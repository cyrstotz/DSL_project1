TASK 3a  
State 1 in the ice-water mixture (EW):  

The pressure in state 1 is calculated as follows:  
\[
p_{\text{EW}} = p_{\text{amb}} + \frac{m_K g}{A}
\]  
where \( A = (0.1 \, \text{m})^2 \pi \), \( p_{\text{amb}} = 1 \, \text{bar} \), \( m_K = 32 \, \text{kg} \), and \( g = 9.81 \, \text{m/s}^2 \). Substituting these values:  
\[
p_{\text{EW}} = \frac{(0.1 \, \text{m})^2 \pi \cdot 10^5 + 32 \cdot 9.81}{(0.1 \, \text{m})^2 \pi} = 1.1 \, \text{bar}
\]  

The gas pressure \( p_g \) is then determined:  
\[
p_g = p_{\text{EW}} + \frac{m_K g}{A} = 1.1 \, \text{bar}
\]  

The mass of the gas \( m_g \) is calculated using:  
\[
m_g = \frac{p_g V_g}{R_g T_g}
\]  
Substituting \( p_g = 1.1 \, \text{bar} \), \( V_g = 3.14 \, \text{L} \), \( R_g = \frac{8314}{50} \), and \( T_g = 773.15 \, \text{K} \):  
\[
m_g = \frac{1.1 \cdot 3.14}{\frac{8314}{50} \cdot 773.15} = 2.68 \, \text{g}
\]  

---

TASK 3b  
Since we assume that solid/liquid water is incompressible and the density difference is negligible, the pressure in state 2 is equal to the pressure in state 1:  
\[
p_{\text{g,2}} = p_{\text{g,1}} = 1.5 \, \text{bar}
\]  

The temperature of the gas in state 2 is equal to the temperature in state 1:  
\[
T_{\text{g,2}} = T_{\text{g,1}} = 500^\circ \text{C}
\]