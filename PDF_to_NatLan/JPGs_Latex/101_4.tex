TASK 3a  
The problem involves determining the gas pressure \( p_{g,1} \) and molar mass \( M_g \) in state 1.  

The molar mass of the gas is given as:  
\[
M_g = 50 \, \text{kg/kmol} = 50 \cdot 10^{-3} \, \text{kg/mol}
\]  

The gas constant \( R \) is calculated using the universal gas constant \( \bar{R} \):  
\[
R = \frac{\bar{R}}{M_g} = \frac{8.314 \, \text{J/(mol·K)}}{50 \cdot 10^{-3} \, \text{kg/mol}} = 166.28 \, \text{J/(kg·K)}
\]  

The temperature \( T \) is converted to Kelvin:  
\[
T = 500^\circ\text{C} = 773.15 \, \text{K}
\]  

The volume \( V \) is converted to cubic meters:  
\[
V = 3.14 \, \text{L} = 3.14 \cdot 10^{-3} \, \text{m}^3
\]  

The relationship between pressure, volume, and temperature is given by the ideal gas law:  
\[
pV = nRT
\]  

The number of moles \( n \) is expressed as:  
\[
n = \frac{M}{m}
\]  

TASK 3b  
The pressure \( p \) is derived using the ideal gas law rearranged for mass \( m \):  
\[
pV = mR T
\]  
\[
m = \frac{pV}{RT}
\]  

Substituting values:  
\[
p = \frac{mR T}{V}
\]  

The pressure is calculated as:  
\[
p = \frac{M_g R T^2}{V^2} = 1.14712 \, \text{bar} \approx 1.15 \, \text{bar} = p_{g,1}
\]  

The mass \( m_g \) is calculated using the ideal gas law:  
\[
m = \frac{pV}{RT}
\]  

Substituting values:  
\[
m = \frac{1.15 \cdot 3.14 \cdot 10^{-3}}{166.28 \cdot 773.15} = 2.81 \cdot 10^{-3} \, \text{kg} = 2.81 \, \text{g}
\]  

The final mass of the gas is:  
\[
m_g = 2.81 \, \text{g}
\]  

No diagrams or figures are present on this page.