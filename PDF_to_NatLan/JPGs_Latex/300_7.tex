TASK 4a  
The diagram is a pressure-temperature (\( p \)-\( T \)) graph. It shows phase regions for gas, liquid, and solid (ice). The curve labeled "Flüssig" represents the boundary between liquid and gas phases, while the curve labeled "Fest" represents the boundary between solid and liquid phases. The point labeled "1" indicates the initial state, and the point labeled "3" represents the state after compression. The process transitions are described as follows:  
- Isobaric freezing occurs first.  
- Sublimation follows, which is described as an isothermal process.  

TASK 4b  
The energy balance equation is written as:  
\[
\dot{m}_{123} \cdot (h_{\text{in}} - h_{\text{aus}}) + Q_{12} = 0
\]  
Here, \( Q_{12} \) represents the heat transfer, and \( h_{\text{in}} \) and \( h_{\text{aus}} \) are enthalpies.  

The enthalpy difference is expressed as:  
\[
Q_{12} = \dot{m}_{123} \cdot (h_{\text{in}} - h_1) \quad \Rightarrow \quad Q_{12} = \dot{m}_{123} \cdot (h_2 - h_1)
\]  

The process steps are described:  
- From state 1 to state 2: Isobaric evaporation occurs at constant temperature (\( T_v \)) with saturated vapor.  
- From state 2 to state 3: Adiabatic and reversible compression occurs.  

The mass flow rate equation is given as:  
\[
\dot{m}_{123} \cdot (h_2 - h_3) + \dot{Q}_K = \dot{W}_K
\]  
Rearranging for the mass flow rate:  
\[
\dot{m}_{123} = \frac{\dot{W}_K}{h_2 - h_3}
\]  

The work input for the compressor is specified as:  
\[
\dot{W}_K = -28 \, \text{W}
\]  

Additional notes mention difficulties in finding \( h_3 \) at 8 bar. The entropy values are noted as:  
\[
s_2 = s_3
\]  

No further calculations or results are provided.