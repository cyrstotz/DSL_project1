TASK 3a  
The equation for the gas pressure \( p_{g,1} \) is derived using the ideal gas law:  
\[
p_{g,1} V_{g,1} = m_g R T_1
\]  
The forces acting on the piston are balanced as follows:  
\[
m_{\text{EW}} g + m_K g + p_{\text{amb}} A = p_{g,1} A
\]  
Where:  
- \( A = \pi r^2 = \pi \left( \frac{D}{2} \right)^2 = 0.007854 \, \text{m}^2 \)  

Substituting values:  
\[
p_{g,1} = \frac{m_{\text{EW}} g}{A} + \frac{m_K g}{A} + p_{\text{amb}} = \frac{0.1 \, \text{kg} \cdot 9.81 \, \text{m/s}^2}{0.007854 \, \text{m}^2} + \frac{32 \, \text{kg} \cdot 9.81 \, \text{m/s}^2}{0.007854 \, \text{m}^2} + 10^5 \, \text{Pa} = 1.4 \, \text{bar}
\]  

The gas mass \( m_g \) is calculated using:  
\[
m_g = \frac{p_{g,1} V_{g,1}}{R T_1}
\]  
Where:  
\[
R = \frac{\bar{R}}{M} = \frac{8.314 \, \text{kJ/kmol·K}}{50 \, \text{kg/kmol}} = 166.28 \, \text{J/kg·K}
\]  

Substituting values:  
\[
m_g = \frac{1.4 \cdot 10^5 \, \text{Pa} \cdot 0.00314 \, \text{m}^3}{166.28 \, \text{J/kg·K} \cdot (500 + 273.15) \, \text{K}} = 3.449 \, \text{g}
\]  

TASK 3b  
The pressure remains the same as in state 1 because the forces acting on the piston have not changed. Thus:  
\[
p_{g,2} = p_{g,1} = 1.4 \, \text{bar}
\]  
The temperature has changed because energy was transferred to melt the ice.  

The specific heat capacities are given as:  
\[
c_V = 0.633 \, \text{kJ/kg·K}
\]  
\[
c_p = R + c_V = 166.28 \, \text{J/kg·K} + 633 \, \text{J/kg·K} = 799.28 \, \text{J/kg·K}
\]