TASK 4a  
The diagram is a pressure-temperature (\(p\)-\(T\)) graph illustrating the freeze-drying process. It includes the following labeled features:  
- A curve labeled "gas" representing the gaseous phase.  
- A point labeled "Triple point," indicating the intersection of phase boundaries.  
- A horizontal line labeled "isobar" and "Filling," representing an isobaric process.  
- A vertical line labeled "isochor," representing an isochoric process.  
- The axes are labeled \(p\) (pressure) on the vertical axis and \(T\) (temperature) on the horizontal axis.  

TASK 4b  
A table is drawn with columns labeled \(T\), \(p\), and \(h\), corresponding to temperature, pressure, and enthalpy. The rows are numbered 1, 2, 3, and 4, with the following values:  
- Row 1: \(T = -60^\circ\text{C}\)  
- Row 2: \(T = -60^\circ\text{C}\), \(p = 8 \, \text{bar}\)  
- Row 3: \(T = 37.33^\circ\text{C}\), \(p = 8 \, \text{bar}\)  
- Row 4: \(T = 37.33^\circ\text{C}\)  

The following calculations are provided:  
1. **Entropy calculation**:  
   The process is described as adiabatic, reversible, and isentropic. The entropy values are calculated as:  
   \[
   s_2 = s_3 = 0.9273 + \left(-60 + 8.4\right) \cdot 0.9235 - 0.9273 = 0.9226 \, \text{kJ/kg·K}
   \]

2. **Mass flow rate calculation**:  
   Using the energy balance equation:  
   \[
   \dot{m} \cdot (h_2 - h_3) - W_{\text{K}} = 0
   \]  
   Rearranging for \(\dot{m}\):  
   \[
   \dot{m} = \frac{W_{\text{K}}}{(h_2 - h_3)} = 0.00170 \, \text{kg/s} = 3.573 \, \text{kg/h}
   \]

3. **Enthalpy values**:  
   The enthalpy values are calculated as follows:  
   \[
   h_2 = 264.7 + \left(-60 + 8.4\right) \cdot 0.9235 - 0.9273 = 243.72 \, \text{kJ/kg}
   \]  
   \[
   h_3 = 264.7 + 0.9066 = 273.66 \, \text{kJ/kg}
   \]  
   \[
   h_4 = 264.7 + 0.9324 - 0.9066 = 273.46 \, \text{kJ/kg}
   \]  

The final mass flow rate is noted as:  
\[
\dot{m} = 65.050 \, \text{kg/h}
\]