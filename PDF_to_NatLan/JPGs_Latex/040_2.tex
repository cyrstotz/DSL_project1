TASK 2a  
The process is illustrated in a T-s diagram. The diagram shows six states labeled 1 through 6, with isobars \( p_0 \), \( p_1 \), \( p_2 \), and \( p_3 \) drawn as curves. The states are connected by arrows indicating the transitions between them.  

State descriptions are provided in a table:  
- **State 1**: \( T = -30^\circ\text{C} \), \( s = s_1 \), \( p = p_0 \).  
- **State 2**: \( T = T_2 \), \( s_3 > s_1 \), \( p_2 > p_0 \).  
- **State 3**: \( T = T_3 \), \( s_2 > s_2 \), \( p_3 > p_2 \).  
- **State 4**: \( T = T_4 \), \( s_3 > s_3 \), \( p_4 < p_3 \).  
- **State 5**: \( T = 98.9^\circ\text{C} \), \( s_5 > s_3 \), \( p = p_4 \).  
- **State 6**: \( T = T_6 \), \( s_6 = s_5 \), \( p = p_0 \).  

The table also includes notes about pressure relationships and entropy changes between states.  

TASK 2b  
The goal is to determine \( w_6 \) and \( T_6 \).  

The isentropic relationships are used, with known values of \( w_5 \), \( p_5 \), and \( T_5 \). The equation for energy balance is given:  
\[
Q = \dot{m} \left( h_s - h_6 + \frac{\Delta k_e}{2} \right), \quad \Delta k_e = \frac{w_5^2 - w_6^2}{2}
\]  

The enthalpy \( h_s \) is calculated using tabulated values:  
\[
h_s = h(930 \, \text{K}) + h(990 \, \text{K}) - h(950 \, \text{K}) \quad \text{with corrections of } 1.9 \, \text{kJ/kg}.
\]  

The equations are set up to solve for \( w_6 \) and \( T_6 \) based on the given data and relationships.  

No additional diagrams or graphs are present beyond the T-s diagram described above.