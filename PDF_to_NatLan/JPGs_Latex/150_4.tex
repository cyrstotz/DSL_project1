TASK 3a  
The universal gas constant \( R \) is calculated as:  
\[
R = \frac{\bar{R}}{M_g} = \frac{766.28 \, \text{J}}{\text{kg·K}} = 0.766 \, \text{J}/\text{g·K}
\]

The gas pressure \( p_{g,1} \) is determined using the formula:  
\[
p_{g,1} = \frac{m_{\text{EW}} \cdot g}{D^2 \cdot \pi} + p_0 + \frac{m_K \cdot g}{D^2 \cdot \pi}
\]  
Substituting values, \( p_{g,1} = 1.4 \, \text{bar} \).

The gas mass \( m_g \) is calculated using the ideal gas law:  
\[
p_{g,1} V_{g,1} = m_g R T_{g,1} \quad \Rightarrow \quad m_g = \frac{p_{g,1} V_{g,1}}{R T_{g,1}}
\]  
Substituting values, \( m_g = 3.42 \, \text{g} \).

---

TASK 3b  
The energy balance is expressed as:  
\[
O = \dot{Q}_{12} = C_g - h_{\text{EW}}
\]

The gas pressure \( p_{g,2} \) remains constant at \( p_{g,2} = p_{g,1} = 1.4 \, \text{bar} \), as the weight of the piston (\( CF_g \)) remains constant and exerts the same pressure.

The equilibrium temperature \( T_2 \) of the ice-water mixture is:  
\[
T_2 = T_{\text{EW},1} = 0^\circ\text{C}
\]  
This is because the heat capacity and mass of water are significantly larger than the latent heat of fusion, and the mixture still contains ice. Therefore, the equilibrium temperature remains constant:  
\[
T_2 = T_{\text{EW},1} = T_{\text{15W}}
\]