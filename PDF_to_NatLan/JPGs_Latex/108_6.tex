TASK 4a  
A graph is drawn representing the freeze-drying process in a \( p \)-\( T \) diagram. The graph includes a blue curve indicating phase boundaries. Four states (1, 2, 3, and 4) are labeled, forming a cycle. The pressure axis is labeled \( p \, [\text{bar}] \), and the temperature axis is labeled \( T \, [\text{K}] \). The pressure at state 3 is marked as \( 150 \, \text{bar} \), and the pressure at state 1 is marked as \( 1 \, \text{bar} \).  

TASK 4b  
The vapor quality at state 1 is given as \( x_1 = 1 \).  
The pressures at state 2 and state 3 are \( p_2 = p_3 = 8 \, \text{bar} \), and the vapor quality at state 3 is \( x_3 = 0 \).  

The energy balance equation is written as:  
\[
\dot{m} \cdot (h_2 - h_3) = \dot{W}_K
\]  
where \( \dot{m} \) is the mass flow rate, \( h_2 \) and \( h_3 \) are enthalpies, and \( \dot{W}_K \) is the work input to the compressor.  

Entropy values are noted as \( s_2 = s_3 \).  
The pressure at state 4 is \( p_4 = p_3 = 8 \, \text{bar} \), and the internal chamber pressure is \( p_{\text{Innenraum}} = 1 \, \text{bar} \).  

TASK 4c  
The mass flow rate \( \dot{m} \) is calculated using the following assumptions:  
The temperature at state 2 is \( T_2 = -22^\circ\text{C} \), and data is taken from Table A-10.  

The pressure at state 2 is \( p_2 = 1.292 \, \text{bar} \), and the pressure at state 4 is \( p_4 = 8 \, \text{bar} \).  

The energy balance equation is written as:  
\[
\dot{m} \cdot (h_{u,4} - h_{u,1}) = 0 \implies h_{u,4} = h_{u,1}
\]  
where \( h_{u,4} \) and \( h_{u,1} \) are enthalpies.  

From Table A-10:  
\[
h_{u,4} = 93.62 \, \frac{\text{kJ}}{\text{kg}}, \quad h_{u,1} = 234.08 \, \frac{\text{kJ}}{\text{kg}}
\]  

The vapor quality \( x \) is calculated as:  
\[
x = \frac{h_{u,4} - h_f}{h_g - h_f} = 0.337
\]  

TASK 4d  
The coefficient of performance \( \epsilon_K \) is calculated as:  
\[
\epsilon_K = \frac{\dot{Q}_K}{\dot{W}_K}
\]  

The energy balance for the evaporator is written as:  
\[
\dot{Q}_K = \dot{m} \cdot (h_2 - h_1) + \dot{Q}_{\text{zu}}
\]  

From Table A-10:  
\[
h_2 = 234.08 \, \frac{\text{kJ}}{\text{kg}}
\]  

The heat transfer rate \( \dot{Q}_K \) is calculated as:  
\[
\dot{Q}_K = \dot{m} \cdot (h_2 - h_1) = 0.005 \, \frac{\text{kg}}{\text{s}} \cdot (234.08 \, \frac{\text{kJ}}{\text{kg}} - 93.62 \, \frac{\text{kJ}}{\text{kg}}) = 0.146 \, \text{kW}
\]  

The coefficient of performance is then calculated as:  
\[
\epsilon_K = \frac{0.146}{38 \, \text{kW}} = 0.005
\]  

TASK 4e  
The temperature would decrease. The internal pressure remains constant, and the volume also remains unchanged. The only way to remove energy from the system is by reducing the temperature.