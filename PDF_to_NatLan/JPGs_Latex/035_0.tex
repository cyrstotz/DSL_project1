TASK 4a  
The task involves drawing the freeze-drying process in a \( p \)-\( T \) diagram and labeling the phase regions.  

### Description of the first graph:  
The graph is a \( p \)-\( T \) diagram with pressure (\( p \)) on the vertical axis labeled in bar and temperature (\( T \)) on the horizontal axis labeled in Kelvin (\( K \)). The diagram includes a wavy curve representing phase boundaries. There is a shaded region in the lower part of the graph, likely indicating a specific phase region.  

### Process steps:  
1. **Step 1-2:**  
   - Temperature (\( T \)) decreases.  
   - Pressure (\( p \)) decreases.  

2. **Step 2-3:**  
   - Entropy remains constant (\( S_2 = S_3 \)).  
   - Pressure (\( p \)) decreases further.  

3. **Step 3-4:**  
   - Isobaric process (\( p = \text{const.} \)).  
   - Temperature (\( T \)) increases.  

4. **Step 4-1:**  
   - Isenthalpic process (\( h_4 = h_1 \)).  
   - Pressure (\( p \)) decreases.  
   - Temperature (\( T \)) remains constant.  

---

### Description of the second graph:  
The second graph is another \( p \)-\( T \) diagram with pressure (\( p \)) on the vertical axis labeled in bar and temperature (\( T \)) on the horizontal axis labeled in Kelvin (\( K \)).  

The diagram shows a cycle with four states labeled:  
- **State 1:** Starting point.  
- **State 2:** Pressure decreases, temperature remains constant (\( T = \text{const.} \)).  
- **State 3:** Pressure remains constant (\( p = \text{const.} \)), temperature increases.  
- **State 4:** Pressure remains constant (\( p = \text{const.} \)), temperature decreases back to the starting point.  

The cycle is drawn as a closed loop connecting these states.  

