TASK 3a  
To determine the gas pressure \( p_{g,1} \) and mass \( m_g \) in state 1:  

The given values are:  
\[
T_{g,1} = 500^\circ\text{C} = 773 \, \text{K}
\]
\[
V_{g,1} = 3.14 \, \text{L} = 0.00314 \, \text{m}^3
\]
\[
M_g = 50 \, \text{kg/kmol}
\]

The ideal gas law is applied:  
\[
pV = nRT
\]
Where \( n \) is the number of moles, \( R \) is the specific gas constant, and \( m \) is the mass of the gas.  

The specific gas constant is calculated as:  
\[
R = \frac{\bar{R}}{M_g} = \frac{8.314}{50} = 0.166 \, \text{kJ/kg·K}
\]

The mass of the gas is expressed as:  
\[
m = \frac{pV}{RT}
\]

The pressure is derived as:  
\[
p = \frac{mRT}{V}
\]

Some calculations and derivations are crossed out and not legible.  

---

TASK 3b  
Given \( x_{\text{ice},2} > 0 \), the temperature \( T_{g,2} \) and pressure \( p_{g,2} \) are determined.  

The following values are provided:  
\[
p_{g,1} = 1.56 \, \text{bar}
\]
\[
m_g = 3.6 \, \text{g} = 0.0036 \, \text{kg}
\]
\[
V_{g,2} = 0.00314 \, \text{m}^3
\]

Using the relation for temperature and pressure:  
\[
\frac{p_2 V_2}{R m_g} = T_2
\]

The temperature ratio is expressed as:  
\[
\frac{T_2}{T_1} = \left( \frac{V_1}{V_2} \right)^{\kappa - 1}
\]
Where \( \kappa \) is the adiabatic index.  

This simplifies to:  
\[
T_2 = T_1
\]

The pressure ratio is expressed as:  
\[
\frac{p_2}{p_1} = \left( \frac{T_2}{T_1} \right)^{\frac{\kappa}{\kappa - 1}}
\]
This leads to:  
\[
p_2 = p_1
\]

Additional notes state:  
"Mass and volume remain constant, so \( V \) remains equal."

---

No diagrams or graphs are present on the page.