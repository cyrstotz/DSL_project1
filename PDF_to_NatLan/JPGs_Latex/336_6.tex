TASK 4a  
A p-T diagram is drawn to represent the freeze-drying process. The diagram includes the following features:  
- The x-axis represents temperature \( T \) (in Kelvin), and the y-axis represents pressure \( p \) (in bar).  
- The diagram shows phase regions labeled as "solid," "liquid," and "gas."  
- The triple point is marked as a specific point where the three phases coexist.  
- An isothermal expansion process is indicated at the lower left, starting at 0.5 bar.  
- The process path includes isobaric cooling and transitions between phases.  
- A temperature difference \( \Delta T = 10 \, \text{K} \) is noted near the isothermal expansion.  

TASK 4b  
The equation for the rate of change of energy is written as:  
\[
\frac{dE}{dt} = \sum \dot{m} \cdot (h_{\text{in}} + v^2 + gz) + \dot{Q} - \dot{W}
\]  
The energy balance is simplified as:  
\[
\sum E_{\text{in}} - \sum E_{\text{out}} = -Q_K - Q_{\text{ab}} = W_K = 28 \, \text{W}
\]  
The system is described as adiabatic.  

TASK 4c  
A schematic diagram is drawn, representing the refrigeration cycle.  
- The diagram includes two intersecting triangles, symbolizing the cycle components.  
- Labels indicate the following:  
  - \( m \): mass flow rate  
  - \( \dot{Q}_K \): heat removed  
  - \( \Pi_4 \): pressure at state 4, given as \( 8 \, \text{bar} \)  
  - \( T_4(8 \, \text{bar}) = 31.33^\circ\text{C} \)  
  - \( x = 0 \): vapor quality at state 4  
  - \( p = 8 \, \text{bar} \): pressure  

No further explanation or calculations are provided.