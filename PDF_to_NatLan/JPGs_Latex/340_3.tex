TASK 3a  
The gas pressure \( p_{g,1} \) is calculated using the formula:  
\[
p_{g,1} = \frac{R \cdot m}{V}
\]  
where \( R = 0.633 \, \text{kJ/kg·K} \), \( m = 32 \, \text{kg} \), and \( V = 3.14 \, \text{L} \).  
The calculation yields:  
\[
p_{g,1} = 766.26 \, \text{kPa}
\]  

The sketch shows a cylinder divided into two chambers: the bottom chamber contains gas, and the top chamber contains the ice-water mixture. A piston rests on the ice-water mixture, exerting pressure.  

TASK 3b  
The temperature \( T_{g,2} \) in state 2 is equal to \( T_{\text{EW},2} \), as thermal equilibrium is reached.  
\[
T_{g,2} = T_{\text{EW},2} = 273.15 \, \text{K}
\]  

The pressure \( p_{g,2} \) is calculated using the formula:  
\[
p_{g,2} = p_{\text{amb}} + \frac{m_K \cdot g}{A}
\]  
where \( p_{\text{amb}} = 1 \, \text{bar} \), \( m_K = 32 \, \text{kg} \), \( g = 9.81 \, \text{m/s}^2 \), and \( A = 0.00785 \, \text{m}^2 \).  
The calculation yields:  
\[
p_{g,2} = 1.769 \, \text{bar}
\]  

The explanation states that the temperature remains constant because the gas transfers heat to the ice-water mixture, melting the ice. The pressure changes due to the piston maintaining equilibrium.  

TASK 3c  
The energy balance equation is used to calculate the heat transferred \( Q_{12} \):  
\[
Q_{12} = \Delta E_{\text{int}} + W
\]  
where \( \Delta E_{\text{int}} = m \cdot c_V \cdot (T_2 - T_1) \) and \( W = \int p \, dV = p(V_2 - V_1) \).  

The work \( W \) is calculated as:  
\[
W = p \cdot (V_2 - V_1) = 1.769 \cdot 10^5 \cdot (0.00695 - 0.00314) \, \text{m}^3
\]  
\[
W = 6.36 \, \text{kJ}
\]  

The heat transfer \( Q_{12} \) is then calculated:  
\[
Q_{12} = 32 \cdot 0.633 \cdot (273.15 - 773.15) + 6.36
\]  
\[
Q_{12} = -30,496.96 + 6.36 = -30,490.6 \, \text{kJ}
\]  

The final result is:  
\[
Q_{12} = 7,719.257 \, \text{J}
\]  

No diagrams or additional figures are described beyond the cylinder sketch.