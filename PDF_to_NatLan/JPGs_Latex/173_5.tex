TASK 4a  
A graph is drawn showing a pressure-temperature (\(p\)-\(T\)) diagram. The axes are labeled as follows:  
- The vertical axis represents pressure (\(p\)) in bar, ranging from 1 bar to 8 bar.  
- The horizontal axis represents temperature (\(T\)) in Kelvin (\(K\)).  

The graph depicts a cycle with four states labeled 1, 2, 3, and 4.  
- State 1 is at low pressure and low temperature.  
- State 2 is at low pressure and higher temperature.  
- State 3 is at high pressure and high temperature.  
- State 4 is at high pressure and lower temperature.  

The cycle forms a closed loop, with arrows indicating the direction of the process.  

TASK 4b  
A schematic diagram of a refrigeration system is drawn. It shows the following:  
- An inlet stream labeled \( \dot{m}_R \), entering at pressure \( p_2 = p_1 \) and temperature \( T_2 \).  
- An outlet stream labeled \( \dot{m} \), exiting at temperature \( T_1 \).  
- Heat flow \( \dot{Q}_K \) is shown being removed from the system.  

Equations are written below the diagram:  
\[
\dot{E}_{\text{in}} = \sum \dot{m} \left( h + \frac{v^2}{2} + gz \right) + \sum \dot{Q} - \sum \dot{W}
\]  
\[
0 = \dot{m} \sum h + \dot{Q}_K
\]  
\[
\dot{Q}_K = \dot{m} \left( h_{\text{aus}} - h_{\text{ein}} \right)
\]  
\[
\dot{m}_{\text{R134a}} = \frac{\dot{Q}_K}{h_{\text{aus}} - h_{\text{ein}}}
\]  

Where:  
- \( h_{\text{aus}} \) represents the enthalpy at the outlet.  
- \( h_{\text{ein}} \) represents the enthalpy at the inlet.  
- \( \dot{m}_{\text{R134a}} \) is the mass flow rate of the refrigerant.  

No additional diagrams or content are present.