TASK 2a  
The diagram is a qualitative representation of the jet engine process in a \( T \)-\( s \) diagram. It includes labeled isobars and states:  
- State 0 represents ambient conditions (\( p_0 \)).  
- State 1 is the inlet air condition.  
- State 2 is after the compression process.  
- State 3 is the combustion chamber.  
- State 4 is the turbine outlet.  
- State 5 is the mixing chamber.  
- State 6 is the nozzle exit.  

The isobars are clearly marked, showing the pressure levels \( p_0 \), \( p_2 \), and \( p_5 \). The entropy axis (\( s \)) and enthalpy axis (\( h \)) are labeled, with units \( \text{kJ/kg·K} \).  

TASK 2b  
The nozzle is modeled as adiabatic and reversible:  
\[
s_5 = s_6 \quad \text{(isentropic process)}  
\]  

The temperature \( T_6 \) is calculated using the isentropic relation:  
\[
\frac{T_6}{T_5} = \left( \frac{p_6}{p_5} \right)^{\frac{\kappa - 1}{\kappa}}  
\]  
Substituting values:  
\[
T_6 = T_5 \left( \frac{p_6}{p_5} \right)^{\frac{\kappa - 1}{\kappa}} = 328.075 \, \text{K}  
\]  

Using the energy balance at the nozzle:  
\[
\dot{Q} = \dot{m}_K \left( h_5 - h_6 + \frac{w_5^2}{2} - \frac{w_6^2}{2} \right)  
\]  

Rearranging for \( w_6^2 \):  
\[
\frac{w_6^2}{2} = h_5 - h_6 + \frac{w_5^2}{2}  
\]  

The enthalpy difference is calculated as:  
\[
h_5 - h_6 = c_p \cdot (T_5 - T_6) = 103.888 \, \text{kJ/kg}  
\]  

Finally, the outlet velocity \( w_6 \) is determined:  
\[
w_6 = \sqrt{2 \left( h_5 - h_6 + \frac{w_5^2}{2} \right)} = 506.14 \, \text{m/s}  
\]