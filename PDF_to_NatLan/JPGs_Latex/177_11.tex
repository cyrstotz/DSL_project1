TASK 4b  
The refrigerant mass flow rate \( \dot{m}_{\text{R134a}} \) is calculated using the first law of thermodynamics applied to the compressor between states 2 and 3:  
\[
0 = \dot{m}_{\text{R134a}} (h_2 - h_3) + \dot{W}_K
\]  
Here, \( h_2 \) is the enthalpy at state 2, and \( h_3 \) is the enthalpy at state 3.  

Given:  
\[
h_2 = h_g(-22^\circ\text{C}) = 237.08 \, \frac{\text{kJ}}{\text{kg}}
\]  
At state 3, entropy is conserved (\( s_2 = s_3 \)):  
\[
s_2 = s_g(-22^\circ\text{C}) = 0.9351 \, \frac{\text{kJ}}{\text{kg·K}}
\]  
\[
s_3 = 0.9351 \, \frac{\text{kJ}}{\text{kg·K}}
\]  

Using the entropy relation for \( h_3 \):  
\[
h_3(p_{\text{bar}}, s_3) = 0.9351 \, \frac{\text{kJ}}{\text{kg·K}} = 269.15 \, \frac{\text{kJ}}{\text{kg}}
\]  
\[
h_3 = 264.18 \, \frac{\text{kJ}}{\text{kg}}
\]  

Substituting into the first law equation:  
\[
\dot{m}_{\text{R134a}} = \frac{-\dot{W}_K}{h_2 - h_3} = \frac{-0.028 \, \text{kW}}{234.08 \, \frac{\text{kJ}}{\text{kg}} - 269.18 \, \frac{\text{kJ}}{\text{kg}}}
\]  
\[
\dot{m}_{\text{R134a}} = 3.349 \, \frac{\text{kg}}{\text{h}}
\]  

---

TASK 4c  
The vapor quality \( x_1 \) of the refrigerant at state 1 after expansion is determined.  

Given:  
\[
x_1 = 0
\]  
At state 4:  
\[
p_4 = 8 \, \text{bar}
\]  
The throttling process is isenthalpic and adiabatic, with no work done:  
\[
h_4 = h_1
\]  

From the refrigerant table:  
\[
h_f(p_{\text{bar}}) = 93.42 \, \frac{\text{kJ}}{\text{kg}}
\]  

Using the enthalpy relation:  
\[
h_1 = h_f + x(h_g - h_f)
\]  
At \( 1 \, \text{bar} \), \( T_{\text{bar}} \), and \( p_{\text{amb}} \), the enthalpy values are referenced from the refrigerant table.  

