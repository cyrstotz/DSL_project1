TASK 4b  
From Table A-12, the enthalpy can now be determined using entropy. It is known that:  
\[
s_2 = s_3 = 6.528 \, \frac{\text{kJ}}{\text{kg·K}}
\]  

The enthalpy \( h_3 \) is calculated as:  
\[
h_3 = h(8 \, \text{bar}, 40^\circ\text{C}) - h(8 \, \text{bar}, T_{\text{sat}}) + \frac{s_3 - s(8 \, \text{bar}, T_{\text{sat}})}{s(8 \, \text{bar}, 40^\circ\text{C}) - s(8 \, \text{bar}, T_{\text{sat}})} \cdot h(8 \, \text{bar}, T_{\text{sat}})
\]  
\[
h_3 = 277.37 \, \frac{\text{kJ}}{\text{kg}}
\]  

Using this, for the balance equation under the compressor, which is adiabatic:  
\[
0 = \dot{m}(h_2 - h_3) + \dot{Q} - \dot{W}
\]  

The mass flow rate \( \dot{m} \) is calculated as:  
\[
\dot{m} = \frac{\dot{W}}{h_2 - h_3} = 8.34 \cdot 10^{-4} \, \frac{\text{kg}}{\text{s}}
\]  

TASK 4c  
For the adiabatic throttle (isenthalpic process), using Table A-11:  
\[
h_a = h_n
\]  
\[
h_n = h_f(8.0 \, \text{bar}) = 93.92 \, \frac{\text{kJ}}{\text{kg}}
\]  

For the evaporator operating in water vapor at \( -16^\circ\text{C} \), the vapor quality can be calculated using Table A-10:  
\[
x_n = \frac{h_n - h_f(-16^\circ\text{C})}{h_g(-16^\circ\text{C}) - h_f(-16^\circ\text{C})}
\]  
\[
x_n = 30.76\%
\]