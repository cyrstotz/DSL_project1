TASK 3a  
The pressure \( p_{g,1} \) and mass \( m_g \) of the gas in state 1 are calculated as follows:  

The cross-sectional area \( A \) of the cylinder is determined using the formula:  
\[
A = (5 \times 10^{-2})^2 \pi = 7.854 \, \text{m}^2 \times 10^{-3}
\]  

The pressure \( p_{g,1} \) is calculated as:  
\[
p_{g,1} = 1 \, \text{bar} + \frac{32 \, \text{kg}}{A} + 0.1 \, \text{bar} = 1.041 \, \text{bar}
\]  

The ideal gas law is used to calculate the mass \( m_g \):  
\[
pV = mRT
\]  
where \( R = \frac{8.314}{50} \).  

Substituting values:  
\[
m = \frac{1.041 \times 3.141 \times 10^{-3}}{\frac{8.314}{50} \times 773.15} = 2.5426 \, \text{kg}
\]  

---

TASK 3b  
The pressure \( p_{g,2} \) and mass \( m_g \) in state 2 are given as:  
\[
p_{g,2} = 1.5 \, \text{bar}, \quad m_g = 3.6 \, \text{g}
\]  

It is stated that \( p_{g,2} = p_{g,1} = 1.5 \, \text{bar} \) because the mass of the ice-water mixture (\( m_{\text{EW}} \)) remains constant.  

The temperature \( T_{g,2} \) is equal to \( T_{\text{EW},2} \), which is \( 0^\circ\text{C} \), because \( x_{\text{ice},2} > 0 \).  

---

TASK 3c  
The heat transferred \( Q_{12} \) from the gas to the ice-water mixture is calculated as follows:  

The specific heat capacity \( c_V \) of the gas is given as:  
\[
c_V = 0.633 \, \text{kJ/kg·K}
\]  

Since \( \Delta p = 0 \), the change in internal energy \( \Delta U \) is equal to the heat transferred \( Q \):  
\[
\Delta U = Q
\]  

Using the formula:  
\[
m \cdot c_V \cdot \Delta T = Q_{12}
\]  

The temperature difference \( \Delta T \) is:  
\[
\Delta T = -500^\circ\text{C}
\]  

Substituting values:  
\[
Q_{12} = 1140 \, \text{J}
\]  

--- 

No diagrams or graphs are present on the page.