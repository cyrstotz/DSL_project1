TASK 4c  
The vapor quality \( x_1 \) is determined for the transition from state 4 to state 1.  

The energy balance equation is written as:  
\[
0 = \dot{m} \left[ O \right] + \dot{Q} - \dot{W}_u
\]  
Here, \( \dot{W}_u = \dot{Q} = 0 \), as the process is isentropic.  

---

TASK 4d  
The coefficient of performance \( \epsilon_K \) is calculated using the formula:  
\[
\epsilon_K = \frac{\dot{Q}_{\text{zu}}}{\dot{W}_u}
\]  
This can also be expressed as:  
\[
\epsilon_K = \frac{\dot{Q}_{\text{zu}}}{\left( \dot{Q}_{\text{ab}} - \dot{Q}_{\text{zu}} \right)}
\]  
This formula applies specifically for refrigeration cycles.  

In this case, the coefficient of performance is given by:  
\[
\epsilon_K = \frac{\dot{Q}_K}{\left( \dot{Q}_{\text{ab}} - \dot{Q}_x \right)}
\]  

---

TASK 4e  
The temperature evolves toward equilibrium.  

The internal temperature can only become as cold as the surrounding environment. If \( \dot{Q}_{\text{ab}} \) gradually approaches zero, the lowest achievable internal temperature is reached.  

