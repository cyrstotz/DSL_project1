TASK 2a  
The task involves drawing a qualitative T-s diagram for the jet engine process.  

**Description of the diagram:**  
The diagram consists of two parts:  
1. The left graph is a T-s diagram showing the thermodynamic process of the jet engine. It includes labeled states (1, 2, 3, 4, 5, and 6) and isobars (horizontal lines). The process starts at state 1, moves through compression (state 2), combustion (state 3), expansion (state 4), mixing (state 5), and ends at state 6. The axes are labeled as \( T \) (temperature in Kelvin) and \( s \) (specific entropy).  
2. The right graph is a zoomed-in section showing the mixing process between states 5 and 6. It includes labeled isobars and arrows indicating heat transfer (\( \dot{Q} \)) and work (\( \dot{W} \)).  

TASK 2b  
The task involves determining the outlet velocity \( w_6 \) and temperature \( T_6 \).  

**Text and calculations:**  
The gas is modeled as an ideal gas with the following properties:  
- \( c_{p,\text{air}} = 1.006 \, \text{kJ/kg·K} \)  
- \( \kappa = 1.4 \)  

The ambient pressure \( p_0 \) is given as \( 0.191 \, \text{bar} \).  

For the process from state 5 to state 6, it is stated that the process is reversible and adiabatic, implying isentropic behavior. Using the isentropic relation:  
\[
\frac{T_6}{T_5} = \left( \frac{p_6}{p_5} \right)^{\frac{\kappa - 1}{\kappa}}
\]  

Substituting values:  
\[
\frac{T_6}{T_5} = \left( \frac{0.191 \, \text{bar}}{0.5 \, \text{bar}} \right)^{\frac{1.4 - 1}{1.4}}
\]  
\[
T_6 = 431.9 \, \text{K} \times \left( \frac{0.191}{0.5} \right)^{0.286}
\]  
\[
T_6 = 328.074 \, \text{K}
\]  

For \( w_6 \), the velocity can be calculated using the energy equation, but no explicit calculation is shown here.  

Additional notes:  
The specific gas constant \( R \) is calculated as:  
\[
R = c_{p,\text{air}} - c_{v,\text{air}}
\]  
\[
R = 1.006 \, \text{kJ/kg·K} - 0.7186 \, \text{kJ/kg·K} = 0.2874 \, \text{kJ/kg·K}
\]  

No further calculations for \( w_6 \) are visible.