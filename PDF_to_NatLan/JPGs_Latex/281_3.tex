TASK 2a  
The diagram represents a qualitative \( T \)-\( s \) (temperature-entropy) process diagram for a jet engine. The axes are labeled as follows:  
- The vertical axis is \( T \) (temperature) with units \( \frac{u}{\text{kg} \cdot u} \).  
- The horizontal axis is \( s \) (entropy) with units \( \frac{u}{\text{kg} \cdot u} \).  

Key features of the diagram:  
- The process begins at state 1 and follows a vertical line labeled \( s = \text{const} \) (isentropic process).  
- State 2 transitions to state 3 along an isobaric curve labeled "Isobare 2,3".  
- State 3 transitions to state 4 along another isobaric curve labeled "Isobare 4,5".  
- State 4 transitions to state 5 along a vertical line labeled \( s = \text{const} \).  

TASK 2b  
The task involves determining \( w_6 \) (outlet velocity) and \( T_6 \) (outlet temperature).  

The process is described as adiabatic and reversible, implying \( s = \text{const} \) (isentropic process).  

The temperature ratio is given by:  
\[
\frac{T_6}{T_5} = \left( \frac{p_6}{p_5} \right)^{\frac{\kappa - 1}{\kappa}}
\]  
where \( \kappa \) is the specific heat ratio.  

The outlet temperature \( T_6 \) is calculated as:  
\[
T_6 = T_5 \left( \frac{p_6}{p_5} \right)^{\frac{\kappa - 1}{\kappa}}
\]  
Substituting values:  
\[
T_6 = 431.9 \, \text{K} \left( \frac{0.5 \, \text{bar}}{0.191 \, \text{bar}} \right)^{\frac{1.4 - 1}{1.4}}
\]  
\[
T_6 = 568.6 \, \text{K}
\]  

The entropy remains constant (\( s = \text{const} \)).  

The energy balance equation is provided as:  
\[
0 = \dot{m} \left( h_5 - h_6 + \frac{w_5^2 - w_6^2}{2} \right)
\]  
This equation relates the enthalpy difference and velocity terms for the flow.