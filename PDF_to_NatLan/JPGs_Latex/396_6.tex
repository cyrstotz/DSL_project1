TASK 4a  
The page contains two diagrams illustrating phase regions in a pressure-temperature (\( p \)-\( T \)) diagram.  

1. **First Diagram**:  
   - The axes are labeled as \( p \) (pressure in bar) on the vertical axis and \( T \) (temperature in \( ^\circ \text{C} \)) on the horizontal axis.  
   - Three distinct regions are marked: "solid," "fluid," and "gas."  
   - The "solid" region is located at low temperatures and high pressures, while the "gas" region is at high temperatures and low pressures. The "fluid" region lies between them.  
   - A curve labeled "Tripel" (triple point) separates the three phases.  
   - Two points, labeled "1" and "2," are marked along the curve.  

2. **Second Diagram**:  
   - Similar axes are used: \( p \) (pressure in bar) on the vertical axis and \( T \) (temperature in \( ^\circ \text{C} \)) on the horizontal axis.  
   - Three regions are again identified: "solid," "fluid," and "gas."  
   - The "solid" region is at low temperatures and high pressures, while the "gas" region is at high temperatures and low pressures. The "fluid" region lies between them.  
   - A curve separates the phases, and two points, labeled "1" and "2," are marked along the curve.  

Both diagrams visually represent phase transitions and equilibrium points in the \( p \)-\( T \) space.