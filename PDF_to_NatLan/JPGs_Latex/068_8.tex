TASK 4d  
The coefficient of performance \( \epsilon_K \) is calculated using the formula:  
\[
\epsilon_K = \frac{\dot{Q}_{\text{cool}}}{\dot{W}_K} = \frac{\dot{W}_K + \dot{m}_{\text{R134a}} (h_{2} - h_{1})}{\dot{m}_{\text{R134a}} (h_{2} - h_{1})}
\]  
Substituting the values:  
\[
\epsilon_K = \frac{128 \, \text{J/s} + 1.3333 \, \text{g/s} \cdot (264.25 - 93.42) \, \text{J/g}}{1.3333 \, \text{g/s} \cdot (264.25 - 93.42) \, \text{J/g}}
\]  
\[
\epsilon_K = \frac{128 + 1.3333 \cdot 170.83}{1.3333 \cdot 170.83}
\]  
\[
\epsilon_K = 4.228
\]  

TASK 4c  
The work done by the refrigerant is calculated using the integral:  
\[
W_{4 \to 1} = \int_{4}^{1} p \, dV = m_{\text{R134a}} \cdot \frac{R}{M} \cdot (T_4 - T_1)
\]  
Here, \( R \) is the specific gas constant, \( M \) is the molar mass, and \( T_4 \) and \( T_1 \) are the temperatures at states 4 and 1, respectively.  

Additional notes:  
- The term \( \frac{1}{k} \) is referenced, likely related to the isentropic process.  
- The refrigerant used is R134a.  

No diagrams or figures are present on the page.