TASK 2a  
The diagram is a T-s (temperature-entropy) graph illustrating the thermodynamic process of a jet engine.  
- The x-axis represents entropy \( s \) in units of \( \frac{\Delta s}{c_p \cdot k} \).  
- The y-axis represents temperature \( T \) in Kelvin.  
- The process begins at state \( 0 \) with \( p_0 = 0.191 \, \text{bar} \).  
- The graph shows transitions through states \( 1 \), \( 2 \), \( 3 \), \( 4 \), and \( 5 \), ending at state \( 6 \).  
- The pressure at state \( 5 \) is labeled as \( p_5 = 0.5 \, \text{bar} \).  
- The curve includes isobaric and adiabatic sections, with arrows indicating the direction of the process.  

TASK 2b  
The process is described as isentropic, with \( n = \kappa = 1.4 \).  

The temperature at state \( 6 \) is calculated using the following formula:  
\[
T_6 = T_5 \left( \frac{p_6}{p_5} \right)^{\frac{n-1}{n}}
\]  
Substituting the given values:  
\[
T_6 = 431.9 \, \text{K} \left( \frac{0.191 \, \text{bar}}{0.5 \, \text{bar}} \right)^{\frac{1.4-1}{1.4}}
\]  
\[
T_6 = 328.075 \, \text{K}
\]  

The final temperature at state \( 6 \) is:  
\[
T_6 = 328.075 \, \text{K}
\]