TASK 2a  
A temperature-entropy (\( T \)-\( s \)) diagram is drawn. The axes are labeled as follows:  
- The vertical axis represents temperature (\( T \)) in Kelvin (\( \text{K} \)).  
- The horizontal axis represents entropy (\( s \)) in \( \text{kJ}/\text{kg·K} \).  

The diagram shows a qualitative process with four states labeled \( 1 \), \( 2 \), \( 3 \), and \( 4 \).  
- State \( 1 \) to \( 2 \) is an isentropic process (vertical line).  
- State \( 2 \) to \( 3 \) is a curved line indicating an increase in entropy.  
- State \( 3 \) to \( 4 \) is another curved line with decreasing entropy.  

---

TASK 2b  
The goal is to calculate the outlet temperature \( T_6 \).  

Given:  
\[
p_5 = 0.5 \, \text{bar}, \quad w_5 = 220 \, \text{m/s}
\]  

The equation for \( T_6 \) is derived using the pressure ratio and the isentropic relation:  
\[
T_6 = T_5 \left( \frac{p_6}{p_5} \right)^{\frac{\kappa - 1}{\kappa}}
\]  
Substituting values:  
\[
T_6 = 431.5 \, \text{K} \left( \frac{1.013 \, \text{bar}}{0.5 \, \text{bar}} \right)^{\frac{0.4}{1.4}}
\]  
The calculated result is:  
\[
T_6 = 328.07 \, \text{K}
\]  

---

TASK 2c  
The mass-specific increase in flow exergy is calculated using the following equation:  
\[
\dot{m} \left[ h_5 - h_6 \right] + \frac{w_5^2 - w_6^2}{2} = 0
\]  

Additional notes:  
- \( w_5 = w_6 \).  
