TASK 3c  
The calculation begins with the assumption that the gas temperature \( T_{g,2} = 0.003^\circ\text{C} \). The goal is to determine \( Q_{12} \).  

Using the energy balance:  
\[
\Delta E = E_2 - E_1 = Q_{12}
\]  

The internal energy change is expressed as:  
\[
\Delta E = \Delta U = m_g (u_2 - u_1)
\]  

The specific internal energy difference \( u_2 - u_1 \) is calculated using:  
\[
u_2 - u_1 = c_V^{\text{pg}} (T_{g,2} - T_{g,1})
\]  

Substituting values:  
\[
u_2 - u_1 = 0.633 \, \frac{\text{kJ}}{\text{kg·K}} \cdot (273.153 \, \text{K} - 773.15 \, \text{K})
\]  
\[
u_2 - u_1 = -346.50 \, \frac{\text{kJ}}{\text{kg}}
\]  

The total internal energy change is:  
\[
\Delta U = m_g \cdot (u_2 - u_1) = 0.00342 \, \text{kg} \cdot (-346.50 \, \frac{\text{kJ}}{\text{kg}})
\]  
\[
\Delta U = -1.08 \, \text{kJ}
\]  

Thus, the heat transfer is:  
\[
Q_{12} = -1.08 \, \text{kJ}
\]  

---

TASK 3d  
The calculation now considers the case where \( Q_{12} = 1500 \, \text{J} \).  

The specific internal energy \( u \) is given by:  
\[
u = u_{\text{frost}} + x \cdot (u_{\text{flüssig}} - u_{\text{frost}})
\]  

Substituting \( u = \Delta U = -316.50 \, \frac{\text{kJ}}{\text{kg}} \):  
\[
x = \frac{u - u_{\text{frost}}}{u_{\text{flüssig}} - u_{\text{frost}}}
\]  

Substituting values:  
\[
x = \frac{-316.50 \, \frac{\text{kJ}}{\text{kg}} + 333.492 \, \frac{\text{kJ}}{\text{kg}}}{-0.293 \, \frac{\text{kJ}}{\text{kg}} + 333.492 \, \frac{\text{kJ}}{\text{kg}}}
\]  
\[
x = \frac{16.942}{333.405} = 0.051
\]  

The final ice fraction is \( x = 0.051 \).