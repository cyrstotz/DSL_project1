TASK 3c  
The heat transfer \( Q_{12} \) from the gas to the ice-water mixture is calculated using the first law of thermodynamics applied to the gas side.  

The equation for heat transfer is:  
\[
Q = m \cdot c_V \cdot \Delta T
\]  
where \( m \) is the mass of the gas, \( c_V \) is the specific heat capacity at constant volume, and \( \Delta T \) is the temperature difference.  

A schematic diagram is shown with a box labeled "Gas" and an arrow labeled \( Q \) pointing outward, indicating heat transfer.  

The heat transfer equation is expressed as:  
\[
Q_{12} = m_g \cdot c_V \cdot (T_2 - T_1)
\]  
with the assumption that \( p_1 = p_2 \).  

The first law of thermodynamics applied to the piston is:  
\[
\dot{m} \cdot (U_2 - U_1) = \dot{Q}_{12} - W_{12}
\]  
where \( U \) represents internal energy, \( \dot{Q}_{12} \) is the heat transfer, and \( W_{12} \) is the work done.  

For an isochoric process (\( h = 0 \)) and assuming a perfect gas, the work term \( W_{12} \) is zero. The equation simplifies to:  
\[
R \cdot (T_2 - T_1) = \frac{8.314 \, \text{kJ}}{50 \, \text{kg}} \cdot (0.00326 \, \text{kg} \cdot 500^\circ\text{C}) = -83.14 \, \text{kJ}
\]  

The heat transfer \( Q_{12} \) is then calculated as:  
\[
Q_{12} = m_g \cdot c_p \cdot (T_2 - T_1)
\]  
where \( c_p = R + c_V \).  

Substituting values:  
\[
Q_{12} = m_g \cdot c_V \cdot (T_2 - T_1) + W_{12}
\]  
\[
Q_{12} = 0.00386 \, \text{kg} \cdot 0.633 \, \text{kJ/kg·K} \cdot (449.997 \, \text{K}) - 83.14 \, \text{kJ}
\]  

Final result:  
\[
Q_{12} = \ldots
\]  

(Note: The calculation is incomplete, and some parts are crossed out. The final numerical value for \( Q_{12} \) is not provided.)  

No additional diagrams or graphs are described.