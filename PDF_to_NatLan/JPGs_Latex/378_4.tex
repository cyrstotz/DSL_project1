TASK 2a  
The diagram provided is a qualitative representation of the jet engine process on a temperature-entropy (\( T \)-\( s \)) diagram. The graph includes labeled isobars and key states of the process:  
- State 1 represents the inlet conditions.  
- State 2 shows the compression process.  
- State 3 corresponds to the combustion chamber.  
- State 4 represents the turbine outlet.  
- State 5 is the mixing chamber.  
- State 6 is the nozzle exit.  

The isobars (\( p_0, p_1, p_2, p_3, p_4 \)) are clearly marked, and arrows indicate the direction of the thermodynamic processes. The diagram also shows the entropy increase and temperature changes during the various stages of the jet engine operation.

---

TASK 2b  
The following calculations are provided for determining the outlet velocity (\( w_6 \)) and temperature (\( T_6 \)):  

Given:  
\[
w_5 = 220 \, \text{m/s}, \quad p_5 = 0.5 \, \text{bar}, \quad T_5 = 431.9 \, \text{K}
\]

The energy balance equation is written as:  
\[
\frac{dE}{dt} = \dot{m} \left( h_5 - h_6 + \frac{w_6^2 - w_5^2}{2} \right) + \dot{Q} - \dot{W}
\]

Assumptions:  
- Reversible and adiabatic turbine operation.  
- Isentropic process with \( n = 1.4 \).  

The enthalpy change is calculated as:  
\[
\Delta h = c_p (T_5 - T_c) \quad \text{where} \quad T_c = T_5 \left( \frac{p_c}{p_5} \right)^{\frac{n-1}{n}}
\]

Substituting values:  
\[
T_c = 431.9 \left( \frac{0.191}{0.5} \right)^{\frac{1.4-1}{1.4}} = 328.07 \, \text{K}
\]
\[
\Delta h = 1.006 \cdot (431.9 - 328.07) = 104.488 \, \text{kJ/kg}
\]

The outlet velocity (\( w_6 \)) is calculated using:  
\[
w_6 = \sqrt{w_5^2 + \Delta h \cdot 2}
\]
\[
w_6 = \sqrt{220^2 + 104.488 \cdot 2} = 498.79 \, \text{m/s}
\]