TASK 4a  
A graph is drawn representing the freeze-drying process on a pressure-temperature (\( p \)-\( T \)) diagram. The axes are labeled as follows:  
- \( p \) [bar] on the vertical axis.  
- \( T \) [K] on the horizontal axis.  

The graph includes the following features:  
- A curved line labeled "isobaric evaporation" connecting points 1 and 4.  
- Points 1, 2, 3, and 4 are marked along the curve, with arrows indicating transitions between states.  
- The region labeled "Nassdampf Gebiet" (wet steam region) is shown.  
- Another region labeled "gesättigte Flüssigkeit" (saturated liquid) is indicated.  

TASK 4b  
The energy balance for the cooling process is written as:  
\[
\dot{Q}_K = \dot{m} \left[ h_e - h_a \right]
\]  
where \( h_e \) and \( h_a \) are enthalpies at specific states.  

Another equation is provided:  
\[
Q_{\text{in}} = \left[ h_4 - h_2 \right] + Q_K
\]  

The enthalpy difference is expressed as:  
\[
\dot{Q}_K = \dot{m} \left[ h_2 - h_1 \right]
\]  

For the transition from state 2 to state 3:  
- \( s \) is constant (\( s_2 = s_3 \)).  
- \( s_3 \) corresponds to 8 bar.  

The work done by the compressor is given as:  
\[
W_K = 28 \, \text{W}
\]  

The mass flow rate equation is written as:  
\[
\dot{m} \left[ h_2 - h_3 \right] = 28 \, \text{W}
\]  

TASK 4ii  
The sublimation temperature is calculated as follows:  
- At 466.5 mbar: \( T = 273.15 \, \text{K} \).  
- At 10 K above the sublimation temperature: \( T = -30^\circ\text{C} \).  
- Converted to Kelvin: \( T = 243.15 \, \text{K} \).  

The temperature of the food is noted as:  
\[
T_{\text{food}} = 237.15 \, \text{K}
\]  

No additional diagrams or figures are present for this task.