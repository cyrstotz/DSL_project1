TASK 2a  
The process is represented on a \( T \)-\( s \) diagram with labeled isobars and states. The diagram includes the following features:  
- States 0, 2, 3, 4, 5, and 6 are marked.  
- The process transitions between states with arrows indicating the direction of flow.  
- Isobaric lines are labeled for \( p_5 = 0.5 \, \text{bar} \) and \( p_6 = p_0 - 0.191 \, \text{bar} \).  
- The axes are labeled as \( T \) (temperature in Kelvin) on the vertical axis and \( s \) (specific entropy in \( \frac{\text{kJ}}{\text{kg·K}} \)) on the horizontal axis.  

TASK 2b  
The energy balance for the steady flow process is derived. The temperature \( T_6 \) is calculated using the following equation:  
\[
\frac{T_6}{T_5} = \left( \frac{p_6}{p_5} \right)^{\frac{\kappa - 1}{\kappa}} \implies T_6 = T_5 \cdot \left( \frac{p_6}{p_5} \right)^{\frac{\kappa - 1}{\kappa}}
\]  
Substituting values:  
\[
T_6 = 431.9 \, \text{K} \cdot \left( \frac{0.191 \, \text{bar}}{0.5 \, \text{bar}} \right)^{\frac{0.4}{1.4}}
\]  
\[
T_6 = 328.07 \, \text{K}
\]  

TASK 2c  
The energy balance for the steady flow process is expressed as:  
\[
0 = \dot{m} \left( h_5 - h_6 + \frac{w_5^2 - w_6^2}{2} \right)
\]  
Simplified to:  
\[
0 = h_5 - h_6 + \frac{w_5^2 - w_6^2}{2}
\]  
Additional terms related to heat transfer and dissipation are crossed out, indicating an adiabatic and steady-state process.