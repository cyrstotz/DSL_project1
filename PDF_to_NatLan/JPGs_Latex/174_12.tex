TASK 3c  
The temperature difference between the gas and the ice-water mixture is calculated as:  
\[
\Delta T = T_{g,1} - T_{\text{EW},2} = 500^\circ\text{C} - 0^\circ\text{C} = 773.15 \, \text{K} - 273.15 \, \text{K} = 500 \, \text{K}.
\]

The gas volume at state 2 is determined using the ideal gas law:  
\[
V_2 = \frac{m R T_2}{M p_2} = \frac{3.425 \cdot 8314 \, \text{kJ/kmol·K} \cdot 273.15 \, \text{K}}{50 \, \text{kg/kmol} \cdot 1.401 \, \text{bar}} = 1.108 \, \text{L}.
\]

The work done by the gas during the volume change is given by:  
\[
W_{12} = \int_{V_1}^{V_2} p \, dV = 1.401 \, \text{bar} \cdot (V_2 - V_1) = 1.401 \cdot (1.108 - 0.0041) = 1.401 \cdot 1.104 \, \text{L}.
\]

The heat transferred from the gas to the ice-water mixture is calculated as:  
\[
Q_{12} = c_V \cdot m \cdot \Delta T = 0.633 \, \text{kJ/kg·K} \cdot 3.425 \, \text{kg} \cdot 500 \, \text{K} = 1.081 \, \text{kJ}.
\]

TASK 3d  
The internal energy of the ice-water mixture is calculated using the formula:  
\[
U = U_s + x (U_f - U_s),
\]
where \( x \) is the ice fraction. Solving for \( x \):  
\[
x = \frac{U - U_s}{U_f - U_s}.
\]

Substituting values:  
\[
x = \frac{-315.785 + 333.458}{-0.045 + 333.458}.
\]

The internal energy per unit mass of the gas is given by:  
\[
U = -\frac{Q_{12}}{m_g} = -\frac{315.785}{m_g}.
\]

The ice fraction \( x \) is calculated as:  
\[
x = 0.0529.
\]

The final ice fraction \( x_{\text{ice},2} \) is expressed as a percentage:  
\[
x_{\text{ice},2} = 5.29\%.
\]