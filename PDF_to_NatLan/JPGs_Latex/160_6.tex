TASK 3a  
The problem begins with the ideal gas law \( pV = mRT \). The gas constant \( R \) is calculated as:  
\[
R = \frac{R_u}{M_g} = \frac{8.314 \, \text{kJ/kmol·K}}{50 \, \text{kg/kmol}} = 0.16628 \, \text{kJ/kg·K}.
\]  

Given initial conditions:  
- \( V_{g,1} = 3.14 \, \text{L} \),  
- \( T_{g,1} = 500^\circ\text{C} \),  
- \( m_{\text{EW}} = 0.1 \, \text{kg} \).  

The pressure \( p_{g,1} \) is determined using the equilibrium of forces acting on the piston. The equation for \( p_{g,1} \) is:  
\[
p_{g,1} \cdot A = p_{\text{amb}} \cdot A + m_K \cdot g + m_{\text{EW}} \cdot g,
\]  
where \( A \) is the cross-sectional area of the cylinder.  

The area \( A \) is calculated as:  
\[
A = \frac{D^2 \pi}{4} = \frac{(0.1 \, \text{m})^2 \pi}{4} = 0.007853983 \, \text{m}^2.
\]  

Substituting values:  
\[
p_{g,1} = \frac{p_{\text{amb}} \cdot A + m_K \cdot g + m_{\text{EW}} \cdot g}{A}.
\]  
\[
p_{g,1} = \frac{1.4 \cdot 10^5 \, \text{N/m}^2 + 32 \cdot 9.81 \, \text{N} + 0.1 \cdot 9.81 \, \text{N}}{0.007853983 \, \text{m}^2} = 140034.4406 \, \text{N/m}^2 \approx 1.4 \, \text{bar}.
\]  

The gas mass \( m_{g,1} \) is calculated using the ideal gas law:  
\[
m_{g,1} = \frac{p_{g,1} \cdot V_{g,1}}{R \cdot T_{g,1}}.
\]  
Substituting values:  
\[
m_{g,1} = \frac{1.4 \cdot 10^5 \, \text{Pa} \cdot 3.14 \cdot 10^{-3} \, \text{m}^3}{0.16628 \, \text{kJ/kg·K} \cdot (500 + 273.15) \, \text{K}} = 0.00634 \, \text{kg} \approx 3.422 \, \text{g}.
\]  

---

TASK 3b  
The ice fraction in state 2 is given as \( x_{\text{ice},2} \geq 0 \). The initial ice fraction is \( x_{\text{ice},1} = 0.6 \).  

The mass of ice \( m_{\text{ice}} \) is calculated as:  
\[
m_{\text{ice}} = x_{\text{ice},1} \cdot m_{\text{EW}} = 0.6 \cdot 0.1 \, \text{kg} = 0.06 \, \text{kg}.
\]  

Thermodynamic reasoning:  
Since the densities of ice and water are equal, the mass (and volume) of the ice-water mixture does not change. From the equilibrium condition, it is evident that \( p_{g,2} = p_{g,1} \). Thus:  
\[
p_{g,2} = 1.4 \, \text{bar}.
\]