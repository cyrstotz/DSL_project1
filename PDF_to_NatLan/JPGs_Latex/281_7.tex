TASK 3b  
The temperature \( T_{g,2} \) is equal to \( T_{\text{EW},2} \), which is \( 0^\circ\text{C} \). This is because the ice fraction \( x > 0 \). If the temperature were above \( 0^\circ\text{C} \), the ice fraction \( x_{\text{ice}} \) would be \( 0 \).  

The gas pressure \( p_{g,2} \) can be calculated using the ideal gas law:  
\[
p_{g,2} V_{g,2} = m_g R T_{g,2}
\]  
Rearranging for \( p_{g,2} \):  
\[
p_{g,2} = \frac{m_g R T_{g,2}}{V_{g,2}}
\]  
Substituting values:  
\[
p_{g,2} = \frac{3.6 \cdot 10^{-3} \cdot 8.314 \cdot (273.15)}{50 \cdot 10^{-3}}
\]  

TASK 3c  
Ice and water are incompressible. Therefore, the pressure \( p_{g,2} \) is equal to \( p_{g,1} \), which is \( 1.5 \, \text{bar} \).