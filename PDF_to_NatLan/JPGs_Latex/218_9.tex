TASK 3a  
The force \( F_n \) acting on the piston is calculated as the sum of the gravitational forces on the piston and the ice-water mixture, plus the atmospheric pressure acting on the piston:  
\[
F_n = m_K \cdot g + m_{\text{EW}} \cdot g + p_0
\]  
Substituting the given values:  
\[
m_K = 32 \, \text{kg}, \quad m_{\text{EW}} = 0.1 \, \text{kg}, \quad g = 9.81 \, \text{m/s}^2, \quad p_0 = 1 \cdot 10^5 \, \text{Pa}
\]  
\[
F_n = 32 \cdot 9.81 + 0.1 \cdot 9.81 + 1 \cdot 10^5 = 314.98 \, \text{N} + 1 \cdot 10^5 \, \text{Pa}
\]  

The gas pressure \( p_g \) is calculated using the formula:  
\[
p_g = \frac{F_n}{A}
\]  
where the area \( A \) of the piston is determined as:  
\[
A = r^2 \pi = (0.05 \, \text{m})^2 \pi = 0.00785 \, \text{m}^2
\]  
Substituting the values:  
\[
p_g = \frac{314.98 + 1 \cdot 10^5}{0.00785} = 1.4 \, \text{bar}
\]  

The mass of the gas \( m_g \) is calculated using the ideal gas law:  
\[
m_g = \frac{p_g \cdot V}{R \cdot T}
\]  
Substituting the given values:  
\[
p_g = 1.5 \, \text{bar} = 1.5 \cdot 10^5 \, \text{Pa}, \quad T = 773 \, \text{K}, \quad V = 3.14 \cdot 10^{-3} \, \text{m}^3
\]  
The gas constant \( R \) is calculated as:  
\[
R = \frac{8.314 \, \text{kJ/kmol·K}}{M}, \quad M = 50 \, \text{kg/kmol}
\]  
\[
R = \frac{8.314}{50} = 166.28 \, \text{J/kg·K}
\]  
Substituting into the ideal gas law:  
\[
m_g = \frac{1.5 \cdot 10^5 \cdot 3.14 \cdot 10^{-3}}{166.28 \cdot 773} = 3.666 \, \text{g}
\]  

TASK 3b  
No content found.