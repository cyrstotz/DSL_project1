TASK 3d  
The final ice fraction \( x_{\text{ice},2} \) is calculated using the following relationships:  

1. The ice mass fraction \( x_{\text{ice}} \) is defined as:  
\[
x_{\text{ice}} = \frac{m_{\text{ice}}}{m_{\text{EW}}}
\]  

2. The total energy change \( \Delta U \) is zero, and the internal energy change of the ice-water mixture (\( \Delta U_{\text{EW}} \)) is equal to the internal energy change of the gas (\( \Delta U_g \)):  
\[
\Delta U_g = \Delta U_{\text{EW}}
\]  

3. The internal energy change of the ice-water mixture is expressed as:  
\[
\Delta U_{\text{EW}} = 3.16 \times 10^3 + U_{1,\text{EW}} = U_{2,\text{EW}}
\]  

4. Using the energy balance:  
\[
Q_{12} = -3.16 \times 10^3 + 1 - 0.6 \cdot 45 + 0.6 \cdot (-333 + 158 + 0.6 \cdot 45) = -546.56 \, \text{kJ}
\]  

From Table 1:  
\[
U_{2,\text{EW}} - U_{1,\text{EW}} = -546.56 + 0.083
\]  
\[
U_{\text{ice}} - U_{\text{liquid}} = -333.442 + 0.083
\]  

5. The internal energy change is expressed as:  
\[
\Delta U = G_{12} + U_{12}
\]  

6. The internal energy of state 2 is calculated as:  
\[
\Delta U_{2,\text{EW}} = \frac{1500}{m} \rightarrow U_2 = \frac{1500}{m} + U_1
\]  

7. The internal energy of state 2 is further expanded:  
\[
U_2 = U_{\text{fluid}} + x_2 (U_{\text{ice}} - U_{\text{fluid}})
\]  
\[
U_1 = -0.6 \cdot 45 + 0.6 \cdot (-333 + 158 + 0.6 \cdot 45) = -200.083
\]  

8. The ice fraction \( x_2 \) is calculated as:  
\[
x_2 = \frac{Q_{12} + U_1 - U_{\text{fluid}}}{U_{\text{ice}} - U_{\text{fluid}}}
\]  
\[
x_2 = \frac{-1500}{0.1} - 200.083 + 0.083}{-333.442 + 0.083}
\]  

Finally, it is concluded that:  
\[
x_{\text{ice},2} \text{ is greater than } 0.6 \text{ since the temperature increases.}
\]