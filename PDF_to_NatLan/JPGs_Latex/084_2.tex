TASK 4a  
A pressure-temperature (\(p\)-\(T\)) diagram is drawn. The diagram shows the phase regions for a substance, including the liquid, vapor, and superheated gas phases. The curve represents the saturation line, with a labeled triple point ("TP") and two steps marked as "i" and "ii". The x-axis is labeled as temperature (\(T\)), and the y-axis is labeled as pressure (\(p\)). The region below the saturation line is labeled "Nassdampf" (wet steam), and the region above is labeled "Gas".

---

TASK 4c  
The vapor quality (\(x_1\)) immediately after expansion is calculated using the formula:  
\[
x_1 = \frac{S - S_f}{S_g - S_f}
\]  
where \(S\) is the specific entropy, \(S_f\) is the entropy of the saturated liquid, and \(S_g\) is the entropy of the saturated vapor.

---

TASK 4d  
The coefficient of performance (\(\epsilon\)) is calculated using the formula:  
\[
\epsilon = \frac{\dot{Q}_K}{\dot{W}_K} = \frac{\lvert \dot{Q}_K \rvert}{\lvert \dot{Q}_{ab} \rvert - \lvert \dot{Q}_K \rvert}
\]  
where \(\dot{Q}_K\) is the heat removed, \(\dot{W}_K\) is the work input, and \(\dot{Q}_{ab}\) is the heat absorbed.