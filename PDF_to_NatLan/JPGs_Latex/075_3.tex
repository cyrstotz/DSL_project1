TASK 2a  
The process is represented qualitatively in a \( T \)-\( s \) diagram. The diagram includes labeled isobars and key states (0, 1, 2, 3, 4, 5, and 6).  
- The curve begins at state 0 and progresses through states 1, 2, 3, 4, 5, and 6.  
- State transitions are marked with processes such as "isentropic" and "isobaric."  
- The axes are labeled: \( T \) (temperature) on the vertical axis and \( s \) (specific entropy) on the horizontal axis.  
- The isobars are shown as curved lines, and the transitions between states are indicated with arrows.  

TASK 2b  
To calculate the outlet velocity \( w_6 \) and temperature \( T_6 \):  
Given:  
\[
w_5 = 220 \, \text{m/s}, \quad p_5 = 0.5 \, \text{bar}, \quad T_5 = 431.9 \, \text{K}, \quad p_0 = 0.191 \, \text{bar}
\]  
The process is assumed to be isentropic (\( s_5 = s_6 \)).  

The polytropic temperature \( T_6 \) is calculated using:  
\[
T_6 = T_5 \left( \frac{p_0}{p_5} \right)^{\frac{\kappa - 1}{\kappa}}
\]  
Substituting values:  
\[
T_6 = 431.9 \left( \frac{0.191}{0.5} \right)^{\frac{1.4 - 1}{1.4}} = 328.07 \, \text{K}
\]  

The outlet velocity \( w_6 \) is determined using the energy balance:  
\[
\frac{w_6^2}{2} = c_p (T_5 - T_6) + \frac{w_5^2}{2}
\]  
Substituting values:  
\[
\frac{w_6^2}{2} = 1.006 (431.9 - 328.07) + \frac{220^2}{2}
\]  
\[
w_6^2 = 2304.45 + 77.95
\]  
\[
w_6 = \sqrt{220.7} \, \text{m/s}
\]  

Final results:  
\[
T_6 = 328.07 \, \text{K}, \quad w_6 = 220.7 \, \text{m/s}
\]