TASK 2a  
A graph is drawn representing a temperature-entropy (T-s) diagram. The axes are labeled as follows:  
- The vertical axis is labeled \( T \, [K] \), representing temperature in Kelvin.  
- The horizontal axis is labeled \( S \, \left[\frac{\text{kJ}}{\text{kg·K}}\right] \), representing entropy in kilojoules per kilogram per Kelvin.  

The diagram includes the following features:  
- A curve starting at point \( 0 \), rising to point \( 3 \), and labeled as "isobar (geht durch 3)" (isobar passing through point 3).  
- A downward slope from point \( 3 \) to point \( 4 \).  
- A curve rising again from point \( 4 \) to point \( 5 \), labeled "isobar."  
- A vertical line from point \( 5 \) to point \( 6 \), labeled "isobar."  

TASK 2b  
The problem involves calculating the outlet velocity \( w_6 \) and temperature \( T_6 \).  

The entropy values are given as \( S_5 = S_6 \). The pressure at state 6 is provided:  
\[
p_6 = 0.191 \, \text{bar}.
\]  

The temperature ratio is calculated using the isentropic relation:  
\[
\frac{T_6}{T_5} = \left(\frac{p_6}{p_5}\right)^{\frac{n-1}{n}} \implies T_6 = \left(\frac{p_6}{p_5}\right)^{\frac{0.4}{1.4}} \cdot T_5.
\]  

Substituting values:  
\[
T_6 = \left(\frac{p_6}{p_5}\right)^{\frac{0.4}{1.4}} \cdot T_5 = 32.07 \, \text{K}.
\]  

The outlet velocity is given as:  
\[
w_6 = 220 \, \text{m/s}.
\]  

Additional note: "Durchschnitt Dose" (average value) is written near the temperature calculation.