TASK 4a  
A p-T diagram is drawn with labeled axes: pressure \( p \) on the vertical axis and temperature \( T \) in Kelvin on the horizontal axis. The diagram depicts a cycle with four states labeled as 1, 2, 3, and 4.  
- The transitions between states 1 and 2, and states 3 and 4, are marked as "isobar" (constant pressure).  
- The cycle forms a closed loop, indicating a thermodynamic process.  
- A smaller, unlabeled sketch of the cycle appears to the right of the main diagram.

---

TASK 4b  
The equation for energy balance is written as:  
\[
0 = \dot{m}_R \left[ h_2 - h_3 \right] + \dot{Q}_{23} - \dot{W}_K
\]  
Where:  
- \( h_3(8 \, \text{bar}, x_3 = 1) = h_g(8 \, \text{bar}) = 93.42 \, \text{kJ/kg} \)  
- \( x_3 = 1 \) because the refrigerant is fully evaporated (isentropic condition).  

The entropy at state 3 is given as:  
\[
s_{3f} = s_{3g} = 0.385 \, \text{kJ/(kg·K)}
\]  
This is valid because the process is isentropic.  

The enthalpy at state 2 (\( h_2 \)) is left incomplete.  

A small arrow diagram is drawn, representing the direction of energy flow.