TASK 3a  
The force equilibrium is used to determine \( p_{g,1} \):  
\[
p_{g,1} \cdot \left(\frac{D}{2}\right)^2 \pi = p_{\text{amb}} \cdot \left(\frac{D}{2}\right)^2 \pi + m_K \cdot g
\]  
Rearranging gives:  
\[
p_{g,1} = p_{\text{amb}} + \frac{m_K \cdot g}{\left(\frac{D}{2}\right)^2 \pi}
\]  
Substituting values:  
\[
p_{g,1} = 100{,}000 \, \text{Pa} + \frac{32 \, \text{kg} \cdot 9.81 \, \text{N/kg}}{\left(0.05 \, \text{m}\right)^2 \pi}
\]  
This results in:  
\[
p_{g,1} = 1.3997 \, \text{bar}
\]  

TASK 3b  
Using the ideal gas law:  
\[
m_g = \frac{p_{g,1} \cdot V_{g,1}}{R_g \cdot T_{g,1}}
\]  
The specific gas constant \( R_g \) is calculated as:  
\[
R_g = \frac{\bar{R}}{M_g}
\]  
Substituting values:  
\[
R_g = \frac{8.314 \, \text{kJ/(kmol·K)}}{50 \, \text{kg/kmol}} = 166.3 \, \text{J/(kg·K)}
\]  

Now, substituting into the mass equation:  
\[
m_g = \frac{139970 \, \text{Pa} \cdot 3.14 \cdot 10^{-3} \, \text{m}^3}{166.3 \, \text{J/(kg·K)} \cdot 773.15 \, \text{K}}
\]  
This results in:  
\[
m_g = 3.42 \, \text{g}
\]  

No diagrams or graphs are present on the page.