TASK 4d  
The coefficient of performance \( \epsilon_K \) is defined as:  
\[
\epsilon_K = \frac{\lvert \dot{Q}_{\text{zu}} \rvert}{\lvert \dot{W}_K \rvert}
\]  
Here, \( \dot{Q}_{\text{zu}} \) represents the useful heat transfer, and \( \dot{W}_K \) is the work input to the system. The term "nutz" refers to the useful energy, while "Aufwand" refers to the effort or input energy. A note indicates a value of \( 28 \, \text{W} \) for \( \dot{W}_K \).  

TASK 4b  
The mass flow rate of the refrigerant \( \dot{m}_{\text{R134a}} \) is calculated using the energy balance equation:  
\[
0 = \dot{m} \cdot (h_e - h_a) + \dot{Q} - \dot{W}
\]  
Rearranging for \( \dot{m} \):  
\[
\dot{m} = \frac{\dot{W}_K}{h_2 - h_3}
\]  
Here, \( h_2 \) corresponds to the enthalpy at state 2, and \( h_3 \) corresponds to the enthalpy at state 3.  

Additional notes suggest:  
- \( h_2 \) is evaluated at \( x_2 = 1 \) and \( T_2 \).  
- \( h_3 \) is evaluated at \( p_3 = 8 \, \text{bar} \) and \( T_3 \).  

TASK 4c  
The vapor quality \( x \) is calculated using the formula:  
\[
\Phi = \Phi_f + x \cdot (\Phi_g - \Phi_f)
\]  
Rearranging for \( x \):  
\[
x = \frac{q - \Phi_f}{\Phi_g - \Phi_f}
\]  
Here, \( \Phi_f \) and \( \Phi_g \) represent the specific properties of the fluid and gas phases, respectively.  

TASK 4d (Diagram Description)  
A sketch shows a simplified refrigeration cycle with labeled states:  
- State 2: inlet to the compressor.  
- State 3: outlet from the compressor.  
The process is marked as adiabatic. The diagram visually represents the flow of refrigerant through the compressor.  

