TASK 4a  
A pressure-temperature (\(p\)-\(T\)) diagram is drawn to represent the freeze-drying process. The diagram includes phase regions and key states labeled as 1, 2, and 3.  
- The curve labeled "Triple" represents the triple point of the substance.  
- State 1 is marked on the curve at a lower temperature and pressure.  
- State 2 is shown at a higher temperature (\(T_z\)) and pressure (\(p_z\)).  
- State 3 is connected to State 2 via a dashed line, indicating an isentropic process.  
The axes are labeled as \(p\) (pressure) on the vertical axis and \(T\) (temperature) on the horizontal axis.

---

TASK 4b  
The energy balance equation is written as:  
\[
0 = \dot{m}(h) + \dot{Q} - \dot{W}^o
\]  
Simplified to:  
\[
0 = \dot{m}(h) + \dot{Q} - W
\]  
The work \(W\) is expressed as:  
\[
W = \dot{m}(h_2 - h_3)
\]  
The mass flow rate \(\dot{m}\) is calculated as:  
\[
\dot{m} = \frac{W}{h_2 - h_3}
\]  
The enthalpy \(h_2\) is defined as:  
\[
h_2 = h_g(T = T_z)
\]  
For an isentropic process, the entropy at State 3 equals the entropy at State 2:  
\[
s_3 = s_2
\]  
The temperature \(T_z\) is calculated using the isentropic relation:  
\[
\frac{p_3}{p_2} = \left(\frac{T_3}{T_2}\right)^{\frac{n-1}{n}} \quad \Rightarrow \quad T_z = T_2 \left(\frac{p_2}{p_3}\right)^{\frac{n-1}{n}}
\]

---

TASK 4d  
The coefficient of performance (\(\epsilon\)) is defined as:  
\[
\epsilon = \frac{\dot{Q}_K}{\dot{W}_K}
\]  
This equation is written but not further elaborated.