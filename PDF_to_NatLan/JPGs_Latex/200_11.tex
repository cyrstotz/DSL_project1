TASK 4b  
The refrigerant is at state 3 with a pressure \( p = 8 \, \text{bar} \) and specific entropy \( s = 0.9298 \, \frac{\text{kJ}}{\text{kg·K}} \). Using the tables, the enthalpy at state 3 is calculated as:  
\[
h_3 = 264.15 - 273.66 \cdot \frac{(0.9298 - 0.9066)}{(0.9066 - 0.9324)} + 267.75 = 277.37 \, \frac{\text{kJ}}{\text{kg}}
\]

The enthalpy at state 2 is determined for \( T = -16^\circ \text{C} \):  
\[
h_2 = h_g(T = -16^\circ \text{C}) = 237.74 \, \frac{\text{kJ}}{\text{kg}}
\]

The mass flow rate of the refrigerant is calculated using the heat transfer rate \( \dot{Q}_K = 28 \, \text{kW} \):  
\[
\dot{m}_{\text{R134a}} = \frac{\dot{Q}_K}{h_3 - h_2} = \frac{28}{277.37 - 237.74} = 0.834 \, \frac{\text{kg}}{\text{s}} = 3 \, \frac{\text{kg}}{\text{h}}
\]

TASK 4c  
The process is illustrated with a graph showing the enthalpy \( h \) versus pressure \( p \).  
- At state 3: \( p = 8 \, \text{bar} \), \( h = 277.37 \, \frac{\text{kJ}}{\text{kg}} \).  
- At state 1: \( p = 1.5748 \, \text{bar} \), \( h = 93.42 \, \frac{\text{kJ}}{\text{kg}} \).  

The enthalpy at state 1 is determined as \( h_f(8 \, \text{bar}) = 93.42 \, \frac{\text{kJ}}{\text{kg}} \). The process is isenthalpic due to the throttling.

Using Table A-10 for \( p = 1.5748 \, \text{bar} \), the vapor quality \( x_1 \) is calculated:  
\[
x_1 = \frac{h_1 - h_f}{h_g - h_f} = \frac{93.42 - h_f}{h_g - h_f} = 0.308
\]  
This corresponds to a vapor quality of \( x_1 = 0.308 \).