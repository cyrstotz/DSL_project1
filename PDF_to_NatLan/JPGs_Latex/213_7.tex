TASK 4a  
The diagram is a pressure-temperature (\( p \)-\( T \)) graph showing the freeze-drying process. It includes labeled phase regions: "Solid I" and "Fluid". The curve represents the phase boundary between solid and fluid. The process steps are marked as follows:  
- Point 1: Isobaric evaporation.  
- Point 2: Reversible adiabatic compression.  
- Point 3: Isobaric condensation.  

The temperature axis (\( T \)) is labeled in Kelvin, and the pressure axis (\( p \)) is labeled in unspecified units. A horizontal scale indicates a temperature difference of 10 K.  

TASK 4b  
The pressure at state 3 (\( p_3 \)) is equal to the pressure at state 4 (\( p_4 \)).  
\[
p_3 = p_4
\]  

The pressure at state 1 (\( p_1 \)) is equal to the pressure at state 2 (\( p_2 \)).  
\[
p_1 = p_2
\]  

The temperature of the refrigerant (\( T_q \)) is calculated as:  
\[
T_q = 31.33^\circ \text{C}
\]  

The enthalpy at state 4 (\( h_4 \)) is equal to the enthalpy at state 3 (\( h_3 \)), which corresponds to the enthalpy of saturated liquid at 8 bar:  
\[
h_4 = h_3 = h_f(8 \, \text{bar}) = 93.42 \, \text{kJ/kg}
\]  

The temperature at state 2 (\( T_2 \)) is equal to the temperature at state 1 (\( T_1 \)):  
\[
T_2 = T_1
\]