TASK 4a  
A graph is drawn representing the freeze-drying process on a pressure-temperature (\(p\)-\(T\)) diagram.  
- The curve labeled "isobar" represents constant pressure.  
- The phase regions are marked as "Gas" and "Flüssig" (Liquid).  
- The triple point (\(T_{\text{Triple}}\)) is indicated.  
- The process steps are labeled as follows:  
  - Step 1: Starting in the gas phase.  
  - Step 2: Moving isothermally towards the liquid phase.  
  - Step 3: Returning to the gas phase.  

TASK 4b  
The problem asks to calculate the required mass flow rate of the refrigerant \( \dot{m}_{\text{R134a}} \).  

Given data:  
- Isobaric evaporation of R134a.  
- Compressor work \( W_K = 28 \, \text{kW} \).  
- Evaporator temperature: \( T_{\text{Verdampfer}} = T_i - 6 \, \text{K} = 257.15 \, \text{K} \).  
- Initial temperature: \( T_i = -10^\circ\text{C} = 263.15 \, \text{K} \).  

Energy balance for the compressor:  
\[
0 = \dot{m} (h_e - h_a) + \dot{W}_{\text{K,n}}
\]  
\[
\dot{m} = \frac{\dot{W}_K}{h_e - h_a}
\]  

Additional notes:  
- \( h_e \): Enthalpy at \( T = -16^\circ\text{C} \) for R134a.  
- \( h_a \): Unknown enthalpy.  
- From Table A-10: At \( T = -16^\circ\text{C} \), \( h_g = 237.9 \, \text{kJ/kg} \).  
- Assumption: \( h_2 = h_e \).  

The enthalpy values need to be determined to proceed with the calculation.  

Crossed-out content:  
Some calculations and notes are crossed out and are not transcribed.