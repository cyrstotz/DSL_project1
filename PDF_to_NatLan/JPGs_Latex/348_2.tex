TASK 2c  
The mass-specific exergy destruction \( ex_{\text{dest}} \) is calculated as the difference between the exergy at state 6 and state 0, with reference to ambient conditions. The formula used is:  
\[
ex_{\text{dest}} = ex_{6} - ex_{0}
\]  
This is based on the ideal gas assumption and includes terms for enthalpy and entropy differences, as well as kinetic energy contributions.  

The detailed calculation is:  
\[
ex_{\text{dest}} = (h_6 - h_0) - T_0 (s_6 - s_0) + \frac{w_6^2}{2} - \frac{w_0^2}{2}
\]  
Substituting values:  
\[
T_0 = 243.15 \, \text{K}, \quad w_6 = 220 \, \text{m/s}, \quad w_0 = 200 \, \text{m/s}
\]  
\[
s_6 - s_0 = c_p \ln\left(\frac{T_6}{T_0}\right) = 85.46 \, \frac{\text{J}}{\text{kg·K}} \ln\left(\frac{431.9}{243.15}\right) = 30.14 \, \frac{\text{J}}{\text{kg·K}}
\]  
\[
ex_{\text{dest}} = 120.8 \, \frac{\text{kJ}}{\text{kg}}
\]  

---

TASK 2d  
The exergy destruction for the entire system is calculated using the exergy balance equation. The steady-state assumption is applied:  
\[
\dot{Q} = \dot{m} \Delta ex_{\text{st}} + \left(1 - \frac{T_0}{T_B}\right) \dot{Q}_B - \dot{W}_{\text{turbine}}
\]  
Substituting values:  
\[
\dot{Q}_{\text{total}} = -\Delta ex_{\text{st}} + C_p \left(\frac{T_0}{T_B}\right) \cdot 98 - 948.8 \, \text{kW}
\]  
The final result is:  
\[
\dot{Q}_{\text{total}} = 948.8 \, \text{kW}
\]  

Additional notes:  
- \( p_0 = p_6 \) is used in the calculation for the nozzle.  
- \( \dot{Q}_B \) is the heat input during combustion.  
