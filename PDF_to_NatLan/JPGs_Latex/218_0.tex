TASK 1a  
The goal is to determine the heat flow \( \dot{Q}_{\text{out}} \) removed by the coolant.  

Using the first law of thermodynamics:  
\[
0 = \dot{m}(h_e - h_a) + \dot{Q} - \dot{Q}_R
\]  
Rearranging for \( \dot{Q}_{\text{out}} \):  
\[
\dot{Q}_{\text{out}} = \dot{m}(h_a - h_e) - \dot{Q}_R
\]  

From the water tables:  
\[
h_e = h_f(70^\circ\text{C}, x=0) = 292.38 \, \text{kJ/kg} \quad \text{(Table A-2)}
\]  
\[
h_a = h_f(100^\circ\text{C}, x=0) = 419.04 \, \text{kJ/kg} \quad \text{(Table A-2)}
\]  

Substituting values:  
\[
\dot{Q}_{\text{out}} = 0.3 \, \text{kg/s} \left( 419.04 - 292.38 \right) - 100 \, \text{kW}
\]  
\[
\dot{Q}_{\text{out}} = -62.18 \, \text{kJ/s} \, \text{(kW)}
\]  

---

TASK 1b  
The goal is to determine the thermodynamic mean temperature \( \bar{T}_{\text{KF}} \) of the coolant.  

The mean temperature is defined as:  
\[
\bar{T}_{\text{KF}} = \frac{\int_{s_a}^{s_e} T \, ds}{s_a - s_e}
\]  

Using the enthalpy-entropy relationship:  
\[
\bar{T}_{\text{KF}} = \frac{h_a - h_e}{s_a - s_e} \quad \text{(from the first law of thermodynamics)}
\]  

For an ideal liquid:  
\[
h_a - h_e = c_{\text{KF}}(T_2 - T_1)
\]  
\[
s_a - s_e = c_{\text{KF}} \ln \left( \frac{T_2}{T_1} \right)
\]  

Substituting these into the mean temperature formula:  
\[
\bar{T}_{\text{KF}} = \frac{c_{\text{KF}}(T_2 - T_1)}{c_{\text{KF}} \ln \left( \frac{T_2}{T_1} \right)}
\]  
\[
\bar{T}_{\text{KF}} = \frac{T_2 - T_1}{\ln \left( \frac{T_2}{T_1} \right)}
\]  

Substituting numerical values:  
\[
\bar{T}_{\text{KF}} = \frac{288.15 \, \text{K} - 278.15 \, \text{K}}{\ln \left( \frac{288.15}{278.15} \right)}
\]  
\[
\bar{T}_{\text{KF}} = 293.12 \, \text{K}
\]  

The thermodynamic mean temperature of the coolant is \( \bar{T}_{\text{KF}} = 293.12 \, \text{K} \).