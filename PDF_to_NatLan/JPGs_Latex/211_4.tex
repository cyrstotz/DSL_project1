TASK 4a  
Two diagrams are drawn to represent the freeze-drying process.  

1. **Diagram i**:  
   - The graph is a pressure-temperature (\(P\)-\(T\)) diagram.  
   - The labeled states (1, 2, 3, 4) are shown in the three-phase region.  
   - The process transitions between gas and liquid phases.  
   - A note states: "All states are in the three-phase region."  

2. **Diagram ii**:  
   - Another \(P\)-\(T\) diagram is shown with a focus on the liquid and gas phases.  
   - The temperature difference (\(\Delta T = 10 \, \text{K}\)) is highlighted.  
   - The process transitions between liquid and gas phases.  

TASK 4b  
The system around the compressor is analyzed.  

The energy balance equation is written as:  
\[
0 = \dot{m} (h_2 - h_3) - \dot{W}
\]  
The mass flow rate is expressed as:  
\[
\dot{m} = \frac{\dot{W}}{h_2 - h_3}
\]  

The process is labeled as adiabatic.  

Given values:  
\[
h_1 = h_4 = 93.42 \, \text{kJ/kg}
\]  
\[
h_2 = 249.53 \, \text{kJ/kg}
\]  
Temperatures:  
\[
T_i = 10^\circ\text{C} = 283.15 \, \text{K}
\]  
\[
T_v = 4^\circ\text{C} = 277.15 \, \text{K}
\]  

TASK 4c  
The vapor quality (\(x_1\)) is calculated using the formula:  
\[
x_1 = \frac{h_1 - h_f}{h_g - h_f}
\]  

For \(h_3\):  
\[
s_3 = s_2 = 0.9169 \, \text{kJ/kg·K}
\]  
This indicates the state is superheated (\( \text{überhitzt} \)).  

Interpolation is performed to find \(h_3\):  
\[
h_3 = 264.15 \, \text{kJ/kg} + \frac{1}{\dots} \cdot \frac{273.66 - 264.15}{0.9374 - 0.93066} \cdot (0.9169 - 0.93066)
\]  