TASK 4a  
A pressure-temperature (\(p\)-\(T\)) diagram is drawn, showing the phase regions of a substance. The diagram includes the following labeled features:  
- The "Fest" (solid) region on the left.  
- The "Flüssig" (liquid) region in the middle.  
- The "Gas" (gas) region on the bottom right.  
- The triple point is marked as "Tripel."  
- Two states, labeled as "1" and "2," are connected by a horizontal line within the liquid region.  
- The diagram is shaded to indicate phase boundaries and transitions.  

TASK 4b  
The enthalpy at state \(q\) is given as:  
\[
h_q = 93.92 \, \frac{\text{kJ}}{\text{kg}} \quad (\text{from Table A-11})
\]  
The enthalpy after throttling remains constant:  
\[
h_1 = h_q
\]  
This is noted with the explanation: "Throttling is isenthalpic."  

The inlet temperature \(T_i\) is determined from the diagram as:  
\[
T_i = -10^\circ\text{C}
\]  
The outlet temperature \(T_{w2}\) is calculated as:  
\[
T_{w2} = T_i - 6 \, \text{K} = -16^\circ\text{C}
\]  

The enthalpy at state \(h_2\) is calculated using tabulated values:  
\[
h_2 = 49.54 \, \frac{\text{kJ}}{\text{kg}} + 293.66 \, \frac{\text{kJ}}{\text{kg}} - 269.15 \, \frac{\text{kJ}}{\text{kg}} = 237.71 \, \frac{\text{kJ}}{\text{kg}}
\]  

The entropy at state \(s_2\) is calculated as:  
\[
s_2 = 6.2258 \, \frac{\text{kJ}}{\text{kg·K}} - s_3
\]  

TASK 4c  
The enthalpy values are used from part b:  
\[
h_q = 93.92 \, \frac{\text{kJ}}{\text{kg}} \quad (\text{from above})
\]  
\[
h_1 = h_q = 93.92 \, \frac{\text{kJ}}{\text{kg}}
\]  

The enthalpy at \(h_f(-6^\circ\text{C})\) is calculated as:  
\[
h_f(-6^\circ\text{C}) = 305.51 \, \frac{\text{kJ}}{\text{kg}} - (19.25 \, \frac{\text{kJ}}{\text{kg}})
\]  
\[
h_f(-6^\circ\text{C}) = 492.945 \, \frac{\text{kJ}}{\text{kg}}
\]  

The vapor quality \(x_1\) is calculated using the formula:  
\[
x_1 = \frac{h_1 - h_f}{h_g - h_f}
\]  
Substituting the values:  
\[
x_1 = \frac{93.92 - 492.945}{293.72 - 492.945} = 0.2599
\]  

An additional note explains that an error was made in earlier calculations due to using incorrect temperatures, which affected the results.