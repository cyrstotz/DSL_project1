TASK 4a  
A graph is drawn with the x-axis labeled as temperature \( T \) and the y-axis labeled as pressure \( p \). The graph shows phase regions labeled as "fest" (solid), "flüssig" (liquid), and "gasförmig" (gaseous). There are lines separating the regions, indicating phase boundaries. The graph includes annotations for states 1, 2, 3, and 4, with arrows showing transitions between these states.  

TASK 4b  
The energy balance equation is written as:  
\[
\frac{dE}{dt} = \sum_i \dot{m}_i \left( h_i + \frac{v_i^2}{2} + g z_i \right) + \sum_j \dot{Q}_j - \sum_n \dot{W}_n
\]  
Simplified further:  
\[
\frac{dE}{dt} = \dot{m}_i \left( h_e - h_a \right) + \dot{Q}_K - \dot{W}_K
\]  
The change in energy is expressed as:  
\[
\Delta E = \Delta U + \Delta KE + \Delta PE
\]  

The work and heat transfer relationship is given as:  
\[
\dot{W}_K - \dot{Q}_K = \dot{m}_{\text{R134a}} \left( h_e - h_a \right)
\]  
It is noted that \( \dot{W}_K = 26 \, \text{W} \) (from the task).  

TASK 4c  
The vapor quality \( x_1 \) is calculated using the formula:  
\[
x = \frac{d_g}{d_f}
\]  

TASK 4d  
The coefficient of performance \( \epsilon_K \) is expressed as:  
\[
\epsilon_K = \frac{\dot{Q}_K}{\dot{W}_K}
\]  
Expanded further:  
\[
\epsilon_K = \left[ \frac{\dot{Q}_{\text{zu}}}{\dot{W}_K} \right] - \left[ \frac{\dot{Q}_{\text{abl}}}{\dot{W}_K} \right]
\]  
Where:  
\[
\dot{Q}_K = \text{"Energieübertrag"}
\]  

The energy balance is revisited:  
\[
\dot{Q}_K = \dot{m} \left( h_e - h_a + \frac{w_e^2 - w_a^2}{2} + g \left( z_e - z_a \right) \right)
\]  
Simplified to:  
\[
\Delta E = \dot{Q}_K
\]  

TASK 4e  
The temperature \( T_i \) would continue to decrease until the triple point is reached, as energy is continuously removed.  

Additional notes include:  
\[
\dot{m} = 0, \quad T_2 = T_1, \quad p_2 = p_1
\]  
It is indicated that \( T_1 \to \infty \), \( T_2 \to T_1 \), and \( p_2 \to p_1 \).  

No further diagrams or content are described.