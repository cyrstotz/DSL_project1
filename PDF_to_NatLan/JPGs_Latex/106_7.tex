TASK 3b  
Since all the heat \( Q_{12} \) flows into the process of melting ice and some ice remains, it follows that \( T_{2,\text{ice}} = 0^\circ\text{C} \). Consequently, \( T_{2,\text{gas}} = 0^\circ\text{C} \) as well (due to thermal equilibrium). The pressure remains constant in this configuration:  
\[
p_{g,2} = p_{\text{amb}} = 1.4 \, \text{bar}.
\]

---

TASK 3c  
The energy balance is expressed as:  
\[
\Delta E = \Delta U + \text{KE} + \text{PE} = Q_{12} - W_{12}.
\]  
Neglecting kinetic and potential energy changes, the heat transfer \( Q_{12} \) is calculated as:  
\[
Q_{12} = c_V (T_2 - T_1) + p \Delta V.
\]  
Substituting values:  
\[
Q_{12} = -316.5 \, \text{kJ}.
\]

---

TASK 3d  
To determine the final ice fraction \( x_{\text{ice},2} \):  
The energy balance for the system is:  
\[
\Delta U = Q_{12} = m_2 u_2 - m_1 u_1.
\]  
The heat \( Q_{12} \) is distributed between the ice and water phases. Using the specific internal energy values:  
\[
u_2 = -185.05 \, \frac{\text{kJ}}{\text{kg}}.
\]  
From the tables:  
\[
u_1 = u_{\text{fluid}} + x_{\text{ice},1} (u_{\text{ice}} - u_{\text{fluid}}),
\]  
where:  
\[
u_1[1.4 \, \text{bar}, 0^\circ\text{C}] = u_{\text{fluid}} + x_{\text{ice},1} (u_{\text{ice}} - u_{\text{fluid}}) = -200.07 \, \frac{\text{kJ}}{\text{kg}}.
\]  
The final ice fraction is calculated as:  
\[
x_{\text{ice},2} = \frac{u_2 - u_{\text{fluid}}}{u_{\text{ice}} - u_{\text{fluid}}}.
\]  
Substituting values:  
\[
x_{\text{ice},2} = 0.555.
\]