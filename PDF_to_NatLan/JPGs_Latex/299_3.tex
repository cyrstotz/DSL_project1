TASK 3a  
The gas pressure \( p_{g,1} \) and mass \( m_g \) are to be determined using the ideal gas law:  
\[
p_g V_g = m_g R T_g
\]

---

TASK 3b  
The pressure \( p_{g,1} \) is calculated as \( 1.5 \, \text{bar} \). The mass \( m_g \) is determined as \( 3.6 \, \text{g} \). The temperature \( T_{g,2} \) is related to \( T_{g,1} \) by the following expression:  
\[
T_2 = T_1 \left( \frac{p_2}{p_1} \right)^{\frac{\kappa - 1}{\kappa}}
\]  
The pressure \( p_{g,2} \) is also calculated.

---

TASK 3c  
The heat transferred \( Q_{12} \) between states 1 and 2 is calculated using the energy balance:  
\[
Q_{12} = \dot{m} (h_1 - h_2) + \dot{Q}_{\text{ext}}
\]  
Since \( \dot{Q}_{\text{ext}} = 0 \), this simplifies to:  
\[
Q_{12} = -\dot{m} (h_1 - h_2)
\]  
Using specific heat capacity \( c_p \), the expression becomes:  
\[
Q_{12} = c_p (T_2 - T_1)
\]  
The specific heat capacity \( c_p \) is calculated as:  
\[
c_p = \frac{R}{M} + c_V
\]  
Substituting values:  
\[
c_p = \frac{8.314}{50} + 0.633 = 0.7992 \, \text{kJ/kg·K}
\]  
Finally, the heat transfer is calculated:  
\[
Q_{12} = c_p (T_2 - T_1) = 0.7992 (0.003^\circ\text{C} - 500^\circ\text{C}) = -399.6376 \, \text{W}
\]

---

TASK 3d  
The final ice fraction \( x_{\text{ice},2} \) is calculated using the enthalpy difference:  
\[
x_{\text{ice},2} = \frac{u_2 - u_f}{u_g - u_f}
\]  
Where:  
\[
u_2 = u(0.003^\circ\text{C}), \quad u_f = -0.033 \times 10^3 \, \text{kJ/kg}, \quad u_g = -333.492 \times 10^3 \, \text{kJ/kg}
\]  
Substituting these values gives the final ice fraction \( x_{\text{ice},2} \).