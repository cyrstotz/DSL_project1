TASK 1c  
The entropy production rate \( \dot{S}_{\text{gen}} \) is calculated using the entropy balance equation:  
\[
\dot{S}_{\text{gen}} = \dot{m} \left( s_{\text{out}} - s_{\text{in}} \right) + \frac{\dot{Q}_{\text{out}}}{T_{\text{mean}}}
\]  
where \( T_{\text{mean}} \) is the thermodynamic mean temperature of the coolant.  

The diagram shows a rectangular reactor with labeled inlet and outlet streams for the coolant (\( T_{\text{KF,in}} \) and \( T_{\text{KF,out}} \)), as well as the heat transfer \( \dot{Q}_{\text{out}} \) from the reactor to the coolant.  

From the water tables:  
\[
s_f(70^\circ\text{C}) = 0.9584 \, \text{kJ/kg·K}, \quad s_g(70^\circ\text{C}) = 7.7555 \, \text{kJ/kg·K}
\]  
\[
s_f(100^\circ\text{C}) = 1.3069 \, \text{kJ/kg·K}, \quad s_g(100^\circ\text{C}) = 7.3599 \, \text{kJ/kg·K}
\]  

For saturated liquid at \( T_{\text{Reactor}} = 100^\circ\text{C} \):  
\[
s = s_f + x \cdot (s_g - s_f) = 1.33779 \, \text{kJ/kg·K}
\]  

The entropy production rate is then calculated as:  
\[
\dot{S}_{\text{gen}} = \frac{\dot{Q}_{\text{out}}}{T_{\text{mean}}} = \frac{100 \, \text{kW}}{373.15 \, \text{K}} = 0.268 \, \text{kW/K}
\]  

---

TASK 1d  
The temperature of the reactor contents changes from \( T_1 = 100^\circ\text{C} \) (\( 373.15 \, \text{K} \)) to \( T_2 = 70^\circ\text{C} \) (\( 343.15 \, \text{K} \)). A mass \( \Delta m \) of saturated liquid water at \( T_{\text{in,12}} = 20^\circ\text{C} \) (\( 293.15 \, \text{K} \)) is added. The heat released during cooling is \( Q_{\text{out},12} = 35 \, \text{MJ} \).  

The energy balance equation is:  
\[
\Delta E = m_2 u_2 - m_1 u_1 = \Delta m \cdot h_{\text{in}} + Q_{\text{out},12}
\]  
where \( m_1 = 5755 \, \text{kg} \).  

From the water tables:  
\[
u_f(70^\circ\text{C}) = 292.95 \, \text{kJ/kg}, \quad u_f(100^\circ\text{C}) = 419.94 \, \text{kJ/kg}, \quad u_f(20^\circ\text{C}) = 83.96 \, \text{kJ/kg}
\]  

Substituting into the energy balance:  
\[
m_1 u_f(100^\circ\text{C}) + \Delta m \cdot u_f(20^\circ\text{C}) = m_1 u_f(70^\circ\text{C}) + Q_{\text{out},12}
\]  
Rearranging for \( \Delta m \):  
\[
\Delta m = \frac{m_1 \left( u_f(70^\circ\text{C}) - u_f(100^\circ\text{C}) \right) + Q_{\text{out},12}}{u_f(20^\circ\text{C})}
\]  

Substituting values:  
\[
\Delta m = \frac{5755 \cdot (292.95 - 419.94) + 35,000}{83.96} = 3637.9 \, \text{kg}
\]  

Thus, the mass of water added is \( \Delta m = 3637.9 \, \text{kg} \).