TASK 3a  
The problem involves determining the gas pressure \( p_{g,1} \) and mass \( m_g \) in state 1.  

The ideal gas law is applied:  
\[
p \cdot V = m \cdot R \cdot T
\]  
Rearranging for \( m_g \):  
\[
m_g = \frac{p \cdot V}{R \cdot T}
\]  
Given values:  
- \( R = \frac{R}{M} = \frac{166.28 \, \text{J/(kg·K)}}{0.050 \, \text{kg/mol}} \)  
- \( T = 738.15 \, \text{K} \)  
- \( V = 0.00314 \, \text{L} \)  

The calculated mass is:  
\[
m_g = 2.687 \, \text{g}
\]  

The pressure \( p_g \) is determined using the equation:  
\[
p_g \cdot A = p_0 \cdot A + m_g \cdot g
\]  
Where:  
- \( A = (0.1 \, \text{m})^2 \cdot \pi \)  
- \( p_g = 1.100 \, \text{bar} \).  

---

TASK 3b  
The problem involves determining \( T_{g,2} \), \( p_{g,2} \), and justifying the equilibrium condition.  

Given values:  
- \( p_0 = 1.56 \, \text{bar} \)  
- \( m_g = 3.6 \, \text{g} \)  

The ideal gas law is applied again:  
\[
p \cdot V = m \cdot R \cdot T
\]  

It is stated that \( p_2 = p_1 \), and the pressure must remain constant because the system is in equilibrium and the weight of the piston does not change.  

The temperature \( T \) is calculated using:  
\[
T = \frac{p \cdot V}{m \cdot R}
\]  

Additional explanation: "The pressure remains constant because the system is in equilibrium, and the weight of the piston does not change after the air pressure is adjusted."  

No further numerical results are provided for \( T_{g,2} \) or \( p_{g,2} \).  

---  
No diagrams or graphs are present on the page.