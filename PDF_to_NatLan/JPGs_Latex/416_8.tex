TASK 3a  
The gas pressure \( p_{g,1} \) is calculated using the equation:  
\[
p_{g,1} = p_{\text{amb}} + \frac{1}{A}(m_K g + m_{\text{EW}} g)
\]  
Substituting the values:  
\[
p_{g,1} = 120005 \, \text{Pa} = 1.20 \, \text{bar}
\]  
The cross-sectional area \( A \) of the cylinder is calculated as:  
\[
A = \pi (0.05 \, \text{m})^2 = 0.00785 \, \text{m}^2
\]  
The mass of the gas \( m_g \) is determined using:  
\[
m_g = \frac{p_{g,1} V_{g,1}}{R T_{g,1}}
\]  
Substituting the values:  
\[
m_g = 3422 \, \text{g}
\]  

---

TASK 3b  
The temperature of the gas in state 2 is equal to the temperature of the ice-water mixture:  
\[
T_{g,2} = T_{\text{EW},2} = 0.0^\circ\text{C}
\]  
The pressure of the gas in state 2 remains the same as in state 1:  
\[
p_{g,2} = p_{g,1}
\]  
Explanation:  
- The external pressure remains constant, as in state 1.  
- Since ice is still present in the water, all heat is used for melting the ice.  
- At equilibrium, the temperature does not change further, and the ice fraction remains constant.  

---

TASK 3c  
The energy balance for the gas is written as:  
\[
-\Delta U + w = Q_{12}
\]  
Substituting:  
\[
-m_g c_V (-500 \, \text{K}) + \int_{V_1}^{V_2} p_g \, dv = -1.367 \, \text{kJ}
\]  
The integral term is simplified:  
\[
\int_{V_1}^{V_2} p_g \, dv = p_g (V_2 - V_1)
\]  
The final volume \( V_2 \) is calculated using:  
\[
V_2 = \frac{m_g R T_{g,2}}{p_g M_g}
\]  
Substituting the values:  
\[
V_2 = 1.105 \, \text{L}
\]