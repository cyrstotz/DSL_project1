TASK 4a  
A graph is drawn with pressure \( p \) on the vertical axis and temperature \( T \) on the horizontal axis. The graph shows phase regions labeled as "gas," "flüssig" (liquid), and "mixed." The curve separates the gas and liquid regions, and points labeled \( 1 \), \( 2 \), and \( 3 \) are marked along the curve.  

TASK 4b  
A schematic diagram of a refrigeration cycle is shown. It includes three labeled states: \( 2 \), \( 3 \), and \( 1 \). Arrows indicate the flow direction, and a work input \( \dot{W}_K \) is shown entering the system.  

The equation for energy balance is written as:  
\[
0 = \dot{m} (h_2 - h_3) - \dot{W}_K
\]  
Rearranged, the mass flow rate \( \dot{m} \) is expressed as:  
\[
\dot{m} = \frac{\dot{W}_K}{h_2 - h_3}
\]  

Additional notes include:  
- \( h_2 \) and \( h_3 \) are enthalpies at states 2 and 3, respectively.  
- \( h_3 \) corresponds to 8 bar pressure.  
- \( s_2 = s_3 \) (isentropic process).  
- \( s_3 \) is superheated, and \( s_3 \) at 8 bar can be found in Table 12.  

TASK 4c  
The vapor quality \( x \) is calculated using the formula:  
\[
x = \frac{h_1 - h_f}{h_g - h_f}
\]  
where \( h_1 \) is the enthalpy at state 1, \( h_f \) is the enthalpy of saturated liquid, and \( h_g \) is the enthalpy of saturated vapor.  

TASK 4d  
The coefficient of performance \( \epsilon_K \) is defined as:  
\[
\epsilon_K = \frac{\lvert \dot{Q}_K \rvert}{\lvert \dot{W}_K \rvert} = \frac{\lvert \dot{Q}_K \rvert}{\lvert \dot{Q}_{\text{ab}} - \dot{Q}_K \rvert}
\]  
where \( \dot{Q}_K \) is the heat removed, and \( \dot{Q}_{\text{ab}} \) is the absorbed heat.  

TASK 4e  
The temperature would continue to decrease until it eventually stagnates because no more heat can be removed.