TASK 4a  
Two diagrams are drawn to represent thermodynamic processes:  

1. The first diagram is a pressure-enthalpy (\(p\)-\(h\)) graph. It shows two isobaric lines labeled \(p_1\) and \(p_2\), with enthalpy increasing along the x-axis. The diagram includes points labeled \(1\), \(2\), and \(A\), connected by curves. The process transitions from \(1\) to \(2\) along an isobaric line, and from \(2\) to \(A\) along another isobaric line.  

2. The second diagram is a temperature-enthalpy (\(T\)-\(h\)) graph. It shows two isothermal lines labeled \(T_1\) and \(T_2\), with enthalpy increasing along the x-axis. Points \(1\), \(2\), and \(A\) are connected by curves, representing transitions between states.  

TASK 4b  
The refrigerant mass flow rate (\(m_R\)) is calculated using an energy balance approach.  

A table is provided with the following information:  
- Row 1: "NO reg sat"  
- Row 2: "vap sat" — Reference to Table A-10 at state \(1\), with \(p_2 = p_1\).  
- Row 3: "VSupen." — Reference to Table A-15 at state \(2\), with pressure \(p = 8\).  
- Row 4: "liq sat" — Value \(31.33\).  

No further numerical calculations or explanations are visible.  

Descriptions of the diagrams and table are provided as written.