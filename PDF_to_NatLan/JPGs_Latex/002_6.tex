TASK 3a  
The initial conditions of the system are provided:  
- Gas temperature: \( T_{g,1} = 500^\circ\text{C} \)  
- Gas volume: \( V_{g,1} = 3.14 \, \text{L} \)  
- Specific heat capacity of the gas: \( c_V = 0.633 \, \text{kJ/kg·K} \)  
- Molar mass of the gas: \( M_g = 50 \, \text{kg/kmol} \)  
- Ice-water mixture mass: \( m_{\text{EW}} = 0.1 \, \text{kg} \)  
- Ice-water temperature: \( T_{\text{EW}} = 0^\circ\text{C} \)  
- Ice mass fraction: \( x_{\text{ice},1} = 0.6 \)  

The pressure of the gas in state 1 is calculated using the formula:  
\[
p_{g,1} = \frac{F}{A}
\]  
where:  
\[
A = \left( \frac{D}{2} \right)^2 \pi
\]  
\[
F = (m_{\text{EW}} + m_K) g + p_{\text{amb}} A
\]  
The resulting pressure is:  
\[
p_{g,1} = 1.4 \, \text{bar}
\]  

The mass of the gas is determined using the ideal gas law:  
\[
p_{g,1} V_{g,1} = m R T_{g,1}
\]  
Rearranging for \( m \):  
\[
m = \frac{p_{g,1} V_{g,1}}{R T_{g,1}}
\]  
Substituting values yields:  
\[
m = 3.42 \, \text{g}
\]  

---

TASK 3b  
The pressure of the gas remains constant:  
\[
p_{g,2} = p_{g,1} = 1.4 \, \text{bar}
\]  
The external pressure acting on the gas has not changed.  

The ice mass fraction in state 2 is greater than zero (\( x_{\text{ice},2} > 0 \)), meaning not all the ice has melted. Therefore, the temperatures of the gas and the ice-water mixture are equal in state 2:  
\[
T_{\text{EW},2} = T_{g,2}
\]  
The temperature in state 2 is:  
\[
T_{g,2} = T_{\text{EW},2} = 0^\circ\text{C}
\]  

---

TASK 3c  
The energy balance is expressed as:  
\[
dU = Q - W
\]  
The change in internal energy is calculated as:  
\[
u_2 - u_1 = c_V (T_2 - T_1)
\]  
Substituting values:  
\[
dU = 316.5 \, \text{kJ/kg}
\]  

The work done per unit mass is given by:  
\[
\frac{W}{m} = \int p \, dV
\]  
The specific work is calculated using the change in volume:  
\[
W = m p (V_2 - V_1)
\]  
Substituting values yields:  
\[
W = 0.972 \, \text{kJ}
\]  

The final volume is determined using the ideal gas law:  
\[
V_2 = \frac{m R T_2}{p_{g,2}}
\]  
Substituting values yields:  
\[
V_2 = 1.11 \, \text{L}
\]  

The total heat transferred is calculated as:  
\[
Q = dU + W
\]  
Substituting values yields:  
\[
Q = 2052 \, \text{J}
\]  

The total heat transferred is approximately:  
\[
Q = 108 \, \text{kJ}
\]  