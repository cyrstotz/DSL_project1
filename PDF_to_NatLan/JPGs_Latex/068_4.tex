TASK 4b  
The process involves sublimation. The pressure \( p \) is reduced to \( 1 \, \text{mbar} \) or \( 1.2 \times 10^{-3} \, \text{bar} \), which corresponds to \( 2.0 \times 10^{-3} \, \text{kg/m}^3 \). Sublimation occurs at this pressure.  

From Table A-6, interpolation is performed to determine the temperature \( T \) at \( 0.21 \, \text{kPa} \). The formula used is:  
\[
T(0.21 \, \text{kPa}) = T(0.20388 \, \text{kPa}) + \frac{T(0.1635 \, \text{kPa}) - T(0.0883 \, \text{kPa})}{0.1635 - 0.0883} \times (0.21 - 0.0883)
\]  
This yields \( T \approx -22^\circ\text{C} \approx 20.385^\circ\text{C} \).  

The sub-point temperature \( T_2 \) is calculated as \( T_2 = T_{\text{sub-point}} + 20 \, \text{K} \), which is \( 20.385^\circ\text{C} + 20^\circ\text{C} \).  

TASK 4c  
The process from state 1 to state 2 involves compression. The temperature difference \( \Delta T \) is \( 6 \, \text{K} \), resulting in \( T_i = 26^\circ\text{C} \).  

For the stationary equilibrium process, the equation is:  
\[
0 = \dot{m}_{\text{R134a}} \left( h_1 - h_2 + \frac{v_2^2}{2} - \frac{v_1^2}{2} \right) + \dot{Q}_K
\]  
This simplifies to:  
\[
\dot{m}_{\text{R134a}} = \frac{\dot{Q}_K}{h_e - h_a}
\]  
where \( h_a \) and \( h_e \) are enthalpies.  

TASK 4d  
The coefficient of performance \( \epsilon_K \) is calculated as:  
\[
\epsilon_K = \frac{\dot{Q}_K}{\dot{W}_K}
\]  
This involves the heat transfer \( \dot{Q}_K \) and work \( \dot{W}_K \).  

The work \( W_K \) is expressed as:  
\[
W_K = \dot{m} \cdot \left( h_3 - h_4 \right) - \dot{Q}_{AB}
\]  
Thus, \( \epsilon_K \) can be rewritten as:  
\[
\epsilon_K = \frac{\dot{Q}_K}{\dot{m} \cdot \left( h_3 - h_4 \right) - \dot{Q}_{AB}}
\]  

TASK 4e  
In Step ii, if the cooling cycle from Step i continued with constant \( \dot{Q}_K \), the temperature \( T_i \) would evolve based on sublimation conditions.  

No further content or diagrams are visible.