TASK 4b  
The work done by the compressor is \( W_K = 288 \, \text{W} \).  

The energy balance for the compressor is given as:  
\[
Q = \dot{m} \left( h_2 - h_1 \right) + W_K
\]  
The process is adiabatic and reversible, so \( Q = 0 \) and \( s_2 = s_1 \).  

---

TASK 4c  
The energy balance for the throttling process is:  
\[
Q = 0, \, W = 0 \implies h_4 = h_1
\]  

The temperature of the refrigerant is \( T_{\text{verd}} = T_i - 6 \, \text{K} = 40^\circ\text{C} \).  

The pressure relationship is \( p_4 = p_1 \).  

---

TASK 4a  
A graph is drawn showing the freeze-drying process in a \( p \)-\( T \) diagram.  
- The diagram includes phase regions labeled as "gas," "liquid," and "solid."  
- The sublimation curve is marked, along with the triple point.  
- Axes are labeled: \( p \) (pressure in bar) on the vertical axis and \( T \) (temperature in Kelvin) on the horizontal axis.  

---

TASK 4e  
The temperature \( T_i = 10^\circ\text{C} \) and the pressure \( p_i = 2 \, \text{mbar} \).  

If the cooling cycle from Step i continues with constant \( \dot{Q}_K \), the temperature decreases continuously because \( \dot{Q}_K \) is applied and no additional work is performed.  

---

Description of the diagram:  
The diagram shows the phase regions for gas, liquid, and solid. The sublimation curve is drawn, and the triple point is marked. The axes are labeled with pressure (\( p \)) in bar and temperature (\( T \)) in Kelvin.