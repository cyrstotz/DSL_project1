TASK 2a  
The diagram represents a qualitative \( T \)-\( s \) diagram for the jet engine process. The temperature \( T \) is plotted on the vertical axis (in Kelvin), and the entropy \( s \) is plotted on the horizontal axis (in \( \text{kJ}/\text{kg·K} \)). The process includes labeled states:  
- State 0 represents the ambient conditions.  
- States 1, 2, and 3 correspond to the compression process, with \( p_1 = p_3 \).  
- State 4 represents the combustion process.  
- State 5 corresponds to the mixing chamber, and \( p_4 = p_5 \).  
- State 6 represents the nozzle exit, where \( p_6 = p_0 \).  

The diagram shows isobars and process paths connecting the states, indicating compression, combustion, mixing, and expansion processes.

---

TASK 2b  
The first law of thermodynamics applied to the entire system is written as:  
\[
0 = \dot{m}_0 \left( h_0 + \frac{w_{\text{Luft}}^2}{2} \right) - h_6 - \frac{w_6^2}{2}
\]  
Rearranging this equation gives:  
\[
2 \left( h_0 - h_6 + \frac{w_{\text{Luft}}^2}{2} \right) = w_6^2
\]  

The enthalpy difference \( h_0 - h_6 \) is expressed as:  
\[
h_0 - h_6 = c_{p,\text{Luft}} \left( T_0 - T_6 \right)
\]  

The first law of thermodynamics applied to the mixing chamber is noted, but the equation is partially crossed out. The equation suggests:  
\[
T_5 = T_6
\]  

No further explanation or derivation is provided for the mixing chamber.