TASK 1a  
The task involves determining \( \dot{Q}_{\text{out}} \), the heat flow removed by the coolant, using a steady-state flow process.  

The energy balance equation is written as:  
\[
\dot{m}_{\text{wasser}} (h_{1,w} - h_{2,w}) + \dot{Q}_R - \dot{Q}_{\text{out}} = 0
\]  

Here, \( h_{1,w} \) and \( h_{2,w} \) represent the specific enthalpies of water at the inlet and outlet temperatures, respectively.  

From the water tables:  
\[
h_{1,w} = h_f(70^\circ\text{C}) = 202.98 \, \frac{\text{kJ}}{\text{kg}}
\]  
\[
h_{2,w} = h_f(100^\circ\text{C}) = 419.09 \, \frac{\text{kJ}}{\text{kg}}
\]  

Substituting these values into the energy balance equation:  
\[
\dot{m}_w (h_{1,w} - h_{2,w}) + \dot{Q}_R = \dot{Q}_{\text{out}}
\]  
\[
\dot{Q}_{\text{out}} = 62.782 \, \text{kW}
\]  

A graph is included, showing a phase diagram with specific volume \( v \) on the x-axis and pressure \( p \) on the y-axis. The graph highlights the region of saturated liquid and the transition to the wet steam (mass-vapor) region.  

