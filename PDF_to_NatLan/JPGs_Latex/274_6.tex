TASK 4a  
A pressure-temperature (\( p \)-\( T \)) diagram is drawn to represent the freeze-drying process (Steps i and ii). The diagram includes labeled points corresponding to states 1, 2, 3, and 4.  
- The \( y \)-axis is labeled as \( p \) (pressure in bar).  
- The \( x \)-axis is labeled as \( T \) (temperature in Kelvin).  
- The curve shows phase regions, with transitions between states marked.  
- State 3 is at the highest pressure, and state 4 is at a lower pressure.  
- State 2 is indicated near the sublimation temperature.  

TASK 4b  
The equation for the refrigerant mass flow rate \( \dot{m}_{\text{R134a}} \) is given as:  
\[
0 = \dot{m}_{\text{R134a}} \left[ h_2 - h_3 \right] + \dot{W}_K
\]  
Rearranging for \( \dot{W}_K \):  
\[
\dot{W}_K = \dot{m}_{\text{R134a}} \left[ h_2 - h_3 \right]
\]  
The mass flow rate is calculated as:  
\[
\dot{m}_{\text{R134a}} = 0.721 \, \text{kg/s}
\]  

The enthalpy difference \( h_3 - h_2 \) is calculated using tabulated values:  
\[
h_3 = 61.9351 \, \text{kJ/kg}, \quad h_2 = 234.08 \, \text{kJ/kg}
\]  
Intermediate calculations are shown using values from Table A10.  

TASK 4c  
The vapor quality \( x_1 \) at state 1 is determined using the formula:  
\[
x_1 = \frac{s_4 - s_f}{s_g - s_f}
\]  
Where:  
- \( s_4 \) is the entropy at state 4, calculated as the average of \( s_f \) values at 2.8 bar and 3.2 bar:  
\[
s_4 = \frac{s_f(2.8 \, \text{bar}) + s_f(3.2 \, \text{bar})}{2} = 0.2
\]  
- \( s_f \) and \( s_g \) are the saturated liquid and vapor entropies at state 1:  
\[
s_f = 0.08959, \quad s_g = 1.9350
\]  

The pressure \( p_1 = p_2 \), and \( p_2 = p_{\text{sat}}(T = 272 \, \text{K}) = 1.2432 \, \text{bar} \).  
Using the entropy values:  
\[
x_1 = \frac{0.2 - 0.08959}{1.9350 - 0.08959} = 0.131
\]  

No additional diagrams or graphs are present.