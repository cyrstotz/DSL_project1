TASK 4a  
A pressure-temperature (\( p \)-\( T \)) diagram is drawn to represent the freeze-drying process. The diagram includes the following elements:  
- The \( p \)-axis is labeled in mbar, with points marked at 0.5 mbar and 1 mbar.  
- The \( T \)-axis is labeled in \( ^\circ \text{C} \).  
- A curve labeled "gas" separates the liquid and gas regions.  
- The triple point is marked on the curve.  
- States 1, 2, and 3 are labeled, with arrows indicating transitions between them.  
- State 1 is above the curve, state 2 is on the curve, and state 3 is below the curve.  

TASK 4b  
A table is provided to describe the refrigerant properties at different states:  

| \( P \)       | \( T \)       | \( x \) | \( s \) |  
|---------------|---------------|---------|---------|  
| 1             | \(-16^\circ \text{C}\) |         |         |  
| 2             | \(-16^\circ \text{C}\) | 1       |         |  
| 3             | \( 8 \, \text{bar} \)  |         |         |  
| 4             | \( 8 \, \text{bar} \)  | 0       |         |  

Additional notes:  
- \( T_i = 6 \, \text{K} \)  
- \( p_i = 1 \, \text{mbar} \)  
- \( T_i = -10^\circ \text{C} \)  

TASK 4c  
An energy balance equation for the compressor is written:  
\[
0 = \dot{m}_{\text{R134a}} \left( h_2 - h_3 \right) - \dot{W}_K
\]  
Where:  
\[
\dot{W}_K = \dot{m}_{\text{R134a}} \left( h_2 - h_3 \right)
\]  

From Table A-10:  
\[
h_2 = h_g(-16^\circ \text{C}) = 237.74  
\]  
\[
s_2 = s_g(-16^\circ \text{C}) = 0.9298 \, \text{kJ/kg·K}  
\]  
\[
s_3 = s_2 \, \text{(because the process is reversible)}  
\]