TASK 4a  
The diagram is a pressure-enthalpy (\(p\)-\(h\)) graph illustrating the freeze-drying process with R134a. It shows four states connected by processes:  
- From state 1 to state 2, the process is isobaric.  
- From state 3 to state 4, the process is also isobaric.  
- The curve labeled "ND" represents the non-dimensional region, likely indicating a phase change or transition.  
- Arrows indicate the direction of the processes.  

TASK 4b  
The energy balance for the compressor is given as:  
\[
0 = \dot{m} \left[ h_2 - h_3 \right] + \dot{Q} - \dot{W}_{\text{K}}
\]  
where \(h_2\) and \(h_3\) are enthalpies at states 2 and 3, respectively, and \(\dot{W}_{\text{K}}\) is the work done by the compressor.  

The entropy at state 2 is equal to the entropy at state 3:  
\[
s_2 = s_3 = 0.9066 \, \text{kJ/kg·K}
\]  

The enthalpy at state 3 is calculated using the isobaric condition:  
\[
h_3 = h_3(8 \, \text{bar}) = 264.15 \, \text{kJ/kg}
\]  

For the throttle (expansion valve), the energy balance is:  
\[
0 = \dot{m} \left[ h_4 - h_1 \right]
\]  
where \(h_4 = h_1\), indicating that enthalpy remains constant during throttling.  

The pressure at state 4 is equal to the pressure at state 3:  
\[
p_4 = p_3
\]