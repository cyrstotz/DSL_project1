TASK 3a  
The problem involves an ideal gas system and the equilibrium of forces in state 1.  

### Force Equilibrium  
A diagram is provided showing a piston resting on a membrane, separating a gas chamber below from an ice-water mixture above. The forces acting on the piston are:  
- \( F_{\text{EW}} \): Force due to the ice-water mixture.  
- \( F_{\text{amb}} \): Force due to ambient pressure.  
- \( F_K \): Force due to the piston mass.  
- \( F_g \): Force exerted by the gas.  

The force equilibrium is expressed as:  
\[
F_g = F_{\text{EW}} + F_{\text{amb}} + F_K
\]

The cross-sectional area of the cylinder is calculated as:  
\[
A_z = \text{Diameter of cylinder} = \pi r^2
\]
Given \( r = 5 \, \text{cm} = 0.05 \, \text{m} \),  
\[
A_z = \pi \cdot (0.05 \, \text{m})^2 = 7.85 \cdot 10^{-3} \, \text{m}^2
\]

### Calculation of Forces  
1. **Force due to ice-water mixture**:  
\[
F_{\text{EW}} = m_{\text{EW}} \cdot g
\]
Given \( m_{\text{EW}} = 0.1 \, \text{kg} \) and \( g = 9.81 \, \text{m/s}^2 \),  
\[
F_{\text{EW}} = 0.1 \, \text{kg} \cdot 9.81 \, \text{m/s}^2 = 0.981 \, \text{N}
\]

2. **Force due to piston mass**:  
\[
F_K = m_K \cdot g
\]
Given \( m_K = 32 \, \text{kg} \),  
\[
F_K = 32 \, \text{kg} \cdot 9.81 \, \text{m/s}^2 = 313.92 \, \text{N}
\]

3. **Force due to ambient pressure**:  
\[
F_{\text{amb}} = p_{\text{amb}} \cdot A_z
\]
Given \( p_{\text{amb}} = 100 \, \text{kPa} = 100,000 \, \text{N/m}^2 \),  
\[
F_{\text{amb}} = 100,000 \, \text{N/m}^2 \cdot 7.85 \cdot 10^{-3} \, \text{m}^2 = 785 \, \text{N}
\]

4. **Total force exerted by the gas**:  
\[
F_g = F_{\text{EW}} + F_K + F_{\text{amb}}
\]
Substituting values:  
\[
F_g = 0.981 \, \text{N} + 313.92 \, \text{N} + 785 \, \text{N} = 1099.901 \, \text{N}
\]

### Gas Pressure Calculation  
The gas pressure is calculated as:  
\[
F_g = p_{g,1} \cdot A_z \implies p_{g,1} = \frac{F_g}{A_z}
\]
Substituting values:  
\[
p_{g,1} = \frac{1099.901 \, \text{N}}{7.85 \cdot 10^{-3} \, \text{m}^2} = 140,144.78 \, \text{Pa}
\]
Converting to kilopascals and bar:  
\[
p_{g,1} = 140.14 \, \text{kPa} = 1.4 \, \text{bar}
\]

### Ideal Gas Law  
The ideal gas law is applied:  
\[
pV = mRT \implies m = \frac{pV}{RT}
\]

1. **Specific gas constant**:  
\[
R_g = \frac{R}{M}
\]
Given \( R = 8.314 \, \text{kJ/kmol·K} \) and \( M = 50 \, \text{kg/kmol} \),  
\[
R_g = \frac{8.314 \, \text{kJ/kmol·K}}{50 \, \text{kg/kmol}} = 0.166 \, \text{kJ/kg·K}
\]

2. **Gas volume**:  
\[
V_{g,1} = 3.14 \, \text{L} = 3.14 \cdot 10^{-3} \, \text{m}^3
\]

3. **Gas temperature**:  
\[
T_{g,1} = 500^\circ\text{C} = 773.15 \, \text{K}
\]

Substituting into the ideal gas law:  
\[
m_g = \frac{p_{g,1} \cdot V_{g,1}}{R_g \cdot T_{g,1}}
\]