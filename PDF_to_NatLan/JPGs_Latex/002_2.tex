TASK 2a  
The problem setup includes the following parameters:  
- Airspeed \( w_{\text{air}} = 200 \, \text{m/s} \)  
- Ambient pressure \( p_0 = 0.191 \, \text{bar} \)  
- Ambient temperature \( T_0 = -30^\circ\text{C} \)  
- Heat added per unit mass \( q_B = 1195 \, \text{kJ/kg} \)  
- Mean temperature during combustion \( \bar{T}_B = 1289 \, \text{K} \)  
- Air is modeled as a perfect gas.  

A qualitative \( T \)-\( s \) diagram is drawn to represent the jet engine process.  
- The diagram shows labeled states (0, 1, 2, 3, 4, 5, 6).  
- The process begins at state 0 and progresses through compression (state 1 to state 2), combustion (state 3), turbine expansion (state 4), mixing (state 5), and nozzle exit (state 6).  
- Isobars are indicated with red lines, and the entropy axis is labeled as \( s \) in \( \text{kJ/kg·K} \).  
- The temperature axis is labeled as \( T \) in \( \text{K} \).  
- The pressure at state 2 is equal to the pressure at state 3 (\( p_2 = p_3 \)), and the pressure at state 6 equals the ambient pressure (\( p_6 = p_0 \)).  

TASK 2b  
An energy balance is performed around the jet engine.  
The following equations and relationships are derived:  
1. The temperature at state 6 (\( T_6 \)) is calculated using the isentropic relation:  
\[
T_6 = T_5 \left( \frac{p_6}{p_5} \right)^{\frac{\kappa - 1}{\kappa}}
\]  
where \( \kappa = 1.4 \).  
The result is \( T_6 = 325.07 \, \text{K} \).  

2. The energy balance equation is written as:  
\[
0 = \dot{m} \left( h_0 - h_1 \right) + q_B - \dot{Q}
\]  
\[
0 = \dot{m} \left( h_1 - h_B \right) + \left( w_e^2 - w_a^2 \right) + Q
\]  
\[
0 = 2 \left( h_1 - h_B \right) + \left( w_e^2 - w_a^2 \right)
\]  
\[
2 \left( h_1 - h_B \right) = Q + \left( w_e^2 - w_a^2 \right)
\]  

The equations are used to calculate the outlet velocity \( w_6 \) and other thermodynamic properties.  

A block diagram is included to visually represent the energy flow and key states in the jet engine process. The diagram shows the inlet and outlet streams, heat addition \( q_B \), and the relationship between pressures and temperatures at various states.  

Additional notes:  
- \( n = 1.4 \) is used for the isentropic exponent.  
- Intermediate calculations for \( T_6 \) and other variables are shown.  

No further content is visible.