TASK 1a  
The first law of thermodynamics is applied:  
\[
0 = \dot{m} (h_e - h_a) + \dot{Q} - \dot{W}
\]  
The heat flow removed by the coolant is expressed as:  
\[
\dot{Q}_{\text{ab}} = \dot{m} (h_a - h_e)
\]  

The enthalpy values are determined from water tables:  
\[
h_a = h_f \, \text{at} \, T = 100^\circ\text{C} = 2676.16 \, \frac{\text{kJ}}{\text{kg}}
\]  
\[
h_e = h_f \, \text{at} \, T = 70^\circ\text{C} = 2626.8 \, \frac{\text{kJ}}{\text{kg}}
\]  
(using Table A-2).  

Substituting into the equation:  
\[
\dot{Q}_{\text{ab}} = 0.3 \, \frac{\text{kg}}{\text{s}} \left( 2676.16 \, \frac{\text{kJ}}{\text{kg}} - 2626.8 \, \frac{\text{kJ}}{\text{kg}} \right) = 14.79 \, \text{kW}
\]  

TASK 1b  
The thermodynamic mean temperature of the coolant is calculated using the integral formula:  
\[
T_{\text{KF}} = \frac{\int_{e}^{a} T \, dS}{S_a - S_e}
\]  

A small sketch is visible in the top right corner, showing a simplified diagram of heat flow \( \dot{Q} \) through a wall labeled "Fall."