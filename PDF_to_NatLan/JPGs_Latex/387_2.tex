TASK 3c  
The heat transfer \( Q_{12} \) is calculated using the energy balance equation:  
\[
\Delta U = Q - W
\]  
The work \( W_{12} \) is expressed as:  
\[
W_{12} = \frac{R(T_2 - T_1)}{1 - n}
\]  
where \( n = \frac{c_p}{c_v} \), and \( c_p \) is calculated as:  
\[
c_p = R + c_v = 799 \, \frac{\text{J}}{\text{kg·K}}
\]  
Thus, \( n = \frac{c_p}{c_v} = 1.2666 \).  

The work \( W_{12} \) is determined to be:  
\[
W_{12} = -1082.4 \, \text{J}
\]  

The internal energy change \( \Delta U \) is calculated as:  
\[
\Delta U = c_v (T_2 - T_1) m = 866.33 \, \text{J}
\]  

Finally, the heat transfer \( Q_{12} \) is:  
\[
Q_{12} = 1,945.4 \, \text{J}
\]  

---

TASK 3d  
The final ice fraction \( x_{\text{ice},2} \) is determined using the following equation:  
\[
- m_{\text{EW}} \cdot 0.6 \cdot U_{\text{fresh}}(T_2) + m_{\text{EW}} \cdot x_{\text{ice},2} \cdot U_{\text{fresh}}(T_2) + m_{\text{EW}} \cdot (1 - x_{\text{ice},2}) \cdot U_{\text{fossils}}(T_2) - Q_{12} = 0
\]  

This equation balances the energy contributions from the ice fraction, the water fraction, and the heat transfer \( Q_{12} \).