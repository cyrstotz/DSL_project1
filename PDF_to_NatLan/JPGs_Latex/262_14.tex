TASK 4a  
The diagram represents the freeze-drying process in a pressure-temperature (\(p\)-\(T\)) diagram. The curve shows the phase regions for the refrigerant R134a, including liquid, vapor, and supercritical states. Key points labeled on the diagram are:  
- Point 1: Isobaric evaporation at low pressure.  
- Point 2: Isobaric compression leading to the supercritical region.  
- Point 3: Isobaric condensation at high pressure.  
- Point 4: Adiabatic expansion returning to the liquid-vapor region.  

The critical temperature (\(T_{\text{crit}}\)) is marked at the peak of the curve, and the transitions between phases are labeled as "isobar" (constant pressure). The diagram also includes arrows indicating the direction of the process flow.

---

TASK 4b  
The energy balance equation for the refrigerant cycle is written as:  
\[
0 = \dot{m}_{\text{R134a}} \left( h_2 - h_3 \right) + \dot{Q}_K - W_K
\]  
This equation accounts for the mass flow rate of R134a (\(\dot{m}_{\text{R134a}}\)), the enthalpy difference between states 2 and 3 (\(h_2 - h_3\)), the heat removed (\(\dot{Q}_K\)), and the work done by the compressor (\(W_K\)). The system is described as stationary, meaning there is no accumulation of energy over time.