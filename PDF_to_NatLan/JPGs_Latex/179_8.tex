TASK 3a  
The pressure \( p_{g,1} \) and mass \( m_g \) are calculated as follows:  

The force per unit area is given by:  
\[
\frac{F}{A} = p
\]  

The pressure is calculated using:  
\[
p_{\text{g,1}} = 200000 \, \text{Pa} - p_{\text{amb}} = 190095 \, \text{Pa} = 1.9 \, \text{bar}
\]  

The cross-sectional area of the cylinder is determined using:  
\[
A = \left(\frac{D}{2}\right)^2 \pi = 0.007853 \, \text{m}^2
\]  

The ideal gas law is applied to find the mass:  
\[
\frac{pV}{RT} = m = 0.0034 \, \text{kg} = 3.4 \, \text{g}
\]  

Given values:  
\[
p = 7.2 \, \text{bar}, \, V = 3.14 \, \text{L}, \, T = 500 \, \text{K}
\]  

The gas constant \( R \) is calculated as:  
\[
R = \frac{R_u}{M}
\]  

Where \( R_u \) is the universal gas constant, and \( M \) is the molar mass of the gas.  

TASK 3b  
The ice fraction \( x_{\text{ice},2} > 0 \).  

The temperature \( T_{g,2} \) and pressure \( p_{g,2} \) are determined.