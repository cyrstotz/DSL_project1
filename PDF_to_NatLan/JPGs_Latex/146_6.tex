TASK 4a  
A graph is drawn with pressure \( p \) (in bar) on the vertical axis and temperature \( T \) (in Kelvin) on the horizontal axis. The graph shows a curve labeled "Triple point" at its peak, with phase regions labeled as "Liquid," "Vapor," and "Solid." The curve represents the phase boundaries between these regions.

---

TASK 4b  
The equation for the work rate \( \dot{W}_K \) is given as:  
\[
\dot{W}_K = 28 \, \text{W}
\]  
The pressure at state 3 is \( p_3 = 8 \, \text{bar} \), and the enthalpy at state 2 is \( h_2 \). It is noted that \( s_2 = s_3 \) (isentropic process).  

The energy balance equation is written as:  
\[
0 = \dot{m} [h_2 - h_3] - \dot{W}_K \quad \Rightarrow \quad \dot{W}_K = \dot{m} [h_2 - h_3]
\]  
Rearranging for mass flow rate \( \dot{m} \):  
\[
\dot{m} = \frac{\dot{W}_K}{h_2 - h_3}
\]  

The enthalpy \( h_2 \) is left undefined.

---

TASK 4c  
The vapor quality \( x_1 \) is calculated using the formula:  
\[
x_1 = \frac{h_1 - h_f}{h_g - h_f}
\]  
where \( h_1 \) is the enthalpy at state 1, \( h_f \) is the enthalpy of saturated liquid, and \( h_g \) is the enthalpy of saturated vapor.