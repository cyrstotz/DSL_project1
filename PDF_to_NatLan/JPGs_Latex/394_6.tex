TASK 4a  
The diagram is a pressure-temperature (\(p\)-\(T\)) graph illustrating the freeze-drying process. It shows a rectangular cycle with four states labeled as 1, 2, 3, and 4.  
- State 1 is at the bottom left, representing the low-pressure, low-temperature region.  
- State 2 is at the bottom right, indicating the same pressure as state 1 but at a higher temperature.  
- State 3 is at the top right, showing high pressure and high temperature.  
- State 4 is at the top left, indicating the same pressure as state 3 but at a lower temperature.  

TASK 4b  
The mass flow rate of the refrigerant \( \dot{m}_{\text{R134a}} \) is calculated using the energy balance equation:  
\[
\dot{Q}_K = \dot{m} (h_2 - h_3) + Q_{23}^{\text{adiabat}} - \dot{W}_K
\]  
where:  
- \( h_2 \) is the enthalpy at state 2, calculated as \( h_2 = h_f + x_2 (h_g - h_f) \), with \( s_2 = s_3 \) and \( p_2 = p_4 \).  
- \( h_3 \) is the enthalpy at state 3, given as \( h_3 = h(8 \, \text{bar}) \).  

TASK 4c  
The vapor quality \( x_1 \) at state 1 is determined using the following relationships:  
\[
\dot{Q}_K = \dot{m} (h_4 - h_1) + Q_{34}^{\text{adiabat}} - \dot{W}_{34}
\]  
where:  
- \( h_4 = h_s(8 \, \text{bar}) = 264.15 \, \text{kJ/kg} \).  
- \( p_4 = p_3 \).  
- \( h_1 = h_f + x_1 (h_g - h_f) \), with \( h_g - h_f = x_1 (p_1) \).  

Additional notes:  
- \( h_4 \) is the enthalpy at state 4, and \( h_1 \) is the enthalpy at state 1.  
- The vapor quality \( x_1 \) is calculated using the enthalpy difference \( h_g - h_f \).  

No additional diagrams or figures are described beyond the \(p\)-\(T\) graph in TASK 4a.