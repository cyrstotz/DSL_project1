TASK 2b  
The energy balance for the steady-state flow process (SFP) is written as:  
\[
0 = \dot{m}_g \left[ h_5 - h_6 + \frac{w_5^2 - w_6^2}{2} \right] + 0
\]  
This equation accounts for the enthalpy difference and kinetic energy terms. The term "Schaufelrad IG" (likely referring to a turbine or impeller) is noted.  

Rewriting the energy balance:  
\[
0 = \dot{m}_g \left[ \int c_p (T_5 - T_6) + \frac{w_5^2 - w_6^2}{2} \right]
\]  

From this, the velocity term is isolated:  
\[
\frac{w_6^2 - w_5^2}{2} = \dot{m}_g \int c_p (T_5 - T_6)
\]  

The outlet velocity \( w_6 \) is derived as:  
\[
w_6^2 = 2 \left( c_p (T_5 - T_6) \right) - w_5^2
\]  

Finally, solving for \( w_6 \):  
\[
w_6 = \sqrt{2 \left( c_p (T_5 - T_6) \right) - w_5^2}
\]  

---

TASK 2c  
The exergy balance for the steady-state flow process (SFP) is expressed as:  
\[
0 = \dot{m}_g \Delta ex_{\text{thr}}
\]  

The specific exergy change is given by:  
\[
\Delta ex_{\text{thr}} = \left[ h_6 - h_0 - T_0 (s_6 - s_0) + \frac{w_6^2 - w_0^2}{2} \right]
\]  
The term "Schaufelrad IG" is again noted, and the following relationships are used:  
\[
h_6 - h_0 = c_p^w (T_6 - T_0)
\]  
\[
s_6 - s_0 = c_p^w \ln \left( \frac{T_6}{T_0} \right) - R \ln \left( \frac{p_6}{p_0} \right)
\]  

Here, \( c_p = R + c_v \) is noted as the specific heat relationship.