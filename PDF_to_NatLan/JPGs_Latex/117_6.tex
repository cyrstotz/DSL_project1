TASK 3c  
The problem involves determining \( Q_{12} \), the heat transferred from the gas to the ice-water mixture between states 1 and 2.

The temperature of the gas at state 2 is given as:  
\[
T_{g,2} = 0.003^\circ \text{C}
\]

### Energy Balance  
The energy balance equation is written as:  
\[
\frac{dE}{dt} = \dot{Q}_{12} - \dot{W}_{12} \quad \text{(steady-state assumption)}
\]

For the system, the internal energy change is expressed as:  
\[
m \Delta U = Q_{12} - W_{12}
\]

The work term \( W_{12} \) is defined as:  
\[
W_{12} = \int_{1}^{2} p \, dV
\]

The change in internal energy \( \Delta U \) is calculated as:  
\[
\Delta U = U_2 - U_1 = c_V \cdot (T_{g,2} - T_1)
\]

Substituting values:  
\[
\Delta U = 0.633 \, \frac{\text{kJ}}{\text{kg·K}} \cdot (0.003^\circ \text{C} - 500^\circ \text{C})
\]

This results in:  
\[
\Delta U = -376.498 \, \frac{\text{kJ}}{\text{kg}}
\]

### Final Equation for \( Q_{12} \)  
The mass of the gas is given as:  
\[
m_g = 3.6 \, \text{g} \quad \text{(assumed)}
\]

The heat transfer \( Q_{12} \) is expressed as:  
\[
m_g \cdot \Delta U = Q_{12} - W_{12}
\]

Rearranging:  
\[
Q_{12} = m_g \cdot \Delta U + W_{12}
\]  

Some terms are crossed out, but the final expression for \( Q_{12} \) remains valid.