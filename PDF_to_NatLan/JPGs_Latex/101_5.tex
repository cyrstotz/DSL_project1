TASK 2a  
A table is presented with columns labeled \( p \), \( T \), and \( S \). The rows correspond to states 0 through 6, with the following observations:  
- State 1: \( p < p_c \).  
- State 2: \( S_2 = S_1 \).  
- State 3: \( T > T_2 \).  
- State 5: \( p = 0.5 \, \text{bar} \), \( T = 431.9 \, \text{K} \).  
- State 6: \( S_5 = S_6 \).  

Below the table, a graph is drawn with temperature \( T \, [\text{K}] \) on the vertical axis and entropy \( S \, [\text{kJ/kg·K}] \) on the horizontal axis. The graph depicts the thermodynamic process of a jet engine with labeled states (1 through 6) and transitions:  
- State 1 to 2: Isentropic compression.  
- State 2 to 3: Isobaric heating.  
- State 3 to 4: Isothermal process.  
- State 4 to 5: Isobaric cooling.  
- State 5 to 6: Isentropic expansion.  

The graph also includes annotations such as "steeper than isobar" and "adiabatic."  

---

TASK 2b  
The following calculations are shown to determine the temperature \( T_6 \):  
- Given values:  
  \( w_5 = 220 \, \text{m/s} \), \( p_5 = 0.5 \, \text{bar} \), \( T_5 = 431.9 \, \text{K} \).  
  The process from state 5 to 6 is labeled as "isentropic."  

- The temperature ratio is calculated using the isentropic relation:  
  \[
  \frac{T_c}{T_5} = \left( \frac{p_c}{p_5} \right)^{\frac{\kappa - 1}{\kappa}}
  \]  
  Substituting values:  
  \[
  \frac{T_c}{T_5} = \left( \frac{0.191 \, \text{bar}}{0.5 \, \text{bar}} \right)^{\frac{0.4}{1.4}}
  \]  
  \[
  T_c = 328.07 \, \text{K}
  \]  

- An energy balance equation is written:  
  \[
  0 = \dot{m} \left( h_5 - h_6 + \frac{w_5^2 - w_6^2}{2} \right) + \dot{Q} - \dot{W}
  \]  
  Additional terms are included to account for enthalpy differences and velocity changes.  

