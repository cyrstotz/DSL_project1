TASK 4a  
A graph is drawn with axes labeled \( p \) (pressure) on the vertical axis and \( T \) (temperature) on the horizontal axis. The graph depicts a phase diagram with regions for different phases. A line labeled "4" is shown, representing isobaric evaporation at a temperature 6 K below \( T_i \). The diagram includes arrows indicating transitions between states, but some details are unclear.

---

TASK 4b  
The energy balance equation is written as:  
\[
0 = \dot{m} \left[ h_e - h_a + ke + gz(z_e - z_a) \right] + \Delta \dot{Q} - \sum \dot{W}_{\text{in}}
\]  
Simplified to:  
\[
0 = \dot{m}_{\text{R134a}} \left[ h_1 - h_p \right] + (-\dot{Q}_{\text{abs}}) + Q_K + 28W
\]  

Further simplifications:  
\[
0 = \dot{m} \left[ h_1 - h_2 \right] + Q_K
\]  
\[
Q_K = \dot{m}_{\text{R134a}} \left[ h_2 - h_1 \right]
\]  

For the transition between states 2 and 3:  
\[
0 = \dot{m}_{\text{R134a}} \left[ h_2 - h_3 \right] + Q_K + 28W
\]  
\[
= \dot{m} \left[ \int_{T_1}^{T_2} c_p(T) dT + v \Delta p (p_2 - p_1) \right] + Q_K + 28W
\]  

Additional notes mention \( h_2 \) as "gesättigt dampf" (saturated vapor).  

---

TASK 4c  
The entropy change is described as:  
\[
\Delta Q \rightarrow \phi = \phi_f + x (\phi_g - \phi_f)
\]  
\[
S = S_f + x (S_g - S_f)
\]  
Where \( S_1 = S_f \).  

It is noted that in state 4, the refrigerant is fully condensed. Reference is made to Table A7 for further data.

---

TASK 4d  
The coefficient of performance \( \epsilon_K \) is defined as:  
\[
\epsilon_K = \frac{\left| Q_{\text{in}} \right|}{\left| W_e \right|} = \frac{\left| Q_{\text{abs}} \right|}{\left| Q_{\text{abs}} - \left| W_e \right| \right|}
\]  