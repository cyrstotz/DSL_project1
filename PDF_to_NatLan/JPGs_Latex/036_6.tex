TASK 4a  
The diagram is a pressure-temperature (\(p\)-\(T\)) plot illustrating the freeze-drying process. It consists of a rectangular cycle with four labeled states:  
- State \(1 \to 2\): Isobaric process.  
- State \(2 \to 3\): Isothermal process.  
- State \(3 \to 4\): Isobaric process.  
- State \(4 \to 1\): Isothermal process.  

The axes are labeled as follows:  
- The vertical axis represents pressure (\(p\)).  
- The horizontal axis represents temperature (\(T\)).  

TASK 4b  
The mass flow rate of the refrigerant (\( \dot{m}_{\text{R134a}} \)) is calculated using an energy balance for the evaporator.  

The energy balance equation is:  
\[
0 = \dot{m} \left[ h_e - h_a \right] + \sum \dot{Q}_n - \sum \dot{W}_{t,n}
\]  
Since the process is adiabatic (\( \dot{Q}_n = 0 \)), the equation simplifies to:  
\[
\dot{W} = 2.8 \, \text{W} = \dot{W}_t
\]  

From the given data:  
- \( h_e = h_2 \)  
- \( h_u = h_3 = 2.619.75 \, \text{kJ/kg} \) (from Table A-11).  
- \( p_2 = p_1 \) (adiabatic throttling implies constant pressure).  

The mass flow rate is calculated as:  
\[
\dot{m} = \frac{\dot{W}_{t,\text{in}}}{h_a - h_e} = \frac{2.8 \, \text{W}}{h_3 - h_2}
\]  

Additional notes:  
- Adiabatic throttling is isenthalpic.