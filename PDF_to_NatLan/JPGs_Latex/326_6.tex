TASK 4a  
The diagram represents the freeze-drying process in a pressure-temperature (\(p\)-\(T\)) diagram.  
- The phase regions for solid, liquid, and gas are labeled.  
- The triple point is marked as the intersection of the three phase boundaries.  
- A horizontal line is drawn below the triple point, indicating the sublimation process at a pressure lower than the triple point.  
- The temperature is labeled as \(T [^\circ C]\), and the pressure is labeled as \(p [\text{bar}]\).  
- The process includes two steps:  
  1. The system starts below the triple point.  
  2. The temperature is held 10 K above the sublimation point during the sublimation process.  

TASK 4b  
The refrigerant mass flow rate (\(\dot{m}_{\text{R134a}}\)) is calculated using the following equations:  
\[
\dot{m}_R = \frac{\dot{W}_K}{h_2 - h_3}
\]  
Where:  
- \(T_i = -10^\circ\text{C}\)  
- \(T_2 = -16^\circ\text{C}\)  
- \(h_2 = h_g(-16^\circ\text{C}) = 237.79 \, \text{kJ/kg}\)  

The entropy values are:  
\[
s_2 = s_3 = 0.5288 \, \text{kJ/kg·K}
\]  

The vapor quality (\(x_3\)) at state 3 is determined using:  
\[
x_3 = \frac{s_2 - s_f(8 \, \text{bar})}{s_g(8 \, \text{bar}) - s_f(8 \, \text{bar})}
\]  
Substituting values:  
\[
x_3 = \frac{0.5288 - 0.3859}{6.966 - 0.3859} = 1.064 \, (\text{not physically valid})
\]  

The refrigerant mass flow rate is expressed as:  
\[
\dot{m}_R = \frac{\dot{W}_K}{h_2 - h_3}
\]  

No further numerical results are provided for \(\dot{m}_R\).