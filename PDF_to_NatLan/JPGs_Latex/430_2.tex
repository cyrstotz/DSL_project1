TASK 2a  
A qualitative \( T \)-\( s \) diagram is drawn to represent the jet engine process. The diagram includes labeled isobars and process states (0, 1, 2, 3, 4, 5, and 6).  
- State 0 is the ambient condition.  
- States 1 to 2 represent the compression process.  
- States 3 to 4 correspond to isobaric combustion.  
- States 4 to 5 represent the turbine process.  
- State 6 is the nozzle exit.  
The diagram shows arrows indicating the direction of the process flow, and the isentropic lines are clearly marked.  

---

TASK 2b  
The control volume for the jet engine is sketched, showing inlet at state 0 and outlet at state 6. The diagram includes arrows for mass flow and energy transfer.  

The energy balance equation is written as:  
\[
\frac{dE}{dt} = \sum \dot{m}_i \left( h_i + \frac{w_i^2}{2} + k_i + p_i \right) + \sum \dot{Q}_i - \sum \dot{m}_o \left( h_o + \frac{w_o^2}{2} + k_o + p_o \right) - \sum \dot{W}_i
\]

The outlet velocity \( w_6 \) is calculated using the following steps:  
1. The enthalpy difference is expressed as:  
\[
h_6 = h_0 + \frac{w_0^2}{2} - \frac{w_6^2}{2}
\]  
2. The initial enthalpy \( h_0 \) is calculated:  
\[
h_0 = \frac{w_0^2}{2} = \frac{(200 \, \text{m/s})^2}{2} = 20,000 \, \text{J/kg}
\]  
3. The enthalpy difference \( h_0 - h_6 \) is integrated:  
\[
h_0 - h_6 = \int_{T_0}^{T_6} c_p \, dT = 1,006 \, \frac{\text{J}}{\text{kg·K}} \cdot (431.9 \, \text{K} - 243.15 \, \text{K}) + 328,107 \, \frac{\text{J}}{\text{kg}}
\]  
This results in:  
\[
h_0 - h_6 = 855,928 \, \frac{\text{J}}{\text{kg}}
\]

The temperature ratio \( \frac{T_6}{T_0} \) is calculated using:  
\[
\frac{T_6}{T_0} = \left( \frac{p_6}{p_0} \right)^{\frac{\kappa - 1}{\kappa}}
\]  
Substituting values:  
\[
\frac{T_6}{T_0} = \left( \frac{0.196 \, \text{bar}}{0.5 \, \text{bar}} \right)^{\frac{1.4 - 1}{1.4}} = 0.4379
\]  
Thus, \( T_6 = 328.07 \, \text{K} \).  

The outlet temperature \( T_6 \) is boxed as:  
\[
T_6 = 328.07 \, \text{K}
\]  

No additional diagrams or figures are described.