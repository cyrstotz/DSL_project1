TASK 2a  
The diagram is a qualitative representation of the jet engine process on a temperature-entropy (\( T \)-\( s \)) diagram. It includes labeled points corresponding to states \( 0 \), \( 1 \), \( 2 \), \( 3 \), \( 4 \), \( 5 \), and \( 6 \). The process involves isobars and curves indicating compression, combustion, and expansion.  
- The curve starts at state \( 0 \) and rises sharply to state \( 2 \), representing compression.  
- From state \( 2 \) to \( 3 \), there is a further increase in temperature and entropy, indicating combustion.  
- The curve then drops from state \( 3 \) to \( 4 \), representing turbine expansion.  
- States \( 5 \) and \( 6 \) are connected by a steep curve, indicating the nozzle process.  
The isobars are marked, and the general flow of the process is shown with arrows.  

TASK 2b  
The temperature \( T_6 \) is calculated using the following formula:  
\[
T_6 = \left( \frac{p_6}{p_5} \right)^{\frac{\kappa - 1}{\kappa}} \cdot T_5
\]  
Here, \( \kappa = 1.4 \), \( p_6 \) is the pressure at state \( 6 \), \( p_5 \) is the pressure at state \( 5 \), and \( T_5 \) is the temperature at state \( 5 \).  

The result is \( T_6 = 520.45 \, \text{K} \).  

TASK 2c  
The outlet velocity \( w_6 \) is determined using the energy balance equation:  
\[
0 = \dot{m} \left( h_5 - h_6 + \frac{w_5^2 - w_6^2}{2} \right)
\]  
Rearranging for \( w_6 \):  
\[
w_6 = \sqrt{-2 \left( \frac{h_5 - h_6}{\dot{m}} \right) + w_5^2}
\]  
This equation relates the enthalpy difference \( h_5 - h_6 \), mass flow rate \( \dot{m} \), and the velocity at state \( 5 \), \( w_5 \), to the outlet velocity \( w_6 \).