TASK 4a  
The graph is a pressure-temperature (\( p \)-\( T \)) diagram. It shows the freeze-drying process with two steps labeled as "i" and "ii."  
- The x-axis represents temperature (\( T \)) in degrees Celsius (\( ^\circ \text{C} \)), ranging from approximately \(-70^\circ \text{C}\) to \( 20^\circ \text{C}\).  
- The y-axis represents pressure (\( p \)) in millibars (\( \text{mbar} \)), ranging from \( 0.1 \, \text{mbar} \) to \( 70 \, \text{mbar} \).  
- The curve labeled "Sublimationspunkt" (sublimation point) separates the "flüssig" (liquid) region from the "flüssig-gasförmig" (liquid-gaseous) region.  
- Step "i" is depicted as a horizontal line at a constant pressure, while step "ii" involves a vertical drop in pressure below the sublimation point.  

TASK 4e  
The temperature would decrease very slowly because only a small amount of heat would be transferred through convection.  

TASK 4b  
A table is drawn with columns labeled \( P \), \( T \), and \( x \). Rows are numbered from 1 to 4, but the entries are incomplete:  
- Row 1: No values are provided.  
- Row 2: Pressure (\( P \)) is marked, but temperature (\( T \)) and vapor quality (\( x \)) are blank.  
- Row 3: Pressure (\( P \)) is marked, but temperature (\( T \)) and vapor quality (\( x \)) are blank.  
- Row 4: Vapor quality (\( x \)) is marked as \( 0 \), but pressure (\( P \)) and temperature (\( T \)) are blank.  

No further explanation or calculations are provided for the table.