TASK 2a  
The process steps for the jet engine are described as follows:  
- **0 → 1**: Adiabatic compression, \( p = 0.191 \, \text{bar} \).  
- **1 → 2**: Adiabatic and reversible compression (isentropic).  
- **2 → 3**: Isobaric heat addition.  
- **3 → 4**: Adiabatic and irreversible turbine process with entropy generation.  
- **4 → 5**: Isobaric mixing chamber, \( p = 0.5 \, \text{bar} \).  
- **5 → 6**: Reversible adiabatic nozzle (isentropic), \( p = 0.191 \, \text{bar} \).  

A graph is drawn showing a \( T \) vs. \( s \) diagram (temperature vs. entropy).  
- The diagram includes labeled states (0, 1, 2, 3, 4, 5, 6).  
- The temperature axis is marked with values: \( 1289 \, \text{K} \), \( 431.9 \, \text{K} \), \( 243.15 \, \text{K} \), and \( 200 \, \text{K} \).  
- The entropy axis is labeled \( s \, [\text{kJ}/\text{kg·K}] \).  
- Processes are shown as isentropic (vertical lines) and isobaric (horizontal lines).  
- The curve for heat addition is marked as "isobaric."  

TASK 2b  
The first law of thermodynamics is applied to the nozzle:  
\[
0 = \dot{m} \left( h_e - h_a + \frac{w_e^2 - w_a^2}{2} \right)
\]
Simplified to:  
\[
0 = h_5 - h_6 + \frac{w_5^2 - w_6^2}{2}
\]  
Further rearranged:  
\[
h_5 - h_6 = c_p (T_5 - T_6)
\]  

The nozzle is described as isentropic.