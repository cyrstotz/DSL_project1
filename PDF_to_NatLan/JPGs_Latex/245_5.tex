TASK 4a  
Two diagrams are drawn, both labeled with axes \( p \, [\text{mbar}] \) and \( T \, [\text{K}] \).  

1. The first diagram illustrates the freeze-drying process in a pressure-temperature (\( p \)-\( T \)) diagram. It shows four states connected by arrows:  
   - State 1 to State 2 is labeled "isobar."  
   - State 2 to State 3 is labeled "adiabat."  
   - State 3 to State 4 is labeled "isobar."  
   - State 4 to State 1 is labeled "adiabat."  

2. The second diagram also represents the process in a \( p \)-\( T \) diagram, with similar states and transitions:  
   - State 1 to State 2 is labeled "isobar."  
   - State 2 to State 3 is labeled "adiabat."  
   - State 3 to State 4 is labeled "isobar."  
   - State 4 to State 1 is labeled "adiabat."  

TASK 4b  
The temperature \( T_2 \) is calculated as:  
\[
T_2 = T_i - 6 \, \text{K}
\]  
The inlet temperature \( T_i \) is determined as:  
\[
T_i = T_{\text{triple}} + 10 \, \text{K}
\]  
Substituting values:  
\[
T_2 = 0^\circ \text{C} + 10 \, \text{K} - 6 \, \text{K} = 4^\circ \text{C}
\]  

For the transition from state 2 to state 3, the energy balance is expressed as:  
\[
0 = \dot{m} (h_2 - h_3) + \dot{Q}^0 - \dot{W}
\]  
The refrigerant mass flow rate \( \dot{m} \) is calculated as:  
\[
\dot{m} = \frac{\dot{W}}{h_2 - h_3}
\]  

TASK 4c  
For the transition from state 4 to state 1, the energy balance is expressed as:  
\[
0 = \dot{m} (h_4 - h_1) + \dot{Q}^0 + \dot{W}^0
\]  
It is noted that:  
\[
h_4 = h_1
\]  