TASK 4a  
The page contains several diagrams related to thermodynamic processes.  

1. **Diagram 1**: A pressure-volume (\(p\)-\(V\)) diagram is shown with a curve labeled "1" and "2". The curve starts at point 1 and transitions to point 2, indicating a thermodynamic process.  
2. **Diagram 2**: A temperature-entropy (\(T\)-\(S\)) diagram is depicted with multiple points labeled "1", "2", "3", and "4". The process transitions between these points, forming a closed loop. The arrows indicate the direction of the process.  

TASK 4b  
The energy balance equation is partially written:  
\[
\dot{m} (h_1 - h_2) + \dot{Q} - \dot{W} \geq 0
\]  
This equation represents the energy conservation for a thermodynamic system.  

Further notes mention:  
- \(h_2\) is related to \(x\), implying the vapor quality.  
- \(h_3 (8 \, \text{bar}) = h_3 (\text{value unclear})\).  

TASK 4c  
Another \(p\)-\(V\) diagram is shown with points labeled "1", "2", and "3". The process transitions between these points, with arrows indicating the direction of the process.  

Additional notes:  
- \(T_i = 16 \, \text{K}\) above the sublimation temperature.  
- Chamber pressure is \(5 \, \text{mbar}\) below the triple point of water.  

TASK 4d  
The energy balance equation is repeated with some crossed-out text:  
\[
\dot{m} (h_1 - h_2) + \dot{Q} - \dot{W} = 0
\]  
This equation is used to calculate the heat transfer or work done in the system.  

No further clear content is visible.  

Descriptions of diagrams:  
- The diagrams illustrate thermodynamic processes in \(p\)-\(V\) and \(T\)-\(S\) spaces, with labeled points and arrows indicating the direction of the processes.  
- The diagrams are consistent with the freeze-drying process described in Task 4.