TASK 4a  
Two diagrams are drawn to represent the freeze-drying process.  

1. **First diagram**:  
   - The graph is a pressure-temperature (\( p \)-\( T \)) diagram.  
   - The phase regions are indicated with curves, showing transitions between solid, liquid, and vapor phases.  
   - Two points are labeled:  
     - \( (i) \): Corresponds to the initial state, with pressure at approximately 5 mbar.  
     - \( (ii) \): Corresponds to the sublimation step, with pressure reduced further below the triple point of water.  
   - The axes are labeled \( p \) (pressure) and \( T \) (temperature).  

2. **Second diagram**:  
   - The graph is a pressure-volume (\( p \)-\( V \)) diagram.  
   - The process transitions are shown as lines, with two states labeled:  
     - \( (i) \): Initial state at 5 mbar.  
     - \( (ii) \): Final state after sublimation.  
   - The axes are labeled \( p \) (pressure) and \( V \) (volume).  

TASK 4b  
The equation for the work done by the compressor is given as:  
\[
-\dot{W}_K = \dot{m} \left( h_2 - h_3 \right)
\]  
Where:  
- \( h_2 = h_g \) at \( T_{\text{evap}} - 6 \, \text{K} = 249.3 \, \frac{\text{kJ}}{\text{kg}} \)  
- \( h_3 = h_f \) at \( 8 \, \text{bar} = 269.15 \, \frac{\text{kJ}}{\text{kg}} \)  

The mass flow rate \( \dot{m} \) is calculated as:  
\[
\dot{m} = \frac{-\dot{W}_K}{h_2 - h_3} = \frac{1}{249.3 - 269.15} = 1.22378 \, \frac{\text{kg}}{\text{s}}
\]