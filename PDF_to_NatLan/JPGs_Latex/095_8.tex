TASK 3a  
The initial temperature of the gas is \( T_{g,1} = 500^\circ\text{C} \), and its volume is \( V_{g,1} = 3.14 \, \text{L} \). The diameter of the cylinder is \( D = 0.1 \, \text{m} \), and the piston mass is \( m_K = 32 \, \text{kg} \). The gravitational force acting on the piston is calculated as:  
\[
F = m_K \cdot g = 32 \cdot 9.81 = 313.92 \, \text{N}.
\]  

The cross-sectional area of the cylinder is determined using the formula for the area of a circle:  
\[
A = \pi r^2, \quad r = \frac{D}{2} = 0.05 \, \text{m}.
\]  
\[
A = \pi \cdot (0.05)^2 = 7.853 \cdot 10^{-3} \, \text{m}^2.
\]  

The pressure exerted by the piston is calculated as:  
\[
p_{\text{durch Gewicht}} = \frac{F}{A} = \frac{313.92}{7.853 \cdot 10^{-3}} = 0.04 \, \text{bar}.
\]  

In the ice-water chamber (EW), the pressure is \( p_{\text{EW}} = 1.4 \, \text{bar} \), so the gas chamber pressure is also \( p_{g,1} = 1.4 \, \text{bar} \).  

The gas mass \( m_g \) is calculated using the ideal gas law:  
\[
p_g V_g = m_g R_g T_g.
\]  
Rearranging for \( m_g \):  
\[
m_g = \frac{p_g V_g}{R_g T_g}.
\]  
Substituting values:  
\[
p_g = 1.4 \, \text{bar} = 1.4 \cdot 10^5 \, \text{Pa}, \quad V_g = 3.14 \cdot 10^{-3} \, \text{m}^3, \quad R_g = \frac{R}{M_g} = \frac{8.314}{50} = 0.166 \, \text{kJ/kg·K}, \quad T_g = 773.15 \, \text{K}.
\]  
\[
m_g = \frac{1.4 \cdot 10^5 \cdot 3.14 \cdot 10^{-3}}{0.166 \cdot 773.15} = 3.405 \cdot 10^{-3} \, \text{kg} = 3.405 \, \text{g}.
\]  

A sketch of the cylinder is drawn, showing the piston, gas chamber, and ice-water chamber separated by a membrane.  

---

TASK 3b  
The final temperature of the gas is \( T_{g,2} = T_{\text{EW},2} \), and the pressure in the ice-water chamber remains \( p_{\text{EW}} = 1.4 \, \text{bar} \).  

Since equilibrium is maintained, the gas pressure \( p_{g,2} \) also remains at \( 1.4 \, \text{bar} \).  

