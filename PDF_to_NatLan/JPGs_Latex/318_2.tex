TASK 3a  
To calculate the gas pressure \( p_{g,1} \) and mass \( m_g \) in state 1:  

The gas constant \( R \) is determined as:  
\[
R = \frac{\bar{R}}{M} = 166.28 \, \frac{\text{J}}{\text{kg·K}}
\]  

The pressure \( p_{g,1} \) is calculated using the formula:  
\[
p_{g,1} = p_{\text{amb}} + \frac{m_K \cdot g}{\pi \cdot \left(0.05 \, \text{m}\right)^2} = 40,094.49 \, \text{Pa}
\]  
This corresponds to \( p_{g,1} = 1.8 \, \text{bar} \).  

The gas mass \( m_g \) is calculated using the ideal gas law:  
\[
m_g = \frac{p \cdot V}{R \cdot T} = \frac{1.4 \, \text{bar} \cdot 0.00314 \, \text{m}^3}{166.28 \, \frac{\text{J}}{\text{kg·K}} \cdot 773.15 \, \text{K}} = 3.1 \, \text{g}
\]  

---

TASK 3b  
The specific heat capacity \( c_p \) is calculated as:  
\[
c_p = R + c_v = 799.28 \, \frac{\text{J}}{\text{kg·K}}
\]  

The ratio of specific heat capacities \( \frac{c_p}{c_v} \) is given as:  
\[
\frac{c_p}{c_v} = n = 1.26
\]  

The pressure remains constant from state 1 to state 2:  
\[
p_{g,2} = 1.4 \, \text{bar}
\]  

The temperature \( T_{g,2} \) of the gas will be slightly above \( 0^\circ\text{C} \), as ice is still present, and the ice-water mixture has a temperature of approximately \( 0^\circ\text{C} \).  

A small sketch is drawn showing the relationship between \( T_{g,1} \), \( T_{g,2} \), and \( p_{g,1} \). It indicates that \( T_{g,2} \) is close to \( 0^\circ\text{C} \).  

No additional diagrams or graphs are provided.