TASK 2a  
The diagram depicts a phase diagram with temperature \( T \) on the x-axis and pressure \( p \) on the y-axis. The curve represents the phase boundary between solid, liquid, and vapor states. Key points and regions are labeled:  
- "fest" (solid) is below the curve.  
- "flüssig" (liquid) is to the right of the curve.  
- "dampf" (vapor) is above the curve.  
- The "Tripelpunkt" (triple point) is marked where the solid, liquid, and vapor phases coexist.  
- An arrow labeled "2a" points to a region near the vapor phase.  

TASK 2b  
Energy balance for the compressor:  
\[
Q = \dot{m}_{\text{coolant}} \cdot (h_{\text{ein}} - h_{\text{aus}}) - \dot{W}_{\text{K}}
\]  
The mass flow rate of the coolant is given by:  
\[
\dot{m}_{\text{coolant}} = \frac{\dot{W}_{\text{K}}}{h_2 - h_3}
\]  
Where:  
- \( h_2 \) is the enthalpy at state 2 (saturated vapor).  
- \( h_3 \) is the enthalpy at state 3.  

Additional notes:  
- Pressure interval is specified as 1 bar.  
- \( h_2 \) at 8 bar is referenced, and \( \Delta h \) is calculated.  
- \( h_3 \) is determined using data tables, yielding \( h_3 = 264.16 \, \text{kJ/kg} \).  
- The enthalpy difference \( \Delta h \) is used for further calculations.