TASK 2a  
The diagram represents a qualitative T-s diagram for the jet engine process. The axes are labeled as follows:  
- The vertical axis is temperature \( T \) in Kelvin \([K]\).  
- The horizontal axis is entropy \( S \) in \([R_2/k]\).  

The diagram includes several isobars labeled \( p_2 = p_3 \), \( p_5 = p_4 \), and \( p_0 \). The process is drawn with orange curves and arrows, indicating the thermodynamic transitions between states:  
- State 0 to State 2: Compression.  
- State 2 to State 3: Isobaric heat addition.  
- State 3 to State 4: Expansion in the turbine.  
- State 4 to State 5: Mixing.  
- State 5 to State 6: Isentropic nozzle expansion.  

TASK 2b  
The transition from state 5 to state 6 is described as an isentropic expansion process.  

The following formulas and values are provided:  
1. The kinetic energy per unit mass is given as:  
\[
ke = \frac{w^2}{2}
\]  
where \( w \) is the velocity.  

2. The relationship for temperature \( T_6 \) is derived using the isentropic relation:  
\[
T_6 = T_5 \cdot \left( \frac{p_5}{p_6} \right)^{\frac{k-1}{k}}
\]  
where \( k = 1.4 \) is the specific heat ratio.  

Given values:  
\[
p_0 = p_6 = 0.191 \, \text{bar}, \quad T_5 = 431.9 \, \text{K}, \quad p_5 = 0.5 \, \text{bar}
\]  

Substituting into the formula:  
\[
T_6 = T_5 \cdot \left( \frac{p_5}{p_6} \right)^{\frac{k-1}{k}} = 431.9 \cdot \left( \frac{0.5}{0.191} \right)^{\frac{1.4-1}{1.4}}
\]  
\[
T_6 = 568.58 \, \text{K}
\]  

The final temperature at state 6 is calculated as \( T_6 = 568.58 \, \text{K} \).