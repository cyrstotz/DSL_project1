TASK 3a  
The gas pressure \( p_{g,1} \) is calculated using the equation:  
\[
p_{g,1} V_{g,1} = m_{g,1} R T_{g,1}
\]  
The pressure \( p_{g,1} \) is given as:  
\[
p_{g,1} = 1 \, \text{bar} + \frac{m_{\text{EW}} g}{A} + \frac{m_K g}{A}
\]  
Substituting values:  
\[
p_{g,1} = 100000 \, \text{Pa} + \frac{0.1 \, \text{kg} \cdot 9.81 \, \text{m/s}^2}{25 \, \text{cm}^2} + \frac{32 \, \text{kg} \cdot 9.81 \, \text{m/s}^2}{25 \, \text{cm}^2}
\]  
\[
p_{g,1} = 1.40094 \, \text{bar}
\]  

The mass \( m_{g,1} \) is calculated as:  
\[
m_{g,1} = \frac{p_{g,1} V_{g,1}}{R T_{g,1}} = 0.008422 \, \text{kg}
\]  

The gas constant \( R \) is given as:  
\[
R = \frac{R_u}{M}
\]  

The volume \( V_{g,1} \) is:  
\[
V_{g,1} = 3.14 \, \text{L} = 0.00314 \, \text{m}^3
\]  

TASK 3b  
The pressure in state 2 remains unchanged:  
\[
p_{g,2} = p_{g,1} = 1.40094 \, \text{bar}
\]  

It is stated that nothing has changed in terms of pressure or mass. The temperature \( T_{g,2} \) is equal to \( T_{\text{EW},2} \), indicating equilibrium:  
\[
T_{g,2} = T_{\text{EW},2} \rightarrow \text{Equilibrium}
\]  

The mass \( m_{g,2} \) remains the same as \( m_{g,1} \).  

TASK 3c  
The heat transferred between states 1 and 2 is given as:  
\[
Q_{12} = W_{12}
\]  

No further explanation or derivation is provided for this subtask.  

No diagrams or figures are present on this page.