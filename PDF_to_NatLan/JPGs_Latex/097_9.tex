TASK 4a  
The page contains two diagrams illustrating the freeze-drying process in a pressure-temperature (\(P\)-\(T\)) diagram.  

**Description of the first diagram (top):**  
- The vertical axis represents pressure (\(P\)) in millibars (\(mbar\)), ranging from 5 mbar to higher values.  
- The horizontal axis represents temperature (\(T\)) in degrees Celsius (\(^\circ\text{C}\)), ranging from \(-50^\circ\text{C}\) to \(10^\circ\text{C}\).  
- The diagram shows phase regions labeled as "Flüssig" (liquid), "Sgas" (solid-gas equilibrium), and "Wasser in Lebensmitteln" (water in food).  
- The triple point (\(T_{\text{Tripel}}\)) is marked, and the sublimation curve extends from the triple point into the solid-gas region.  
- A horizontal line at \(T = 10^\circ\text{C}\) is drawn, indicating the temperature above the sublimation temperature during the freeze-drying process.  

**Description of the second diagram (bottom):**  
- The vertical axis again represents pressure (\(P\)) in millibars (\(mbar\)), with a focus on the region around 5 mbar.  
- The horizontal axis represents temperature (\(T\)) in degrees Celsius (\(^\circ\text{C}\)), ranging from \(-50^\circ\text{C}\) to \(10^\circ\text{C}\).  
- The phase regions are labeled as "Flüssig" (liquid), "Sgas" (solid-gas equilibrium), and the triple point (\(T_{\text{Tripel}}\)) is marked.  
- Two points are labeled:  
  - Point 1 corresponds to the initial state during freeze-drying.  
  - Point 2 corresponds to the final state, where \(T = \text{constant}\).  
- A vertical dashed line connects the two points, indicating the transition during the process.  

Both diagrams visually represent the freeze-drying process and the phase changes of water under reduced pressure and controlled temperature conditions.