TASK 4a  
A graph is drawn showing a pressure-temperature (\(p-T\)) diagram. The diagram includes a dome-shaped curve representing the phase boundary between liquid and vapor regions. Two horizontal lines are drawn across the dome, indicating isobaric processes. The x-axis is labeled as \(T\) (temperature), and the y-axis is labeled as \(p\) (pressure).  

TASK 4b  
The energy balance equation for the process between states 2 and 3 is written as:  
\[
0 = \dot{m} (h_2 - h_3) + \dot{W}
\]  
Rearranging for the mass flow rate:  
\[
\dot{m} (h_3 - h_2) = \dot{W}
\]  
\[
\dot{m} = \frac{\dot{W}}{(h_3 - h_2)}
\]  
It is noted that \(p_3 = 8 \, \text{bar}\).  

For the process between states 3 and 4, the energy balance is written as:  
\[
0 = \dot{m} (h_4 - h_2) + \dot{Q}_K
\]  
The entropy condition is stated:  
\[
S_2 = S_3
\]  

The enthalpy at state 3 (\(h_3\)) is calculated using interpolation:  
\[
h_3 = h_f + \frac{S_2 - S_f}{S_g - S_f} (h_g - h_f)
\]  

The enthalpy at state 2 (\(h_2\)) is determined as:  
\[
h_2 = h(GK)
\]  
This is noted to involve interpolation from Table A-12.