TASK 4a  
The diagram is a pressure-temperature (\( p \)-\( T \)) graph illustrating the freeze-drying process.  
- The graph has labeled regions for "solid" (\( \text{Fest} \)), "liquid" (\( \text{Flüssig} \)), and "gas" (\( \text{Gas} \)).  
- The process is divided into two steps:  
  - Step \( i \): An isobaric process between states \( 1 \) and \( 2 \).  
  - Step \( ii \): An isothermal process between states \( 2 \) and \( 3 \).  
- The curve separating the phase regions represents the equilibrium lines for phase transitions.  

TASK 4e  
The equation for the energy balance is given as:  
\[
Q = \dot{m}_{\text{R134a}} \cdot (h_4 - h_2) + \dot{Q}_K
\]  
The initial temperature of the system is specified as \( T_i = -20^\circ\text{C} \).  

The enthalpy at state \( 1 \) is partially written as:  
\[
h_1 = (\ldots)
\]  

No further details are provided for \( h_1 \).