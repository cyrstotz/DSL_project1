TASK 1a  
A graph is drawn with pressure \( P \) on the vertical axis and temperature \( T \) on the horizontal axis. The graph appears to be a simple qualitative representation, with no specific data points or curves labeled.

---

TASK 1b  
The entropy at state 2 is equal to the entropy at state 3:  
\[
S_2 = S_3
\]

An energy balance equation is written:  
\[
0 = \dot{m}_{\text{R134a}} \left( h_2 - h_3 \right) + \dot{Q}_{23} - \dot{W}_{23}
\]

The work term is defined as:  
\[
\dot{W}_{23} = \dot{W}_K = -28 \, \text{kW}
\]

The mass flow rate of the refrigerant is calculated:  
\[
\dot{m}_{\text{R134a}} = \frac{\dot{Q}_K}{\Delta h} = \frac{4}{7.111 \times 10^{-3}} \, \text{kg/s}
\]

Another energy balance equation is written:  
\[
0 = \dot{m}_{\text{R134a}} \left( h_2 - h_1 \right)
\]

The temperature at state 2 is given as:  
\[
T_2 = -22^\circ\text{C} = 251.15 \, \text{K}
\]

The entropy at state 3 is interpolated using data from Table A-10:  
\[
S_3 = S_2 = S(T = -22^\circ\text{C}) = 0.9089 - 0.9102 \frac{24 - 22}{24 - 20} + 0.9089 = 0.90955 \, \text{kJ/kg·K}
\]