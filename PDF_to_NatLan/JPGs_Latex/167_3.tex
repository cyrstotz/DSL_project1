TASK 2a  
A qualitative \( T \)-\( s \) diagram is drawn, showing the thermodynamic process for the jet engine. The diagram includes labeled isobars and states:  
- State 1 is at the lowest temperature and entropy.  
- State 2 shows an increase in temperature and entropy due to compression.  
- State 3 represents the combustion process, with a significant increase in temperature.  
- State 4 shows a decrease in temperature and entropy due to expansion in the turbine.  
- States 5 and 6 represent mixing and nozzle exit, respectively, with entropy increasing slightly.  

The axes are labeled as follows:  
- \( T \) (temperature) on the vertical axis, in Kelvin (\( K \)).  
- \( s \) (specific entropy) on the horizontal axis, in \( \text{kJ}/\text{kg·K} \).  

TASK 2b  
The energy balance equation for the jet engine is written as:  
\[
Q = \dot{m} \left( h_e - h_a + \frac{w_e^2 - w_a^2}{2} \right) + Q - W
\]  
From this, the temperature at state \( T_2 \) is calculated using the isentropic relation:  
\[
T_2 = T_1 \left( \frac{p_2}{p_1} \right)^{\frac{\kappa - 1}{\kappa}}
\]  
The ideal gas law is also noted:  
\[
p = \rho R T
\]  

TASK 3a  
The pressure \( p \) is calculated using the formula:  
\[
p = \frac{m_g}{V_g} R T
\]  
Substituting values, the pressure is determined to be approximately \( 7.40094 \, \text{bar} \).  

The mass \( m \) of the gas is calculated using the ideal gas law:  
\[
p V = m R T
\]  
Rearranging for \( m \):  
\[
m = \frac{p V}{R T}
\]  
The result is \( m = 3.4017 \, \text{g} \).  

TASK 3b  
The pressure \( p_{g,2} \) is calculated as:  
\[
p_{g,2} = \frac{m_g R T}{V_g} + p_{\text{amb}}
\]  
Substituting values, the pressure is determined to be \( 7.294 \, \text{bar} \).  

A note is added:  
"Der Druck ist eine Kombination des Atmosphärendrucks und der Last am Kolben."  
Translation: "The pressure is a combination of atmospheric pressure and the load on the piston."