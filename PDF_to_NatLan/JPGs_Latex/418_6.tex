TASK 3a  
The universal gas constant \( R \) is calculated as:  
\[
R = \frac{\bar{R}}{M_g} = \frac{8.314 \, \text{J/mol·K}}{50 \, \text{kg/kmol}} = 0.1663 \, \text{kJ/kg·K}.
\]

The gas volume \( V_{g,1} \) is given as:  
\[
V_{g,1} = 3.14 \, \text{L} = 3.14 \cdot 10^{-3} \, \text{m}^3.
\]

The gas pressure \( p_{g,1} \) is determined using the ideal gas law:  
\[
p_{g,1} V_{g,1} = R \cdot T_{g,1} \cdot m_{g,1}.
\]

Rearranging for \( p_{g,1} \):  
\[
p_{g,1} = \frac{R \cdot T_{g,1} \cdot m_{g,1}}{V_{g,1}} = \frac{0.1663 \, \text{kJ/kg·K} \cdot (500 + 273.15) \, \text{K}}{3.14 \cdot 10^{-3} \, \text{m}^3}.
\]

The calculated pressure is:  
\[
p_{g,1} = 28636 \, \text{kPa}.
\]

The pressure \( p_{\text{EW}} \) is calculated as:  
\[
p_{\text{EW}} = \frac{g \cdot (m_K + m_{\text{EW}})}{\pi \cdot (0.1 \, \text{m})^2} + p_{\text{amb}}.
\]

Substituting values:  
\[
p_{\text{EW}} = \frac{9.81 \, \text{m/s}^2 \cdot (0.1 \, \text{kg} + 32 \, \text{kg})}{\pi \cdot (0.1 \, \text{m})^2} + 100023.61 \, \text{Pa}.
\]

The result is:  
\[
p_{\text{EW}} = 111348.6101 \, \text{Pa}.
\]

The gas mass \( m_{g,1} \) is calculated as:  
\[
m_{g,1} = \frac{p_{g,1} \cdot V_{g,1}}{R \cdot T_{g,1}} = \frac{28636 \, \text{kPa} \cdot 3.14 \cdot 10^{-3} \, \text{m}^3}{0.1663 \, \text{kJ/kg·K} \cdot (500 + 273.15) \, \text{K}}.
\]

The calculated mass is approximately:  
\[
m_{g,1} \approx 2.445 \cdot 10^{-4} \, \text{kg}.
\]

---

TASK 3b  
The pressure \( p_{g,2} \) remains unchanged because the temperature difference caused by the melting process is compensated by the volume change. The top of the cylinder is sealed, so the pressure remains constant throughout the process.