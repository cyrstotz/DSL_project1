TASK 3a  
The gas pressure \( p_{g,1} \) is calculated using the equation:  
\[
p_g = p_0 + \frac{F}{A}
\]  
The result is:  
\[
p_{g,1} = 1.4 \, \text{bar}
\]  

The gas constant \( R \) is determined as:  
\[
R = \frac{\bar{R}}{M} = 0.16628 \, \text{kJ/(kg·K)}
\]  

The mass of the gas \( m_g \) is calculated using the ideal gas law:  
\[
PV = mRT \quad \Rightarrow \quad m_g = \frac{p_g V_g}{R T_g} = 3.42 \, \text{g}
\]  

---

TASK 3b  
The temperature \( T_{g,2} \) and pressure \( p_{g,2} \) are discussed. It is noted that the mass on the piston does not change, and the volume of the ice-water mixture (EW) remains constant. The gas compresses, resulting in higher pressure but a lower temperature.  

The temperature \( T_{g,2} \) is calculated as:  
\[
T_{g,2} = \frac{p_{g,2} V_{g,2}}{m_g R}
\]  

The relationship between pressure and volume is described using the polytropic process equation:  
\[
p V^n = \text{constant} \quad \text{(where \( n = v \))}
\]  

The volume \( V_2 \) is calculated as:  
\[
V_2 = \frac{RT_g}{p} = 3.14 \times 10^{-3} \, \text{m}^3
\]  

---

TASK 3c  
The heat transferred \( Q_{12} \) is calculated using the energy balance equation:  
\[
\Delta E = m_2 u_2 - m_1 u_1 + \Delta KE + \Delta PE = \sum \Delta m_i \left( h_i + \frac{w_i^2}{2} + g z_i \right) + \sum Q - \sum W
\]  

The heat \( Q \) is expressed as:  
\[
Q = m_g (u_2 - u_1)
\]  

Using specific heat capacity \( c_v \):  
\[
Q_{12} = m_g c_v (T_2 - T_1)
\]  

The result is:  
\[
Q_{12} = -1082.42 \, \text{J}
\]  

---

TASK 3d  
The final ice fraction \( x_{\text{ice},2} \) is determined.  

The heat \( Q \) is expressed as:  
\[
Q = m_{\text{EW}} (u_2 - u_1)
\]  

The internal energy \( u_2 \) is calculated as:  
\[
u_2 = u_{\text{ice}} + x \cdot (u_{\text{water}} - u_{\text{ice}})
\]  

Values used:  
\[
u_{\text{ice}} = -444.23 \, \text{kJ/kg}, \quad u_{\text{water}} = -122.59 \, \text{kJ/kg}
\]  

The heat transfer equation is:  
\[
Q_{12} = m_{\text{EW}} c_{\text{EW}} (T_{\text{EW},2} - T_{\text{EW},1})
\]  

It is noted that:  
\[
Q + T_{\text{EW},1} = T_{\text{EW},2}
\]  

The volume relationship is described as:  
\[
V_A = V_2 = \text{constant}
\]  

No further numerical results are provided for \( x_{\text{ice},2} \).  

---  
Descriptions of diagrams or graphs: None are present on this page.