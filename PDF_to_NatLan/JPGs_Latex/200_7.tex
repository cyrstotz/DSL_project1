TASK 3c  
The energy balance equation is written as:  
\[
\Delta E = \Sigma Q - \Sigma EW
\]  
This simplifies to:  
\[
\Delta E = Q_{12} - W_v
\]  
Further, the change in energy is expressed as:  
\[
\Delta E = \Delta U = Q_{12} - W_v
\]  
The change in internal energy is calculated using:  
\[
\Delta U = m \cdot c_V \cdot (T_2 - T_1)
\]  
Substituting values:  
\[
\Delta U = 3.479 \, \text{g} \cdot 0.633 \, \frac{\text{kJ}}{\text{kg·K}} \cdot (0 - 500 \, \text{K})
\]  
\[
\Delta U = -1082 \, \text{J}
\]  

The work done by the gas is calculated as:  
\[
W_v = \int_{V_1}^{V_2} p_{g,1} \, dv = p_{g,1} \cdot (V_2 - V_1)
\]  
The volume \( V_2 \) is determined using:  
\[
V_2 = \frac{m \cdot R \cdot T_2}{p_2}
\]  
Substituting values:  
\[
V_2 = 0.00711 \, \text{m}^3 = 1.71 \, \text{L}
\]  
The work done is then:  
\[
W_v = -285.6 \, \text{J}
\]  

The heat transfer is calculated as:  
\[
Q_{12} = \Delta U + W_v
\]  
\[
Q_{12} = -1082 \, \text{J} + (-285.6 \, \text{J})
\]  
\[
Q_{12} = -1367.6 \, \text{J}
\]  

TASK 3d  
The initial ice fraction is given as:  
\[
x_{\text{ice},1} = 0.6
\]  
The specific internal energy of the ice-water mixture is calculated as:  
\[
U_1 = 0.6 \cdot (-333.858) + (1 - 0.6) \cdot (-0.005 \, \frac{\text{kJ}}{\text{kg}})
\]  
\[
U_1 = -200.1 \, \frac{\text{kJ}}{\text{kg}}
\]  

The change in internal energy per unit mass of the ice-water mixture is expressed as:  
\[
\Delta U_{\text{EW}} = \frac{Q_{12}}{m_{\text{EW}}}
\]