TASK 4b  
The energy balance for the compressor is given as:  
\[
0 = \dot{m} (h_2 - h_3) + \dot{W}_K
\]  
where \( \dot{W}_K \) represents the work input to the compressor, \( h_2 \) and \( h_3 \) are the specific enthalpies at the inlet and outlet of the compressor, and \( \dot{m} \) is the mass flow rate.  

A diagram is sketched showing the flow of energy through the compressor, with arrows indicating work input (\( \dot{W}_K \)) and enthalpy changes (\( h_2 \) and \( h_3 \)).  

The mass flow rate is calculated as:  
\[
\dot{m} = \frac{\dot{W}_K}{h_2 - h_3}
\]  

It is noted that entropy remains constant during the compression process:  
\[
s_2 = s_3
\]  

TASK 4c  
The pressure at state 4 is equal to the pressure at state 2:  
\[
p_4 = p_2
\]  

The vapor quality at state 4 is zero (\( x_4 = 0 \)), indicating that the refrigerant is fully condensed. The pressure at state 4 is given as:  
\[
p_4 = 8 \, \text{bar}
\]  

The temperature at state 4 is approximately:  
\[
T_4 = 31 \, \text{to} \, 33^\circ\text{C}
\]  

TASK 4d  
For the throttling process (AHS Drossel), the energy balance is written as:  
\[
\dot{m} (h_4 - h_1) = 0
\]  
This implies that the specific enthalpy remains constant during throttling:  
\[
h_4 = h_1
\]  

No diagrams or additional figures are provided beyond the brief sketches mentioned above.