TASK 4a  
The diagram is a pressure-temperature (\( p \)-\( T \)) graph illustrating the freeze-drying process. It includes labeled regions for "mass", "vapor", and "wet vapor" (or "mixed phase"). The curve represents the phase boundary between liquid and vapor. Key states are marked:  
- State 1: Located in the vapor region.  
- State 2: Transitioning into the wet vapor region.  
- State 3: Positioned in the wet vapor region.  
- State 4: Condensation occurs at the boundary between wet vapor and liquid.  
The horizontal axis is labeled as temperature (\( T \)) in degrees Celsius, and the vertical axis is labeled as pressure (\( p \)) in bar.

---

TASK 4b  
The cooling cycle is described in terms of the states:  
- **State 1**: \( p_1 \) and \( x_1 = 0.9 \), indicating partial evaporation.  
- **State 2**: \( p_2 = p_1 \), \( x_2 = 1 \), meaning the refrigerant is fully evaporated as gas.  
- **State 3**: \( p_3 = 8 \, \text{bar} \), transitioning into the wet vapor region.  
- **State 4**: \( p_4 = p_3 \), \( x = 0 \), indicating complete condensation at \( 8 \, \text{bar} \).  

The ambient temperature \( T_a \) is given as \( 3.1 \, \text{°C} \).  

The initial temperature \( T_i \) is 10 K above the sublimation point of water and slightly below the triple point.  

The temperature difference \( \Delta T_i \) is calculated as:  
\[
\Delta T_i = -10^\circ\text{C} \implies T_{\text{evaporator}} = -16^\circ\text{C} \implies T_2 = -7^\circ\text{C}
\]

For state 2, \( x = 1 \), \( T_2 = -7^\circ\text{C} \), \( T_{\text{ambient}} = -10^\circ\text{C} \), and \( p_2 = p_1 = 2.5748 \, \text{bar} \).  

It is noted that \( p_2 \) and \( p_3 \) are given, and the work \( W_k \) of the compressor is adiabatic and reversible.  

The equation for \( \frac{W_{23}}{\dot{m}} \) cannot be solved because no polytropic process is defined.  

A small arrow labeled "R134a" is drawn at the bottom of the page.