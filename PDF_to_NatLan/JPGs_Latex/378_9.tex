TASK 4a  
Two diagrams are drawn:  

1. **Diagram 1**:  
   - A rectangular cycle is plotted on a \( T \)-\( p \) diagram.  
   - The cycle consists of four numbered points (1, 2, 3, 4) connected by arrows indicating the direction of the process.  
   - The process moves sequentially:  
     - From point 1 to point 2 (upward in \( T \)),  
     - From point 2 to point 3 (rightward in \( p \)),  
     - From point 3 to point 4 (downward in \( T \)),  
     - From point 4 back to point 1 (leftward in \( p \)).  

2. **Diagram 2**:  
   - A triangular cycle is plotted on a \( T \)-\( p \) diagram.  
   - The cycle includes labeled processes:  
     - "isotherm" (horizontal line),  
     - "isochore" (vertical line),  
     - "isobar" (diagonal line).  
   - The cycle also includes a region labeled "Nassdampf" (wet steam).  

---

TASK 4b  
The following equation is written:  
\[
\frac{d\dot{Q}}{dt} = \dot{m} \cdot (h_2 - h_3) + \dot{Q}_{ev} - \dot{W}_K
\]  

Two enthalpy definitions are provided:  
\[
h_2 = h(p_4, \, h_g(p_4, s_2 = s_1))
\]  
\[
h_3 = h(s_3 = s_2, \, 8 \, \text{bar})
\]  

A value is given for \( \dot{W}_K \):  
\[
\dot{W}_K = -28 \, \text{W}
\]  