TASK 3a  
The problem involves determining the force equilibrium in the system.  

The pressure \( p_{g,1} \) in the gas chamber is given by:  
\[
p_{g,1} = p_{\text{amb}} + \frac{(m_K + m_{\text{EW}}) g}{A}
\]  
where \( A = \frac{\pi D^2}{4} \).  

Substituting values:  
\[
p_{g,1} = p_{\text{amb}} + \frac{(m_K + m_{\text{EW}}) g}{A} = 1.4 \, \text{bar}
\]  

The ideal gas law is applied:  
\[
p_{g,1} V_{g,1} = m_g R T_{g,1}
\]  
where \( R = \frac{\bar{R}}{M_g} = 0.16628 \, \text{kJ/(kg·K)} \).  

Rearranging to find the gas mass \( m_g \):  
\[
m_g = \frac{p_{g,1} V_{g,1}}{R T_{g,1}}
\]  

Substituting values:  
\[
m_g = 2.1087 \times 10^{-3} \, \text{kg}
\]  

TASK 3b  
The first law of thermodynamics is applied to the entire system:  
\[
\Delta U = -W
\]  

No diagrams or additional figures are present on the page.