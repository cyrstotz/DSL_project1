TASK 3c  
The first law of thermodynamics for the gas chamber is written as:  
\[
m_{\text{gas}} [u_2 - u_1] = Q - W \quad \Rightarrow \quad Q = -1361.57 \, \text{kJ}
\]  
This represents the heat transfer from the gas to the ice-water mixture, where \( Q \) is negative, indicating heat loss.

The work \( W \) is calculated as:  
\[
W = p_{\text{g}} \left[ V_2 - V_1 \right] = -285.17 \, \text{kJ}
\]  
This work is done by the gas (negative work).

The volumes \( V_2 \) and \( V_1 \) are determined using the ideal gas law:  
\[
V = \frac{m R T}{p}
\]  
For \( V_2 \):  
\[
V_2 = \frac{0.0031 \cdot 8.314 \cdot 273.15}{50 \cdot 10^3} = 1.1028 \times 10^{-3} \, \text{m}^3
\]  
For \( V_1 \):  
\[
V_1 = \frac{0.0031 \cdot 8.314 \cdot (273.15 + 500)}{50 \cdot 10^3} = 3.1203 \times 10^{-3} \, \text{m}^3
\]

The internal energy change \( u_2 - u_1 \) is calculated using the specific heat capacity:  
\[
u_2 - u_1 = c_V (T_2 - T_1) = 0.633 \cdot (-500) = -316.5 \, \text{kJ/kg}
\]  
The total change in internal energy is then:  
\[
m_{\text{gas}} \cdot (u_2 - u_1) = -1076.15 \, \text{kJ}
\]

---

TASK 3d  
The first law of thermodynamics for the ice chamber is written as:  
\[
m_{\text{ice-water}} [u_2 - u_1] = Q = 1361.57 \, \text{kJ}
\]  

The internal energy \( u_2 \) is calculated as:  
\[
u_2 = 1361.57 - 200.059 = -186.1771 \, \text{kJ/kg}
\]  

The ice fraction \( x_2 \) in the final state is determined using the solid-liquid equilibrium:  
\[
x_2 = \frac{-186.1771 + 0.045}{-333.458 + 0.045} = 0.559
\]  

A small diagram is drawn showing the ice-water mixture with a piston above it, indicating the pressure exerted by the piston and atmospheric conditions.