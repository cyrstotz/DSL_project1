TASK 4c  
At state 4, the pressure is \( 8 \, \text{bar} \) and the vapor quality \( x_4 = 0 \).  

For an adiabatic throttle, the enthalpy remains constant:  
\[
h_2 = h_1
\]  

From the refrigerant table (Table A11), the enthalpy at \( 8 \, \text{bar} \) and \( x = 0 \) is:  
\[
h_4 = 93.42 \, \frac{\text{kJ}}{\text{kg}}
\]  

The enthalpy at state 2 is calculated using the mixture equation:  
\[
(1-x) \cdot h_f + x \cdot h_g = h_2
\]  

At \( -22^\circ\text{C} \), the pressure is \( p_2 = 1.2452 \, \text{bar} \). Adding the pressures:  
\[
p_1 + p_2 = 6.2452 \, \text{bar}
\]  

Rearranging for vapor quality \( x \):  
\[
93.42 \, \frac{\text{kJ}}{\text{kg}} = (1-x) \cdot h_f + x \cdot h_g
\]  

---

TASK 4d  
The coefficient of performance \( \epsilon_K \) is given by:  
\[
\epsilon_K = \frac{\dot{Q}_K}{\dot{W}_K}
\]  

This represents the ratio of useful cooling to work input.