TASK 3a  
The gas pressure in state 1 is calculated using the equation:  
\[
p_{g,1} = p_{\text{amb}} + \frac{m_K \cdot g}{A} + \frac{m_{\text{EW}} \cdot g}{A}
\]  
where \( A = \frac{\pi D^2}{4} = 0.03142 \, \text{m}^2 \).  
Substituting the values, the result is:  
\[
p_{g,1} = 1.1 \, \text{bar}
\]  

The mass of the gas is determined using the ideal gas law:  
\[
p_1 V_1 = m R T_1 \quad \Rightarrow \quad m = \frac{p_1 V_1}{R T_1}
\]  
Substituting \( p_1 = 1 \, \text{bar} \), \( V_1 = 3.14 \, \text{L} \), \( R = 8.314 \, \text{J/(mol·K)} \), and \( T_1 = 500^\circ\text{C} = 773.15 \, \text{K} \), the result is:  
\[
m = 2.687 \, \text{g}
\]  

---

TASK 3b  
The temperature of the gas in state 2 is equal to the equilibrium temperature of the ice-water mixture:  
\[
T_{g,2} = T_{\text{EW},1} = 0^\circ\text{C}
\]  
Since the ice does not completely melt, the temperature of the ice-water mixture remains constant. Additionally, because the gas and the ice-water mixture are in thermal equilibrium, they share the same temperature.  

The gas pressure in state 2 remains constant:  
\[
p_{g,2} = p_{g,1} = 1.1 \, \text{bar}
\]  
This is because the external load on the membrane does not change (same mass and atmospheric pressure), so the gas pressure remains constant.  

---

TASK 3c  
The volume of the gas in state 2 is calculated using the ideal gas law:  
\[
V_2 = \frac{m R T_2}{p_2}
\]  
Substituting \( m = 2.687 \, \text{g} \), \( R = 8.314 \, \text{J/(mol·K)} \), \( T_2 = 273.15 \, \text{K} \), and \( p_2 = 1.1 \, \text{bar} \), the result is:  
\[
V_2 = 0.00111 \, \text{m}^3
\]  

The initial volume of the gas is:  
\[
V_1 = 0.00314 \, \text{m}^3
\]  

The work done by the gas during isobaric expansion is:  
\[
W = \int_{V_1}^{V_2} p \, dV = p \cdot (V_2 - V_1)
\]  
Substituting \( p = 1.1 \, \text{bar} = 110,000 \, \text{Pa} \), \( V_2 = 0.00111 \, \text{m}^3 \), and \( V_1 = 0.00314 \, \text{m}^3 \), the result is:  
\[
W = -0.222 \, \text{J}
\]  

The change in internal energy is:  
\[
\Delta U = U_2 - U_1 = Q - W
\]  
The heat transferred is:  
\[
Q = U_2 - U_1 + W
\]  
Using \( U_2 - U_1 = m c_V (T_2 - T_1) \), where \( c_V = 0.633 \, \text{kJ/(kg·K)} \), \( T_2 = 273.15 \, \text{K} \), \( T_1 = 773.15 \, \text{K} \), and \( m = 2.687 \, \text{g} \), the result is:  
\[
U_2 - U_1 = -850.521 \, \text{J}
\]  
Thus, the heat transferred is:  
\[
Q_{12} = -Q = 1.092 \, \text{kJ}
\]  