TASK 3a  
The mass \( m \) is calculated using the ideal gas law:  
\[
m = \frac{p \cdot V}{R \cdot T}
\]  
Given values:  
- \( p = 1.3997 \, \text{bar} \)  
- \( V = 3.14 \cdot 10^{-3} \, \text{m}^3 \)  
- \( T = 773 \, \text{K} \)  
- \( R = \frac{8.314 \, \text{kJ}}{\text{kmol} \cdot \text{K}} \div 50 \, \text{kg/kmol} = 0.1663 \, \frac{\text{kJ}}{\text{kg} \cdot \text{K}} \)  

Substituting these values:  
\[
m = \frac{1.3997 \cdot 10^5 \, \text{Pa} \cdot 3.14 \cdot 10^{-3} \, \text{m}^3}{0.1663 \, \frac{\text{kJ}}{\text{kg} \cdot \text{K}} \cdot 773 \, \text{K}}
\]  
Simplifying:  
\[
m = 3.92 \cdot 10^{-3} \, \text{kg} = 3.92 \, \text{g}
\]  

TASK 3b  
The following questions are posed:  
- \( x_{\text{ice},2} > 0 \)  
- \( T_{g,2} \) is unknown.  
- \( p_{g,2} \) is unknown.  

Key assumptions and observations:  
- \( R \) and \( m \) are constant.  
- \( V_{\text{ist}} \) is smaller because \( p_{g,2} \) does not depend on the volume but rather on the pressure and mass.  

The temperature \( T_2 \) is calculated using the polytropic temperature relationship:  
\[
T_2 = T_1 \left( \frac{p_2}{p_1} \right)^{\frac{n-1}{n}}
\]  
Where:  
- \( C_p = R + C_v = 0.1663 \, \frac{\text{kJ}}{\text{kg} \cdot \text{K}} + 0.633 \, \frac{\text{kJ}}{\text{kg} \cdot \text{K}} = 0.7993 \, \frac{\text{kJ}}{\text{kg} \cdot \text{K}} \)  

Additional notes mention that \( T \) is smaller due to the polytropic process.  

No diagrams or figures are present on this page.