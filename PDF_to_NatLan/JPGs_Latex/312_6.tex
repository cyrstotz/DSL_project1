TASK 4a  
The pressure-temperature (\(p-T\)) diagram is drawn to represent the freeze-drying process. The diagram includes the following features:  
- A phase envelope showing the relationship between pressure and temperature.  
- The triple point is labeled, indicating the coexistence of solid, liquid, and vapor phases.  
- States 1, 2, 3, and 4 are marked along the diagram, with arrows showing transitions between states.  
- The region labeled "Phasengebiet" (phase region) is highlighted.  

TASK 4b  
The equation for the energy balance is written as:  
\[
O = \dot{m}_{\text{R134a}} (h_g - h_h) = -\dot{Q}_{\text{ab}} = \dot{m}_{\text{R134a}} (h_n - h_2) + \dot{Q}_K
\]  
This expression is simplified to:  
\[
\dot{m}_{\text{R134a}} \cdot \frac{\dot{Q}_{\text{ab}}}{h_s - h_4} = -\frac{\dot{Q}_K}{h_n - h_2}
\]  
Further, the energy balance is rewritten as:  
\[
O = \dot{m}_{\text{R134a}} (h_2 - h_3) - \dot{W}_K
\]  
The mass flow rate of the refrigerant is calculated using:  
\[
\dot{m}_{\text{R134a}} = \frac{\dot{W}_K}{h_2 - h_3}
\]  
The enthalpy at state 2 is determined:  
\[
h_2 = f(p_2, x=1) = h_g(p_2) = 259 \, \text{kJ/kg}
\]  
The pressure at state 2 is calculated:  
\[
p_2 = 6 \, \text{bar} - 5 \, \text{mbar} = 5.995 \, \text{bar} = p_1
\]  
The pressures at states 3 and 4 are given as:  
\[
p_3 = p_4 = 8 \, \text{bar}
\]