TASK 4b  
The enthalpy \( h_3 \) is interpolated using the formula:  
\[
h_3 = h_{\text{sat}} + \frac{h(40) - h_{\text{sat}}}{s(r0) - s_{\text{sat}}} \cdot (s_{82} - s_{\text{sat}})
\]  
The interpolated value for \( h_3 \) is calculated as \( h_3 = 272.95 \, \text{kJ/kg} \).  

The work \( -W_{12} \) is calculated using:  
\[
-W_{12} = m(h_2 - h_3)
\]  
Rearranging for \( m \):  
\[
m = \frac{-W_{12}}{h_2 - h_3} = 1.52 \, \text{kg}
\]  

---

TASK 4c  
The pressure \( p \) is given as \( p_1 = 1.219 \, \text{bar} \) (from Table A10).  

The enthalpy \( h_1 \) is equal to \( h_{\text{ev}} \).  

The vapor quality \( x \) is calculated using:  
\[
x = \frac{h_4 - h_f}{h_g - h_f}
\]  
Substituting values:  
\[
x = 0.39
\]  

---

TASK 4d  
The coefficient of performance \( \epsilon_K \) is calculated as:  
\[
\epsilon_K = \frac{\dot{Q}_K}{\dot{W}_K} = \frac{Q_K}{W_K}
\]  

The heat \( Q_K \) is determined using:  
\[
Q_K = m(h_2 - h_1)
\]  
Substituting values:  
\[
Q_K = m \cdot \frac{G(40)}{60^2} \cdot (251.8 - 95.42)
\]  
The result is:  
\[
Q_K = 0.76 \, \text{kJ}
\]