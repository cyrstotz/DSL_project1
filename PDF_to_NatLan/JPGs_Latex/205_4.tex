TASK 3a  
The equilibrium of forces is considered for the piston resting on the ice-water mixture (EW).  

The forces acting on the piston are:  
\[
F_T = F_{\text{EW}} = F_g
\]  
where \( F_T \) is the total force, \( F_{\text{EW}} \) is the force exerted by the EW, and \( F_g \) is the gravitational force.  

The pressure \( p \) is calculated using:  
\[
p = \frac{F}{A} \quad \text{(in \(\text{Pa}\))}
\]  
where \( A \) is the area of the piston.  

The piston radius is given as \( r = 0.05 \, \text{m} \), and the area is calculated as:  
\[
A = \pi r^2 = \pi (0.05)^2 = 7.854 \times 10^{-3} \, \text{m}^2
\]  

The gravitational force acting on the piston is:  
\[
F_g = m_K \cdot g = 32 \, \text{kg} \cdot 9.81 \, \text{m/s}^2 = 313.92 \, \text{N}
\]  

The total force \( F_T \) is:  
\[
F_T = F_g + F_{\text{amb}} \cdot A
\]  
where \( F_{\text{amb}} \) is the force due to atmospheric pressure.  

TASK 3b  
The ideal gas law is used to determine the pressure \( p_{g,1} \) and mass \( m_g \) of the gas in state 1.  

The gas pressure is:  
\[
p_{g,1} = 1.4 \, \text{bar} = 1.4 \times 10^5 \, \text{Pa}
\]  

The ideal gas law is expressed as:  
\[
p V = m R T
\]  
where \( R \) is the specific gas constant, \( T \) is the temperature, \( V \) is the volume, and \( m \) is the mass.  

The specific gas constant is calculated as:  
\[
R = \frac{R_u}{M_g} = \frac{8.314 \, \text{J/mol·K}}{0.05 \, \text{kg/mol}} = 166.28 \, \text{J/kg·K}
\]  

The gas temperature is:  
\[
T_{g,1} = 773.15 \, \text{K}
\]  

The volume is converted to cubic meters:  
\[
V_{g,1} = 3.14 \, \text{L} = 3.14 \, \text{dm}^3 = 3.14 \times 10^{-3} \, \text{m}^3
\]  

The gas mass is calculated as:  
\[
m_g = \frac{p V}{R T} = \frac{1.4 \times 10^5 \cdot 3.14 \times 10^{-3}}{166.28 \cdot 773.15} = 0.0342 \, \text{kg}
\]