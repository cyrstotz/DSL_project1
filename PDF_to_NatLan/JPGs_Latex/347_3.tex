TASK 3a  
The gas pressure \( p_{g,1} \) and mass \( m_g \) are calculated as follows:  

The total force exerted by the piston is the sum of the piston weight, atmospheric pressure, and the pressure exerted by the ice-water mixture.  
\[
F_g = m_K \cdot g = 32.169 \, \text{kg} \cdot 9.81 \, \text{m/s}^2 = 319.907 \, \text{N}
\]  
The pressure exerted by the gas is given by:  
\[
p_g = p_0 + \frac{F_g}{A} = p_{\text{atm}} + \frac{319.907}{0.05 \, \text{m}^2}
\]  
Substituting values:  
\[
p_g = 1.401 \, \text{bar}
\]  

To calculate the mass of the gas \( m_g \), the ideal gas law is used:  
\[
m_g = \frac{p_g \cdot V_{g,1}}{R_g \cdot T_{g,1}}
\]  
With \( R_g = \frac{8.314 \, \text{m}^3 \, \text{kPa}/\text{mol·K}}{50 \, \text{kg/kmol}} = 0.16623 \, \text{J/g·K} \):  
\[
m_g = \frac{1.401 \cdot 10^5 \cdot 0.00314}{0.16623 \cdot (500 + 273.15)} = 3.422 \, \text{g}
\]  

---

TASK 3b  
The temperature \( T_{g,2} \) and pressure \( p_{g,2} \) in state 2 are determined as follows:  

The gas and ice-water mixture reach equilibrium in state 2.  
- In state 1, the water is at \( 0^\circ\text{C} \) with \( x_{\text{ice},1} > 0 \), meaning it is in a two-phase state.  
- From the table, at \( 0^\circ\text{C} \), the equilibrium pressure \( p_{\text{EW}} \) is \( 1.4 \, \text{bar} \).  

In state 2:  
\[
p_{\text{EW}} = p_{\text{gas}} = p_{\text{atm}} + p_{\text{EW}}
\]  
Thus, \( T_{g,2} = T_{\text{EW},2} = 0^\circ\text{C} \).  

---

TASK 3c  
The heat transferred \( Q_{12} \) from the gas to the ice-water mixture is calculated using an energy balance:  

The total energy change in the system is:  
\[
\Delta oE = m_g \cdot h_{g,2} - m_g \cdot h_{g,1} = Q_{12} - W_{12}
\]  
Where:  
\[
W_{12} = \int p \, dV = p_a \cdot (V_{g,2} - V_{g,1})
\]  
Given:  
\[
V_{g,2} = 0.00111 \, \text{m}^3, \quad V_{g,1} = 0.00314 \, \text{m}^3
\]  
\[
W_{12} = 1.401 \cdot 10^5 \cdot (0.00111 - 0.00314) = -294.5 \, \text{J}
\]  

The internal energy change of the gas is:  
\[
\Delta u_g = c_V \cdot (T_{g,2} - T_{g,1}) = 0.633 \, \text{kJ/kg·K} \cdot (-500 \, \text{K}) = -316.5 \, \text{J/g}
\]  
\[
\Delta u_g = -316.5 \cdot 3.422 = -1084.7 \, \text{J}
\]  

Thus:  
\[
Q_{12} = W_{12} + m_g \cdot \Delta u_g = -294.5 + (-1084.7) = -1367.5 \, \text{J}
\]  

This represents the heat transferred away from the gas.  

---  
No diagrams or figures are present on this page.