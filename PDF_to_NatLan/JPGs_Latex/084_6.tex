TASK 2a  
The process is described qualitatively in a \( T \)-\( s \) diagram. The diagram includes labeled isobars and transitions between states. The following processes are indicated:  
- \( 0 \to 1 \): Isentropic compression, \( T_1 \).  
- \( 1 \to 2 \): Isentropic compression, \( T_2 \).  
- \( 2 \to 3 \): Isobaric heating, \( T_3 \).  
- \( 3 \to 4 \): Irreversible turbine process, \( T_4 \).  
- \( 4 \to 5 \): Isobaric mixing of bypass and core streams, \( T_5 \).  
- \( 5 \to 6 \): Isentropic expansion, \( T_6 \).  

The diagram shows increasing entropy for the turbine process and isothermal mixing at state \( 5 \). The student notes that the drawing is incomplete and apologizes for missing details.  

TASK 2b  
To determine \( w_6 \) and \( T_6 \):  
The air is modeled as an ideal gas. The specific gas constant \( R \) is calculated as:  
\[
R = \frac{c_{p,\text{air}}}{\kappa - 1} = \frac{8.314 \, \text{kJ/(kmol·K)}}{28.97 \times 10^{-3} \, \text{kg/mol}} = 287 \, \text{J/(kg·K)}.
\]  

For the isentropic process \( 5 \to 6 \), the entropy remains constant (\( s_6 = s_5 \)). Using the given values:  
\[
p_5 = 0.5 \times 10^5 \, \text{Pa}, \quad T_5 = 431.9 \, \text{K}.
\]  
From Table A-22, the corresponding values for \( T_6 \) and \( w_6 \) can be determined.  

Further calculations are incomplete on the page.  

No additional diagrams or graphs are described.