TASK 4a  
A graph is drawn with pressure \( p \) on the vertical axis and temperature \( T \) on the horizontal axis. The diagram represents the freeze-drying process. It includes labeled points:  
- Point 1: The starting state.  
- Point 2: Isobaric evaporation.  
- Point 3: Compression.  
- Point 3': Condensation.  
Arrows indicate the transitions between these states, and the process is described as isobaric and adiabatic in different regions.

---

TASK 4b  
The equation for the cooling process is given as:  
\[
Q = \dot{m}_2 \left[ h_2 - h_3 \right] - \dot{W}_K
\]  
The inlet temperature is:  
\[
T_i = -10^\circ\text{C}
\]  
The temperature after evaporation is:  
\[
T_2 = T_i - 6 \, \text{K} = -16^\circ\text{C}
\]  
From Table A-10:  
\[
h_2 = h_g = 237.74 \, \frac{\text{kJ}}{\text{kg}}
\]  
From Table A-10:  
\[
p_2 = 1.57945 \, \text{bar}
\]

---

TASK 4c  
The pressure is given as:  
\[
p_1 = p_2 = 1.57945 \, \text{bar}
\]  
The temperature is:  
\[
T_1 = T_i = -10^\circ\text{C}
\]  
The vapor quality \( x_1 \) is left incomplete.

---

TASK 4d  
The coefficient of performance is defined as:  
\[
\epsilon_K = \frac{\dot{Q}_K}{\dot{W}_K}
\]  
It is stated that:  
\[
\dot{Q}_{ZU} = \dot{Q}_K
\]  
The work input is:  
\[
\dot{W}_K = \dot{W}_K = 286 \, \text{W}
\]

---

TASK 4e  
No content is provided for this subtask.