TASK 2a  
The page contains a graph and an equation related to thermodynamic processes.  

**Graph Description:**  
The graph is a \( T \) vs. \( s \) diagram, where \( T \) represents temperature in Kelvin and \( s \) represents specific entropy in \( \text{kJ/kg·K} \). The diagram includes labeled isobars and process points:  
- Point 1 is the starting state.  
- Point 2 represents an isentropic compression.  
- Point 3 shows an isobaric heat addition.  
- Point 4 represents an isentropic expansion.  
The isobars are curved lines, and the processes are connected by straight lines between the points.  

TASK 2b  
The equation for the stationary thermodynamic process is written as:  
\[
0 = \dot{m} \left( h_4 - h_5 - T_0 (s_4 - s_5) \right) + \frac{1}{2} \dot{m} \left( w_6^2 - w_5^2 \right) - \dot{W}_{\text{diss}}
\]  
Where:  
- \( \dot{m} \) is the mass flow rate.  
- \( h_4 \) and \( h_5 \) are specific enthalpies at states 4 and 5, respectively.  
- \( T_0 \) is the ambient temperature.  
- \( s_4 \) and \( s_5 \) are specific entropies at states 4 and 5, respectively.  
- \( w_6 \) and \( w_5 \) are velocities at states 6 and 5, respectively.  
- \( \dot{W}_{\text{diss}} \) is the rate of dissipated work.  