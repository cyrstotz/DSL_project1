TASK 3a  
The ideal gas law is used:  
\[
PV = mRT
\]  
The specific gas constant \( R \) is calculated as:  
\[
R = \frac{\bar{R}}{M_g} = \frac{8.314 \, \text{kJ}/\text{kmol·K}}{50 \, \text{kg}/\text{kmol}} = 0.16628 \, \text{kJ}/\text{kg·K}
\]  

The gas pressure \( p_g \) is determined by summing the atmospheric pressure and the pressure exerted by the piston:  
\[
p_g = 1 \, \text{bar} + \frac{F}{A} = 1 \, \text{bar} + \frac{32 \, \text{kg} \cdot 9.81 \, \text{m/s}^2}{\left(\frac{10 \, \text{cm}}{2}\right)^2 \pi}
\]  
Simplifying:  
\[
p_g = 1 \, \text{bar} + 0.396 \, \text{bar} = 1.396 \, \text{bar}
\]  

The mass of the gas \( m_g \) is calculated using the ideal gas law:  
\[
m_g = \frac{p_g V}{RT} = \frac{0.16628 \, \text{kJ}/\text{kg·K} \cdot 500^\circ\text{C}}{1.46 \, \text{cm} \cdot 3.14}
\]  
Result:  
\[
m_g = 2.92 \, \text{g}
\]  

TASK 3b  
No content found.