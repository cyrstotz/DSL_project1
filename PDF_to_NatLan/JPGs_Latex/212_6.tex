TASK 4a  
A graph is drawn showing a pressure-temperature (\( p \)-\( T \)) diagram. The diagram includes four labeled points (1, 2, 3, and 4) connected by lines to represent the freeze-drying process.  
- Point 1 is at the lower left, connected to point 4 by a vertical line.  
- Point 4 is connected to point 3 by a horizontal line.  
- Point 3 is connected to point 2 by a diagonal line sloping upward.  
- Point 2 is connected back to point 1 by a horizontal line.  

The axes are labeled:  
- \( p \) [Pa] on the vertical axis.  
- \( T \) [K] on the horizontal axis.  

TASK 4b  
The mass flow rate of the refrigerant (\( \dot{m}_{\text{R134a}} \)) is calculated using the energy balance:  
\[
0 = \dot{m} \left[ h_2 - h_3 \right] + \dot{W}_K
\]  
The result is:  
\[
\dot{m} = 0.827 \, \text{kg/h}
\]  

From the tables:  
- \( h_2 = h(x_2 = 1) \)  
- \( h_3 = h(8 \, \text{bar}, s_2) \)  

The temperature is given as:  
\[
T = T_i - 6 \, \text{K}
\]  
where \( T_i = -10^\circ\text{C} \), resulting in \( T = -16^\circ\text{C} \).  

From the tables:  
- \( h = 237.94 \, \text{kJ/kg} \)  
- \( s = 0.9288 \, \text{kJ/kg·K} \)  

TASK 4c  
The vapor quality (\( x_1 \)) is determined.  

Given:  
\[
x_1 = 0
\]  

The temperature is:  
\[
T_i = -10^\circ\text{C}
\]  

Using the energy balance:  
\[
0 = \dot{m} \left[ h_4 - h_1 \right]
\]  
where \( h_1 = h_4 \).  

From the tables:  
\[
h = s_{\text{ext}} + s_{40} - s_{5,\text{ext}}
\]  
\[
h = 271.313 \, \text{kJ/kg}
\]  

The vapor quality is confirmed as:  
\[
x_1 = 0
\]