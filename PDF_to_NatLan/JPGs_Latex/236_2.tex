TASK 3a  
The mass of the gas is calculated using the ideal gas law:  
\[
M_g = \frac{140.14 \, \text{kPa} \cdot 3.14 \cdot 10^{-3} \, \text{m}^3}{0.166 \, \frac{\text{kg}}{\text{kJ}} \cdot 773.15 \, \text{K}} = 3.43 \cdot 10^{-3} \, \text{kg} = 3.43 \, \text{g}
\]

---

TASK 3b  
The energy balance (EB) for a piston system is derived from the general energy equation:  
\[
\frac{dE}{dt} = \sum \dot{Q}_j - \sum \dot{W}_n
\]  
For a steady-state process:  
\[
\Delta E = E_2 - E_1 = \sum Q_j - \sum W_n
\]  
Rewriting:  
\[
E_2 - E_1 = Q_{12} - W_{12} \quad \Rightarrow \quad Q_{12} = E_2 - E_1 + W_{12}
\]  

The system boundary is defined as containing only the gas. A small sketch is provided showing a rectangular boundary labeled "Gas" with a membrane separating the system.

---

TASK 3c  
The pressure of the gas is calculated using the ideal gas law:  
\[
pV = mRT \quad \Rightarrow \quad p = \frac{mRT}{V} = \frac{RT}{v}
\]  
The specific volume is determined:  
\[
v = \frac{V}{m} = \frac{3.14 \cdot 10^{-3} \, \text{m}^3}{3.43 \cdot 10^{-3} \, \text{kg}} = 0.915 \, \text{m}^3/\text{kg}
\]  
The pressure at state 2 is:  
\[
p_2 = \frac{RT}{v_1} = 49.56 \, \text{kPa}
\]

The total energy is expressed as:  
\[
E = U + \text{KE} + \text{PE}
\]  
Kinetic and potential energy terms are negligible.  

For a perfect gas, the change in internal energy is:  
\[
\Delta U = c_V (T_2 - T_1)
\]  
Substituting values:  
\[
\Delta U = 0.633 \, \frac{\text{kJ}}{\text{kg·K}} \cdot (273.153 \, \text{K} - 773.15 \, \text{K}) = -316.5 \, \frac{\text{kJ}}{\text{kg}}
\]

The heat transfer is calculated as:  
\[
Q_{12} = \Delta U + W_{12} = W_{12} - 316.5 \, \frac{\text{kJ}}{\text{kg}}
\]