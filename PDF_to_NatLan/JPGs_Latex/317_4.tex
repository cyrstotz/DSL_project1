TASK 3a  
The pressure \( p_{g,1} \) is calculated as the sum of the ambient pressure \( p_{\text{amb}} \) and the pressure exerted by the piston \( p_{\text{piston}} \):  
\[
p_{g,1} = p_{\text{amb}} + p_{\text{piston}}
\]  
The piston pressure is determined using the formula:  
\[
p_{\text{piston}} = \frac{F}{A} = \frac{m_K \cdot g}{\pi \left(\frac{D}{2}\right)^2}
\]  
Substituting the values:  
\[
p_{\text{piston}} = \frac{32 \, \text{kg} \cdot 9.81 \, \text{m/s}^2}{\pi \left(\frac{10 \, \text{cm}}{2}\right)^2} = 0.900 \, \text{bar}
\]  
Thus:  
\[
p_{g,1} = 1.0 \, \text{bar} + 0.9 \, \text{bar} = 1.9 \, \text{bar}
\]  

The ideal gas law is used to calculate the gas mass \( m_g \):  
\[
p_{g,1} \cdot V_1 = m_g \cdot R \cdot T_1
\]  
The specific gas constant \( R \) is calculated as:  
\[
R = \frac{\bar{R}}{M} = \frac{8.314 \, \text{J/mol·K}}{50 \, \text{kg/kmol}} = 166.28 \, \text{J/kg·K}
\]  
Rearranging for \( m_g \):  
\[
m_g = \frac{p_{g,1} \cdot V_1}{R \cdot T_1} = \frac{1.9 \, \text{bar} \cdot 0.00372 \, \text{m}^3}{166.28 \, \text{J/kg·K} \cdot 773.15 \, \text{K}}
\]  
Converting \( \text{bar} \) to \( \text{Pa} \):  
\[
m_g = \frac{1.9 \cdot 10^5 \, \text{Pa} \cdot 0.00372 \, \text{m}^3}{166.28 \, \text{J/kg·K} \cdot 773.15 \, \text{K}} = 0.00392 \, \text{kg} = 3.92 \, \text{g}
\]  

---

TASK 3b  
The system is in thermodynamic equilibrium and isolated from the surroundings (no heat exchange). Therefore, the temperature \( T_{g,2} \) must equal \( T_{\text{EW},2} = 0^\circ\text{C} \). Since the ambient pressure and piston pressure remain unchanged, the gas pressure in state 2 is:  
\[
p_{g,2} = p_{g,1} = 1.9 \, \text{bar}
\]  

---

TASK 3c  
Using the first law of thermodynamics:  
\[
\Delta W = Q_{12} - W_{12}
\]  
The change in internal energy \( \Delta U \) is calculated as:  
\[
\Delta U = m_g \cdot c_V \cdot \Delta T
\]  
Substituting values:  
\[
\Delta U = 3.92 \, \text{g} \cdot 0.633 \, \text{kJ/kg·K} \cdot (0) = 0
\]  

The work \( W_{12} \) is calculated as:  
\[
W_{12} = \int p \, dV = p_{g,1} \cdot (V_2 - V_1)
\]  
The final volume \( V_2 \) is:  
\[
V_2 = \frac{m_g \cdot R \cdot T_2}{p_{g,2}} = \frac{0.00392 \, \text{kg} \cdot 166.28 \, \text{J/kg·K} \cdot 273.15 \, \text{K}}{1.9 \cdot 10^5 \, \text{Pa}} = 0.00110 \, \text{m}^3
\]  
Substituting into the work equation:  
\[
W_{12} = 1.9 \, \text{bar} \cdot (0.00110 \, \text{m}^3 - 0.00372 \, \text{m}^3)
\]  
Converting \( \text{bar} \) to \( \text{Pa} \):  
\[
W_{12} = 1.9 \cdot 10^5 \, \text{Pa} \cdot (-0.00262 \, \text{m}^3) = -289.2 \, \text{J}
\]