TASK 4a  
The task involves drawing the freeze-drying process in a \( p \)-\( T \) diagram. Several diagrams are sketched, showing the phase regions and transitions.  

1. **First diagram**:  
   - The \( p \)-\( T \) curve is drawn with labeled axes (\( P \) in bar and \( T \) in Kelvin).  
   - A dome-shaped curve represents the phase boundary.  
   - Two points are marked:  
     - Point 1 is near the lower pressure region.  
     - Point 2 is inside the dome, indicating a phase transition.  
   - An isobaric line is drawn across the dome.  

2. **Second diagram**:  
   - Similar \( p \)-\( T \) curve with the dome-shaped phase boundary.  
   - Point 2 is marked inside the dome, and an arrow indicates a transition.  

3. **Table of values**:  
   - A table is provided with columns for \( P \), \( T \), and other variables:  
     - Row 1: \( P_u \), unspecified temperature.  
     - Row 2: \( P = 8 \, \text{bar} \), \( T = -22^\circ\text{C} \), \( x_2 = 1 \).  
     - Row 3: \( P = 8 \, \text{bar} \), unspecified temperature, \( x_3 = 1 \).  
     - Row 4: \( P = 8 \, \text{bar} \), unspecified temperature, \( x_4 = 0 \).  

4. **Equations and notes**:  
   - \( a_{41} = 0 \).  
   - \( P_u = S_{\text{mbar}} + P_{\text{tp}} \).  
   - \( T_i - 10 \, \text{K} = T_{\text{sublimation point}} \).  
   - \( S_3 = S_2 \).  
   - \( T_i - T_3 = 6 \, \text{K} \).  

5. **Additional diagrams**:  
   - Another \( p \)-\( T \) curve is drawn with labeled points 2, 3, and 4.  
   - Isobaric and isentropic lines are indicated.  

TASK 4b  
The task involves determining the required refrigerant mass flow rate \( \dot{m}_{\text{R134a}} \).  

1. **Equation for energy balance**:  
   \[
   \frac{dE}{dt} = \dot{m}_R (h_2 - h_3) + \dot{Q}_K - \dot{W}_K  
   \]  
   \[
   \dot{W}_K = \dot{m}_{\text{R134a}} (h_2 - h_3)  
   \]  

2. **Values and interpolation**:  
   - \( h_2(8 \, \text{bar}, x_2 = 1) = h_g(8 \, \text{bar}) = 264.15 \, \text{kJ/kg} \) (from Table A-11).  
   - \( h_3(8 \, \text{bar}, S_2) \): Interpolation is performed at \( 8 \, \text{bar} \) with \( S_2 \).  
   - \( S_2(8 \, \text{bar}, x = 1) = 0.9066 \, \text{kJ/kg·K} = S_3 \).  

3. **Interpolation formula**:  
   \[
   h_3 = h(S_x) - h(S_r) \cdot \frac{(S_2 - S_r)}{x - S_x}  
   \]  

No further calculations are provided.  

Descriptions of diagrams and equations are included, but no final numerical results are visible.