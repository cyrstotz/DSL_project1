TASK 2a  
The diagram represents a qualitative T-s diagram for the jet engine process. It includes labeled isobars and states:  
- State 1: Isentropic compression.  
- State 2: Combustion process at constant pressure \( p_2 \).  
- State 3: Expansion in the turbine.  
- State 4: Mixing process.  
- State 5: Outlet conditions with \( T_5 = 431.9 \, \text{K} \) and \( p_5 = 0.5 \, \text{bar} \).  
- State 6: Isentropic expansion in the nozzle with \( p_6 = p_0 = 0.191 \, \text{bar} \).  

The graph shows temperature \( T \) on the vertical axis and entropy \( s \) on the horizontal axis. The isobars \( p_0 \), \( p_5 \), and \( p_2 \) are drawn as curves, with transitions between states labeled clearly.  

---

TASK 2b  
The outlet temperature \( T_6 \) is calculated using the isentropic relation:  
\[
T_6 = T_5 \left( \frac{p_6}{p_5} \right)^{\frac{\kappa - 1}{\kappa}}
\]  
Substituting values:  
\[
T_6 = 431.9 \left( \frac{0.191}{0.5} \right)^{\frac{0.4}{1.4}} = 328.07 \, \text{K}
\]  

The outlet velocity \( w_6 \) is derived from the energy balance:  
\[
w_6^2 = 2 \left( h_5 - h_6 \right) + w_5^2
\]  
where \( h_5 \) and \( h_6 \) are specific enthalpies at states 5 and 6, respectively, and \( w_5 \) is the velocity at state 5.  

Further derivations and calculations are partially crossed out and unclear.