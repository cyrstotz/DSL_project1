TASK 4a  
A graph is drawn representing a pressure-temperature (\(p\)-\(T\)) diagram. The diagram includes the following labeled regions:  
- "gesättigte Flüssigkeit" (saturated liquid) on the left side.  
- "Nassdampfgebiet" (wet steam region) in the middle, enclosed by a dome-shaped curve.  
- "überhitzter Dampf" (superheated steam) on the right side.  

The dome-shaped curve represents the phase boundary between saturated liquid and saturated vapor. Three points are marked:  
- Point 1 is located within the wet steam region.  
- Point 2 is also within the wet steam region, closer to the curve's apex.  
- Point 3 is located in the superheated steam region, outside the dome.  

The temperature axis (\(T\)) is labeled with a value of \(-2^\circ\text{C}\) near the saturated liquid region. The pressure axis (\(p\)) is shown vertically but has no numerical values marked.

---

TASK 4b  
The equations for stationary processes are written as follows:  
\[
0 = \dot{m} \left[ h_e - h_a \right] + \sum Q - \sum W
\]
\[
W = \dot{m} \left[ h_e - h_a \right]
\]  
Where:  
- \( \dot{m} \) is the mass flow rate.  
- \( h_e \) and \( h_a \) are the specific enthalpies at the exit and inlet, respectively.  
- \( Q \) represents heat transfer.  
- \( W \) represents work.