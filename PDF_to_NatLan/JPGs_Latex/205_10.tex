TASK 4d  
The task involves calculating the coefficient of performance \( \epsilon_K \) for a refrigeration cycle. The following equations and steps are provided:

1. The coefficient of performance \( \epsilon_K \) is defined as:  
   \[
   \epsilon_K = \frac{\dot{Q}_K}{\dot{W}_K}
   \]

2. The heat removed \( \dot{Q}_K \) is given as \( 0.12 \, \text{kJ} \).

3. The work input \( \dot{W}_K \) is calculated using enthalpy differences:  
   \[
   \dot{W}_K = \dot{m} \cdot (h_2 - h_1)
   \]

4. Enthalpy values are interpolated:  
   - \( h_2 = 237.78 \, \text{kJ/kg} \)  
   - \( h_1 = 93.42 \, \text{kJ/kg} \)

5. Substituting values:  
   \[
   \dot{W}_K = 0.628 \, \text{kW}
   \]

6. Finally, the coefficient of performance is calculated:  
   \[
   \epsilon_K = \frac{0.12}{0.628} = 4.2857
   \]

No diagrams or figures are present on this page.