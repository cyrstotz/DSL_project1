TASK 4a  
The diagram is a pressure-temperature (\(p\)-\(T\)) graph. It shows the phase regions of a refrigerant, with labeled points corresponding to states in the freeze-drying process.  
- The x-axis represents temperature (\(T\)) in Kelvin, and the y-axis represents pressure (\(p\)) in bar.  
- The curve separates the liquid and vapor phases.  
- State 2 is marked on the saturated vapor line, while state 1 is on the saturated liquid line.  
- A horizontal line connects states 1 and 2, indicating isobaric evaporation.  
- State 3 is shown at a higher pressure (\(p_3 = 8 \, \text{bar}\)) and lies in the superheated vapor region.  

TASK 4b  
The inlet temperature is given as \( T_i = -70^\circ\text{C} \).  
The evaporation temperature is \( T_z = T_i - 6 \, \text{K} = -78^\circ\text{C} \).  
The pressure at state 3 is \( p_3 = 8 \, \text{bar} \).  
The entropy at state 3 is equal to the entropy at state 2:  
\[
S_3 = S_2
\]  

TASK 4c  
No content found.