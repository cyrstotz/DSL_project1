TASK 3a  
The pressure of the gas in state 1 (\( p_{g,1} \)) is calculated as:  
\[
p_{g,1} = p_{\text{EW}} + \frac{m_K \cdot g}{A}
\]  
Substituting values:  
\[
p_{g,1} = 1.4 \, \text{bar}
\]  

The mass of the gas (\( m_{\text{gas}} \)) is determined using the ideal gas law:  
\[
m_{\text{gas}} = \frac{p_{g,1} \cdot V_{g,1}}{R \cdot M \cdot T_{g,1}}
\]  
Substituting values:  
\[
m_{\text{gas}} = 3.4224 \, \text{g}
\]  

Two diagrams are drawn to depict the forces acting on the piston and the pressure contributions. The first diagram shows the pressure from the ice-water mixture (\( p_{\text{EW}} \cdot A \)) and the weight of the piston (\( m_K \cdot g \)) acting downward. The second diagram includes the same forces but explicitly labels the pressure contributions from the ambient pressure (\( p_0 \)) and the piston weight divided by the area (\( \frac{m_K \cdot g}{A} \)).  

---

TASK 3b  
The temperature of the gas in state 2 (\( T_{g,2} \)) is equal to the temperature of the ice-water mixture in state 1 (\( T_{\text{EW},1} \)):  
\[
T_{g,2} = T_{\text{EW},1} = 0^\circ \text{C}
\]  

Explanation:  
The ice-water mixture does not change its temperature due to the heat flow because it is already in the solid-liquid equilibrium region. This process is isothermal.  

In equilibrium, the gas and the ice-water mixture have the same temperature.  

A scribbled-out section appears below this explanation.  

The pressure in state 2 (\( p_{a2} \)) remains constant:  
\[
p_{a2} = p_{a1} = 1.4 \, \text{bar}
\]  

Further explanation:  
The pressure remains unchanged because the force balance does not change, even though heat is transferred (refer to Task 2).