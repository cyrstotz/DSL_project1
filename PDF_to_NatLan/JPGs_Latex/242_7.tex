TASK 3d  
The energy balance equation is written as:  
\[
\Delta U = Q - W \quad \text{(with \( W = 0 \), as the process is adiabatic)}  
\]  
This simplifies to:  
\[
m \cdot (u_2 - u_1) = Q  
\]  
where \( u_2 \) and \( u_1 \) represent the specific internal energies at states 2 and 1, respectively.  

The ice mass fraction \( x_2 \) at state 2 is calculated using:  
\[
Q = \dot{m} \cdot \Delta u \cdot (0.003) + x_2 \cdot (\Delta u_{\text{ice}} - \Delta u_{\text{water}})  
\]  
Substituting values:  
\[
Q = -200.1 - m \cdot u_{\text{ice}}(0.003) - x_2 \cdot (u_{\text{ice}} - u_{\text{water}})  
\]  
Rearranging for \( x_2 \):  
\[
x_2 = \frac{Q - (-200.1) - u_{\text{ice}}(0.003)}{- (u_{\text{ice}} - u_{\text{water}})(0.003)}  
\]  

Specific internal energy values are substituted:  
\[
u_{\text{ice}}(0.003) = -0.033  
\]  
\[
u_{\text{water}}(0.003) = -333.64  
\]  

This yields:  
\[
x_2 = 0.5696  
\]  
or equivalently:  
\[
x_2 = 56.96\%  
\]  

Additional notes clarify that the initial ice mass fraction \( x_{\text{ice},1} \) is 0.6, corresponding to 60% ice.  

No diagrams or figures are present on the page.