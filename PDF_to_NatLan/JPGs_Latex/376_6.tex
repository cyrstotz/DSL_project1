TASK 3a  
The mass of the gas (\( m_g \)) is calculated using the ideal gas law:  
\[
m_g = \frac{p V}{R T}
\]  
The specific gas constant (\( R \)) is derived from the universal gas constant (\( \bar{R} \)) divided by the molar mass (\( \mu \)):  
\[
R = \frac{\bar{R}}{\mu} = \frac{8.314 \, \text{J/(mol·K)}}{50 \, \text{kg/kmol}}
\]  
This simplifies to:  
\[
R = 0.16628 \, \text{J/(g·K)} = 166.28 \, \text{J/(kg·K)}
\]  
Substituting into the equation, the mass of the gas is calculated as:  
\[
m_g = 0.003922 \, \text{kg} = 3.922 \, \text{g}
\]  

TASK 3b  
The variables \( T_{g,2} \) and \( p_{g,2} \) are mentioned but no further calculations or explanations are provided.