TASK 2b  
The calculation begins with the expression for the outlet velocity \( w_6 \):  
\[
1.006 \, \frac{\text{kJ}}{\text{kg}} \left( 431.9 \, \text{K} - 328.1 \, \text{K} \right) - 74.51 \, \text{kW} + \frac{w_0^2}{2} = \frac{w_6^2}{2}
\]  
From this, the outlet velocity squared is calculated:  
\[
w_6^2 = 54122.8
\]  
The outlet velocity is then determined:  
\[
w_6 = 329 \, \frac{\text{m}}{\text{s}}
\]  

---

TASK 2c  
The exergy destruction rate is calculated using the following equation:  
\[
-\dot{E}_{\text{x,str}} = \dot{E}_{\text{x,str,0}} - \dot{E}_{\text{x,str,6}}
\]  
This expands to:  
\[
\dot{E}_{\text{x,str}} = \dot{m} \left( h_0 - h_6 - T_0 \left( s_0 - s_6 \right) + \Delta \text{ke} \right)
\]  
Substituting the terms:  
\[
-\dot{E}_{\text{x,str}} = h_0 - h_6 - T_0 \left( s_0 - s_6 \right) + \frac{w_0^2 - w_6^2}{2}
\]  
Further expanded:  
\[
-\dot{E}_{\text{x,str}} = c_p \left( T_0 - T_6 \right) - T_0 \left( \ln \frac{T_0}{T_6} - R \ln \frac{p_0}{p_6} \right) + \frac{w_0^2 - w_6^2}{2}
\]  
Numerical substitution:  
\[
-1.006 \, \frac{\text{kJ}}{\text{kg}} \left( 297.15 \, \text{K} - 328.1 \, \text{K} \right) - 243.751 \, \text{K} \left( \ln \frac{297.15 \, \text{K}}{328.1 \, \text{K}} \right) - 0.287 \, \frac{\text{kJ}}{\text{kg}} \left( \ln(1) \right)
\]  
\[
+ \frac{200^2}{2} - \frac{329^2}{2}
\]  
The result is:  
\[
\Delta E_{\text{x,str}} = 119.5 \, \frac{\text{kJ}}{\text{kg}}
\]  

---

TASK 2d  
The exergy balance equation is written as:  
\[
\dot{E}_{\text{x,verl}} = \dot{E}_{\text{x,verl,mixes}}
\]  
Expanded:  
\[
0 = -\Delta \dot{E}_{\text{x,str}} + \dot{E}_{\text{x,aj}} - W_{\text{in}} - \dot{E}_{\text{x,verl}}
\]  
The specific exergy is calculated as:  
\[
ex_{\text{a}} = \left( 1 - \frac{T_0}{T_B} \right) \frac{\dot{Q}}{\dot{m}}
\]  

---

No diagrams or figures are present on this page.