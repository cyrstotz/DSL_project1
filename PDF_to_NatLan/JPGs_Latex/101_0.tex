TASK 4a  
The table describes the states of the refrigerant during the freeze-drying process. The columns represent pressure (\(p\)), temperature (\(T\)), specific volume (\(V\)), specific work (\(W\)), specific entropy (\(s\)), and heat transfer rate (\(\dot{Q}\)).  

- **State 1**:  
  \( p_1 < 8 \, \text{bar} \), \( T_1 = T_i - 6 \, \text{K} \), \( \dot{Q}_K \).  
  The refrigerant is in a gaseous, saturated state.  

- **State 2**:  
  \( p_2 = p_1 \), \( T_2 = T_i \), \( W_K \), \( s_2 = s_3 \).  

- **State 3**:  
  \( p_3 = 8 \, \text{bar} \), \( T_3 \), liquid phase.  

- **State 4**:  
  \( p_4 = 8 \, \text{bar} \), \( T_4 \), liquid phase.  

---

The page includes three diagrams:  

1. **First diagram**:  
   A pressure-temperature (\(p\)-\(T\)) graph with labeled saturation curve.  
   - The saturation line is shown, and state \(i\) is marked on the curve.  
   - The pressure \(p_1 = p_2\) is indicated, and the entropy \(s_2 = s_3\) is noted.  

2. **Second diagram**:  
   Another \(p\)-\(T\) graph with labeled states \(1\), \(2\), \(3\), and \(4\).  
   - The states are connected by arrows indicating the transitions between processes.  
   - The saturation curve is visible, and the pressure at \(1 \, \text{bar}\) is marked.  

3. **Third diagram**:  
   A \(p\)-\(T\) graph with labeled processes:  
   - Isobaric and isentropic transitions are shown.  
   - States \(1\), \(2\), \(3\), and \(4\) are labeled, with arrows indicating the direction of the processes.  

Descriptions of the diagrams emphasize the thermodynamic processes and transitions between states during the freeze-drying cycle.