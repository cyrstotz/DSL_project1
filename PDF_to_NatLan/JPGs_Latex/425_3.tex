TASK 3a  
The gas pressure \( p_{g,1} \) is calculated using the formula:  
\[
p_{g,1} = \frac{-m_{\text{EW}} g}{A} + \frac{m_K g}{A} + p_{\text{amb}}
\]  
The cross-sectional area \( A \) is determined as:  
\[
A = \pi \left(\frac{D}{2}\right)^2 = 0.0079 \, \text{m}^2
\]  
Substituting values, the pressure is:  
\[
p_{g,1} = 1.33861 \, \text{bar}
\]  

The mass of the gas \( m_g \) is calculated using the ideal gas law:  
\[
p V = m_g R T
\]  
where \( R = \frac{\bar{R}}{M_g} = 0.16628 \, \text{kJ/kg·K} \).  
Substituting values, the mass is:  
\[
m_g = 0.00394 \, \text{kg} \quad \text{or} \quad 3.94 \times 10^{-3} \, \text{kg}
\]  

TASK 3b  
The system is in equilibrium.

TASK 3c  
The heat transferred \( Q_{12} \) from the gas to the ice-water mixture is calculated as:  
\[
Q_{12} = m_g c_V \Delta T
\]  
Substituting values:  
\[
Q_{12} = 0.00394 \, \text{kg} \cdot 0.633 \, \text{kJ/kg·K} \cdot (500 \, \text{K} - 273.15 \, \text{K})
\]  
\[
Q_{12} = 1.133 \, \text{kJ}
\]  

TASK 3d  
The equilibrium temperature \( T_{\text{EW},2} \) is equal to \( T_{g,2} \), as the system reaches thermal equilibrium.  

The final ice fraction \( x_{\text{ice},2} \) is calculated using the energy balance:  
\[
x_{\text{ice},1} + x_{\text{ice},2} = \frac{u_2 - u_{\text{frost}}}{u_{\text{flüssig}} - u_{\text{frost}}}
\]  
where all values are taken at \( p = 1 \, \text{bar} \) and \( T = 0^\circ\text{C} \).  

The heat \( Q_{12} \) is equal to the change in internal energy \( \Delta U \), as the process is isochoric:  
\[
Q_{12} = \Delta U = u_2 - u_1
\]  
Substituting values:  
\[
u_2 = u_1 + \frac{Q_{12}}{m_{\text{EW}}} = -133.2 \, \text{kJ/kg} + \frac{1.500 \, \text{kJ}}{0.1 \, \text{kg}}
\]  
\[
u_2 = -118.2 \, \text{kJ/kg}
\]  

Finally, the ice fraction \( x_{\text{ice},2} \) is determined as:  
\[
x_{\text{ice},2} = 0.555
\]