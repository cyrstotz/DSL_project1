TASK 4a  
The page contains two diagrams related to the freeze-drying process described in Task 4.

1. **First Diagram**:  
   - The graph is a pressure-temperature (\( p \)-\( T \)) diagram.  
   - The axes are labeled as \( p \) (pressure) on the vertical axis and \( T \) (temperature) on the horizontal axis.  
   - The diagram includes phase regions separated by lines.  
   - A curved line represents the boundary between liquid and vapor phases.  
   - The label "NO" appears near the curve, possibly indicating a region where sublimation or phase transition does not occur.  
   - The diagram also includes straight lines crossing the phase boundary, which may represent isobaric or isothermal processes.

2. **Second Diagram**:  
   - Another \( p \)-\( T \) graph is shown below the first.  
   - The vertical axis is labeled \( p \) (pressure), and the horizontal axis is labeled \( T \) (temperature).  
   - A point labeled \( T_i \) appears on the horizontal axis, indicating the initial temperature of the freeze-drying process.  
   - The graph is mostly empty, with no additional curves or annotations beyond the axes and \( T_i \).

No further textual explanation or mathematical expressions are visible on the page.