TASK 4a  
The diagram shows a pressure-temperature (\(p\)-\(T\)) graph. The pressure (\(p\)) is plotted on the vertical axis, and the temperature (\(T\)) in Kelvin is plotted on the horizontal axis. The graph includes a dome-shaped curve representing phase regions, with a horizontal line drawn within the dome to indicate isobaric evaporation.

---

TASK 4b  
The process from state 2 to state 3 is described as adiabatic and reversible, leading to an isentropic change (\(s_2 = s_3\)). The process from state 3 to state 4 is isobaric.

---

TASK 4c  
The second diagram is another \(p\)-\(T\) graph, similar to the first, but with labeled states 1, 2, 3, and 4. The curve represents phase regions, and the states are connected by lines indicating the thermodynamic processes described.

---

TASK 4d  
The coefficient of performance (\(\epsilon_K\)) is calculated using the formula:  
\[
\epsilon_K = \frac{\lvert Q_{\text{zu}} \rvert}{\lvert W_{\text{K}} \rvert} = \frac{\lvert Q_{\text{ab}} \rvert}{\lvert Q_{\text{ab}} - Q_{\text{zu}} \rvert}
\]

---

TASK 4e  
The text states that the temperature (\(T_i\)) would cool down faster ("würde schneller abkühlen") if the cooling cycle from Step i continued with constant heat removal (\(\dot{Q}_K\)).

---

Additional Formula:  
The vapor quality (\(x_1\)) at state 1 is calculated using:  
\[
x_1 = \frac{u_1 - u_{f,1}}{u_{g,1} - u_{f,1}}
\]