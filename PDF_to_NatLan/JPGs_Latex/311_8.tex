TASK 4a  
The page contains three diagrams illustrating thermodynamic processes in a \( p \)-\( V \) (pressure-volume) diagram. Each diagram is labeled with key regions and states. Below is a description of the diagrams:

1. **First Diagram**:  
   - The graph shows a curved line representing the phase regions: "Flüssig" (liquid), "Nassdampf" (wet steam), and "Dampf" (steam).  
   - States 1 and 4 are marked within the "Nassdampf" region.  
   - The curve transitions from the liquid phase to the wet steam region and then to the steam phase.  

2. **Second Diagram**:  
   - Similar to the first diagram, the phase regions "Flüssig," "Nassdampf," and "Dampf" are labeled.  
   - States 1 and 4 are again marked within the "Nassdampf" region.  
   - The curve transitions through the same phases, with an additional label "Isobare" (isobaric process) near the steam region.  

3. **Third Diagram**:  
   - This diagram is more detailed, showing states 1, 2, 3, and 4.  
   - The phase regions "Flüssig" (liquid) and "Nass Dampf" (wet steam) are labeled.  
   - The curve transitions through the states, with arrows indicating the direction of the process.  
   - Labels "P3" and "P4" are visible, indicating pressure levels, and "Isobare" is written near the steam region.  

The diagrams visually represent the thermodynamic cycle involving phase changes and pressure-volume relationships.