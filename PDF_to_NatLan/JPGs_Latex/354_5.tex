TASK 3a  
The pressure of the gas in state 1 is given as:  
\[
p_1 = 1.49 \, \text{bar}
\]  
The mass of the gas is:  
\[
m_g = m_2 = 0.1 \, \text{kg}
\]  
The specific internal energy \( U_1 \) is calculated using the following equation:  
\[
U_1 = U(0^\circ\text{C}, 1.49 \, \text{bar}) = U_f(0^\circ\text{C}, 1.49 \, \text{bar}) + x(U_g(0^\circ\text{C}, 1.49 \, \text{bar}) - U_f(0^\circ\text{C}, 1.49 \, \text{bar}))
\]  
Substituting values:  
\[
U_1 = -133.4 \, \frac{\text{kJ}}{\text{kg}}
\]  

TASK 3d  
The final ice fraction \( x_2 \) is determined using the following formula:  
\[
x_2 = \frac{U_2 - U_{\text{fest}}}{U_{\text{flüssig}} - U_{\text{fest}}}
\]  

The specific volume of the gas in state 2 is calculated as:  
\[
V_{2g} = \frac{m \, R \, T_2}{p} = 1.11 \cdot 10^{-3} \, \text{m}^3
\]  

The work done is given by:  
\[
W = p_1 (V_2 - V_1) = 28.27 \, \text{J}
\]  

The mass relationship is stated as:  
\[
m_g = m_2 = m_w
\]  

The specific internal energy \( U_2 \) is calculated using the energy balance:  
\[
m_2 U_2 - m_w U_w = Q_{12} - W_{12}
\]  
Rearranging:  
\[
U_2 = \frac{Q_{12} - W_{12}}{m_w} + U_w
\]  
Substituting values:  
\[
U_2 = -122.59 \, \frac{\text{kJ}}{\text{kg}}
\]  

Finally, the ice fraction \( x_2 \) is calculated as:  
\[
x_2 = \frac{U_2 - U_{\text{fest}}}{U_{\text{flüssig}} - U_{\text{fest}}} = 63.28 \, \%
\]