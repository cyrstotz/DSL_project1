TASK 4a  
A p-T diagram is drawn, showing the freeze-drying process. The diagram includes labeled phase regions: "solid," "gasförmig" (gaseous), and "ND" (likely referring to the triple point or phase transition region). The diagram indicates isochoric and isobaric processes, as well as the transition between states (labeled i, ii, iii, and iv). The x-axis is labeled as \( T \, [K] \), and the y-axis is labeled as \( p \, [\text{bar}] \).  

TASK 4b  
The refrigerant mass flow rate \( \dot{m}_{\text{R134a}} \) is calculated using the formula:  
\[
\dot{m}_{\text{R134a}} = \frac{\dot{Q}_K}{h_{4} - h_{1}} = \frac{6.2455 \, \text{kW}}{93.42 \, \frac{\text{kJ}}{\text{kg}}} \quad \text{(from Table A.11)}
\]  
The enthalpy values are referenced from Table A.11, and the process from state 4 to state 1 is described as isenthalpic, meaning \( h_4 = h_1 = 93.42 \, \frac{\text{kJ}}{\text{kg}} \).  

TASK 4c  
The mass flow rate \( \dot{m}_{\text{R134a}} \) is given as \( 6.62 \, \frac{\text{kg}}{\text{h}} \).  
The temperature at state 2 is \( T_2 = -72^\circ\text{C} \).  
The enthalpy values are described as equal:  
\[
h_4 = h_1 \quad \text{and} \quad h_1 = h_f(8 \, \text{bar}) = 93.42 \, \frac{\text{kJ}}{\text{kg}} = h_4
\]  

TASK 4d  
No content is provided for this subtask.  

