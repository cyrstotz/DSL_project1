TASK 4a  
A graph is drawn representing the freeze-drying process in a pressure-temperature (\(p\)-\(T\)) diagram. The graph includes the following features:  
- The x-axis is labeled as \(T [K]\), representing temperature in Kelvin.  
- The y-axis is labeled as \(p [\text{bar}]\), representing pressure in bar.  
- Three distinct regions are marked: "Solid-Dampf" (solid-vapor), "Dampf" (vapor), and the "Triple Point," which is indicated as a specific point on the curve.  
- The curve transitions smoothly between the solid-vapor region and the vapor region, illustrating phase boundaries.  

TASK 4b  
The following thermodynamic steps and calculations are described:  
1. **Isobaric evaporation**: \(p_1 = p_2\), with \(x_1 = 1\), indicating complete evaporation.  
2. **Adiabatic isochoric process**: \(s_2 = s_3\), where entropy remains constant.  
3. **Isobaric process**: \(p_3 = p_4\), with \(x_4 = 0\), indicating no vapor quality.  

The energy balance equation is given as:  
\[
0 = \dot{m} (h_2 - h_3) - \dot{W}_K
\]  

Specific enthalpy and entropy values are calculated:  
- \(h_3 = h_2(p = 8 \, \text{bar}) = 264.15 \, \frac{\text{kJ}}{\text{kg}}\)  
- \(s_3 = s_2 = s_1(p = 8 \, \text{bar}) = 0.9069 \, \frac{\text{kJ}}{\text{kg·K}}\)  

For the isobaric process:  
- \(p_4 = 8 \, \text{bar}\)  
- \(h_4 = h_2(p = 8 \, \text{bar}) = 53.42 \, \frac{\text{kJ}}{\text{kg}}\)  
- \(x_4(p = 8 \, \text{bar}) = 0.3455 \, \frac{\text{kJ}}{\text{kg}}\)  

No diagrams are present for this part.