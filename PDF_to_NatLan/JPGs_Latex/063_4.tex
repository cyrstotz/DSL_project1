TASK 2a  
The diagram is a qualitative \( T \)-\( s \) plot representing the thermodynamic process of a jet engine. The temperature \( T \) is plotted on the vertical axis (in Kelvin), and the entropy \( s \) is plotted on the horizontal axis (in \( \text{kJ}/\text{kg·K} \)). The process includes labeled states:  
- State 0 (ambient conditions)  
- State 1 (after compression)  
- State 2 (bypass/core split)  
- State 3 (combustion)  
- State 4 (turbine exit)  
- State 5 (mixer exit)  
- State 6 (nozzle exit).  

The diagram shows isobaric and adiabatic processes, with arrows indicating the flow direction. The combustion process is highlighted as a significant increase in temperature.  

TASK 2b  
Using the first law of thermodynamics:  
\[
0 = \dot{m}_{\text{gas}} \left[ h_5 - h_6 + \frac{w_5^2 - w_6^2}{2} \right]
\]  
Rearranging:  
\[
0 = h_5 - h_6 + \frac{w_5^2}{2} - \frac{w_6^2}{2}
\]  
Solving for \( w_6 \):  
\[
w_6 = \sqrt{2 \left[ h_5 - h_6 + \frac{w_5^2}{2} \right]}
\]  

The temperature \( T_6 \) is calculated using the isentropic relation:  
\[
T_6 = T_s \left( \frac{p_6}{p_s} \right)^{\frac{\kappa - 1}{\kappa}}
\]  
Substituting values:  
\[
T_s = 437.9 \, \text{K}, \quad p_6 = 0.5 \, \text{bar}, \quad p_s = 0.791 \, \text{bar}
\]  
\[
T_6 = 437.9 \, \text{K} \times \left( \frac{0.5 \, \text{bar}}{0.791 \, \text{bar}} \right)^{\frac{1.4 - 1}{1.4}} = 328 \, \text{K}
\]  

Finally, calculating \( w_6 \):  
\[
w_6 = \sqrt{2 \cdot 7.006 \, \text{kJ}/\text{kg·K} \cdot (437.9 \, \text{K} - 328 \, \text{K}) + 220^2 \, \text{m}^2/\text{s}^2}
\]  
\[
w_6 = 220.9 \, \text{m/s}
\]