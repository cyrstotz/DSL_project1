TASK 4a  
A graph is drawn with pressure \( P \) on the vertical axis and temperature \( T \) on the horizontal axis. The graph represents a pressure-temperature (p-T) diagram for the freeze-drying process. Key points and processes are labeled:  
- Point 4: "mass dampf" (mass vapor)  
- Point 3: "isobar Kondensort" (isobaric condensation)  
- Point 2: "adiabatic und reversibel (isentrop) Verdichtung" (adiabatic and reversible (isentropic) compression)  
- Point 1: "isobar vollständige Verdampfung" (isobaric complete evaporation)  
- The process transitions are described as "adiabatic expansion" and "ND Gebiet" (ND region).  

TASK 4b  
The task involves determining the mass flow rate \( \dot{m}_{\text{R134a}} \). The following equations and values are provided:  
1. Energy balance equation:  
\[
0 = \dot{m} \left[ h_2 - h_3 \right] + \dot{Q} - \dot{W}_u
\]  
2. Rearranged equation for \( \dot{m}_{\text{R134a}} \):  
\[
\dot{W}_u = \dot{m}_{\text{R134a}} \frac{\dot{W}_u}{h_2 - h_3}
\]  
3. Heat transfer equation:  
\[
\dot{W}_u = c_p \left( T_2 - T_3 \right)
\]  

Given values:  
- \( \dot{W}_u = 28 \, \text{W} \)  
- \( T_i = -10^\circ\text{C} = 263.15 \, \text{K} \)  
- \( h_3 = h_f = 264.15 \, \text{kJ/kg} \) (from Table A-11)  
- \( T_3 = 31.33^\circ\text{C} \)  

Additional notes mention that \( s_2 = s_3 \) (entropy equality) and reference the graphical representation for further calculations.  

No further numerical results or conclusions are provided.