TASK 4a  
Two diagrams are drawn to represent the freeze-drying process in a pressure-temperature (\(P\)-\(T\)) diagram.  

- The first diagram shows three distinct regions labeled "Fest" (solid), "Flüssig" (liquid), and "Gas" (gas). The phase boundaries are drawn, and three states are marked:  
  1. State 1 is located in the solid region.  
  2. State 2 is on the boundary between solid and liquid.  
  3. State 3 is on the boundary between liquid and gas.  

- The second diagram simplifies the phase regions and shows the transitions between states 1, 2, and 3 along the phase boundaries.  

TASK 4b  
The initial temperature \(T_i\) is given as \(-10^\circ\text{C}\). The evaporation temperature is calculated as:  
\[
T_{\text{verdampf}} = T_i - 6^\circ\text{C} = -16^\circ\text{C}
\]  
The pressure at state 3 is \(p_3 = 8 \, \text{bar}\), and the pressure at state 4 is \(p_4 = p_3\). The vapor quality at state 2 is \(x_2 = 1\), and at state 4 it is \(x_4 = 0\).  

The energy balance for the system is expressed as:  
\[
O = \dot{m}_{\text{R134a}} (h_3 - h_4) + \dot{Q}_{\text{K}}
\]  
Alternatively:  
\[
O = \dot{m}_{\text{R134a}} (h_2 - h_3) - \dot{W}_{\text{K}}
\]  

The mass flow rate of the refrigerant is calculated using:  
\[
\dot{m}_{\text{R134a}} = \frac{\dot{W}_{\text{K}}}{h_2 - h_3}
\]  

The enthalpy values are provided:  
\[
h_3 = h(8 \, \text{bar}) = 264.15 \, \frac{\text{kJ}}{\text{kg}}
\]  
\[
h_2 = h(-16^\circ\text{C}) = 237.74 \, \frac{\text{kJ}}{\text{kg}}
\]  

Substituting these values, the refrigerant mass flow rate is calculated as:  
\[
\dot{m}_{\text{R134a}} = 3.8167 \, \frac{\text{kg}}{\text{h}}
\]  