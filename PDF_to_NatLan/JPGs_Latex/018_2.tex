TASK 2a  
The problem involves a jet engine operating under specified conditions. Key parameters are listed:  
- \( w_5 = 220 \, \text{m/s} \)  
- \( w_0 = 200 \, \text{m/s} \)  
- \( p_0 = 0.191 \, \text{bar} \), \( T_0 = -30^\circ\text{C} \)  
- \( p_5 = 0.5 \, \text{bar} \), \( T_5 = 431.9 \, \text{K} \)  
- \( \frac{\dot{m}_M}{\dot{m}_K} = 5.293 \)  
- \( q_B = \frac{\dot{Q}_B}{\dot{m}_K} = 1195 \, \text{kJ/kg} \), \( \bar{T}_B = 1289 \, \text{K} \)  

Constants:  
- \( c_p = 1.006 \, \text{kJ/kg·K} \)  
- \( n = \kappa = 1.4 \)  

TASK 2b  
a) A qualitative \( T \)-\( s \) diagram is drawn to represent the jet engine process. The diagram includes labeled isobars and process states:  
- State 0: Ambient conditions (\( p_0, T_0 \))  
- State 2: Compression  
- State 3: Combustion  
- State 5: Mixing chamber (\( p_5, T_5 \))  
- State 6: Nozzle exit (\( p_6 = p_0 \))  

The diagram shows increasing entropy during combustion and decreasing entropy during compression and expansion. Axes are labeled as \( T \, [\text{K}] \) and \( s \, [\text{kJ/kg·K}] \).  

TASK 2b  
b) To determine \( w_6 \) and \( T_6 \):  

The temperature \( T_6 \) is calculated using the isentropic relation:  
\[
\frac{T_6}{T_5} = \left( \frac{p_0}{p_5} \right)^{\frac{n-1}{n}}
\]  
Substituting values:  
\[
T_6 = T_5 \cdot \left( \frac{p_0}{p_5} \right)^{\frac{n-1}{n}} = 431.9 \, \text{K} \cdot \left( \frac{0.191 \, \text{bar}}{0.5 \, \text{bar}} \right)^{\frac{0.4}{1.4}}
\]  
\[
T_6 = 328.075 \, \text{K}
\]  

The velocity \( w_6 \) is determined using the steady-state energy balance:  
\[
0 = \dot{m} \left( h_2 - h_0 \right) + \frac{w_2^2 - w_0^2}{2} + \sum \dot{Q} - \sum \dot{W}
\]  

This equation accounts for enthalpy change, kinetic energy, heat transfer, and work. Further calculations are required to solve for \( w_6 \).  

No additional diagrams or graphs are present.