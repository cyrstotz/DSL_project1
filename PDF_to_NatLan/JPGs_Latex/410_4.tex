TASK 3a  
Determine \( p_{g,1} \) and \( m_g \):  

The mass of the gas is calculated using the formula:  
\[
m_g = \frac{p_{g,1} V_{g,1}}{R_g T_{g,1}}
\]  
Substituting values, \( m_g = 3.422 \, \text{g} \).  

The pressure \( p_{g,1} \) is given by:  
\[
p_{g,1} = p_{\text{amb}} + \frac{m_K \cdot g}{\frac{D^2 \pi}{4}} + \frac{m_{\text{EW}} \cdot g}{\frac{D^2 \pi}{4}}
\]  
Substituting values, \( p_{g,1} = 1.401 \, \text{bar} \).  

The specific gas constant is calculated as:  
\[
R_g = \frac{R}{M_g} = 0.16628 \, \frac{\text{kJ}}{\text{kg·K}}
\]  

---

TASK 3b  
No content found.  

---

TASK 3c  
Energy balance for the gas:  

The energy change is expressed as:  
\[
\Delta E = Q_{12} - W_u
\]  

Using internal energy and heat transfer:  
\[
m(u_2 - u_1) + U_v = Q_{12}
\]  

Expanding the energy balance:  
\[
m c_V (T_{g,2} - T_{g,1}) + \frac{R (T_{g,2} - T_{g,1})}{1 - \kappa} = Q_{12}
\]  

The volume at state 2 is calculated as:  
\[
V_2 = \frac{m_g R_g T_{g,2}}{p_{g,2}}
\]  