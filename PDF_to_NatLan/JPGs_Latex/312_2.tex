TASK 2a  
The graph represents a qualitative \( T \)-\( s \) diagram for the jet engine process. The temperature \( T \) is plotted on the vertical axis (in Kelvin), and the specific entropy \( s \) is plotted on the horizontal axis (in \( \frac{\text{kJ}}{\text{kg·K}} \)).  

Key features of the diagram:  
- The process begins at state \( 0 \), with entropy increasing as the air is compressed to state \( 2 \).  
- States \( 2 \) and \( 3 \) are connected by an isobaric combustion process, where temperature increases significantly.  
- The entropy decreases from state \( 3 \) to state \( 4 \) during the turbine process.  
- States \( 5 \) and \( 6 \) represent the mixing and nozzle exit, respectively, with entropy decreasing further.  
- The isobars are labeled as \( p_2 = p_3 \) and \( p_4 = p_5 = p_6 \).  
- The ambient pressure \( p_0 = p_6 \) is indicated.  

Dashed lines represent isobaric processes.  

---

TASK 2b  
The flow process equations are provided as follows:  

1. Energy balance:  
\[
O = \dot{m} \left( h_5 - h_6 + \frac{w_5^2 - w_6^2}{2} \right) + \dot{Q} \neq \dot{W}
\]  

2. Simplified energy equation using specific heat capacity:  
\[
O = c_{p,\text{Luft}} \left( T_5 - T_6 \right) + \frac{w_5^2 - w_6^2}{2}
\]  

3. Entropy balance:  
\[
O = \dot{m} \left( s_5 - s_6 \right) + \dot{Q} \neq \dot{W} + \dot{S}_{\text{erzeugt}}
\]  

4. Entropy difference calculation:  
\[
s_5 - s_6 = s^0(T_5) - s^0(T_6) - R_{\text{Luft}} \ln \left( \frac{p_5}{p_6} \right) = 0
\]  

5. Gas constant for air:  
\[
R_{\text{Luft}} = \frac{R}{M_{\text{Luft}}} = \frac{8.314 \, \frac{\text{kJ}}{\text{kmol·K}}}{28.97 \, \frac{\text{kg}}{\text{kmol}}} = 0.287 \, \frac{\text{kJ}}{\text{kg·K}}
\]  

Annotations clarify that the process is adiabatic (\( O_{\text{adiabat}} \)) and reversible (\( O_{\text{reversibel}} \)).