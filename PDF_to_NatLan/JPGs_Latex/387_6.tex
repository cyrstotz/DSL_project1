TASK 4a  
A graph is drawn representing the freeze-drying process in a pressure-temperature (\(p\)-\(T\)) diagram. The axes are labeled as follows:  
- The vertical axis represents pressure (\(p\)), with arrows indicating increasing pressure.  
- The horizontal axis represents temperature (\(T\)).  

The graph includes curves that depict phase transitions and regions corresponding to different phases (solid, liquid, and vapor). The curves are qualitatively drawn to show the relationships between pressure and temperature during the freeze-drying process.

---

TASK 4b  
The energy balance equation for the refrigerant mass flow rate is written as:  
\[
0 = -\dot{m}_{\text{R134a}} \cdot \left( h_1 - h_4 \right) + Q_K
\]  

Values for enthalpy (\(h\)) are provided:  
\[
h_4 = 32.75 \, \text{kJ/kg}
\]  
\[
h_2 = h_g(-16^\circ\text{C}) = 237.7 \, \text{kJ/kg} \quad \text{(from Table A-10)}
\]  

Entropy values are also noted:  
\[
s_2 = s_3 = 0.9528 \, \text{kJ/kg·K}
\]  

---

TASK 4c  
The refrigerant temperature at state \(i\) is corrected:  
\[
T_i = -10^\circ\text{C}
\]  

No further explanation or calculations are provided for this subtask.  

