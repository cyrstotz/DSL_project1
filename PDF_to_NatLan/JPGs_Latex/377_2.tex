TASK 2a  
A graph is drawn representing the thermodynamic process in a jet engine on a \( T \)-\( s \) diagram. The axes are labeled as follows:  
- The vertical axis is \( T \) (temperature) in Kelvin (\( \text{K} \)).  
- The horizontal axis is \( s \) (specific entropy) in \( \frac{\text{kJ}}{\text{kg·K}} \).  

The process includes the following states and transitions:  
1. State 0: Ambient conditions (\( p_0 \), \( T_0 \)).  
2. State 1: Isentropic compression.  
3. State 3: Combustion process (isobaric).  
4. State 4: Expansion in the turbine (adiabatic).  
5. State 5: Mixing chamber.  
6. State 6: Nozzle exit (isentropic).  

Key annotations:  
- The curve from state 0 to state 1 is labeled "isentropic."  
- The curve from state 3 to state 4 is labeled "polytrope."  
- State 5 is labeled with \( p_5 = 0.5 \, \text{bar} \).  
- State 6 is labeled with \( p_6 = p_0 = 0.191 \, \text{bar} \).  

TASK 2b  
The task involves calculating the outlet velocity \( w_6 \) and temperature \( T_6 \) at the nozzle exit.  

The energy balance equation for the nozzle is written:  
\[
\dot{E} = \dot{m} \left( h_0 - h_6 + \frac{w_6^2 - w_0^2}{2} \right) + \dot{Q} - \dot{W}
\]  
Assuming steady-state operation and neglecting heat transfer (\( \dot{Q} = 0 \)) and work (\( \dot{W} = 0 \)), the equation simplifies to:  
\[
h_0 - h_6 + \frac{w_6^2 - w_0^2}{2} = 0
\]  
Rearranging:  
\[
w_6^2 = 2(h_0 - h_6) + w_0^2
\]  

Using the ideal gas model:  
\[
h_0 - h_6 = c_p (T_0 - T_6)
\]  

For the transition from state 5 to state 6:  
- Adiabatic and reversible (\( S_{5} = S_{6} \)).  
- Isentropic relation:  
\[
\frac{T_6}{T_5} = \left( \frac{p_6}{p_5} \right)^{\frac{\kappa - 1}{\kappa}}
\]  
where \( \kappa = 1.4 \).  

Substituting values:  
\[
T_6 = T_5 \left( \frac{p_6}{p_5} \right)^{\frac{\kappa - 1}{\kappa}}
\]  
\[
T_6 = 431.9 \, \text{K} \left( \frac{0.191}{0.5} \right)^{\frac{1.4 - 1}{1.4}}
\]  
\[
T_6 = 328.094 \, \text{K}
\]  

Finally, calculating \( w_6 \):  
\[
w_6 = \sqrt{2 c_p (T_0 - T_6) + w_0^2}
\]  
Substituting \( c_p = 1.006 \, \frac{\text{kJ}}{\text{kg·K}} \), \( T_0 = -30^\circ\text{C} + 273.15 = 243.15 \, \text{K} \), \( T_6 = 328.094 \, \text{K} \), and \( w_0 = 200 \, \frac{\text{m}}{\text{s}} \):  
\[
w_6 = \sqrt{2 \cdot 1.006 \cdot (243.15 - 328.094) + 200^2}
\]  
\[
w_6 = \sqrt{453,202 \, \frac{\text{m}^2}{\text{s}^2}}
\]  
\[
w_6 = 453.202 \, \frac{\text{m}}{\text{s}}
\]  

The outlet velocity at the nozzle exit is \( w_6 = 453.202 \, \frac{\text{m}}{\text{s}} \).