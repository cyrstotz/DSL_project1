TASK 3a  
The ideal gas law is used to calculate pressure and mass. The equation is:  
\[
pV = m \frac{R}{M} T
\]  
where \( R = \frac{8.314 \, \text{kJ/(kmol·K)}}{50 \, \text{kg/kmol}} = 0.16628 \, \text{kJ/(kg·K)} \).  

The specific gas constant is calculated as:  
\[
R = 0.16628 \, \text{kJ/(kg·K)} = 0.16628 \, \text{kNm/(kg·K)}
\]  

The area of the piston is determined using:  
\[
A = \pi r^2 = \pi (5 \times 10^{-2} \, \text{m})^2 = 7.853981 \times 10^{-3} \, \text{m}^2
\]  

The pressure exerted by the piston is calculated as:  
\[
p_{s,1} = \frac{F}{A} + p_{\text{amb}} = \frac{32 \, \text{kg} \cdot 9.81 \, \text{m/s}^2}{7.853981 \times 10^{-3} \, \text{m}^2} + 100,000 \, \text{N/m}^2
\]  
\[
= 139,969.538 \, \text{N/m}^2 = 1.3967 \, \text{bar}
\]  

The mass of the gas is calculated using:  
\[
m_g = \frac{pV}{RT} = \frac{(139,969.538 \, \text{Pa})(3.14 \times 10^{-3} \, \text{m}^3)}{(0.16628 \, \text{kJ/(kg·K)})(500 + 273.15) \, \text{K}}
\]  
\[
= 3.418687423 \, \text{kg}
\]  
\[
m_g = 3.41879 \, \text{kg}
\]  

---

TASK 3b  
The mass of the gas is confirmed as:  
\[
m_g = 3.41879 \, \text{kg}
\]  

---

TASK 3c  
Using \( T_{g,2} = 0.003^\circ \text{C} \), the energy balance is applied:  
\[
\frac{dE}{dt} = \dot{Q}_{12} - \dot{W} \implies \Delta E = Q_{12} - W
\]  

The formula for heat transfer is:  
\[
Q_{12} = m_g c_V \Delta T
\]  

Substituting values:  
\[
Q_{12} = (3.41879 \times 10^3 \, \text{kg}) \cdot (0.633 \, \text{kJ/(kg·K)}) \cdot (500 - 0.003) \, \text{K}
\]  
\[
= 1.08200 \, \text{kJ} = 1082 \, \text{J}
\]  

The heat transferred is:  
\[
Q_{12} = 1082 \, \text{J}
\]  

---

No diagrams or graphs are present on the page.