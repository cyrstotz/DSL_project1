TASK 1a  
The graph depicts a pressure-temperature (\(P\)-\(T\)) diagram. The curve represents the phase boundary between the liquid and gaseous states. Key points labeled on the graph include:  
- "fuel" near the liquid region,  
- "gas" near the gaseous region,  
- "T_{mix}" marked along the curve, indicating a mixing temperature.  

The horizontal axis is labeled \(T\) (temperature), and the vertical axis is labeled \(P\) (pressure).  

TASK 1b  
The equation provided is:  
\[
0 = \dot{m}_2 \cdot m_c \cdot (h_2 - h_5) - \dot{W}_K
\]  
This represents an energy balance equation, where:  
- \(\dot{m}_2\) is the mass flow rate,  
- \(m_c\) is a constant or specific mass,  
- \(h_2\) and \(h_5\) are specific enthalpies at states 2 and 5, respectively,  
- \(\dot{W}_K\) is the work rate of the compressor.