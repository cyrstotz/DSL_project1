TASK 4a  
A pressure-temperature (\(p\)-\(T\)) diagram is drawn, showing the freeze-drying process. The diagram includes labeled points corresponding to states \(1\), \(2\), \(3\), and \(4\).  
- The curve represents the phase boundaries, with the region labeled "Nassdampfgebiet" (wet steam region).  
- Isobaric processes are indicated between states \(1\) and \(2\), and between states \(3\) and \(4\).  
- The transition from state \(2\) to \(3\) is depicted as a steep line, likely representing compression.  
- The \(T\)-axis is horizontal, and the \(p\)-axis is vertical.  

TASK 4b  
The energy balance equation is written as:  
\[
0 = \dot{m} (h_c - h_a) + Q - W
\]  
where \( \dot{m} \) is the mass flow rate, \( h_c \) and \( h_a \) are specific enthalpies, \( Q \) is heat transfer, and \( W \) is work.  

TASK 4c  
The vapor quality \(x\) is defined as:  
\[
x = \frac{m_g}{m_g + m_f}
\]  
where \( m_g \) is the mass of the gas phase and \( m_f \) is the mass of the liquid phase.  

TASK 4d  
The coefficient of performance (\( \epsilon_K \)) is given as:  
\[
\epsilon_K = \frac{| \dot{Q}_2 |}{| W |} = \frac{| \dot{Q}_2 |}{| \dot{Q}_4 - \dot{Q}_2 |}
\]  

TASK 4e  
No content is provided for this subtask.