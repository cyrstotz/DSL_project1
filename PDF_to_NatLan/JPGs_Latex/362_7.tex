TASK 4a  
A pressure-temperature (\(p\)-\(T\)) diagram is drawn to represent the freeze-drying process. The diagram includes the following features:  
- The curve shows the phase boundaries between liquid, vapor, and saturated states.  
- State 1 is labeled as "liquid, \(x = 0\)" and lies on the saturated liquid line.  
- State 2 is labeled as "vapor" and lies on the saturated vapor line.  
- State 3 is located above the saturated vapor line, indicating a superheated vapor region.  
- Arrows indicate the transitions between states (1 → 2 → 3).  

TASK 4b  
An energy balance is applied to the compressor for the transition from state 2 to state 3:  
\[
0 = \dot{m} \left( h_{\text{out}} - h_{\text{in}} \right) - W_{\text{c}}
\]  
Rearranging for the compressor work:  
\[
W_{\text{c}} = \dot{m} \left( h_{\text{out}} - h_{\text{in}} \right)
\]  

The following temperature values are provided:  
\[
T_2 = T_i + 20 \, \text{K}
\]  
\[
T_1 = T_i - 6 \, \text{K}
\]  

From the tables:  
\[
T_i = -10^\circ\text{C} \quad \text{(as per the table)}
\]  
\[
T_2 = -10^\circ\text{C} + 20 \, \text{K} = 10^\circ\text{C}
\]  
\[
T_1 = -10^\circ\text{C} - 6 \, \text{K} = -16^\circ\text{C}
\]  

At state 2, the vapor quality is \(x = 1\), indicating saturated vapor.  

From the table (A-10):  
\[
h_2 = h_{\text{out}} = 237.74 \, \text{kJ/kg}
\]  

No further calculations are visible.