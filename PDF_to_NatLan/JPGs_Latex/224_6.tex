TASK 4a  
The diagram is a pressure-temperature (\(p\)-\(T\)) plot illustrating the freeze-drying process.  
- The phase regions are labeled as "fest" (solid), "flüssig" (liquid), and "gasförmig" (gaseous).  
- The triple point is marked, and the chamber pressure is reduced to 5 mbar below the triple point.  
- The process steps are labeled as 1, 2, 3, and 4, with arrows indicating transitions between states.  
- \(T_i\) is shown as a horizontal line above the sublimation temperature.  

TASK 4b  
The entropy at state 2 is equal to the entropy at state 3:  
\[
s_2 = s_3
\]  
The required refrigerant mass flow rate is calculated using the energy balance:  
\[
\dot{m}_R = \frac{\dot{Q}_K}{h_2 - h_3} + \dot{W}_K
\]  

TASK 4d  
The coefficient of performance (\(\epsilon_K\)) is calculated as:  
\[
\epsilon_K = \frac{\dot{Q}_K}{\dot{W}_K} = \frac{\dot{Q}_K}{\dot{Q}_{30} - \dot{Q}_{20}}
\]  
Given:  
\[
h_4 = 0.342 \, \text{kJ/kg}, \quad h_3 = 260.15 \, \text{kJ/kg} \, (8 \, \text{bar})
\]  
The heat transfer rate is:  
\[
\dot{Q}_K = \dot{m} \cdot (h_2 - h_1)
\]  
The result for \(\dot{Q}_{20}\) is:  
\[
\dot{Q}_{20} = 0.1897 \, \text{kW}
\]  

No additional content is visible.