TASK 3a  
The universal gas constant \( R \) is calculated as:  
\[
R = \frac{\bar{R}}{M} = \frac{8.314 \, \text{kJ/(kmol·K)}}{50 \, \text{kg/kmol}} = 0.16628 \, \text{kJ/(kg·K)}
\]

The gas pressure \( p_{\text{gas}} \) is determined using the formula:  
\[
p_{\text{gas}} = p_{\text{amb}} + \frac{F_L}{A} + \frac{F_{\text{EW}}}{A}
\]
where \( F_L = m_K \cdot g \) and \( F_{\text{EW}} = m_{\text{EW}} \cdot g \). Substituting values:  
\[
p_{\text{gas}} = p_{\text{amb}} + \frac{32 \, \text{kg} \cdot 9.81 \, \text{m/s}^2}{\left(\frac{\pi}{4} \cdot (0.1 \, \text{m})^2\right)} + \frac{0.1 \, \text{kg} \cdot 9.81 \, \text{m/s}^2}{\left(\frac{\pi}{4} \cdot (0.1 \, \text{m})^2\right)}
\]
\[
= 1 \cdot 10^5 \, \text{N/m}^2 + 1.400 \cdot 10^5 \, \text{N/m}^2 = 1.4 \, \text{bar}
\]

The gas mass \( m_g \) is calculated using the ideal gas law:  
\[
p \cdot V = n \cdot R \cdot T
\]
\[
n = \frac{p \cdot V}{R \cdot T}
\]
\[
m_g = n \cdot M = \frac{p_{\text{gas}} \cdot V_{g,1}}{R \cdot T_{g,1}} \cdot M
\]
Substituting values:  
\[
m_g = \frac{1.4 \cdot 10^5 \, \text{N/m}^2 \cdot 3.14 \cdot 10^{-3} \, \text{m}^3}{0.16628 \, \text{kJ/(kg·K)} \cdot 773.15 \, \text{K}} \cdot 50 \, \text{kg/kmol}
\]
\[
= 3.419 \cdot 10^{-3} \, \text{kg} = 3.42 \, \text{g}
\]

---

TASK 3b  
The temperature \( T_{g,2} \) is given as \( 0^\circ\text{C} \), because the ice cannot melt further at \( 0^\circ\text{C} \).  

The gas pressure \( p_{g,2} \) is equal to the ambient pressure \( p_{\text{amb}} \):  
\[
p_{g,2} = p_{\text{amb}} = 1.4 \, \text{bar}
\]
(This pressure was calculated in part (a).)  

Crossed-out content is ignored.