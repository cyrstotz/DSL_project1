TASK 2a  
The diagram is a qualitative \( T \)-\( s \) diagram representing the jet engine process. It includes labeled isobars \( p_0 \), \( p_1 \), \( p_3 \), \( p_5 \), and \( p_6 \). The process starts at state \( 0 \), moves through states \( 1 \), \( 2 \), \( 3 \), \( 4 \), \( 5 \), and ends at state \( 6 \). The curve shows the thermodynamic transitions, with arrows indicating the flow direction. The axes are labeled \( T \) (temperature) on the vertical axis and \( s \) (entropy) on the horizontal axis.

---

TASK 2b  
The outlet velocity \( w_6 \) is given as:  
\[
w_6 = 220 \, \text{m/s}
\]  
The pressure \( p_5 \) is given as:  
\[
p_5 = 6.5 \, \text{bar}
\]  

The mass flow rate \( \dot{m} \) is calculated using the equation:  
\[
\dot{m} = \frac{p_0 v_0}{R T_0}
\]  
where \( v_0 = v_6 \), \( p_0 = p_6 \), and \( T_0 \) is the ambient temperature.  

Since the nozzle is adiabatic and reversible, the temperature \( T_6 \) is calculated using:  
\[
T_6 = T_0 \left( \frac{p_6}{p_0} \right)^{\frac{n-1}{n}}
\]  
Substituting values:  
\[
T_6 = 328.07 \, \text{K}
\]  

For an adiabatic nozzle, the heat transfer \( Q \) is zero:  
\[
Q = 0
\]  

The energy balance equation is:  
\[
0 = \dot{m} (h_e - h_a) + \frac{\dot{m}}{2} (w_6^2 - w_5^2)
\]  

Rearranging and solving for \( w_6^2 \):  
\[
\frac{w_6^2}{2} = c_p (T_5 - T_6) + \frac{w_5^2}{2}
\]  

Finally, the outlet velocity \( w_6 \) is calculated as:  
\[
w_6 = \sqrt{2 c_p (T_5 - T_6) + w_5^2} = 506.02 \, \text{m/s}
\]