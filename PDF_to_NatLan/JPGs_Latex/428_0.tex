TASK 1a  
The system is steady-state.  
The mass flow rate is given as:  
\[
\dot{m}_{\text{in}} = 0.3 \, \text{kg/s}.
\]  

The inlet temperature \( T_{\text{in}} \) and outlet temperature \( T_{\text{out}} \) are both \( 70^\circ\text{C} \). Both states are saturated liquid (pure water assumed).  

From Table A-2:  
\[
h(T_{\text{in}}) = 259.88 \, \text{kJ/kg}, \quad h(T_{\text{out}}) = 419.04 \, \text{kJ/kg}.
\]  

Using the energy balance:  
\[
0 = \dot{m} (h_{\text{in}} - h_{\text{out}}) + \dot{Q}_R - \dot{Q}_{\text{out}},
\]  
where \( \dot{Q}_R = 100 \, \text{kW} \). Rearranging:  
\[
\dot{Q}_{\text{out}} = \dot{m} (h_{\text{in}} - h_{\text{out}}) + \dot{Q}_R.
\]  

Substituting values:  
\[
\dot{Q}_{\text{out}} = (0.3 \, \text{kg/s}) \cdot (259.88 \, \text{kJ/kg} - 419.04 \, \text{kJ/kg}) + 100 \, \text{kW}.
\]  
\[
\dot{Q}_{\text{out}} = -62.182 \, \text{kW}.
\]  

TASK 1b  
The thermodynamic mean temperature \( T_{\text{KF}} \) of the coolant is calculated assuming ideal liquid behavior with constant heat capacity \( c_p(T) = c_p \).  

The formula for \( T_{\text{KF}} \) is:  
\[
T_{\text{KF}} = \frac{\int_{T_{\text{KF,in}}}^{T_{\text{KF,out}}} T \, c_p(T) \, dT}{\int_{T_{\text{KF,in}}}^{T_{\text{KF,out}}} c_p(T) \, dT}.
\]  

Assuming \( c_p(T) \) is constant:  
\[
T_{\text{KF}} = \frac{T_{\text{KF,in}} + T_{\text{KF,out}}}{2}.
\]  

Substituting values:  
\[
T_{\text{KF}} = \frac{288.15 \, \text{K} + 298.15 \, \text{K}}{2}.
\]  
\[
T_{\text{KF}} = 293.15 \, \text{K}.
\]