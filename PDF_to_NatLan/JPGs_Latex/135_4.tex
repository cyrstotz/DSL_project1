TASK 3a  
The gas pressure \( p_{g,1} \) is calculated using the formula:  
\[
p_{g,1} = \frac{m_K \cdot g + m_{\text{EW}} \cdot g + p_{\text{amb}} \cdot A}{A}
\]  
Substituting the values:  
\[
p_{g,1} = \frac{32 \, \text{kg} \cdot 9.81 \, \text{m/s}^2 + 0.1 \, \text{kg} \cdot 9.81 \, \text{m/s}^2 + 1.05 \, \text{N/m}^2}{\pi \cdot (0.1 \, \text{m})^2}
\]  
This results in:  
\[
p_{g,1} = 1.40 \, \text{bar}
\]  

The mass \( m_g \) of the gas is determined using the ideal gas law:  
\[
pV = mRT \implies m = \frac{pV}{RT} \cdot M_g
\]  
Substituting the values:  
\[
m = \frac{7.4 \cdot 10^4 \, \text{Pa} \cdot 3.14 \cdot 10^{-3} \, \text{m}^3}{8.31 \, \frac{\text{kJ}}{\text{kmol·K}} \cdot 773.15 \, \text{K}} \cdot 50 \, \frac{\text{kg}}{\text{kmol}}
\]  
This results in:  
\[
m = 3.66 \, \text{g}
\]  

---

TASK 3b  
The pressure remains constant at \( p_{g,2} = 1.40 \, \text{bar} \).  

This can be explained using the formula for \( p_{g,1} \), as none of the quantities used in the calculation change. Therefore, the pressure does not vary.