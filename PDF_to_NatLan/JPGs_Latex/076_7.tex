TASK 4a  
A graph is drawn with axes labeled \( p \) (pressure) on the vertical axis and \( T \) (temperature) on the horizontal axis. The graph shows three regions: "Gasförmig" (gaseous), "Fest" (solid), and a curve labeled "Triple point." The curve separates the solid and gaseous regions.  

---

TASK 4b  
The mass flow rate of the refrigerant (\( \dot{m}_{\text{R134a}} \)) is being calculated.  

The vapor quality at state 2 is given as \( x_2 = 1 \).  

The process from state 2 to state 3 is described as reversible (\( s_2 = s_3 \)) and isentropic.  

The temperature at state 2 is calculated as:  
\[
T_2 = T_i - 6 \, \text{K}
\]  

The pressure at state 2 is calculated using:  
\[
p_i = 1 \, \text{mbar} = 10^{-3} \, \text{bar} = 100 \, \text{Pa} = 0.1 \, \text{kPa}
\]  
From Table A-6, the temperature corresponding to \( p = 0.1 \, \text{kPa} \) is:  
\[
T_{i,p} = -20.385^\circ \text{C}
\]  
Thus:  
\[
T_2 = -20.385 - 6 = -26.385^\circ \text{C}
\]  

The enthalpy at state 2 is:  
\[
h_2 = h_g(-26.385^\circ \text{C})
\]  

The mass flow rate is calculated using:  
\[
\dot{m}_{\text{R134a}} = \frac{\dot{Q}_K}{h_2 - h_3}
\]  

The entropy at state 2 is:  
\[
s_2 = s_3 = s(-26.385^\circ \text{C})
\]  

The vapor quality at state 3 is calculated as:  
\[
x_3 = \frac{s_3 - s_f(8 \, \text{bar})}{s_g(8 \, \text{bar}) - s_f(8 \, \text{bar})}
\]  
From Table A-11:  
\[
s_f(8 \, \text{bar}) = 0.345 \, \frac{\text{kJ}}{\text{kg·K}}, \quad s_g(8 \, \text{bar}) = 0.906 \, \frac{\text{kJ}}{\text{kg·K}}
\]  

The enthalpy at state 3 is calculated as:  
\[
h_3 = h_f(8 \, \text{bar}) + x_3 \cdot \left(h_g(8 \, \text{bar}) - h_f(8 \, \text{bar})\right)
\]  
From Table A-11:  
\[
h_f(8 \, \text{bar}) = 93.42 \, \frac{\text{kJ}}{\text{kg}}, \quad h_g(8 \, \text{bar}) = 264.45 \, \frac{\text{kJ}}{\text{kg}}
\]  

---  
No additional diagrams or figures are described beyond the graph in TASK 4a.