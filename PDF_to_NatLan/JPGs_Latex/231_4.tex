TASK 3a  
The pressure \( p_{g,1} \) and mass \( m_g \) of the gas in state 1 are calculated as follows:  

The cross-sectional area \( A \) of the cylinder is determined using the formula for the area of a circle:  
\[
A = \pi r^2 = 0.031415 \, \text{m}^2
\]  

The pressure \( p \) is calculated using the force \( F \) and area \( A \):  
\[
p = \frac{F}{A}
\]  
\[
p_{g,1} = \frac{(0.1 \, \text{kg} + 32 \, \text{kg}) \cdot g}{A} + 1 \, \text{bar}
\]  
Substituting values:  
\[
p_{g,1} = \frac{0.031415 \, \text{m}^2 \cdot (32 \, \text{kg} + 0.1 \, \text{kg}) \cdot 9.81 \, \text{m/s}^2}{1} + 10^5 \, \text{Pa}
\]  
\[
p_{g,1} = 100003.893 \, \text{Pa}
\]  

The molar mass of the gas is given as \( M_g = 50 \, \text{kg/kmol} \). The volume \( V \) is converted to cubic meters:  
\[
V = 3.14 \, \text{L} = 3.14 \cdot 10^{-3} \, \text{m}^3
\]  

The temperature \( T \) is converted to Kelvin:  
\[
T = 500^\circ\text{C} = 773.15 \, \text{K}
\]  

Using the ideal gas law:  
\[
p_{g,1} \cdot V_{g,1} = \frac{M}{M_g} \cdot R \cdot T
\]  
Rearranging to solve for \( M \):  
\[
M = \frac{p_{g,1} \cdot V_{g,1} \cdot M_g}{R \cdot T}
\]  
Substituting values:  
\[
M = \frac{100003.893 \, \text{Pa} \cdot 3.14 \cdot 10^{-3} \, \text{m}^3 \cdot 50 \, \text{kg/kmol}}{8.314 \, \text{J/mol·K} \cdot 773.15 \, \text{K}}
\]  
\[
M = 2.44 \, \text{g}
\]  

TASK 3b  
The ideal gas law is stated again:  
\[
pV = \frac{M}{M_g} \cdot R \cdot T
\]