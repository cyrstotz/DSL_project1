TASK 3a  
The cross-sectional area of the cylinder is calculated as:  
\[
A = \pi \left( \frac{D}{2} \right)^2 = 25 \pi \, \text{cm}^2 = 25 \pi \times 10^{-4} \, \text{m}^2
\]  

The pressure exerted by the piston and EW mixture is given by:  
\[
p_{\text{oben}} = \frac{(m_K + m_{\text{EW}}) g}{A} + p_{\text{amb}}
\]  
Substituting values:  
\[
p_{\text{oben}} = \frac{(32 \, \text{kg} + 0.1 \, \text{kg}) \cdot 9.81 \, \text{N/kg}}{25 \pi \times 10^{-4} \, \text{m}^2} + 1 \times 10^5 \, \text{Pa} \approx 140 \, \text{kPa} = 1.4 \, \text{bar}
\]  

Mechanical equilibrium is established:  
\[
p_{g,1} = p_{\text{oben}} \approx 1.4 \, \text{bar}
\]  

The specific gas constant is calculated as:  
\[
R_g = \frac{R}{M_g} = \frac{8.314 \, \text{J/mol·K}}{50 \times 10^{-3} \, \text{kg/mol}} = 166.28 \, \text{J/kg·K}
\]  

The mass of the gas is determined using the ideal gas law:  
\[
m_g = \frac{p_{g,1} V_{g,1}}{R_g T_{g,1}} = \frac{1.4 \times 10^5 \, \text{Pa} \cdot 3.14 \times 10^{-3} \, \text{m}^3}{166.28 \, \text{J/kg·K} \cdot 773.15 \, \text{K}} \approx 0.0034 \, \text{kg} = 3.4 \, \text{g}
\]  

---

TASK 3b  
The temperature of the gas in state 2 is:  
\[
T_{g,2} = 0^\circ \text{C} = 273.15 \, \text{K}
\]  

Since \( x_{\text{ice},2} > 0 \), the temperature of the EW in state 2 is \( 0^\circ \text{C} \), as it is at equilibrium (end state). Therefore, the temperature of the gas equals that of the EW, which is \( 0^\circ \text{C} \).  

The pressure remains constant:  
\[
p_{g,2} = p_{\text{EW},2} = p_{\text{oben}} = 1.4 \, \text{bar}
\]  

As the mass above remains unchanged, the pressure must remain constant to maintain equilibrium, meaning the gas "supports" the mass in equilibrium.

---

TASK 3c  
The specific heat capacity of the gas is calculated as:  
\[
C_p = R_g + C_v = 166.28 \, \text{J/kg·K} + 633 \, \text{J/kg·K} = 799.28 \, \text{J/kg·K}
\]  

The change in internal energy is:  
\[
\Delta U_g = C_p m_g \Delta T = 799.28 \, \text{J/kg·K} \cdot 0.0034 \, \text{kg} \cdot (273.15 \, \text{K} - 773.15 \, \text{K}) \approx -1358.8 \, \text{J}
\]  

The final gas volume is calculated as:  
\[
V_{g,2} = \frac{m_g R_g T_{g,2}}{p_{g,2}} = \frac{0.0034 \, \text{kg} \cdot 166.28 \, \text{J/kg·K} \cdot 273.15 \, \text{K}}{1.4 \times 10^5 \, \text{Pa}} \approx 1.103 \, \text{L}
\]  

The change in height of the piston is:  
\[
\Delta h = \frac{\Delta V}{A} = \frac{(1.103 \, \text{L} - 3.14 \, \text{L}) \cdot 1 \times 10^{-3} \, \text{m}^3/\text{L}}{25 \pi \times 10^{-4} \, \text{m}^2} \approx -0.26 \, \text{m}
\]  

The work done by the gas is:  
\[
W = (m_K + m_{\text{EW}}) g \Delta h = (32 \, \text{kg} + 0.1 \, \text{kg}) \cdot 9.81 \, \text{N/kg} \cdot (-0.26 \, \text{m}) \approx -81.9 \, \text{J}
\]  

The total heat transferred is:  
\[
Q_{12} = \Delta U_g + W = -1358.8 \, \text{J} - 81.9 \, \text{J} = -1440.7 \, \text{J}
\]  

The magnitude of the heat transfer is:  
\[
|Q_{12}| = 1440.7 \, \text{J}
\]  