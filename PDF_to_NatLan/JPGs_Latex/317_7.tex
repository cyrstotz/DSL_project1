TASK 4a  
A pressure-temperature (\( p \)-\( T \)) diagram is drawn.  
- The diagram shows phase regions labeled as "gas," "liquid," and "solid."  
- The triple point is marked where the three phases coexist.  
- Isobars are drawn, with one labeled as "isobar" and another labeled as "isotherm."  
- States 1, 3, and 4 are marked on the diagram, connected by lines indicating transitions.  
- The diagram includes arrows pointing to the "solid" and "liquid" regions, with annotations "fest" (solid) and "flüssig" (liquid).  

TASK 4b  
Energy balance at the compressor is written as:  
\[
\dot{Q} = \dot{m}_{\text{R134a}} (h_2 - h_3) - \dot{W}_K
\]

The following assumptions and calculations are noted:  
- The process is adiabatic and reversible, implying isentropic behavior (\( s_2 = s_3 \)).  
- The temperature at state 2 is calculated as \( T_2 = T_i - 5 \, \text{K} = -22^\circ\text{C} \).  

From Table A-10:  
- \( h_2(T = -22^\circ\text{C}, \text{saturated}) \) is interpolated, yielding \( h_g = 239.08 \, \text{kJ/kg} \).  
- \( s_2 = s_g = 0.9351 \, \text{kJ/(kg·K)} \).  

The note mentions that \( h_3 \) should be interpolated from Table A-11.  

No further calculations or results are visible.