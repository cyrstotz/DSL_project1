TASK 4a  
A pressure-temperature (\( p \)-\( T \)) diagram is drawn, showing the freeze-drying process. The diagram includes four states labeled \( 1 \), \( 2 \), \( 3 \), and \( 4 \). The process transitions are described as follows:  
- \( 1 \to 2 \): Isobaric process.  
- \( 2 \to 3 \): Adiabatic process with isentropic compression.  
- \( 3 \to 4 \): Isobaric process.  
- \( 4 \to 1 \): Isentropic process.  

The axes are labeled:  
- \( p \) [bar] on the vertical axis.  
- \( T \) [K] on the horizontal axis.  

TASK 4b  
The mass flow rate of the refrigerant (\( \dot{m}_{\text{R134a}} \)) is calculated using the energy balance equation:  
\[
0 = \dot{m}_{\text{R134a}} (h_3 - h_2) + \dot{Q}_{\text{ab}}
\]  
Rearranging the equation:  
\[
0 = \dot{m}_{\text{R134a}} (h_2 - h_3) - \dot{W}_K \implies \dot{m}_{\text{R134a}} = \frac{\dot{W}_K}{h_2 - h_3}
\]  

Additional assumptions:  
- \( h_2 = h_g \), \( s_2 = s_3 \).  
- \( h_3 \) is determined at \( 8 \, \text{bar} \) and \( T_3 \).  

A crossed-out term (\( \dot{Q}_K \)) is visible but ignored in the calculations.