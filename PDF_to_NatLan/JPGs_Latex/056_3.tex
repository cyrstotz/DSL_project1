TASK 3a  
The pressure of the gas (\( p_{\text{gas}} \)) is calculated as:  
\[
p_{\text{gas}} = p_{\text{amb}} + \frac{m_K g}{A}
\]  
where:  
- \( p_{\text{amb}} = 1 \, \text{bar} \) (ambient pressure),  
- \( m_K = 32 \, \text{kg} \) (mass of the piston),  
- \( g \) is the gravitational acceleration,  
- \( A = \frac{D^2}{4} \pi \) is the cross-sectional area of the cylinder, with \( D = 10 \, \text{cm} \).  

Substituting values:  
\[
p_{\text{gas}} = 1.399 \, \text{bar} \approx 1.4 \, \text{bar}.
\]  

TASK 3b  
For the perfect gas, the following relationships are used:  
\[
p_{\text{gas}} V_g = m_{\text{gas}} R T_g
\]  
Rearranging to find the gas mass:  
\[
m_{\text{gas}} = \frac{p_{\text{gas}} \cdot V_g}{R \cdot T_g}.
\]  

Given:  
- \( V_g = 3.14 \, \text{L} \),  
- \( R = \frac{R_u}{M} = \frac{8314}{50} = 166.28 \, \text{J/(kg·K)} \),  
- \( T_g = 500^\circ\text{C} = 773.15 \, \text{K} \).  

Substituting values:  
\[
m_{\text{gas}} = \frac{1.4 \cdot 3.14}{166.28 \cdot 773.15} = 3.42 \, \text{g}.
\]  

TASK 3c  
Given \( x_{\text{ice},2} > 0 \):  
The temperature of the gas decreases as heat flows into the ice-water mixture (EW). However, the pressure remains constant because equilibrium depends on the external pressure.  

The equilibrium point shifts to a new state.  

TASK 3d  
For the closed system, the energy balance is expressed as:  
\[
\Delta E = \Sigma Q - \Sigma W
\]  
and:  
\[
\Delta U = \Sigma Q.
\]  

The heat transfer is given by:  
\[
m_{\text{EW}} (u_2 - u_1) = Q.
\]  

Using the enthalpy values from the table:  
\[
u_1 = u_{\text{flüssig}} + x_1 (u_{\text{fest}} - u_{\text{flüssig}}),
\]  
where:  
- \( u_{\text{flüssig}} = -200.0928 \, \text{kJ/kg} \),  
- \( u_{\text{fest}} = -333.458 \, \text{kJ/kg} \),  
- \( x_1 = 0.6 \).  

Substituting:  
\[
u_1 = -200.0928 + 0.6 (-333.458 - (-200.0928)) = -200.0928 - 0.6 \cdot (-133.3652).
\]  

For \( u_2 \):  
\[
u_2 = u_{\text{flüssig}} + x_2 (u_{\text{fest}} - u_{\text{flüssig}}).
\]  

This calculation continues for \( x_2 \) and \( T_2 \).  

The solution references "zu Aufgabe (3d)" for further calculations.  

No diagrams or figures are present on this page.