TASK 2a  
The process is described qualitatively in a temperature-entropy (\( T \)-\( s \)) diagram. The diagram includes the following labeled states and processes:  
- State 0: Ambient conditions.  
- State 1: Adiabatic compression.  
- State 2: Isobaric heating.  
- State 3: Adiabatic expansion.  
- State 4: Isobaric cooling.  
- State 5: Mixing chamber.  
- State 6: Nozzle exit.  

The diagram shows the temperature (\( T \)) on the vertical axis and entropy (\( s \)) on the horizontal axis. Isobars are drawn as horizontal lines, and the processes are labeled as adiabatic or isobaric transitions. Key points include:  
- \( p_1 = p_5 = 0.5 \, \text{bar} \).  
- \( p_0 = 0.191 \, \text{bar} \).  

TASK 2b  
The outlet velocity (\( w_6 \)) and temperature (\( T_6 \)) are calculated.  

Given:  
\[
T_5 = 431.9 \, \text{K}, \, p_5 = 0.5 \, \text{bar}, \, w_5 = 220 \, \text{m/s}, \, p_6 = p_0 = 0.191 \, \text{bar}.
\]  

The polytropic relation is used to find \( T_6 \):  
\[
n = \kappa = 1.4, \quad \frac{T_2}{T_1} = \left( \frac{p_2}{p_1} \right)^{\frac{n-1}{n}}.
\]  
\[
\frac{T_6}{T_5} = \left( \frac{p_6}{p_5} \right)^{\frac{n-1}{n}} \quad \Rightarrow \quad T_6 = T_5 \cdot \left( \frac{p_6}{p_5} \right)^{\frac{n-1}{n}}.
\]  
Substituting values:  
\[
T_6 = 431.9 \cdot \left( \frac{0.191}{0.5} \right)^{\frac{1.4-1}{1.4}} = 328.07 \, \text{K}.
\]  

The energy balance is applied to find \( w_6 \):  
\[
0 = \dot{m} \left[ h_6 - h_5 + \frac{w_6^2}{2} - \frac{w_5^2}{2} \right].
\]  
Rearranging:  
\[
w_6 = \sqrt{w_5^2 + 2 \cdot (h_5 - h_6)}.
\]  

Using enthalpy values from tables:  
\[
h_5 = A-21(431.9 \, \text{K}) = 433.64 \, \frac{\text{kJ}}{\text{kg}}, \quad h_6 = A-21(328.07 \, \text{K}) = 328.42 \, \frac{\text{kJ}}{\text{kg}}.
\]  
Substituting:  
\[
w_6 = \sqrt{220^2 + 2 \cdot (433.64 - 328.42)}.
\]  

Final calculation for \( w_6 \) is left incomplete on the page.  

TASK 2a Diagram Description  
The diagram visually represents the thermodynamic processes in the jet engine cycle. It includes labeled states (0 through 6) and transitions between them. Isobars are drawn horizontally, and the processes are marked as adiabatic or isobaric. The diagram highlights the mixing chamber and nozzle exit conditions.