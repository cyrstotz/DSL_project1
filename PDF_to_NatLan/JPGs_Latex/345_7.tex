TASK 4a  
A p-T diagram is drawn to represent the freeze-drying process. The diagram includes labeled phase regions and process steps.  
- The vertical axis represents pressure (\( p \)), and the horizontal axis represents temperature (\( T \)).  
- The diagram shows three states labeled as 1, 2, and 3, connected by arrows to indicate transitions.  
- State 1 is at low pressure and temperature, while state 3 is at higher pressure and temperature.  
- The transitions between states are marked as isobaric and adiabatic processes.  

TASK 4b  
The equation for work done by the compressor is written as:  
\[
W_K = \dot{m} \cdot (h_2 - h_3)
\]  
where \( \dot{m} \) is the mass flow rate, \( h_2 \) and \( h_3 \) are specific enthalpies at states 2 and 3, respectively.  

Given data:  
- \( p_1 = 1 \, \text{mbar} \), \( T_i = -10^\circ\text{C} \)  
- \( T_2 = -22^\circ\text{C} \), \( h_2 = 234.08 \, \text{kJ/kg} \), \( s_2 = 0.9 \)  
- \( h_3 = 284.75 + (273.15 - 268.15) \cdot 1.006 = 277.95 \, \text{kJ/kg} \)  

Using Table A-10:  
- \( s_2 = 0.9066 \)  
- \( h_3 = 0.9066 \cdot 1.006 = 0.9066 \, \text{kJ/kg} \)  

TASK 4c  
The mass flow rate of the refrigerant is calculated as:  
\[
\dot{m}_{\text{R134a}} = 0.337 \, \text{kg/h}
\]  

No additional diagrams or explanations are provided for this task.