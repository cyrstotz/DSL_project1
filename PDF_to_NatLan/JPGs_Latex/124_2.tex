TASK 3a  
The gas pressure \( p_{g,1} \) is calculated using the formula:  
\[
p_{g,1} = \frac{m_g}{A} + p_{\text{atm}}
\]  
where \( A = \pi \cdot D^2 / 4 \) and \( D = 0.1 \, \text{m} \). Substituting values:  
\[
A = \pi \cdot (0.1 \, \text{m})^2 / 4 = 7.85 \cdot 10^{-3} \, \text{m}^2
\]  
The pressure is then calculated as:  
\[
p_{g,1} = \frac{(m_{\text{EW}} + m_K) \cdot g}{A} + p_{\text{atm}}
\]  
Substituting \( m_{\text{EW}} = 0.7 \, \text{kg} \), \( m_K = 32 \, \text{kg} \), \( g = 9.87 \, \text{m/s}^2 \), \( p_{\text{atm}} = 70 \cdot 10^5 \, \text{Pa} \), and \( A = 7.85 \cdot 10^{-3} \, \text{m}^2 \):  
\[
p_{g,1} = \frac{(0.7 + 32) \cdot 9.87}{7.85 \cdot 10^{-3}} + 70 \cdot 10^5 = 7.7 \cdot 10^5 \, \text{Pa} = 7.7 \, \text{bar}
\]  

The gas mass \( m_g \) is calculated using the ideal gas law:  
\[
p \cdot V = m \cdot R \cdot T
\]  
Rearranging:  
\[
m_g = \frac{p_{g,1} \cdot V_{g,1} \cdot M_g}{R \cdot T}
\]  
Substituting \( R = 8.374 \, \text{kJ/kmol·K} \), \( M_g = 50 \, \text{kg/kmol} \), \( T = 773.75 \, \text{K} \), \( V_{g,1} = 3.79 \cdot 10^{-3} \, \text{m}^3 \), and \( p_{g,1} = 7.7 \cdot 10^5 \, \text{Pa} \):  
\[
m_g = \frac{7.7 \cdot 10^5 \cdot 3.79 \cdot 10^{-3} \cdot 50}{8.374 \cdot 773.75} = 28.40 \, \text{g}
\]  

---

TASK 3b  
Since \( x > 0 \), the system is in equilibrium, meaning not all the ice has melted. Therefore, the ice-water mixture remains at the equilibrium temperature of \( 0^\circ\text{C} \).  

The pressure in this case is independent of the gas and depends only on the mass of the water/ice, the weight, and the external pressure, which remains constant without mass exchange.  

---

TASK 3c  
The transferred energy is lost by the gas due to heat conduction:  
\[
Q_{12} = m \cdot (h_2 - h_1) = m_g \int_{T_1}^{T_2} c_p \, dT
\]  
Substituting \( c_p = R + c_v \), where \( R = 8.374 \, \text{kJ/kmol·K} \), \( c_v = 0.633 \, \text{kJ/kg·K} \), and \( M_g = 50 \, \text{kg/kmol} \):  
\[
c_p = \frac{R}{M_g} + c_v = \frac{8.374}{50} + 0.633 = 0.7593 \, \text{kJ/kg·K}
\]  
The heat transfer is then:  
\[
Q_{12} = m_g \cdot c_p \cdot (T_2 - T_1)
\]  
Substituting \( m_g = 0.0284 \, \text{kg} \), \( c_p = 0.7593 \, \text{kJ/kg·K} \), \( T_2 = 500 \, \text{K} \), and \( T_1 = 0^\circ\text{C} = 273.15 \, \text{K} \):  
\[
Q_{12} = 0.0284 \cdot 0.7593 \cdot (500 - 273.15) = 77.35 \, \text{kJ}
\]  

---

TASK 3d  
The final ice fraction \( x_{\text{ice},2} \) is calculated using the energy balance:  
\[
u_{\text{EW}} = x \cdot u_{\text{flüssig}} + (1 - x) \cdot u_{\text{fest}}
\]  
Substituting \( x = 0.6 \), \( u_{\text{flüssig}} = 0 \), \( u_{\text{fest}} = -333.4702 \, \text{kJ/kg} \):  
\[
u_{\text{EW}} = 0.6 \cdot 0 + (1 - 0.6) \cdot (-333.4702) = -733.4702 \, \text{kJ/kg}
\]  

Note: The calculation continues on the next page.