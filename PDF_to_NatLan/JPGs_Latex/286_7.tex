TASK 4a  
Three diagrams are drawn to represent the freeze-drying process in a pressure-temperature (\(p\)-\(T\)) diagram.  

1. **First diagram**:  
   - The phase regions for gas and liquid are depicted.  
   - A curve separates the liquid and gas phases.  
   - Points labeled 1, 2, 3, and 4 are marked along the curve, with arrows indicating transitions between states.  
   - The region labeled "flüssig" (liquid) is below the curve, and "gas" (gas) is above the curve.  
   - A vertical line connects state 1 to state 2, indicating a phase change.  

2. **Second diagram**:  
   - A rectangular cycle is drawn, with states 1, 2, 3, and 4 labeled.  
   - The region labeled "flüssig" (liquid) is below the curve, and "gas" (gas) is above the curve.  
   - Arrows indicate the transitions between states, forming a closed loop.  

3. **Third diagram**:  
   - A simpler diagram showing a linear progression between states 1, 2, 3, and 4.  
   - The region labeled "flüssig" (liquid) is below the curve, and "gasförmig" (gaseous) is above the curve.  
   - The transitions between states are represented by straight lines.  

TASK 4b  
The first law of thermodynamics is applied to the system:  
\[
0 = \dot{m} (h_{\text{in}} - h_{\text{out}}) + \dot{Q} - \dot{W}
\]  
Rearranging for the mass flow rate:  
\[
\dot{m} = \frac{\dot{W}}{h_3 - h_2}
\]