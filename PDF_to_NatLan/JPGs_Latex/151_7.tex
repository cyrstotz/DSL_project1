TASK 4c  
The vapor quality \( x_1 \) is calculated using the formula:  
\[
x_1 = \frac{h_1 - h_f}{h_{fg}}
\]  
where \( h_f \) is the enthalpy of the saturated liquid, \( h_{fg} \) is the enthalpy of vaporization, and \( h_1 \) is the enthalpy at state 1.  

It is stated that \( x_4 = 0 \), meaning the refrigerant is fully condensed at state 4.  

The pressure \( p_1 \) and enthalpy \( h_4 \) are also referenced but not further elaborated.

---

TASK 4d  
The coefficient of performance \( \epsilon_K \) for the cooling cycle is given by:  
\[
\epsilon_K = \frac{\dot{Q}_K}{\dot{W}_K} = \frac{\dot{Q}_K}{\dot{Q}_{\text{out}} - \dot{Q}_K} = \frac{\dot{Q}_K}{\dot{Q}_{\text{out}} - \dot{Q}_K}
\]  
This equation relates the heat removed (\( \dot{Q}_K \)) to the work input (\( \dot{W}_K \)) and other heat flows in the system.

---

TASK 4e  
The temperature in the freeze dryer would continue to sink until the temperature in the freeze dryer chamber reaches 0 K.