TASK 3a  
The gas pressure \( p_{g,1} \) and mass \( m_g \) are determined as follows:  

The pressure difference is calculated using:  
\[
p_{g,1} = p_{\text{atm}} + \frac{F}{A} = p_{\text{atm}} + \frac{m_s g}{A}
\]  
where \( A = \pi \frac{D^2}{4} = 0.00785 \, \text{m}^2 \).  

The force exerted by the piston is:  
\[
F = m_s g = 32.76 \, \text{kg} \cdot 9.81 \, \text{m/s}^2 = 321.801 \, \text{N}
\]  

Substituting values:  
\[
p_{g,1} = 1 \, \text{bar} + 0.4 \, \text{bar} = 1.4 \, \text{bar}
\]  

The gas pressure is therefore \( p_{g,1} = 1.4 \, \text{bar} \).  

The gas mass \( m_g \) is calculated using the ideal gas law:  
\[
p V = m R T
\]  
Rearranging:  
\[
m_g = \frac{p V}{R T}
\]  

Given:  
\[
V = 3.14 \times 10^{-3} \, \text{m}^3, \quad T = 773.15 \, \text{K}, \quad R = \frac{R_u}{M} = \frac{8.314}{50} = 166.28 \, \text{J/kg·K}
\]  

Substituting values:  
\[
m_g = \frac{1.4 \times 10^5 \cdot 3.14 \times 10^{-3}}{166.28 \cdot 773.15} = 3.42 \, \text{g}
\]  

The gas mass is \( m_g = 3.42 \, \text{g} \).  

---

TASK 3b  
The temperature \( T_{g,2} \) and pressure \( p_{g,2} \) are determined as follows:  

The temperature \( T_{g,2} \) is equal to \( T_{\text{EW},2} \), as the system reaches thermal equilibrium. Thus:  
\[
T_{g,2} = T_{\text{EW},2} = 273.15 \, \text{K}
\]  

The pressure \( p_{g,2} \) is calculated using the ideal gas law:  
\[
p_{g,2} = \frac{m_g R T_{g,2}}{V}
\]  

Substituting values:  
\[
p_{g,2} = \frac{3.42 \times 10^{-3} \cdot 166.28 \cdot 273.15}{3.14 \times 10^{-3}} = 48 \, \text{kPa} = 0.48 \, \text{bar}
\]  

The pressure is \( p_{g,2} = 0.48 \, \text{bar} \).  

---

TASK 3c  
The energy balance equation is written as:  
\[
m_g (u_2 - u_1) + m_{\text{EW}} (u_{\text{EW},2} - u_{\text{EW},1}) = 0
\]  

The change in internal energy for the gas is:  
\[
u_2 - u_1 = c_V (T_2 - T_1)
\]  

Substituting into the energy balance:  
\[
m_g c_V (T_2 - T_1) = m_{\text{EW}} \Delta u_{\text{EW}}
\]  

Where \( \Delta u_{\text{EW}} \) represents the change in internal energy of the ice-water mixture.  

This equation can be used to calculate the heat transferred \( Q_{12} \).  

