TASK 4a  
The diagram shows a pressure-temperature (\( p(T) \)) graph with phase regions labeled. The graph includes the following features:  
- A curve representing the phase boundary between liquid and vapor.  
- The region labeled "unterkühlte Flüssigkeit" (subcooled liquid) is below the curve.  
- The region labeled "Massedampf" (wet vapor) is along the curve.  
- The region labeled "übersättigter Dampf" (superheated vapor) is above the curve.  

TASK 4b  
The second diagram illustrates the freeze-drying process in two steps:  
- **Step ii**: A pressure-temperature (\( p(T) \)) graph with a vertical line labeled "gasförmig" (gaseous) and "flüssig" (liquid). The transition point is marked with a star, indicating the sublimation process.  
- **Step i**: Another pressure-temperature (\( p(T) \)) graph with a dome-shaped curve representing phase boundaries. States 1, 2, and 3 are labeled along the curve, indicating the refrigerant cycle.  

The mathematical expression for energy balance is:  
\[
0 = \dot{m}_2 (h_2 - h_3) + \dot{Q}_{\text{K}}
\]  

Using Table A-10:  
\[
h_2 = h_g(27.7 \, \text{K}) = 295.55 \, \frac{\text{kJ}}{\text{kg}}
\]  
\[
s_2 = s_3 = s_g(27.7^\circ\text{C}) = 0.916 \, \frac{\text{kJ}}{\text{kg·K}}
\]  

Given:  
\[
T_i = 10^\circ\text{C} = 283.15 \, \text{K}
\]  
\[
p_{\text{freeze}} = \text{constant}
\]  
\[
p_1 = 1 \, \text{mbar} = p_2
\]  
\[
T_2 = 283.15 \, \text{K} - 6 \, \text{K} = 277.15 \, \text{K}
\]  

Additional calculations (partially crossed out):  
\[
h_1 = h_f(283.15 \, \text{K}), \quad s_1 = s_f
\]  
\[
s_5 = s_4 = s_g(295.55 \, \text{K}) + s_g
\]  

No further content is clearly legible.