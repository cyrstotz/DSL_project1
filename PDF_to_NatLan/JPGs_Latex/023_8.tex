TASK 4a  
Two graphs are drawn:  

1. The first graph is a pressure-temperature (\( p \)-\( T \)) diagram. It shows phase regions with a curve representing the phase boundary. The curve includes oscillations, labeled "ik," likely indicating a process within the phase boundary.  
2. The second graph is also a \( p \)-\( T \) diagram. It includes labeled regions for "Gas/Fest" (gas-solid) and "Flüssig" (liquid). A line separates these regions, and a point labeled \( T_i \) is marked on the diagram.  

Below the diagrams, the following information is written:  
\[
T_i = -10^\circ\text{C}, \, p_i = 1 \, \text{mbar}
\]  
Additionally, "Temp. Lockpunkt = -16^\circ\text{C}" is noted, indicating a temperature locking point of \(-16^\circ\text{C}\).  

---

TASK 4b  
The energy balance for the process from state 1 to state 2 is written as:  
\[
Q = \dot{m} \left[ h_1 - h_2 \right] + \dot{Q}_K
\]  
It is further specified that:  
\[
h_2 = h_g(-22) = 234.8 \, \text{kJ/kg}
\]  

For the process from state 2 to state 3, the energy balance is:  
\[
Q = \dot{m} \left[ h_2 - h_3 \right] - \dot{W}_K
\]  

---

TASK 4c  
The enthalpy at state 1 is given as:  
\[
h_1 = h_f
\]  
The enthalpy at state 4 is calculated as:  
\[
h_4 = h_f(8 \, \text{bar}) = 93.42 \, \text{kJ/kg}
\]  

The vapor quality \( x \) is determined using the formula:  
\[
x = \frac{h_g - h_f}{h_g - h_f}
\]  
Arrows indicate the calculation steps for \( h_f \) and \( h_g \).  

No further content is visible.