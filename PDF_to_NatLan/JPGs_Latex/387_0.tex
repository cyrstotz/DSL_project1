TASK 3a  
The specific gas constant \( R \) is calculated using the formula:  
\[
R = \frac{\bar{R}}{M} = \frac{166.28 \, \text{J/(kg·K)}}{50 \, \text{kg/kmol}} = 3.3256 \, \text{J/(g·K)}
\]  

The pressure \( p_{g,1} \) is determined using the equation:  
\[
p_{g,1} = p_{\text{amb}} + \frac{m_K \cdot g}{\pi \cdot (0.05)^2} + \frac{0.1 \cdot g}{\pi \cdot (0.05)^2}
\]  
Substituting values:  
\[
p_{g,1} = 1 \, \text{bar} + \frac{32 \cdot 9.81}{\pi \cdot (0.05)^2} + \frac{0.1 \cdot 9.81}{\pi \cdot (0.05)^2} = 1.160015 \, \text{bar} \approx 1.16 \, \text{bar}
\]  

The mass of the gas \( m_g \) is calculated using the ideal gas law:  
\[
m_g = \frac{p \cdot V}{R \cdot T} = \frac{1.16 \cdot 3.14}{3.3256 \cdot 773.15} \approx 1.42 \, \text{g}
\]  

TASK 3b  
The heat transfer \( Q \) is given by:  
\[
Q = c_V \cdot (T_{g,2} - T_{g,1}) \cdot m_g
\]  
Assuming \( T_{g,2} = 0^\circ\text{C} \), the minimum heat transfer is calculated as:  
\[
Q_{\text{min}} = c_V \cdot (T_{g,2} - T_{g,1}) \cdot m_g = 0.633 \cdot (273.15 - 773.15) \cdot 1.42 \approx -1.08245 \, \text{kJ}
\]  

TASK 3c  
The maximum heat transfer \( Q_{\text{max}} \) is calculated using the enthalpy difference:  
\[
Q_{\text{max}} = m_{\text{EW}} \cdot (u_f(145) - u_i(145)) = -20 \, \text{kJ}
\]  

TASK 3d  
The explanation states that less heat \( Q \) is transferred to the ice-water mixture as the temperature approaches \( 0^\circ\text{C} \). This is because the exergy of the system decreases as the temperature gradient diminishes.  

The final pressure \( p_{g,2} \) is equal to \( p_{g,1} \):  
\[
p_{g,2} = p_{g,1} = 1.160015 \, \text{bar}
\]  

No diagrams or figures are present on the page.