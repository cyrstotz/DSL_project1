TASK 4c  
The vapor quality \( x_1 \) is calculated as:  
\[
x_1 = \frac{s_4 - s_{\text{liq}}}{s_{\text{liq}} - s_{\text{vap}}}
\]  
The refrigerant enthalpy at state 4 is given as:  
\[
h_4 = h_{\text{vap}}(8 \, \text{bar}) = 264.11 \, \frac{\text{kJ}}{\text{kg}}
\]  
The mass flow rate of the refrigerant is expressed as:  
\[
\dot{m} = \frac{\dot{Q}_K}{h_f}
\]  

---

TASK 4d  
The coefficient of performance (\( \epsilon_K \)) is calculated as:  
\[
\epsilon_K = \frac{\dot{Q}_K}{\dot{W}_K} = \frac{\dot{Q}_K}{T_{\text{evap}}}
\]  
The heat transfer rate is expressed as:  
\[
\dot{Q}_K = \dot{m} (h_2 - h_1)
\]  

---

TASK 4e  
The temperature inside the chamber would decrease if the cooling cycle continued with constant \( \dot{Q}_K \). This is because the interior of the freeze dryer is adiabatic to the surroundings, and heat removal would progressively lower the temperature.