TASK 3a  
To determine the gas pressure \( p_{g,1} \) and mass \( m_g \) in state 1:  

The gas pressure is calculated as:  
\[
p_{g,1} = p_{\text{amb}} + p_{\text{g,2}} = 1 \, \text{bar} + 0.1002 \, \text{bar} = 1.1002 \, \text{bar} = 1.1 \, \text{bar}
\]  

The mass is calculated using the ideal gas law:  
\[
p V = m R T \quad \Rightarrow \quad m = \frac{p V}{R T}
\]  
Given:  
\[
V = 3.14 \times 10^{-3} \, \text{m}^3, \quad T_{g,1} = 773.15 \, \text{K}, \quad R = \frac{8.314}{M} = 0.166 \, \text{kJ/kg·K}, \quad p = 11002.36 \, \text{Pa}
\]  

Substituting values:  
\[
m = \frac{11002.36 \, \text{Pa} \cdot 3.14 \times 10^{-3} \, \text{m}^3}{773.15 \, \text{K} \cdot 0.166 \, \text{kJ/kg·K}} = 2.692 \, \text{kg}
\]  

The weight balance is calculated as:  
\[
m_K \cdot g + m_{\text{EW}} \cdot g = 32 \, \text{kg} \cdot 9.81 \, \text{m/s}^2 + 0.1 \, \text{kg} \cdot 9.81 \, \text{m/s}^2 = 1002.36 \, \text{Pa}
\]  

---

TASK 3b  
Given \( x_{\text{ice},2} > 0 \), the temperature \( T_{g,2} \) and pressure \( p_{g,2} \) are determined.  

The temperature \( T_{g,2} \) is equal to \( T_{\text{EW},2} \), as the EW remains in the solid-liquid equilibrium region.  

---

TASK 3c  
To calculate the transferred heat \( Q_{12} \) from gas to EW between states 1 and 2:  

Using the energy balance:  
\[
\frac{d}{dt} \left( U + KE + PE \right) = \Delta Q
\]  

For internal energy change:  
\[
\Delta U_{12} = \Delta Q_{12} \quad \Rightarrow \quad \Delta u_{12} = \int_{T_1}^{T_2} c_V \, dT = c_V \cdot (T_2 - T_1)
\]  

Substituting values:  
\[
\Delta u_{12} = 0.633 \cdot (0.003) = 0.0019 \, \text{kJ/kg}
\]  

The heat transfer is:  
\[
\Delta q_{12} = 0.0019 \, \text{kJ/kg} \quad \Rightarrow \quad m \cdot \Delta q_{12} = 1900 \, \text{kJ}
\]  

This represents the heat transfer between the gas and EW.  

---

No diagrams or figures are present on this page.