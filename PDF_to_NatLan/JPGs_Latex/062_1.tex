TASK 3a  
The heat transferred \( Q_{12} \) is given as:  
\[
Q_{12} = 1300 \, \text{J}
\]

The volume of the ice-water mixture remains constant:  
\[
V_{\text{EW}} = V_{2,\text{EW}} \quad \text{and} \quad v_{1,\text{EW}} = v_{2,\text{EW}}
\]

The specific volume of the ice-water mixture is calculated using the ice fraction \( x_{\text{ice}} \):  
\[
v_{\text{EW}} = 0.6 \cdot v_g(0^\circ\text{C}) + (1 - 0.6) \cdot v_f(0^\circ\text{C}) = 125.98 \, \text{m}^3/\text{kg}
\]

TASK 3d  
The final ice fraction \( x_2 \) is determined using the relationship:  
\[
x_2 = \frac{v_2 - v_f}{v_g - v_f} = \frac{u_2 - u_f}{u_g - u_f}
\]

TASK 3c  
The change in internal energy \( \Delta U \) is calculated as:  
\[
\Delta U = Q_{12} - W_{12}
\]

Since the work term \( W_{12} \) is negligible, the internal energy change simplifies to:  
\[
\Delta U = m(u_2 - u_1)
\]

The specific internal energy difference is expressed as:  
\[
u_2 - u_1 = i_f(T_2 - T_1)
\]

Description of the graph:  
A wavy line is drawn, possibly representing a qualitative temperature or energy profile over time or states. No axes or labels are provided.