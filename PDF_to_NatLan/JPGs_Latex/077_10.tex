TASK 4a  
The diagram is a pressure-temperature (\(p\)-\(T\)) plot illustrating the freeze-drying process. It includes the phase regions for solid, liquid, and gas, with the triple point marked.  
- The pressure axis (\(p\)) is labeled, with a point at 5 mbar, which is below the triple point pressure.  
- The temperature axis (\(T\)) is labeled in degrees Celsius (\(^\circ\text{C}\)).  
- The process steps are indicated:  
  1. Isobaric freezing: Pressure remains constant, temperature decreases, and the state transitions to solid.  
  2. Isothermal sublimation: Pressure decreases, temperature remains constant, and the state transitions to gas.  

TASK 4b  
The refrigerant mass flow rate (\(\dot{m}_{\text{R134a}}\)) is to be determined.  

The state transition from 2 to 3 is described as isentropic because it is reversible and adiabatic.  

The energy balance equation for the process is given as:  
\[
\dot{m}_{\text{R134a}} (h_e - h_a) + \sum \dot{Q}_j - \sum \dot{W}_j = 0
\]  
where \(h_e\) and \(h_a\) represent specific enthalpies, \(\dot{Q}_j\) is the heat transfer rate, and \(\dot{W}_j\) is the work rate.  

Additional notes:  
- The piston is described as mechanical.  
- Energy balance for the compressor is highlighted.  

The energy balance for compression is expressed as:  
\[
\dot{m} (h_2 - h_3) = 28 \, \text{W}
\]