TASK 2a  
A graph is drawn representing the jet engine process on a \( T \)-\( s \) diagram. The axes are labeled as follows:  
- The vertical axis is \( T \) (temperature) in Kelvin \([K]\).  
- The horizontal axis is \( s \) (specific entropy) in \([ \frac{k}{\text{kg} \cdot \text{K}} ]\).  

The process includes the following labeled points and transitions:  
- Point 1: Starting point, labeled with "nus < 1".  
- Point 2: Isentropic process, labeled "adiabatic rev".  
- Point 3: Isobaric process, labeled "isobar".  
- Point 5: Isentropic process, labeled "adiabatic rev".  
- Point 6: Isentropic process, labeled "adiabatic rev".  

Additional annotations include:  
- \( p_5 = p_6 = p_0 \), indicating constant pressure at points 5 and 6.  
- "Mischkammer isobar", referring to the mixing chamber operating isobarically.  
- \( p_0 = p_8 \), indicating constant pressure at points 0 and 8.  

TASK 2b  
The task involves determining the outlet velocity \( w_6 \) and temperature \( T_6 \).  

The process from point 5 to point 6 is described as a reversible adiabatic (isentropic) process. The following equations and calculations are provided:  

1. The temperature \( T_6 \) is calculated using the isentropic relation:  
\[
\frac{T_6}{T_5} = \left( \frac{p_6}{p_5} \right)^{\frac{k-1}{k}}
\]  
Substituting values:  
\[
T_6 = T_5 \cdot \left( \frac{p_6}{p_5} \right)^{\frac{k-1}{k}}
\]  
\[
T_6 = 328.07 \, \text{K}
\]  

2. The outlet velocity \( w_6 \) is calculated using the energy balance:  
\[
w_6^2 = 2 \cdot c_p \cdot k \cdot (T_5 - T_6)
\]  
Substituting values:  
\[
w_6 = \sqrt{2 \cdot c_p \cdot k \cdot (T_5 - T_6)}
\]  
\[
w_6 = 540.8 \, \text{m/s}
\]  

Additional notes:  
- \( k = \frac{c_p}{c_v} \), and \( c_v = c_p \cdot k \).  

No further content is visible.