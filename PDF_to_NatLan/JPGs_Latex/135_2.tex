TASK 2a  
A qualitative \( T \)-\( s \) diagram is drawn to represent the jet engine process. The diagram includes labeled isobars and key states:  
- State 0 is at ambient pressure \( p_0 = 0.191 \, \text{bar} \).  
- State 2 is labeled with \( p = 13 \, \text{bar} \).  
- States 3, 4, and 5 are connected with curved lines representing the compression and combustion processes.  
- State 5 is labeled with \( p = 0.5 \, \text{bar} \).  
- State 6 is labeled with \( p = 0.191 \, \text{bar} \).  

The diagram shows the thermodynamic process qualitatively, with transitions between states and isobaric lines clearly marked.

---

TASK 2b  
The nozzle is modeled as an isentropic process. The temperature at state 6 (\( T_6 \)) is calculated using the isentropic relation:  
\[
\frac{T_0}{T_5} = \left( \frac{p_0}{p_5} \right)^{\frac{\kappa-1}{\kappa}}
\]  
Substituting values:  
\[
T_6 = T_5 \left( \frac{p_0}{p_5} \right)^{\frac{\kappa-1}{\kappa}}
\]  
\[
T_6 = 431.9 \, \text{K} \left( \frac{0.191 \, \text{bar}}{0.5 \, \text{bar}} \right)^{\frac{0.4}{1.4}}
\]  
\[
T_6 = 328.025 \, \text{K}
\]  

The outlet velocity \( w_6 \) is calculated using the steady-flow energy equation:  
\[
0 = \dot{m} \left( h_5 - h_6 + \frac{w_6^2 - w_5^2}{2} \right) + \dot{Q} - \dot{W}
\]  
Assuming no heat transfer (\( \dot{Q} = 0 \)) and no work (\( \dot{W} = 0 \)):  
\[
w_6 = \sqrt{2 \cdot (h_5 - h_6) + w_5^2}
\]  
Using \( h = c_p \cdot T \):  
\[
w_6 = \sqrt{2 \cdot c_p \cdot (T_5 - T_6) + w_5^2}
\]  
Substituting values:  
\[
w_6 = \sqrt{2 \cdot 1.006 \, \text{kJ/kg·K} \cdot (431.9 \, \text{K} - 328.025 \, \text{K}) + (220 \, \text{m/s})^2}
\]  
\[
w_6 = 507.24 \, \text{m/s}
\]