TASK 1d  
The task involves determining \( \Delta m_{12} \), the mass of saturated liquid water added during the cooling process from \( T_1 = 100^\circ\text{C} \) to \( T_2 = 70^\circ\text{C} \). The inlet temperature of the added water is \( T_{\text{in,12}} = 20^\circ\text{C} \).

Energy balance for the system is written as:  
\[
\Delta E = m_2 u_2 - m_1 u_1 + \Delta KE + \Delta PE
\]  
Expanding the equation:  
\[
\Delta E = -\dot{m}_j \left[ h_i + \frac{w^2}{2} + g z_i \right] + \sum_j Q_j - \sum_n \psi_u
\]  

Here:  
- \( \Delta E \) represents the change in energy.  
- \( m_2 u_2 \) and \( m_1 u_1 \) are the internal energy terms for states 2 and 1, respectively.  
- \( \Delta KE \) and \( \Delta PE \) are changes in kinetic and potential energy, which are negligible.  
- \( \dot{m}_j \) is the mass flow rate.  
- \( h_i \) is the enthalpy at the inlet.  
- \( w \) is the velocity.  
- \( g z_i \) is the gravitational potential energy term.  
- \( \sum_j Q_j \) is the heat transfer term.  
- \( \sum_n \psi_u \) accounts for other energy interactions.  

No further calculations or results are visible on the page.  

No diagrams or graphs are present.