TASK 4a  
A graph is drawn showing the freeze-drying process in a pressure-temperature (\( p \)-\( T \)) diagram. The axes are labeled as \( p \, [\text{bar}] \) for pressure and \( T \, [\text{K}] \) for temperature. The graph includes three labeled states (1, 2, and 3) and a curve representing the phase boundary.  
- State 1 is in the liquid region.  
- State 2 is near the phase boundary, indicating evaporation.  
- State 3 is in the vapor region.  
Arrows indicate transitions between states, and the process is labeled as "isobaric" and "reversible adiabatic."  

TASK 4b  
A schematic diagram of a refrigeration cycle is drawn, showing three states (1, 2, and 3).  
- \( p_2 = 1 \, \text{bar} \)  
- \( p_3 = 8 \, \text{bar} \)  

The energy balance equation for the system is written as:  
\[
\frac{dE}{dt} = \sum_i \dot{m}_i \left( h_{\text{in},i} - h_{\text{out},i} \right) + \sum \dot{Q}_i - \sum \dot{W}_i
\]  
where \( h \) represents enthalpy, \( \dot{Q} \) is heat transfer, and \( \dot{W} \) is work.  

The work equation is given as:  
\[
\omega = \dot{m}_{\text{R134a}} \left( h_2 - h_3 \right) - W_i
\]  

TASK 4c  
The pressure at state 1 is given as \( p_1 = 8 \, \text{bar} \), and the vapor quality at state 1 is \( x_1 = 0 \).  

The heat transfer equation is written as:  
\[
Q_q = \dot{m}_{\text{R134a}} \cdot h_{\text{fg}} = 93.42 \, \frac{\text{kJ}}{\text{kg}}
\]  

The mass flow rate is calculated as:  
\[
\dot{m}_{\text{R134a}} = 93.42 \, \frac{\text{kg}}{\text{s}}
\]  

TASK 4d  
The coefficient of performance (\( \epsilon_K \)) is calculated using the formula:  
\[
\epsilon_K = \frac{1}{\dot{W}_K} - \frac{\dot{Q}_{\text{out}}}{\dot{W}_K}  
\]  
where \( \dot{Q}_{\text{out}} \) and \( \dot{W}_K \) represent heat transfer and work, respectively.  

No additional diagrams or explanations are provided for this task.