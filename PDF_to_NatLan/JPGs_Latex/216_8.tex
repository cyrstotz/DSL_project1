TASK 4c  
The refrigerant R134a enters the system with a mass flow rate of \( \dot{m}_{\text{R134a}} = 4 \, \text{kg/s} \) at a temperature \( T = -22^\circ\text{C} \).  

For the throttling process:  
\[
h_4 = h_1
\]  

From Table A-11:  
\[
h_1(8 \, \text{bar}, x = 0) = 93.42 \, \frac{\text{kJ}}{\text{kg}}
\]  
\[
h_3(8 \, \text{bar}) = 93.42 \, \frac{\text{kJ}}{\text{kg}}
\]  

Entropy remains constant:  
\[
s_2 = s_3
\]  

The enthalpy at state 4 is calculated as:  
\[
h_4 = h_1 + x \cdot (h_{g4} - h_{f4})
\]  
Where:  
\[
x = \frac{h_1 - h_{f4}}{h_{g4} - h_{f4}}
\]  
Using the values:  
\[
h_{f4}(-22^\circ\text{C}, x = 0) = 40 \, \frac{\text{kJ}}{\text{kg}}
\]  
\[
h_{g4}(-22^\circ\text{C}, x = 1) = 247.24 \, \frac{\text{kJ}}{\text{kg}}
\]  
\[
x = \frac{93.42 - 40}{247.24 - 40} = 0.27
\]  

TASK 4d  
The coefficient of performance \( \epsilon_K \) is calculated as:  
\[
\epsilon_K = \frac{\dot{Q}_K}{\dot{W}_K}
\]  
\[
\epsilon_K = 5.87
\]  

The heat transfer \( Q_{12} \) is given by:  
\[
Q_{12} = \dot{m} \cdot (h_2 - h_1)
\]  
\[
Q_{12} = 4 \cdot (247.24 - 93.42) = 614.72 \, \text{kW}
\]  

TASK 4e  
Interpolation is performed to find intermediate enthalpy values:  
\[
h_{f1} = -20 + 12 \cdot \frac{39.54 - 34.97}{24 + 34.39} = 36.965 \, \frac{\text{kJ}}{\text{kg}}
\]  
\[
h_{g1} = -10 + 12 \cdot \frac{205.57 - 203}{20 + 203} = 203.389 \, \frac{\text{kJ}}{\text{kg}}
\]  

The explanation for the evolution of \( T_i \):  
The temperature \( T_i \) would first increase due to the heat transfer from the refrigerant. Once equilibrium is reached, the temperature would stabilize.  

TASK 4e (continued)  
The final statement explains that the water vapor would first flow into the chamber and only later condense when the temperature stabilizes.