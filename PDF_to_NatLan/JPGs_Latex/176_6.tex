TASK 4a  
The diagram is a pressure-temperature (\(p\)-\(T\)) graph. The vertical axis represents pressure in bar, and the horizontal axis represents temperature in degrees Celsius (\(^\circ\text{C}\)). The graph shows phase regions, including a line labeled "flowing" and another labeled "fast." The phase change boundaries are indicated, and the graph includes arrows pointing toward specific regions.

---

TASK 4b  
The energy balance equation is written as:  
\[
0 = \dot{m}(h_2 - h_1) + Q - \dot{W}_K
\]  
Rearranged for the adiabatic case:  
\[
0 = \dot{m}(h_2 - h_3) - \dot{W}_K
\]  
The work done by the compressor is expressed as:  
\[
\dot{W}_K = \dot{m}(h_2 - h_3)
\]  

The enthalpy at state 2 (\(h_2\)) is calculated as:  
\[
h_2 = h_g(T_2, p_2) = 259.460 \, \frac{\text{kJ}}{\text{kg}}
\]  

The enthalpy at state 3 (\(h_3\)) is determined using the refrigerant properties:  
\[
h_3 = h_f(8 \, \text{bar}) = 93.42 \, \frac{\text{kJ}}{\text{kg}}
\]  

The temperature at state 2 (\(T_2\)) is calculated as:  
\[
T_2 = T_i - 6 \, \text{K} = -22^\circ\text{C}
\]  

The enthalpy of vaporization at \(220^\circ\text{C}\) is calculated using interpolation:  
\[
h_g(220^\circ\text{C}) = \frac{260.45 - 258.36}{264 - 20} (22 - 20) + 258.36 = 259.407 \, \frac{\text{kJ}}{\text{kg}}
\]  

---

TASK 4c  
The vapor quality (\(x_4\)) at state 4 is calculated using the enthalpy relationship:  
\[
h_4 = h_f + x_4(h_g - h_f)
\]  

Given:  
\[
h_4 = h_f(8 \, \text{bar}) = 93.42 \, \frac{\text{kJ}}{\text{kg}}
\]  
\[
x_4 = 0
\]  

The pressure at state 4 (\(p_4\)) is equal to \(p_3 = 8 \, \text{bar}\).  

The temperature at state 2 (\(T_2\)) is confirmed as:  
\[
T_2 = -22^\circ\text{C}
\]  