TASK 4a  
The page contains two diagrams labeled as part of the freeze-drying process described in Task 4.  

**Diagram (i):**  
This is a pressure-temperature (\( p \)-\( T \)) diagram illustrating the refrigeration cycle.  
- The diagram shows four states labeled \( 1 \), \( 2 \), \( 3 \), and \( 4 \).  
- The process between states \( 1 \) and \( 2 \) is marked as "isobaric."  
- The process between states \( 2 \) and \( 3 \) is labeled "adiabatic reversible."  
- The process between states \( 3 \) and \( 4 \) is marked as "isobaric."  
- The process between states \( 4 \) and \( 1 \) is labeled "adiabatic reversible."  
- The curve labeled "NS" represents the saturation line of the refrigerant.  

**Diagram (ii):**  
This is another \( p \)-\( T \) diagram illustrating the sublimation process in Step ii of freeze-drying.  
- The diagram shows a curve labeled "NS," which represents the saturation line.  
- A point labeled "Triple point" is marked at the peak of the curve.  
- The process is labeled "isotherm," indicating constant temperature during sublimation.  

TASK 4b  
The page contains mathematical expressions related to energy balance and thermodynamic calculations.  

1. The general energy balance equation is written as:  
\[
\frac{\partial E}{\partial t} = \sum_i \dot{m}_i(t) \left[ h_i(t) + \frac{v_i^2(t)}{2} + g z_i(t) \right] + \sum_j \dot{Q}_j(t) - \sum_k \dot{W}_k(t)
\]  
Where:  
- \( \dot{m}_i(t) \) represents mass flow rate.  
- \( h_i(t) \) is specific enthalpy.  
- \( v_i(t) \) is velocity.  
- \( z_i(t) \) is height.  
- \( \dot{Q}_j(t) \) is heat transfer rate.  
- \( \dot{W}_k(t) \) is work transfer rate.  

2. The work rate equation is given as:  
\[
\dot{W}_u = \dot{m} \left[ h_2 - h_3 \right]
\]  

3. The temperature relationship is expressed as:  
\[
T_2 = T_1 - CU
\]  

4. Additional notes include:  
- \( p_1 = p_2 \)  
- \( p_3 = p_4 = \Delta p_{\text{min}} \)  

No further explanation or context is provided for these equations.