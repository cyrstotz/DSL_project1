TASK 2a  
A graph is drawn representing a qualitative \( T \)-\( s \) diagram. The vertical axis is labeled \( T \) [K], and the horizontal axis is labeled \( s \) [kJ/kg·K]. The diagram includes several labeled points and curves:  
- Points 0, 2, 3, 4, 5, and 6 are marked along the curves.  
- The curve from point 0 to point 2 is labeled "drive line."  
- The curve from point 2 to point 3 is labeled "rotor."  
- The curve from point 3 to point 4 is labeled "rotor."  
- The curve from point 4 to point 5 is labeled "stator."  
- The curve from point 5 to point 6 is labeled "stator."  
The diagram visually represents the thermodynamic process in the jet engine, with isobars and qualitative transitions between states.

---

TASK 2b  
The following calculations and equations are provided:  

1. The pressure at state 6 is equal to the pressure at state 1:  
\[
p_6 = p_1
\]

2. The mass flow rate is expressed as:  
\[
\dot{m} = \frac{\dot{V}}{v} = A \cdot \dot{\omega}
\]

3. The temperature ratio between states 5 and 6 is given by:  
\[
\frac{T_6}{T_5} = \left( \frac{p_6}{p_5} \right)^{\frac{n-1}{n}}
\]  
where \( n = k = 1.4 \).  

4. Substituting values, the temperature at state 6 is calculated as:  
\[
T_6 = 328.07 \, \text{K}
\]

5. The ideal gas law is referenced:  
\[
p \cdot \dot{V} = \dot{m} \cdot R \cdot T
\]

6. The specific volume is expressed as:  
\[
v = \frac{A \cdot \dot{\omega}}{\dot{m}}
\]