TASK 1a  
The problem involves calculating the heat flow \( \dot{Q}_{\text{out}} \) removed by the coolant through the reactor wall.  

A diagram is drawn showing a reactor with inlet and outlet streams labeled \( T_{\text{in}} \) and \( T_{\text{out}} \), respectively. Heat flow \( \dot{Q}_{\text{out}} \) is indicated as leaving the reactor and entering the coolant stream labeled \( T_{\text{KF,in}} \) and \( T_{\text{KF,out}} \).  

The steady-state energy balance is written as:  
\[
\dot{m}_{\text{in}} \cdot h_{\text{in}} = \dot{m}_{\text{out}} \cdot h_{\text{out}} + \dot{Q}_{\text{out}} + \dot{Q}_R
\]  
where \( \dot{m}_{\text{in}} = \dot{m}_{\text{out}} \) due to steady-state operation.  

The heat flow \( \dot{Q}_{\text{out}} \) is expressed as:  
\[
\dot{Q}_{\text{out}} = \dot{m}_{\text{in}} \cdot (h_{\text{out}} - h_{\text{in}}) - \dot{Q}_R
\]  

Using water table values:  
\[
h_{\text{in}} = h(70^\circ\text{C, liquid}) = 292.88 \, \text{kJ/kg}
\]  
\[
h_{\text{out}} = h(100^\circ\text{C, liquid}) = 419.04 \, \text{kJ/kg}
\]  

Substituting values:  
\[
\dot{Q}_{\text{out}} = 0.3 \cdot (419.04 - 292.88) - 100 \, \text{kW}
\]  
\[
\dot{Q}_{\text{out}} = 62.182 \, \text{kW}
\]  

TASK 1b  
The problem involves determining the thermodynamic mean temperature \( T_{\text{KF}} \) of the coolant.  

The mean temperature is calculated using the logarithmic mean temperature difference formula:  
\[
T_{\text{KF}} = \frac{T_2 - T_1}{\ln\left(\frac{T_2}{T_1}\right)}
\]  
where \( T_1 = 288.15 \, \text{K} \) and \( T_2 = 298.15 \, \text{K} \).  

Substituting values:  
\[
T_{\text{KF}} = \frac{298.15 - 288.15}{\ln\left(\frac{298.15}{288.15}\right)}
\]  
\[
T_{\text{KF}} = 203.12 \, \text{K}
\]  

This concludes the calculations for \( \dot{Q}_{\text{out}} \) and \( T_{\text{KF}} \).