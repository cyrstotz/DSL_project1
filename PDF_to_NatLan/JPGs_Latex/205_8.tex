TASK 4a  
The diagram is a qualitative representation of the freeze-drying process in a \( p \)-\( T \) diagram. It includes labeled phase regions and key states.  
- The vertical axis represents pressure \( p \) in bar.  
- The horizontal axis represents temperature \( T \) in degrees Celsius.  
- The diagram shows a curve separating phase regions, with labeled points:  
  - State 1 is in the vapor region.  
  - State 2 is at the saturated vapor line.  
  - State 3 is in the liquid region.  
  - State 4 is also in the liquid region.  
Arrows indicate transitions between states, with processes such as compression and expansion marked.  

TASK 4b  
The rotational flow process is described as follows:  
\[
O = \dot{m} \cdot (h_2 - h_1 + \frac{w_2^2}{2} - \frac{w_1^2}{2}) + \dot{Q} - \dot{W}
\]  
For the transition from state 2 to state 3:  
\[
O = \dot{m} \cdot (h_2 - h_3)
\]  
The enthalpy \( h_2 \) is determined from tables (Table A-11) for \( p_3 = 8 \, \text{bar} \).  
\[
h_2 = h_f(p_3) = 93 \, \text{kJ/kg}
\]  

TASK 4c  
The entropy balance for the process from state 2 to state 3 is described as reversible and adiabatic:  
\[
\Delta s = 0
\]  
Thus, the entropy change is zero:  
\[
\Delta s = 0
\]  

TASK 4e  
The evolution of \( T_i \) in Step ii is described:  
If the cooling cycle from Step i continued with constant \( \dot{Q}_K \), \( T_i \) would decrease further below the triple point of water. This is because the sublimation process requires heat removal, and maintaining \( \dot{Q}_K \) would drive the temperature lower.