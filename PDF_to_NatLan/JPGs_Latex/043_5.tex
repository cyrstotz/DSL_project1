TASK 3c  
The heat transfer \( Q_{12} \) is determined using an energy balance.  

The energy balance equation is:  
\[
\Delta E = E_2 - E_1 = Q_{12} - W_{12}
\]  
Since \( W_{12} = 0 \), the equation simplifies to:  
\[
E_2 - E_1 = Q_{12}
\]  

The internal energy change is expressed as:  
\[
m_g (u_2 - u_1) = Q_{12}
\]  

Substituting values:  
\[
0.0036 \cdot c_V \cdot (T_2 - T_1) = Q_{12}
\]  

Using \( c_V = 0.633 \, \text{kJ/kg·K} \), \( T_2 = 3 \, \text{K} \), and \( T_1 = 500 \, \text{K} \):  
\[
0.0036 \cdot 0.633 \cdot (0.003 - 500) = Q_{12}
\]  

The calculated heat transfer is:  
\[
Q_{12} = -1.13933 \, \text{kJ}
\]  

---

TASK 3d  
The final ice fraction \( x_2 \) is calculated using the formula:  
\[
x_2 = \frac{u_2 - u_{\text{flüssig}}}{u_{\text{fest}} - u_{\text{flüssig}}}
\]  

Where:  
- \( u_{\text{flüssig}} = -0.033 \, \text{kJ/kg} \)  
- \( u_{\text{fest}} = -333.492 \, \text{kJ/kg} \) (from Table A-1 provided during the exam)  

The internal energy \( u_2 \) is determined as:  
\[
u_2 = u_1 + \frac{Q_{12}}{m_{\text{EW}}}
\]  

Substituting values:  
\[
u_2 = -0 + \frac{-1.13933}{0.15} = -7.59553 \, \text{kJ/kg}
\]  

Now, solving for \( x_2 \):  
\[
x_2 = \frac{-7.59553 - (-0.033)}{-333.492 - (-0.033)}
\]  

Simplifying:  
\[
x_2 = \frac{-7.59553 + 0.033}{-333.492 + 0.033}
\]  

\[
x_2 = \frac{-7.56253}{-333.459}
\]  

The final ice fraction is:  
\[
x_2 = 0.0227
\]  

This result is boxed in the handwritten solution.