TASK 4a  
A graph is drawn with pressure \( p \) (in bar) on the vertical axis and temperature \( T \) (in Kelvin) on the horizontal axis. The graph illustrates the freeze-drying process.  
- Point 1 represents "isobaric freezing."  
- Point 2 represents "isothermal pressure reduction."  
- Point 3 is connected back to Point 1, completing the cycle.  

TASK 4b  
From state 3 to state 4:  
\[
\dot{m}_{\text{R134a}} = \frac{\dot{W}_K}{h_2 - h_3}
\]  

From state 2 to state 3:  
\[
0 = -\dot{W}_K + \dot{m} (h_2 - h_3)
\]  

The enthalpy at state 3 is given as:  
\[
h_3 = 264.15 \, \text{kJ/kg}
\]  

The temperature at state 2 is calculated as:  
\[
T_2 = T_i - 6 \, \text{K}
\]  

Reference is made to Table A.11 for further data.