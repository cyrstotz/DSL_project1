TASK 2a  
The process is represented qualitatively in a \( T \)-\( s \) diagram. The diagram includes labeled isobars and key states of the jet engine process.  
- State 1: Inlet conditions.  
- State 2: Compression (adiabatic).  
- State 3: Combustion (isobaric).  
- State 4: Expansion in the turbine (adiabatic).  
- State 5: Mixing chamber.  
- State 6: Nozzle exit.  

The diagram shows increasing entropy (\( s \)) along the x-axis and temperature (\( T \)) along the y-axis. The compression and turbine processes are curved, indicating adiabatic changes, while the combustion process is horizontal, representing isobaric heat addition.  

---

TASK 2b  
The outlet temperature \( T_6 \) is calculated using the energy balance equation:  
\[
T_6 = T_5 \left( \frac{p_6}{p_5} \right)^{\frac{\kappa - 1}{\kappa}}
\]  
Substituting values:  
\[
T_6 = 431.9 \, \text{K} \left( \frac{0.191}{0.5} \right)^{\frac{1.4 - 1}{1.4}} = 328.6 \, \text{K}
\]  

The outlet velocity \( w_6 \) is determined using the energy equation:  
\[
w_6 = \sqrt{2 \left( h_5 - h_6 \right) + w_5^2}
\]  
Substituting enthalpy and velocity values:  
\[
w_6 = \sqrt{2 \left( c_p \left( T_5 - T_6 \right) \right) + w_5^2}
\]  
\[
w_6 = \sqrt{2 \left( 1.006 \, \text{kJ/kg·K} \cdot (431.9 - 328.6) \, \text{K} \right) + (220 \, \text{m/s})^2}
\]  
\[
w_6 = \sqrt{2 \cdot 103.6 \, \text{kJ/kg} + 48,400 \, \text{m}^2/\text{s}^2}
\]  
\[
w_6 = \sqrt{206.6 \, \text{kJ/kg} + 48,400 \, \text{m}^2/\text{s}^2}
\]  
\[
w_6 = 328.7 \, \text{m/s}
\]  

---

TASK 2c  
The mass-specific increase in flow exergy is calculated as:  
\[
\Delta ex_{\text{flow}} = ex_{\text{flow},6} - ex_{\text{flow},0}
\]  
Using the equation:  
\[
ex_{\text{flow}} = R \left( T_0 \ln \frac{T}{T_0} - T + T_0 \right) + \frac{w^2}{2}
\]  

For state 6:  
\[
ex_{\text{flow},6} = R \left( T_0 \ln \frac{T_6}{T_0} - T_6 + T_0 \right) + \frac{w_6^2}{2}
\]  
Substituting values:  
\[
R = c_p - c_v = 1.006 - 0.717 = 0.289 \, \text{kJ/kg·K}
\]  
\[
ex_{\text{flow},6} = 0.289 \left( 243.15 \ln \frac{328.6}{243.15} - 328.6 + 243.15 \right) + \frac{328.7^2}{2}
\]  
\[
ex_{\text{flow},6} = 77.659 \, \text{kJ/kg}
\]  

For state 0:  
\[
ex_{\text{flow},0} = R \left( T_0 \ln \frac{T_0}{T_0} - T_0 + T_0 \right) + \frac{w_0^2}{2}
\]  
\[
ex_{\text{flow},0} = 0.289 \left( 243.15 \ln 1 - 243.15 + 243.15 \right) + \frac{200^2}{2}
\]  
\[
ex_{\text{flow},0} = 0.289 \cdot 0 + \frac{200^2}{2}
\]  
\[
ex_{\text{flow},0} = 20,000 \, \text{m}^2/\text{s}^2
\]  

Finally:  
\[
\Delta ex_{\text{flow}} = 77.659 - 20,000 = 57.659 \, \text{kJ/kg}
\]  

---

Description of diagrams:  
The \( T \)-\( s \) diagram visually represents the thermodynamic processes in the jet engine, including compression, combustion, and expansion.