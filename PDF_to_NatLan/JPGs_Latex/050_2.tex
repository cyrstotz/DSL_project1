TASK 2a  
The diagram represents a qualitative \( T \)-\( s \) (temperature-entropy) diagram for the jet engine process. It includes labeled isobars and key states (0 to 6). The following features are visible:  
- State 0: Ambient conditions at \( -30^\circ\text{C} \).  
- State 1: Compression begins, \( p_2 = p_3 \).  
- State 3: Combustion occurs, \( p_4 < p_3 \).  
- State 4: Expansion in the turbine, \( p_5 = p_6 \).  
- State 5: Mixing chamber, \( T_5 = 431.9 \, \text{K} \).  
- State 6: Nozzle exit, \( p_6 = p_0 \).  

The diagram shows reversible adiabatic processes (marked "ad.rev"), isobaric combustion, and the nozzle exit conditions. The entropy increases during irreversible processes.  

---

TASK 2b  
The energy balance equation is written as:  
\[
\frac{dE}{dt} = 0 = \dot{m} \left( h_0 - h_6 \right) + \frac{\left( w_0^2 - w_6^2 \right)}{2} + \frac{q_{B}}{\dot{m}_K} + W
\]  
This equation accounts for enthalpy differences, kinetic energy changes, heat transfer, and work done.  

Another form of the equation is shown:  
\[
0 = \dot{m} c_p \left( T_0 - T_6 \right) + \frac{\left( w_0^2 - w_6^2 \right)}{2} = -\frac{q_{B}}{\dot{m}_K}
\]  
This version uses specific heat capacity \( c_p \) and temperature differences.  

Crossed-out calculations for mass flow rates and other parameters are ignored.  

---

TASK 2a (continued)  
A second \( T \)-\( s \) diagram is drawn, showing the same process with additional annotations:  
- States 0 to 6 are labeled.  
- Isobars are marked at \( 0.5 \, \text{bar} \) and \( 0.191 \, \text{bar} \).  
- The mixing chamber and nozzle exit are highlighted.  
- The entropy increases during irreversible processes, and the temperature changes are consistent with the engine's operation.  

No additional numerical values are provided.