TASK 3a  
The universal gas constant \( R \) is calculated as:  
\[
R = \frac{\bar{R}}{M} = \frac{8314 \, \text{J/(kmol·K)}}{50 \, \text{kg/kmol}} = 166.28 \, \text{J/(kg·K)}.
\]  

The total mass \( m \) is given by:  
\[
m = m_K + m_{\text{EW}} = 32 \, \text{kg} + 0.1 \, \text{kg} = 32.1 \, \text{kg}.
\]  

The gas pressure \( p_{g,1} \) is calculated as:  
\[
p_{g,1} = p_{\text{EW},1} = p_{\text{amb}} + \frac{m \cdot g}{\pi \cdot \left(\frac{D}{2}\right)^2} = 1.40 \times 10^5 \, \text{Pa}.
\]  

The gas mass \( m \) is determined using the ideal gas law:  
\[
\rho V = mRT \quad \text{or} \quad m = \frac{pV}{RT}.
\]  
Substituting values:  
\[
m = \frac{1.4 \, \text{bar} \cdot 3.14 \, \text{L}}{166.28 \, \text{J/(kg·K)} \cdot 500^\circ\text{C}} = 3.49 \, \text{kg}.
\]  

---

TASK 3b  
The pressure \( p_{g,2} \) is equal to \( p_{g,1} \):  
\[
p_{g,2} = p_{g,1} = 1.4 \times 10^5 \, \text{Pa}.
\]  

The explanation states:  
"The pressure remains constant because it is balanced by external forces (atmospheric pressure and piston)."

---

TASK 3c  
Several equations and diagrams are partially visible but crossed out. These include expressions for heat transfer and temperature relationships, such as:  
\[
Q_{\text{EW}} + Q_{g,1} = Q_{g,2} + Q_{\text{EW},2}.
\]  
Other crossed-out equations involve entropy and temperature ratios:  
\[
T_{g,2} = T_{g,1}, \quad \frac{p_{g,2}}{p_{g,1}} = \left(\frac{T_{g,2}}{T_{g,1}}\right)^{\kappa - 1}.
\]  

A small diagram shows intersecting lines labeled with \( T_{g,2} \), \( T_{g,1} \), \( p_{g,2} \), and \( p_{g,1} \), but it is crossed out and not fully legible.  

---

No further content is clearly visible.