TASK 4a  
Two diagrams are drawn:  
1. The first diagram is a pressure-temperature (\(p\)-\(T\)) plot. It shows phase regions labeled as "liquid," "vapor," and "mixed phase." The curve represents the phase boundary between these regions.  
2. The second diagram is another \(p\)-\(T\) plot. It includes a labeled point for \(T_i\) and arrows indicating the cooling process. The curve represents the refrigerant cycle, with labels for "R134a" and "phase transitions."  

TASK 4b  
The equation for the heat exchanger is given:  
\[
Q = \dot{m} (h_2 - h_1) - \dot{U}_\text{in}
\]  
The mass flow rate is expressed as:  
\[
\dot{m} = \frac{\dot{U}_\text{in}}{h_2 - h_1}
\]  

Values are calculated:  
\[
T_2 = T_i - 6 = 273.15 - 10 - 6 = 259.15 \, \text{K} = -76^\circ\text{C}
\]  
\[
h_2 = h(x=1, T=22^\circ\text{C}) = 234.08 \, \frac{\text{kJ}}{\text{kg}}
\]  
\[
x = 0.3385
\]  

TASK 4c  
The enthalpy difference is calculated:  
\[
h_2 - h_1 = h(5 \, \text{bar}, x=0) = 31.92 \, \frac{\text{kJ}}{\text{kg}}
\]  
The pressure is given as:  
\[
p_1 = p_2 = 1.957 \, \text{bar}
\]  

TASK 4d  
The coefficient of performance is expressed as:  
\[
\epsilon_K = \frac{\dot{Q}_K}{\dot{W}_K} = \frac{\dot{Q}_K}{\dot{Q}_\text{in} - \dot{Q}_\text{out}}
\]  

TASK 4e  
The text states:  
"The temperature would stabilize at a middle value."  

No additional calculations or explanations are provided.