TASK 3c  
The heat transferred \( Q \) is calculated using the change in internal energy:  
\[
Q = \Delta E = \Delta U = U_{g,2} - U_{g,1}
\]  
Using the specific heat capacity \( c_V \), the temperature difference \( T_2 - T_1 \), and the mass \( m \):  
\[
Q = c_V (T_2 - T_1) m = 0.635 \, \frac{\text{kJ}}{\text{kg·K}} \cdot 3.92 \cdot 10^{-2} \, \text{kg} \cdot (500 \, \text{K})
\]  
The result is:  
\[
|Q_{12}| = 1.08294 \, \text{kJ}
\]  

---

TASK 3d  
The heat transferred \( Q_{12} \) is given as \( 1500 \, \text{J} \) from the solid-liquid equilibrium.  

The pressure in the ice-water mixture \( p_{2,\text{EW}} \) is calculated as:  
\[
p_{2,\text{EW}} = p_{\text{amb}} + \frac{m_K \cdot g}{A} = 1.4 \, \text{bar}
\]  
(from Table 1).  

The initial ice mass fraction \( x_{\text{EW},1} \) is:  
\[
x_{\text{EW},1} = 0.6
\]  

The specific internal energy of the ice-water mixture \( U_{\text{EW},1} \) is calculated as:  
\[
U_{\text{EW},1} = x \cdot (-333.458 \, \frac{\text{kJ}}{\text{kg}}) + (1 - x) \cdot (-0.085 \, \frac{\text{kJ}}{\text{kg}})
\]  
Substituting \( x = 0.6 \):  
\[
U_{\text{EW},1} = -20.093 \, \frac{\text{kJ}}{\text{kg}}
\]  

The internal energy at state 2 \( U_{\text{EW},2} \) is calculated as:  
\[
U_{\text{EW},2} = U_{\text{EW},1} + \Delta Q = U_{\text{EW},1} + Q_{12}
\]  
\[
U_{\text{EW},2} = -20.093 \, \frac{\text{kJ}}{\text{kg}} + 1.5 \, \frac{\text{kJ}}{\text{kg}} = -18.593 \, \frac{\text{kJ}}{\text{kg}}
\]  

The final ice fraction \( x_2 \) is determined using:  
\[
U_{\text{EW},2} = x_2 \cdot (-333.458 \, \frac{\text{kJ}}{\text{kg}}) + (1 - x_2) \cdot (-0.085 \, \frac{\text{kJ}}{\text{kg}})
\]  
Solving for \( x_2 \):  
\[
x_2 = 0.555
\]  

