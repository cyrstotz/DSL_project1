TASK 3a  
The temperature \( T_1 \) is calculated as:  
\[
T_1 = 500 + 273.15
\]  

Using the ideal gas law \( pV = mRT \), the mass of the gas \( m_g \) is determined:  
\[
m_g = \frac{p \cdot V}{R \cdot T} = \frac{1.4009 \, \text{bar} \cdot 3.14 \, \text{L}}{50 \, \text{kg/kmol} \cdot 8.314 \, \text{kJ/(kmol·K)} \cdot T_1}
\]  
This results in:  
\[
m_g = 3.4217 \, \text{g}
\]  

The pressure in the gas chamber is:  
\[
p_{g,1} = 1.4009 \, \text{bar}
\]  

TASK 3b  
The pressure in the gas remains constant because the piston weight continues to exert force. Therefore:  
\[
p_{g,2} = p_{g,1} = 1.4009 \, \text{bar}
\]  

The temperature of the gas in state 2 is equal to the temperature of the ice-water mixture:  
\[
T_{g,2} = T_{\text{EW},2} = 0^\circ\text{C}
\]  

TASK 3c  
The heat transferred \( Q_{12} \) is calculated using the specific heat capacity \( c_V \):  
\[
Q_{12} = m_g \cdot c_V \cdot \Delta T
\]  
Substituting values:  
\[
Q_{12} = 0.0034217 \, \text{kg} \cdot 0.633 \, \text{kJ/(kg·K)} \cdot (500 - 0) \, \text{K}
\]  
This results in:  
\[
Q_{12} = 1.083 \, \text{kJ}
\]  

TASK 3d  
The pressure in the ice-water mixture (EW) is calculated as:  
\[
p_{\text{EW}} = p_{\text{amb}} + \frac{m_K \cdot g}{A}
\]  
Where:  
\[
A = \left( \frac{10 \, \text{cm}}{2} \right)^2 \cdot \pi = 0.007853 \, \text{m}^2
\]  
Substituting values:  
\[
p_{\text{EW}} = 1 \, \text{bar} + \frac{32 \, \text{kg} \cdot 9.81 \, \text{m/s}^2}{0.007853 \, \text{m}^2}
\]  
This results in:  
\[
p_{\text{EW}} = 1.33937 \, \text{bar}
\]  

The final ice fraction \( x_{\text{ice},2} \) is determined using the equilibrium table for solid-liquid phase changes.  

Description of Diagram:  
The diagram shows a cylinder divided into two chambers. The top chamber contains the ice-water mixture (EW), while the bottom chamber contains the gas. A piston rests on the EW, exerting pressure downward. Forces acting on the piston are labeled, including atmospheric pressure and the weight of the piston.