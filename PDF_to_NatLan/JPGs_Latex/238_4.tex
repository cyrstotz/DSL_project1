TASK 3a  
The gas pressure \( p_{g,1} \) is calculated as the sum of the ambient pressure \( p_{\text{amb}} \) and the pressure exerted by the piston \( \frac{F_g}{A} \):  
\[
p_{g,1} = p_{\text{amb}} + \frac{F_g}{A}
\]  
The force exerted by the piston is given as:  
\[
F_g = 32 \, \text{kg} \cdot 9.81 \, \text{m/s}^2 = 313.92 \, \text{N}
\]  
The cross-sectional area of the cylinder is calculated using the diameter \( D = 10 \, \text{cm} \):  
\[
A = \pi \left( \frac{D}{2} \right)^2 = \pi \left( \frac{0.1 \, \text{m}}{2} \right)^2 = 0.007853982 \, \text{m}^2
\]  
Substituting these values:  
\[
p_{g,1} = 1 \, \text{bar} = 10^5 \, \text{Pa} + \frac{313.92 \, \text{N}}{0.007853982 \, \text{m}^2} = 10^5 \, \text{Pa} + 39963.59 \, \text{Pa} = 1.4 \, \text{bar}
\]  

To determine the mass of the gas \( m_g \), the ideal gas law is applied:  
\[
m_g = \frac{p_{g,1} V_{g,1}}{R T_{g,1}}
\]  
Given:  
\[
p_{g,1} = 1.4 \, \text{bar}, \, V_{g,1} = 3.14 \cdot 10^{-3} \, \text{m}^3, \, T_{g,1} = 773.15 \, \text{K}
\]  
The specific gas constant \( R \) is calculated as:  
\[
R = \frac{R_{\text{univ}}}{M_g} = \frac{8.314 \, \text{J/mol·K}}{50 \, \text{kg/kmol}} = 0.16625 \, \text{J/g·K}
\]  
Substituting into the ideal gas law:  
\[
m_g = \frac{1.4 \, \text{bar} \cdot 3.14 \cdot 10^{-3} \, \text{m}^3}{0.16625 \, \text{J/g·K} \cdot 773.15 \, \text{K}} = 0.00342 \, \text{kg}
\]  
Thus, the mass of the gas is:  
\[
m_g = 3.42 \, \text{g}
\]  

TASK 3b  
The pressure of the gas has not changed because no new components have been added to exert additional pressure. Therefore:  
\[
p_{2,g} = p_{1,g} = 1.4 \, \text{bar}
\]  

If, in state 2, no heat is transferred between the gas and the ice-water mixture (EW), there can be no temperature difference between the gas and the EW. This implies:  
\[
T_{2,g} = 0^\circ\text{C}
\]  