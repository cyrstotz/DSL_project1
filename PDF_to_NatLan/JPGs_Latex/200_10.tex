TASK 9a  
A pressure-temperature (\( p \)-\( T \)) diagram is drawn to represent the freeze-drying process. The diagram includes the following features:  
- The \( p \)-axis is labeled in millibar, and the \( T \)-axis is labeled in degrees Celsius (\( ^\circ \text{C} \)).  
- The diagram shows phase regions labeled as "Fest" (solid), "Flüssig" (liquid), and "Gas" (gas).  
- An isobaric line is drawn at 5 mbar, and an isothermal line is drawn at a temperature difference of 10 K.  
- Points labeled "1", "2", and "3" indicate states in the process.  
- The transitions between states are marked as isothermal and isobaric processes.  

TASK 9b  
The refrigerant mass flow rate \( \dot{m}_{\text{R134a}} \) is calculated using the following equation:  
\[
\dot{W}_K = 28 \, \text{W}
\]  

From Table A-10, the properties of R134a at state 2 are given:  
- Pressure \( p = 7.5748 \, \text{bar} \)  
- Temperature \( T_2 = T_i - 6^\circ \text{C} \)  
\[
T_i = -10^\circ \text{C}, \quad T_2 = -76^\circ \text{C}
\]  

The entropy at state 2 is:  
\[
S_g = 0.9298 \, \frac{\text{kJ}}{\text{kg·K}}
\]  

The energy balance equation is written as:  
\[
0 = \dot{m} (h_e - h_a) + \dot{W}_K
\]  

Rearranging for \( \dot{m} \):  
\[
\dot{W}_K = \dot{m} \frac{h_a}{h_3 - h_2}
\]  

Where \( h_3 \) and \( h_2 \) are enthalpies at states 3 and 2, respectively.