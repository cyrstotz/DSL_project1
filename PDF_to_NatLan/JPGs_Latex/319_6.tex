TASK 4a  
Two diagrams are drawn to represent the freeze-drying process in a pressure-temperature (\(p\)-\(T\)) diagram.  

1. The first diagram shows phase regions labeled as "solid," "liquid," and "gas." The lines separating these regions represent phase boundaries. The triple point is marked, and the sublimation line is highlighted. The diagram includes arrows indicating transitions between states.  

2. The second diagram zooms into the sublimation process, showing the transition from solid to gas. The labels indicate "fast," "slow," and "missing" sublimation rates.  

TASK 4b  
The energy balance equation for the refrigerant is written as:  
\[
\frac{dE}{dt} = \sum \dot{m}_i (h_i + ke_i + pe_i) + \sum \dot{Q} - \sum \dot{W}
\]  
For the specific case, the equation simplifies to:  
\[
0 = \dot{m}_{\text{R134a}} (h_2 - h_3) + 28 \, W
\]  
The enthalpy at state 2 (\(h_2\)) is calculated using the relation:  
\[
h_2 = h_3 (T = T_i - 6 \, \text{K})
\]  
Interpolation is required for \(h_3\) using the refrigerant tables.  

TASK 4c  
The pressure at state 2 (\(p_2\)) is equal to the pressure at state 3 (\(p_3\)), which is given as:  
\[
p_3 = p_2 = 8 \, \text{bar}
\]  
The energy balance equation is written as:  
\[
\frac{dE}{dt} = \sum \dot{m}_i (h_i) + \sum \dot{Q} - \sum \dot{W}
\]  
Simplified for this case:  
\[
0 = \dot{m} (h_1 - h_2) - \dot{W}
\]  

TASK 4d  
The coefficient of performance (\(\epsilon_K\)) is defined as:  
\[
\epsilon_K = \frac{\dot{Q}_K}{\dot{W}_K}
\]  
This is further expressed as:  
\[
\epsilon_K = \frac{\dot{m} (h_2 - h_1)}{\dot{W}_K}
\]  

A small diagram is drawn showing a curve in the \(p\)-\(T\) space, with annotations indicating the phase transitions and a marked point labeled as "state 1."  

Descriptions of diagrams are provided in detail above.