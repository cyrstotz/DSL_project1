TASK 4a  
Two diagrams are drawn:  
1. The first diagram is a pressure-temperature (\(p\)-\(T\)) plot with shaded regions. The axes are labeled \(p\) (pressure) and \(T\) (temperature). The shaded region represents the phase change area.  
2. The second diagram is also a \(p\)-\(T\) plot, showing the phase regions labeled "Flüssig" (liquid), "Gas" (gas), and "Nass-Dampf" (wet steam). Points 1, 2, 3, and 4 are marked on the diagram, representing different states in the cycle.  

TASK 4b  
The entropy at state 2 is equal to the entropy at state 3:  
\[
S_2 = S_3
\]  
The energy balance equation is given as:  
\[
0 = \dot{m}(h_2 - h_3) - \dot{W}_K
\]  
Additional notes mention \(T_i\) and \(p = 1 \, \text{mbar}\).  

TASK 4c  
The enthalpy at state 1 is equal to the enthalpy at state 4:  
\[
h_1 = h_4
\]  

TASK 4e  
The text states that the exergy loss would increase significantly ("Sie würde größer, da mehr Exergieverlust").  

TASK 4d  
The coefficient of performance (\(\epsilon_K\)) is defined as:  
\[
\epsilon_K = \frac{\dot{Q}_K}{\dot{W}_K}
\]  