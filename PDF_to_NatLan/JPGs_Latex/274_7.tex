TASK 4d  
The coefficient of performance \( \epsilon_K \) is calculated using the formula:  
\[
\epsilon_K = \frac{\dot{Q}_K}{\dot{W}_K} = \frac{\dot{Q}_K}{\dot{Q}_{ab} - \dot{Q}_K} = \frac{\dot{Q}_K}{\dot{Q}_{ab} - \dot{Q}_x}
\]

The heat removed by the refrigerant is given by:  
\[
\dot{Q}_K = \dot{m}_{R134a} \cdot (h_2 - h_1)
\]  
Substituting values:  
\[
\dot{Q}_K = 205.79 \, \text{kJ}
\]

The heat added during condensation is:  
\[
\dot{Q}_{ab} = \dot{m}_{R134a} \cdot (h_4 - h_3)
\]  
Substituting values:  
\[
\dot{Q}_{ab} = 156.29 \, \text{kJ}
\]

Finally, the coefficient of performance is:  
\[
\epsilon_K = \frac{\dot{Q}_K}{\dot{W}_K} = 7349.64
\]

TASK 4e  
If the cooling cycle from Step i continued with constant \( \dot{Q}_K \), the temperature \( T_i \) would not remain above the sublimation point. As a result, the food would not freeze-dry properly.