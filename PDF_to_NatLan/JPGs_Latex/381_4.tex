TASK 2a  
The diagram represents a qualitative \( T \)-\( s \) (temperature-entropy) plot for the jet engine process. The axes are labeled as follows:  
- The vertical axis is \( T \) (temperature in Kelvin).  
- The horizontal axis is \( s \) (specific entropy in \( \text{kJ}/\text{kg·K} \)).  

The process includes the following states and transitions:  
1. Starting at state 1, an isentropic compression occurs, moving vertically upwards to state 2.  
2. From state 2 to state 3, heat is added isobarically, increasing entropy.  
3. From state 3 to state 4, an isentropic expansion occurs, moving vertically downwards.  
4. From state 4 to state 5, heat is removed isobarically, decreasing entropy.  
5. The final state (state 6) is marked as an isobaric process at \( p_6 = p_0 \).  

The diagram also includes annotations for isentropic and isobaric processes, with specific pressure values labeled (e.g., \( 0.5 \, \text{bar} \), \( 1.5 \, \text{bar} \)).  

---

TASK 2b  
The energy balance for the process from state 5 to state 6 is written as:  
\[
0 = \dot{m}_{\text{ges}} \left( h_5 - h_6 \right) + \frac{w_5^2 - w_6^2}{2} + \dot{Q}  
\]
This is simplified using the specific heat capacity \( c_p \) and temperature difference:  
\[
0 = m_f \left( c_p \left( T_5 - T_6 \right) \right) + \frac{w_5^2 - w_6^2}{2}  
\]

For an ideal gas, the relationship between pressures and temperatures is given by:  
\[
\left( \frac{p_6}{p_5} \right)^{\frac{k-1}{k}} = \frac{T_6}{T_5}  
\]
Rearranging for \( T_6 \):  
\[
T_6 = T_5 \left( \frac{p_6}{p_5} \right)^{\frac{k-1}{k}}  
\]

Substituting values:  
\[
T_6 = 328.07 \, \text{K}  
\]  

The equation also includes the kinetic energy term:  
\[
\frac{w_6^2}{2}  
\]  
but no further calculation is shown for this term.