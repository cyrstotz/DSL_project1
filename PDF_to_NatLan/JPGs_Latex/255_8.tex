TASK 4b  
The pressure at state 1 is equal to the pressure at state 2:  
\[
p_1 = p_2
\]  
The temperature difference between states 1 and 2 is given as:  
\[
T = 6 \, \text{K} = T_1 - T_2
\]  

Using the first law of thermodynamics for the compressor, the power input is:  
\[
\dot{W}_K = 295 \, \text{W}
\]  

The mass flow rate of the refrigerant is calculated using:  
\[
\dot{m}_{\text{R134a}} = \frac{-W_K}{h_2 - h_3}
\]  

The enthalpy values are:  
\[
h_2 = 247.53 \, \text{kJ/kg} \quad \text{(from Table A-10)}
\]  
\[
h_3 = \text{value from saturated liquid data at state 3}
\]  

TASK 4c  
The entropy at state 2 is equal to the entropy at state 3:  
\[
s_2 = s_3
\]  
From the saturated liquid data:  
\[
s_3 = s_f + x_3 (s_g - s_f)
\]  
The vapor quality at state 3 is calculated as:  
\[
x_3 = \frac{s_3 - s_f}{s_g - s_f}
\]  

The enthalpy at state 3 is given by:  
\[
h_3 = h_f + x_3 (h_g - h_f)
\]  

TASK 4b (continued)  
The mass flow rate of the refrigerant can also be expressed as:  
\[
\dot{m}_{\text{R134a}} = \frac{W_K}{h_4 + x_3 (h_g - h_f) - h_2}
\]  

No diagrams or graphs are present on the page.