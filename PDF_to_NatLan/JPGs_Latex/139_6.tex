TASK 3a  
The ideal gas law is applied to calculate the pressure and mass of the gas in state 1:  
\[
\rho V = mRT
\]  
The mass of the gas is calculated as:  
\[
m_{\text{gas}} = \frac{p V_{g,1}}{R T_{g,1}} = \frac{\frac{F_K}{A} \cdot V_{g,1}}{R T_{g,1}}
\]  
Substituting the values:  
\[
m_{\text{gas}} = \frac{\frac{32.1 \, \text{kg} \cdot 9.81 \, \text{m/s}^2}{0.005 \, \text{m}^2} \cdot 3.14 \, \text{L}}{50 \cdot (500 + 273.15)} = 9.982539 \, \text{g}
\]  
Approximated as:  
\[
m_{\text{gas}} \approx 9.983 \, \text{g}
\]  

The pressure is calculated as:  
\[
p = \frac{F_K + F_{\text{EW}}}{A} = \frac{4087.059939 \, \text{N}}{0.005 \, \text{m}^2} \approx 4087.059939 \, \text{Pa}
\]  

TASK 3b  
The membrane is a heat conductor, ensuring that the gas and EW (ice-water mixture) reach the same temperature. The same applies to the pressure, as the membrane is frictionless.  

The pressure in state 2 is calculated as:  
\[
p_2 = \frac{F_K}{A} = \frac{32 \, \text{kg} \cdot 9.80665 \, \text{m/s}^2}{0.005 \, \text{m}^2} = 3955.59847 \, \text{N/m}^2
\]  
Approximated as:  
\[
p_2 \approx 3955.6 \, \text{N/m}^2
\]  

TASK 3c  
The energy balance is applied:  
\[
\Delta E = \Sigma Q - \Sigma W_n
\]  
Where:  
\[
E = U + KE + PE
\]  
Since kinetic and potential energy are negligible:  
\[
\Delta U = Q_{12} - W_n
\]  
Thus:  
\[
Q_{12} = \Delta U + W_n
\]  

TASK 3d  
The total energy is expressed as:  
\[
E = U + KE + PE
\]  

No diagrams or additional figures are present on the page.