TASK 4b  
The refrigerant temperature \( T_i \) is given as \( -8^\circ\text{C} \).  
The temperature \( T_2 \) is equal to \( T_i \).  

The enthalpy \( h_{12} \) is interpolated from Table A-10:  
\[
h_{12} = \frac{h_g(-8^\circ\text{C}) - h_g(-12^\circ\text{C})}{-8^\circ\text{C} - (-12^\circ\text{C})} \cdot (-8^\circ\text{C} - (-10^\circ\text{C})) + h_g(-12^\circ\text{C})
\]  
Values are substituted into the equation:  
\[
h_{12} = \frac{264.78 - 243.73}{4} \cdot 2 + 243.73
\]  

The mass flow rate \( \dot{m}_{\text{R134a}} \) is calculated using the formula:  
\[
\dot{m}_{\text{R134a}} = \frac{\dot{Q}_K}{h_{12}}
\]  
From Table A-12, \( h_{12} = 264.15 \). Substituting values:  
\[
\dot{m}_{\text{R134a}} = \frac{28 \, \text{kW}}{264.15 - 243.73} = \frac{28}{20.42} = 1.37 \, \text{kg/s}
\]  

TASK 4c  
The vapor quality \( x_1 \) is calculated using the formula:  
\[
x = \frac{h - h_f}{h_g - h_f}
\]  

TASK 4d  
The coefficient of performance \( \epsilon_K \) is calculated using:  
\[
\epsilon_K = \frac{\dot{Q}_K}{\dot{W}_K} = \frac{\dot{Q}_K}{\dot{Q}_K - \dot{Q}_{\text{cond}}}
\]  

TASK 4e  
The pressure would continuously decrease, and sublimation would stop. Eventually, the water would reach the triple point and no longer escape from the food.  

No figures or diagrams are present on the page.