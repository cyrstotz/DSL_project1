TASK 3c  

The change in energy is expressed as:  
\[
\Delta E = E_2 - E_1
\]  
This assumes no kinetic or potential energy contributions.  

The energy balance is given by:  
\[
\Delta E = U_1 - U_2 = Q - W
\]  

TASK 3b  

The pressure \( p_2 \) is equal to \( p_1 \), as the pressure is determined by \( p_0 \) (ambient pressure), and the mass does not change.  

The temperature ratio \( \frac{T_2}{T_1} \) is used to calculate the pressure ratio:  
\[
\frac{p_2}{p_1} = \left( \frac{T_2}{T_1} \right)^{\frac{R}{c_v}}
\]  
where \( R \) is the gas constant and \( c_v = 0.633 \, \text{kJ/kg·K} \).  

The specific heat ratio \( n \) is calculated as:  
\[
n = \frac{c_p}{c_v} = 7.2637
\]  

TASK 3d  

The energy balance is expressed as:  
\[
\Delta E = E_2 - E_1
\]  
This assumes no kinetic or potential energy contributions.  

The energy balance is also written as:  
\[
\Delta E = U_2 - U_1 = Q - W
\]  

The work done by the system is calculated using the formula for volumetric work:  
\[
W = m \cdot \left( \frac{1}{n-1} \cdot (p_2 v_2 - p_1 v_1) \right)
\]  
where \( v_2 - v_1 \) represents the change in specific volume.  

No further numerical results are provided.  

No diagrams or graphs are present on the page.