TASK 3b  
The ice fraction \( x \) must remain greater than zero to ensure equilibrium between the ice and water phases.  

The temperature \( T_{g,2} \) is given as \( 0.003^\circ\text{C} \), and the pressure \( p_{g,2} \) is constant at \( 140 \, \text{kPa} \). This pressure is sufficient to maintain the conditions for equilibrium.  

---

TASK 3c  
The energy balance is expressed as:  
\[
\Delta E = Q - W
\]  
Since there is no work done, this simplifies to:  
\[
\Delta E = \Delta U
\]  

The change in internal energy is given by:  
\[
\Delta U = m_2 u_2 - m_1 u_1
\]  
or equivalently:  
\[
\Delta U = m (u_2 - u_1)
\]  

The relevant temperatures are:  
\[
T_1 = 273.15 \, \text{K}, \quad T_2 = 273.183 \, \text{K}
\]  

The specific internal energy at state 2 is:  
\[
u_2 = -333.401 \, \text{kJ/kg}
\]  

---

TASK 3d  
To determine the final ice fraction \( x_2 \), the following equation is used:  
\[
x_2 = \frac{u - u_f}{u_i - u_f}
\]  

Given:  
\[
T_{g,2} = 0.003^\circ\text{C}, \quad p_{g,2} = 1.5 \, \text{bar}, \quad m_g = 3.6 \, \text{g}
\]  

The specific internal energy values are:  
\[
u = -333.401 \, \text{kJ/kg}, \quad u_f = 0, \quad u_i = -333.492 \, \text{kJ/kg}
\]  

Substituting into the equation:  
\[
x_2 = \frac{-333.401 - 0}{-333.492 - 0} = 0.999877
\]  

Thus, the final ice fraction \( x_2 \) is approximately \( 0.999877 \).