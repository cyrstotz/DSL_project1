TASK 4a  
A p-T diagram is drawn to represent the freeze-drying process. The diagram includes the following labeled points and processes:  
- Point 1: The starting state.  
- Point 2: Isobaric evaporation.  
- Point 3: Reversible adiabatic compression.  
- Point 4: Isobaric condensation.  
The diagram also shows the triple point of the refrigerant and the phase boundaries. The x-axis represents temperature \( T \) (in degrees Celsius), and the y-axis represents pressure \( p \). The processes are labeled as "isobar" and "adiabat" to indicate the thermodynamic paths.

---

TASK 4b  
The stationary flow process is described with the following energy balance equation:  
\[
0 = \dot{m} (h_2 - h_3) + \dot{Q} - \dot{W}
\]  
Rearranging for the mass flow rate \( \dot{m} \):  
\[
\dot{m} = \frac{\dot{W}_u}{h_2 - h_3} = \frac{\dot{W}_u}{h_c - h_a}
\]  

Values for enthalpy are provided using Table A-12:  
- Saturated \( h_3 \) at 8 bar: \( h_3 = 264.15 \, \text{kJ/kg} \).  
- Saturated \( h_2 \) at 231.16 kPa: \( h_2 = 231.35 \, \text{kJ/kg} \).  

The calculation for \( \dot{m} \) is left incomplete.