TASK 3a  
The specific gas constant \( R \) is calculated as:  
\[
R = \frac{\bar{R}}{M} = \frac{83.49}{50} = 166.98 \, \text{J/kg·K}
\]

The pressure in the ice-water mixture (EW) is determined as:  
\[
p_{\text{EW}} = p_0 + F \cdot A = 1 \, \text{bar} + m \cdot g \cdot \left(\frac{4}{\pi D^2}\right) = 1.247 \, \text{bar}
\]  
Thus, the pressure of the gas in state 1 is equal to the pressure of the EW:  
\[
p_{g,1} = p_{\text{EW}} = 1.247 \, \text{bar}
\]

The ideal gas law is applied:  
\[
pV = mRT
\]  
Substituting the known values:  
\[
V_1 = 3.19 \cdot 10^{-3} \, \text{m}^3, \quad T_1 = 773.15 \, \text{K}
\]  
The mass of the gas is calculated as:  
\[
m_g = \frac{p_1 V_1}{R T_1} = 0.037126 \, \text{kg}
\]

---

TASK 3b  
Since the given conditions do not change, the piston still exerts a mass of \( 32 \, \text{kg} \) and 1 bar atmospheric pressure. Therefore, the gas pressure in state 2 remains the same:  
\[
p_{g,2} = p_{g,1} = 1.247 \, \text{bar}
\]

The energy balance equation is written as:  
\[
\Delta E = 0 = m_{\text{EW}} (u_{2g} - u_{1g}) + m_g (u_2 - u_1)
\]  
Substituting for internal energy changes:  
\[
\Delta E = m_{\text{EW}} (u_{2g} - u_{1g}) + m_g \cdot c_V (T_2 - T_1)
\]