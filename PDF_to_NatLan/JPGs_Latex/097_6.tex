TASK 3a  
The mass of the gas \( m \) is calculated using the ideal gas law:  
\[
m = \frac{p V}{R T}
\]  
Substituting the given values:  
\[
R = 162.28 \, \text{J/(kg·K)}, \, T = 773.15 \, \text{K}, \, p = 1 \, \text{bar} = 10^5 \, \text{Pa}, \, V = 3.14 \times 10^{-3} \, \text{m}^3
\]  
The calculation yields:  
\[
m = 0.00244 \, \text{kg} \quad \text{or} \quad m = 0.1402 \, \text{kg} \, \text{(crossed out)}
\]  

TASK 3b  
The final temperature \( T_{g,2} \) and pressure \( p_{g,2} \) are determined as follows:  
\[
T_{g,2} = T_{\text{EW}}
\]  
This implies that the gas temperature in state 2 equals the equilibrium temperature of the ice-water mixture.  
\[
p_{g,2} = p_{\text{EW}}
\]  
This indicates that the gas pressure in state 2 equals the pressure exerted by the ice-water mixture.  

Additional notes:  
The equation \( \dot{p} V = \dot{m} R T \) is written but not further elaborated.  

No diagrams or graphs are present.