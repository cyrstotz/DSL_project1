TASK 3a  
The gas pressure \( p_{g,1} \) is calculated using the following formula:  
\[
p_{g,1} = p_{g,2} + p_{\text{EW}} + p_{\text{amb}}
\]  
where:  
\[
p_{g,2} = \frac{m_g \cdot g}{A} + \frac{m_{\text{EW}} \cdot g}{A} + p_{\text{amb}}
\]  
The piston area \( A \) is given as \( A = 0.008 \, \text{m}^2 \).  

The calculation yields:  
\[
p_{g,1} = 1.4 \, \text{bar}
\]  

The gas mass \( m_g \) is determined using the equation:  
\[
m_g = \frac{p_{g,1} \cdot V_{g,1}}{R \cdot T_{g,1}}
\]  
Substituting values:  
\[
m_g \approx 0.0034 \, \text{kg} = 3.4 \, \text{g}
\]  

---

TASK 3b  
The temperature \( T_{g,2} \) is given as:  
\[
T_{g,2} = 0^\circ\text{C}
\]  

Explanation: Not all the ice has melted, and yet both chambers are in thermodynamic equilibrium. This means the EW has a temperature of \( 0^\circ\text{C} \), and the gas does as well.  

The pressure \( p_{g,2} \) is calculated as:  
\[
p_{g,2} = p_{g,1} = 1.4 \, \text{bar}
\]  

Further explanation: The atmospheric pressure and the weight force of the piston and EW act on the same area. The EW remains constant in density, so the pressure also remains constant.