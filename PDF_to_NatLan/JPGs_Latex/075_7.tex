TASK 4a  
A graph is drawn with pressure \( P \) on the vertical axis and temperature \( T \) on the horizontal axis. The diagram represents the freeze-drying process. Key features include:  
- A labeled region for "gas" at higher temperatures and lower pressures.  
- A point labeled "10 K" indicating a temperature difference.  
- A pressure of "3 mbar" marked near the sublimation region.  
- Phase regions are indicated, with arrows showing transitions between states.  
- The axes are labeled, and the phase boundaries are sketched qualitatively.  

TASK 4b  
The refrigerant mass flow rate \( \dot{m}_{\text{R134a}} \) is mentioned. A small diagram is drawn showing heat flow \( \dot{Q}_K \) into state 2.  
The equation for the thermodynamic mean temperature \( \bar{T}_v \) is partially written but incomplete.  

TASK 4d  
The coefficient of performance \( \epsilon_K \) is calculated using the formula:  
\[
\epsilon_K = \frac{\dot{Q}_K}{\dot{W}_K}
\]  
An alternative form is shown:  
\[
\epsilon_K = \frac{\dot{Q}_K}{Q_{\text{ob}} - Q_{\text{in}}}
\]  

Additional notes:  
- A table is partially filled with columns labeled \( T \), \( p \), \( h \), and \( S \) for states 1 through 4.  
- Some entries are left blank, while pressures for states 3 and 4 are noted as "8 bar."  

No further content is legible or complete.