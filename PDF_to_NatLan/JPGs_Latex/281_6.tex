TASK 3a  
The problem involves determining the gas pressure \( p_{g,1} \) and mass \( m_g \) in state 1.  

For a perfect gas, the relationship is given as:  
\[
p_1 V_1 = m_g R T_1
\]  

Using the equilibrium of forces:  
\[
\Delta p_{\text{amb}} + m_{\text{piston}} g + m_{\text{EW}} g = p_{g,1} A
\]  
where \( A = \pi r_{\text{cylinder}}^2 \).  

The pressure calculation proceeds as follows:  
\[
p_{g,1} = p_{\text{amb}} + \frac{32 \cdot 9.81 + 0.1 \cdot 9.81}{\pi \cdot (5 \cdot 10^{-2})^2}
\]  
Substituting values:  
\[
p_{g,1} = 1 \cdot 10^5 + \frac{32 \cdot 9.81 + 0.1 \cdot 9.81}{\pi \cdot (5 \cdot 10^{-2})^2}
\]  
\[
p_{g,1} = 1.40094 \, \text{bar}
\]  

Next, the gas mass \( m_g \) is calculated using:  
\[
m_g = \frac{p_1 V_1}{R T_1}
\]  
where \( R = \frac{8.314}{50 \cdot 10^{-3}} \).  

Substituting values:  
\[
m_g = \frac{1.400 \cdot 10^5 \cdot 3.14 \cdot 10^{-3}}{8.314 / 50 \cdot 10^{-3} \cdot 773.15}
\]  
\[
m_g = 3.421 \, \text{g}
\]  

TASK 3b  
The next part of the solution is on the following page.  

No diagrams or figures are present.