TASK 2a  
The page contains three diagrams related to thermodynamic processes. Below is a description of each diagram:

1. **First Diagram**:  
   - The graph is a \( T \)-\( s \) diagram (temperature vs. entropy).  
   - The curve starts at a high temperature and entropy, rises to a peak, and then decreases steadily.  
   - The axes are labeled: \( T \) (temperature) on the vertical axis and \( s \) (entropy) on the horizontal axis.  
   - The entropy axis is marked with "in \( \frac{\text{kJ}}{\text{kg}} \)".

2. **Second Diagram**:  
   - This is another \( T \)-\( s \) diagram, but it shows a stepwise process with four distinct states labeled \( 1 \), \( 2 \), \( 3 \), and \( 4 \).  
   - The process begins at state \( 1 \), moves vertically to state \( 2 \), then horizontally to state \( 3 \), and finally descends diagonally to state \( 4 \).  
   - The axes are labeled \( T \) (temperature) on the vertical axis and \( s \) (entropy) on the horizontal axis.

3. **Third Diagram**:  
   - This is a detailed \( T \)-\( s \) diagram with six states labeled \( 1 \), \( 2 \), \( 3 \), \( 4 \), \( 5 \), and \( 6 \).  
   - The process begins at state \( 1 \), rises vertically to state \( 2 \), moves horizontally to state \( 3 \) (marked "isobar"), and then descends through states \( 4 \), \( 5 \), and \( 6 \).  
   - The axes are labeled \( T \) (temperature) on the vertical axis and \( s \) (entropy) on the horizontal axis.  
   - The entropy axis is marked with "in \( \frac{\text{kJ}}{\text{kg}} \)".  
   - A temperature value \( T_5 = 431.94 \, \text{K} \) is noted near state \( 5 \).

No additional textual explanations or calculations are visible on the page.