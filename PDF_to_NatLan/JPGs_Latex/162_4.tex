TASK 3a  
For a perfect gas, the mass \( M \) is calculated using the ideal gas law:  
\[
M = \frac{\rho V}{RT}
\]  
Substituting the given values:  
\[
M = \frac{\rho V_{g,1}}{RT} = 3.42 \, \text{kg}
\]  
The gas constant \( R \) is determined as:  
\[
R = \frac{R_u}{M} = 0.166 \, \text{kJ/(kg·K)}
\]  

The pressure exerted by the piston is calculated as:  
\[
F_g + p_{\text{amb}} A = p_g A
\]  
Where:  
\[
p_g = \frac{F_g}{A} + p_{\text{amb}}
\]  
Substituting values:  
\[
F_g = 9.81 \cdot (m_K + m_{\text{EW}}) = 314.901 \, \text{N}
\]  
\[
A = 0.00785 \, \text{m}^2
\]  
\[
p_g = \frac{314.901}{0.00785} + 1 \, \text{bar} = 1.46 \, \text{bar}
\]  

A small sketch is included showing a piston exerting pressure on a gas chamber.  

---

TASK 3b  
The volume \( V_1 \) is equal to \( V_2 \):  
\[
V_1 = V_2 = V_m = 0.0314 \, \text{m}^3
\]  
The temperature ratio \( \frac{T_2}{T_1} \) is used to calculate the vapor quality \( X \):  
\[
X = \frac{V_{g,2} - V_{f} C_{T_2}}{V_{g,1} T_{f} C_{T_2}}
\]  
The equation for energy balance is:  
\[
A + U = Q \implies m (u_2 - u_1) = m_{\text{phase}} (T_2 - T_1)
\]  

---

TASK 3c  
The energy change is expressed as:  
\[
\Delta E = \dot{m} \cdot \dot{Q} + \dot{W}
\]  
The heat transfer \( Q_{12} \) is calculated as:  
\[
Q_{12} = m_{\text{EW}} \cdot m \cdot c_p (T_2 - T_1) = 1899 \, \text{J}
\]  
The specific heat capacity \( c_p \) is given as:  
\[
c_p = R + c_v = 0.79928 \, \text{kJ/(kg·K)}
\]  

---

TASK 3d  
The energy change is expressed as:  
\[
\Delta E = \dot{Q} \implies m_2 u_2 - m_1 u_1 = \dot{Q}
\]  
The internal energy \( u_2 \) is calculated as:  
\[
u_2 = u_f + X (u_{g} - u_f)
\]  
Substituting into the equation:  
\[
m_2 (u_f + X (u_{g} - u_f)) = m_1 u_1 + \dot{Q}
\]  
Rearranging for \( X \):  
\[
X = \frac{m_1}{m_2} \cdot \frac{1}{(u_{g} - u_f)} \left( m_1 u_1 + \dot{Q} - u_f \right)
\]  

The equation for \( u_g \) is left incomplete.  

---  
Descriptions of diagrams:  
- A small sketch of a piston exerting pressure on a gas chamber is included. It shows the mass \( m \), pressure \( p \), and area \( A \).  

