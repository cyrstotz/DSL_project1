TASK 4a  
The diagram is a pressure-temperature (\( p \)-\( T \)) graph showing the freeze-drying process.  
- The graph includes labeled phase regions: "Gas", "Absorption", and "Desorption".  
- The temperature axis (\( T \)) ranges from \(-50^\circ\text{C}\) to \( 0^\circ\text{C}\), with increments of \( 10^\circ\text{C} \).  
- A point labeled "1" is marked at \( 0^\circ\text{C} \), and another point labeled "2" is marked at \(-10^\circ\text{C}\).  
- The graph indicates a temperature difference of \( 10 \, \text{K} \) and a pressure reduction of \( 5 \, \text{mbar} \) below the triple point of water.  

TASK 4b  
The required refrigerant mass flow rate (\( \dot{m}_{\text{R134a}} \)) is to be calculated.  

The temperature in the evaporator is given as \( T_i - 6 \, \text{K} \).  
From the diagram:  
\[
T_i = -10^\circ\text{C}, \quad T_f = -16^\circ\text{C}
\]  

Using \( x_2 = n \), entropy and enthalpy values are obtained from Table A-10:  
\[
s_g(-16^\circ\text{C}) = 0.5298 \, \frac{\text{kJ}}{\text{kg·K}}, \quad h_g(-16^\circ\text{C}) = 237.74 \, \frac{\text{kJ}}{\text{kg}}
\]  

Since the compressor is adiabatic and reversible, the entropy remains constant:  
\[
S_2 = S_3
\]