TASK 2a  
The diagram represents a qualitative \( T \)-\( s \) (temperature-entropy) diagram for the jet engine process. The axes are labeled as follows:  
- The vertical axis represents temperature \( T \) in Kelvin (K), with specific values marked at 1289, 431.9, 322.08, and 243.15 K.  
- The horizontal axis represents entropy \( s \) in \( \text{kJ/kg·K} \).  

The process is divided into several states:  
- State 0 is the ambient condition.  
- State 1 is the inlet condition after compression.  
- State 2 is the bypass stream.  
- State 3 represents the combustion process, with a sharp increase in temperature.  
- State 4 is the turbine outlet.  
- State 5 is the mixing chamber outlet, and state 6 is the nozzle exit.  

Two pressure levels are indicated:  
- \( P = P_0 \), representing ambient pressure.  
- \( P = P_2 \), representing the higher pressure after compression.  

The curves are color-coded:  
- The magenta curve represents the high-pressure process.  
- The blue curve represents the low-pressure process.  

TASK 2c  
The equation for the mass-specific increase in flow exergy is given as:  
\[
\dot{E}_{\text{exergy}} = \dot{m} \left[ h_e - h_a - T_0 (s_e - s_a) \right]
\]  
Where:  
- \( \dot{E}_{\text{exergy}} \) is the exergy flow rate.  
- \( \dot{m} \) is the mass flow rate.  
- \( h_e \) and \( h_a \) are the specific enthalpies at the exit and ambient conditions, respectively.  
- \( T_0 \) is the ambient temperature.  
- \( s_e \) and \( s_a \) are the specific entropies at the exit and ambient conditions, respectively.