TASK 3a  
The gas pressure \( p_{g,1} \) is calculated as:  
\[
p_{g,1} = p_0 + \frac{(m_K + m_{\text{EW}}) \cdot g}{\pi \left( \frac{D}{2} \right)^2} = 1.4 \, \text{bar}
\]  

The mass of the gas \( m_g \) is determined using the ideal gas law:  
\[
m_g = \frac{p_{g,1} \cdot V_{g,1}}{R \cdot T_{g,1}} = 5.29 \, \text{g}
\]  

The specific gas constant \( R \) is calculated as:  
\[
R = \frac{\bar{R}}{M_g} = 166.28 \, \frac{\text{J}}{\text{kg·K}}
\]  

No further explanation is provided for the crossed-out text.  

---

TASK 3b  
The gas pressure \( p_{g,2} \) is equal to \( p_{g,1} \), as the mass acting on the cylinder remains unchanged:  
\[
p_{g,2} = p_{g,1}
\]  

The temperature \( T_{g,2} \) will cool toward \( 0^\circ\text{C} \), as heat exchange occurs through the membrane.  

No additional valid content is visible for this task.  