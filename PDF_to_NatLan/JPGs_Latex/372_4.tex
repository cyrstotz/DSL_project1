TASK 2a  
A graph is drawn representing a qualitative \( T \)-\( s \) diagram for the jet engine process. The axes are labeled as follows:  
- The vertical axis is \( T \) (temperature).  
- The horizontal axis is \( s \) (specific entropy) with units \( \frac{\text{kJ}}{\text{kg·K}} \).  

The diagram includes labeled points and processes:  
- Point \( O \) represents the ambient state.  
- Point \( 1 \) is connected to \( 2 \) via an adiabatic compression process.  
- Point \( 2 \) to \( 3 \) represents an isobaric heat addition process.  
- Point \( 3 \) to \( 4 \) is an adiabatic expansion process.  
- Point \( 4 \) to \( 5 \) represents a mixing process.  
- Point \( 5 \) to \( 6 \) is an isobaric process.  
- The pressures \( p_3 = p_2 \), \( p_4 = p_5 \), and \( p_6 = p_0 \) are indicated.  

TASK 2b  
The outlet velocity \( w_6 \) and temperature \( T_6 \) are calculated.  

The temperature \( T_6 \) is determined using the formula:  
\[
T_6 = T_5 \left( \frac{p_6}{p_5} \right)^{\frac{\kappa - 1}{\kappa}}
\]  
Substituting values:  
\[
T_6 = 431.9 \left( \frac{0.191}{0.5} \right)^{\frac{0.4}{1.4}} = 431.9 \cdot 0.101 = 328.075 \, \text{K}
\]  

The outlet velocity \( w_6 \) is calculated using the energy balance:  
\[
w_6 = \sqrt{2 \left( h_5 - h_6 + \frac{w_5^2}{2} \right)}
\]  

No further numerical values are provided for \( w_6 \).