TASK 4a  
The diagram is a pressure-temperature (\( p \)-\( T \)) plot illustrating the freeze-drying process.  
- The x-axis represents temperature (\( T \)) in Kelvin, and the y-axis represents pressure (\( p \)) in bar.  
- The curve shows the phase boundaries between different states of the refrigerant.  
- The left side of the curve is labeled "saturated liquid," while the right side is labeled "saturated vapor."  
- The region under the curve is labeled "wet steam" (Nassdampf).  
- The process includes the following steps:  
  - Adiabatic compression (state 1 to state 2).  
  - Isobaric heat transfer (state 2 to state 3).  
  - Adiabatic expansion (state 3 to state 4).  

TASK 4b  
The first law of thermodynamics is applied to the compressor:  
\[
0 = m (h_2 - h_3) + \dot{W}_K
\]  
Rearranging for the mass flow rate:  
\[
\dot{m} = \frac{\dot{W}_K}{h_2 - h_3}
\]  
Given:  
\[
\dot{W}_K = 28 \, \text{W}
\]  

For the enthalpy difference:  
- \( h_2 - h_3 \): The process is adiabatic and reversible, meaning it is isentropic. Thus, \( s_2 = s_3 \).  
- At \( p = 8 \, \text{bar} \), \( s_2 = s_3 = 0.506 \, \frac{\text{kJ}}{\text{kg·K}} \) (from Table A-11).  

The enthalpy at state 2 (\( h_2 \)) is:  
\[
h_2 = h_g(\text{8 bar}) = 264 \, \frac{\text{kJ}}{\text{kg}} \quad (\text{from Table A-11})
\]  

The enthalpy at state 3 (\( h_3 \)) is not explicitly calculated but depends on \( s_3 \).