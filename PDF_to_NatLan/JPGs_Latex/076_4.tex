TASK 3a  
To determine the gas pressure \( p_{g,1} \) and mass \( m_g \) in state 1, the following calculations are performed:

1. **Given constants and equations**:  
   - \( c_V, \text{gas} = 0.633 \, \frac{\text{kJ}}{\text{kg·K}} \)  
   - \( M_g = 50 \, \frac{\text{kg}}{\text{kmol}} \)  
   - \( R_g = \frac{8.314}{M_g} = 0.16628 \, \frac{\text{kJ}}{\text{kg·K}} \)  

2. **Temperature and volume**:  
   - \( T_{g,1} = 500^\circ\text{C} = 773.15 \, \text{K} \)  
   - \( V_{g,1} = 3.14 \, \text{L} = 3.14 \times 10^{-3} \, \text{m}^3 \)  

3. **Cross-sectional area of the cylinder**:  
   - \( A_K = \pi \left(\frac{D}{2}\right)^2 = \pi \left(\frac{0.1}{2}\right)^2 = 0.00785 \, \text{m}^2 \)  

4. **Gas pressure calculation**:  
   The gas pressure \( p_g \) is calculated using:  
   \[
   p_g = p_{\text{amb}} + m_K \cdot g \cdot \frac{1}{A_K} + m_{\text{EW}} \cdot g \cdot \frac{1}{A_K}
   \]  
   Substituting values:  
   \[
   p_{g,1} = 1 \, \text{bar} + \frac{g \cdot (32 + 0.1)}{A_K} = 1.4 \, \text{bar}
   \]  

5. **Gas mass calculation**:  
   Using the ideal gas law \( \rho \cdot V = m \cdot R \cdot T \):  
   \[
   m = \frac{p_g \cdot V}{R_g \cdot T}
   \]  
   Substituting values:  
   \[
   m = \frac{1.4 \, \text{bar} \cdot 3.14 \times 10^{-3} \, \text{m}^3}{0.16628 \, \frac{\text{kJ}}{\text{kg·K}} \cdot 773.15 \, \text{K}} = 3.47 \, \text{g}
   \]  

---

TASK 3b  
To determine \( T_{g,2} \) and \( p_{g,2} \), the following reasoning is applied:

1. **Pressure remains constant**:  
   \[
   p_{g,2} = p_{g,1} = 1.4 \, \text{bar}
   \]  
   This is because the pressure due to the piston and weight remains unchanged.

2. **Thermal equilibrium**:  
   At state 2, the gas and EW are in thermal equilibrium, so:  
   \[
   U_g = U_{\text{EW}}
   \]  

No further numerical calculations are provided for \( T_{g,2} \).  

---