TASK 4a  
The graph depicts the freeze-drying process in a pressure-temperature (\(p\)-\(T\)) diagram. The vertical axis represents pressure in millibars (\(p\)), and the horizontal axis represents temperature in degrees Celsius (\(T\)).  
- The curve labeled "Triple point" indicates the phase equilibrium between solid, liquid, and vapor.  
- Point 2 is connected to point 1 via an "isobar single curve," indicating constant pressure.  
- Point 3 is connected to point 1 via an "isotherm," indicating constant temperature.  

TASK 4b  
A table is provided to describe the cooling cycle. It includes the following states:  
1. Pressure (\(p\)) is \(p_0\).  
2. Pressure (\(p\)) is \(p_0 - 16^\circ\text{C}\), and the vapor quality (\(x_2\)) is equal to 1.  
3. Pressure (\(p\)) is \(8 \, \text{bar}\).  
4. Pressure (\(p\)) is \(8 \, \text{bar}\), and the vapor quality (\(x_4\)) is equal to 0.  

The refrigerant enters at state 1 and undergoes changes in pressure and temperature as it moves through the cycle. The heat transfer (\(h_q\)) and work (\(\dot{Q}_K\)) are indicated in the table.  

The initial temperature (\(T_i\)) is determined from the diagram as \(T_i = -20^\circ\text{C}\).