TASK 4c  
The vapor quality \( x \) is calculated using the formula:  
\[
x = \frac{h - h_f}{h_g - h_f}
\]  

Given:  
- \( p_4 = 8 \, \text{bar} \)  
- \( x_4 = 0 \)  
- \( T = 31.33 \, \text{°C} \)  

The process is described as isothermal. The derivative of enthalpy difference with respect to time is zero:  
\[
\frac{d}{dt} (h_4 - h_1) = 0
\]  

From this, \( h_4 = h_1 \).  

The enthalpy \( h_4 \) is equal to \( h_f(8 \, \text{bar}) \), which can be found in Table A-11:  
\[
h_4 = h_f(8 \, \text{bar}) = 55.42 \, \frac{\text{kJ}}{\text{kg}}
\]  

Next, the vapor quality \( x \) is recalculated for a different state:  
\[
x = \frac{h_4 - h_f(4 \, \text{°C})}{h_g(4 \, \text{°C}) - h_f(4 \, \text{°C})}
\]  

Using values from Table A-10:  
\[
h_4 = 55.42 \, \frac{\text{kJ}}{\text{kg}}, \quad h_f(4 \, \text{°C}) = 55.35 \, \frac{\text{kJ}}{\text{kg}}, \quad h_g(4 \, \text{°C}) = 2478.53 \, \frac{\text{kJ}}{\text{kg}}
\]  

Substituting these values:  
\[
x = \frac{55.42 - 55.35}{2478.53 - 55.35}
\]  
\[
x = \frac{0.07}{2423.18} = 0.000029
\]  

The triple point temperature is given as:  
\[
T_{\text{triple}} = 0 \, \text{°C}
\]  

The initial temperature \( T_i \) is calculated as:  
\[
T_i = 0 \, \text{°C} + 10 = 10 \, \text{°C}
\]  

The final temperature \( T_1 \) is given as:  
\[
T_1 = 4 \, \text{°C}
\]  

This concludes the calculation of vapor quality and temperature values.