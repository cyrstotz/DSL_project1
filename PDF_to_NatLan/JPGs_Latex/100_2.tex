TASK 2b  
The following calculations are related to the jet engine problem.  

The temperature ratio \( T_6 / T_5 \) is calculated using the pressure ratio \( p_6 / p_5 \) and the isentropic relation:  
\[
\frac{T_6}{T_5} = \left( \frac{p_6}{p_5} \right)^{\frac{\kappa - 1}{\kappa}}
\]  
Substituting values:  
\[
T_6 = \left( \frac{p_6}{p_5} \right)^{\frac{0.4}{1.4}} T_5
\]  
Given \( T_5 = 328.07 \, \text{K} \), the equation is used to determine \( T_6 \).  

The work \( w_{5/6} \) is calculated using the formula:  
\[
w_{5/6} = R \frac{T_B - T_B}{-0.4}
\]  
The result is incomplete, but the work is approximately \( 78.71 \, \text{kJ/kg} \).  

TASK 2c  
The specific exergy destruction is calculated using the energy balance:  
\[
w_e = h_e - h_a + \frac{w_e^2 - w_a^2}{2}
\]  

Further calculations include:  
\[
w_A^2 = 2h_5 - 2h_6 + w_e^2 - 2w_e
\]  
\[
w_A' = 2c_p (T_5 - T_6) + (1000 \, \text{m/s})^2 - 2w_e
\]  
\[
w_s = 3.16 \, \text{m/s}
\]  

TASK 2a  
A qualitative \( T \)-\( s \) diagram is drawn. The diagram shows the thermodynamic process of the jet engine with labeled states (1, 2, 3, 4, 5). The curve includes isentropic compression (1 to 2), isobaric heat addition (2 to 3), adiabatic expansion (3 to 4), and mixing (4 to 5).  

No additional explanation is provided for the diagram.  

TASK 2  
Constants and parameters are noted:  
- \( M = 28.97 \, \text{kg/kmol} \)  
- \( R = 287 \, \text{J/(kg·K)} \)  

No further explanation is provided for these constants.  

TASK 2d  
No content is clearly visible for this subtask.  

TASK 2e  
No content is clearly visible for this subtask.