TASK 3a  
The cross-sectional area \( A \) of the cylinder is calculated using the formula:  
\[
A = \pi r^2
\]  
where \( r = 0.05 \, \text{m} \). Substituting the value:  
\[
A = 7.854 \times 10^{-3} \, \text{m}^2
\]  

The pressure exerted by the piston \( p_K \) is determined using the formula:  
\[
p_K = \frac{F}{A}
\]  
where \( F = m_K \cdot g \). Substituting \( m_K = 32 \, \text{kg} \) and \( g = 9.81 \, \text{m/s}^2 \):  
\[
F = 32 \cdot 9.81 = 313.92 \, \text{N}
\]  
\[
p_K = \frac{313.92}{7.854 \times 10^{-3}} = 39969.53 \, \text{Pa} \approx 0.399 \, \text{bar}
\]  

The pressure exerted by the ice-water mixture \( p_{\text{EW}} \) is calculated as:  
\[
p_{\text{EW}} = \frac{m_{\text{EW}} \cdot g}{A}
\]  
Substituting \( m_{\text{EW}} = 0.1 \, \text{kg} \):  
\[
p_{\text{EW}} = \frac{0.1 \cdot 9.81}{7.854 \times 10^{-3}} = 124.90 \, \text{Pa} \approx 0.001249 \, \text{bar}
\]  

The total gas pressure \( p_{g,1} \) in state 1 is the sum of the piston pressure, the ice-water pressure, and the ambient pressure:  
\[
p_{g,1} = p_K + p_{\text{EW}} + p_{\text{amb}}
\]  
Substituting \( p_{\text{amb}} = 1 \, \text{bar} \):  
\[
p_{g,1} = 0.399 + 0.001249 + 1 = 1.4 \, \text{bar}
\]  

The gas mass \( m_g \) is calculated using the ideal gas law:  
\[
p_{g,1} V_{g,1} = m_g R T_1
\]  
Rearranging:  
\[
m_g = \frac{p_{g,1} V_{g,1} M}{R T_1}
\]  
Substituting \( p_{g,1} = 1.4 \times 10^5 \, \text{Pa} \), \( V_{g,1} = 3.14 \times 10^{-3} \, \text{m}^3 \), \( M = 50 \, \text{kg/kmol} \), \( R = 8.314 \, \text{J/mol·K} \), and \( T_1 = 500 + 273.15 \, \text{K} \):  
\[
m_g = \frac{1.4 \times 10^5 \cdot 3.14 \times 10^{-3} \cdot 50}{8.314 \cdot (500 + 273.15)} = 0.818 \, \text{g}
\]  

---

TASK 3c  
The heat transferred \( \Delta Q \) from the gas to the ice-water mixture is calculated using:  
\[
\Delta Q = m_g \cdot C_p \cdot \Delta T
\]  
where \( C_p \) is the specific heat capacity of the gas:  
\[
C_p = \frac{R}{M} + C_v
\]  
Substituting \( R = 8.314 \, \text{J/mol·K} \), \( M = 50 \, \text{kg/kmol} \), and \( C_v = 0.633 \, \text{kJ/kg·K} \):  
\[
C_p = \frac{8.314}{50} + 0.633 = 0.793 \, \text{kJ/kg·K} = 0.8 \, \text{kJ/kg·K}
\]  

Substituting \( m_g = 3.6 \times 10^{-3} \, \text{kg} \), \( \Delta T = T_2 - T_1 = 0^\circ\text{C} - 500^\circ\text{C} \):  
\[
\Delta Q = 3.6 \times 10^{-3} \cdot 0.8 \cdot (-500) = -1.4393 \, \text{J}
\]  

---

TASK 3d  
The internal energy \( u_2 \) at state 2 is calculated using:  
\[
u_2 = u_f + x (u_{\text{rest}} - u_f)
\]  

The change in internal energy \( \Delta U \) is equal to the heat transferred:  
\[
\Delta U = \Delta Q
\]  

The difference in internal energy between states 2 and 1 is given by:  
\[
u_2(T_2) - u_1(T_1) = C_r (T_2 - T_1)
\]  
where \( T_1 = 0^\circ\text{C} \), \( T_2 = 0^\circ\text{C} \), and \( C_r \) is the specific heat capacity of the ice-water mixture.  

Substituting \( \Delta Q = 15 \, \text{kJ} \), \( m_{\text{EW}} = 0.1 \, \text{kg} \):  
\[
u_2(T_2) - u_1(T_1) = \frac{\Delta Q}{m_{\text{EW}}} = \frac{15}{0.1} = 150 \, \text{kJ/kg}
\]  

For the ice fraction \( x_1 \):  
\[
u_1(T_1) = u_f + x_1 (u_{\text{rest}} - u_f)
\]  
Substituting \( u_f = -0.045 \, \text{kJ/kg} \), \( u_{\text{rest}} = -333.458 \, \text{kJ/kg} \), and \( x_1 = 0.6 \):  
\[
u_1(T_1) = -0.045 + 0.6 \cdot (-333.458 - (-0.045)) = -200.082 \, \text{kJ/kg}
\]  

(Note: Some calculations and expressions are crossed out and not considered here.)  

---  
No diagrams or graphs are present on this page.