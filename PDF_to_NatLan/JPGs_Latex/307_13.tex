TASK 4d  
The coefficient of performance \( \epsilon_K \) is calculated using the formula:  
\[
\epsilon_K = \frac{\lvert \dot{Q}_{\text{zu}} \rvert}{\lvert \dot{Q}_{\text{ab}} \rvert - \lvert \dot{Q}_{\text{zu}} \rvert}
\]  

The energy balance is expressed as:  
\[
0 = \dot{m}_{\text{R134a}} (h_3 - h_2) + \dot{Q}_{\text{ab}}
\]  
where \( \dot{Q}_{\text{ab}} = \dot{m}_{\text{R134a}} (h_4 - h_3) \).  

Substituting, the equation becomes:  
\[
0 = \dot{m}_{\text{R134a}} (h_1 - h_2) + \dot{Q}_{\text{zu}}
\]  
and \( \dot{Q}_{\text{zu}} = \dot{m}_{\text{R134a}} (h_2 - h_1) \).  

Thus, the coefficient of performance simplifies to:  
\[
\epsilon_K = \frac{\lvert h_3 - h_4 \rvert}{\lvert h_3 - h_4 \rvert - \lvert h_2 - h_1 \rvert}
\]  

---

TASK 4e  
The temperature \( T_i \) in Step ii will decrease further but at an increasingly slower rate. It cannot decrease indefinitely because the system reaches thermal equilibrium.  

A sketch is provided showing the qualitative behavior of \( T \) over time \( t \):  
- The curve starts at a higher temperature and decreases steeply initially.
- The slope flattens as time progresses, indicating the slowing rate of temperature decrease.  

The diagram is labeled as "so ein Verlauf" (such a progression), with \( T \) on the vertical axis and \( t \) on the horizontal axis.