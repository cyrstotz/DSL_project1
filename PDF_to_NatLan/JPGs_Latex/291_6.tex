TASK 3a  
The gas pressure \( p_{g,1} \) and mass \( m_g \) in state 1 are calculated as follows:  

The molar mass of the gas is given as \( M_g = 50 \, \text{kg/kmol} \). The area of the cylinder is calculated using the formula for the cross-sectional area:  
\[
A = R^2 \pi = \frac{D^2 \pi}{4}
\]  

The pressure \( p_{g,1} \) is determined using the equation:  
\[
p_{g,1} = \frac{m_K \cdot g}{A} + m_{\text{EW}} \cdot g + p_{\text{amb}}
\]  
Substituting values:  
\[
p_{g,1} = \frac{(m_K + m_{\text{EW}}) \cdot g}{\frac{D^2 \pi}{4}} + p_{\text{amb}} = 14 \, \text{bar}
\]  

The mass \( m_g \) is calculated using the ideal gas law:  
\[
m_g = \frac{p_1 V_1}{R T_1}
\]  
Where \( R = \frac{R_u}{M_g} \), and \( R_u = 8.314 \, \text{kJ/kmol·K} \). Substituting values:  
\[
R = \frac{8.314 \, \text{kJ/kmol·K}}{50 \, \text{kg/kmol}} = 0.16628 \, \text{kJ/kg·K}
\]  
\[
m_g = \frac{p_1 V_1}{R T_1} = 0.003422 \, \text{kg} = 3.422 \, \text{g}
\]  

---

TASK 3b  
The gas pressure \( p_{g,2} \) is equal to \( p_{g,1} \) because the same weight is still pressing on the gas. Therefore:  
\[
p_{g,2} = 14 \, \text{bar}
\]  

The temperature \( T_{g,2} \) is equal to \( T_{\text{EW}} \), as the gas and EW reach thermal equilibrium.  

---

TASK 3c  
The heat transferred \( Q_{12} \) from the gas to the EW is calculated under isobaric conditions:  
\[
Q_{12} = \Delta E = m(u_2 - u_1) = Q_{\text{ab}} = \int_1^2 p \, dv = R T \int_1^2 \frac{1}{v} \, dv
\]  

Substituting values:  
\[
Q_{\text{ab}} = m c_V (T_2 - T_1) = 0.00584 \, \text{J} = 5884 \, \text{mJ}
\]  

