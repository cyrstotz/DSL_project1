TASK 4a  
A graph is drawn showing the freeze-drying process in a pressure-temperature (\( p \)-\( T \)) diagram. The graph includes the following points:  
- Point 1 is at the top right, representing the initial state.  
- Point 2 is horizontally aligned with Point 1 but at a lower pressure.  
- Point 3 is vertically below Point 2, representing the sublimation pressure at 5 mbar.  
- The triple point is marked on the graph, and the pressure axis includes labels for "5 mbar" and "Triple Point."  
The temperature axis is labeled as \( T \), and the pressure axis is labeled as \( p \).  

TASK 4b  
Energy balance for the compressor is written as:  
\[
\dot{Q} = \dot{m}_{\text{R134a}} \cdot (h_2 - h_3) + \dot{W}_c
\]  
From this, the mass flow rate of the refrigerant is derived:  
\[
\dot{m}_{\text{R134a}} = \frac{\dot{W}_c}{h_3 - h_2}
\]  

The enthalpy values are calculated as follows:  
1. \( h_2 \):  
   - Isobaric process from state 1 to state 2.  
   - \( p_2 = p_a \) (ambient pressure).  
   - \( T_2 = 26^\circ\text{C} \).  
   - Using the refrigerant table (Table A-10 at \( T = 26^\circ\text{C} \)):  
     \[
     h_2 = h_g(-26^\circ\text{C}) = 231.62 \, \text{kJ/kg}
     \]  

2. \( h_3 \):  
   - Isentropic process from state 2 to state 3.  
   - \( s_2 = s_3 \).  
   - Using the refrigerant table (Table A-12 at \( s_3 \)):  
     \[
     s_2 = s_g(-26^\circ\text{C}) = 0.9390 \, \text{kJ/kg·K}
     \]  
   - Interpolation is performed using Table A-12 with \( s_3 \):  
     \[
     h_3 = 273.66 + (284.39 - 273.66) \cdot \frac{0.9390 - 0.9374}{0.9391 - 0.9374} = 274.74 \, \text{kJ/kg}
     \]  

Finally, the refrigerant mass flow rate is calculated:  
\[
\dot{m}_{\text{R134a}} = \frac{0.000648 \, \text{kJ/s}}{h_3 - h_2} = 0.05291 \, \text{kg/s}
\]