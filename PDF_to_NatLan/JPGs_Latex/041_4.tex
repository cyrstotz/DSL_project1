TASK 3a  
The initial temperature of the gas is given as \( T_{g,1} = 500^\circ\text{C} \), and the initial volume is \( V_{g,1} = 3.14 \, \text{L} \).  

The pressure \( p_{g,1} \) is calculated using the ideal gas law:  
\[
p_{g,1} V_{g,1} = R T_{g,1} m_g
\]  
Rearranging for \( p_{g,1} \):  
\[
p_{g,1} = \frac{R T_{g,1} m_g}{V_{g,1}}
\]  

The specific volume \( v_{g,1} \) is calculated as:  
\[
v_{g,1} = \frac{V_{g,1}}{m_g} = \frac{V_{g,1}}{M_g} = 0.0000628 \, \text{m}^3/\text{kg}
\]  

A sketch of the cylinder is provided, showing the gas chamber below a piston. The piston exerts a force \( F \) due to its weight:  
\[
F = m_K \cdot g = 32.1 \cdot 9.81
\]  

The area \( A \) of the piston is calculated as:  
\[
A = \pi \cdot (5 \, \text{cm})^2
\]  

The pressure exerted by the piston is:  
\[
p = \frac{F}{A} = \frac{32.1 \cdot 9.81}{\pi \cdot (5 \, \text{cm})^2} = 0.409 \, \text{bar}
\]  

The mass of the gas \( m_g \) is calculated using the ideal gas law:  
\[
m_g = \frac{p_{g,1} V_{g,1}}{R T_{g,1}}
\]  
Substituting values:  
\[
m_g = 0.9769 \, \text{kg} \approx 0.98 \, \text{kg}
\]  

TASK 3b  
The system reaches thermodynamic equilibrium in state 2. The temperature of the gas and the ice-water mixture are equal:  
\[
T_{g,2} = T_{\text{EW},2}
\]  

Since the temperature remains constant in the solid-liquid equilibrium region:  
\[
T_{\text{EW},2} = T_{\text{EW},1}
\]  
Thus:  
\[
T_{g,2} = T_{\text{EW},1} = 0^\circ\text{C}
\]  

The pressure in state 2 remains equal to the pressure in state 1:  
\[
p_{g,2} = p_{g,1} = 0.4 \, \text{bar}
\]  

Because no mass is lost and the area remains constant, the same mass continues to exert pressure on the same surface area.