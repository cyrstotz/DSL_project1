TASK 3b  
The ice-water mixture remains in thermodynamic equilibrium at the solid-liquid phase boundary until all the ice (solid phase) has melted. Therefore, the ice-water mixture (EW) is still at \( T_{\text{EW},2} = 0^\circ\text{C} \). Consequently, the gas must also be at \( T_{\text{g},2} = 0^\circ\text{C} \) to maintain thermodynamic equilibrium.  

\[
T_{\text{g},2} = 0^\circ\text{C}
\]

The masses that exert pressure on the gas remain constant. Thus, the gas pressure is calculated as:  
\[
p_{\text{g},2} = \frac{m_K}{V_{\text{g},2}}
\]
Substituting values:  
\[
p_{\text{g},2} = 1.46 \, \text{bar}
\]

---

TASK 3c  
An energy balance is applied to the system:  
\[
\frac{dE}{dt} = \sum_j \dot{Q}_j - \sum_n \dot{W}_n
\]
This simplifies to:  
\[
U_2 - U_1 = Q_{12} - W_{12}
\]

Since there is no friction, the work \( W_{12} \) is reversible:  
\[
W_{12} = \int p \, dV
\]

The gas mass is calculated as:  
\[
m_g = \frac{V_{g,1}}{v_{g,1}}
\]
Substituting values:  
\[
m_g = \frac{3.14 \cdot 10^{-3} \, \text{m}^3}{3.429 \cdot 10^{-3} \, \text{m}^3/\text{kg}} = 0.9168397 \, \text{kg}
\]

---

Description of Diagram:  
A small sketch shows a system with labeled heat transfer (\( Q_{12} \)) and work (\( W_{12} \)) arrows. The system boundary is indicated, representing the energy balance applied to the gas and ice-water mixture.