TASK 1d  
The enthalpy \( h_1 \) is calculated as:  
\[
h_1 = h(100^\circ\text{C}, x = 0.005) = h_f(100^\circ\text{C}) + x \cdot \left(h_g(100^\circ\text{C}) - h_f(100^\circ\text{C})\right)
\]  
Using values from Table A-2:  
\[
h_1 \approx 430.3 \, \text{kJ/kg}
\]  

The enthalpy \( h_2 \) is calculated as:  
\[
h_2 = h(30^\circ\text{C}, x = 0) = h_f(30^\circ\text{C})
\]  
From Table A-2:  
\[
h_2 = 292.88 \, \text{kJ/kg}
\]  

The enthalpy \( h_{\text{in,12}} \) is calculated as:  
\[
h_{\text{in,12}} = h(20^\circ\text{C}, x = 0) = h_f(20^\circ\text{C})
\]  
From Table A-2:  
\[
h_{\text{in,12}} = 83.96 \, \text{kJ/kg}
\]  

The mass \( \Delta m_{12} \) is determined using the energy balance:  
\[
\Delta m_{12} = \frac{5755}{83.96} \cdot \frac{(292.88 - 430.3)}{(1 - 83.96)}
\]  
This simplifies to:  
\[
\Delta m_{12} \approx 109.5 \, \text{kg}
\]  

---

TASK 1e  
The entropy change \( \Delta S \) is calculated as:  
\[
\Delta S = S_2 - S_1 = m_2 \cdot s_2 - m_1 \cdot s_1
\]  

The entropy \( s_2 \) is determined as:  
\[
s_2 = s_f(30^\circ\text{C}, \text{saturated liquid}) \approx 0.9543 \, \text{kJ/kg·K}
\]  
From Table A-2.  

The entropy \( s_1 \) is calculated as:  
\[
s_1 = s_f(100^\circ\text{C}) + x \cdot \left(s_g(100^\circ\text{C}) - s_f(100^\circ\text{C})\right) \approx 1.337 \, \text{kJ/kg·K}
\]  

Substituting into the entropy change equation:  
\[
\Delta S = (5755 + 109.5) \cdot 0.9543 - 5755 \cdot 1.337
\]  
This simplifies to:  
\[
\Delta S \approx 2095 \, \text{kJ/K}
\]