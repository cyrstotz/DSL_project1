TASK 2a  
A T-s diagram is drawn to represent the jet engine process. The diagram includes labeled states (0, 1, 2, 3, 4, 5, 6) and processes:  
- Between states 0 and 1: Isobaric process.  
- Between states 1 and 2: Adiabatic, reversible, and isentropic compression.  
- Between states 2 and 3: Isobaric heat addition.  
- Between states 3 and 4: Adiabatic, irreversible expansion (entropy increases).  
- Between states 4 and 5: Isobaric mixing.  
- Between states 5 and 6: Isentropic expansion in the nozzle.  

The axes are labeled as follows:  
- The x-axis represents entropy \( s \) in \( \text{kJ}/\text{kg·K} \).  
- The y-axis represents temperature \( T \) in Kelvin (K).  

TASK 2b  
The outlet velocity \( w_6 \) and temperature \( T_6 \) are to be determined.  

Given data:  
\[
w_5 = 200 \, \text{m/s}, \, p_5 = 0.5 \, \text{bar}, \, T_5 = 431.9 \, \text{K}, \, p_6 = p_0 = 0.191 \, \text{bar}.
\]

The energy balance equation is written as:  
\[
0 = \dot{m} (s_c - s_a) + \frac{Q_i}{T_j} + \frac{s_e}{2}.
\]  
It is noted that \( Q_i = 0 \) for adiabatic processes and the process is irreversible.  

Entropy difference \( s_c - s_a \) is considered.  

Polytropic temperature relationship is used:  
\[
T_2 = T_1 \left( \frac{p_2}{p_1} \right)^{\frac{n-1}{n}}, \quad n = 1.4.
\]  

For \( T_6 \):  
\[
T_6 = T_5 \left( \frac{p_6}{p_5} \right)^{\frac{\kappa - 1}{\kappa}}, \quad \kappa = 1.4.
\]  
Substituting values:  
\[
T_6 = 431.9 \, \text{K} \cdot \left( \frac{0.191 \, \text{bar}}{0.5 \, \text{bar}} \right)^{\frac{1.4 - 1}{1.4}} = 328.07 \, \text{K}.
\]

Energy balance for velocity:  
\[
0 = h_e - h_a + \frac{w_e^2}{2} - \frac{w_a^2}{2}.
\]  
Rearranging:  
\[
w_a = \sqrt{2 (h_e - h_a)} + w_e^2.
\]