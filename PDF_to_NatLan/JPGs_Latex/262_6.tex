TASK 2a  
The task involves drawing a qualitative \( T \)-\( S \) diagram for a thermodynamic process. The diagram is labeled as follows:

- The vertical axis represents temperature (\( T \)) in Kelvin, and the horizontal axis represents entropy (\( S \)) in \( \text{kJ}/\text{kg·K} \).  
- The reference temperature is given as \( T_0 = 293.15 \, \text{K} \).  

Key points and processes are marked:  
1. Point \( 0 \) is the starting state.  
2. From \( 0 \) to \( 1 \), the process is labeled as "reversible, adiabatic."  
3. From \( 1 \) to \( 2 \), the process is labeled as "reversible, isobaric."  
4. From \( 2 \) to \( 3 \), the process is labeled as "isobaric."  
5. From \( 3 \) to \( 7 \), the process is labeled as "adiabatic, irreversible."  
6. From \( 7 \) to \( S \), the process is labeled as "isobaric."  
7. From \( S \) to \( 6 \), the process is labeled as "reversible."  

The entropy values \( S_0, S_2, S_4 \) are marked along the horizontal axis.  

The diagram visually represents the thermodynamic cycle, with curved and straight lines connecting the states to indicate the nature of the processes (e.g., isobaric, adiabatic).