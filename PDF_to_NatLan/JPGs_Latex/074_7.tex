TASK 3b  
The content begins with a labeled section "b)" and includes calculations and explanations related to thermodynamic processes involving ice melting and gas behavior.  

The mass of the ice-water mixture is given as:  
\[
m_{\text{EW}} = 0.1 \, \text{kg}
\]  

The heat transfer is assumed to be zero:  
\[
Q = 0
\]  

An energy balance is written as:  
\[
\frac{dE}{dt} = \sum \dot{Q} - \sum \dot{W}
\]  

The specific heat capacity at constant volume is provided:  
\[
C_v = 0.633 \, \text{kJ/kg·K}
\]  

The melting point of ice is noted:  
"Eis schmilzt bei \( 0^\circ\text{C} \)" (Ice melts at \( 0^\circ\text{C} \)).  

The ice fraction \( x_{\text{ice}} \) is stated to be greater than zero:  
\[
x_{\text{ice}} > 0
\]  

The equilibrium temperatures for the ice-water mixture and the gas are given:  
\[
T_{\text{EW},2} = 0^\circ\text{C}
\]  
\[
T_{g,2} = 0^\circ\text{C}
\]  

A note explains that not all the ice melts, and the temperature must remain at \( 0^\circ\text{C} \) for both the ice-water mixture and the gas.  

The ideal gas relationship is used to calculate pressure:  
\[
T_2 = T_1 \left( \frac{p_2}{p_1} \right)^{\frac{n-1}{n}}
\]  
This is rearranged to:  
\[
\frac{T_2}{T_1} = \left( \frac{p_2}{p_1} \right)^{\frac{n-1}{n}}
\]  

The specific heat capacity at constant pressure is calculated:  
\[
C_p = R + C_v = 8.314 \, \text{J/mol·K} + 0.633 \, \text{kJ/kg·K} = 735.28 \, \text{J/kg·K}
\]  

The ratio of specific heats is given:  
\[
k = \frac{C_p}{C_v} = 4.187
\]  

The pressure \( p_2 \) is calculated using the ideal gas law:  
\[
p_2 = p_1 \left( \frac{T_2}{T_1} \right)^{\frac{k-1}{k}}
\]  
Substituting values:  
\[
p_2 = 1.5 \, \text{bar} \left( \frac{273.15 \, \text{K}}{500 + 273.15 \, \text{K}} \right)^{\frac{4.187 - 1}{4.187}}
\]  
The result is approximately:  
\[
p_2 = 0.403 \, \text{bar}
\]  

No diagrams or graphs are present on the page.