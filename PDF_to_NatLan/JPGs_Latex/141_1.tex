TASK 4c  
The vapor quality \( x_1 \) directly after the throttle is calculated using:  
\[
x_1 = \frac{m_g}{m_g + m_p}
\]  
where \( m_g \) and \( m_p \) represent the respective masses.  

TASK 4d  
The coefficient of performance \( \epsilon_K \) is defined as:  
\[
\epsilon_K = \frac{\dot{Q}_K}{\dot{W}_K} = \frac{Q_{\text{out}}}{\dot{W}_K} - 25 \, \text{kW}
\]  
The heat transfer rate \( \dot{Q}_K \) is calculated as:  
\[
\dot{Q}_K = -\dot{m} \cdot (\text{h}_{\text{e}} - \text{h}_{\text{a}}) = -4 \, \text{kg/h}
\]  

TASK 4e  
If there were a continuous decrease in heat during sublimation and the cooling cycle continued at constant \( \dot{Q}_K \), the temperature would remain constant for a long time until the phase transition is complete. Afterward, the temperature would begin to decrease further.  

No figures or diagrams are present on this page.