TASK 4a  
A diagram is drawn showing the freeze-drying process with a refrigeration cycle using R134a. The diagram is a p-T graph with labeled regions and states:  
- State 1 is in the "Dampf-Flüssig zweiphasiges Gebiet" (vapor-liquid two-phase region).  
- State 2 is labeled as "Gas."  
- State 3 is labeled as "Isobar \( p_3 > 8 \, \text{bar} \)."  
- State 4 is labeled as "Flüssig" (liquid).  
The process transitions between these states, with arrows indicating the flow of the cycle.  

TASK 4b  
From Table A-15, interpolation is used to calculate the mass flow rate of the refrigerant:  
\[
\dot{m}_{\text{R134a}} = \frac{\dot{Q}_K}{h_3 - h_1}
\]  
where \( \dot{Q}_K \) is the heat removed, and \( h_3 \) and \( h_1 \) are specific enthalpies.  
The calculation yields:  
\[
\dot{m}_{\text{R134a}} \approx 4 \, \text{kg/h}.
\]  

TASK 4c  
The vapor quality \( x_1 \) at state 1 is calculated using the formula:  
\[
x_1 = \frac{u_1 - u_f}{u_g - u_f}
\]  
where \( u_1 \) is the specific internal energy at state 1, \( u_f \) is the specific internal energy of the saturated liquid, and \( u_g \) is the specific internal energy of the saturated vapor.  
It is noted that \( u_1 = u_f \).