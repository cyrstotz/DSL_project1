TASK 3a  
The gas pressure \( p_g \) is calculated using the following formula:  
\[
p_g = p_{\text{amb}} + \frac{4 m_K g}{D^2 \pi} + \frac{4 m_{\text{EW}} g}{D^2 \pi}
\]
where \( A = \frac{D^2}{4 \pi} \).  

Substituting values:  
\[
p_g = 1 \, \text{bar} + \frac{4 g}{D^2 \pi} (m_K + m_{\text{EW}})
\]
\[
p_g = 1 \, \text{bar} + \frac{4 \cdot 9.81}{(0.1 \, \text{m})^2 \pi} (32 \, \text{kg} + 0.1 \, \text{kg})
\]
\[
p_g = 1.4 \, \text{bar}
\]

The mass of the gas \( m_g \) is calculated using the ideal gas law:  
\[
m_g = \frac{p_g V_{g,1} M}{R T_{g,1}}
\]
Substituting values:  
\[
m_g = \frac{1.4 \, \text{bar} \cdot 3.74 \, \text{L} \cdot 50 \, \text{kg/kmol}}{8.314 \, \text{J/mol·K} \cdot 773.15 \, \text{K}}
\]
\[
m_g = 3.453 \, \text{g}
\]

TASK 3b  
The pressure \( p_{g,2} \) is equal to \( p_{g,1} \) because the equilibrium from state 1 does not change.  

The temperature \( T_2 \) is calculated using the isentropic relation:  
\[
T_2 = T_1 \left( \frac{p_2}{p_1} \right)^{\frac{n-1}{n}}
\]
where \( n = \frac{c_p}{c_v} \), and \( c_p \) and \( c_v \) are related by:  
\[
n = \frac{R + c_v}{c_v} = \frac{\frac{R}{M} + c_v}{c_v}
\]
Substituting values:  
\[
n = \frac{8.314}{50} + 0.633 = 1.2627
\]

For \( T_2 \):  
\[
T_2 = 500 \, \text{K} \left( \frac{1.4}{1.4} \right)^{\frac{1.2627 - 1}{1.2627}}
\]
\[
T_2 = T_1
\]

Thus, \( T_2 = T_1 \) because the pressure ratio does not change, and the temperature remains constant.