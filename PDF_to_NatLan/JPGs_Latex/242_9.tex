TASK 4c  
The vapor quality \( x_1 \) is calculated using the formula:  
\[
h_1 = h_f + x_1 h_g = h_f + x_1 (h_g - h_f)
\]  
Substituting values:  
\[
h_1 = h_f (1.00489 \, \text{kJ/kg}) + x_1 (h_g (1.00489 \, \text{kJ/kg}) - h_f (1.00459 \, \text{kJ/kg}))
\]  
Rearranging for \( x_1 \):  
\[
x_1 = \frac{h_1 - h_f}{h_g - h_f} = 0.3566
\]  
Thus, the vapor quality \( x_1 \) is \( 35.66\% \).  

---

TASK 4d  
The coefficient of performance \( \epsilon_K \) is calculated using:  
\[
\epsilon_K = \frac{Q_{zu}}{W_K}
\]  
Here, \( Q_{zu} = Q_u \).  

Using the energy balance:  
\[
0 = \dot{m}(h_2 - h_1) + Q_u
\]  
Substituting values:  
\[
Q_u = \dot{m}(h_2 - h_1)
\]  
\[
h_2 = h_f + 35.66\% (h_g - h_f) = 93.42 \, \text{kJ/kg}
\]  
\[
h_1 = 23.16 \, \text{kJ/kg}
\]  
\[
Q_u = 0.094 \, \text{kg/s} \cdot (93.42 - 23.16) = 88.91 \, \text{W}
\]  

Finally, the coefficient of performance:  
\[
\epsilon_K = \frac{Q_{zu}}{W_K} = \frac{88.91 \, \text{W}}{28 \, \text{W}} = 3.25
\]  

A simple diagram is drawn showing a refrigeration loop with labeled states \( 1 \) and \( 2 \), and heat flow \( Q_u \) entering the system.  

---

TASK 4e  
The temperature \( T_i \) would decrease to \( -26^\circ\text{C} \), as the temperature of the cooling medium is lower.  

No additional mathematical expressions or diagrams are provided for this part.