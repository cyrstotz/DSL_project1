TASK 3a  
The gas pressure \( p_{g,1} \) is calculated using the formula:  
\[
p_{g,1} = p_{\text{atm}} + \frac{F}{A}
\]  
Where:  
- \( F = 32 \, \text{kg} \cdot g = 32 \cdot 9.81 = 314 \, \text{N} \)  
- \( A = \pi \cdot r^2 = \pi \cdot (0.05 \, \text{m})^2 \)  
- \( \frac{F}{A} = \frac{314}{0.00785} = 40 \, \text{kPa} \)  

Thus:  
\[
p_{g,1} = p_{\text{atm}} + \frac{F}{A} = 1.1 \, \text{bar}
\]  

The ideal gas law is used to calculate the gas mass \( m_g \):  
\[
p V = m R T
\]  
Rearranging:  
\[
m_g = \frac{p V}{R T}
\]  
Where:  
- \( R = \frac{R_u}{M} = \frac{8.314}{0.05} = 166.28 \, \text{J/(kg·K)} \)  
- \( p = 1.1 \, \text{bar} = 1.1 \cdot 10^5 \, \text{Pa} \)  
- \( V = 3.14 \, \text{L} = 3.14 \cdot 10^{-3} \, \text{m}^3 \)  
- \( T = T_{g,1} = 773 \, \text{K} \)  

Substituting values:  
\[
m_g = \frac{1.1 \cdot 10^5 \cdot 3.14 \cdot 10^{-3}}{166.28 \cdot 773} = 3.42 \, \text{g}
\]  

TASK 3b  
Since \( x_{\text{ice},2} > 0 \) and the temperature is homogeneous throughout the system, \( T_{g,2} \) must equal \( T_{12} = 0^\circ\text{C} \) at constant \( p = 1.1 \, \text{bar} \).  

In the two-phase region, \( p \) remains constant at \( 1.1 \, \text{bar} \), and the temperature is \( 0^\circ\text{C} \) according to the equilibrium table.  

The pressure also remains constant because \( p_{\text{atm}} \) and \( m_K \) are constant.