TASK 4c  
The process involves an adiabatic throttle, where \( h_4 = h_1 \) according to the first law of thermodynamics.  

The vapor quality at state 4 is given as \( x_4 = 0 \). The pressure at state 4 is \( p_4 = 8 \, \text{bar} \) (isobaric process from state 3 to state 4).  

From Table A-11:  
\[
h_4 = 93.42 \, \frac{\text{kJ}}{\text{kg}} = h_1
\]  

For state 2, the pressure is \( p_2 = p_1 \) (isobaric process). From Table A-10, at \( T_2 = -22^\circ\text{C} \), the pressure is:  
\[
p_2 = 1.21926 \, \text{bar} = p_1
\]  

The vapor quality \( x_A \) is calculated as:  
\[
x_A = \frac{h_A - h_f}{h_g - h_f}
\]  
Substituting values:  
\[
x_A = 0.337
\]  

---

TASK 4d  
The coefficient of performance \( \epsilon_K \) is calculated as:  
\[
\epsilon_K = \frac{\dot{Q}_K}{|\dot{W}_K|}
\]  

The heat transfer rate \( \dot{Q}_{\text{zu}} = \dot{Q}_K \) is given by:  
\[
\dot{Q}_{\text{zu}} = \dot{m}_{\text{R134a}} \cdot (h_2 - h_1)
\]  
Substituting values:  
\[
\dot{Q}_{\text{zu}} = 0.156 \, \text{kW}
\]  

Thus, the coefficient of performance is:  
\[
\epsilon_K = 5.58 \cdot 10^{-3}
\]  

---

TASK 4e  
The temperature would increase.  

