TASK 3d  
The task involves determining the final ice fraction \( x_{\text{ice},2} \) in state 2 using an energy balance and the solid-liquid equilibrium table.

The energy balance equation is written as:  
\[
\frac{dE}{dt} = \dot{m} \left[ h_1 - h_2 \right] + Q
\]

The change in internal energy is given as:  
\[
\Delta U = Q
\]

The internal energy equation is expressed as:  
\[
u_1(T_1) - u_2(T_2) = q_{12}
\]

The detailed energy balance for the ice-water mixture is expanded as:  
\[
x_2 \cdot u_{\text{ice}}(0^\circ\text{C}) + (1 - x_2) \cdot u_{\text{flowing}}(0^\circ\text{C}) - x_2 \cdot u_{2,\text{fusion}}(0.005^\circ\text{C}) - (1 - x_2) \cdot u_{\text{flowing}}(0.005^\circ\text{C}) = q_{12}
\]

Further simplifications lead to:  
\[
x_1 \cdot u_{\text{H}} + (1 - x) \cdot u_{\text{FA}} - x_L \cdot u_{\text{H2}} - (1 - x_L) \cdot u_{\text{FA}} + u_{\text{H2}} - u_{\text{F2}} = q_{12}
\]

Rearranging terms:  
\[
q_{12} - x_1 \cdot u_{\text{H2}} - (1 - x) \cdot u_{\text{FA}} + u_{\text{H2}} - u_{\text{F2}} = x_L
\]

Substituting numerical values:  
\[
-316.5 + 0.6 \cdot (-333.450) - (1 - 0.6) \cdot (-0.045) - 0.031
\]

The final ice fraction is calculated as:  
\[
x_L = 0.548
\]

The student notes that all values are taken from Table A.

---

No diagrams or figures are present on this page.