TASK 1a  
The heat flow removed by the coolant, \( \dot{Q}_{\text{out}} \), is determined using the energy balance:  
\[
\dot{m}(h_{\text{in}} - h_{\text{out}}) + \dot{Q}_{\text{out}} = 0
\]  
Rearranging for \( \dot{Q}_{\text{out}} \):  
\[
\dot{Q}_{\text{out}} = \dot{m} \cdot (h_{\text{out}} - h_{\text{in}})
\]  

The enthalpy values are calculated using water tables:  
\[
h_{\text{out}} = h(T = 100^\circ\text{C}, x = 0.005) = x \cdot h_g + (1 - x) \cdot h_f = 430.325 \, \text{kJ/kg}
\]  
\[
h_{\text{in}} = h_{\text{saturated liquid}} = 300.647 \, \text{kJ/kg}
\]  

Substituting into the equation:  
\[
\dot{Q}_{\text{out}} = 37.702774 \, \text{kW}
\]  

---

TASK 1b  
The thermodynamic mean temperature \( T_{\text{KF}} \) of the coolant is calculated as:  
\[
T_{\text{KF}} = \frac{\int T \, dQ}{\dot{Q}}
\]  

The entropy difference \( s_2 - s_1 \) is calculated as:  
\[
s_2 - s_1 = s_g - s_f = 7.72 - 6.384 = 1.336 \, \text{kJ/kg·K}
\]  

The entropy integral is expressed as:  
\[
s_2 - s_1 = \int \frac{c_p}{T} \, dT = c_p \cdot \ln \left( \frac{T_2}{T_1} \right)
\]  

