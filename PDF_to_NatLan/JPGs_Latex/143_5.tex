TASK 3a  
The specific gas constant \( R_g \) is calculated as:  
\[
R_g = \frac{R}{M_g} = 266.28 \, \text{J/kg·K}
\]  
The cross-sectional area \( A \) of the cylinder is determined using the formula:  
\[
A = \frac{\pi D^2}{4} = 0.0078 \, \text{m}^2
\]  
The force balance equation for the gas pressure \( p_{g,1} \) is given as:  
\[
p_{g,1} \cdot A = m_K \cdot g + p_{\text{amb}} \cdot A + m_{\text{EW}} \cdot g
\]  
Rearranging and solving for \( p_{g,1} \):  
\[
p_{g,1} = \frac{m_K \cdot g + m_{\text{EW}} \cdot g}{A} + p_{\text{amb}} = 1.4 \, \text{bar}
\]  
The mass of the gas \( m_g \) is calculated using the ideal gas law:  
\[
m_g = \frac{p_{g,1} \cdot V_{g,1}}{R_g \cdot T_{g,1}} = 3.42 \, \text{g}
\]  

---

TASK 3b  
The gas pressure in state 2, \( p_{g,2} \), is equal to \( p_{g,1} \):  
\[
p_{g,2} = p_{g,1} = 1.4 \, \text{bar}
\]  
This is because the same pressure from atmospheric conditions and the piston weight still applies.  

For the polytropic process, the following equation is used:  
\[
k = \frac{R + c_V}{c_V} = 1.263
\]  
The temperature \( T_{g,2} \) is calculated using the polytropic relation:  
\[
T_{g,2} = T_{g,1} \cdot \left( \frac{p_2}{p_1} \right)^{\frac{k-1}{k}} = 773.75 \, \text{K}
\]  

---

TASK 3c  
The ice fraction in state 2 is given as:  
\[
x_{\text{ice},2} = 0.0005
\]  

---

No diagrams or figures are present on this page.