TASK 4a  
The diagram is a pressure-temperature (\(p\)-\(T\)) graph illustrating the freeze-drying process. It includes the following labeled regions and transitions:  
- An **isobaric** process at 8 bar (horizontal line from state 1 to state 4).  
- An **isenthalpic** process (diagonal line from state 4 to state 1).  
- Another **isobaric** process at a lower pressure (horizontal line from state 2 to state 3).  
- An **isentropic** process (diagonal line from state 3 to state 2).  

The graph visually represents the refrigeration cycle with phase transitions and thermodynamic processes.

---

TASK 4b  
The mass flow rate of R134a (\( \dot{m}_{\text{R134a}} \)) is calculated using the energy balance for the compressor:  
\[
0 = \dot{m} \left[ h_2 - h_3 \right] + \dot{W}_K
\]  
The process is described as adiabatic and reversible, with the entropy condition:  
\[
s_2 = s_3
\]  

Additional expressions are provided for enthalpy values:  
\[
h_2 = 
\]  
\[
h_3 = 
\]  

The temperature at state 2 is given as:  
\[
T_2 = T_i + 6 \, \text{K} = 94 \, \text{K} + 7.5 \, \text{u}6
\]  

The initial temperature \( T_i \) is expressed as:  
\[
T_i = 10 \, \text{K} + T_{\text{sub}}
\]  

No further numerical values are provided for \( h_2 \), \( h_3 \), or \( \dot{m}_{\text{R134a}} \).