TASK 4a  
The table outlines the states of the freeze-drying process with the following columns: \( M \), \( V \), \( P \), \( T \), \( Q \), \( W \), and Notes.  

- State \( z1 \):  
  - \( P = p_1 \)  
  - \( T = T_i - 6 \, \text{K} \)  

- State \( z2 \):  
  - \( P = p_2 \)  

- State \( z3 \):  
  - \( P = 8 \, \text{bar} \)  

- State \( z4 \):  
  - \( P = 8 \, \text{bar} \)  

Additional notes:  
- The evaporator operates at \( T_i - 6 \, \text{K} \).  
- In the chamber, the process is isothermal and occurs below the triple point. \( T_i = 10 \, \text{K} \) above the sublimation temperature.  
- The process between states 2 and 3 is isentropic.  
- The refrigerant mass flow rate is \( \dot{m}_{\text{R134a}} \), and the work \( W = 26 \, \text{kW} \).  

TASK 4a  
The graph is a qualitative \( p-T \) diagram illustrating the freeze-drying process.  

- The curve represents the phase regions.  
- Points 1 through 4 are labeled to indicate the states of the process.  
- The path from state 1 to state 2 is a downward curve, representing cooling and evaporation.  
- The path from state 2 to state 3 is a steep upward curve, indicating compression.  
- The path from state 3 to state 4 is horizontal, representing isobaric condensation.  

No additional axes or units are explicitly labeled.