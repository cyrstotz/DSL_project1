TASK 4a  
A p-T diagram is drawn. The diagram shows pressure (\( p \)) on the vertical axis and temperature (\( T \)) on the horizontal axis. Four states are labeled:  
- State 1 is in the two-phase region, near the saturated liquid line.  
- State 2 is in the two-phase region, near the saturated vapor line.  
- State 3 is in the superheated vapor region, at higher pressure.  
- State 4 is in the subcooled liquid region, at the same pressure as state 3.  

The diagram qualitatively represents the refrigeration cycle with transitions between phase regions.

---

TASK 4b  
The mass flow rate of the refrigerant is calculated using the energy balance for the heat exchanger process:  
\[
\dot{Q} = \dot{m} \left( h_e - h_a \right) = \dot{m} \left( h_2 - h_3 \right) + \dot{W}_{\text{trim}}
\]  
Rearranging for \( \dot{m} \):  
\[
\dot{m} = \frac{\dot{Q}}{h_2 - h_3}
\]  
The equation assumes steady-state operation and neglects kinetic and potential energy changes.  

A note indicates \( \dot{Q} = 1 \, \text{kW} \).

---

TASK 4c  
The pressure values are given:  
\[
p_i = p_3 = 8 \, \text{bar}
\]  
\[
p_1 = p_{\text{water}}
\]  
This indicates the pressure at state 1 corresponds to the water vapor pressure at the sublimation temperature.

---

TASK 4d  
The coefficient of performance (\( \epsilon_K \)) is calculated using the formula:  
\[
\epsilon_K = \frac{\dot{Q}_K}{\dot{W}_K} = \frac{\dot{Q}_K}{\dot{Q}_{\text{in}} - \dot{Q}_{\text{out}}}
\]  
This represents the efficiency of the refrigeration cycle.

---

TASK 4e  
The temperature \( T_i \) would decrease further during Step ii due to the lower mass flow rate through state 2 and state 1. This is caused by the large pressure difference at the compressor (states 2 to 3).