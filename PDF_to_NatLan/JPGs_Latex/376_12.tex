TASK 2a  
The diagram represents a qualitative \( T \)-\( s \) (temperature-entropy) diagram for the jet engine process. The axes are labeled as follows:  
- The vertical axis represents temperature \( T \) in Kelvin (\( K \)).  
- The horizontal axis represents entropy \( s \) in \( \frac{\text{kJ}}{\text{kg·K}} \).  

Key features of the diagram:  
1. Several isobars are drawn, labeled with pressures \( p_0 = 0.191 \, \text{bar} \), \( p_1 \), and \( p_5 = 0.5 \, \text{bar} \).  
2. The process begins at point \( 0 \), which corresponds to ambient conditions.  
3. The compression process is shown as a steep curve from \( 0 \) to \( 3 \), indicating an increase in pressure and temperature.  
4. The combustion process is represented by a horizontal line from \( 3 \) to \( 4 \), showing an isobaric heat addition.  
5. The turbine process is depicted as a downward curve from \( 4 \) to \( 5 \), indicating a decrease in temperature and entropy.  
6. The mixing process occurs between \( 5 \) and \( 6 \), followed by the nozzle expansion from \( 6 \) back to ambient pressure \( p_0 \).  

Additional annotations:  
- The saturated regions are marked near the isobars.  
- Points \( 1 \), \( 2 \), \( 3 \), \( 4 \), \( 5 \), and \( 6 \) are clearly labeled along the process path.  
- Arrows indicate the direction of the process flow.  

This diagram visually represents the thermodynamic processes occurring in the jet engine, including compression, combustion, turbine expansion, and nozzle expansion.