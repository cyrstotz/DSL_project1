TASK 3a  
The pressure of the gas in state 1 is calculated using the formula:  
\[
p_1 = 1 \, \text{bar} + \frac{(m_K + m_{\text{mem}}) \cdot g}{\frac{(0.1 \, \text{m})^2 \cdot \pi}{4}}
\]  
Substituting the values:  
\[
p_1 = 1 \, \text{bar} + 0.40094 \, \text{bar} = 1.4 \, \text{bar}
\]  

The mass of the gas is determined using the ideal gas law:  
\[
p \cdot V = m \cdot R \cdot T
\]  
Rearranging for \( m \):  
\[
m = \frac{p \cdot V}{R \cdot T}
\]  
Substituting the values:  
\[
m = 3.92 \, \text{g}
\]  

---

TASK 3b  
The system is isolated, meaning the volume does not change, and neither does the mass. Therefore:  
\[
Q = 0
\]  
The energy balance is expressed as:  
\[
m_g \cdot (u_2 - u_1) = m_{\text{EW}} \cdot (u_1 - u_2)
\]  
This simplifies to:  
\[
m_g \cdot (u_2 - u_1) + m_{\text{EW}} \cdot (u_1 - u_2) = 0
\]  

---

TASK 3c  
The heat transferred between states 1 and 2 is calculated using:  
\[
Q_{12} = m \cdot (u_2 - u_1)
\]  
Substituting the values:  
\[
Q_{12} = m \cdot 0.633 \, \frac{\text{kJ}}{\text{kg·K}} \cdot (T_2 - T_1)
\]  
The result is:  
\[
Q_{12} = -1082.183 \, \text{kJ}
\]  

Note: The work in this task is referenced as being completed in a previous part of the exam.