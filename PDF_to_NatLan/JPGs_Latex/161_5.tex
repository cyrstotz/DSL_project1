TASK 4a  
The diagram is a pressure-temperature (\(p\)-\(T\)) graph illustrating the freeze-drying process.  
- The graph shows phase regions labeled as "liquid," "vapor," and "solid."  
- A curved line separates the liquid and vapor regions, representing the saturation curve.  
- The process steps are marked as "i" and "ii," with arrows indicating transitions.  
- Step "i" begins in the vapor region and moves toward the liquid region.  
- Step "ii" transitions from the liquid region toward the solid region.  
- The triple point is labeled, and the sublimation line is shown extending from the triple point into the solid-vapor region.  

TASK 4b  
Another diagram is provided, showing a pressure-temperature (\(p\)-\(T\)) graph with labeled points and transitions:  
- The graph includes regions labeled "liquid" (\(flüssig\)), "vapor" (\(dampf\)), and "solid" (\(fest\)).  
- Points 1, 2, and 3 are marked, with transitions between them.  
- Point 1 is in the liquid region, point 2 is in the vapor region, and point 3 is near the sublimation line.  
- The transitions are labeled "i" and "ii," corresponding to the freeze-drying process steps.  

TASK 4d  
The coefficient of performance (\(\epsilon_K\)) is defined as:  
\[
\epsilon_K = \frac{\dot{Q}_K}{\dot{W}_K}
\]  
The heat transfer rate (\(\dot{Q}_K\)) is calculated using:  
\[
\dot{Q}_K = \dot{m}_{\text{R134a}} \left( h_3 - h_1 \right)
\]  
Where:  
- \(h_3\) and \(h_1\) are enthalpies at states 3 and 1, respectively.  
- \(p_1 = p_3\).  

Additional calculations:  
- \(h_2 = h_g(8 \, \text{bar}) = g_2 \frac{R}{M}\), using Table A11.  
- \(s_2 = s_3\), and \(s_2 = s_g(T_i - 6^\circ\text{C})\).  
- Vapor quality (\(x_1\)) is calculated as:  
\[
x_1 = \frac{s_2 - s_f(8 \, \text{bar})}{s_g(8 \, \text{bar}) - s_f(8 \, \text{bar})}
\]  
Where \(s_f\) and \(s_g\) are entropy values for saturated liquid and vapor at 8 bar.  

Enthalpy at state 3 (\(h_3\)) is determined as:  
\[
h_3 = h_f(8 \, \text{bar}) + x \left( h_g(8 \, \text{bar}) - h_f(8 \, \text{bar}) \right)
\]  
Values for \(s_f\), \(s_g\), \(h_f\), and \(h_g\) are obtained from Table A11.  

No additional content is visible.