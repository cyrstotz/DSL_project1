TASK 4b  
The mass flow rate of the refrigerant \( \dot{m}_{\text{R134a}} \) is being calculated for an isentropic process.  

A diagram is drawn showing a compressor labeled "isentrop" with an inlet at \( x = 1 \) and an outlet at \( 8 \, \text{bar} \). The compressor is connected to a work input labeled \( 28 \, \text{W} \).  

The entropy at the outlet is equal to the entropy at the inlet:  
\[
s_2 = s_3
\]  
\[
s_2 = s_g(T_2)
\]  

Using the first law of thermodynamics for a steady-state flow process:  
\[
0 = \dot{m} [h_2 - h_3] + \sum \dot{Q} - \dot{W}_k
\]  
Since the process is adiabatic, \( \sum \dot{Q} = 0 \).  

Rearranging for the mass flow rate:  
\[
\dot{m} = \frac{\dot{W}_k}{h_2 - h_3}
\]  

The saturation temperature at \( 8 \, \text{bar} \) is given as:  
\[
T_{\text{sat}} = 31.33^\circ \text{C} \, @ \, 8 \, \text{bar}
\]  

---

TASK 4c  
The vapor quality \( x \) after the throttle is being determined.  

A schematic diagram of a throttle is drawn, showing an inlet at \( p_1 = 8 \, \text{bar} \) and an outlet at \( p_2 = 0 \, \text{bar} \).  

The enthalpy after the throttle remains constant:  
\[
h_4 = h_1
\]  

The vapor quality \( x \) is calculated using the formula:  
\[
\phi = \phi_f + x (\phi_g - \phi_f)
\]  

For an isenthalpic process:  
\[
x = \frac{h_4 - h_f}{h_g - h_f}
\]  

Values are substituted:  
\[
h_4 = h_1 = 264.15 \, \text{kJ/kg}
\]  
\[
h_g(p_2) = 732.02 \, \text{kJ/kg}, \, h_f(p_2) = 253.13 \, \text{kJ/kg}
\]  

The vapor quality is calculated as:  
\[
x = \frac{264.15 - 253.13}{732.02 - 253.13}
\]  

The enthalpy at \( 8 \, \text{bar} \) is referenced from a table:  
\[
h_g \, \text{from table at} \, 8 \, \text{bar} = 264.15 \, \text{kJ/kg}
\]  