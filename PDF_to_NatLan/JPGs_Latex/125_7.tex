TASK 4a

The page contains two diagrams related to the freeze-drying process described in Task 4.

1. **First Diagram**:
   - The graph is a qualitative representation of the freeze-drying process in a \( T \)-\( s \) diagram.
   - The axes are labeled as follows:
     - \( T \) (temperature) on the horizontal axis.
     - \( s \) (entropy) on the vertical axis.
   - The process consists of four distinct states labeled \( 1 \), \( 2 \), \( 3 \), and \( 4 \), forming a closed loop. 
     - State \( 1 \) to \( 2 \): A curved line indicating an increase in entropy and temperature.
     - State \( 2 \) to \( 3 \): A horizontal line indicating constant entropy.
     - State \( 3 \) to \( 4 \): A curved line showing a decrease in entropy and temperature.
     - State \( 4 \) to \( 1 \): A horizontal line indicating constant entropy.

2. **Second Diagram**:
   - The graph is a \( p \)-\( T \) diagram showing the phase regions of the refrigerant during the freeze-drying process.
   - The axes are labeled as follows:
     - \( T \) (temperature) on the horizontal axis.
     - \( p \) (pressure) on the vertical axis.
   - The diagram includes the phase boundaries:
     - "Fest" (solid phase) is labeled on the left side.
     - "Flüssig" (liquid phase) is labeled in the middle region.
     - "Gas/Förmig" (gas phase) is labeled on the right side.
   - Two steps are marked:
     - Step \( i \): A vertical line indicating a change in pressure at constant temperature.
     - Step \( ii \): A horizontal line indicating a change in temperature at constant pressure.

These diagrams visually represent the thermodynamic processes involved in the freeze-drying cycle.