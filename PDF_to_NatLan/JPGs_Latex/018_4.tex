TASK 3a  
The gas pressure \( p_{g,1} \) and mass \( m_g \) in state 1 are calculated as follows:

1. The cross-sectional area of the cylinder is determined using the formula for the area of a circle:  
\[
A = \left( \frac{D}{2} \right)^2 \pi = 0.007854 \, \text{m}^2
\]

2. The pressure exerted by the piston and EW (ice-water mixture) is calculated:  
\[
p_{\text{neu}} = \frac{F}{A} = \frac{m_K g}{A} + \frac{m_{\text{EW}} g}{A}
\]  
Substituting values:  
\[
p_{\text{neu}} = \frac{32 \cdot 9.81}{0.007854} + \frac{0.1 \cdot 9.81}{0.007854} = 40,004.35 \, \text{Pa} = 0.4 \, \text{bar}
\]

3. The total gas pressure is then:  
\[
p_{g,1} = p_0 + p_{\text{neu}} = 1 \, \text{bar} + 0.4 \, \text{bar} = 1.4 \, \text{bar}
\]

4. To calculate the gas mass \( m_g \), the ideal gas law is used:  
\[
p V = m R T
\]  
Rearranging for \( m \):  
\[
m = \frac{p V}{R T}
\]  
The specific gas constant \( R \) is calculated as:  
\[
R = \frac{\bar{R}}{M} = \frac{8.314}{50} = 0.16628 \, \text{kJ/kg·K}
\]  
Substituting values:  
\[
m_g = \frac{p_{g,1} V}{R T} = \frac{1.4 \, \text{bar} \cdot 3.14 \, \text{L}}{0.16628 \, \text{kJ/kg·K} \cdot (500 + 273.15) \, \text{K} \cdot 1000}
\]  
\[
m_g = 0.003427 \, \text{kg} = 3.42 \, \text{g}
\]

---

TASK 3b  
If \( x = 0 \), the mixture consists entirely of ice and water.  
If \( x = 1 \), the mixture consists of ice and gas.  
Thus, \( T_2 \) must be lower to maintain equilibrium.

---

TASK 3c  
The energy balance equation is written as:  
\[
\frac{dE}{dt} = \sum \dot{Q} - \sum \dot{W}
\]  
For work:  
\[
\Delta E = \Delta Q - \Delta W
\]  

The detailed energy balance is expressed as:  
\[
m_2 u_2 - m_1 u_1 + \Delta KE + \Delta PE = \Delta Q
\]  
Where \( u_2 \) and \( u_1 \) are the specific internal energies at states 2 and 1, respectively.  

Given:  
\[
u_1 \, (500^\circ\text{C}, 1.4 \, \text{bar}), \quad u_2 \, (0.003^\circ\text{C})
\]  

No further numerical solution is provided on the page.  

---  
Descriptions of diagrams or graphs are not present on this page.