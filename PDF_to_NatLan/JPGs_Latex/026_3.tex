TASK 1a  
The interpolation formula is given as:  
\[
y = \frac{y_2 - y_1}{x_2 - x_1} (x - x_1) + y_1
\]  

The energy balance equation for the reactor is:  
\[
0 = \dot{m} (h_{\text{ein}} - h_{\text{aus}}) + \dot{Q}_{\text{out}}
\]  
Rearranging to solve for \( \dot{Q}_{\text{out}} \):  
\[
\dot{Q}_{\text{out}} = \dot{m} (h_{\text{ein}} - h_{\text{aus}})
\]  
Substituting values, the heat flow removed by the coolant is calculated as:  
\[
\dot{Q}_{\text{out}} = -37.818 \, \text{kJ}
\]  

A graph is sketched showing a downward slope labeled "abgeführt" (removed).  

TASK 1b  
The enthalpy values at the given temperatures are:  
\[
h_{\text{ein}} \, (@70^\circ\text{C}) = 292.98 \, \frac{\text{kJ}}{\text{kg}}
\]  
\[
h_{\text{aus}} \, (@100^\circ\text{C}) = 418.04 \, \frac{\text{kJ}}{\text{kg}}
\]  

Using the energy balance equation:  
\[
0 = \dot{m} (h_{\text{ein}} - h_{\text{aus}}) - \dot{Q} + \dot{Q}_R
\]  
Rearranging to solve for \( \dot{Q}_{\text{out}} \):  
\[
\dot{Q}_{\text{out}} = \dot{m} (h_{\text{ein}} - h_{\text{aus}}) + \dot{Q}_R
\]  

Additional enthalpy values are provided for steam quality \( x = 0.005 \):  
\[
h_{\text{ein}} \, (@70^\circ\text{C}, x = 0.005) = 384.64 \, \frac{\text{kJ}}{\text{kg}}
\]  
\[
h_{\text{aus}} \, (@100^\circ\text{C}, x = 0.005) = 421.1 \, \frac{\text{kJ}}{\text{kg}}
\]  

The calculated heat flow values are:  
\[
\dot{Q}_{\text{out}} = 62.249 \, \text{kW}
\]  
\[
\dot{Q}_{\text{out}} = 65.05 \, \text{kW}
\]  