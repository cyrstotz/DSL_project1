TASK 2a  
The page begins with a labeled T-s diagram representing the thermodynamic processes of a jet engine. The vertical axis is labeled \( T \, [\text{K}] \), and the horizontal axis is labeled \( s \, [\text{kJ/kg·K}] \). The diagram includes six states (0, 1, 2, 3, 4, 5, and 6) connected by various processes.  

Description of the diagram:  
- State 0 to 1: A steep curve indicating an adiabatic, isentropic compression.  
- State 1 to 2: A vertical line representing an adiabatic, non-reversible process.  
- State 2 to 3: A horizontal line indicating an isobaric process.  
- State 3 to 4: A downward curve showing a non-reversible process.  
- State 4 to 5: A vertical line representing isentropic expansion.  
- State 5 to 6: A horizontal line indicating an isobaric process.  

Below the diagram, the processes are described in text:  
- \( 0 \to 1 \): Adiabatic and isentropic compression (\( \eta_{\text{comp}} < 1 \)), where \( p_1 > p_0 \) and \( T_1 > T_0 \).  
- \( 1 \to 2 \): Adiabatic, non-reversible process (\( s_1 \neq s_2 \)).  
- \( 2 \to 3 \): Isobaric process.  
- \( 3 \to 4 \): Non-reversible process.  
- \( 4 \to 5 \): Isentropic expansion (\( p_5 < p_4 \)).  
- \( 5 \to 6 \): Isobaric process.  

TASK 2b  
The task involves determining the outlet velocity \( w_6 \) and temperature \( T_6 \).  

The following equation is written:  
\[
w_6 = w_5 \cdot \frac{p_5}{p_0}
\]  
where \( \frac{p_5}{p_0} = 2.678 \).  

The corrected equation for \( w_6 \) is:  
\[
w_6 = w_5 \cdot \frac{p_5}{p_0} = 575.06 \, \text{m/s}
\]  

Crossed-out content: Several intermediate calculations and incorrect equations are crossed out and ignored in the transcription.  

No additional diagrams or figures are present beyond the T-s diagram described above.