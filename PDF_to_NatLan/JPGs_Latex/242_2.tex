TASK 2a  
The page contains two diagrams representing thermodynamic processes in a jet engine. Both diagrams are plotted on \( T \) (temperature) versus \( s \) (specific entropy) axes.  

### Diagram 1 (Top):  
- The diagram begins at point \( 0 \), representing ambient conditions (\( p_0 = 0.191 \, \text{bar} \)).  
- A vertical line from \( 0 \) to \( 2 \) indicates an adiabatic compression process.  
- The curve from \( 2 \) to \( 3 \) represents isobaric heat addition during combustion.  
- The process continues from \( 3 \) to \( 4 \) as an adiabatic expansion in the turbine.  
- The final curve from \( 4 \) to \( 5 \) represents mixing and heat exchange, ending at \( p = 0.5 \, \text{bar} \).  

### Diagram 2 (Bottom):  
- This diagram is more detailed and includes additional points.  
- The process starts at \( 0 \), representing ambient conditions.  
- A vertical line from \( 0 \) to \( 2 \) indicates adiabatic compression.  
- The curve from \( 2 \) to \( 3 \) represents isobaric heat addition during combustion.  
- The process continues from \( 3 \) to \( 4 \) as an adiabatic expansion in the turbine, ending at \( p = 0.5 \, \text{bar} \).  
- A mixing process occurs from \( 4 \) to \( 5 \), followed by a nozzle expansion from \( 5 \) to \( 6 \).  
- The label "clear axial gas flows" is written near the nozzle expansion.  

Both diagrams are qualitative representations of the jet engine cycle, showing the thermodynamic states and processes involved.