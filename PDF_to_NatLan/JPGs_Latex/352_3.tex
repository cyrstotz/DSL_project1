TASK 2a  
A T-s diagram is drawn, illustrating the thermodynamic process of a jet engine. The diagram includes labeled points corresponding to different states in the process:  
- Point 0: Ambient conditions.  
- Point 1: Air compression.  
- Point 2: Isentropic compression.  
- Point 3: Combustion.  
- Point 4: Expansion in the turbine.  
- Point 5: Mixing chamber.  
- Point 6: Nozzle exit.  

The x-axis represents entropy \( S \) in \( \text{kJ/kg·K} \), and the y-axis represents temperature \( T \). The process lines are drawn to show isentropic compression, isobaric combustion, and adiabatic expansion.  

Below the diagram, the text explains:  
At point 1, the air is compressed through the turbine process, while it is then directed to the mixing chamber.  

TASK 2b  
The energy balance equation is written as:  
\[
\dot{E} = \sum \dot{h} + \sum \dot{Q} - \sum \dot{W}
\]  
This equation accounts for enthalpy \( h \), heat transfer \( Q \), and work \( W \).  

Additionally, the equation for mass flow rate is provided:  
\[
0 = \dot{m} (h_{\text{in}} - h_{\text{out}})
\]  
This represents the steady-state energy balance for the system.