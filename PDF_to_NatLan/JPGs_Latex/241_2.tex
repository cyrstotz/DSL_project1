TASK 1d  
The table lists the temperatures for two states:  
- State 1: \( T = 100^\circ\text{C} \)  
- State 2: \( T = 70^\circ\text{C} \)  

The inlet temperature for the added mass is given as \( T_{\text{in,12}} = 20^\circ\text{C} \), and the task involves determining \( \Delta m_{12} \).  

A schematic diagram is drawn showing a system labeled "adiabatic system" with heat flow \( Q_{\text{out}} \) leaving the system, steam quality \( x = 0 \), and a reaction heat \( \dot{Q}_R \) entering the system. The inlet temperature \( T_{\text{in,12}} = 20^\circ\text{C} \) is noted.  

The energy balance equation is written as:  
\[
\Delta E = m_2 \cdot u_2 - m_1 \cdot u_1 + \Delta KE + \Delta PE  
\]  
Neglecting kinetic and potential energy changes, this simplifies to:  
\[
\Delta E = \Delta m_{12} \cdot u_{12} + Q_{\text{out}} + \dot{Q}_R  
\]  
The heat terms \( Q_{\text{out}} \) and \( \dot{Q}_R \) are crossed out, leaving:  
\[
m_2 \cdot u_2 - m_1 \cdot u_1 = \Delta m_{12} \cdot u_{12}  
\]  

Further simplification gives:  
\[
(m_1 + \Delta m_{12}) \cdot u_2 - m_1 \cdot u_1 = \Delta m_{12} \cdot u_{12}  
\]  

Using water table data (Table A-2):  
- \( u_2(x = 0, 70^\circ\text{C}) = u_f(70^\circ\text{C}) = 292.95 \, \text{kJ/kg} \)  
- \( u_1(x = 0.005, 100^\circ\text{C}) = u_f(100^\circ\text{C}) + x \cdot u_{fg}(100^\circ\text{C}) = 423.64 \, \text{kJ/kg} \)  
- \( u_{12}(x = 0, 20^\circ\text{C}) = u_f(20^\circ\text{C}) = 83.96 \, \text{kJ/kg} \)  

These values are used to calculate \( \Delta m_{12} \).