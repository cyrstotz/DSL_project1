TASK 4a  
The diagram shows a pressure-temperature (\( p \)-\( T \)) plot for the freeze-drying process. The graph includes labeled regions for "Dampf" (vapor), "Nassdampf" (wet vapor), and "Sumpf" (liquid). The process steps are marked as points 1, 2, and 3, with arrows indicating transitions between states. The temperature \( T_i \) is given as \( 10^\circ\text{C} \).  

TASK 4b  
The energy balance for the compressor is written as:  
\[
0 = \dot{m}(h_1 - h_2) + \dot{Q} - \dot{W}
\]  
At state 4, \( x_4 = 0 \) at 8 bar. Using Table A-11, \( h_4 = 93.42 \, \text{kJ/kg} \).  
For state 2, \( T_2 = T_1 - 6 \, \text{K} \), and \( T_2 = -273.15 \, \text{K} \).  

TASK 4c  
The vapor quality \( x_1 \) is calculated at \( T_2 = -22^\circ\text{C} \):  
\[
x_1 = \frac{r_{134a}}{4.25 \, \text{kJ/kg}}
\]  
The energy balance for the expansion valve is given as:  
\[
0 = \dot{m}(h_2 - h_3) + \dot{Q} - \dot{W}
\]  
At state 3, \( h_3 = 81 \, \text{kJ/kg} \) and \( x_3 = 0 \). Using Table A-11, \( h_4 = 93.42 \, \text{kJ/kg} \).  

TASK 4d  
The coefficient of performance (\( \epsilon_K \)) is calculated as:  
\[
\epsilon_K = \frac{\dot{Q}_K}{\dot{W}_K} = \frac{10 \, \text{kJ}}{25 \, \text{W}}
\]  
Interpolation is used to find \( h_2 \) at the outlet pressure, assuming \( h_2 = h_4 \) due to isenthalpic conditions.  

TASK 4e  
The temperature \( T_i \) would remain nearly constant during Step ii, as the cooling cycle continues with constant \( \dot{Q}_K \).