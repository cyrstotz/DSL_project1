TASK 4a  
A pressure-temperature (\( p \)-\( T \)) diagram is drawn, showing the phase regions for solid, liquid, and gas. The curve represents the phase boundaries, with labels indicating "Fest" (solid), "Flüssig" (liquid), and "Gas" (gas). The diagram includes a marked point labeled "Schluss" (end) and a region labeled "noch nicht gefroren" (not yet frozen). The temperature axis is labeled \( T \), and the pressure axis is labeled \( p \).  

TASK 4b  
A table is presented with columns labeled \( P \), \( V \), \( h \), \( x \), and \( S \). The rows contain values for different states:  
- Row 1: \( P = 1 \), \( V = \), \( h = \), \( x = 1 \), \( S = 0.9294 \)  
- Row 2: \( P = 2 \), \( V = \), \( h = \), \( x = 1 \), \( S = 0.9298 \)  
- Row 3: \( P = 8 \), \( V = \), \( h = \), \( x = 1 \), \( S = 0.9298 \)  
- Row 4: \( P = 8 \), \( V = \), \( h = \), \( x = 0 \), \( S = \)  

Additional notes are written:  
- \( T_i = -70^\circ\text{C} \), \( T_2 = -76^\circ\text{C} \)  
- \( h_2 = 237.74 \), \( S_2 = 0.9298 \)  

TASK 4b (continued calculations)  
The mass flow rate is calculated using the formula:  
\[
\dot{m} (\Delta h) = \dot{W}_{23}
\]  
Rearranging for \( \dot{m} \):  
\[
\dot{m} = \frac{\dot{W}}{h_3 - h_2}
\]  
Substituting values:  
\[
\dot{m} = \frac{28 \, \text{W}}{277.91 - 237.74} = 0.000634 \, \text{kg/s}
\]  
Converting to hourly flow rate:  
\[
\dot{m} \cdot 3600 = 3.0 \, \text{kg/h}
\]  

TASK 4b (additional notes)  
The temperature \( T_3 \) is calculated using interpolation:  
\[
T_3 = y_1 + \frac{(x_0 - x_1)}{x_2 - x_1} (y_2 - y_1)
\]  
Where \( x \) represents temperature and \( y \) represents enthalpy.  

Further notes:  
1) \( T \) at \( T_{\text{sat}} \)  
2) \( T \) at \( 40^\circ\text{C} \)  

TASK 4b (final values)  
The temperature \( T_3 \) is determined to be:  
\[
T_3 = 37.86^\circ\text{C}
\]  
The enthalpy \( h_3 \) is given as:  
\[
h_3 = 277.37
\]  

Additional clarification:  
- \( x \) represents temperature.  
- \( y \) represents enthalpy.  
1) \( T \) at \( T_{\text{sat}} \)  
2) \( T \) at \( 40^\circ\text{C} \)  