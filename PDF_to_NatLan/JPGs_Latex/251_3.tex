TASK 3a  
The gas pressure \( p_g \) is calculated using the formula:  
\[
p_g = \frac{F + p_{\text{amb}} \cdot m_g + p_{\text{amb}} \cdot m_{\text{EW}}}{A} = \frac{32 \cdot 10^3}{\pi \cdot (0.1)^2} + 1.5 \, \text{bar} = 1.5 \, \text{bar} \, \text{(approx.)}.
\]  

The mass of the gas \( m_g \) is determined as:  
\[
m_g = m_{\text{EW}} + m_K = 32.1 \, \text{kg}.
\]  

The ideal gas law is used to calculate \( m_g \):  
\[
m_g = \frac{p_g \cdot V}{R \cdot T_1} = 3.663 \, \text{kg}.
\]  

The gas constant \( R \) is calculated as:  
\[
R = \frac{1}{M} = 0.1663 \, \frac{\text{J}}{\text{g·K}} = 166.263 \, \frac{\text{J}}{\text{kg·K}}.
\]  

---

TASK 3b  
The temperature of the ice-water mixture \( T_{\text{EW}} \) remains constant at \( 0^\circ\text{C} \) because the system is in the two-phase region.  

The pressure \( p_2 \) equals \( p_1 \) because the solid and liquid phases are incompressible.  

---

TASK 3c  
The energy change \( \Delta E \) is calculated as:  
\[
\Delta E = E_2 - E_1 = Q_{12} = m_g \cdot (\Delta u) = m_g \cdot c_p \cdot (T_2 - T_1).
\]  

The internal energy change \( \Delta u \) is expressed as:  
\[
\Delta u = c_p \cdot (\Delta T).
\]  

The specific heat capacity \( c_p \) is calculated as:  
\[
c_p = R + c_V = 0.7553 \, \frac{\text{kJ}}{\text{kg·K}}.
\]  

The heat transferred \( Q_{12} \) is:  
\[
Q_{12} = 2.225 \, \text{kJ}.
\]  

---

TASK 3d  
The final ice fraction \( x_{\text{ice},2} \) is calculated using the formula:  
\[
x_{\text{ice},2} = \frac{u_{E2} - u_f}{u_g - u_f}.
\]  

The internal energy \( u_{E2} \) is expressed as:  
\[
u_{E2} = u_{E1} + \frac{Q_{12}}{m_{\text{EW}}}.
\]  

The internal energy \( u_{E1} \) is calculated as:  
\[
u_{E1} = u_f + x \cdot u_g.
\]  

The final expression for \( u_{E2} \) is:  
\[
u_{E2} = (1 - x) \cdot u_f + x \cdot u_{\text{next}}.
\]  