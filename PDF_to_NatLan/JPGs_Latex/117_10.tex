TASK 4c  
The vapor quality \( x_1 \) of the refrigerant at state 1 is being calculated. The following steps and equations are used:

1. The mass flow rate of the refrigerant is given as \( \dot{m}_{\text{R134a}} = 4 \, \text{kg/h} \), and the temperature \( T_2 = -22^\circ\text{C} \).  
   The inlet temperature \( T_i \) is calculated as:  
   \[
   T_i = T_2 + 6 \, \text{K} = -16^\circ\text{C}
   \]

2. Using the steady-state energy balance for the process from state 4 to state 1:  
   \[
   0 = \dot{m}_{\text{R134a}} \cdot (h_4 - h_1) + \frac{w_4^2 - w_1^2}{2} + g(z_4 - z_1) + \dot{Q}_K - \dot{W}_K
   \]  
   Kinetic energy (\( ke \)), potential energy (\( pe \)), and additional heat transfer terms are neglected, simplifying the equation to:  
   \[
   h_4 = h_1
   \]  
   The enthalpy at state 4 is given as \( h_4 = h_f(8 \, \text{bar}) = 93.42 \, \frac{\text{kJ}}{\text{kg}} \), so \( h_1 = h_4 = 93.42 \, \frac{\text{kJ}}{\text{kg}} \).

3. For the process from state 1 to state 2, the steady-state energy balance is applied:  
   \[
   0 = \dot{m}_{\text{R134a}} \cdot (h_1 - h_2) + \frac{w_1^2 - w_2^2}{2} + g(z_1 - z_2) + \dot{Q}_K - \dot{W}_K
   \]  
   Neglecting kinetic energy, potential energy, and work terms, the equation simplifies to:  
   \[
   \dot{Q}_K = \dot{m}_{\text{R134a}} \cdot (h_2 - h_1)
   \]  
   The enthalpy at state 2 is given as \( h_2 = h_f(-22^\circ\text{C}) = 234.08 \, \frac{\text{kJ}}{\text{kg}} \).  

4. The heat transfer rate \( \dot{Q}_K \) is calculated as:  
   \[
   \dot{Q}_K = \frac{4 \, \text{kg}}{3600 \, \text{s}} \cdot \left( 234.08 \, \frac{\text{kJ}}{\text{kg}} - 93.42 \, \frac{\text{kJ}}{\text{kg}} \right)
   \]  
   \[
   \dot{Q}_K = 0.75629 \, \text{kW} = 756.29 \, \text{W}
   \]

5. The vapor quality \( x_1 \) is calculated using the formula:  
   \[
   x_1 = \frac{h_1 - h_f}{h_{fg} - h_f}
   \]  
   However, the calculation is incomplete, and no final value for \( x_1 \) is provided.

No diagrams or figures are present on this page.