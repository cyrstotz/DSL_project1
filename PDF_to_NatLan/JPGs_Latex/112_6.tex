TASK 4a  
The first diagram is a pressure-temperature (\(p\)-\(T\)) diagram illustrating the freeze-drying process. It includes labeled phase regions for solid, liquid, and gas. The process path is marked with states 1, 2, 3, and 4, showing transitions between phases. The refrigerant undergoes evaporation, compression, condensation, and expansion, forming a closed cycle. The sublimation region is clearly indicated.

The second diagram is another \(p\)-\(T\) diagram, focusing on the phase transitions. It highlights the sublimation process and the refrigerant cycle, with states labeled as 1, 2, 3, and 4. The diagram includes a shaded area representing the isobaric evaporation step.

TASK 4b  
The energy balance for the compressor is given as:  
\[
0 = \dot{W}_{\text{R134a}} (h_2 - h_3) - \dot{W}_K
\]  
where:  
\[
\dot{W}_{\text{R134a}} = \frac{\dot{W}_K}{h_2 - h_3}
\]  
The enthalpy values are defined as:  
\[
h_2 = h_g
\]  
\[
h_3 = h_f + v_f \cdot (p - p_{\text{sat}})
\]  
The pressure \(p\) is given as 8 bar.

TASK 4c  
The vapor quality \(x_1\) at state 1 is calculated using:  
\[
x_1 = \frac{h_1 - h_f}{h_g - h_f}
\]