TASK 3c  
The energy balance for the gas container is expressed using the first law of thermodynamics:  
\[
\frac{dE}{dt} = \dot{Q}_{12}
\]  

The heat transfer \( Q_{12} \) is calculated as:  
\[
m (U_2 - U_1) = Q_{12}
\]  
Substituting the internal energy difference in terms of specific heat capacity and temperature change:  
\[
m \left( c_V^{\text{ig}} (T_2 - T_1) \right) = Q_{12}
\]  

Given:  
\[
m = 0.00342 \, \text{kg}, \quad c_V^{\text{ig}} = 0.633 \, \frac{\text{kJ}}{\text{kg·K}}, \quad T_2 = 273.15 \, \text{K}, \quad T_1 = 773.15 \, \text{K}
\]  

The heat transfer is calculated as:  
\[
Q_{12} = 0.00342 \cdot 0.633 \cdot (273.15 - 773.15) = -1082 \, \text{J}
\]  

The negative sign indicates that heat flows out of the system boundary (from the gas to the EW mixture).  

---

TASK 3d  
The internal energy \( u_2 \) is equal to \( u_1 \) because the temperature remains constant.  

The ice mass fraction in state 1 is given as:  
\[
x_{\text{ice},1} = \frac{m_{\text{ice}}}{m_{\text{EW}}} = 0.6
\]  

The internal energy \( u_2 \) is calculated using the mixture formula:  
\[
u_2 = u_f + x (u_g - u_f)
\]  

The ice mass in state 1 is calculated as:  
\[
m_{\text{ice},1} = 0.6 \cdot 0.1 \, \text{kg} = 0.06 \, \text{kg}
\]  