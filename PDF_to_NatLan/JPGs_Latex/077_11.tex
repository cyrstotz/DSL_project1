TASK 4b  
The mass flow rate \( \dot{m} \) is calculated using the equation:  
\[
\dot{m} = \frac{28 \, \text{W}}{h_2 - h_3}
\]  
where \( h_2 - h_3 \) represents the enthalpy difference.  

At \( x_2 = 1 \), the refrigerant is fully vaporized (saturated vapor).  

---

TASK 4c  
To determine \( T_i \):  
The pressure is 5 mbar below the triple point, and the temperature is 10 K above the sublimation temperature.  
Thus, \( p = 1 \, \text{mbar} \) and \( T_i = -10^\circ\text{C} = 263 \, \text{K} \).  

The evaporator temperature is given as:  
\[
T_{\text{Verdampfer}} = 257 \, \text{K}
\]  

The enthalpy values are:  
\[
h_2 = h_{2g}, \quad h_3 = h_{3,f} \, (\text{fully compressed at 8 bar})
\]  
From the table (A.11):  
\[
h_{3,f} = h_3 = 93.42 \, \text{kJ/kg}
\]  

---

TASK 4d  
The entropy between states 2 and 3 remains constant:  
\[
s_2 = s_3
\]  
From the table:  
\[
s_3 = s_{3,f} = 0.3459 \, \text{kJ/kg·K}
\]  

At \( x_2 = 1 \), the entropy values are:  
\[
s_{3,f} = s_{2,g} = s_2 = 0.3459 \, \text{kJ/kg·K}
\]  

The pressure \( p_2 \) needs to be determined.