TASK 4a  
The initial temperature is given as \( T_i = 0^\circ\text{C} \).  

Two graphs are drawn:  
1. The first graph is a pressure-volume (\( p \)-\( v \)) diagram. It shows several curves, including one labeled "nicht beachtet" (not considered). The curves represent different thermodynamic states, with one curve crossing over a peak and others diverging.  
2. The second graph is also a \( p \)-\( v \) diagram. It includes labeled regions for "gasförmig" (gaseous), "flüssig" (liquid), and "fest" (solid). A rectangular cycle is drawn with points labeled 1, 2, 3, and 4, representing transitions between phases. The cycle moves through the regions of "flüssig" (liquid) and "gasförmig" (gaseous).  

TASK 4b  
The refrigerant does not evaporate completely at \(-6^\circ\text{C}\).  
At \(-6^\circ\text{C}\), the refrigerant fully evaporates.  

Reference is made to Table A10.