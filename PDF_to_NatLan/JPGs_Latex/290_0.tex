TASK 4a  
Two diagrams are drawn to represent the freeze-drying process.  

1. **First Diagram**:  
   - The graph is labeled with axes: pressure \( P \) (in bar) on the vertical axis and specific volume \( v \) (in \( \text{m}^3/\text{kg} \)) on the horizontal axis.  
   - The diagram shows phase regions, including saturated liquid and saturated vapor.  
   - Four states are marked: \( T_i \), \( T_4 \), \( T_3 \), and \( T_1 \).  
   - Processes are labeled as "isobaric" and "isentropic."  
   - The student writes "sorry, falsch gelesen" (sorry, misread) with a sad face.  

2. **Second Diagram**:  
   - The graph is labeled with axes: pressure \( P \) (in bar) on the vertical axis and temperature \( T \) (in Kelvin) on the horizontal axis.  
   - Phase regions are depicted, including saturated liquid and saturated vapor.  
   - Four states are marked: \( x = 0 \) at state 4 and \( x = 1 \) at state 1.  
   - Processes are labeled as "isobaric" and "isentropic."  

TASK 4b  
The refrigerant used is R134a, and the process is described as stationary and isentropic.  

- The energy balance equation is written as:  
  \[
  \dot{E} = \dot{m} \left[ h_2 - h_3 \right] - \dot{W}_K
  \]  
- Initial conditions:  
  - \( T_i = 10 \, \text{K} \) above the sublimation temperature.  
  - \( T_i = -10^\circ\text{C} \), \( T_4 = -16^\circ\text{C} \).  

- Calculations:  
  - \( h_2 (x=1, T_2=-16^\circ\text{C}) = 237.74 \, \text{kJ/kg} \) (from Table A-10).  
  - \( h_3 (x=0, P_3=8 \, \text{bar}) = h(T=37.375^\circ\text{C}) + \dots = 266.497 \, \text{kJ/kg} \) (from Table A-10).  
  - \( s_2 = 0.929875 \, \text{kJ/kg·K} \) (from Table A-10).  
  - \( s_3 = 0.9066 + 0.9374 \dots = 0.9374 \, \text{kJ/kg·K} \) (from Table A-12).  

- Mass flow rate calculation:  
  \[
  \dot{m} = \frac{237.74 - 266.497}{-28.757} = 0.974 \, \text{kg/s} = 3.5 \, \text{kg/h}
  \]  

TASK 4c  
The vapor quality \( x \) at state 1 is calculated.  

- Conditions:  
  - \( P_1 = P_2 \), \( T_2 = -16^\circ\text{C} \), \( P_4 = 8 \, \text{bar} \), \( x_4 = 0 \).  

- Heat transfer:  
  \[
  \dot{Q} = \dot{m} \left( h_4 - h_3 \right) = 0.974 \cdot 10 \cdot \left( 93.42 - 266.497 \right) = -1709 \, \text{W}
  \]  

- Enthalpy at state 4:  
  \[
  h_4 (x=0, P_4=8 \, \text{bar}) = 95.42 \, \text{kJ/kg} \quad \text{(from Table A-11)}  
  \]  

TASK 4d  
The coefficient of performance \( \epsilon_K \) is calculated using the formula:  
\[
\epsilon_K = \frac{\dot{Q}_K}{\dot{W}_K}
\]  

- Heat transfer:  
  \[
  \dot{Q}_K = m \left( h_2 - h_4 \right)  
  \]  
- The student notes that \( h_4 \) is uncertain due to missing data.  

No further calculations are provided for \( \epsilon_K \).  

TASK 4e  
The student writes a question about how \( T_i \) evolves in Step ii if the cooling cycle from Step i continues with constant \( \dot{Q}_K \).  

No content found for this subtask.