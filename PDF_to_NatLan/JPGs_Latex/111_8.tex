TASK 4c  
The process described is an isenthalpic throttling (adiabatic). The enthalpy remains constant during throttling:  
\[
h_u = h_1
\]  
The enthalpy at state 1 is calculated using the formula:  
\[
h_{g,1} = h_f + x_1 (h_g - h_f)
\]  
where \( x_1 \) is the vapor quality, \( h_g \) is the enthalpy of the vapor phase, and \( h_f \) is the enthalpy of the liquid phase.  

Given:  
- \( p_2 = 1.2192 \, \text{bar} \)  
- \( T_2 = -22^\circ\text{C} \) (state 2 is in two-phase region)  

From the tables:  
\[
h_f = 211.77 \, \frac{\text{kJ}}{\text{kg}}, \quad h_g = 234.08 \, \frac{\text{kJ}}{\text{kg}}
\]  

The vapor quality \( x_1 \) can be calculated as:  
\[
x_1 = \frac{h_u - h_f}{h_g - h_f}
\]  

TASK 4d  
The coefficient of performance \( \epsilon_K \) is defined as:  
\[
\epsilon_K = \frac{|\dot{Q}_K|}{|\dot{W}_K|} = \frac{|\dot{Q}_K|}{|\dot{W}_{\text{ind}}|}
\]  

Using the energy balance:  
\[
\dot{Q}_K = \dot{m}_{\text{R134a}} \left[ h_1 - h_2 \right]
\]  

Substituting values:  
\[
\dot{Q}_K = 4 \, \left[ \ldots \right]
\]  

(Note: The calculation is incomplete and the final numerical values are not provided.)