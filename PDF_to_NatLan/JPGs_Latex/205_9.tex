TASK 4b  
The sublimation point of water is determined as 5 mbar below the triple point. The initial temperature \( T_i \) is given as \( T_i = 10^\circ\text{C} \).  

The total temperature difference is calculated as:  
\[
T_{\text{total}} = T_i - T_{\text{triple}} = 10^\circ\text{C} - 0^\circ\text{C} = 10^\circ\text{C}
\]  

From the tables:  
- \( h_2 = h_g(10^\circ\text{C}) \) is found in Table A-10.  
- \( h_2 = 2337.94 \, \text{kJ/kg} \).  

The mass flow rate is calculated as:  
\[
\dot{m}_{\text{R134a}} = \dot{Q}_K / h_2 = 0.529 \, \text{kg/s}
\]  

TASK 4c  
Interpolation is performed to determine \( h_3 \) using the values from Table A-12:  
- \( s_2 = s_g(10^\circ\text{C}) \) and \( s_3 = s_g(8 \, \text{bar}) \).  
- \( h_g(1333 \, \text{Pa}) = 0.9066 \, \text{kJ/kg} \).  
- \( h_g(1400 \, \text{Pa}) = 0.9374 \, \text{kJ/kg} \).  

Interpolating for \( h_3 \):  
\[
h_3 = h_g(1333 \, \text{Pa}) + \frac{(h_g(1400 \, \text{Pa}) - h_g(1333 \, \text{Pa}))}{(1400 - 1333)} \cdot (1333 - 1333)
\]  
\[
h_3 = 2743.43 \, \text{kJ/kg}
\]  

TASK 4d  
The coefficient of performance \( \epsilon_K \) is calculated as:  
\[
\epsilon_K = \frac{\dot{Q}_K}{\dot{W}_K}
\]  

No diagrams or graphs are present on this page.