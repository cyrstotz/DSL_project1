TASK 3a  
The gas pressure \( p_{g,1} \) and mass \( m_g \) are calculated as follows:  

1. The equilibrium pressure \( p_{\text{Eis}} \) is determined using the formula:  
\[
p_{\text{Eis}} = p_{\text{gas}} \quad \text{(since equilibrium is assumed)}  
\]  
The total pressure \( p_a \) is given by:  
\[
p_a + \frac{m_K \cdot g}{\frac{\pi \cdot (0.1)^2}{4}} = p_{\text{Eis}}
\]  
Substituting values:  
\[
1.10^5 + \frac{32 \cdot 9.81}{\frac{\pi \cdot (0.1)^2}{4}} - p_{\text{Eis}} = 1.399969 \, \text{bar}
\]  
Thus, \( p_{g,1} = 1.4 \, \text{bar} \).  

2. The mass of the gas \( m_g \) is calculated using the ideal gas law:  
\[
m_g = \frac{p_1 \cdot V}{R \cdot T_1}
\]  
Where \( R = \frac{R_{\text{universal}}}{M} = \frac{8314}{50} = 166.28 \).  
Substituting values:  
\[
m_g = \frac{1.4 \cdot 10^5 \cdot 3.14 \cdot 0.001}{166.28 \cdot (500 + 273.15)} = 0.003418 \, \text{kg} = 3.418 \, \text{g}.
\]  

---

TASK 3b  
Since no additional forces are acting on the system, the process in the cylinder is isobaric, and the density of water remains constant. Therefore, \( p_{Eis} = p_{g,1} = 1.4 \, \text{bar} \).  

In state 2, the ice-water mixture is in equilibrium, and the temperature \( T_{2w} = T_{2g} = 0^\circ\text{C} \). The system remains in thermodynamic equilibrium. Hence:  
\[
T_{2w} = T_{2g} = 0^\circ\text{C}.
\]  

---

TASK 3c  
The heat transferred \( Q_{12} \) from the gas to the ice-water mixture is calculated as follows:  

1. The change in internal energy \( \Delta U \) is given by:  
\[
\Delta U = \dot{m} \cdot c_V \cdot (T_2 - T_1) = Q_{12}.
\]  
Potential energy changes are neglected.  

2. Substituting values:  
\[
Q_{12} = 3.418 \cdot 10^{-3} \cdot 0.633 \cdot 10^3 \cdot (0 - 500^\circ\text{C}) = -1,087.55 \, \text{kJ}.
\]  
Converting to joules:  
\[
Q_{12} = -1,087,554 \, \text{J}.
\]  

Thus, the heat transferred is:  
\[
Q_{12} = -1,087,554 \, \text{J}.
\]