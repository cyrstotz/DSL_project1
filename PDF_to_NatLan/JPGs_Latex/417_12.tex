TASK 4a  
A graph is drawn on a \( T \)-\( p \) diagram. The graph shows a cycle with four labeled points: 1, 2, 3, and 4.  
- Point 1 is at the lower left, representing the starting state.  
- Point 2 is higher on the \( T \)-axis, indicating an increase in temperature.  
- Point 3 is at the peak of the cycle, showing the highest temperature and pressure.  
- Point 4 is on the downward slope, returning to lower pressure and temperature.  
The cycle appears to represent a thermodynamic process, possibly related to a refrigeration or freeze-drying cycle.  

TASK 4b  
The energy balance equation is written as:  
\[
\frac{dE}{dt} = \sum \dot{m}_i (h_e - h_k) - \sum \dot{Q}_i - \sum \dot{W}_i
\]  
The equation for \( Q \) is given as:  
\[
Q = \dot{m}_i (h_3 - h_2) + \dot{W}_K
\]  
The mass flow rate \( \dot{m}_{\text{R134a}} \) is calculated as:  
\[
\dot{m}_{\text{R134a}} = -\frac{\dot{W}_K}{h_3 - h_2}
\]  
It is noted that \( h_3 \) corresponds to a pressure of 8 bar.  

TASK 4c  
The internal energy \( u \) is expressed as:  
\[
u = u_f + x (u_g - u_f)
\]  
The vapor quality \( x \) is calculated using:  
\[
x = \frac{u - u_f}{u_g - u_f}
\]