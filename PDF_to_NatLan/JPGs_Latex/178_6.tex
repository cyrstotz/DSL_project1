TASK 3a  
The pressure \( p \) is calculated using the formula:  
\[
p = p_{\text{amb}} + \frac{F}{\pi r^2}
\]  
Substituting values:  
\[
p = p_{\text{amb}} + \frac{m_K \cdot g}{\pi r^2}
\]  
\[
p = 10^5 + \frac{32 \cdot 9.81}{\pi \cdot (5 \cdot 10^{-2})^2}
\]  
\[
p = 1.4 \, \text{bar}
\]  

The diameter \( D \) is given as \( D = 2r \), with \( r = 5 \, \text{cm} \).  

The mass \( m_g \) is calculated using the ideal gas law:  
\[
m_g = \frac{p \cdot V}{R \cdot T}
\]  
The specific gas constant \( R \) is determined as:  
\[
R = \frac{R_{\text{univ}}}{M_g} = \frac{8314}{50 \cdot 10^{-3}} = 166.28 \, \text{J/kg·K}
\]  
Substituting values:  
\[
m_g = \frac{1.4 \cdot 10^5 \cdot 3.14 \cdot 10^{-3}}{166.28 \cdot (500 + 273.15)}
\]  
\[
m_g = 0.34 \, \text{kg}
\]  

TASK 3b  
The ice fraction \( x_{\text{ice}} > 0 \).  

The pressure exerted by the gas is equal to the pressure exerted by the ice:  
\[
p_{g,2} = 1.4 \cdot 10^5 \, \text{Pa} = 1.4 \, \text{bar}
\]  
The temperature \( T_{g,2} \) is \( 0^\circ\text{C} \), as per the equilibrium table.  

TASK 3c  
The energy balance equation is written as:  
\[
\frac{dE}{dt} = \dot{Q} + \dot{m} \cdot h_2 - \dot{m} \cdot h_1
\]  
This simplifies to:  
\[
\Delta Q = m_g \cdot c_V \cdot (T_1 - T_2)
\]  
Substituting values:  
\[
\Delta Q = 0.633 \cdot (500 - 0) \cdot 0.005
\]  
\[
\Delta Q = 316.5 \, \text{J}
\]  

TASK 3d  
The final ice fraction \( x_{\text{ice},2} \) is calculated using the equilibrium table.  
\[
Q = m_{\text{EW}} \cdot L
\]  
Substituting values:  
\[
Q = 0.1 \cdot 36 \cdot 10^3 - 316.5
\]  
\[
Q = 3596.5 \, \text{J}
\]  
The final ice fraction is determined from the table as \( x_{\text{ice},2} = 0.005 \).