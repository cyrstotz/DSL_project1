TASK 4a  
A pressure-temperature (\(p\)-\(T\)) diagram is drawn to represent the freeze-drying process. The diagram includes the following features:  
- The vertical axis is labeled as \(p\) in mbar.  
- The horizontal axis is labeled as \(T\) in \(^\circ\text{C}\).  
- Three regions are marked: "Fest" (solid), "Flüssig" (liquid), and "Gas" (gas).  
- The sublimation point is indicated on the curve separating the solid and gas regions.  
- The triple point is labeled, with a pressure of 5 mbar below it.  
- Two steps are shown:  
  - Step I involves a vertical line indicating cooling at constant pressure.  
  - Step II involves a horizontal line indicating sublimation at constant temperature, 10 K above the sublimation temperature.  

TASK 4b  
The mass flow rate of the refrigerant R134a is given as:  
\[
\dot{m}_{\text{R134a}} = 4 \, \frac{\text{kg}}{\text{h}}, \quad T_2 = -22^\circ\text{C}.
\]  

The pressure at state 2 is calculated as:  
\[
p_2 = p(-22^\circ\text{C}) = 1.2192 \, \text{bar}.
\]  

Since \(T_1 = T_2\), it follows that:  
\[
h_1 = h_2.
\]  

The enthalpy at state 4 is determined using the isenthalpic throttling process:  
\[
h_4 = h_1.
\]  

From the refrigerant tables:  
\[
h_{f}(8 \, \text{bar}) = 93.42 \, \frac{\text{kJ}}{\text{kg}}.
\]  

The vapor quality at state 1 is calculated as:  
\[
x_1 = \frac{h_4 - h_f(-22^\circ\text{C})}{h_{fg}(-22^\circ\text{C})} \approx 0.337.
\]  

TASK 4d  
The coefficient of performance (\(\epsilon_K\)) for the cooling cycle is expressed as:  
\[
\epsilon_K = \frac{\dot{Q}_K - \dot{Q}_{\text{AB}}}{\dot{W}_K}.
\]  