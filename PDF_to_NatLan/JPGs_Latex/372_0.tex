TASK 4a  
Two graphs are drawn to represent thermodynamic processes.  

1. **First graph**:  
   - The graph is labeled \( p \) (pressure) on the vertical axis and \( T \) (temperature) on the horizontal axis.  
   - The curve shows a cyclic process with multiple states labeled as 2, 3, 4, and 5.  
   - The region is divided into "Flüssig" (liquid) and "Nassdampf" (wet steam).  
   - The curve transitions between these regions, indicating phase changes.  

2. **Second graph**:  
   - The graph is also labeled \( p \) (pressure) on the vertical axis and \( T \) (temperature) on the horizontal axis.  
   - States 2, 3, and 4 are marked, with arrows indicating transitions between these states.  
   - The region is similarly divided into "Flüssig" (liquid) and "Nassdampf" (wet steam).  
   - The curve represents a thermodynamic cycle involving phase changes.  

TASK 4b  
The following equations and calculations are written:  

1. The heat transfer equation:  
\[
Q = \dot{m} \cdot (h_2 - h_3) + \dot{W}
\]  

2. A simplified form of the equation:  
\[
Q = \dot{m} \cdot (h_2 - h_3)
\]  

3. The entropy relationship:  
\[
s_3 = s_2 = s_{2, \text{liquid}}
\]  

4. The temperature at state 2 is given:  
\[
T_i = 20^\circ\text{C} = 293.15 \, \text{K}
\]  

5. Additional calculations and expressions are partially crossed out and illegible.  

No further descriptions or explanations are provided.