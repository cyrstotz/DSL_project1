TASK 4a  
Two graphs are drawn to represent the freeze-drying process in a \( p \)-\( T \) diagram.  

1. **First graph**:  
   - The vertical axis is labeled \( p \) (pressure), and the horizontal axis is labeled \( T \) (temperature).  
   - The graph shows phase regions labeled "ice," "water," and "vapor."  
   - A curve separates the "ice" and "water" regions, and another curve separates "water" and "vapor."  
   - A rectangular box is drawn in the "vapor" region, labeled with state points corresponding to the refrigeration cycle.  

2. **Second graph**:  
   - Similar axes are used: \( p \) (pressure) on the vertical axis and \( T \) (temperature) on the horizontal axis.  
   - The graph shows the refrigeration cycle with states labeled \( 1 \), \( 2 \), \( 3 \), and \( 4 \).  
   - The cycle is represented as a closed loop, with arrows indicating the direction of the process.  

TASK 4b  
The equation for the refrigerant mass flow rate \( \dot{m}_{\text{R134a}} \) is written as:  
\[
0 = \dot{m}_{\text{R134a}} (h_2 - h_3) - W_K
\]  
Where:  
- \( W_K = 28 \, \text{W} \)  

TASK 4c  
The vapor quality \( x_1 \) at state 1 is calculated using the formula:  
\[
x_1 = \frac{\dot{m}_{\text{R134a}} (h_4 - h_{c,1})}{h_{g,1} - h_{c,1}}
\]  

Values are referenced from Table A-10:  
- \( h_{c,1} = \text{value at } -6^\circ\text{C} \)  
- \( h_{g,1} = \text{value at } -6^\circ\text{C} \)  

Intermediate calculations:  
\[
h_{c,1} = 242.54 \, \text{kJ/kg}, \quad h_{g,1} = 247.20 \, \text{kJ/kg}
\]  

Final result:  
\[
x_1 = \frac{\text{calculated numerator}}{247.20 - 242.54}
\]  

TASK 4d  
No explicit calculation for the coefficient of performance \( \epsilon_K \) is visible, but the general formula is implied:  
\[
\epsilon_K = \frac{\dot{Q}_K}{\dot{W}_K}
\]  

No further details are provided.  

TASK 4e  
No content visible for this subtask.  

Descriptions of calculations and graphs are based on visible content.