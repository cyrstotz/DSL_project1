TASK 3a  
The pressure \( p_1 \) is calculated using the formula:  
\[
p_1 = p_{\text{amb}} + m_k \cdot \frac{1}{\pi \left( \frac{D}{2} \right)^2} g + m_{\text{EW}} \cdot \frac{1}{\pi \left( \frac{D}{2} \right)^2} g
\]  
Substituting values:  
\[
p_1 = p_{\text{amb}} + m_k \cdot \frac{1}{\pi \left( \frac{D}{2} \right)^2} g + m_{\text{EW}} \cdot \frac{1}{\pi \left( \frac{D}{2} \right)^2} g = 1.9005 \, \text{bar}
\]  

The ideal gas law is used to calculate the mass \( m_1 \):  
\[
p \cdot V = m \cdot R \cdot T
\]  
Where \( R = \frac{\bar{R}}{M} = 166.28 \, \text{J/(kg·K)} \).  
Substituting values:  
\[
m_1 = \frac{p_1 \cdot V_1}{R \cdot T_1} = 3.427 \cdot 10^{-3} \, \text{kg}
\]  

---

TASK 3b  
It is stated that \( m_1 = m_2 \) and \( p_1 = p_2 \) because the same "weight" is exerted by the piston.  

An isentropic polytropic process is assumed:  
\[
T_2 = T_1 \left( \frac{V_1}{V_2} \right)^{\gamma - 1}
\]  
Using the ideal gas law:  
\[
p_2 \cdot v_2 = R \cdot T_2 \quad \Rightarrow \quad v_2 = \frac{R \cdot T_2}{p_2}
\]  
Substituting into the temperature equation:  
\[
T_2 = T_1 \left( \frac{V_1}{\frac{R \cdot T_2}{p_2}} \right)^{\gamma - 1} = T_1 \cdot \frac{R \cdot T_2}{V_1 \cdot p_2}
\]  

This concludes the derivation for \( T_2 \).