TASK 2a  
The diagram is a qualitative representation of the jet engine process on a temperature-entropy (\( T \)-\( S \)) diagram. It includes labeled isobars and isentropic processes:  
- The curve starts at \( P_0 \), representing the ambient pressure.  
- The process moves isentropically to \( P_2 \), indicating compression.  
- From \( P_2 \), an isobaric process leads to \( P_3 \), representing combustion.  
- The process then moves isentropically to \( P_4 \), indicating expansion in the turbine.  
- Finally, an isobaric process leads to \( P_5 \), with \( P = 0.5 \, \text{bar} \), and reverses isentropically to \( P_6 \), returning to ambient pressure \( P_0 \).  

The axes are labeled as follows:  
- \( T \) [K] on the vertical axis (temperature).  
- \( S \) [kJ/kg·K] on the horizontal axis (entropy).  

TASK 2b  
The energy balance for the nozzle is written as:  
\[
0 = \dot{h}_5 - \dot{h}_6 + \frac{w_5^2 - w_6^2}{2}
\]  

The temperature \( T_6 \) is calculated using the isentropic relation:  
\[
T_6 = T_5 \left( \frac{P_6}{P_5} \right)^{\frac{k-1}{k}}
\]  
where \( k = 1.4 \). Substituting values:  
\[
T_6 = 451.9 \, \text{K} \left( \frac{0.191}{0.5} \right)^{\frac{1.4-1}{1.4}} = 828.075 \, \text{K}
\]  

For an ideal gas, the outlet velocity \( w_6 \) is determined using:  
\[
w_6 = \sqrt{2 \cdot \left[ \dot{m}_{\text{gas}} \cdot c_p \cdot (T_5 - T_6) + \frac{w_5^2}{2} \right]}
\]  
Substituting values:  
\[
w_6 = \sqrt{2 \cdot \left[ \dot{m}_{\text{gas}} \cdot c_p \cdot (451.9 - 828.075) + \frac{220^2}{2} \right]} = 220.47 \, \text{m/s}
\]  

Additional notes:  
- \( P_6 = P_0 = 0.191 \, \text{bar} \).  
- \( w_5 = 220 \, \text{m/s} \).  
- \( T_5 = 451.9 \, \text{K} \).  
- \( \dot{m}_{\text{gas}} \) and \( c_p \) are constants for the ideal gas.  
- Work interactions (\( W \)) are zero.