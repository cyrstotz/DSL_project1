TASK 3a  
The gas pressure \( p_{g,1} \) is calculated using the formula:  
\[
p_{g,1} = p_0 + \frac{m_{\text{total}} g}{A}
\]  
where \( A = \frac{\pi}{4} \left( \frac{D}{2} \right)^2 = 0.00785398 \, \text{m}^2 \).  

The total mass is given as:  
\[
m_{\text{total}} = m_K + m_{\text{EW}} = 32.1 \, \text{kg}
\]  

The pressure \( p_{g,1} \) is calculated to be:  
\[
p_{g,1} = 1.80094 \, \text{bar}
\]  

The gas mass \( m_g \) is determined using the ideal gas law:  
\[
m_g = \frac{p V}{R T}
\]  
where \( R = \frac{R_u}{M} = 166.28 \, \text{J/(kg·K)} \).  

The gas mass is calculated as:  
\[
m_g = 3.422 \, \text{g}
\]  

---

TASK 3b  
The pressure \( p_{g,2} \) is given as:  
\[
p_{g,2} = p_{g,1} = 1.80094 \, \text{bar}
\]  

The pressure does not change because the system is closed and the same external pressure is still applied.  

It is valid that:  
\[
T_{g,2} = T_{\text{EW}} = 0^\circ\text{C}
\]  

There is still ice present (\( T < 0^\circ\text{C} \)), and no heat exchange occurs anymore.