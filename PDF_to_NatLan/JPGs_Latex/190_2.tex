TASK 2a  
The process is represented qualitatively in a \( T \)-\( S \) diagram. The diagram includes labeled isobars and states:  
- State \( 0 \) corresponds to ambient conditions (\( p_0 \)).  
- States \( 2 \), \( 3 \), \( 5 \), and \( 6 \) are shown along the process path.  
- The diagram includes dashed lines for isentropic processes and solid lines for other transitions.  
- The axes are labeled: \( T \) (temperature) on the vertical axis and \( S \) (specific entropy, \( \text{kJ/kg·K} \)) on the horizontal axis.  

TASK 2b  
To determine \( w_6 \) and \( T_6 \):  
- The outlet velocity is given as \( w_6 = 220 \, \text{m/s} \).  
- The pressure at state \( 5 \) is \( p_5 = 0.5 \, \text{bar} \), and the temperature is \( T_5 = 431.9 \, \text{K} \).  
- The entropy at state \( 5 \) equals the entropy at state \( 6 \) (\( S_5 = S_6 \)) because the process is isentropic.  
- Air is modeled as an ideal gas with \( c_p = 1.006 \, \text{kJ/kg·K} \).  

Since the process is adiabatic:  
\[
\frac{T_6}{T_5} = \left( \frac{p_6}{p_5} \right)^{\frac{n-1}{n}}
\]  
Substituting values:  
\[
T_6 = 431.9 \, \text{K} \cdot \left( \frac{0.191 \, \text{bar}}{0.5 \, \text{bar}} \right)^{\frac{1.4-1}{1.4}} = 328.075 \, \text{K}
\]  

The change in internal energy is given by:  
\[
u_6 - u_5 = c_v \Delta T
\]  
Where \( c_v = c_p - R \).