TASK 2a  
The process is depicted qualitatively in a \( T \)-\( s \) diagram. The diagram shows six states labeled 1 through 6, connected by various processes. The isobars are labeled as \( p = 0.191 \, \text{bar} \), \( p = 0.5 \, \text{bar} \), and \( p = p_2 \). The ambient temperature is given as \( T_0 = 243.15 \, \text{K} \), calculated from \( T_0 = 273.15 - 30 \). The entropy axis is labeled \( s \, [\frac{\text{kJ}}{\text{K}}] \), and the temperature axis is labeled \( T \, [\text{K}] \).  

TASK 2b  
The thermodynamic mean temperature \( T_6 \) is calculated using the formula:  
\[
T_6 = T_5 \left( \frac{p_6}{p_5} \right)^{\frac{\kappa - 1}{\kappa}}
\]  
Substituting values:  
\[
T_6 = 431.9 \, \text{K} \left( \frac{0.191}{0.5} \right)^{\frac{0.4}{1.4}} = 327.39 \, \text{K}
\]  

An energy balance is illustrated with a schematic of the nozzle, showing inlet at state 5 and outlet at state 6. The velocity at state 6, \( w_6 \), is calculated using:  
\[
w_6 = \sqrt{w_5^2 + 2 \left( h_5 - h_6 \right)} = \sqrt{w_5^2 + 2 c_p \left( T_6 - T_5 \right)}
\]  
Substituting values:  
\[
w_6 = 508.6 \, \frac{\text{m}}{\text{s}}
\]  

TASK 2c  
The mass-specific increase in flow exergy \( \Delta ex_{\text{flow}} \) is calculated as:  
\[
\Delta ex_{\text{flow}} = ex_{\text{flow},6} - ex_{\text{flow},0} = h_6 - h_0 - T_0 \left( s_6 - s_0 \right) + \frac{1}{2} \left( w_6^2 - w_0^2 \right)
\]  
Expanding the terms:  
\[
\Delta ex_{\text{flow}} = c_p \left( T_6 - T_0 \right) - T_0 \left( c_p \ln \left( \frac{T_6}{T_0} \right) - R \ln \left( \frac{p_6}{p_0} \right) \right) + \frac{1}{2} \left( w_6^2 - w_0^2 \right)
\]  
Substituting values:  
\[
\Delta ex_{\text{flow}} = 120.32 \, \frac{\text{kJ}}{\text{kg}}
\]  

TASK 2d  
An exergy balance is performed using the equation:  
\[
0 = -\Delta ex_{\text{str}} + \left( 1 - \frac{T_0}{T_B} \right) q_B - ex_{\text{vol}}
\]  
Rearranging:  
\[
ex_{\text{vol}} = \left( 1 - \frac{T_0}{T_B} \right) q_B - \Delta ex_{\text{str}}
\]  
Substituting values:  
\[
ex_{\text{vol}} = 84.826 \, \frac{\text{kJ}}{\text{kg}}
\]