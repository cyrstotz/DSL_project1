TASK 2a  
The process is represented in a \( T \)-\( s \) diagram. The diagram shows the thermodynamic process of the jet engine, with labeled states 1 through 5. The following features are visible:  
- State 1 to 2: Isentropic compression.  
- State 2 to 3: Isobaric heat addition.  
- State 3 to 4: Adiabatic expansion.  
- State 4 to 5: Isobaric mixing.  
The axes are labeled as \( T \) (temperature) on the vertical axis and \( s \) (entropy) on the horizontal axis, with units \( \text{kJ/kg·K} \). The curves represent isobars \( p_2 = p_3 \) and \( p_4 = p_5 \).  

TASK 2b  
The outlet velocity \( w_6 \) and temperature \( T_6 \) are calculated as follows:  

The temperature ratio is given by:  
\[
\frac{T_6}{T_5} = \left( \frac{p_6}{p_5} \right)^{\frac{\kappa - 1}{\kappa}}
\]  
Substituting values:  
\[
T_6 = T_5 \left( \frac{p_6}{p_5} \right)^{\frac{\kappa - 1}{\kappa}}
\]  
\[
T_6 = 431.9 \, \text{K} \left( \frac{0.191}{0.5} \right)^{\frac{0.4}{1.4}}
\]  
\[
T_6 = 328.07 \, \text{K}
\]  

The ambient temperature \( T_0 \) is also noted as \( T_0 = 328.07 \, \text{K} \).  

TASK 2c  
The mass-specific increase in flow exergy is calculated using the energy balance equation:  
\[
\frac{dE}{dt} = \dot{m} \left( h_6 - h_0 \right) + \dot{Q} - \dot{W}
\]  
For steady-state operation:  
\[
0 = \dot{m} \left( h_5 - h_6 \right) + \frac{w_6^2}{2} - \frac{w_5^2}{2}
\]  
Rewriting:  
\[
0 = \dot{m} \left( c_p (T_5 - T_6) \right) + \frac{w_6^2}{2} - \frac{w_5^2}{2}
\]  

Additional relationships are noted:  
\[
\frac{\dot{m}_M}{\dot{m}_K} = 5.293
\]  
\[
q_B = \frac{\dot{Q}_B}{\dot{m}_K} = 1195 \, \text{kJ/kg}
\]