TASK 2a  
The problem setup begins with the following parameters:  
- Ambient pressure \( p_0 = 0.191 \, \text{bar} \)  
- Ambient temperature \( T_0 = -30^\circ\text{C} \)  
- Airspeed \( w_{\text{air}} = 200 \, \text{m/s} \)  
- The ratio of bypass to core mass flow rates is \( \frac{\dot{m}_M}{\dot{m}_K} = 5.293 \).  
- Specific heat capacity of air \( c_{p,\text{air}} = 1.006 \, \text{kJ/kg·K} \).  
- Adiabatic index \( \kappa = 1.4 \).  

The jet engine operates through several thermodynamic states:  
1. \( p_1 = p_3 \)  
2. \( p_2 = p_3 \)  
3. \( p_3 = p_4 \)  
4. \( T_5 = 437.9 \, \text{K} \), \( p_5 = 0.5 \, \text{bar} \), \( w_5 = 220 \, \text{m/s} \).  
5. \( p_6 = p_0 = 0.191 \, \text{bar} \).  

---

TASK 2b  
The task involves determining the outlet velocity \( w_6 \) and temperature \( T_6 \).  

The energy balance equation is applied:  
\[
O = \dot{m}_{\text{gas}} \left( h_5 - h_6 + \frac{1}{2} w_5^2 - \frac{1}{2} w_6^2 \right)
\]

The enthalpy \( h_5 \) is calculated using the given temperature \( T_5 = 437.9 \, \text{K} \) and pressure \( p_5 = 0.5 \, \text{bar} \):  
\[
h_5 = h(T_5, p_5) = A - B + C - D
\]
Where:  
- \( A = 490 \), \( B = 490 \), \( C = 431.9 \times 10^3 \), \( D = 441.6 \times 10^3 \).  

The enthalpy \( h_6 \) is calculated using \( T_6 = 328.07 \, \text{K} \) and \( p_6 = 0.191 \, \text{bar} \):  
\[
h_6 = h(T_6, p_6) = A - B + C - D
\]
Where:  
- \( A = 325 \), \( B = 325 \), \( C = 330.3 \times 10^3 \), \( D = 330.3 \times 10^3 \).  

The temperature ratio is used to calculate \( T_6 \):  
\[
\frac{T_6}{T_5} = \left( \frac{p_6}{p_5} \right)^{\frac{\kappa - 1}{\kappa}}
\]
Substituting values:  
\[
\frac{T_6}{T_5} = \left( \frac{0.191}{0.5} \right)^{\frac{0.4}{1.4}} \implies T_6 = 328.07 \, \text{K}
\]

The outlet velocity \( w_6 \) is determined using:  
\[
w_6^2 = \frac{2}{m} \left( h_5 - h_6 \right) + w_5^2
\]
Substituting values:  
\[
w_6 = \sqrt{\frac{2}{m} \left( h_5 - h_6 \right) + w_5^2} = 508.26 \, \text{m/s}
\]  

---

TASK 2a (Graph Description)  
A T-s diagram is drawn to represent the thermodynamic process. The diagram includes:  
- Isobars labeled for states \( p_0 \), \( p_3 \), \( p_5 \), and \( p_6 \).  
- The temperature axis starts at \( T_0 = -30^\circ\text{C} \) and extends to \( T_5 = 437.9 \, \text{K} \).  
- The entropy axis shows qualitative changes during compression, combustion, and expansion.  
- The process path includes labeled points for states 0, 1, 2, 3, 4, 5, and 6.  
