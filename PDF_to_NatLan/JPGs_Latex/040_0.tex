TASK 1a  
The first law of thermodynamics is applied to the reactor system:  
\[
Q = \dot{m}_{\text{in}} (h_{\text{ein}} - h_{\text{aus}}) + \dot{Q}_R - \dot{Q}_{\text{aus}}
\]  
The enthalpy values are taken from the water tables:  
\[
h_{\text{ein}} \, (\text{Table A2}) = 297.28 \, \frac{\text{kJ}}{\text{kg}}
\]  
\[
h_{\text{aus}} \, (\text{Table A3}) = 419.07 \, \frac{\text{kJ}}{\text{kg}}
\]  
The heat flow removed by the coolant is calculated as:  
\[
\dot{Q}_{\text{aus}} = \dot{m}_{\text{in}} (h_{\text{ein}} - h_{\text{aus}}) + \dot{Q}_R = 62.2 \, \text{kW}
\]  

TASK 1b  
The thermodynamic mean temperature of the coolant can be determined using the arithmetic mean:  
\[
T_{\text{KF}} = \frac{T_{\text{KF,in}} + T_{\text{KF,out}}}{2} = 293.15 \, \text{K}
\]  
This value is highlighted as the final result.