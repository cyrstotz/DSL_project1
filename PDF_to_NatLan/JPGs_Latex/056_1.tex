TASK 4a  
The page contains two graphs related to the freeze-drying process described in Task 4.  

**Graph 1 (Top):**  
This graph is a pressure-temperature (\( p \)-\( T \)) diagram.  
- The x-axis represents temperature (\( T \)) in degrees Celsius (\( ^\circ \text{C} \)), ranging from negative values to above zero.  
- The y-axis represents pressure (\( p \)) in mbar, ranging from 0.01 to 1 and higher.  
- The graph includes phase regions labeled as "fest" (solid), "flüssig" (liquid), and "gas" (gas).  
- The triple point is marked, indicating the intersection of the solid, liquid, and gas phases.  
- A region labeled "flüssig" transitions into "gas," with a dashed line labeled "Nassdampf gas" (wet steam gas).  
- The diagram shows the boundaries between the phases, with curved lines separating solid, liquid, and gas regions.  

**Graph 2 (Bottom):**  
This graph is also a pressure-temperature (\( p \)-\( T \)) diagram but zoomed in on specific details.  
- The x-axis represents temperature (\( T \)) in degrees Celsius (\( ^\circ \text{C} \)), ranging from \(-30^\circ \text{C}\) to \( 0^\circ \text{C} \).  
- The y-axis represents pressure (\( p \)) in mbar, ranging from 0.01 to 10.  
- The phase regions are labeled "fest" (solid), "flüssig" (liquid), and "gas" (gas).  
- The triple point is marked again, showing the transition between phases.  
- A note indicates the relationship between Step ii and \( T_i \), suggesting that the temperature \( T_i \) is linked to the sublimation process.  

Both graphs visually describe the phase transitions and conditions relevant to the freeze-drying process.