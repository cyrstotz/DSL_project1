TASK 4a  
A graph is drawn showing the freeze-drying process in a \( p \)-\( T \) diagram. The diagram includes labeled states: 1, 2, 3, 4, and 5. The curve represents phase regions, with state 1 starting in the liquid-vapor region, state 2 in the vapor region, state 3 in the superheated vapor region, state 4 in the liquid region, and state 5 returning to the vapor region. The axes are labeled \( p \) (pressure) on the vertical axis and \( T \) (temperature) on the horizontal axis.  

TASK 4b  
The process from state 2 to state 3 is described as isentropic, meaning \( s_2 = s_3 \).  

A table is provided with the following columns:  
- \( \text{State} \): 1, 2, 3, 4  
- \( p \, (\text{bar}) \): 1, 1, 8, 8  
- \( T \): unspecified values  
- \( x \): 1 (state 2), 0 (state 4)  
- \( s \): \( s_2 = s_3 \)  
- \( h \): \( h_4 \)  

It is noted that \( x_3 \) is calculated.  

The enthalpy at state 1 is determined using the equation:  
\[
h_1 = h_f(36 \, \text{kPa}) = 93.42 \, \text{kJ/kg}
\]