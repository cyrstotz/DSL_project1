TASK 2a  
The diagram is a qualitative representation of the jet engine process on a \( T \)-\( s \) diagram. It includes labeled isobars and key states (0, 1, 2, 3, 4, 5, 6). The axes are labeled as follows:  
- \( T \) (temperature) on the vertical axis, in Kelvin (\( \text{K} \)).  
- \( s \) (specific entropy) on the horizontal axis, in \( \text{kJ}/\text{kg·K} \).  

The process transitions are shown with arrows connecting the states:  
- State 0 to 1 represents compression.  
- State 1 to 2 is the bypass flow.  
- State 2 to 3 represents combustion.  
- State 3 to 4 is turbine expansion.  
- State 4 to 5 is mixing.  
- State 5 to 6 represents nozzle expansion.  

Key annotations:  
- The process is described as "qualitative" and "from the solution sheet in the graph phase."  
- The student notes to "transfer to another sheet" for further elaboration.  

---

TASK 2b  
To determine the outlet velocity \( w_6 \) and temperature \( T_6 \):  

Given data:  
- Ambient pressure \( p_0 = 0.191 \, \text{bar} \).  

The student mentions "balance in the boundary" and writes a crossed-out equation involving mass flow rates.  

Assumptions:  
- \( \dot{m}_{\text{in}} = \dot{m}_{\text{out}} \) (mass conservation).  
- \( p \cdot V^n = \text{constant} \) (likely referring to a polytropic process).  

No further calculations or results are visible.  

---  
Description of crossed-out content:  
The student attempted to write an equation involving \( \dot{m} \) and \( h_{\text{in}} \), but it was crossed out and not completed.  

