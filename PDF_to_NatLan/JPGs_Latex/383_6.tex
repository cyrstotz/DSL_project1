TASK 4a  
The table lists the states of the freeze-drying process with the following parameters:  
- State 1: \( T \) and \( p_1 = p_2 \), vapor quality \( x = 1 \).  
- State 2: \( T \) and \( p_2 = p_1 \), vapor quality \( x = 1 \).  
- State 3: \( T \) and \( p_3 = 8 \, \text{bar} \), vapor quality unspecified.  
- State 4: \( T \) and \( p_4 = p_3 \), vapor quality \( x = 0 \).  

Below the table, a \( p \)-\( T \) diagram is drawn.  
- The diagram shows a dome-shaped curve representing the phase boundary.  
- States 1 and 2 are located on the left side of the dome, indicating saturated vapor conditions.  
- State 3 is at the top of the dome, representing high pressure.  
- State 4 is on the right side of the dome, indicating saturated liquid conditions.  
Arrows indicate transitions between states:  
1 → 2 (horizontal), 2 → 3 (upward), 3 → 4 (horizontal).

---

TASK 4b  
The mass flow rate of the refrigerant \( \dot{m}_{\text{R134a}} \) is calculated using an energy balance around the compressor under steady-state conditions:  
\[
0 = \dot{m}_{\text{R134a}} \left( h_2 - h_3 \right) + \dot{Q}_{\text{radial}} + \dot{W}_K
\]  
Here, \( \dot{Q}_{\text{radial}} \) is neglected. Rearranging gives:  
\[
\dot{m}_{\text{R134a}} = \frac{\dot{W}_K}{h_3 - h_2}
\]  

The enthalpy values are provided:  
\[
h_2 = h_f(-22^\circ\text{C}) = 271.77 \, \text{kJ/kg}
\]  
\( h_3 \) is left unspecified.