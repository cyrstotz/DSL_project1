TASK 4a  
The triple point temperature is \( T_{\text{triple}} = 0^\circ\text{C} \).  
The sublimation temperature at 5 mbar is \( T_{\text{sub}} = -20^\circ\text{C} \).  
The initial temperature \( T_i \) is set 10 K above \( T_{\text{sub}} \), resulting in \( T_i = -10^\circ\text{C} \).  

A graph is drawn showing pressure (\( p \)) on the y-axis and temperature (\( T \)) on the x-axis. The graph includes the following labeled regions:  
- "Triple point" at the intersection of the solid, liquid, and gas phase boundaries.  
- "Gas" region above the sublimation curve.  
- "Sublimation" curve separating the solid and gas phases.  
- "Solid" region below the sublimation curve.  
- "Frost fusion" curve separating the solid and liquid phases.  
The graph also includes arrows indicating the direction of sublimation and isothermal processes.

---

TASK 4b  
The stationary flow process is described using the energy balance:  
\[
0 = \dot{m} (h_2 - h_3) + \dot{Q} - \dot{W}_K
\]  
The work done by the compressor is:  
\[
\dot{W}_K = \dot{m} (h_2 - h_3)
\]  
Rearranging to solve for the mass flow rate:  
\[
\dot{m}_{\text{R134a}} = \frac{\dot{W}_K}{h_2 - h_3}
\]  
Substituting values:  
\[
\dot{W}_K = 28 \times 10^{-3} \, \text{kW}, \quad h_2 = 237.796 \, \text{kJ/kg}, \quad h_3 = 227.314 \, \text{kJ/kg}
\]  
\[
\dot{m}_{\text{R134a}} = \frac{28 \times 10^{-3}}{237.796 - 227.314}
\]  
\[
\dot{m}_{\text{R134a}} = 8.39 \times 10^{-4} \, \text{kg/s}
\]  
\[
\dot{m}_{\text{R134a}} = 0.834 \, \text{g/s}
\]  

Additional notes:  
- The temperature after isobaric evaporation is \( T_v = T_i - 6 = -16^\circ\text{C} \).  
- From Table A-11, \( h_2 \) is determined for \( x_2 = 1 \) and \( T_v = -16^\circ\text{C} \):  
\[
h_2 = 237.796 \, \text{kJ/kg}
\]