TASK 3d  
The internal energy \( u_2 \) is calculated as:  
\[
u_2 = -733.4702 \, \frac{\text{kJ}}{\text{kg}}
\]  

The total energy \( U_2 \) is expressed as:  
\[
U_2 = m_{\text{EW}} u_2 + Q_{12} = m_{\text{EW}} u_2
\]  

Substituting values:  
\[
u_2 = u_1 + \frac{Q_{12}}{m_{\text{EW}}} = -733.4702 \, \frac{\text{kJ}}{\text{kg}} + \frac{77.35 \, \text{kJ}}{0.7 \, \text{kg}}
\]  
\[
u_2 = -733.4702 \, \frac{\text{kJ}}{\text{kg}} + 110.5 \, \frac{\text{kJ}}{\text{kg}} = -79.97 \, \frac{\text{kJ}}{\text{kg}}
\]  

The ice fraction \( x_2 \) is calculated using the equation:  
\[
x_2 (u_{FI} - u_{FE}) = u_2 - u_{FE}
\]  

Rewriting:  
\[
x_2 = \frac{u_2 - u_{FE}}{u_{FI} - u_{FE}}
\]  

Substituting values:  
\[
x_2 = \frac{-79.97 \, \frac{\text{kJ}}{\text{kg}} + 333.958 \, \frac{\text{kJ}}{\text{kg}}}{-0.085 \, \frac{\text{kJ}}{\text{kg}} + 333.958 \, \frac{\text{kJ}}{\text{kg}}}
\]  
\[
x_2 = 0.990
\]  

The final ice fraction \( x_2 \) is approximately \( 0.990 \).