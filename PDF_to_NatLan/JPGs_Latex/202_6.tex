TASK 4a  
The diagram is a pressure-temperature (\(p\)-\(T\)) graph illustrating the freeze-drying process. It includes two steps:  
- Step (i): Compression, labeled as "Komprimiert," showing an increase in pressure.  
- Step (ii): Isobaric evaporation, labeled as "isobare Verdampfung," showing a horizontal line at constant pressure.  
The axes are labeled:  
- \(p\) (pressure) on the vertical axis.  
- \(T\) (temperature) on the horizontal axis.  

TASK 4b  
The energy balance equation is written as:  
\[
Q = \dot{m} \cdot \left(h_e - h_a\right) + \dot{Q}_K
\]  
The heat transfer rate is expressed as:  
\[
\dot{Q}_K = \dot{m} \cdot \left(h_a - h_e\right)
\]  
Specific enthalpy relationships are given:  
\[
h_a = h_2, \quad h_e = h_1, \quad h_g = h_4
\]  
The vapor quality (\(x_0\)) is defined at \(T = -6^\circ\text{C}\), and the pressure is calculated as:  
\[
p_1 = 3.376 \, \text{bar}
\]  
It is noted that \(p_2 = p_1\).  

TASK 4c  
The vapor quality (\(x\)) is calculated using:  
\[
x = \frac{h_f}{h_f + x \cdot \left(h_0 - h_f\right)}
\]  
It is noted that \(p_4 = p_3\).  

TASK 4d  
The coefficient of performance (\(\epsilon_K\)) is expressed as:  
\[
\epsilon_K = \frac{\dot{Q}_K}{\dot{W}_K} = \frac{\dot{Q}_K}{\dot{Q}_{ab} - \dot{Q}_K}
\]  

TASK 4e  
The explanation states:  
"Temperatures remain constant as heat continues to be removed. If \(T_i = -70^\circ\text{C}\), the process is no longer isobaric."  
Some text is crossed out and ignored.  

No additional diagrams or content are present.