TASK 4a  
The diagram is a pressure-temperature (\( p \)-\( T \)) phase diagram. It shows the phase regions for a substance: solid ("Fest"), liquid ("flüssig"), and gas ("Gas").  

- The curve labeled "Tripel" represents the triple point where solid, liquid, and gas phases coexist in equilibrium.  
- A horizontal line is drawn in the liquid region, indicating isobaric processes.  
- Two points are marked along this line, connected by a vertical line, which suggests a transition between phases (possibly evaporation or condensation).  
- The axes are labeled: the vertical axis is pressure (\( p \)) and the horizontal axis is temperature (\( T \)).  

No further numerical or textual information is provided.