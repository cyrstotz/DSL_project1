TASK 4a  
Two diagrams are drawn to represent the freeze-drying process.  

1. **First Diagram**:  
   - The graph is labeled as \( p(T)_{\text{sur}} \) on the y-axis and \( T \, [K] \) on the x-axis.  
   - It shows phase regions for the refrigerant, including "Gas" and "Flüssig" (liquid).  
   - The curves intersect, indicating phase transitions.  

2. **Second Diagram**:  
   - The graph is labeled with \( p \) on the y-axis and \( T \) on the x-axis.  
   - It depicts a typical phase diagram with a dome-shaped curve representing the saturation region.  
   - Points labeled "1" and "2" are marked on the curve, with arrows indicating transitions between states.  
   - The labels "Tripelpunkt" (triple point) and "Gas" are included.  

TASK 4b  
The mass flow rate of the refrigerant is calculated using an energy balance for the compressor and evaporator.  

- Energy balance equation:  
  \[
  \sum \dot{m}_i \left( h_i + \frac{w_i^2}{2} + g z_i \right) + \dot{Q}_{\text{in}} - \dot{W}_{\text{out}} = \sum \dot{m}_e \left( h_e + \frac{w_e^2}{2} + g z_e \right)
  \]  

- Refrigerant mass flow rate:  
  \[
  \dot{m}_{\text{R134a}} = \frac{\dot{Q}_K}{h_2 - h_3}
  \]  

- Key assumptions and values:  
  - \( h_2 = h(p_2, T_i) \)  
  - \( h_4 = h(p_4) \)  
  - \( h_4 \) at 8 bar is given as \( 195.92 \, \text{kJ/kg} \).  
  - The process is reversible, so \( s_2 = s_3 \).  
  - \( p_2 \) and \( p_1 \) are constant.  

- Final equation:  
  \[
  \dot{Q}_K = \dot{m}_{\text{R134a}} \cdot (h_4 - h_1)
  \]  

Additional notes:  
- Energy balance is applied to the compressor and evaporator.  
- \( h_1 \) and \( h_4 \) are determined from refrigerant tables.