TASK 4a  
A graph is drawn showing the freeze-drying process in a pressure-temperature (\( p \)-\( T \)) diagram. The axes are labeled as \( p \, [\text{mbar}] \) for pressure and \( T \, [\text{K}] \) for temperature. The graph includes four distinct states labeled as 1, 2, 3, and 4, connected by arrows indicating the direction of the process.  
- State 1 starts at low pressure and low temperature.  
- State 2 shows an increase in temperature while maintaining low pressure.  
- State 3 indicates a significant increase in pressure and temperature.  
- State 4 returns to lower pressure while maintaining high temperature.  

TASK 4b  
The goal is to determine the mass flow rate of the refrigerant (\( \dot{m}_{\text{R134a}} \)).  

From the energy balance equation:  
\[
0 = \dot{m}_{\text{R134a}} \cdot (h_2 - h_3) + \dot{Q} - \dot{W}_k
\]  
Here:  
- \( \dot{m}_{\text{R134a}} \) is the refrigerant mass flow rate.  
- \( h_2 \) and \( h_3 \) are enthalpies at states 2 and 3, respectively.  
- \( \dot{Q} \) represents the heat transfer rate.  
- \( \dot{W}_k \) is the work done by the compressor.  

Note: The term \( \dot{Q} \) is crossed out in the equation.