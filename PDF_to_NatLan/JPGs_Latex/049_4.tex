TASK 3a  
In state 1, the pressure balance is given by:  
\[
p_{\text{amb}} + \frac{F_S}{A} = p_i \implies p_{\text{amb}} + \frac{(m_K + m_{\text{EW}}) g}{\frac{D^2}{4} \pi} = 140089.4 \, \text{Pa}
\]  

Using the ideal gas law:  
\[
p V = n R T
\]  
where \( T = 500 + 273.15 = 773.15 \, \text{K} \).  

The number of moles \( n \) is calculated as:  
\[
n = \frac{p V}{R T} \approx 0.068
\]  

Given \( M_g = 50 \, \text{kg/kmol} \), the mass of the gas is:  
\[
m_g = M_g \frac{n}{1000} \implies m_g \approx 3.4 \, \text{g}
\]  

---

TASK 3b  
Since no more heat is transferred, the gas and EW reach thermal equilibrium:  
\[
T_{g,2} = T_{\text{EW},2}
\]  

For \( x > 0 \), the equilibrium condition holds:  
\[
T_{\text{EW},1} = 0^\circ\text{C} = T_{\text{EW},2}
\]  

The pressure balance remains constant:  
\[
p_2 = p = 140089.4 \, \text{Pa}
\]  

---

TASK 3c  
This is an isobaric process, meaning \( p = \text{const} \).  

From the first law of thermodynamics:  
\[
\Delta U = Q - W \implies Q = \Delta U + W = m_g c_V \Delta T + p \Delta V
\]  

The heat transferred is:  
\[
Q = 1772.275 \, \text{J}
\]  

The change in volume is negligible:  
\[
\Delta V = \left| V_2 - V_1 \right| = \frac{R T}{p} \approx 0.002
\]