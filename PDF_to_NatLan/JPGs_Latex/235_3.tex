TASK 3a  
The gas constant \( R \) is calculated as:  
\[
R = \frac{R}{M_g} = 166.28 \, \frac{\text{J}}{\text{kg·K}}
\]  

The pressure \( p_g \) is determined using the following equation:  
\[
p_g = \frac{m_{\text{EW}} g}{\left(\frac{d}{2}\right)^2} + \frac{m_K g}{\left(\frac{d}{2}\right)^2} + p_{\text{amb}} = 2.259 \, \text{bar}
\]  
where \( d = 0.1 \, \text{m} \).  

Using the ideal gas law:  
\[
p V = m R T
\]  
The temperature \( T_g \) is given as \( T_g = 773.15 \, \text{K} \), and the volume \( V_g \) is \( V_g = 3.14 \cdot 10^{-3} \, \text{m}^3 \).  

The mass of the gas \( m_g \) is calculated as:  
\[
m_g = \frac{p_g V_g}{R T_g} = 5.52 \, \text{g}
\]  

---

TASK 3b  
The pressure \( p_{g,2} \) corresponds to the pressure \( p_{g,1} \), as the mass exerting the pressure does not change.  

---

TASK 3c  
Using the given values, further calculations are performed:  
\[
m_g = 3.6 \, \text{g}, \quad T_{2,g} = 0.003^\circ\text{C} = 273.153 \, \text{K}
\]  

The change in internal energy \( \Delta U_{12} \) is expressed as:  
\[
\Delta U_{12} = \sum_j Q - \sum_n W_n
\]  

The heat transferred \( |Q_{12}| \) is calculated as:  
\[
|Q_{12}| = m_g (u_2 - u_1) = m_g c_V (T_2 - T_1) = 1139.4 \, \text{J}
\]  