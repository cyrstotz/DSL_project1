TASK 3a  
The gas pressure \( p_{g,1} \) is determined using the force equilibrium at the membrane:  
\[
F_g = F_{\text{EW}} + F_{\text{p,amb}}
\]  

The area of the membrane is calculated as:  
\[
A = \left(\frac{D}{2}\right)^2 \pi = \left(\frac{0.04 \, \text{m}}{2}\right)^2 \pi = 6.7559 \times 10^{-4} \, \text{m}^2
\]  

The gas pressure is given by:  
\[
p_{g,1} \cdot A = m_K \cdot g + c_{\text{EW}} \cdot g + 10^5 \, \text{Pa} \cdot A
\]  

Substituting values:  
\[
p_{g,1} = 1.05 \, \text{bar}
\]  

The mass of the gas \( m_g \) is calculated using the ideal gas law:  
\[
m_g = \frac{p_{g,1} \cdot V_{g,1}}{R \cdot T_{g,1}}
\]  

Substituting values:  
\[
m_g = \frac{94.05 \cdot 205.3 \cdot 10^{-3}}{0.166 \cdot 10^3 \cdot (500 + 273.15)} = 0.1 \, \text{kg}
\]  

The specific gas constant \( R \) is calculated as:  
\[
R = \frac{\bar{R}}{M} = \frac{8.314}{50} = 0.166 \, \frac{\text{kJ}}{\text{kg·K}}
\]  

---

TASK 3b  
The final temperature of the gas is given as:  
\[
T_{g,2} = 0.003^\circ\text{C}
\]  

It is noted that the equilibrium state has a larger volume.  

---

TASK 3c  
Using the first law of thermodynamics:  
\[
\Delta E_{\text{gas}} + \Delta E_{\text{EW}} = 0
\]  

This expands to:  
\[
m_g \cdot (u_2 - u_1) = m_{\text{EW}} \cdot (u_1 - u_2)
\]  

Substituting for internal energy changes:  
\[
m_g \cdot c_V \cdot (T_2 - T_1) = m_{\text{EW}} \cdot (u_1 - u_2)
\]  

The heat transferred \( Q \) is calculated as:  
\[
Q = 0.1 \, \text{kg} \cdot 0.633 \, \frac{\text{kJ}}{\text{kg·K}} \cdot (0.003 - 500) = -31.651 \, \text{kJ}
\]  

This value is underlined in the solution.  

---  
No diagrams or figures are present on this page.