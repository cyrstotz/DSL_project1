TASK 3a  
The ideal gas law is used to calculate the pressure and mass of the gas in state 1:  
\[
p_{g,1} V_{g,1} = m_g R T_{g,1}
\]  
where \( R = \frac{\bar{R}}{M} \).  

The force equilibrium on the membrane is described as:  
\[
p_{g,1} \cdot \left( \frac{1}{2} D^2 \pi \right) = g \cdot (m_K + m_{\text{EW}})
\]  
Rearranging and solving for \( p_{g,1} \):  
\[
p_{g,1} = \frac{4 \cdot g \cdot (m_K + m_{\text{EW}})}{D^2 \pi}
\]  
Substituting values:  
\[
p_{g,1} = 4 \cdot \frac{9.81 \, \text{m/s}^2 \cdot (32 \, \text{kg} + 0.1 \, \text{kg})}{(0.1 \, \text{m})^2 \cdot \pi}
\]  
\[
p_{g,1} = 4.01 \, \text{bar}
\]  

Next, the mass of the gas \( m_g \) is calculated:  
\[
m_g = \frac{p_{g,1} \cdot V_{g,1}}{R \cdot T_{g,1}}
\]  
Substituting values:  
\[
m_g = \frac{40,080 \, \text{Pa} \cdot 0.00314 \, \text{m}^3}{\frac{8.314 \, \text{J/mol·K}}{50 \, \text{kg/kmol}} \cdot (500 + 273.15) \, \text{K}}
\]  
\[
m_g = 0.978 \, \text{kg}
\]  

No diagrams or graphs are present on this page.