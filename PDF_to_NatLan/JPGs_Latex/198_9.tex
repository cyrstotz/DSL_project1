TASK 4a  
The diagram is a pressure-temperature (\( p \)-\( T \)) plot illustrating the freeze-drying process. It consists of four distinct states labeled \( 1 \), \( 2 \), \( 3 \), and \( 4 \).  
- Between states \( 1 \) and \( 2 \), the process is isobaric evaporation.  
- Between states \( 2 \) and \( 3 \), the process is a reversible adiabatic compression.  
- Between states \( 3 \) and \( 4 \), the process is isobaric condensation.  
- Between states \( 4 \) and \( 1 \), the process is an isenthalpic expansion.  
The graph includes annotations such as "geregelte vollständige Verdampfung" (controlled complete evaporation) and "geregelte vollständige Kondensation" (controlled complete condensation).  

TASK 4b  
The energy balance for the compressor is written as:  
\[
0 = \dot{m}_{\text{R134a}} (h_2 - h_3) + \dot{Q} - \dot{W}_K  
\]  
Since the process is adiabatic, \( \dot{Q} = 0 \), simplifying the equation to:  
\[
\dot{W}_K = \dot{m}_{\text{R134a}} (h_2 - h_3)  
\]  

The enthalpy \( h_2 \) is determined from the refrigerant properties at \( T_2 \):  
\[
h_2 = h_{\text{g}}(T_2)  
\]  

The temperature \( T_2 \) is given as:  
\[
T_2 = T_i - 6 \, \text{K}  
\]  

Using the conditions from Step ii, \( T_i \) can be determined from the \( p \)-\( T \) diagram. The value of \( T_i \) is calculated as:  
\[
T_i = 10^\circ\text{C} = 283.15 \, \text{K}  
\]