TASK 4a  
Two diagrams are drawn, both labeled with axes \( p \) (pressure) and \( T \) (temperature).  

1. **First Diagram**:  
   - The graph depicts a dome-shaped curve representing phase regions.  
   - Points labeled "1", "2", "3", and "4" are marked on the curve.  
   - Two horizontal lines are drawn across the dome, labeled "Isobar" (constant pressure).  
   - The curve represents the phase change process, with transitions between liquid, vapor, and mixed-phase regions.  

2. **Second Diagram**:  
   - Similar dome-shaped curve with points labeled "1", "2", "3", and "4".  
   - The regions are labeled as follows:  
     - "Flüssigkeit" (liquid) on the left side of the dome.  
     - "gesättigter Dampf" (saturated vapor) on the right side of the dome.  
     - "Mischgebiet" (mixed region) under the dome.  
   - Horizontal lines connect points "1" to "2" and "3" to "4", representing isobaric processes.  

TASK 4b  
The following equations are written:  
\[
\dot{m}(h_2 - h_3) = \dot{W}_\text{K}
\]  
\[
\dot{m} = \frac{\dot{W}_\text{K}}{h_2 - h_3}
\]  
\[
p_3 = p_4 = 8 \, \text{bar}
\]  

Explanation:  
- The first equation represents the energy balance for the compressor, where \( \dot{m} \) is the mass flow rate, \( h_2 \) and \( h_3 \) are specific enthalpies, and \( \dot{W}_\text{K} \) is the work input to the compressor.  
- The second equation rearranges the energy balance to solve for the mass flow rate \( \dot{m} \).  
- The third equation states that the pressure at points 3 and 4 is constant at 8 bar.