TASK 3a  
The initial temperature of the gas is \( T_{g,1} = 500^\circ\text{C} \).  
The initial volume of the gas is \( V_{g,1} = 3.14 \, \text{L} \), which is converted to \( V_{g,1} = 3.14 \times 10^{-3} \, \text{m}^3 \).  

The pressure of the gas \( p_{g,1} \) is calculated using the following equation:  
\[
p_{g,1} = p_{\text{amb}} + \frac{m_K \cdot g}{A} + \frac{m_{\text{EW}} \cdot g}{A}
\]  
where:  
- \( p_{\text{amb}} = 10^5 \, \text{N/m}^2 \),  
- \( m_K \cdot g = 313.92 \, \text{N} \),  
- \( m_{\text{EW}} \cdot g = 0.981 \, \text{N} \),  
- \( A = 7.854 \times 10^{-3} \, \text{m}^2 \).  

Substituting the values:  
\[
p_{g,1} = \frac{10^5 \, \text{N/m}^2}{A} + \frac{313.92 \, \text{N}}{7.854 \times 10^{-3} \, \text{m}^2} + \frac{0.981 \, \text{N}}{7.854 \times 10^{-3} \, \text{m}^2}
\]  
\[
p_{g,1} = 1.4 \, \text{bar}
\]  

The gas pressure is determined to be \( p_{g,1} = 1.4 \, \text{bar} \).  

The mass of the gas \( m_g \) is calculated using the ideal gas law:  
\[
p_{g,1} V_{g,1} = m_g R T_{g,1}
\]  
where \( R = \frac{\bar{R}}{M_g} = 0.16628 \, \text{kJ/kg·K} \).  

Rearranging for \( m_g \):  
\[
m_g = \frac{p_{g,1} V_{g,1}}{R T_{g,1}}
\]  
Substituting the values:  
\[
m_g = \frac{1.4 \, \text{bar} \cdot 3.14 \times 10^{-3} \, \text{m}^3}{0.16628 \, \text{kJ/kg·K} \cdot 500^\circ\text{C}}
\]  
\[
m_g = 3.42 \, \text{g}
\]  

The mass of the gas is \( m_g = 3.42 \, \text{g} \).  

**Figure Description:**  
The diagram shows a cylinder divided into two chambers. The top chamber contains an ice-water mixture (EW), while the bottom chamber contains a gas. A piston rests on the ice-water mixture, exerting pressure. The forces acting on the piston include the atmospheric pressure (\( p_{\text{amb}} \)), the weight of the piston (\( m_K \cdot g \)), and the weight of the ice-water mixture (\( m_{\text{EW}} \cdot g \)). The cross-sectional area of the cylinder is labeled as \( A \).