TASK 2a  
The diagram shows a qualitative \( T \)-\( s \) diagram for the jet engine process. The temperature \( T \) is plotted on the vertical axis, and the entropy \( s \) is plotted on the horizontal axis. The diagram includes labeled isobars for pressures \( p_2 = p_3 \), \( p_1 = p_4 = p_5 \), and \( p_6 = p_0 \). The states are connected by curves representing the thermodynamic processes:  
- State 1 to 2: Compression  
- State 2 to 3: Isobaric combustion  
- State 3 to 4: Expansion in the turbine  
- State 4 to 5: Mixing  
- State 5 to 6: Expansion in the nozzle  

TASK 2b  
The following values are provided:  
\[
p_0 = 0.151 \, \text{bar}, \quad T_0 = 243.15 \, \text{K}, \quad w_{\text{air}} = 200 \, \text{m/s}, \quad v = 0.5 \, \text{m}^3/\text{kg}
\]  
At state 5:  
\[
T_5 = 431.3 \, \text{K}, \quad w_5 = 220 \, \text{m/s}
\]  

TASK 2c  
The following calculations are performed:  
1. The gas constant \( R \) is calculated as:  
\[
R = c_p - c_v = 2.87943 \, \text{kJ/kg·K} - 2.8743 \, \text{kJ/kg·K} = 0.00513 \, \text{kJ/kg·K}
\]  
2. The specific heat at constant volume \( c_v \) is calculated as:  
\[
c_v = \frac{c_p}{\kappa} = \frac{2.87943}{1.4} = 2.057 \, \text{kJ/kg·K}
\]  

For the process from state 5 to state 6:  
- The entropy change is calculated using the adiabatic reversible relation:  
\[
s(T_5) = 2.66533 \, \text{kJ/kg·K}, \quad s(T_6) = 2.66533 \, \text{kJ/kg·K}
\]  
- The temperature \( T_6 \) is determined using the relation:  
\[
T_6 = T_5 \left( \frac{p_6}{p_5} \right)^{\frac{R}{c_p}} = 431.3 \left( \frac{0.151}{0.5} \right)^{\frac{0.00513}{2.87943}} = 328.097 \, \text{K}
\]  

TASK 2d  
The first law of thermodynamics is applied for states 5 to 6:  
\[
\Delta ex_{\text{flow}} = c_p (T_6 - T_5) + \frac{w_6^2}{2} - \frac{w_5^2}{2}
\]  
Substituting values:  
\[
\Delta ex_{\text{flow}} = 2.87943 (328.097 - 431.3) + \frac{220^2}{2} - \frac{200^2}{2} = 236.755 \, \text{m/s}
\]  

The outlet velocity \( w_6 \) is calculated using the equation:  
\[
w_6 = \sqrt{2 \left( c_p (T_6 - T_5) + \frac{w_5^2}{2} \right)} = \sqrt{2 \left( 2.87943 (328.097 - 431.3) + \frac{200^2}{2} \right)}
\]  
The final value for \( w_6 \) is boxed in the solution.