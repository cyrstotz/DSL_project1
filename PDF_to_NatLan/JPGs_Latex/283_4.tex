TASK 4a  
The graph depicts a pressure-temperature (\(P\)-\(T\)) diagram. The pressure axis is labeled in bar (\([ \text{bar} ]\)), and the temperature axis is labeled in Kelvin (\([ \text{K} ]\)). The curve represents the phase boundary between different states of the refrigerant. The peak of the curve is marked, indicating the critical point. The diagram shows the general shape of a phase diagram with a rising and falling curve.

---

TASK 4b  
The graph again represents a pressure-temperature (\(P\)-\(T\)) diagram. The pressure axis is labeled in bar (\([ \text{bar} ]\)), and the temperature axis is labeled in Kelvin (\([ \text{K} ]\)). The diagram includes additional annotations:  
- A horizontal line labeled "isobaric condensation" connects points 1 and 2.  
- A vertical line labeled "isothermal pressure reduction" connects points 2 and 3.  
- Two dashed horizontal lines are drawn at pressures of 10 mbar and 5 mbar, indicating specific pressure levels relevant to the sublimation process.  

The curve represents the phase boundary, and the critical point is marked similarly to the previous graph.