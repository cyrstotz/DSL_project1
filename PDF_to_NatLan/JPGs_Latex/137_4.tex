TASK 3a  
The gas pressure \( p_{g,1} \) is calculated as:  
\[
p_{g} = p_{w} = p_{1} = p_{2} = \frac{m_K \cdot g + p_{\text{amb}} \cdot A}{A} = 0.38397 \, \text{bar} + 1 \, \text{bar} = 1.38397 \, \text{bar}.
\]  
The area \( A \) is determined using the cylinder diameter \( D = 10 \, \text{cm} \):  
\[
A = \pi \cdot r^2 = \pi \cdot (5 \, \text{cm})^2 = 0.00785 \, \text{m}^2.
\]  

The gas mass \( m_g \) is calculated using the ideal gas law:  
\[
pV = mRT \quad \Rightarrow \quad m_g = \frac{pV}{RT}.
\]  
Substituting values:  
\[
V_g = A \cdot h = 0.00785 \, \text{m}^2 \cdot 0.05 \, \text{m} = 0.0003949 \, \text{m}^3,
\]  
\[
m_g = \frac{1.38397 \, \text{bar} \cdot 0.0003949 \, \text{m}^3}{R \cdot T_g} = 3.92 \, \text{g}.
\]  

---

TASK 3b  
Since ice is still present, the equilibrium temperature of the ice-water mixture remains constant at \( T_{\text{EW},2} = 0^\circ\text{C} \).  

The system is isobaric because the piston is free to move. Therefore:  
\[
p_{g,2} = p_{g,1} = 1.38397 \, \text{bar}.
\]  

The final state is thermodynamic equilibrium, meaning ice is still present, and all temperatures are \( 0^\circ\text{C} \):  
\[
T_{g,2} = T_{\text{EW},2} = 0^\circ\text{C}.
\]  

---

TASK 3c  
Initial state:  
\[
T_{g,1} = 500^\circ\text{C}, \quad V_{g,1} = 3.14 \, \text{L}, \quad p_{g,1} = 1.38397 \, \text{bar}.
\]  
Final state:  
\[
T_{g,2} = 0^\circ\text{C}.
\]  

The system is closed, and work \( W = 0 \). The gas mass remains constant:  
\[
m_2 = m_1.
\]  

The heat transferred \( Q \) is calculated as:  
\[
Q = m_g \cdot c_V \cdot (T_2 - T_1).
\]  
Substituting values:  
\[
Q = 3.92 \, \text{g} \cdot 0.633 \, \text{kJ/kg·K} \cdot (0^\circ\text{C} - 500^\circ\text{C}) = -1082.63 \, \text{J}.
\]  

This heat is absorbed by the ice-water mixture, melting some ice.