TASK 3a  
The pressure of the gas in state 1, \( p_{g,1} \), is calculated using the ideal gas law:  
\[
p_{g,1} \cdot V_{g,1} = m_g \cdot T_1 \cdot R
\]  
The specific gas constant \( R \) is determined as:  
\[
R = \frac{R_u}{M_g} = \frac{8314 \, \text{J/(kmol·K)}}{50 \, \text{kg/kmol}} = 166.28 \, \text{J/(kg·K)}
\]  
The pressure \( p_{g,1} \) is expressed as:  
\[
p_{g,1} = p_{\text{inner}} = p_0 + \frac{F_A}{A}
\]  
Where:  
\[
F_A = m_K \cdot g = 32 \, \text{kg} \cdot 9.81 \, \text{m/s}^2 = 313.92 \, \text{N}
\]  
The cross-sectional area \( A \) of the cylinder is calculated as:  
\[
A = \frac{\pi \cdot D^2}{4} = \frac{\pi \cdot (0.1 \, \text{m})^2}{4} = 0.00785 \, \text{m}^2
\]  
Thus:  
\[
p_{g,1} = p_0 + \frac{F_A}{A} = 1 \, \text{bar} + \frac{313.92 \, \text{N}}{0.00785 \, \text{m}^2} = 1 \, \text{bar} + 0.448 \, \text{bar} = 1.448 \, \text{bar}
\]  
The mass of the gas \( m_g \) is calculated as:  
\[
m_g = \frac{p_{g,1} \cdot V_{g,1}}{T_1 \cdot R} = \frac{1.448 \, \text{bar} \cdot 0.00314 \, \text{m}^3}{500 \, \text{K} \cdot 166.28 \, \text{J/(kg·K)}} = 0.00353 \, \text{kg} = 3.53 \, \text{g}
\]  

TASK 3b  
The equilibrium temperature of the ice-water mixture is given as \( T_{\text{EW}} = 0^\circ\text{C} \). The ice mass fraction \( x_{\text{ice},2} \) is calculated as:  
\[
x_{\text{ice},2} = \frac{m_{\text{ice}}}{m_{\text{EW}}}
\]  
Where:  
\[
m_{\text{ice}} = 0.6 \cdot 0.1 \, \text{kg} = 0.06 \, \text{kg}
\]  

General system energy balance is written as:  
\[
\Delta E = Q - W_{\text{vu}}
\]  
The energy balance includes contributions from the ice and water masses:  
\[
m_{\text{ice}}(u_2) + m_{\text{EW}}(u_2) + M
\]  
The final volume of the ice-water mixture is expressed as:  
\[
V_{\text{EW}} = V_2 \cdot m_{\text{EW}}
\]  

A small sketch is present, showing equilibrium conditions at \( p_{g,2} = p_{\text{amb}} = 1 \, \text{bar} \). Below the sketch, the text states:  
"Equilibrium reached due to internal energy exchange."  

No further calculations or explanations are visible.