TASK 4b  
The mass flow rate of the refrigerant is calculated using the following equation:  
\[
\dot{m} = \frac{-\dot{W}_K}{h_2 - h_3}
\]  
where \( \dot{W}_K \) is the work done by the compressor, and \( h_2 \) and \( h_3 \) are the specific enthalpies at states 2 and 3, respectively. The enthalpy values \( h_2 \) and \( h_3 \) are to be obtained from the relevant tables.

---

TASK 4a  
A table is provided with labeled columns for temperature (\( T \)), pressure (\( p \)), specific volume (\( V \)), vapor quality (\( x \)), work (\( W \)), and entropy (\( S \)). The entries for the states are as follows:  
- State 1: \( T = 1 \, \text{[bar]} \), \( W = -28W \), \( x = 1 \), \( S = 0 \).  
- State 2: \( p = 8 \, \text{[bar]} \).  
- State 3: \( p = 8 \, \text{[bar]} \).  
- State 4: \( p = 0 \, \text{[bar]} \), \( x = 0 \).  

Additionally, the entropy change between states 2 and 3 is noted as:  
\[
S_{23} = 0 \quad (\text{adiabatic and reversible process}).
\]

---

TASK 4a  
A pressure-temperature (\( p \)-\( T \)) diagram is drawn to represent the freeze-drying process. The diagram includes the following features:  
- The pressure axis is labeled as \( p \, [\text{bar}] \), ranging from 1 bar to 8 bar.  
- The temperature axis is labeled as \( T \, [^\circ\text{C}] \).  
- Four states are marked:  
  - State 1 and State 4 are connected by an isobaric line at 1 bar.  
  - State 2 and State 3 are connected by an isobaric line at 8 bar.  
  - The transition from State 1 to State 2 is labeled as "adiabatic reversible."  
  - The transition from State 3 to State 4 is labeled as "adiabatic reversible."  

Dashed lines are used to indicate the isobaric processes, and curved lines represent the adiabatic transitions.