TASK 2b  
The page begins with a graph labeled as a \( T \)-\( s \) diagram. The axes are marked as follows:  
- The vertical axis is labeled \( T \, [K] \), representing temperature in Kelvin.  
- The horizontal axis is labeled \( s \, \left[\frac{\text{kJ}}{\text{kg·K}}\right] \), representing specific entropy in kilojoules per kilogram per Kelvin.  

The diagram depicts a thermodynamic process with the following states and transitions:  
- State 1 to State 2: Isentropic compression (vertical line).  
- State 2 to State 3: Isobaric heat addition (curved line).  
- State 3 to State 4: Isentropic expansion (vertical line).  
- State 4 to State 5: Isobaric heat rejection (horizontal line).  
- State 5 to State 6: Isentropic compression (curved line).  
- State 6 to State 1: Isobaric heat rejection (horizontal line).  

TASK 2b  
The thermodynamic mean temperature \( \bar{T}_6 \) is calculated using the following formula:  
\[
\bar{T}_6 = T_s \cdot \left(\frac{p_6}{p_s}\right)^{\frac{0.4}{1.4}}
\]  
Substituting values, the result is:  
\[
\bar{T}_6 = 328.07 \, \text{K}
\]  

The exergy balance is expressed as:  
\[
O = \dot{m}_{\text{ges}} \left(h_s - h_6 + \frac{w_s^2 - w_6^2}{2}\right)
\]  
Where:  
\[
h_s - h_6 = c_p \cdot (T_s - T_6)
\]  
Substituting values:  
\[
h_s - h_6 = 404.45 \, \frac{\text{kJ}}{\text{kg}}
\]  

The velocity term \( \frac{w_6^2}{2} \) is calculated as:  
\[
\frac{w_6^2}{2} = c_p \cdot (T_s - T_6) + \frac{w_s^2}{2}
\]  
Substituting values:  
\[
\frac{w_6^2}{2} = 128.65 \, \frac{\text{kJ}}{\text{kg}}
\]  

Finally, the outlet velocity \( w_6 \) is determined as:  
\[
w_6 = \sqrt{2 \cdot 128.65 \, \frac{\text{kJ}}{\text{kg}}}
\]  
The result is:  
\[
w_6 = 507.25 \, \text{m/s}
\]  

A small sketch appears next to the calculations, showing a shaded triangular region, possibly representing an entropy-related area, but it is not labeled or explained further.