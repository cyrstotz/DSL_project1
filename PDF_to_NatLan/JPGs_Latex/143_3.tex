TASK 2a  
The first diagram is a qualitative \( T \)-\( s \) diagram representing the jet engine process. The axes are labeled as follows:  
- The vertical axis is temperature \( T \) in Kelvin (K).  
- The horizontal axis is entropy \( s \) in \( \text{kJ}/\text{kg·K} \).  

Key points and processes are marked:  
- Points 0, 1, 2, 3, 4, and 5 are labeled, corresponding to different states in the jet engine.  
- \( p_2 = p_3 \), \( s_2 = s_3 \), \( p_4 = p_5 = 0.5 \, \text{bar} \), \( p_0 = 0.191 \, \text{bar} \), and \( s_5 = s_6 \).  
- The diagram includes isobars and curves representing compression, combustion, and expansion processes.  

The second diagram is another \( T \)-\( s \) diagram with more detailed isobaric lines and annotations:  
- The states 0, 1, 2, 3, 4, and 5 are connected by polytropic and isentropic processes.  
- The axis labels remain the same as the first diagram.  
- The note "nicht nullpunkt" indicates that the entropy axis does not start at zero.  

TASK 2b  
The enthalpy at state 5 is calculated using the equation:  
\[
h_s = c_p \cdot T_5 = 431.49 \, \text{kJ/kg}
\]  

An energy balance for the nozzle is performed, assuming adiabatic conditions (\( Q = 0 \)) and steady-state operation. The equation is:  
\[
0 = \dot{m}_{\text{ges}} \left( h_5 - h_6 + \frac{1}{2} w_5^2 - \frac{1}{2} w_6^2 \right)
\]  
Rearranging for \( h_6 \):  
\[
h_6 = h_5 + \frac{1}{2} \left( w_5^2 - w_6^2 \right)
\]  

The temperature \( T_6 \) is determined using the polytropic temperature relation:  
\[
T_6 = T_s \left( \frac{p_0}{p_s} \right)^{\frac{k-1}{k}} = 293.9 \, \text{K}
\]  
where \( h_6 = c_p \cdot T_6 \).  

The outlet velocity \( w_6 \) is calculated as:  
\[
w_6 = \left( 2 \cdot (h_5 - h_6) + w_5^2 \right)^{1/2} = 571 \, \text{m/s}
\]  

The page ends with the word "Rückseite!" indicating continuation on the reverse side.