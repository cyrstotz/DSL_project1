TASK 4a  
The diagram is a pressure-temperature (\( p \)-\( T \)) plot illustrating the freeze-drying process. It includes labeled phase regions and key states (1, 2, 3, and 4).  
- State 1: Located in the wet vapor region.  
- State 2: Transitioning through the wet vapor region.  
- State 3: Isobaric compression to the superheated vapor region.  
- State 4: Isobaric condensation back to the liquid phase.  
The process involves isobaric and adiabatic transformations, with arrows indicating the direction of the cycle.  

TASK 4b  
The initial temperature \( T_i \) is given as \( -10^\circ\text{C} \), and the evaporator temperature \( T_{\text{evaporator}} \) is \( -16^\circ\text{C} \).  
The pressure \( p_{\text{evaporator}} \) is 5 bar, and the vapor quality \( x_1 \) is 0.  

Using enthalpy values:  
- \( h_1 = 53.42 \, \text{kJ/kg} \) (from saturated liquid at state 1).  
- \( h_2 = 269.45 \, \text{kJ/kg} \) (from Table A-11).  
- \( h_3 = 837.237.4 \, \text{kJ/kg} \) (from Table A-10).  

The mass flow rate \( \dot{m} \) is calculated using the first law of thermodynamics:  
\[
0 = \dot{m} \left( h_2 - h_3 \right) + \dot{W}_K  
\]
Rearranging:  
\[
\dot{m} = \frac{\dot{W}_K}{h_3 - h_2} = 0.000406 \, \text{kg/s}  
\]

TASK 4c  
The first law of thermodynamics is applied to different components of the refrigeration cycle:  
1. For the evaporator:  
\[
0 = \dot{m} \left( h_1 - h_2 \right) + \dot{Q}_K  
\]  
2. For the condenser:  
\[
0 = \dot{m} \left( h_3 - h_4 \right) + \dot{Q}_{\text{cond}}  
\]  
3. For the compressor:  
\[
0 = \dot{m} \left( h_2 - h_3 \right) - \dot{W}_K  
\]  

No further numerical values are provided for these equations.  

No additional content or figures are visible.