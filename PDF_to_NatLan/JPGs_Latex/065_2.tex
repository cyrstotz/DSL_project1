TASK 2a  
The diagram is a qualitative representation of the jet engine process in a \( T \)-\( s \) diagram. The axes are labeled as follows:  
- The vertical axis represents temperature \( T \) in Kelvin \([K]\).  
- The horizontal axis represents entropy \( s \) in kilojoules per kilogram Kelvin \([kJ/kg·K]\).  

The process is depicted with six states labeled \( 0, 1, 2, 3, 4, 5, \) and \( 6 \).  
- State \( 0 \) is the ambient condition.  
- State \( 1 \) is the inlet air condition.  
- States \( 2 \) and \( 3 \) correspond to the compression process, with \( 3 \) being at the highest temperature and pressure.  
- States \( 4 \) and \( 5 \) represent the combustion and turbine processes, respectively.  
- State \( 6 \) is the nozzle exit condition.  

The diagram includes three isobaric lines:  
- \( p_2 = p_3 \), representing the compression process.  
- \( p_4 = p_5 \), representing the combustion process.  
- \( p_0 \), representing the ambient pressure.  

Dashed lines indicate the isobaric processes, and solid lines represent the transitions between states.

---

TASK 2b  
The process from state \( 5 \) to state \( 6 \) is described as isentropic.  

The temperature at state \( 6 \), \( T_6 \), is calculated using the isentropic relation:  
\[
T_6 = T_5 \left( \frac{p_6}{p_5} \right)^{\frac{\kappa - 1}{\kappa}}
\]  
Substituting the given values:  
\[
T_6 = 431.9 \, \text{K} \left( \frac{0.191 \times 10^5 \, \text{Pa}}{0.5 \times 10^5 \, \text{Pa}} \right)^{\frac{0.4}{1.4}}
\]  
\[
T_6 = 328.07 \, \text{K}
\]  

Next, the energy balance at the nozzle is applied:  
\[
0 = \dot{m} \left[ h_6 - h_5 + \frac{w_6^2}{2} - \frac{w_5^2}{2} \right] + \dot{Q} + \dot{W}
\]  
Since the nozzle is adiabatic and there is no work transfer:  
\[
0 = h_6 - h_5 + \frac{w_6^2}{2} - \frac{w_5^2}{2}
\]  

Using the ideal gas assumption:  
\[
h_5 - h_6 = c_p (T_5 - T_6)
\]  
\[
h_5 - h_6 = 1.006 \, \text{kJ/kg·K} \times (431.9 \, \text{K} - 328.07 \, \text{K}) = 104.45 \, \text{kJ/kg}
\]  

Rearranging for \( w_6 \):  
\[
w_6^2 = w_5^2 + 2 (h_5 - h_6)
\]  
Substituting \( w_5 = 220 \, \text{m/s} \):  
\[
w_6 = \sqrt{220^2 + 2 \times 104.45 \times 10^3}
\]  
\[
w_6 = 507.25 \, \text{m/s}
\]  

The outlet velocity at state \( 6 \) is \( w_6 = 507.25 \, \text{m/s} \).