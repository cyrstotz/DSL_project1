TASK 2a  
Two diagrams are drawn to represent thermodynamic processes on a \( T \)-\( s \) (temperature-entropy) diagram.  

1. **First Diagram**:  
   - The diagram shows a process with labeled isobars \( P_0 \), \( P_4 \), and \( P_5 \).  
   - The curve starts at point \( 1 \), moves through points \( 2 \), \( 3 \), and ends at \( 4 \).  
   - The isobars are curved lines, and the process transitions between them.  
   - The axes are labeled \( T \) (temperature) and \( s \) (entropy).  

2. **Second Diagram**:  
   - This diagram also represents thermodynamic processes on a \( T \)-\( s \) diagram.  
   - It includes labeled isobars \( P_0 \), \( P_4 \), \( P_5 \), and \( P_7 \).  
   - The process starts at point \( 1 \), moves through points \( 2 \), \( 3 \), \( 4 \), \( 5 \), and ends at \( 6 \).  
   - The axes are labeled \( T \) (temperature) and \( s \) (entropy).  

TASK 2b  
The following equations and calculations are provided:  

1. **Energy Balance Equation**:  
   \[
   O = \dot{m} \left( h_5 - h_6 + \frac{w_5^2}{2} - \frac{w_6^2}{2} \right) - \dot{W}_{56}
   \]  
   where \( \dot{W}_{56} \) represents work, and \( \dot{m} \) is the mass flow rate.  

2. **Work per Unit Mass Flow**:  
   \[
   \frac{\dot{W}_{56}}{\dot{m}} = - \int_5^6 v \, dp + \Delta pe
   \]  
   The integral represents the work done during an isobaric process.  

3. **Simplified Energy Balance**:  
   \[
   h_5 - h_6 + \frac{w_5^2}{2} - \frac{w_6^2}{2} = 0
   \]  

4. **Velocity Calculation**:  
   \[
   w_6^2 = 2 \left( h_6 - h_5 \right) + w_5^2
   \]  
   Substituting values:  
   \[
   w_6^2 = 2 c_p \left( T_6 - T_5 \right) + w_5^2
   \]  
   Numerical values are provided:  
   \[
   w_6 = \sqrt{2 \cdot c_p \cdot (T_6 - T_5) + w_5^2}
   \]  

5. **Temperature Calculation**:  
   \[
   T_6 = T_5 \cdot \left( \frac{P_6}{P_5} \right)^{\frac{\kappa - 1}{\kappa}}
   \]  
   Substituting values:  
   \[
   T_6 = 328.07 \, \text{K}
   \]  

No additional diagrams or figures are present beyond the \( T \)-\( s \) diagrams described above.