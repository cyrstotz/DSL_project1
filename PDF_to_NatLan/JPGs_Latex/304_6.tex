TASK 4a  
The graph is a pressure-temperature (\(p\)-\(T\)) diagram. The vertical axis represents pressure (\(p\)), and the horizontal axis represents temperature (\(T\)). The graph includes a curve that starts at low pressure and temperature, increasing steadily. Around \(T = 10\), there is a sharp vertical line indicating a phase change or transition. The axes are labeled, with pressure ranging from 1 to 10 and temperature ranging from -10 to 10.  

TASK 4b  
The flow process around the compressor is described using an energy balance:  
\[
m \cdot (h_1 - h_3) + W_K = 0
\]  
Rearranging gives:  
\[
m \cdot h_1 = \frac{W_K}{h_2 - h_3}
\]  

For an isenthalpic throttle:  
\[
h_u = h_l
\]  
Given \(p_3 = p_4\), it follows that:  
\[
h_u = h_l = h_f \, (8 \, \text{bar})
\]  
From the tables (Table A-11):  
\[
h_f = 93.42 \, \frac{\text{kJ}}{\text{kg}}
\]