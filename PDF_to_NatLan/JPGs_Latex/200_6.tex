TASK 3a  
The gas pressure \( p_{g,1} \) is calculated as follows:  
\[
p_{g,1} = p_{\text{amb}} + \frac{(m_K + m_{\text{EW}}) g}{A}
\]  
The cross-sectional area \( A \) is determined using the formula:  
\[
A = \frac{D^2}{4} \pi = 0.00785 \, \text{m}^2
\]  
Substituting values:  
\[
p_{g,1} = 1 \, \text{bar} + \frac{(32 \, \text{kg} + 0.1 \, \text{kg}) \cdot g}{0.00785 \, \text{m}^2}
\]  
\[
p_{g,1} = 10^5 \, \text{Pa} + 40,779.78 \, \text{Pa}
\]  
\[
p_{g,1} = 7.4 \, \text{bar}
\]  

The gas mass \( m_g \) is calculated using the ideal gas law:  
\[
m_g = \frac{p_{g,1} V_{g,1}}{R T_{g,1}}
\]  
The specific gas constant \( R \) is derived as:  
\[
R = \frac{R}{M_g} = 0.7663 \, \frac{\text{kJ}}{\text{kg·K}}
\]  
Given \( V_{g,1} = 3.14 \cdot 10^{-3} \, \text{m}^3 \), the gas mass is:  
\[
m_g = 3.919 \, \text{g}
\]  

TASK 3b  
The pressure will remain constant because the atmospheric pressure and the weight of the piston do not change. Therefore:  
\[
p_{g,1} = p_{g,2}
\]  
The temperature will decrease to \( 0^\circ\text{C} \) as the ice-water mixture (EW) reaches equilibrium. This means that water and ice coexist at \( 0^\circ\text{C} \), which is the equilibrium temperature for this system.