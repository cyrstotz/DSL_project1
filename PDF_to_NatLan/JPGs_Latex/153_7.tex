TASK 4a  
Two diagrams are drawn to represent the freeze-drying process in a pressure-temperature (\(p\)-\(T\)) diagram.  

1. **Left Diagram**:  
   - The \(p\)-\(T\) curve shows the phase regions of a substance.  
   - The curve includes a labeled "Critical Point" at the top of the dome-shaped region.  
   - The region under the dome is labeled "Nassdampf" (wet steam).  
   - The axes are labeled as \(p\) (pressure in bar) and \(T\) (temperature in Kelvin).  

2. **Right Diagram**:  
   - The diagram shows the sublimation process.  
   - A line labeled "Triple Point" marks the intersection of solid, liquid, and gas phases.  
   - The gas phase is labeled, and the sublimation process is indicated with an arrow labeled "Sublimation."  
   - The axes are labeled as \(p\) (pressure in bar) and \(T\) (temperature in Kelvin).  
   - A label "10 K" appears near the sublimation line, indicating the temperature difference above the sublimation temperature.  

TASK 4b  
The energy balance equation for the refrigerant mass flow rate is written as:  
\[
\dot{m} \cdot [h_2 - h_3] = \dot{W}_\text{K}
\]  
Rearranging for the mass flow rate:  
\[
\dot{m} = \frac{\dot{W}_\text{K}}{h_2 - h_3}
\]  

Additional conditions and parameters are listed:  
- \(p_1 = p_2\), \(s_2 = s_3\), \(p_3 = 8 \, \text{bar}\).  
- \(p_3 = p_4\), \(x_4 = 0\), \(x_2 = 1\).  
- \(T_1 = T_4\), \(T_\text{Verdampfer} = 9^\circ\text{C}\).  
- \(T_\text{Innenraum} = 10^\circ\text{C}\), \(h_1 = h_q\).  