TASK 3c  
The heat transferred from the gas to the ice-water mixture is calculated using the formula:  
\[
Q_{1,2} = |m_{g,1} \, c_V \, (\Delta T)|
\]  
Substituting the given values:  
\[
Q_{1,2} = 3.6 \times 10^{-3} \cdot 0.633 \cdot (500^\circ\text{C} - 0^\circ\text{C})
\]  
\[
Q_{1,2} = 1,139 \, \text{kJ}
\]  

---

TASK 3d  
The equilibrium temperature of the gas and ice-water mixture is given as:  
\[
T_{g,12} = T_{E,12} = 0^\circ\text{C}
\]  

The change in specific internal energy is calculated as:  
\[
\Delta u = \frac{Q_{1,2}}{m} = \frac{1500}{0.1} = 150 \, \frac{\text{kJ}}{\text{kg}} = 0.150 \, \frac{\text{MJ}}{\text{kg}}
\]  

The ice fraction is determined using the formula:  
\[
x = \frac{u_{\text{tot}} - u_{\text{post}}}{u_{\text{ice}} - u_{\text{post}}}
\]  

The change in internal energy for the post-melting state is calculated as:  
\[
\Delta u_{\text{post}} = u_l(0^\circ\text{C}) - u_l(0.003) = (0.045 - 0.033) \, \text{kJ/kg} = 0.012 \, \text{kJ/kg}
\]  

The change in internal energy for the ice state is calculated as:  
\[
\Delta u_{\text{ice}} = -333.458 \, \text{kJ/kg} - 333.442 \, \text{kJ/kg} = -333.16 \, \text{kJ/kg}
\]  

The change in ice fraction is then calculated as:  
\[
\Delta x = \frac{0.150 - 333.16}{0.012 - 333.16} = 0.999
\]  