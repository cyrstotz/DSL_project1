TASK 3a  
The gas pressure \( p_{g,1} \) is calculated using the formula:  
\[
p_{g,1} = \frac{m_{g} \cdot R \cdot T_{g,1}}{V_{g,1}} + \frac{m_{K} \cdot g}{A} + \frac{m_{\text{EW}} \cdot g}{A}
\]  
Substituting the values:  
\[
p_{g,1} = \frac{32 \cdot 9.81}{0.00785} + \frac{0.1 \cdot 9.81}{0.00785} = 140144 \, \text{Pa}
\]  
Thus, \( p_{g,1} = 140.1 \, \text{Bar} \).  

The ideal gas law is used to calculate the molar mass:  
\[
pV = mRT
\]  
The molar mass \( \bar{M} \) is determined as:  
\[
\bar{M} = \frac{\bar{R}}{M} = \frac{8.314}{50} = 0.16628
\]  

The mass of the gas is calculated as:  
\[
m = \frac{pV}{RT} = \frac{1.404 \cdot 10^5 \cdot 3.14 \cdot 10^{-3}}{466.28 \cdot 773.15} = 0.00342 \, \text{kg}
\]  
This corresponds to \( 3.42 \, \text{g} \).  

---

TASK 3b  
The final ice fraction \( x_{\text{ice},2} > 0 \) is determined under equilibrium conditions. The temperature \( T_{g,2} \) and pressure \( p_{g,2} \) are calculated as follows:  

The ice fraction \( x_2 \) is given as \( x_2 = 0.2 \). The mass of the ice-water mixture \( m_{\text{EW}} \) is \( 0.1 \, \text{kg} \).  

The specific heat capacity of the gas is \( c_V = 0.633 \, \text{kJ/kg·K} \). Using the transferred heat:  
\[
Q = m_{\text{gas}} \cdot c_V \cdot (T_2 - T_1)
\]  

Rewriting the energy balance:  
\[
\frac{Q}{m_{\text{gas}} \cdot c_V} = T_2 - T_1
\]  
Substituting values:  
\[
T_2 = T_1 + \frac{Q}{m_{\text{gas}} \cdot c_V}
\]  
\[
T_2 = 773.15 - \frac{0.002464}{3.42 \cdot 10^{-3} \cdot 0.633}
\]  

Further calculations are incomplete on the page.  

---

No diagrams or graphs are present on this page.