TASK 4a  
The diagrams depict the freeze-drying process in a pressure-temperature (\(P\)-\(T\)) diagram.  

1. The first graph shows the compressed gas phase, with a curve labeled "compressed" and "gas phase." The axes are pressure (\(P\)) and temperature (\(T\)).  
2. The second graph shows a phase diagram with labeled regions. The pressure decreases below the sublimation point, and the temperature is held constant. The axes are pressure (\(P\)) and temperature (\(T\)).  
3. The third graph illustrates the refrigeration cycle. It includes labeled states:  
   - State 1: after adiabatic expansion  
   - State 2: isobaric evaporation  
   - State 3: reversible adiabatic compression  
   - State 4: isobaric condensation  
   The axes are pressure (\(P\)) and temperature (\(T\)).  

TASK 4d  
The coefficient of performance (\(\epsilon_K\)) is defined as:  
\[
\epsilon_K = \frac{\dot{Q}_K}{\dot{W}_K} = \frac{\dot{m} \cdot (h_2 - h_1)}{\dot{W}_K}
\]  
where:  
- \(\dot{Q}_K\) is the heat removed by the refrigerant.  
- \(\dot{m}\) is the mass flow rate of the refrigerant.  
- \(h_2\) and \(h_1\) are the specific enthalpies at states 2 and 1, respectively.  

TASK 4e  
The text states:  
"Temperatur würde fallen."  
Translation: "The temperature would decrease."  

No additional explanation is provided.