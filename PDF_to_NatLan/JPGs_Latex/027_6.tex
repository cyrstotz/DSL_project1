TASK 4a  
A p-T diagram is drawn, showing the phase regions for gas, liquid, and solid. The triple point (TP) is labeled, and the diagram includes the following annotations:  
- "Gas" region is above the curve.  
- "Water curves back" is noted near the gas-liquid boundary.  
- "Solid" is labeled below the triple point.  
- "Isotherm under TP" is marked near the solid-liquid boundary.  
- "Isobar (freezer)" is drawn horizontally near the liquid-solid boundary.  

TASK 4b  
The energy balance around the evaporator is written as:  
\[
0 = \dot{m} (h_e - h_a) + \dot{Q} - \dot{W}_k
\]  
This represents a steady-state process.  
- "New Dampf (vollständig verdampft)" indicates fully evaporated vapor.  
- "Dampf-Flüssig-Gemisch (8 bar)" refers to the vapor-liquid mixture at 8 bar.  
From the table (TAB A11):  
\[
h_f = 93.42 \, \text{kJ/kg}, \quad h_g = 266.15 \, \text{kJ/kg}
\]  

TASK 4c  
The vapor quality is expressed as:  
\[
x = h
\]  

TASK 4d  
The coefficient of performance \( \epsilon_K \) is calculated as:  
\[
\epsilon_K = \frac{\dot{Q}_K}{|\dot{W}_K|} = \frac{Q_u - Q_{ab}}{28 \, \text{W}}
\]  

TASK 4e  
The temperature would theoretically decrease to absolute zero and stop there. However, it becomes increasingly difficult to cool the system as it approaches colder temperatures.