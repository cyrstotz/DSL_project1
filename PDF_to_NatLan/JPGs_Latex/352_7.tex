TASK 4a  
A pressure-temperature (\( p \)-\( T \)) diagram is drawn to represent the freeze-drying process. The diagram includes the following features:  
- A curve labeled "Gas" indicating the phase boundary.  
- A horizontal line labeled "Tripel" marking the triple point of water.  
- Three states labeled \( 1 \), \( 2 \), and \( 3 \):  
  - State \( 1 \) is at the beginning of the process.  
  - State \( 2 \) is after step (i), following isobaric cooling.  
  - State \( 3 \) is after step (ii), following isothermal pressure reduction.  
- Vertical arrows indicate transitions between states, labeled "Kühlen" (cooling) and "Druckabl." (pressure reduction).  
- The region above the curve is labeled "Fest" (solid), and the region below is labeled "Flüssig" (liquid).  

TASK 4b  
The following conditions are specified:  
1. \( x_2 = 1 \), indicating the vapor quality at state \( 2 \).  
2. \( p_3 = 8 \, \text{bar} \), and \( s_3 = s_2 \), indicating that the entropy remains constant between states \( 2 \) and \( 3 \).  
3. \( p_4 = p_3 = 8 \, \text{bar} \), indicating that the pressure remains constant between states \( 3 \) and \( 4 \).  
4. \( p_i = 598300 \, \text{Pa} \), specifying the pressure at a certain point.  

A table is provided summarizing the states:  
\[
\begin{array}{|c|c|c|c|}
\hline
\text{Zustand} & p \, (\text{bar}) & \bar{T} \, (\text{K}) & x \\
\hline
1 & & & \\
2 & & 1 & \\
3 & 8 & & \\
4 & 8 & & 0 \\
\hline
\end{array}
\]  
The table lists the state number (\( \text{Zustand} \)), pressure (\( p \)), temperature (\( \bar{T} \)), and vapor quality (\( x \)).