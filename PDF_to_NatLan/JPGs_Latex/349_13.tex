TASK 4a  
The diagram is a pressure-temperature (\( p \)-\( T \)) phase diagram. It shows the phase regions for a substance:  

- The solid phase ("Fest") is located at the lower left.  
- The liquid phase ("Flüssig") is in the middle region.  
- The gaseous phase ("Gasförmig") is at the lower right.  

The triple point ("Tripel") is marked where the solid, liquid, and gaseous phases coexist.  

Two steps are illustrated with arrows:  
- Step "i" starts in the liquid phase and moves downward toward the triple point.  
- Step "ii" moves horizontally to the left, crossing into the gaseous phase.  

The axes are labeled:  
- The vertical axis represents pressure (\( p \)) in bar.  
- The horizontal axis represents temperature (\( T \)) in Kelvin (\( K \)).  

No additional textual content is provided.