TASK 4a  
The page contains three diagrams related to the freeze-drying process with R134a. Each diagram is a pressure-temperature (\( P \)-\( T \)) graph, showing phase regions and transitions.  

1. **First Diagram**:  
   - The graph is labeled "R134a" and shows the phase regions: "fest" (solid), "flüssig" (liquid), and "gas" (gaseous).  
   - The triple point is marked as "Tripel," where the solid, liquid, and gas phases coexist.  
   - The axes are labeled \( P \) (pressure, in unspecified units) on the vertical axis and \( T \) (temperature, in Kelvin) on the horizontal axis.  
   - The lines separating the phases are curved, indicating the boundaries between solid-liquid, liquid-gas, and solid-gas regions.  

2. **Second Diagram**:  
   - This graph also shows the phase regions ("fest," "flüssig," and "gasförmig").  
   - The triple point is marked as "Tripel."  
   - Two steps of the freeze-drying process are illustrated:  
     - Step "i" is shown as a horizontal line, indicating isobaric evaporation.  
     - Step "ii" is shown as a vertical line, indicating pressure reduction below the triple point.  
   - The axes are labeled \( P \) (pressure, in bar) and \( T \) (temperature, in Kelvin).  

3. **Third Diagram**:  
   - This graph shows similar phase regions and transitions as the previous diagrams.  
   - The triple point is marked, and phase boundaries are drawn.  
   - The axes are labeled \( P \) (pressure, in bar) and \( T \) (temperature, in Kelvin).  
   - Additional lines intersect, possibly representing specific processes or states in the freeze-drying cycle.  

These diagrams visually represent the freeze-drying process and the phase behavior of R134a.