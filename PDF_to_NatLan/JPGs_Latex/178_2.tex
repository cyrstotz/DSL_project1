TASK 2a  
The page contains multiple diagrams related to thermodynamic processes, specifically temperature (\( T \)) versus entropy (\( S \)) plots. Below is a description of each diagram:

1. **Top Diagram**:  
   - The graph is labeled with \( T \, [K] \) on the vertical axis and \( S \, [kJ/kg] \) on the horizontal axis.  
   - The process begins at point \( 1 \), moves to point \( 2 \) along an upward curve, and transitions to point \( 3 \) via an isobaric process.  
   - From point \( 3 \), the process moves downward to point \( 4 \), then continues to point \( 5 \) and finally to point \( 6 \).  
   - The diagram includes arrows indicating the direction of the process flow.  

2. **Middle Left Diagram**:  
   - This appears to be an isolated arrow pointing upward, possibly indicating a flow or direction of energy transfer.  

3. **Bottom Left Diagram**:  
   - Similar to the top diagram, this graph is labeled with \( T \, [K] \) on the vertical axis and \( S \) on the horizontal axis.  
   - The process starts at point \( 1 \), moves to point \( 2 \), then transitions to point \( 3 \) along an isobaric path.  
   - The process flows downward to point \( 4 \), continues to point \( 5 \), and ends at point \( 6 \).  
   - Arrows indicate the direction of the process flow.  

4. **Bottom Right Diagram**:  
   - This graph is also labeled with \( T \, [K] \) on the vertical axis and \( S \) on the horizontal axis.  
   - The process begins at point \( 1 \), moves to point \( 2 \), and transitions to point \( 3 \) via an isobaric process.  
   - From point \( 3 \), the process moves downward to point \( 4 \), continues to point \( 5 \), and ends at point \( 6 \).  
   - The diagram explicitly labels the segments as "isobaric" processes where applicable.  

Each diagram represents thermodynamic cycles or processes, with clear labeling of states and transitions.