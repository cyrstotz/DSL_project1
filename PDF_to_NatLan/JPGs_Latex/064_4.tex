TASK 2a  
The page contains a graph labeled as a \( T \)-\( s \) diagram. The axes are marked as follows:  
- The vertical axis is labeled \( T \, [\text{K}] \), representing temperature in Kelvin.  
- The horizontal axis is labeled \( s \, [\frac{\text{kJ}}{\text{kg·K}}] \), representing specific entropy in kilojoules per kilogram per Kelvin.  

The graph depicts a thermodynamic process with several states labeled numerically:  
- State 0 is at the bottom left of the diagram.  
- State 1 is slightly above state 0, connected by a curve.  
- State 2 is higher up and connected to state 1 by another curve.  
- State 3 is at the peak of the diagram, representing the highest temperature.  
- States 4 and 5 are on the descending curve, with state 5 being lower than state 4.  
- State 6 is at the bottom right of the diagram, connected to state 5.  

The graph includes annotations for "isobar" curves, indicating constant pressure processes. These curves are drawn to show transitions between states.  

This diagram qualitatively represents the thermodynamic process of a jet engine, as described in Task 2a.