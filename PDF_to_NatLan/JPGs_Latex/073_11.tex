TASK 4d  
The coefficient of performance \( \epsilon_K \) is defined as:  
\[
\epsilon_K = \frac{\dot{Q}_K}{\dot{W}_K} = \frac{\dot{m}_{\text{R134a}} (h_2 - h_1)}{\dot{m}_{\text{R134a}} (h_4 - h_1) - \dot{W}_K}
\]  

The work done by the compressor \( \dot{W}_K \) is expressed as:  
\[
\dot{W}_K = \dot{m}_{\text{R134a}} (h_4 - h_1)
\]  

The heat removed \( \dot{Q}_K \) is given by:  
\[
\dot{Q}_K = \dot{m}_{\text{R134a}} (h_2 - h_1)
\]  

---

TASK 4e  
The temperature would continue to decrease until the system reaches the critical region.  

---

TASK 4a  
The graph depicts a pressure-temperature (\( p \)-\( T \)) diagram. The vertical axis represents pressure (\( p \) in bar), and the horizontal axis represents temperature (\( T \) in Kelvin). The curve shows a typical phase boundary, starting with a steep increase, reaching a peak, and then gradually decreasing. This represents the transition between different phases of the refrigerant.