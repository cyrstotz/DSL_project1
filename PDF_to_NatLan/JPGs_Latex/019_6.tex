TASK 4a  
The diagram is a pressure-temperature (\(P\)-\(T\)) graph. It shows the freeze-drying process with labeled states.  
- The vertical axis represents pressure in bar (\([ \text{bar} ]\)).  
- The horizontal axis represents temperature in Kelvin (\([ \text{K} ]\)).  
- The curve depicts the phase boundaries.  
- States \(1\), \(2\), \(3\), and \(4\) are marked along the process path.  
- The process transitions from state \(1\) to \(2\), then to \(3\), and finally to \(4\).  
- State \(3\) is labeled at 8 bar, indicating the pressure after compression.  

TASK 4b  
The equation for energy balance is written as:  
\[
0 = \dot{m} (h_e - h_a) + \dot{Q} - \dot{W}
\]  
The process from \(2\) to \(3\) is described as adiabatic and reversible, leading to isentropic conditions.  

Additional notes:  
- \(p_3 = 8 \, \text{bar}\)  
- \(T_3\) is not explicitly provided.  
- \(s_3 = s_2\), indicating entropy remains constant between states \(2\) and \(3\).  