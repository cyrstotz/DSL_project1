TASK 3c  
The temperature \( T_{g,12} \) is given as \( 0.003^\circ\text{C} \), the pressure \( p \) is \( 7.400 \, \text{bar} \), and the mass \( m_g \) is \( 3.427 \times 10^{-3} \, \text{kg} \).  

Using the first law of thermodynamics for the gas:  
\[
\Delta U_{12} = m_g Q_{12} - W_{12}
\]  
Kinetic and potential energy are neglected.  

The change in internal energy for the gas is calculated as:  
\[
\Delta U_{12} = c_V (T_2 - T_1) = -376.496 \, \text{kJ/kg}
\]  
where \( c_V \) is the specific heat capacity of the gas.  

The total internal energy change is:  
\[
U_E = m_g \Delta U_{12} = 1.0824 \, \text{kJ} = Q_{12}
\]  

---

TASK 3d  
Using the first law of thermodynamics for the ice-water mixture (EW):  
\[
\Delta U_{12} = Q_{12} - W_{12}
\]  
where \( W_{12} \) is neglected due to incompressibility.  

The change in internal energy is:  
\[
\Delta U_{12} = Q_{12} / m_{\text{EW}}
\]  

The internal energy at state 2 is given as:  
\[
U_2 = U_{\text{frost}} + x_2 (U_{\text{fusion}} - U_{\text{frost}})
\]  
where \( x_2 \) is the ice fraction at state 2.  

The pressure on the EW is calculated as:  
\[
p_{\text{EW}} = p_{1,\text{EW}} = 7.394 \, \text{bar}
\]  

The internal energy values are derived from tables:  
\[
U_{\text{fusion}} = -200.925 \, \text{kJ/kg}, \quad U_{\text{frost}} = -137.000 \, \text{kJ/kg}
\]  

The ice fraction \( x_2 \) is calculated as:  
\[
x_2 = \frac{U_2 - U_{\text{frost}}}{U_{\text{fusion}} - U_{\text{frost}}} = 0.420
\]  

The remaining liquid fraction is:  
\[
x_{\text{EW}} = 1 - x_2 = 0.570
\]