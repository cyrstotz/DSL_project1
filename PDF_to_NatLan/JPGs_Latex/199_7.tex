TASK 4a  
The diagram is a pressure-temperature (\( p(T) \)) graph. It shows phase regions labeled as "solid" (\( \text{fest} \)), "liquid" (\( \text{flüssig} \)), and "gas." The graph includes two steps:  
- Step i: A horizontal line indicating isobaric evaporation.  
- Step ii: A vertical line showing the pressure reduction below the triple point of water, leading to sublimation.  

TASK 4e  
The text states:  
"The temperature would continue to decrease."  

TASK 4c  
The calculation for the vapor quality \( x \) is as follows:  
\[
h_2 = h_{\text{f}} + x \cdot (h_{\text{g}} - h_{\text{f}})
\]  
Rearranging for \( x \):  
\[
x = \frac{h_2 - h_{\text{f}}}{h_{\text{g}} - h_{\text{f}}}
\]  
The enthalpy \( h_{\text{g}} \) is calculated using:  
\[
h_{\text{g}} = h_{\text{f}} + (8 \, \text{bar}) \cdot \left( \frac{\partial h}{\partial p} \right)_{\text{A-12}}
\]  
Substituting values:  
\[
h_{\text{g}} = 93.62 \, \text{kJ/kg}
\]  
Finally:  
\[
x = \frac{h_2 - h_{\text{f}}}{h_{\text{g}} - h_{\text{f}}}
\]  
The note at the bottom suggests using the values for \( h_{\text{f}} \) and \( h_{\text{g}} \) from the relevant tables.  

No additional content is visible.