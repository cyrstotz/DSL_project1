TASK 4b

The mass flow rate of the refrigerant is calculated using the formula:
\[
\dot{m} = \frac{W_K}{h_3 - h_1} = \frac{-28 \, \text{W}}{237.74 \, \frac{\text{kJ}}{\text{kg}} - 771.3 \, \frac{\text{kJ}}{\text{kg}}} = 0.000835 \, \frac{\text{kg}}{\text{s}}
\]

---

TASK 4c

The enthalpy at state 4 is equal to \( h_4 \), which corresponds to the saturated liquid at 8 bar. Since the vapor quality \( x = 0 \), the enthalpy is:
\[
h_4 = h_f(8 \, \text{bar}) = 93.42 \, \frac{\text{kJ}}{\text{kg}}
\]

---

TASK 4d

The coefficient of performance \( \epsilon_K \) is calculated using the formula:
\[
\epsilon_K = \frac{\dot{Q}_{zu}}{\dot{W}_K} = \frac{\dot{Q}_K}{-28 \, \text{W}}
\]

The energy balance for the cooling process is:
\[
\dot{m}(h_1 - h_4) + \dot{Q}_K = 0
\]

Substituting values:
\[
0.000835 \, \frac{\text{kg}}{\text{s}} \left( 93.42 \, \frac{\text{kJ}}{\text{kg}} - 237.74 \, \frac{\text{kJ}}{\text{kg}} \right) + \dot{Q}_K = 0
\]

Solving for \( \dot{Q}_K \):
\[
\dot{Q}_K = 119.8 \, \text{W}
\]

Thus:
\[
\epsilon_K = \frac{119.8 \, \text{W}}{-28 \, \text{W}} = 4.27
\]

---

TASK 4e

The temperature at state 1 is given as \( T_1 = -16^\circ\text{C} \). At state 1, the enthalpy is calculated as:
\[
h_1 = h_f(-16^\circ\text{C}) + x_1(h_g(-16^\circ\text{C}) - h_f(-16^\circ\text{C})
\]

Substituting values:
\[
h_1 = 93.42 \, \frac{\text{kJ}}{\text{kg}} + x_1(237.79 \, \frac{\text{kJ}}{\text{kg}} - 93.42 \, \frac{\text{kJ}}{\text{kg}})
\]

Solving for \( x_1 \):
\[
x_1 = 0.31
\]

Finally, the temperature remains constant:
\[
T_1 = T_2
\]