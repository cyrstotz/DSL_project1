TASK 4a  
A pressure-temperature (p-T) diagram is drawn to represent the freeze-drying process. The diagram includes the following features:  
- The x-axis is labeled as temperature \( T \) (°C), and the y-axis is labeled as pressure \( p \).  
- The diagram shows phase regions for solid, liquid, and gas.  
- A line labeled "Step i" indicates the isobaric evaporation process.  
- A point labeled \( T_i \) marks the temperature during the freeze-drying process.  
- A shaded region near the triple point is labeled "water can sublimate."  
- The triple point is clearly marked on the diagram.  

TASK 4b  
The mass flow rate of the refrigerant \( \dot{m}_{\text{R134a}} \) is calculated using the following equation:  
\[
\dot{W}_K = 28 \, \text{W} \quad \text{and the compressor operates at 85 Hz.}
\]  
At state 4, the refrigerant is fully condensed, and the pressure is \( p = 8 \, \text{bar} \).  

The energy balance for the process from state 1 to state 2 includes:  
\[
\dot{Q}_{\text{K}} = \dot{m}_{\text{R134a}} \cdot (h_2 - h_1)
\]  
where \( h_3 = h(8 \, \text{bar}) \) and \( h_4 = h(\text{saturated liquid, } 8 \, \text{bar}) \).  

The pressure difference \( p_3 - p_1 \) is noted, as it influences the refrigerant flow.  

Additional crossed-out equations and notes are ignored.