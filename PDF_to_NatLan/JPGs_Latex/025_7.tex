TASK 3a  
The gas pressure \( p_{g,1} \) is calculated using the ideal gas law:  
\[
p_{g,1} V_{g,1} = m_g R T_{g,1}
\]  
The gas constant \( R \) is given as:  
\[
R = 0.166788 \, \text{J/(kg·K)}
\]  
The mass \( m_g \) is determined using the formula:  
\[
m_g = \frac{p_{g,1} V_{g,1}}{R T_{g,1}}
\]  

TASK 3b  
The pressure \( p_{g,2} \) is calculated as:  
\[
p_{g,2} = p_{\text{atm}} + m_K g
\]  
where \( p_{\text{atm}} \) is the atmospheric pressure, \( m_K \) is the mass of the piston, and \( g \) is the acceleration due to gravity.  

The pressure exerted by the piston is calculated as:  
\[
p_{\text{piston}} = \frac{(0.5 \cdot 10^{-2} \, \text{m})^2 \pi}{7.150 \cdot 10^{-3} \, \text{m}^2}
\]  
This results in:  
\[
p_{\text{piston}} = 7.150 \cdot 10^{-3} \, \text{N}
\]  

The total pressure is then:  
\[
p_{g,2} = (0.1 \, \text{bar}) + (0.1 \, \text{bar}) = 0.1 \, \text{bar}
\]  

Crossed-out content is ignored.