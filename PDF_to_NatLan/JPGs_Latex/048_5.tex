TASK 4a  
A graph is drawn showing the phase regions of a substance on a pressure-temperature (\(p\)-\(T\)) diagram. The axes are labeled as follows:  
- The vertical axis represents pressure (\(p\)) in arbitrary units.  
- The horizontal axis represents temperature (\(T\)) in degrees Celsius (\(^\circ\text{C}\)).  

The graph includes the following features:  
- A line separating the gas phase from the liquid phase.  
- A line separating the liquid phase from the solid (ice) phase.  
- A point labeled "Triple Point" where the gas, liquid, and solid phases coexist.  
- The regions are labeled as "Gas," "Liquid," and "Ice."  

TASK 4b  
The energy balance around the compressor is described. The equations provided are:  
\[
\dot{m}_{\text{R134a}} \cdot (h_2 - h_3) - \dot{W}_K = 0
\]
where:  
- \( \dot{m}_{\text{R134a}} \) is the mass flow rate of the refrigerant.  
- \( h_2 \) and \( h_3 \) are the specific enthalpies at states 2 and 3, respectively.  
- \( \dot{W}_K \) is the work input to the compressor.  

The mass flow rate is expressed as:  
\[
\dot{m}_{\text{R134a}} = \frac{\dot{W}_K}{h_2 - h_3}
\]

Another equation is given for the heat transfer rate:  
\[
\dot{Q}_K = \dot{Q}_{\text{ab}} - \dot{W}_K
\]

Additional notes include:  
- \( p_3 = p_4 = 8 \, \text{bar} \).  
- \( h_4 = h_f \), where \( h_f \) is the enthalpy of the saturated liquid at state 4.  
- \( h_f = 93.42 \, \text{kJ/kg} \).  

No further calculations or explanations are provided.  

TASK 4b (continued)  
The enthalpy at state 2 (\( h_2 \)) is left blank, indicating it is to be determined.  

No additional content or figures are visible.