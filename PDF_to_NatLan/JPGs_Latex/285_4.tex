TASK 3a  
The ideal gas law is used to calculate the pressure and mass of the gas in state 1:  
\[
p_{g,1} V = mRT
\]  
The gas constant \( R \) is calculated as:  
\[
R = \frac{8.314}{M} = \frac{8.314}{50} = 166.28 \, \text{J/(kg·K)}
\]  
The total mass \( m \) is the sum of the piston mass \( m_K \) and the ice-water mixture mass \( m_{\text{EW}} \):  
\[
m = m_K + m_{\text{EW}} = 32 \, \text{kg}
\]  
The pressure \( p_{g,1} \) is determined using the atmospheric pressure \( p_{\text{amb}} \) and the piston force:  
\[
p_{g,1} = p_{\text{amb}} + \frac{32 \, \text{kg} \cdot 9.81 \, \text{m/s}^2}{0.05 \, \text{m}^2} = 100000 \, \text{Pa} + 40870.59 \, \text{Pa} = 140870.59 \, \text{Pa}
\]  
The mass \( m \) of the gas is calculated as:  
\[
m = \frac{p V}{RT} = \frac{140870.59 \cdot 0.00314}{166.28 \cdot 773.15} = 3.041 \, \text{g}
\]  

TASK 3b  
The temperature \( T_{g,2} \) and pressure \( p_{g,2} \) in state 2 are noted. The volume remains constant:  
\[
V_2 = V_1
\]  

TASK 3c  
The temperature \( T_{g,2} \) is given as:  
\[
T_{g,2} = 0.003^\circ \text{C}
\]  
The heat transferred \( Q_{12} \) is calculated as:  
\[
[Q_{12}] = 1000 \, \text{J} = 1 \, \text{kJ}
\]  

TASK 3d  
The final ice fraction \( x_{\text{ice},2} \) is calculated using the ratio of the remaining ice mass \( m_{\text{ice},2} \) to the total mass of the ice-water mixture \( m_{\text{EW}} \):  
\[
x_{\text{ice},2} = \frac{m_{\text{ice},2}}{m_{\text{EW}}}
\]  

No diagrams or figures are present on this page.