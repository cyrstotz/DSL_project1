TASK 2a  
The page contains two diagrams related to thermodynamic processes.  

1. **First Diagram (Top)**:  
   - The graph is labeled "T-s" on the axes, indicating a temperature-entropy (\( T \)-\( s \)) diagram.  
   - The curve appears to represent a qualitative thermodynamic process, with peaks and valleys suggesting changes in temperature and entropy during the process.  
   - No specific numerical values or detailed annotations are provided, but the general shape suggests a cyclic process.  

2. **Second Diagram (Bottom)**:  
   - This is another \( T \)-\( s \) diagram with more detailed annotations.  
   - The diagram shows a series of labeled points (0, 1, 2, 3, 4, 5, 6) connected by curves and lines.  
   - The process transitions between states, with arrows indicating the direction of the thermodynamic process.  
   - The isobars (lines of constant pressure) are drawn as curved lines across the diagram.  
   - At state 5, there is a zigzag line indicating mixing or turbulence, and the pressure is labeled as \( p_5 = p_0 \).  
   - The axes are labeled as \( s \, [\frac{kJ}{kg \cdot K}] \) for entropy and \( T \, [K] \) for temperature.  

No further textual explanation or equations are visible on the page.