TASK 4c  
The equation for \( x_3 \) is given as:  
\[
x_3 = \frac{s_3 - s_f}{s_g - s_f} \quad \text{(at 8 bar)}
\]  
where \( s_3 = s_2 \), and \( s_2 = s_g(T_2) \).  

The text mentions needing \( T_2 \), but questions how to determine it:  
"brauche \( T_2 \), aber wie mache ich das?"  
Translation: "I need \( T_2 \), but how do I determine it?"

---

TASK 4c  
The pressure \( p_1 = p_3 = 8 \, \text{bar} \).  

The process is described as an adiabatic throttling, which is isothermal. Therefore, \( T_3 = T_4 \).  

The vapor quality \( x = 0 \), indicating saturated liquid at 8 bar.  

The inlet temperature is given as \( T_{\text{in}} = 31.33^\circ\text{C} \), and reference is made to "TABA-11" for liquid-vapor properties at 8 bar.

---

TASK 4d  
The coefficient of performance \( \epsilon_K \) is defined as:  
\[
\epsilon_K = \frac{\dot{Q}_{\text{zu}}}{|\dot{W}_K|} = \frac{\dot{Q}_{\text{zu}}}{|\dot{Q}_{\text{abl}} - \dot{Q}_{\text{zu}}|} = \frac{|\dot{Q}_K|}{|\dot{Q}_{\text{abl}} - \dot{Q}_K|}
\]  

Further calculations or numerical results are not provided.