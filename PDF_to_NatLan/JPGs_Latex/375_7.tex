TASK 3c  
The change in internal energy \( \Delta U \) is calculated as the difference between the heat transferred \( Q_{12} \) and the work done \( W \):  
\[
\Delta U = Q - W
\]  
The heat transferred \( Q_{12} \) is expressed as:  
\[
Q_{12} = m_g \cdot c_V \cdot (T_2 - T_1)
\]  
Substituting the given values:  
\[
Q_{12} = 0.000003428 \, \text{kg} \cdot 0.633 \, \frac{\text{kJ}}{\text{kg·K}} \cdot (0.0032 - 500^\circ\text{C})
\]  
This results in:  
\[
Q_{12} = -1.0848 \, \text{kJ}
\]  
Thus, \( 1.0848 \, \text{kJ} \) is released.

---

TASK 3d  
The change in internal energy \( \Delta U \) is equal to the heat transferred \( Q_{12} \). The following equation is used:  
\[
\Delta U = Q_{12}
\]  
Here, the positive value is filled in:  
\[
m_{E2} \cdot U_{\text{rest}} + m_{w2} \cdot U_{FL} - m_{w1} \cdot U_{\text{rest}} - m_{w1} \cdot U_p = 1.0848 \, \text{kJ}
\]  

For the ice fraction calculation:  
\[
x = \frac{m_w}{m_{EW}}
\]  
Given \( m_w = 0.06 \, \text{kg} \), the mass of water is:  
\[
m_w = 0.06 \, \text{kg}
\]  
Thus, \( m_{EW} = 0.06 \, \text{kg} \).