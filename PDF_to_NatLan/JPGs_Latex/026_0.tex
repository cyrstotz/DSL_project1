TASK 3a  
The gas pressure \( p_{g,1} \) is calculated using the equation:  
\[
p_{g,1} = p_{\text{amb}} + \frac{m_c g}{A} + \frac{m_{\text{mem}} g}{A}
\]  
The area \( A \) is determined as:  
\[
A = \pi \frac{D^2}{4} = 7.85 \times 10^{-3} \, \text{m}^2
\]  
Substituting values, the result is:  
\[
p_{g,1} = 3.443 \, \text{bar}
\]  
The ambient pressure is given as:  
\[
p_{\text{amb}} = 1.1 \, \text{bar}
\]  

The gas mass \( m_g \) is calculated using the ideal gas law:  
\[
\frac{p_{g,1} V_{g,1}}{R T_{g,1}} = m_g
\]  
Substituting values, the result is:  
\[
m_g = 3.429 \, \text{g}
\]  
The specific gas constant \( R \) is calculated as:  
\[
R = \frac{\bar{R}}{M_g} = 0.16628 \, \text{kJ/kg·K}
\]  

---

TASK 3b  
The energy balance is expressed as:  
\[
\Delta E = Q - W
\]  
For an isolated system, \( Q = 0 \) and \( W = 0 \), so:  
\[
\Delta E = 0
\]  
The change in internal energy is given by:  
\[
\Delta E = m_g (u_2 - u_1)
\]  
For a perfect gas, this simplifies to:  
\[
\Delta E = m_g c_V (T_2 - T_1)
\]  

Since the process is isochoric, the specific volume remains constant:  
\[
V_{g,1} = V_{g,2}
\]  

The final gas pressure \( p_{g,2} \) is calculated using:  
\[
p_{g,2} = \frac{m_g R T_{g,2}}{V_{g,1}}
\]  

