TASK 2c  
The change in flow exergy is calculated using the following equation:  
\[
\Delta ex_{\text{str}} = \dot{m} \left[ h_6 - h_0 - T_0 (s_6 - s_0) + \frac{w_6^2}{2} \right]
\]  
Substituting values:  
\[
h_6 - h_0 = c_p (T_6 - T_0) + v_i (p_6 - p_0) = 85.434 \, \frac{\text{kJ}}{\text{kg}}
\]  
\[
T_0 = 328.075 \, \text{K}, \quad T_6 = -30 + 273.15 = 243.15 \, \text{K}
\]  
\[
s_6 - s_0 = c_p \ln \left( \frac{T_6}{T_0} \right) = 0.301 \, \frac{\text{kJ}}{\text{kg·K}}
\]  
\[
\frac{w_6^2}{2} = 507.24^2 \, \frac{\text{m}^2}{\text{s}^2} = 128646 \, \text{J}
\]  
Thus:  
\[
\Delta ex_{\text{str}} = 12.158 \, \frac{\text{kJ}}{\text{kg}} + \frac{w_6^2}{2} = 116.487 \, \frac{\text{kJ}}{\text{kg}}
\]  

TASK 2d  
The exergy destruction is calculated as follows:  
\[
ex_{\text{verl}} = ?
\]  
The system is adiabatic, and the equation for exergy destruction is:  
\[
\frac{d ex_{\text{str}}}{dt} = \Delta ex_{\text{str}} - ex_{\text{verl}}
\]  
From the calculations:  
\[
ex_{\text{verl}} = \Delta ex_{\text{str}} = 116.487 \, \frac{\text{kJ}}{\text{kg}}
\]  

A diagram is drawn showing a control volume with arrows indicating inlet and outlet flows. Labels include \( ex_q = 0 \) and \( w_c = 0 \), indicating no heat or work transfer.