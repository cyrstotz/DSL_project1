TASK 3a  
The goal is to determine the gas pressure \( p_{g,1} \) and mass \( m_g \) in state 1.  

The temperature of the gas in state 1 is given as \( T_{g,1} = 500^\circ\text{C} = 773.15 \, \text{K} \).  
The volume of the gas is \( V_{g,1} = 3.14 \, \text{L} = 0.00314 \, \text{m}^3 \).  

The specific gas constant is calculated as:  
\[
R_g = \frac{\bar{R}}{M} = 0.1663 \, \frac{\text{kJ}}{\text{kg·K}}
\]  

The gas equation is used:  
\[
p_{g,1} V_{g,1} = m_g R_g T_{g,1}
\]  

A diagram is drawn showing the forces acting on the piston. The piston area is \( A = 5 \, \text{cm diameter} \), which gives:  
\[
A = 0.00785 \, \text{m}^2
\]  

The pressure balance is written as:  
\[
p_{\text{amb}} A + (m_K + m_{\text{EW}}) g = p_{1} A
\]  
\[
p_{\text{amb}} + \frac{g}{A} (m_K + m_{\text{EW}}) = p_{1} = 1.4 \, \text{bar} = p_{g,1}
\]  

The mass of the gas is calculated using:  
\[
m_g = \frac{p_{g,1} V_{g,1}}{R_g T_{g,1}}
\]  

TASK 3b  
The goal is to determine \( T_{g,2} \) and \( p_{g,2} \).  

The pressure in state 2 is equal to the pressure in state 1:  
\[
p_{g,2} = p_{g,1} = 1.4 \, \text{bar}, \text{ since the weight of the EW does not change.}
\]  

The gas equation is used again:  
\[
p_{g,2} V_{g,2} = m_g R T_{g,2}
\]  

Substituting values:  
\[
p_{g,2} V_{g,2} = 1.4 \, \text{bar} \cdot 3.14 \, \text{L}
\]  
\[
m_g R = 3.42 \, \text{g} \cdot 0.1663 \, \frac{\text{kJ}}{\text{kg·K}}
\]  

The result is simplified to:  
\[
T_{g,2} = 7
\]  

TASK 3c  
The goal is to calculate the transferred heat \( Q_{12} \) from gas to EW between states 1 and 2.  

The energy balance is written as:  
\[
\Delta E = Q_{12} - W_{12}
\]  
\[
\Delta E_{12} = m (u_2 - u_1)
\]  

The internal energy difference is calculated using:  
\[
u_2 - u_1 = c_V (T_2 - T_1)
\]  
\[
c_V = 0.633 \, \frac{\text{kJ}}{\text{kg·K}}
\]  
\[
u_2 - u_1 = 0.633 (0.003 - 500) = -316.488 \, \frac{\text{kJ}}{\text{kg}}
\]  

The work done is calculated as:  
\[
W_{12} = \int p_{1} \, dV = p_{1} (V_2 - V_1)
\]  

The system efficiency around the gas is considered.  

The reversible work is calculated as:  
\[
\omega_{12}^{\text{rev}} = \frac{R (T_2 - T_1)}{1 - n}
\]  
\[
n = \frac{c_p}{c_v} = \frac{R + c_v}{c_v} = 1.263
\]  
\[
\omega_{12}^{\text{rev}} = -112.822 \, \frac{\text{kJ}}{\text{kg}}
\]  

The total work is:  
\[
W_{12}^{\text{rev}} = m \omega_{12}^{\text{rev}} = -0.386 \, \frac{\text{kJ}}{\text{kg}}
\]  