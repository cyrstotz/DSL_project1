TASK 4a  
The diagram is a pressure-temperature (\(p\)-\(T\)) graph illustrating the freeze-drying process.  
- The process labeled "i)" is isobaric and occurs at a constant pressure.  
- The process labeled "ii)" involves a reduction in pressure, moving vertically downward on the graph.  
- The horizontal axis represents temperature (\(T\)) in Kelvin, and the vertical axis represents pressure (\(p\)) in bar.  
- The graph includes arrows indicating the direction of the processes.

---

TASK 4b  
The first law of thermodynamics is applied:  
\[
\dot{m}_{\text{R134a}} (h_2 - h_3) = W
\]
Rearranging to solve for the mass flow rate of the refrigerant:  
\[
\dot{m}_{\text{R134a}} = \frac{W}{h_2 - h_3}
\]

Given:  
- \(T_i = -20^\circ\text{C} = T_2\)  
- \(h_{2f} = 29.26 \, \frac{\text{kJ}}{\text{kg}}\)  
- \(T_3 = T_2 \left(\frac{p_2}{p_3}\right)\)  

With \(T_3\) and \(p_3 = 4 \, \text{bar}\) in the table, \(h_3 = h_{3f}\).