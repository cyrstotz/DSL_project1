TASK 4a  
The diagram is a pressure-temperature (\( P \)-\( T \)) graph illustrating the freeze-drying process.  
- The graph includes labeled regions for solid ("Fest"), liquid ("flüssig"), and gaseous ("gasförmig") phases.  
- The triple point ("Tripel") is marked where all three phases coexist.  
- An isobaric process ("isobar, i") and an isothermal process ("isotherm, ii") are indicated with arrows.  
- The axes are labeled \( P \) (pressure) and \( T \) (temperature).  

TASK 4b  
The following calculations are provided:  
- \( s_2 = s_3 \), \( T_i = -10^\circ\text{C} \), \( T_{\text{ver}} = -16^\circ\text{C} = T_2 \).  
- From the tables:  
  \[
  h_2 = 237.74 \, \frac{\text{kJ}}{\text{kg}}, \quad s_2 = s_3 = 0.9289 \, \frac{\text{kJ}}{\text{kg·K}}
  \]  
- Calculation for \( h_3 \):  
  \[
  h_3 = \frac{(273.66 - 264.75) \, \frac{\text{kJ}}{\text{kg}}}{(0.5374 - 0.9066) \, \frac{\text{kJ}}{\text{kg·K}}} + 264.75 \, \frac{\text{kJ}}{\text{kg}}
  \]  
  \[
  h_3 = 271.31 \, \frac{\text{kJ}}{\text{kg}}
  \]  
- Mass flow rate calculation:  
  \[
  Q = \dot{m}_{\text{R134a}} (h_2 - h_3) + \dot{W}_K \quad \Rightarrow \quad \dot{m}_{\text{R134a}} = \frac{-\dot{W}_K}{h_2 - h_3} = 3.00 \, \frac{\text{kg}}{\text{h}}
  \]