TASK 4a  
The diagram is a pressure-temperature (\(p\)-\(T\)) graph illustrating the freeze-drying process. It includes labeled phase regions for gas, liquid, and solid. The graph shows the following transitions:  
- An isobaric evaporation process at state 2, 6 K below \(T_i\).  
- A reversible adiabatic compression from state 2 to state 3.  
- An isobaric condensation at state 4.  
- An adiabatic expansion from state 4 to state 1.  

The axes are labeled as follows:  
- The \(y\)-axis represents pressure (\(p\)) in bar, ranging from 0.01 to 10 bar.  
- The \(x\)-axis represents temperature (\(T\)) in degrees Celsius, ranging from approximately \(-50^\circ\text{C}\) to \(10^\circ\text{C}\).  

Key points are marked on the graph:  
- State 2 is labeled as "saturated cooling vapor."  
- State 3 is labeled at \(p_3 = 8 \, \text{bar}\).  
- State 4 is labeled as "fully condensed refrigerant."  
- State 1 is labeled after adiabatic expansion.  

TASK 4b  
The refrigerant mass flow rate (\(\dot{m}_{\text{R134a}}\)) is calculated using the enthalpy differences between states.  

At state 2:  
The refrigerant is saturated cooling vapor, and the enthalpy \(h_2\) is determined.  

At state 3:  
The pressure is \(p_3 = 8 \, \text{bar}\), and the enthalpy \(h_3\) is calculated using the saturated vapor enthalpy from Table A-11:  
\[
h_3 = 267.15 \, \text{kJ/kg}.
\]  

The work input to the compressor (\(\dot{W}_K\)) is given by:  
\[
\dot{m} \cdot (h_2 - h_3) = \dot{W}_K.
\]  

At state 4:  
The refrigerant is fully condensed, and the enthalpy \(h_4\) is calculated as:  
\[
h_4 = 93.42 \, \text{kJ/kg}.
\]  

The heat removed by the condenser (\(\dot{Q}_{\text{ab}}\)) is given by:  
\[
\dot{m} \cdot (h_3 - h_4) = \dot{Q}_{\text{ab}}.
\]