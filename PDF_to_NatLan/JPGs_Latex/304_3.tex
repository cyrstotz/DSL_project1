TASK 2c  
The mass-specific increase in flow exergy is calculated using the formula:  
\[
ex_{\text{flow}} = h_6 - h_0 - T_0 \cdot (s_6 - s_0) + \frac{w_6^2}{2} - \frac{w_0^2}{2}
\]  
Breaking this down:  
\[
\Delta ex_{\text{flow}} = h_6 - h_0 - T_0 \cdot (s_6 - s_0) + \frac{w_6^2}{2} - \frac{w_0^2}{2} + c_p \cdot (T_6 - T_0) - c_p \cdot \ln\left(\frac{T_6}{T_0}\right) - R \cdot \ln\left(\frac{p_6}{p_0}\right) + \frac{w_6^2}{2} - \frac{w_0^2}{2}
\]  
The numerical calculation yields:  
\[
\Delta ex_{\text{flow}} = 108,745 \, \frac{\text{J}}{\text{kg}}
\]  
\[
\Delta ex_{\text{flow}} = 108,745 \, \frac{\text{kJ}}{\text{kg}}
\]  

---

TASK 2d  
For a stationary flow process over the entire turbine, the exergy balance is expressed as:  
\[
0 = -\Delta ex_{\text{flow}} + \sum \dot{E}_{\text{in}} - \sum \dot{W}_{\text{in}} - \dot{E}_{\text{out}}
\]  
Simplifying:  
\[
ex_{\text{out}} = -\Delta ex_{\text{flow}} + \text{eva} - 0
\]  

The exergy of the heat stream \( q_B \) is given by:  
\[
\dot{E}_{\text{ex},q} = \left(1 - \frac{T_0}{\bar{T}_B}\right) \cdot \dot{Q}_B
\]  
Where:  
\[
\dot{Q}_B = q_B \cdot \dot{m}_K
\]  

Substituting values:  
\[
\dot{E}_{\text{ex},q} = \left(1 - \frac{T_0}{\bar{T}_B}\right) \cdot q_B \cdot \dot{m}_K
\]  
\[
= \left(1 - \frac{30 + 273.15 \, \text{K}}{1289 \, \text{K}}\right) \cdot 1195 \, \frac{\text{kJ}}{\text{kg}} \cdot \dot{m}_K
\]  
\[
= 869.58 \, \frac{\text{kJ}}{\text{kg}}
\]  

Finally, the exergy output is calculated as:  
\[
ex_{\text{out}} = -108,745 + 869.58
\]  
\[
ex_{\text{out}} = 860,835 \, \frac{\text{J}}{\text{kg}}
\]