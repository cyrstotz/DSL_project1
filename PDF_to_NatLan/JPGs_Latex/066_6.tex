TASK 4a  
The table lists the states of the freeze-drying process with the following parameters: temperature \( T \), pressure \( p \), heat \( Q \), work \( W \), enthalpy \( h \), entropy \( s \), and vapor quality \( x \).  

State descriptions:  
1. \( T_i - 6 \, \text{K} \), \( p_1 \), \( Q_K \), \( x_1 = 0.934 \) (initial vapor quality).  
2. \( -9 \, \text{K} \), \( p_1 \), \( Q = 0 \), \( h = 249.53 \), \( s = 0.9267 \), \( x_2 = 1 \) (complete evaporation).  
3. \( 8 \, \text{bar} \), \( Q = 0 \), \( x_3 = 0 \) (fully condensed refrigerant).  
4. \( 8 \, \text{bar} \), \( Q = 0 \), \( x_4 = 0 \) (final state).  

For the adiabatic expansion (throttle), the enthalpy remains constant: \( h_1 = h_4 \).  

Step ii:  
The temperature \( T_i \) remains constant. The pressure \( p_i \) is reduced to \( 1 \, \text{mbar} \), \( -6 \, \text{mbar} \), and \( -5 \, \text{mbar} \) below the triple point. Sublimation occurs.  

The temperature \( T \) is calculated as:  
\[
T = 10 \, \text{K} \, \text{above sublimation point}, \quad T_i = 10 \, \text{K} + 273.15 \, \text{K} = 283.15 \, \text{K}.
\]

---

TASK 4b  
Given: \( m_{\text{R134a}} \)  

The energy balance from state 1 to state 3 is written as:  
\[
\Delta E = \sum \dot{m} \left( h + \frac{w^2}{2} + gz \right)_{\text{in}} - \sum \dot{m} \left( h + \frac{w^2}{2} + gz \right)_{\text{out}} - \sum \dot{W}.
\]  

For the refrigerant:  
\[
\dot{m} \left( h_1 - h_3 \right) - W_K = 0, \quad W_K = 0 \, \text{(work supplied)}.
\]  

From Table A-11:  
\[
h_1 (8 \, \text{bar}, x_1 = 0.934) = 93.48,  
h_g (8 \, \text{bar}) = 264.15,  
s_g (8 \, \text{bar}) = 0.3459,  
s_fg (8 \, \text{bar}) = 0.9006.
\]  

From Table A-10:  
\[
h_4 = h_1 (8 \, \text{bar}, x_1 = 0.934) = 93.48.
\]  

The temperature \( T_i - Q_K \) is calculated as:  
\[
T_i - Q_K = 273.15 \, \text{K}.
\]  

From Table A-10, the enthalpy difference is:  
\[
h_g (4^\circ \text{C}) = 249.53, \quad s_g (4^\circ \text{C}) = 0.9267.
\]  

The vapor quality \( x \) is calculated as:  
\[
x = \frac{s_g - s_f}{s_fg} = \frac{0.9267 - 0.3459}{0.9006} > 1 \quad \text{(indicates saturated state)}.
\]