TASK 2a  
The page begins with a diagram representing the jet engine process flow. The states are labeled sequentially from 1 to 6, with arrows indicating the transitions between states. The following values are noted:  
- \( T = -30^\circ\text{C} \)  
- \( p = 0.191 \, \text{bar} \)  

Below this, a detailed \( T \)-\( s \) diagram is drawn. The axes are labeled as \( T \) (temperature) on the vertical axis and \( s \) (entropy) on the horizontal axis. The diagram includes the following features:  
- States 1 through 6 are marked.  
- Isobaric and isentropic processes are labeled in red and black, respectively.  
- Pressure values are noted at specific points: \( p_2 = 0.5 \, \text{bar} \), \( p_6 = 0.151 \, \text{bar} \).  
- The transitions between states are shown with curved lines, indicating thermodynamic processes.  

TASK 2b  
The temperature at state 6 (\( T_6 \)) is calculated using the formula:  
\[
T_6 = T_5 \left( \frac{p_6}{p_5} \right)^{\frac{\kappa - 1}{\kappa}}
\]  
Substituting the values:  
\[
T_6 = 431.3 \, \text{K} \left( \frac{0.151}{0.5} \right)^{\frac{1.4 - 1}{1.4}} = 328.67 \, \text{K}
\]  

TASK 2c  
The first law of thermodynamics is applied to the transition from state 5 to state 6:  
\[
0 = h_5 - h_6 + \frac{w_5^2 - w_6^2}{2}
\]  
Rearranging and substituting:  
\[
w_6^2 = w_5^2 + 2 c_{p,\text{air}} (T_5 - T_6)
\]  
Substituting the known values:  
\[
w_6 = \sqrt{507.24 \, \text{m}^2/\text{s}^2} = 257.30 \, \text{m/s}
\]  

The final velocity at state 6 is calculated as \( w_6 = 257.30 \, \text{m/s} \).