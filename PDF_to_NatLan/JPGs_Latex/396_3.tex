TASK 3a  
The gas constant \( R \) is calculated as:  
\[
R = \frac{\bar{R}}{M} = 0.166289 \, \frac{\text{J}}{\text{kg·K}}
\]

The force equilibrium is described, with a diagram showing the forces acting on the piston. The forces include the atmospheric pressure \( p_{\text{amb}} \cdot A \), the weight of the piston \( m_K \cdot g \), and the combined weight of the ice-water mixture \( (m_{\text{EW}} + m_{\text{Eis}}) \cdot g \). These are balanced by the gas pressure \( p_{g,1} \cdot A \).  

The area \( A \) of the piston is calculated as:  
\[
A = \pi \frac{D^2}{4} = 0.007853982 \, \text{m}^2
\]

The mass of ice is determined using the ice fraction \( x_{\text{Eis}} \):  
\[
m_{\text{Eis}} = x_{\text{Eis}} \cdot m_{\text{EW}} = 0.06 \, \text{kg}
\]

The gas pressure \( p_{g,1} \) is calculated using the force equilibrium:  
\[
p_{g,1} = \frac{(m_K + m_{\text{EW}} + m_{\text{Eis}}) g}{A} + p_{\text{amb}} = 1.49155 \, \text{bar}
\]

The gas mass \( m_g \) is calculated using the ideal gas law:  
\[
m_g = \frac{p_{g,1} \cdot V_{g,1}}{R \cdot T_{g,1}} = 3.42304 \, \text{g}
\]

---

TASK 3b  
The second state is described with no work being performed. The mass of the gas remains constant.  

The energy balance is written as:  
\[
m_g \cdot u_2 = m_g \cdot u_1
\]

This simplifies to:  
\[
u_2 = u_1
\]

Further simplification leads to:  
\[
c_V \cdot (T_2 - T_1) = 0
\]

Thus, the temperature difference between states 2 and 1 is zero:  
\[
T_2 = T_1
\]  

Some crossed-out text is ignored.  

---

TASK 3c  
No content is provided for this subtask.  

---  
