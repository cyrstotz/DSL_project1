TASK 4a  
Two diagrams are drawn to represent thermodynamic processes:

1. **First Diagram**:  
   - The graph is labeled with \( P \, [\text{Pa}] \) on the vertical axis and \( T \, [\text{K}] \) on the horizontal axis.  
   - The diagram shows intersecting curves. One curve represents the phase boundary (likely sublimation, melting, or evaporation lines), and another curve represents a linear relationship.  
   - A point labeled "Trip" is marked near the intersection of the curves, indicating the triple point of the substance.  
   - The curves are drawn in blue and purple, with the phase boundary curve slightly curved and the other curve linear.

2. **Second Diagram**:  
   - The graph is labeled with \( P \, [\text{Pa}] \) on the vertical axis and \( T \, [\text{K}] \) on the horizontal axis.  
   - A cycle is depicted with four numbered points (1, 2, 3, 4) connected by arrows to indicate the direction of the process.  
   - The process includes an isobaric segment (constant pressure) labeled between points 4 and 1, and another isobaric segment between points 2 and 3.  
   - The cycle appears to represent a thermodynamic refrigeration or freeze-drying process.  
   - The curves are drawn in blue, with annotations in red for the isobaric processes.  

No additional textual explanation is provided on the page.