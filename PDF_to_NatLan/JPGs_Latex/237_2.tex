TASK 2a  
A graph is drawn representing the process in a temperature-entropy (\(T\)-\(S\)) diagram. The axes are labeled as follows:  
- The vertical axis is labeled \(T\) (K), representing temperature in Kelvin.  
- The horizontal axis is labeled \(S\) (\(kJ/\text{kg·K}\)), representing entropy.  

The diagram shows a cycle with numbered states: 1, 2, 3, 4, 5, and 6.  
- State 1 transitions to state 2 with an upward curve.  
- State 2 transitions to state 3 with a steep upward line.  
- State 3 transitions to state 4 with a downward curve.  
- State 4 transitions to state 5 with a zigzag pattern, indicating mixing or turbulence.  
- State 5 transitions to state 6 with a downward curve.  

TASK 2b  
The temperature at state 6 (\(T_6\)) is calculated using the isentropic relation:  
\[
T_6 = T_5 \left( \frac{p_6}{p_5} \right)^{\frac{n-1}{n}}
\]  
Substituting values:  
\[
T_6 = T_5 \left( \frac{p_6}{p_5} \right)^{\frac{n-1}{n}}
\]  
\[
T_6 = 328.07 \, \text{K}
\]  
Where \(n = 1.4\).  

Assumptions:  
\[
\dot{Q} = 0, \quad \dot{W} = 0
\]  

The energy balance equation is written as:  
\[
0 = \dot{m} \left( h_5 - h_6 + \frac{w_5^2 - w_6^2}{2} \right)
\]  

The work done (\(W_{56}\)) is expressed as:  
\[
W_{56} = -\int p \, dV
\]  
In the limit:  
\[
W_{56} = R \frac{T_5 - T_6}{n-1}
\]  
Substituting values:  
\[
W_{56} = -74.6 \, \text{kJ}
\]  

The gas constant \(R\) is calculated as:  
\[
R = c_p - c_v
\]  
\[
R = c_p \left( 1 - \frac{1}{k} \right)
\]  
Substituting \(c_p = 1.006 \, \text{kJ/kg·K}\) and \(k = 1.4\):  
\[
R = 0.2874 \, \text{kJ/kg·K}
\]  

The velocity at state 6 (\(w_6\)) is calculated using:  
\[
w_6 = \sqrt{2 \left( h_5 - h_6 - W_{56} \right)}
\]  
Substituting values:  
\[
w_6 = \sqrt{2 \left( c_p \left( T_5 - T_6 \right) - W_{56} + \frac{w_5^2}{2} \right)}
\]  
\[
w_6 = 220 \, \text{m/s}
\]