TASK 3a  
The mass of the gas is calculated using the ideal gas law:  
\[
m = \frac{p \cdot V}{R \cdot T}
\]  
The gas constant \( R \) is derived as:  
\[
R = \frac{\bar{R}}{M}
\]  
where \( \bar{R} = 8.314 \, \text{J/(mol·K)} \) and \( M = 50 \, \text{kg/kmol} \). Substituting values:  
\[
R = \frac{8.314}{50} = 0.166 \, \text{J/(kg·K)}
\]  
The pressure is expressed as:  
\[
p = \frac{R \cdot T}{V}
\]  

---

TASK 3b  
The initial gas pressure and mass are given as:  
\[
p_{g,1} = 7.35 \, \text{bar}, \quad m_g = 3.6 \, \text{kg}
\]  
The temperature of the gas is stated as:  
\[
T_{g,2} = 0^\circ\text{C}
\]  
The relationship between the temperatures and pressures is given by:  
\[
\frac{T_2}{T_1} = \left( \frac{p_2}{p_1} \right)^{\frac{n-1}{n}}
\]  
where \( n \) is calculated as:  
\[
n = \frac{c_p}{c_v}
\]  
Using \( c_p = 1.006 \, \text{kJ/(kg·K)} \) and \( c_v = 0.633 \, \text{kJ/(kg·K)} \):  
\[
n = \frac{1.006}{0.633} = 1.00025
\]  
The equation for \( p_2 \) is derived as:  
\[
p_2 = \frac{T_2}{T_1} \cdot p_1
\]  

The explanation states: "The temperature is equal to that of the ice because the ice temperature cannot change when \( x_2 > 0 \)."  

---

TASK 3c  
The energy balance is expressed as:  
\[
Q_{12} = m \cdot c_v \cdot (T_2 - T_1)
\]  
Substituting values:  
\[
T_2 = 0.003^\circ\text{C} = 273.153 \, \text{K}, \quad T_1 = 500^\circ\text{C} = 773.15 \, \text{K}
\]  
\[
Q_{12} = 3.6 \cdot 0.633 \cdot (273.153 - 773.15) = -1.146 \, \text{kJ}
\]  
The heat transferred is calculated as:  
\[
Q_{12} = 1140 \, \text{J}
\]  

---

No diagrams or figures are present on the page.