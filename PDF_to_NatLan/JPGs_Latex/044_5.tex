TASK 3a  
The pressure of the gas in state 1 is calculated as:  
\[
p_1 = p_{\text{amb}} + \frac{32 \, \text{kg} \cdot g}{\frac{D^2 \cdot \pi}{4}} + \frac{0.1 \, \text{kg} \cdot g}{\frac{D^2 \cdot \pi}{4}}
\]  
Substituting the values:  
\[
p_1 = 1.40059 \, \text{bar}
\]  

The mass of the gas \( m_g \) is determined using the ideal gas law:  
\[
m_g = \frac{p_1 \cdot V_1}{R \cdot T_1}
\]  
where \( R = \frac{\bar{R}}{M_g} = 166.28 \, \frac{\text{J}}{\text{kg·K}} \).  

Substituting the values:  
\[
m_g = \frac{1.40059 \cdot 3.14 \cdot 10^{-3}}{166.28 \cdot (500 + 273.15)}
\]  
\[
m_g = 3.634 \, \text{g}
\]  

---

TASK 3b  
The temperature \( T_{\text{EW},2} \) is stated to be \( 0^\circ\text{C} \) because \( x_{\text{ice},2} > 0 \), meaning there is still ice present. No heat exchange occurs anymore, and the system is in thermal equilibrium.  

The pressure in state 2 is calculated as:  
\[
p_{g,2} = p_1 = 1.4 \, \text{bar}
\]  

---

TASK 3c  
The volume of the gas in state 2 is calculated using the ideal gas law:  
\[
V_{g,2} = \frac{R \cdot m_g \cdot T_{g,2}}{p_{g,2}}
\]  
Substituting the values:  
\[
V_{g,2} = \frac{166.28 \cdot 3.634 \cdot 273.15}{1.4 \cdot 10^5}
\]  
\[
V_{g,2} = 1.05 \, \text{L}
\]  

The energy balance for the system is written as:  
\[
E_2 - E_1 = Q_{12} - W_v
\]  

The heat transferred \( Q_{12} \) is calculated as:  
\[
Q_{12} = m_g \cdot c_p \cdot (T_2 - T_1)
\]  
Substituting the values:  
\[
Q_{12} = 3.634 \cdot 0.7 \cdot (273.15 - 773.15)
\]  
\[
Q_{12} = 1.4337 \, \text{kJ}
\]  

The specific heat capacity \( c_p \) is given as:  
\[
c_p = R + c_v = 0.16628 + 0.633 = 0.79928 \, \frac{\text{kJ}}{\text{kg·K}}
\]  

---

No diagrams or figures are present on this page.