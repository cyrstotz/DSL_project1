TASK 2a  
The diagram is a qualitative representation of the jet engine process in a \( T \)-\( s \) diagram. It includes labeled isobars and process paths:  
- State 0 represents ambient conditions at \( T_0 = -30^\circ\text{C} \).  
- State 1 shows the adiabatic reversible compression.  
- State 3 represents the isobaric combustion process.  
- State 5 shows the adiabatic reversible mixing.  
- State 6 represents the nozzle exit.  
The isobars are labeled as \( 0.191 \, \text{bar} \) and \( 0.5 \, \text{bar} \). The paths are marked as adiabatic reversible and isobaric processes.  

TASK 2b  
The process involves adiabatic reversible sublimation, where \( s_5 = s_6 \). For an ideal gas, the polytropic equation is applied:  
\[
\frac{T_6}{T_5} = \left( \frac{p_6}{p_5} \right)^{\frac{n-1}{n}}
\]  
Substituting values:  
\[
T_6 = T_5 \left( \frac{p_6}{p_5} \right)^{\frac{n-1}{n}}
\]  
Given \( T_5 = 431.9 \, \text{K} \), \( p_6 = 0.191 \, \text{bar} \), \( p_5 = 0.5 \, \text{bar} \), and \( n = 1.4 \):  
\[
T_6 = 328.07 \, \text{K}
\]  

The energy balance for the nozzle is:  
\[
0 = \dot{m}_5 \left[ \frac{w_6^2 - w_5^2}{2} + h_5 - h_6 \right] - \dot{W}_{\text{nozzle}}
\]  

TASK 2c  
The mass-specific increase in flow exergy is calculated as:  
\[
\Delta ex_{\text{flow}} = (h_0 - h_6) - T_0 (s_0 - s_6)
\]  
Breaking it into components:  
\[
\Delta ex_{\text{flow}} = (h_0 - h_6) - T_0 (s_0 - s_6)
\]  

TASK 2d  
The exergy destruction is expressed as:  
\[
\dot{e}_{\text{dest}} = \Delta ex_{\text{flow}} + ex_{\text{q}} - w_p - ex_{\text{val}}
\]  
Rearranging:  
\[
ex_{\text{val}} = \Delta ex_{\text{flow}} + ex_{\text{q}} - w_p
\]