TASK 4a  
The page contains two diagrams:  

1. The first diagram is a pressure-temperature (\(p\)-\(T\)) graph. It shows a curve representing the phase regions of a substance. The curve peaks at a critical point labeled "crit." The x-axis is labeled \(T\) in Kelvin (\(T \, \text{in K}\)), and the y-axis is labeled \(p\) in bar (\(p \, \text{in bar}\)).  

2. The second diagram is a triangular cycle diagram, also plotted on a pressure-temperature (\(p\)-\(T\)) graph. The cycle is labeled with four states: \(1\), \(2\), \(3\), and \(4\). The x-axis is labeled \(T\) in Kelvin (\(T \, \text{in K}\)), and the y-axis is labeled \(p\) in bar (\(p \, \text{in bar}\)).  

TASK 4b  
The text begins with a description of an adiabatic process between states \(2\) and \(3\).  

The energy balance equation is written as:  
\[
Q = m \cdot (h_e - h_a) + \frac{v^2}{2} + g \cdot z + \dot{Q} - W
\]  
where \(Q\) is the heat transfer, \(m\) is the mass, \(h_e\) and \(h_a\) are specific enthalpies, \(v\) is velocity, \(g\) is gravitational acceleration, \(z\) is height, \(\dot{Q}\) is heat flow rate, and \(W\) is work.  

The work equation is simplified as:  
\[
W = m \cdot (h_e - h_a) = m \cdot (h_2 - h_3)
\]  

The mass flow rate is expressed as:  
\[
m = \frac{W}{h_2 - h_3}
\]  

The enthalpies \(h_2\) and \(h_3\) are left undefined, with placeholders:  
\[
h_2 = \quad h_3 =
\]  

No further numerical values or explanations are provided.