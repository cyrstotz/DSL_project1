TASK 4a  
A pressure-temperature (\( p \)-\( T \)) diagram is drawn to represent the freeze-drying process. The diagram includes the following labeled processes:  
- **Isobaric** processes at states 1 and 4.  
- **Isentropic** processes at states 2 and 3.  
- **Adiabatic** processes connecting states 1 to 2 and 3 to 4.  

The diagram shows transitions between states 1, 2, 3, and 4, with arrows indicating the direction of the processes.

---

TASK 4b  
The mass flow rate of the refrigerant (\( \dot{m}_{\text{R134a}} \)) is calculated using the energy balance for a stationary flow process:  
\[
\dot{Q}_K = \dot{m} (h_2 - h_3)
\]  
where \( \dot{Q}_K \) is the heat removed, \( h_2 \) and \( h_3 \) are the specific enthalpies at states 2 and 3, respectively.

Given conditions:  
- \( p_1 = p_2 \), \( p_3 = 8 \, \text{bar} \).  
- \( s_2 = s_3 \).  
- \( T_2 = 4^\circ\text{C} \), \( T_i = 10^\circ\text{C} \).  

From the tables:  
\[
h_2 = h_g(4^\circ\text{C}) \, \text{from Table A-10} = 249.53 \, \text{kJ/kg}
\]  
\[
h_3 = h_g(8 \, \text{bar}) \, \text{from Table A-11} = 204.15 \, \text{kJ/kg}
\]

Substituting into the formula:  
\[
\dot{m} = \frac{\dot{Q}_K}{h_2 - h_3} = \frac{28}{249.53 - 204.15} \, \text{kJ/kg}
\]

Calculation:  
\[
\dot{m} = 0.0006151 \, \text{kg/s} = 1.815 \, \text{g/s}
\]  

The refrigerant mass flow rate is \( \dot{m} = 1.815 \, \text{g/s} \).