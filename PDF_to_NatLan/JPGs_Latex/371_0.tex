TASK 1a  
To calculate the heat flow \( \dot{Q}_{\text{out}} \), the first law of thermodynamics for steady-state flow processes is applied:  
\[
\dot{m} \cdot (h_{\text{in}} - h_{\text{out}}) + \dot{Q}_R = \dot{Q}_{\text{out}}
\]  
Rearranging for \( \dot{Q}_{\text{out}} \):  
\[
\dot{Q}_{\text{out}} = \dot{Q}_R + \dot{m} \cdot (h_{\text{in}} - h_{\text{out}})
\]  

The enthalpy values are determined using water tables (Table A-2):  
\[
h_{\text{in}} = h_f(T_{\text{in}} = 70^\circ\text{C}) = 2333.8 \, \frac{\text{kJ}}{\text{kg}}
\]  
\[
h_{\text{out}} = h_f(T_{\text{out}} = 100^\circ\text{C}) = 2257.0 \, \frac{\text{kJ}}{\text{kg}}
\]  

Substituting the values:  
\[
\dot{Q}_{\text{out}} = 100 \, \text{kW} + 0.3 \, \frac{\text{kg}}{\text{s}} \cdot \left( 2333.8 \, \frac{\text{kJ}}{\text{kg}} - 2257.0 \, \frac{\text{kJ}}{\text{kg}} \right)
\]  
\[
\dot{Q}_{\text{out}} = 123.24 \, \text{kW}
\]  

TASK 1b  
The cooling jacket operates with \( \dot{Q}_{\text{out}} = 65 \, \text{kW} \).