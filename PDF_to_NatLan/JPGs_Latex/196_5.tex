TASK 3a  
The gas pressure \( p_{g,1} \) and mass \( m_g \) are calculated as follows:  

The diameter of the membrane is \( D = 10 \, \text{cm} = 0.1 \, \text{m} \), and the area is:  
\[
A = \frac{\pi}{4} D^2 = 0.00785 \, \text{m}^2
\]  

The pressure at the membrane is calculated using the equilibrium of forces:  
\[
p_{\text{atm}} \cdot A + m_K \cdot g + m_{\text{EW}} \cdot g = p_{\text{gas}} \cdot A
\]  
Rearranging for \( p_{\text{gas}} \):  
\[
p_{\text{gas}} = p_{\text{atm}} + \frac{(m_K + m_{\text{EW}}) \cdot g}{A}
\]  
Substituting values:  
\[
p_{\text{gas}} = 1.5 \, \text{bar} + \frac{32.1 \, \text{kg} \cdot 9.81 \, \text{m/s}^2}{0.00785 \, \text{m}^2}
\]  
\[
p_{\text{gas}} = 1.5 \, \text{bar} + 40,059.44076 \, \text{Pa}
\]  
\[
p_{\text{gas}} = 15 \, \text{bar} + 0.401 \, \text{bar} = 1.901 \, \text{bar}
\]  

The mass of the gas \( m_g \) is calculated using the ideal gas law:  
\[
p \cdot V = m \cdot R \cdot T
\]  
Rearranging for \( m_g \):  
\[
m_g = \frac{p \cdot V}{R \cdot T}
\]  
Substituting values:  
\[
V = 3.14 \, \text{L} = 0.00314 \, \text{m}^3, \quad T = 500^\circ\text{C} = 773.15 \, \text{K}
\]  
\[
p_{\text{gas}} = 140 \, \text{kPa} = 140,059 \, \text{Pa}, \quad R = \frac{R_{\text{univ}}}{M} = \frac{8.314 \, \text{kJ/kmol·K}}{50 \, \text{kg/kmol}} = 0.16628 \, \text{kJ/kg·K}
\]  
\[
m_g = \frac{140,059 \, \text{Pa} \cdot 0.00314 \, \text{m}^3}{0.16628 \, \text{kJ/kg·K} \cdot 773.15 \, \text{K}}
\]  
\[
m_g = 0.00392 \, \text{kg} = 3.922 \, \text{g}
\]  

TASK 3b  
The ice fraction \( x_{\text{ice},2} > 0 \), meaning the water remains in the solid-liquid equilibrium region.  

The temperature of the gas will be \( 0^\circ\text{C} \). The gas transfers heat to the water as long as the temperature of the gas is higher than the temperature of the water.  

The pressure of the gas will remain the same as in state 1. It must counteract both the gravitational forces and the atmospheric pressure.