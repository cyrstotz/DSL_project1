TASK 3a  
Simplified setup:  
A sketch is shown with a cylinder divided into two chambers. The bottom chamber contains gas, and the top chamber contains an ice-water mixture. A piston rests on the ice-water mixture, exerting pressure. The sketch includes labels for the piston mass (\(m_K\)), ice-water mass (\(m_{\text{EW}}\)), and gravitational acceleration (\(g\)).

The cross-sectional area of the cylinder is calculated as:  
\[
A_{Zyl} = A = \pi \cdot D^2 / 4 = \pi \cdot (25 \, \text{cm})^2 / 4 = \frac{\pi}{400} \, \text{m}^2
\]

Force equilibrium is expressed as:  
\[
p_{g,1} \cdot A = (m_K + m_{\text{EW}}) \cdot g + p_{\text{amb}} \cdot A
\]

Rearranging for \(p_{g,1}\):  
\[
p_{g,1} = \frac{g \cdot (m_K + m_{\text{EW}})}{A} + p_{\text{amb}}
\]

Substituting values:  
\[
p_{g,1} = \frac{9.81 \, \text{m/s}^2 \cdot (32 \, \text{kg})}{\frac{\pi}{400} \, \text{m}^2} + 10^5 \, \text{Pa}
\]

Result:  
\[
p_{g,1} = 1.4008 \, \text{bar}
\]

The gas constant \(R\) is calculated as:  
\[
R = \frac{R_u}{M_g} = \frac{8.314 \, \text{J/(mol·K)}}{50 \, \text{kg/kmol}} = 0.16628 \, \text{kJ/(kg·K)}
\]

The initial temperature is converted:  
\[
T_{g,1} = 500^\circ\text{C} + 273.15 \, \text{K} = 773.15 \, \text{K}
\]

The gas mass \(m_g\) is calculated using the ideal gas law:  
\[
p_{g,1} \cdot V_1 = m_g \cdot R \cdot T_1
\]

Rearranging for \(m_g\):  
\[
m_g = \frac{p_{g,1} \cdot V_1}{R \cdot T_1} = \frac{140090 \, \text{Pa} \cdot 3.14 \cdot 10^{-3} \, \text{m}^3}{0.16628 \, \text{kJ/(kg·K)} \cdot 773.15 \, \text{K}}
\]

Result:  
\[
m_g = 3.422 \, \text{g}
\]

---

TASK 3b  
The pressure \(p_{g,2}\) equals \(p_{g,1}\) because the piston mass and ambient pressure remain unchanged.  

Since ice is still present in state 2 (\(x_{\text{ice},2} > 0\)), thermodynamic equilibrium ensures the temperature of the gas matches that of the ice-water mixture, which is \(0^\circ\text{C}\) or \(273.15 \, \text{K}\).

---

TASK 3c  
For the closed system (\(E_{pe} = 0\)):  
\[
E_2 - E_1 = Q_{12} - W_{v,12}
\]

Internal energy change:  
\[
U_2 - U_1 = Q_{12} - W_{v,12}
\]

Volume \(V_2\) is calculated using the ideal gas law:  
\[
V_2 = \frac{m_g \cdot R \cdot T_2}{p_{g,2}} = \frac{3.422 \, \text{g} \cdot 0.16628 \, \text{kJ/(kg·K)} \cdot 273.15 \, \text{K}}{140090 \, \text{Pa}}
\]

Result:  
\[
V_2 = 1.11 \, \text{L}
\]