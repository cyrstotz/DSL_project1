TASK 4a  
Two diagrams are drawn to represent the freeze-drying process in a pressure-temperature (\( p \)-\( T \)) diagram.  

1. The first diagram shows phase regions with labeled points \( 1 \), \( 2 \), \( 3 \), and \( 4 \).  
   - Point \( 1 \) is near the triple point.  
   - Point \( 2 \) is horizontally aligned with \( 1 \), indicating isobaric evaporation.  
   - Point \( 3 \) is at a higher pressure, representing compression.  
   - Point \( 4 \) is horizontally aligned with \( 3 \), indicating isobaric condensation.  
   - The diagram includes shaded regions to indicate phase boundaries.  

2. The second diagram is simpler, showing the same points \( 1 \), \( 2 \), \( 3 \), and \( 4 \) connected in a rectangular cycle. The triple point is marked.  

The temperature \( T_i = 5^\circ\text{C} \) is noted.  

TASK 4b  
The following conditions are listed for the states:  
- State \( 1 \): \( p_1 = p_2 \)  
- State \( 2 \): \( p_2 = p_1 \), \( x_2 = 1 \)  
- State \( 3 \): \( p_3 = 8 \, \text{bar} \)  
- State \( 4 \): \( p_4 = 8 \, \text{bar} \), \( x = 0 \)  

The work \( w = 28 \, \text{W} \) is noted.  

TASK 4c  
The vapor quality at state \( 1 \) is calculated as:  
\[
x_1 = \frac{m_p}{m_p + m_f}
\]  

TASK 4d  
The coefficient of performance (\( \epsilon_K \)) is calculated as:  
\[
\epsilon_K = \frac{\dot{Q}_K}{\dot{W}_K} = \frac{1 \, \dot{Q}_K}{1 \, \dot{W}_K}
\]