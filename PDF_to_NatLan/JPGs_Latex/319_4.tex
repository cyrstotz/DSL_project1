TASK 3a  
The force balance equation is used to determine the pressure \( p_{g,1} \). Since the ice-water mixture (EW) is incompressible, the area \( A \) is calculated as:  
\[
A = \frac{D^2 \pi}{4}
\]  
The pressure \( p_{g,1} \) is expressed as:  
\[
p_{g,1} = p_{\text{amb}} + \frac{A \cdot m_K \cdot g}{A}
\]  
Substituting values, \( p_{g,1} = 1.3897 \, \text{bar} \).  

The mass of the gas \( m_g \) is determined using the ideal gas law:  
\[
pV = mRT
\]  
Rearranging for \( m \):  
\[
m = \frac{pV}{RT}
\]  
The specific gas constant \( R \) is calculated as:  
\[
R = \frac{R_u}{M} = \frac{8.314}{50} = 0.16628 \, \text{kJ/kg·K}
\]  
Finally, substituting values:  
\[
m_g = \frac{p_{g,1} V_{g,1}}{R T_{g,1}} = 3.478 \, \text{g}
\]  

---

TASK 3b  
The pressure \( p_{g,2} \) is equal to \( p_{g,1} \), as the pressure always pushes the piston upward and remains in equilibrium.  

---

TASK 3c  
The energy balance equation is applied:  
\[
\dot{E}_{\text{in}} - \dot{E}_{\text{out}} = \sum \dot{m}_{\text{in}} \cdot h_{\text{in}} + \text{ke}_{\text{in}} + \text{pe}_{\text{in}} - \sum \dot{Q} - \sum \dot{W}
\]  
This is simplified for the system:  
\[
m(u_2 - u_1) = Q_{12} - W_{12}
\]  
Substituting for heat transfer:  
\[
mc_V (T_2 - T_1) = Q_{12} - p_{\text{amb}} (\Delta V)
\]  
The change in volume \( \Delta V \) is calculated as:  
\[
\Delta V = V_{g,1} + \frac{V_{g,1}}{x_{\text{ice}}} \cdot T_{g,2}
\]  
Substituting values:  
\[
\Delta V = -2.0306 \, \text{L}
\]  

The final temperature of the gas \( T_{g,2} \) is approximately \( 0.003^\circ \text{C} \).  

---  
Descriptions of any figures or diagrams are not present on this page.