TASK 4a  
Two diagrams are drawn to represent the freeze-drying process in a pressure-temperature (\( p-T \)) diagram.  

- The first diagram shows phase regions labeled as "fest" (solid), "flüssig" (liquid), and "gasförmig" (gaseous). The triple point is marked, and arrows indicate transitions between states:  
  - State 1 (\( T_1 \)) is in the gaseous region.  
  - State 2 (\( T_2 \)) is in the liquid region.  
  - State 3 (\( T_3 \)) is in the solid region.  

- The second diagram similarly shows the phase regions "fest," "flüssig," and "gasförmig," with the triple point labeled. Arrows indicate transitions between states, but the layout is slightly simplified.  

TASK 4b  
The temperature \( T_v \) is 4 K above the sublimation temperature \( T_{\text{sub}} \), which corresponds to \( T_v = -18^\circ \text{C} \).  

The process is described as stationary and adiabatic between states 2 and 3.  

The refrigerant mass flow rate \( \dot{m}_{\text{R134a}} \) is calculated as:  
\[
\dot{m}_{\text{R134a}} = \frac{\dot{W}_K}{h_2 - h_3} = 0.834 \, \text{g/s}
\]  

The enthalpy values are determined:  
\[
h_2 = h_g(-16^\circ \text{C}) = 232.74 \, \frac{\text{kJ}}{\text{kg}}
\]  
\[
h_3 = h_f(8 \, \text{bar}) = s_3 \cdot \Delta T + h_f = 0.9298 \cdot (-0.9066) + 246.75 = 277.37 \, \frac{\text{kJ}}{\text{kg}}
\]  

The entropy values are:  
\[
s_2 = s_g(-16^\circ \text{C}) = 0.9298 \, \frac{\text{kJ}}{\text{kg·K}}
\]  
\[
s_3 = s_f(8 \, \text{bar}) = 0.9066 \, \frac{\text{kJ}}{\text{kg·K}}
\]  

TASK 4c  
The enthalpy at state 1 is calculated as:  
\[
h_1 = h_f(8 \, \text{bar}) + x_1 \cdot h_{fg} = 13.42 + 0.3 \cdot 76 = 36.6 \, \frac{\text{kJ}}{\text{kg}}
\]  

The vapor quality \( x_1 \) is determined:  
\[
x_1 = 0.3
\]  

At \( T_1 = -16^\circ \text{C} \), the coefficient of performance \( \epsilon_K \) is calculated as:  
\[
\epsilon_K = \frac{\dot{Q}_K}{\dot{W}_K}
\]  

TASK 4d  
The process between states 1 and 2 is described as isothermal and isolated.  

The heat transfer \( \dot{Q}_K \) is calculated as:  
\[
\dot{Q}_K = \dot{m}_{\text{R134a}} \cdot (h_2 - h_1) = 720.36 \, \text{W}
\]  

The coefficient of performance \( \epsilon_K \) is determined as:  
\[
\epsilon_K = 4.299
\]  

TASK 4e  
The temperature would continue to decrease until \( T_v = T_i \), and then remain constant at the same temperature as in the evaporator (\( T_v \)).