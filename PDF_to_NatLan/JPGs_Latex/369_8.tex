TASK 4a  
The diagram is a pressure-temperature (\( p \)-\( T \)) graph illustrating the phase regions of a substance. The graph includes the following features:  
- The solid phase ("Fest") is on the left, the liquid phase ("Flüssig") is in the middle, and the gas phase ("Gas") is on the right.  
- The triple point is marked where the solid, liquid, and gas phases coexist.  
- An isobaric evaporation process is shown, labeled as "isobar verdampfen."  
- An isobaric pressure reduction process is depicted, labeled as "isobar druckabnahme."  
- The critical point is labeled at the upper end of the gas-liquid boundary curve.  
- The axes are labeled: pressure (\( p \)) in bar or Pascal on the vertical axis, and temperature (\( T \)) in Kelvin on the horizontal axis.  

TASK 4b  
The stationary energy balance is written as:  
\[
\dot{W}_K = 28 \, \text{W}
\]  
\[
Q = \dot{m} (h_e - h_a) \quad \text{and} \quad t = -\dot{W}_K
\]  
\[
\dot{W}_K = 28 \quad V = \dot{m} (h_2 - h_g)
\]  

Definitions and assumptions:  
- \( h_e \): Enthalpy at state \( e \), where \( x_2 = ? \), and the substance is "gesättigt gedampft" (saturated vapor).  
- \( h_a \): Enthalpy at state \( a \), where \( p = 3 \, \text{bar} \), and the process is "adiabatisch reversibel" (adiabatic and reversible).  
- \( s_2 = s_3 \): Entropy remains constant.  
- \( h_g = 93.62 \, \text{kJ/kg} \).  
- \( h_i = h_e \), \( h_f = \) (value unclear).  

Additional notes:  
- \( T_K = -40^\circ\text{C} \).  
- \( T_f = -40^\circ\text{C} \).  

Some values and terms are crossed out or unclear, but the general energy balance and thermodynamic relationships are provided.