TASK 4a  
The diagram is a pressure-temperature (\(p\)-\(T\)) graph illustrating the freeze-drying process. It shows a closed loop with four states labeled as 1, 2, 3, and 4.  
- The x-axis represents temperature (\(T\)) in Kelvin.  
- The y-axis represents pressure (\(p\)) in bar.  
- The curve includes phase regions, with the process moving through states 1 to 4 in a cycle.  
- State 1 is in the low-pressure region, while state 3 is at higher pressure.  
- The transitions between states are connected by arrows indicating the direction of the process.  

TASK 4b  
The refrigerant mass flow rate is calculated using the following energy balance equation:  
\[
\dot{Q}_K = \dot{m} \cdot (h_3 - h_2)
\]  
where:  
- \(x_t = 1\) (vapor quality at state 1).  
- \(h\) represents specific enthalpy.  

The equation for the rate of change of energy is given as:  
\[
\frac{dE}{dt} = \dot{m} \cdot (h_3 - h_2) - \dot{V}_K + \dot{Q}_K
\]  
This simplifies to:  
\[
\dot{V}_K = \dot{m} \cdot (h_3 - h_2)
\]  

Additionally, entropy at states 2 and 3 is equal:  
\[
S_2 = S_3
\]  