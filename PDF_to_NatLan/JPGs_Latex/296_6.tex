TASK 4a  
The diagram is a pressure-temperature (\(P\)-\(T\)) graph. It shows the freeze-drying process with labeled regions and transitions.  
- The curve represents the phase boundary.  
- Point 1 is labeled as "Höher" (higher) and is connected to point 2 by an isothermal line.  
- Point 2 is labeled "Tripel" (triple point), indicating the triple point of water.  
- The process transitions from point 1 to point 2 along an isothermal path.  

TASK 4b  
The goal is to determine the mass flow rate of the refrigerant (\( \dot{m}_{\text{R134a}} \)).  

The energy balance equation is written as:  
\[
\dot{Q}_K = \dot{m} \cdot (h_2 - h_3)
\]  

Values for enthalpy are provided:  
\[
h_3 = h_f(\text{6 bar}) = 93.42 \, \frac{\text{kJ}}{\text{kg}}
\]  
\[
h_2 = h_g(-22^\circ\text{C}) = 235.57 \, \frac{\text{kJ}}{\text{kg}}
\]  

Entropy at state 2 and state 3 is assumed equal:  
\[
s_2 = s_3 = 0.9086 \, \frac{\text{kJ}}{\text{kg·K}}
\]  

Another enthalpy value is given:  
\[
h_4 = h_f(\text{6 bar}) = 264.47 \, \frac{\text{kJ}}{\text{kg}}
\]  

The mass flow rate is calculated as:  
\[
\dot{m} = \frac{\dot{Q}_K}{h_2 - h_3} = \frac{28 \, \text{kW}}{235.57 - 93.42 \, \frac{\text{kJ}}{\text{kg}}} = 0.195 \, \frac{\text{kg}}{\text{s}}
\]