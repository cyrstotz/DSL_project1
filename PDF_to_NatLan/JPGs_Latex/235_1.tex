TASK 2a  
The diagram represents a qualitative T-s diagram for the jet engine process. The temperature \( T \) is plotted on the vertical axis (in Kelvin), and the entropy \( s \) is plotted on the horizontal axis (in \( \frac{\text{kJ}}{\text{K}} \)). The process includes labeled isobars and isentropic lines:  
- State 1 to 2: Isentropic compression.  
- State 2 to 3: Isobaric heat addition.  
- State 3 to 4: Isentropic expansion.  
- State 4 to 5: Mixing process.  
- State 5 to 6: Isentropic expansion through the nozzle.  

Key points:  
- \( p_0 \) (ambient pressure) is labeled as "isobare."  
- \( p_2 \), \( p_3 = p_4 \), and \( p_5 \) are higher pressure levels.  
- The diagram visually shows the thermodynamic transitions between states.  

TASK 2b  
The energy balance for the entire turbine is written as:  
\[
\frac{dE}{dt} = \sum_i \dot{m}_i \left( h_i + k.e. + p.e. \right) + \sum_j \dot{Q}_j - \sum_n \dot{W}_n
\]  
Under steady-state, adiabatic conditions, and neglecting kinetic and potential energy changes, this simplifies to:  
\[
0 = \dot{m}_{\text{ges}} \left( h_6 - h_5 \right) + \frac{w_6^2 - w_5^2}{2}
\]  

For the isentropic process from state 5 to state 6:  
\[
\frac{T_b}{T_5} = \left( \frac{p_6}{p_5} \right)^{\frac{n-1}{n}}
\]  
Rewriting for \( T_b \):  
\[
T_b = \left( \frac{p_6}{p_5} \right)^{\frac{n-1}{n}} T_5
\]  
Substituting values:  
\[
T_b = 328.07 \, \text{K}
\]