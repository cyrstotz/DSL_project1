TASK 3a  
The gas pressure \( p_{g,1} \) is calculated using the formula:  
\[
p_{g,1} = p_{\text{amb}} + \frac{F}{A} + g \cdot (m_K + m_{\text{EW}})
\]  
Substituting the values:  
\[
p_{g,1} = 100,000 + \frac{5.81 \cdot (32 + 0.1)}{\frac{\pi}{4} \cdot (0.1)^2} = 140,089 \, \text{Pa}
\]  

The mass of the gas \( m_g \) is determined using the ideal gas law:  
\[
pV = mRT \quad \Rightarrow \quad m_g = \frac{p_{g,1} \cdot V_{g,1}}{R \cdot T_{g,1}}
\]  
Substituting the values:  
\[
m_g = \frac{140,089 \cdot 0.00314}{8.314 \cdot 773.15} = 0.00391 \, \text{kg}
\]  

---

TASK 3b  
The pressure \( p_{g,2} \) remains constant:  
\[
p_{g,2} = 140,089 \, \text{Pa}
\]  

The external pressure exerted on the gas must remain equal to the internal pressure, so the gas pressure does not change.  

The temperature of the gas in state 2 is:  
\[
T_{g,2} = 0^\circ\text{C}
\]  

Since the ice-water mixture is still present, the temperature of the ice-water system does not change until all the ice has melted. Once the gas reaches equilibrium with the ice-water mixture, it also assumes the temperature of \( 0^\circ\text{C} \).  

---

TASK 3c  
The energy balance is expressed as:  
\[
\Delta E = \Delta U = Q_{12} - W = Q_{12} - \int p \, dV
\]  
Rewriting:  
\[
Q_{12} = \Delta U + \int p \, dV = m_g \int c_V \, dT + \int p \, dV
\]  

The specific gas constant \( R_g \) is calculated:  
\[
R_g = \frac{8.314}{50 \cdot 10^{-3}} = 166.28 \, \text{J/(kg·K)}
\]  
The relationship between \( c_p \) and \( c_V \) is:  
\[
c_p - c_V = R_g
\]  

The final volume \( V_2 \) is determined:  
\[
V_2 = \frac{m_g \cdot R_g \cdot T_{g,2}}{p_{g,2}} = \frac{0.00391 \cdot 166.28 \cdot 273.15}{140,089} = 0.001103 \, \text{m}^3
\]  

The heat transferred \( Q_{12} \) is calculated:  
\[
Q_{12} = m_g \cdot c_V \cdot (T_{g,2} - T_{g,1}) + p_{g,1} \cdot (V_2 - V_1)
\]  
Substituting values:  
\[
Q_{12} = -1.08274 \, \text{kJ} - 0.28553 \, \text{kJ} = -1.3673 \, \text{kJ}
\]  

