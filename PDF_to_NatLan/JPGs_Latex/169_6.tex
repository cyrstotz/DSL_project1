TASK 4a  
Two diagrams are drawn to represent the freeze-drying process in a pressure-temperature (\(p\)-\(T\)) diagram.  

1. **First Diagram**:  
   - The diagram shows phase regions with labeled isobars and states.  
   - States 1, 2, 3, and 4 are marked along the process path.  
   - The isobar at \(p = 5 \, \text{bar}\) is drawn, with the vapor quality (\(x\)) transitioning from \(x = 0\) (saturated liquid) to \(x = 1\) (saturated vapor).  
   - The process path includes evaporation, compression, condensation, and expansion.  

2. **Second Diagram**:  
   - A more detailed view of the phase region is shown with an isobar at \(p = 8 \, \text{bar}\).  
   - States 1, 2, and 3 are labeled, with transitions between \(x = 0\) and \(x = 1\).  
   - The temperature axis (\(T\)) and pressure axis (\(p\)) are labeled.  

TASK 4b  
The energy balance equation for the refrigerant mass flow rate is written as:  
\[
0 = \dot{m}_{\text{R134a}} \left( h_2 - h_3 \right) - \dot{W}_K
\]  
Rearranging for the mass flow rate:  
\[
\dot{m}_{\text{R134a}} = \frac{\dot{W}_K}{h_2 - h_3}
\]  

The enthalpy \(h_2\) is expressed as:  
\[
h_2 = h_f + x \left( h_g - h_f \right) = h_f + h_{fg}
\]  

From Table A-12:  
\[
h_3 \, (p_3 = 8 \, \text{bar}) = ?
\]  

The enthalpy \(h_2\) at \(T_2 = -22^\circ\text{C}\) is calculated using Table A-10:  
\[
h_2 \, (T_2 = -22^\circ\text{C}) = 21.77 + 212.32 = 234.09 \, \text{kJ/kg}
\]