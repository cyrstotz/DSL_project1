TASK 2b  
The outlet velocity is given as \( w_5 = 220 \, \text{m/s} \), with \( p_5 = 0.5 \, \text{bar} \) and \( T_5 = 431.9 \, \text{K} \).  

The nozzle is modeled as adiabatic and reversible, implying \( s_5 = s_6 \). Using the ideal gas assumption:  
\[
\frac{p_6}{p_5} = \left( \frac{T_6}{T_5} \right)^{\frac{\kappa}{\kappa - 1}} \implies T_6 = T_5 \left( \frac{p_6}{p_5} \right)^{\frac{\kappa - 1}{\kappa}}
\]  
Substituting values:  
\[
T_6 = 431.9 \, \text{K} \cdot \left( \frac{0.191 \, \text{bar}}{0.5 \, \text{bar}} \right)^{\frac{0.4}{1.4}} = 328.07 \, \text{K}
\]  

A sketch of the nozzle process is drawn, showing the flow from state 5 to state 6.  

TASK 2c  
The mass-specific increase in flow exergy is calculated as:  
\[
\Delta ex_{\text{flow}} = e_{6,\text{flow}} - e_{0,\text{flow}}
\]  
This expands to:  
\[
\Delta ex_{\text{flow}} = \left( h_6 - h_0 - T_0 (s_6 - s_0) + \frac{w_6^2}{2} - \frac{w_0^2}{2} \right)
\]  

Breaking down the terms:  
\[
\Delta ex_{\text{flow}} = c_p (T_6 - T_0) - T_0 \left( c_p \ln \frac{T_6}{T_0} - R \ln \frac{p_6}{p_0} \right) + \frac{w_6^2}{2} - \frac{w_0^2}{2}
\]  

Substituting numerical values:  
\[
\Delta ex_{\text{flow}} = 1.006 \, \frac{\text{kJ}}{\text{kg·K}} \cdot (328.07 \, \text{K} - 243.15 \, \text{K}) - 243.15 \, \text{K} \cdot \left( 1.006 \ln \frac{328.07}{243.15} - 0.287 \ln \frac{0.191}{0.5} \right) + \frac{550^2}{2} - \frac{200^2}{2}
\]  

After calculation:  
\[
\Delta ex_{\text{flow}} = 60.66 \, \frac{\text{kJ}}{\text{kg}}
\]  

This result is boxed for emphasis.  

No additional diagrams or figures are described explicitly beyond the nozzle sketch.