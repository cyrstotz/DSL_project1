TASK 2a  
The process is represented qualitatively in a temperature-entropy (\(T\)-\(s\)) diagram. The diagram includes labeled isobars and key states (0, 1, 2, 3, 4, 5, 6). The axes are marked as follows:  
- The vertical axis is labeled \(T [K]\), representing temperature in Kelvin.  
- The horizontal axis is labeled \(s [\text{kJ}/\text{kg·K}]\), representing entropy in kilojoules per kilogram per Kelvin.  

The diagram shows a series of curves and lines connecting the states, with arrows indicating the direction of the process. The isobars are clearly drawn as dashed lines, and the transitions between states are highlighted.  

TASK 2b  
The first law of thermodynamics for the steady-state adiabatic process is written as:  
\[
0 = \dot{m} \left[ h_5 - h_6 + \frac{w_5^2}{2} - \frac{w_6^2}{2} \right] + Q - W_e
\]  
For an adiabatic nozzle, \(Q = 0\) and \(W_e = 0\), simplifying to:  
\[
0 = \dot{m} \left[ h_5 - h_6 + \frac{w_5^2}{2} - \frac{w_6^2}{2} \right]
\]  

The entropy balance is expressed as:  
\[
0 = \dot{m} \left[ s_5 - s_6 \right]
\]  

The entropy at state 5 (\(s_5\)) is equal to the entropy at state 6 (\(s_6\)):  
\[
s_5 = s_6
\]  

To determine \(s_5\), interpolation is performed using Table A-22 at a temperature of \(T_5 = 431.9 \, \text{K}\):  
\[
h_5 = \text{interpolation in Table A-22: } 440 - 430 = 433.86 \, \text{kJ/kg}
\]  

Additional calculations for \(s_5\) are crossed out and marked as incorrect.  

TASK 2a (continued)  
The diagram includes a zoomed-in section showing the transitions between states 0, 1, 2, 3, 4, 5, and 6. The states are connected by solid lines, and the process paths are labeled with arrows. The isobaric and adiabatic processes are clearly indicated.