TASK 1d  
The temperature values for the two states are given:  
1. \( T_1 = 100^\circ\text{C} \)  
2. \( T_2 = 70^\circ\text{C} \)  

The inlet temperature of the added mass is specified as \( T_{\text{in}} = 20^\circ\text{C} \).  

The heat released during cooling is given as \( Q_{R,12} = Q_{\text{out},12} = 35 \, \text{MJ} \).  

An energy balance is applied to determine the mass \( \Delta m_{12} \):  
\[
\Delta E = m_2 u_2 - m_1 u_1 = \Delta m_{12} h_e + \sum Q - \sum W
\]  
Here, \( Q_R = Q_{\text{out}} \), and \( Q = 0 \).  

From the water tables (Table A-2), the enthalpy at \( T_{\text{in}} = 20^\circ\text{C} \) is \( h_e = 83.16 \, \frac{\text{kJ}}{\text{kg}} \).  

The total mass at state 2 is expressed as:  
\[
m_2 = m_1 + \Delta m_{12}
\]  
The reactor mass at state 1 is given as \( m_1 = 5755 \, \text{kg} \).  

From Table A-2, the internal energy values are:  
\[
u_1 = u(T = 100^\circ\text{C}) = 426.94 \, \frac{\text{kJ}}{\text{kg}}
\]  
\[
u_2 = u(T = 70^\circ\text{C}) = 292.55 \, \frac{\text{kJ}}{\text{kg}}
\]  

Substituting into the energy balance:  
\[
m_2 u_2 + \Delta m_{12} u_2 - m_1 u_1 = \Delta m_{12} h_e
\]  
Rearranging:  
\[
m_1 (u_2 - u_1) = \Delta m_{12} (h_e - u_2)
\]  
Solving for \( \Delta m_{12} \):  
\[
\Delta m_{12} = \frac{m_1 (u_2 - u_1)}{h_e - u_2}
\]  

Substituting the values:  
\[
\Delta m_{12} = \frac{5755 \cdot (292.55 - 426.94)}{83.16 - 292.55} = 3486.41 \, \text{kg}
\]  

This calculation determines the mass of saturated liquid water added to reduce the reactor temperature from \( 100^\circ\text{C} \) to \( 70^\circ\text{C} \).