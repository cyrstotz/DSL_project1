TASK 4a  
The diagram is a pressure-temperature (\( p \)-\( T \)) graph illustrating the freeze-drying process. It includes the following features:  
- A phase boundary separating the liquid ("flüssig") and gaseous ("gasförmig") regions.  
- Two labeled steps:  
  - Step (i): A horizontal line at constant pressure, representing isobaric evaporation.  
  - Step (ii): A vertical line at constant temperature, representing sublimation.  
- The pressure is reduced to 5 mbar below the triple point of water.  
- The temperature \( T_i \) is marked, and it is 10 K above the sublimation temperature.  

TASK 4b  
Energy balance for the compressor:  
\[
0 = \dot{m} (h_2 - h_3) - W_{\text{tn}}
\]
Rearranging for the mass flow rate \( \dot{m} \):  
\[
\dot{m} = \frac{W_{\text{tn}}}{h_2 - h_3}
\]
Substituting \( W_{\text{tn}} = -28 \, \text{W} \):  
\[
\dot{m} = \frac{-28 \, \text{W}}{h_2 - h_3}
\]

For an adiabatic and reversible process:  
\[
S_2 = S_3 \, \text{(from the entropy balance).}
\]