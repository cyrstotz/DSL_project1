TASK 3a  
The pressure \( p_a \) is equal to \( p_g \), and the membrane is considered.  

The pressure \( p_a \) is calculated as:  
\[
p_a = p_{\text{amb}} + \frac{m_g \cdot g}{A} + \frac{m_{\text{EW}} \cdot g}{A}
\]  
Substituting values:  
\[
p_a = 10^5 + 32 \cdot 9.81 + \frac{0.1 \cdot 9.81}{0.05} = 140039 \, \text{Pa} = 1.4 \, \text{bar}
\]  

The cross-sectional area \( A \) is calculated as:  
\[
A = \frac{D^2}{4} \cdot \pi
\]  

The ideal gas law is applied:  
\[
pV = mRT
\]  
Rearranging for \( m \):  
\[
m = \frac{pV}{RT}
\]  
Substituting values:  
\[
m = \frac{1.4 \cdot 10^5 \cdot 3.14 \cdot 10^{-3}}{8.314 \cdot 500} = 0.05349 \, \text{kg} = 53.49 \, \text{g}
\]  

TASK 3b  
The pressure \( p_{g,2} \) is equal to \( p_{g,1} = 1.4 \, \text{bar}\), as the external forces remain the same.  

The temperature \( T_{g,2} \) is equal to \( T_{\text{EW}} = 0^\circ\text{C}\), as the water remains in the ice-water phase equilibrium and is in the liquid-solid region. No heat transfer occurs, and \( T_{g,2} = T_{\text{EW}} \).  

TASK 3c  
For the gas:  
The equation for heat transfer is:  
\[
Q = m_g \cdot c_V \cdot (T_2 - T_1)
\]  
Substituting values:  
\[
Q = 0.05349 \cdot 0.633 \cdot (273 - 800) = -1739.39 \, \text{J}
\]  

The heat transferred is:  
\[
Q = -1739.39 \, \text{J}
\]