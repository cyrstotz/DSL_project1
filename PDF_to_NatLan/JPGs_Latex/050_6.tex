TASK 4a  
Two diagrams are drawn to represent the freeze-drying process in a pressure-temperature (\(p\)-\(T\)) diagram.  

1. The first diagram shows labeled regions:  
   - "Compressed" region on the left.  
   - "ND" (likely indicating the nozzle/dry region) in the center.  
   - "SH" (superheated) region on the right.  
   - A curved line separates the compressed and superheated regions, with a point labeled "St1" (state 1).  
   - An arrow points from the compressed region toward the superheated region.  

2. The second diagram shows a similar layout but includes additional curves and arrows:  
   - A curved line labeled "Compressed" leads into the "ND" region.  
   - Another curved line separates the "ND" region from the "SH" region.  
   - Arrows indicate transitions between states.  

Both diagrams visually describe the freeze-drying process and the transitions between different thermodynamic states.  

---

TASK 4b  
The process from state 2 to state 3 is described as "adiabatic and reversible."  

An energy balance equation is provided:  
\[
\frac{dE}{dt} = 0 = \dot{m} (h_2 - h_3) + \dot{Q}^{\text{ad}} - \dot{W}
\]  

Additional details:  
- \( h_2 \) corresponds to state 2 with \( x = 1 \) (likely indicating saturated vapor).  
- \( h_3 \) corresponds to state 3 at 8 bar.  

The work done (\( W_K \)) is calculated as:  
\[
W_K = -28 \, \text{W}
\]  

---

TASK 4e  
The process involves an "adiabatic throttle" and is described as "isenthalpic."  

The enthalpy values are given:  
\[
h_1 = h_4
\]  
\[
h_2 (8 \, \text{bar}, x = 0) = 93.42 \, \frac{\text{kJ}}{\text{kg}}
\]  

The final statement reads:  
"Now I need \( T_1 \)."  

No additional calculations or diagrams are provided for this task.