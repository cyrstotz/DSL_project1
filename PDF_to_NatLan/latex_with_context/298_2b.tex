The outlet velocity \( w_6 \) and temperature \( T_6 \) are calculated. The ambient pressure \( p_0 \) is equal to the nozzle exit pressure \( p_6 \). The velocity at state 5 is given as \( w_5 = 220 \, \text{m/s} \). The process is modeled as adiabatic and isentropic.  

The temperature ratio is expressed as:  
\[
\frac{T_6}{T_5} = \left( \frac{p_6}{p_5} \right)^{\frac{\kappa - 1}{\kappa}}
\]  
Substituting \( \kappa = 1.4 \), \( p_6 = p_0 \), and \( p_5 = 0.5 \, \text{bar} \):  
\[
T_6 = T_5 \left( \frac{p_0}{p_5} \right)^{\frac{\kappa - 1}{\kappa}}
\]  
\[
T_6 = 431.9 \, \text{K} \left( \frac{0.191}{0.5} \right)^{\frac{0.4}{1.4}} = 390 \, \text{K}
\]  

The mass flow rate is expressed as:  
\[
\dot{m} = \rho A w
\]  
where \( \rho \) is the density, \( A \) is the cross-sectional area, and \( w \) is the velocity.  

The outlet velocity \( w_6 \) is calculated as:  
\[
w_6 = 510 \, \text{m/s}
\]  

---