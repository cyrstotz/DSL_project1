The gas pressure in state 1 is calculated using the equation:  
\[
p_{g,1} = p_{\text{amb}} + \frac{m_K \cdot g}{A} + \frac{m_{\text{EW}} \cdot g}{A}
\]  
where \( A = \frac{\pi D^2}{4} = 0.03142 \, \text{m}^2 \).  
Substituting the values, the result is:  
\[
p_{g,1} = 1.1 \, \text{bar}
\]  

The mass of the gas is determined using the ideal gas law:  
\[
p_1 V_1 = m R T_1 \quad \Rightarrow \quad m = \frac{p_1 V_1}{R T_1}
\]  
Substituting \( p_1 = 1 \, \text{bar} \), \( V_1 = 3.14 \, \text{L} \), \( R = 8.314 \, \text{J/(mol·K)} \), and \( T_1 = 500^\circ\text{C} = 773.15 \, \text{K} \), the result is:  
\[
m = 2.687 \, \text{g}
\]  

---