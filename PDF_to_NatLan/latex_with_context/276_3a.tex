The gas temperature is given as \( T_g = 500^\circ\text{C} \).  
The gas volume is \( V_s = 3.14 \, \text{L} \).  

The equation for pressure is:  
\[
p_s V_s = M_g \frac{R}{M} T_g
\]  

The cross-sectional area of the cylinder is calculated as:  
\[
A = \left( \frac{D}{2} \right)^2 \pi = 0.00785 \, \text{m}^2
\]  

A sketch is provided showing the forces acting on the piston in equilibrium. The piston is subjected to the ambient pressure \( p_{\text{amb}} \), the pressure from the gas \( p_g \), and the weight of the piston \( M_K g \). The equilibrium condition is expressed as:  
\[
p_g \cdot A = p_{\text{amb}} \cdot A + M_K \cdot g
\]  

Rearranging for \( p_g \):  
\[
p_g = p_{\text{amb}} + \frac{M_K \cdot g}{A}
\]  

Substituting values:  
\[
p_g = 1 \, \text{bar} + \frac{32 \, \text{kg} \cdot 9.81 \, \text{m/s}^2}{0.00785 \, \text{m}^2}
\]  
\[
p_g = 1393870 \, \text{Pa} = 1.4 \, \text{bar}
\]  

The gas mass \( M_g \) is calculated using:  
\[
M_g = \frac{p_g \cdot V_s}{\frac{R}{M} \cdot T_g}
\]  

Substituting values:  
\[
M_g = \frac{1.4 \cdot 10^5 \, \text{Pa} \cdot 3.14 \cdot 10^{-3} \, \text{m}^3}{\frac{8.314 \, \text{J/(mol·K)}}{50 \, \text{kg/kmol}} \cdot 773.15 \, \text{K}}
\]  
\[
M_g = 3.429 \, \text{g} = 3.42 \cdot 10^{-3} \, \text{kg}
\]  

The temperature is converted to Kelvin:  
\[
T_g = 773.15 \, \text{K}
\]  

The volume is converted to cubic meters:  
\[
V_s = 3.14 \cdot 10^{-3} \, \text{m}^3
\]  

---