The equation for the refrigerant mass flow rate is given as:  
\[
\dot{m}_R (h_2 - h_3) = \dot{W}_K
\]  

The vapor quality \( x_2 \) is equal to 1.  

The enthalpy \( h_2 \) corresponds to the saturated vapor enthalpy \( h_g \) at \( T_i = 10^\circ\text{C} \).  

An interpolation is performed between \( 8^\circ\text{C} \) and \( 12^\circ\text{C} \):  
\[
\frac{259.03 - 257.8}{12 - 8} (10^\circ - 8^\circ) + 257.8
\]  

The resulting enthalpy is:  
\[
h_2 = 252.936 \, \frac{\text{kJ}}{\text{kg}}
\]  

The enthalpy \( h_3 \) corresponds to the saturated liquid enthalpy at 8 bar.  
The temperature \( T_i \) is approximately \( 40^\circ\text{C} \).  

The refrigerant mass flow rate is calculated using:  
\[
\dot{m}_{R,\text{R134a}} = \frac{\dot{W}_K}{h_2 - h_3}
\]  

Figure Description:  
A simple graph is drawn with axes labeled \( h \) (enthalpy) and \( T \) (temperature). The graph appears to represent a general thermodynamic relationship between enthalpy and temperature.