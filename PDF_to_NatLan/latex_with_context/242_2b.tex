The pressure at state 6 is given as \( p_6 = p_0 = 0.191 \, \text{bar} \). Using the equation for the outlet velocity \( w_6 \), the following steps are performed:  

The energy balance is written as:  
\[
O = \dot{m} \left( h_5 + \frac{1}{2} w_5^2 - h_6 - \frac{1}{2} w_6^2 \right) = W
\]  

From the isentropic relation:  
\[
\frac{T_6}{T_5} = \left( \frac{p_6}{p_5} \right)^{\frac{\kappa - 1}{\kappa}}
\]  
Substituting values:  
\[
T_6 = T_5 \left( \frac{p_6}{p_5} \right)^{\frac{\kappa - 1}{\kappa}} = 328.14 \, \text{K}
\]  

To determine \( w_6 \), the energy balance is expanded:  
\[
O = \dot{m} \left( h_0 + \frac{1}{2} w_0^2 - h_6 - \frac{1}{2} w_6^2 \right) + \dot{Q} + \dot{W}
\]  

Rewriting for \( w_6 \):  
\[
\frac{1}{2} w_0^2 = c_p \left( T_0 - T_6 \right) + a_5 (0.2035)
\]  
\[
w_6 = \sqrt{2 \cdot \left( c_p \left( T_0 - T_6 \right) + \frac{1}{2} w_0^2 + a_5 (0.2035) \right)}
\]  
Substituting values yields:  
\[
w_6 = 438.866 \, \text{m/s}
\]  

---