A graph is drawn representing the jet engine process on a \( T \)-\( s \) diagram. The axes are labeled as follows:  
- The vertical axis is \( T \) (temperature) in Kelvin \([K]\).  
- The horizontal axis is \( s \) (specific entropy) in \([ \frac{k}{\text{kg} \cdot \text{K}} ]\).  

The process includes the following labeled points and transitions:  
- Point 1: Starting point, labeled with "nus < 1".  
- Point 2: Isentropic process, labeled "adiabatic rev".  
- Point 3: Isobaric process, labeled "isobar".  
- Point 5: Isentropic process, labeled "adiabatic rev".  
- Point 6: Isentropic process, labeled "adiabatic rev".  

Additional annotations include:  
- \( p_5 = p_6 = p_0 \), indicating constant pressure at points 5 and 6.  
- "Mischkammer isobar", referring to the mixing chamber operating isobarically.  
- \( p_0 = p_8 \), indicating constant pressure at points 0 and 8.