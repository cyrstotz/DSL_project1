The diagram is a pressure-temperature (\( p \)-\( T \)) graph illustrating the phase regions of a substance. The graph includes the following labeled regions:  
- "fest" (solid phase)  
- "flüssig" (liquid phase)  
- "gasförmig" (gaseous phase)  

The triple point (\( T_{\text{triple}} \)) is marked where the solid, liquid, and gaseous phases coexist. Two processes are indicated:  
- Process (i): A horizontal line at constant pressure within the liquid phase.  
- Process (ii): A vertical line at constant temperature transitioning from the liquid phase to the gaseous phase.  

The axes are labeled:  
- \( p \) [bar] for pressure (y-axis)  
- \( T \) [°C] for temperature (x-axis).