The mass flow rate of the refrigerant \( \dot{m}_{\text{R134a}} \) is being calculated for an isentropic process.  

A diagram is drawn showing a compressor labeled "isentrop" with an inlet at \( x = 1 \) and an outlet at \( 8 \, \text{bar} \). The compressor is connected to a work input labeled \( 28 \, \text{W} \).  

The entropy at the outlet is equal to the entropy at the inlet:  
\[
s_2 = s_3
\]  
\[
s_2 = s_g(T_2)
\]  

Using the first law of thermodynamics for a steady-state flow process:  
\[
0 = \dot{m} [h_2 - h_3] + \sum \dot{Q} - \dot{W}_k
\]  
Since the process is adiabatic, \( \sum \dot{Q} = 0 \).  

Rearranging for the mass flow rate:  
\[
\dot{m} = \frac{\dot{W}_k}{h_2 - h_3}
\]  

The saturation temperature at \( 8 \, \text{bar} \) is given as:  
\[
T_{\text{sat}} = 31.33^\circ \text{C} \, @ \, 8 \, \text{bar}
\]  

---