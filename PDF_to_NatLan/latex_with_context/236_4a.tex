The table labeled "Zustandstabelle" (state table) contains the following information:  

| Zustand (State) | \( P \) (Pressure) | \( T \) (Temperature) | \( x \) (Vapor Quality) |  
|-----------------|--------------------|-----------------------|-------------------------|  
| 1               | \( P_1 = P_2 = 1.5748 \, \text{bar} \) | \( T = 257.15 \, \text{K} \) | \( x = 1 \) |  
| 2               | \( P_2 = P_1 = 1.5748 \, \text{bar} \) | \( T = 257.15 \, \text{K} \) | \( x = 1 \) |  
| 3               | \( P = 8 \, \text{bar} \) | (No temperature given) | (No vapor quality given) |  
| 4               | \( P = 8 \, \text{bar} \) | (No temperature given) | \( x = 0 \) |  

Additional notes:  
- The process between states 1 and 2 is described as "adiabatic" with \( W_K = 25 \, \text{W} \).

Two diagrams are drawn:  

1. The first diagram is a \( p \)-\( T \) plot with pressure \( p \) (in mbar) on the vertical axis and temperature \( T \) (in \( ^\circ\text{C} \)) on the horizontal axis.  
   - The process is shown as two steps:  
     - Step i is an isobaric process.  
     - Step ii is a sublimation process.  

2. The second diagram is another \( p \)-\( T \) plot with similar axes.  
   - Phase regions are labeled, and the process steps i and ii are indicated.  
   - Step i involves condensation, and step ii involves sublimation.