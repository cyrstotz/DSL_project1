The gas pressure \( p_{g,1} \) is calculated using the formula:  
\[
p_{g,1} = \frac{-m_{\text{EW}} g}{A} + \frac{m_K g}{A} + p_{\text{amb}}
\]  
The cross-sectional area \( A \) is determined as:  
\[
A = \pi \left(\frac{D}{2}\right)^2 = 0.0079 \, \text{m}^2
\]  
Substituting values, the pressure is:  
\[
p_{g,1} = 1.33861 \, \text{bar}
\]  

The mass of the gas \( m_g \) is calculated using the ideal gas law:  
\[
p V = m_g R T
\]  
where \( R = \frac{\bar{R}}{M_g} = 0.16628 \, \text{kJ/kg·K} \).  
Substituting values, the mass is:  
\[
m_g = 0.00394 \, \text{kg} \quad \text{or} \quad 3.94 \times 10^{-3} \, \text{kg}
\]