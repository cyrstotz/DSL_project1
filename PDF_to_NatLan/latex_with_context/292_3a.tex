The gas pressure \( p_{g,1} \) and mass \( m_g \) in state 1 are calculated as follows:  

The pressure \( p_{g,1} \) is given by:  
\[
p_{g,1} = p_{\text{atm}} + \frac{m_K \cdot g}{A}
\]  
Substituting values:  
\[
p_{g,1} = 1.00 \, \text{bar} + \frac{32 \, \text{kg} \cdot 9.81 \, \text{m/s}^2}{\left( \frac{\pi \cdot (5 \cdot 10^{-2} \, \text{m})^2}{4} \right)} = 1.33 \, \text{bar}
\]  

The gas mass \( m_g \) is calculated using the ideal gas law:  
\[
m_g = \frac{p V}{R T}
\]  
Substituting values:  
\[
m_g = \frac{1.33 \cdot 10^5 \, \text{Pa} \cdot 3.14 \cdot 10^{-3} \, \text{m}^3}{106.28 \, \text{J/mol·K} \cdot 773.15 \, \text{K}} = 3.39 \, \text{g}
\]  

Constants used:  
\[
R = \frac{8.314 \, \text{J/mol·K}}{50 \, \text{kg/kmol}} = 106.28 \, \text{J/kg·K}
\]  
\[
T_1 = 773.15 \, \text{K}
\]  

---