The page contains two diagrams labeled as part of the freeze-drying process described in Task 4.  

**Diagram (i):**  
This is a pressure-temperature (\( p \)-\( T \)) diagram illustrating the refrigeration cycle.  
- The diagram shows four states labeled \( 1 \), \( 2 \), \( 3 \), and \( 4 \).  
- The process between states \( 1 \) and \( 2 \) is marked as "isobaric."  
- The process between states \( 2 \) and \( 3 \) is labeled "adiabatic reversible."  
- The process between states \( 3 \) and \( 4 \) is marked as "isobaric."  
- The process between states \( 4 \) and \( 1 \) is labeled "adiabatic reversible."  
- The curve labeled "NS" represents the saturation line of the refrigerant.  

**Diagram (ii):**  
This is another \( p \)-\( T \) diagram illustrating the sublimation process in Step ii of freeze-drying.  
- The diagram shows a curve labeled "NS," which represents the saturation line.  
- A point labeled "Triple point" is marked at the peak of the curve.  
- The process is labeled "isotherm," indicating constant temperature during sublimation.