The pressure \( p \) is calculated using the formula:  
\[
p = p_{\text{amb}} + \frac{F}{\pi r^2}
\]  
Substituting values:  
\[
p = p_{\text{amb}} + \frac{m_K \cdot g}{\pi r^2}
\]  
\[
p = 10^5 + \frac{32 \cdot 9.81}{\pi \cdot (5 \cdot 10^{-2})^2}
\]  
\[
p = 1.4 \, \text{bar}
\]  

The diameter \( D \) is given as \( D = 2r \), with \( r = 5 \, \text{cm} \).  

The mass \( m_g \) is calculated using the ideal gas law:  
\[
m_g = \frac{p \cdot V}{R \cdot T}
\]  
The specific gas constant \( R \) is determined as:  
\[
R = \frac{R_{\text{univ}}}{M_g} = \frac{8314}{50 \cdot 10^{-3}} = 166.28 \, \text{J/kg·K}
\]  
Substituting values:  
\[
m_g = \frac{1.4 \cdot 10^5 \cdot 3.14 \cdot 10^{-3}}{166.28 \cdot (500 + 273.15)}
\]  
\[
m_g = 0.34 \, \text{kg}
\]