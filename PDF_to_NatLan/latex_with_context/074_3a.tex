A table is presented with two states labeled "1" and "2." The table includes columns for temperature \( T \) and volume \( V \):  
- State 1: \( T = 500^\circ\text{C} \), \( V = 3.14 \, \text{L} \)  
- State 2: No values are provided for \( T \) or \( V \).  

The problem begins with the calculation of total pressure \( p_{\text{tot}} \):  
\[
p_{\text{tot}} = \frac{m_K g}{A} + \frac{m_{\text{EW}} g}{A} + p_{\text{amb}}
\]  
The cross-sectional area \( A \) is calculated as:  
\[
A = (0.05 \, \text{m})^2 \cdot \pi = 0.00785 \, \text{m}^2
\]  

Substituting values into the pressure equation:  
\[
p_{\text{tot}} = \frac{32 \, \text{kg} \cdot 9.81 \, \text{m/s}^2}{A} + \frac{0.1 \, \text{kg} \cdot 9.81 \, \text{m/s}^2}{A} + 100000 \, \text{Pa}
\]  
\[
p_{\text{tot}} = 1400000 \, \text{Pa}
\]  

The ideal gas law is applied:  
\[
p \cdot V = m \cdot R \cdot T
\]  
The specific gas constant \( R \) is calculated:  
\[
R = \frac{R_{\text{univ}}}{M} = \frac{8314 \, \text{J/(kmol·K)}}{50 \, \text{kg/kmol}} = 166.28 \, \text{J/(kg·K)}
\]  

Rearranging the ideal gas law to solve for \( m \):  
\[
m = \frac{p \cdot V}{R \cdot T}
\]  
Substituting values:  
\[
m = \frac{1400000 \, \text{Pa} \cdot 0.01007 \, \text{m}^3}{166.28 \, \text{J/(kg·K)} \cdot (500^\circ\text{C} + 273.15 \, \text{K})}
\]  
\[
m = 0.006342 \, \text{kg}
\]  

No diagrams or figures are present on the page.