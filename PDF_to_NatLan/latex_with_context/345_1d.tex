To determine \( \Delta m_{12} \), the energy balance is applied:  
\[
m_2 \cdot h_2 - m_1 \cdot h_1 = \Delta m_{12} \cdot h_{\text{in}} + Q_{R,12}
\]  

From Table A-2:  
\[
h_2 = 2506.5 \, \text{kJ/kg}, \quad h_1 = 835.06 \, \text{kJ/kg}, \quad h_{\text{in}} = 83.94 \, \text{kJ/kg}
\]  

Substituting values:  
\[
5755 \cdot 2506.5 - 5755 \cdot 835.06 = \Delta m_{12} \cdot 83.94 + 35 \, \text{MJ}
\]  

Solving for \( \Delta m_{12} \):  
\[
\Delta m_{12} = \frac{5755 \cdot (2506.5 - 835.06) - 35 \, \text{MJ}}{83.94} = 232.095 \, \text{kg}
\]  

---

The energy balance equation is used to determine the mass \( \Delta m_{12} \) added to the reactor during cooling:  
\[
m_{12} \cdot h_{12} + \Delta m_{12} \cdot h_{\text{in}} = m_{12} \cdot h_{2}
\]  
Rearranging the equation:  
\[
\Delta m_{12} = \frac{m_{12} \cdot (h_{2} - h_{12})}{h_{\text{in}} - h_{2}}
\]  
Substituting values:  
\[
\Delta m_{12} = \frac{3755 \cdot (419.17 - 293.07)}{83.93 - 419.17} = 13.009 \, \text{kg}
\]