The task involves determining \( w_6 \) (outlet velocity) and \( T_6 \) (outlet temperature).  

The following calculations are shown:  
1. The adiabatic relation for temperature:  
   \[
   T_6 = T_5 \left( \frac{p_0}{p_5} \right)^{\frac{\kappa - 1}{\kappa}}
   \]  
   Substituting values:  
   \[
   T_6 = 431.9 \, \text{K} \cdot \left( \frac{0.191}{0.5} \right)^{\frac{0.4}{1.4}} = 328.075 \, \text{K}
   \]  
   The result is boxed: \( T_6 = 328.075 \, \text{K} \).  

2. To determine \( w_6 \), the energy balance equation is used:  
   \[
   0 = h_{\text{in}} - h_{\text{aus}} + \frac{w_{\text{in}}^2 - w_{\text{aus}}^2}{2} + \dot{Q}_B \cdot \frac{1}{\dot{m}_K}
   \]  
   The term \( \dot{Q}_B \cdot \frac{1}{\dot{m}_K} \) is crossed out.  

3. Enthalpy difference is calculated using:  
   \[
   h_{\text{in}} - h_{\text{aus}} = c_p \cdot (T_{\text{ein}} - T_{\text{aus}})
   \]  
   Substituting values:  
   \[
   2 \cdot \left( c_p \cdot (T_{\text{aus}} - T_{\text{ein}}) - \frac{Q_B}{5.293} \right) = 200^2 - w_6^2
   \]  
   Rearranging and solving:  
   \[
   280.68 \, \text{kJ} = 200^2 - w_6^2
   \]  

No final value for \( w_6 \) is explicitly calculated on the page.

The outlet velocity \( w_6 \) is calculated using the energy balance equation:  
\[
2 \cdot (1.006 \cdot (328.047 - 243.15) - 1195) = 200^2 - w_6^2
\]  
Rearranging and solving for \( w_6^2 \):  
\[
200^2 - 280621.233 \, \text{J} = w_6^2
\]  
Taking the square root:  
\[
w_6 = 490.6 \, \text{m/s}
\]  

---