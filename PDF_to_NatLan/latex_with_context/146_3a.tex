The gas pressure \( p_{g,1} \) and mass \( m_g \) in state 1 are calculated as follows:  

The pressure \( p_{g,1} \) is given by:  
\[
p_{g,1} = p_0 + \frac{F_{\text{mem}}}{A} + \frac{F_{\text{piston}}}{A} = p_0 + \frac{32 \, \text{kg} \cdot 9.81 \, \text{m/s}^2 + 0.1 \, \text{kg} \cdot 5.81 \, \text{m/s}^2}{\pi \left(\frac{D}{2}\right)^2}
\]  

Substituting values:  
\[
p_{g,1} = 1.10054 \, \text{bar}
\]  

The ideal gas law is used to calculate \( m_g \):  
\[
PV = mRT \quad \Rightarrow \quad m_g = \frac{p_{g,1} V_{g,1}}{R T_{g,1}}
\]  

Substituting values:  
\[
m_g = \frac{1.4 \, \text{bar} \cdot 3.14 \times 10^{-3} \, \text{m}^3}{8.314 \, \text{J/mol·K} \cdot 773.15 \, \text{K}}
\]  

The mass of the gas is:  
\[
m_g = 0.00342 \, \text{kg} = 3.42 \, \text{g}
\]  

---