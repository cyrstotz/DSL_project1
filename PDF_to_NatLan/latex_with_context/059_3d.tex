The internal energy \( u_2 \) at state 2 is calculated using:  
\[
u_2 = u_f + x (u_{\text{rest}} - u_f)
\]  

The change in internal energy \( \Delta U \) is equal to the heat transferred:  
\[
\Delta U = \Delta Q
\]  

The difference in internal energy between states 2 and 1 is given by:  
\[
u_2(T_2) - u_1(T_1) = C_r (T_2 - T_1)
\]  
where \( T_1 = 0^\circ\text{C} \), \( T_2 = 0^\circ\text{C} \), and \( C_r \) is the specific heat capacity of the ice-water mixture.  

Substituting \( \Delta Q = 15 \, \text{kJ} \), \( m_{\text{EW}} = 0.1 \, \text{kg} \):  
\[
u_2(T_2) - u_1(T_1) = \frac{\Delta Q}{m_{\text{EW}}} = \frac{15}{0.1} = 150 \, \text{kJ/kg}
\]  

For the ice fraction \( x_1 \):  
\[
u_1(T_1) = u_f + x_1 (u_{\text{rest}} - u_f)
\]  
Substituting \( u_f = -0.045 \, \text{kJ/kg} \), \( u_{\text{rest}} = -333.458 \, \text{kJ/kg} \), and \( x_1 = 0.6 \):  
\[
u_1(T_1) = -0.045 + 0.6 \cdot (-333.458 - (-0.045)) = -200.082 \, \text{kJ/kg}
\]  

(Note: Some calculations and expressions are crossed out and not considered here.)  

---  
No diagrams or graphs are present on this page.