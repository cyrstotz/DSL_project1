The problem setup includes the following given values:  
- Airspeed: \( w_{\text{air}} = 200 \, \text{m/s} \)  
- Ambient pressure: \( p_0 = 0.191 \, \text{bar} \)  
- Ambient temperature: \( T_0 = -30^\circ\text{C} \)  
- Heat added per unit mass in the core stream: \( q_B = \frac{\dot{Q}_B}{\dot{m}_K} \)  
- Ratio of bypass to core mass flow rates: \( \frac{\dot{m}_M}{\dot{m}_K} = 5.293 \)  
- Specific heat capacity of air: \( c_{p,\text{air}} = 1.006 \, \text{kJ/kg·K} \)  
- Adiabatic index: \( n = 1.4 \)  

The task involves drawing the process qualitatively in a \( T \)-\( s \) diagram with labeled isobars and units on the axes.  

### Diagram Description  
The \( T \)-\( s \) diagram is drawn with the following features:  
- The vertical axis represents temperature \( T \) in Kelvin.  
- The horizontal axis represents entropy \( s \) in \( \frac{\text{kJ}}{\text{kg·K}} \).  
- The process begins at state \( 0 \), moves to state \( 1 \) via an irreversible compression (\( \eta_{\text{vs}} < 1 \)).  
- From state \( 1 \) to \( 2 \), the process is isentropic.  
- From state \( 2 \) to \( 3 \), the process is isobaric with an increase in temperature (\( ++T \)).  
- From state \( 3 \) to \( 4 \), the process involves an irreversible expansion (\( \eta_{\text{ts}} < 1 \)).  
- From state \( 4 \) to \( 5 \), the process is isobaric, with \( p_4 = p_5 = 0.5 \, \text{bar} \).  
- From state \( 5 \) to \( 6 \), the process is isentropic.  

The diagram includes labeled isobars \( p_2 = p_3 \) and \( p_4 = p_5 \), as well as arrows indicating the direction of the process.  

### Process Steps  
- \( 0 \to 1 \): Irreversible compression (\( \eta_{\text{vs}} < 1 \))  
- \( 1 \to 2 \): Isentropic compression  
- \( 2 \to 3 \): Isobaric heating (\( ++T \))  
- \( 3 \to 4 \): Irreversible expansion (\( \eta_{\text{ts}} < 1 \))  
- \( 4 \to 5 \): Isobaric process (\( p_4 = p_5 = 0.5 \, \text{bar} \))  
- \( 5 \to 6 \): Isentropic expansion  

No additional calculations or explanations are provided on this page.