The goal is to determine \( w_6 \) and \( T_6 \).  

The isentropic relationships are used, with known values of \( w_5 \), \( p_5 \), and \( T_5 \). The equation for energy balance is given:  
\[
Q = \dot{m} \left( h_s - h_6 + \frac{\Delta k_e}{2} \right), \quad \Delta k_e = \frac{w_5^2 - w_6^2}{2}
\]  

The enthalpy \( h_s \) is calculated using tabulated values:  
\[
h_s = h(930 \, \text{K}) + h(990 \, \text{K}) - h(950 \, \text{K}) \quad \text{with corrections of } 1.9 \, \text{kJ/kg}.
\]  

The equations are set up to solve for \( w_6 \) and \( T_6 \) based on the given data and relationships.  

No additional diagrams or graphs are present beyond the T-s diagram described above.