The process is illustrated in a \( T \)-\( s \) diagram, showing the thermodynamic states of the jet engine. The diagram includes labeled points corresponding to states \( 0 \), \( 3 \), \( 4 \), \( 5 \), and \( 6 \).  
- The process between states \( 0 \) and \( 3 \) is marked as "isotrop" (adiabatic and reversible compression).  
- The process between states \( 3 \) and \( 4 \) is labeled "isobar" (constant pressure combustion).  
- The process between states \( 4 \) and \( 5 \) is marked as "isotrop" (adiabatic and reversible expansion in the turbine).  
- The process between states \( 5 \) and \( 6 \) is labeled "same pressure" (isentropic mixing chamber).  
- The isobars are drawn steeply, with a note suggesting they should be adjusted to be more realistic.  

The axes are labeled:  
- \( T(h) \) for temperature on the vertical axis.  
- \( s \, (\text{kJ}/\text{kg·K}) \) for entropy on the horizontal axis.