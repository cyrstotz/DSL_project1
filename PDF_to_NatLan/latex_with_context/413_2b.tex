To calculate \( w_6 \) and \( T_6 \):  
The steady-state energy balance for the entire system is applied, assuming adiabatic conditions.  

\[
0 = h_{\text{in}} - h_{\text{out}} + \frac{w_{\text{in}}^2 - w_{\text{out}}^2}{2}
\]

Rearranging for \( w_{\text{out}} \):  
\[
w_{\text{out}} = \sqrt{2(h_{\text{in}} - h_{\text{out}}) + w_{\text{in}}^2}
\]

For an ideal gas:  
\[
h = c_{p,\text{air}}(T_{\text{in}} - T_{\text{out}}) + v^2 \frac{(p_3 - p_1)}{\rho}
\]

The temperature at the nozzle exit \( T_{\text{out}} \) is calculated using:  
\[
T_{\text{out}} = T_{\text{in}} \left(\frac{p_6}{p_0}\right)^{\frac{\kappa - 1}{\kappa}}
\]

Where \( \kappa = 1.4 \) for air. Substituting:  
\[
T_{\text{out}} = T_{\text{in}}
\]

In the limit, \( w_{\text{out}} = w_{\text{in}} \).

The problem involves a steady flow process in a nozzle. The energy balance for the adiabatic process is written as:  
\[
h_5 - h_6 + \frac{w_5^2 - w_6^2}{2} = 0
\]  
Rearranging for the exit velocity \( w_6 \):  
\[
w_6 = \sqrt{2(h_5 - h_6) + w_5^2}
\]  
Substituting the given values:  
\[
w_6 = 220 \, \frac{\text{m}}{\text{s}}
\]  

The temperature \( T_6 \) is calculated using the isentropic relation:  
\[
T_6 = T_5 \left( \frac{p_6}{p_5} \right)^{\frac{\kappa - 1}{\kappa}}
\]  
Substituting the values:  
\[
T_6 = 75 \left( \frac{p_6}{p_5} \right)^{\frac{\kappa - 1}{\kappa}} = 328.07468 \, \text{K}
\]  

---