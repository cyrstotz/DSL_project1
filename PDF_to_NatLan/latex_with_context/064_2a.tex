A qualitative \( T \)-\( s \) diagram is drawn, showing the thermodynamic process of the jet engine. The diagram includes labeled isobars and key states:  
- State 1: Inlet air conditions.  
- State 2: After compression.  
- State 3: Combustion.  
- State 4: Expansion in the turbine.  
- State 5: Mixing chamber.  
- State 6: Nozzle exit.  

The axes are labeled as follows:  
- \( T \) (temperature) on the vertical axis.  
- \( s \) (entropy) on the horizontal axis.  

The process lines indicate compression, combustion, expansion, and mixing, with arrows showing the direction of the process.  

---

The page contains a graph labeled as a \( T \)-\( s \) diagram. The axes are marked as follows:  
- The vertical axis is labeled \( T \, [\text{K}] \), representing temperature in Kelvin.  
- The horizontal axis is labeled \( s \, [\frac{\text{kJ}}{\text{kg·K}}] \), representing specific entropy in kilojoules per kilogram per Kelvin.  

The graph depicts a thermodynamic process with several states labeled numerically:  
- State 0 is at the bottom left of the diagram.  
- State 1 is slightly above state 0, connected by a curve.  
- State 2 is higher up and connected to state 1 by another curve.  
- State 3 is at the peak of the diagram, representing the highest temperature.  
- States 4 and 5 are on the descending curve, with state 5 being lower than state 4.  
- State 6 is at the bottom right of the diagram, connected to state 5.  

The graph includes annotations for "isobar" curves, indicating constant pressure processes. These curves are drawn to show transitions between states.  

This diagram qualitatively represents the thermodynamic process of a jet engine, as described in Task 2a.

The page contains two diagrams related to thermodynamic processes.  

1. **First Diagram (labeled "a")**:  
   - The graph is a qualitative representation of a thermodynamic process in a \( p \)-\( v \) diagram.  
   - The vertical axis is labeled \( p \) (pressure), and the horizontal axis is labeled \( v \) (specific volume).  
   - Several curves are drawn, with points labeled 1, 2, 3, and 4.  
   - The process appears to involve multiple transitions between states, with arrows indicating the direction of the process.  
   - The curves suggest changes in pressure and specific volume, possibly representing compression, expansion, or other thermodynamic transformations.  

2. **Second Diagram**:  
   - The graph is a \( p \)-\( v \) diagram with the vertical axis labeled \( p \) (pressure) and the horizontal axis labeled \( v \) (specific volume).  
   - The diagram shows a closed cycle with points labeled 1, 2, 3, and 4.  
   - The cycle includes arrows indicating the direction of the process, suggesting a thermodynamic cycle such as a Carnot or Rankine cycle.  
   - The curve transitions between states, with pressure and specific volume changing throughout the cycle.  

No additional textual explanations or equations are visible on the page.