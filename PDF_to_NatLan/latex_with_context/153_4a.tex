Two diagrams are drawn to represent the freeze-drying process in a pressure-temperature (\(p\)-\(T\)) diagram.  

1. **Left Diagram**:  
   - The \(p\)-\(T\) curve shows the phase regions of a substance.  
   - The curve includes a labeled "Critical Point" at the top of the dome-shaped region.  
   - The region under the dome is labeled "Nassdampf" (wet steam).  
   - The axes are labeled as \(p\) (pressure in bar) and \(T\) (temperature in Kelvin).  

2. **Right Diagram**:  
   - The diagram shows the sublimation process.  
   - A line labeled "Triple Point" marks the intersection of solid, liquid, and gas phases.  
   - The gas phase is labeled, and the sublimation process is indicated with an arrow labeled "Sublimation."  
   - The axes are labeled as \(p\) (pressure in bar) and \(T\) (temperature in Kelvin).  
   - A label "10 K" appears near the sublimation line, indicating the temperature difference above the sublimation temperature.

The coefficient of performance \( \epsilon \) is calculated as:  
\[
\epsilon = \frac{\dot{Q}_{\text{ab}}}{\dot{W}_K} = \frac{\dot{Q}_{\text{Zyl}}}{\dot{W}_K} = \frac{\dot{Q}_{\text{K}}}{\dot{Q}_{\text{ab}} - \dot{Q}_{\text{Zyl}}}
\]  

The heat transfer rates are defined as:  
\[
\dot{Q}_{\text{K}} = \dot{m} \left[ h_2 - h_3 \right]
\]  
\[
\dot{Q}_{\text{ab}} = \dot{m} \left[ h_4 - h_3 \right]
\]