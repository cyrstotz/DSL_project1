The pressure \( p_{g,2} \) is equal to \( p_{g,1} \) because the equilibrium from state 1 does not change.  

The temperature \( T_2 \) is calculated using the isentropic relation:  
\[
T_2 = T_1 \left( \frac{p_2}{p_1} \right)^{\frac{n-1}{n}}
\]
where \( n = \frac{c_p}{c_v} \), and \( c_p \) and \( c_v \) are related by:  
\[
n = \frac{R + c_v}{c_v} = \frac{\frac{R}{M} + c_v}{c_v}
\]
Substituting values:  
\[
n = \frac{8.314}{50} + 0.633 = 1.2627
\]

For \( T_2 \):  
\[
T_2 = 500 \, \text{K} \left( \frac{1.4}{1.4} \right)^{\frac{1.2627 - 1}{1.2627}}
\]
\[
T_2 = T_1
\]

Thus, \( T_2 = T_1 \) because the pressure ratio does not change, and the temperature remains constant.