The task involves drawing a qualitative T-s diagram for the jet engine process.  

**Description of the diagram:**  
The diagram consists of two parts:  
1. The left graph is a T-s diagram showing the thermodynamic process of the jet engine. It includes labeled states (1, 2, 3, 4, 5, and 6) and isobars (horizontal lines). The process starts at state 1, moves through compression (state 2), combustion (state 3), expansion (state 4), mixing (state 5), and ends at state 6. The axes are labeled as \( T \) (temperature in Kelvin) and \( s \) (specific entropy).  
2. The right graph is a zoomed-in section showing the mixing process between states 5 and 6. It includes labeled isobars and arrows indicating heat transfer (\( \dot{Q} \)) and work (\( \dot{W} \)).

The ideal gas law is applied:  
\[
p \cdot V = R \cdot T
\]  
From this, the following relationships are derived:  
\[
p_0 \cdot v_0 = R \cdot T_0
\]  
\[
p_6 \cdot v_6 = R \cdot T_6
\]  

Rearranging for specific volume:  
\[
v_0 = \frac{R \cdot T_0}{p_0}, \quad v_6 = \frac{R \cdot T_6}{p_6}
\]  

Using the continuity equation:  
\[
\dot{m} = \rho \cdot A \cdot w
\]  
and assuming \( \dot{m}_6 = \dot{m}_0 \), \( A_6 = A_0 \), the velocity \( w \) is expressed as:  
\[
w = \frac{\dot{m}}{\rho \cdot A}
\]  
Substituting for \( \rho = \frac{p}{R \cdot T} \):  
\[
w_0 = \frac{\dot{m}}{\frac{p_0}{R \cdot T_0} \cdot A_0}
\]  
\[
w_6 = \frac{\dot{m}}{\frac{p_6}{R \cdot T_6} \cdot A_6}
\]