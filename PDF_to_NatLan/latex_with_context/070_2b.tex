The goal is to determine the outlet velocity \( w_6 \) and temperature \( T_6 \).  

Given data:  
\[
w_5 = 220 \, \text{m/s}, \quad T_5 = 431.9 \, \text{K}, \quad p_5 = 0.5 \, \text{bar}, \quad p_6 = p_0 = 0.191 \, \text{bar}.
\]  
Assumptions:  
- Isentropic process: \( s_5 = s_6 \).  

The energy balance equation for the steady-state flow process is written as:  
\[
0 = \dot{m} \left( h_5 - h_6 \right) + \frac{w_5^2 - w_6^2}{2} + \dot{Q} - \dot{W}.
\]  
Since the process is adiabatic and no work is done, \( \dot{Q} = 0 \) and \( \dot{W} = 0 \).  

Rewriting:  
\[
h_5 - h_6 + \frac{w_5^2 - w_6^2}{2} = 0.
\]  

The enthalpy difference is expressed as:  
\[
h_5 - h_6 = c_p \left( T_5 - T_6 \right).
\]  
Substituting into the energy balance:  
\[
c_p \left( T_5 - T_6 \right) + \frac{w_5^2 - w_6^2}{2} = 0.
\]  

Solving for \( w_6 \):  
\[
w_6 = \sqrt{w_5^2 + 2 c_p \left( T_5 - T_6 \right)}.
\]  

Using the given values:  
\[
c_p = 1.006 \, \text{kJ/kg·K}, \quad R = 0.287 \, \text{kJ/kg·K}.
\]  
\[
R = c_p - c_v \quad \Rightarrow \quad c_v = c_p - R = 0.9024 \, \text{kJ/kg·K}.
\]  

For the isentropic process \( s_5 = s_6 \):  
\[
0 = c_p \ln \left( \frac{T_6}{T_5} \right) - R \ln \left( \frac{p_6}{p_5} \right).
\]  
Rewriting:  
\[
\ln \left( \frac{T_6}{T_5} \right) = \frac{R}{c_p} \ln \left( \frac{p_6}{p_5} \right).
\]  
Exponentiating:  
\[
\frac{T_6}{T_5} = \left( \frac{p_6}{p_5} \right)^{\frac{R}{c_p}}.
\]  
Substituting values:  
\[
T_6 = T_5 \left( \frac{p_6}{p_5} \right)^{\frac{R}{c_p}} = 431.9 \left( \frac{0.191}{0.5} \right)^{\frac{0.287}{1.006}}.
\]  
\[
T_6 = 328.075 \, \text{K}.
\]  

Finally, substituting \( T_6 \) into the velocity equation:  
\[
w_6 = \sqrt{220^2 + 2 \cdot 1.006 \cdot (431.9 - 328.075)}.
\]  
\[
w_6 = \sqrt{48400 + 2 \cdot 1.006 \cdot 103.825}.
\]  
\[
w_6 = \sqrt{48400 + 208.9} = \sqrt{48608.9}.
\]  
\[
w_6 = 220.37 \, \text{m/s}.
\]  

Summary:  
\[
T_6 = 328.075 \, \text{K}, \quad w_6 = 220.37 \, \text{m/s}.
\]