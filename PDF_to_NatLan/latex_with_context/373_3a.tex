The gas pressure \( p_{g,1} \) is calculated using the formula:  
\[
p_{g,1} = \frac{m_{K} g}{A} + \frac{m_{\text{EW}} g}{A} + p_{\text{amb}}
\]  
The area \( A \) is determined as:  
\[
A = \pi \left(\frac{D}{2}\right)^2 = 7.854 \times 10^{-3} \, \text{m}^2
\]  
Substituting the values:  
\[
p_{g,1} = \frac{32 \, \text{kg} \cdot 9.81 \, \text{m/s}^2}{7.854 \times 10^{-3} \, \text{m}^2} + \frac{0.1 \, \text{kg} \cdot 9.81 \, \text{m/s}^2}{7.854 \times 10^{-3} \, \text{m}^2} + 1.013 \, \text{bar}
\]  
\[
p_{g,1} = 3.997 \, \text{bar} + 1.013 \, \text{bar} = 5.010 \, \text{bar}
\]  

For the gas:  
\[
p V = n R T
\]  
The gas mass \( m_g \) is calculated as:  
\[
m_g = \frac{p_{g,1} V_{g,1}}{R T_{g,1}}
\]  
Substituting the values:  
\[
m_g = \frac{5.010 \times 10^5 \, \text{N/m}^2 \cdot 3.14 \times 10^{-3} \, \text{m}^3}{8.314 \, \text{J/mol·K} \cdot 273.15 \, \text{K}} \cdot \frac{50 \, \text{kg}}{\text{kmol}}
\]  
\[
m_g = 3.198 \times 10^{-3} \, \text{kg}
\]  

---