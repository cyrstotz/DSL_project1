An energy balance is applied to the compressor for the transition from state 2 to state 3:  
\[
0 = \dot{m} \left( h_{\text{out}} - h_{\text{in}} \right) - W_{\text{c}}
\]  
Rearranging for the compressor work:  
\[
W_{\text{c}} = \dot{m} \left( h_{\text{out}} - h_{\text{in}} \right)
\]  

The following temperature values are provided:  
\[
T_2 = T_i + 20 \, \text{K}
\]  
\[
T_1 = T_i - 6 \, \text{K}
\]  

From the tables:  
\[
T_i = -10^\circ\text{C} \quad \text{(as per the table)}
\]  
\[
T_2 = -10^\circ\text{C} + 20 \, \text{K} = 10^\circ\text{C}
\]  
\[
T_1 = -10^\circ\text{C} - 6 \, \text{K} = -16^\circ\text{C}
\]  

At state 2, the vapor quality is \(x = 1\), indicating saturated vapor.  

From the table (A-10):  
\[
h_2 = h_{\text{out}} = 237.74 \, \text{kJ/kg}
\]  

No further calculations are visible.

The temperature \( T_2 \) is given as \( -16^\circ\text{C} \), and the enthalpy \( h_2 \) is \( 237.74 \, \frac{\text{kJ}}{\text{kg}} \). The pressure \( p_3 \) is \( 8 \, \text{bar} \).

The process is reversible, so \( s_2 = s_3 \).

From Table A-12, the entropy values are:
- \( s_{\text{sat}} = 0.92066 \, \frac{\text{kJ}}{\text{kg·K}} \) at \( 31.33^\circ\text{C} \),
- \( s(20^\circ\text{C}) = 0.8374 \, \frac{\text{kJ}}{\text{kg·K}} \).

Interpolation is used to determine \( T_3 \).

Expression for \( T_3 \):  
\[
T_3 =
\]