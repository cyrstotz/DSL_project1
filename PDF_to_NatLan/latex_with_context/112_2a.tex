The page contains diagrams illustrating the thermodynamic processes of a jet engine in a \( T \)-\( s \) diagram and a simplified process sketch. Below is the description of the content:

---

### Diagram 1: \( T \)-\( s \) Diagram  
The first diagram is a \( T \)-\( s \) plot (temperature vs. entropy) showing the thermodynamic processes in the jet engine. The axes are labeled as follows:  
- \( T \) (temperature) on the vertical axis.  
- \( s \) (specific entropy) on the horizontal axis, with units \( \frac{\text{kJ}}{\text{kg·K}} \).  

The process is divided into six states (0, 1, 2, 3, 4/5, and 6), with the following annotations:  
1. **State 0 to 1**: Adiabatic and irreversible compression.  
2. **State 1 to 2**: Isobaric heat addition.  
3. **State 2 to 3**: Reversible adiabatic compression.  
4. **State 3 to 4/5**: Isobaric combustion.  
5. **State 4/5 to 6**: Adiabatic and irreversible expansion.  

Additional notes:  
- \( p_2 = p_3 \): Pressure remains constant between states 2 and 3.  
- \( p_0 = p_6 \): Pressure at the inlet and outlet is equal.  

---

### Diagram 2: \( T \)-\( s \) Diagram with Expanded Details  
The second \( T \)-\( s \) diagram provides more detailed annotations:  
- \( p_2 = p_3 = p_4 = p_5 \): Pressure is constant across states 2, 3, 4, and 5.  
- \( p_0 = p_6 \): Pressure at the inlet and outlet is equal.  
- The process includes labeled regions for "isentropic" and "adiabatic irreversible" transitions.  
- The combustion process is explicitly labeled as "isobaric."  

---

### Diagram 3: Simplified Process Sketch  
A simplified sketch of the thermodynamic process is shown, with states labeled as 0, 1, 2, 3, 4/5, and 6. The sketch is a qualitative representation of the transitions between states, showing the general shape of the process paths.

---

This page visually represents the thermodynamic processes in a jet engine, focusing on entropy and temperature changes across different states.