The equation for work done by the compressor is written as:  
\[
W_K = \dot{m} \cdot (h_2 - h_3)
\]  
where \( \dot{m} \) is the mass flow rate, \( h_2 \) and \( h_3 \) are specific enthalpies at states 2 and 3, respectively.  

Given data:  
- \( p_1 = 1 \, \text{mbar} \), \( T_i = -10^\circ\text{C} \)  
- \( T_2 = -22^\circ\text{C} \), \( h_2 = 234.08 \, \text{kJ/kg} \), \( s_2 = 0.9 \)  
- \( h_3 = 284.75 + (273.15 - 268.15) \cdot 1.006 = 277.95 \, \text{kJ/kg} \)  

Using Table A-10:  
- \( s_2 = 0.9066 \)  
- \( h_3 = 0.9066 \cdot 1.006 = 0.9066 \, \text{kJ/kg} \)