The triple point temperature is \( T_{\text{triple}} = 0^\circ\text{C} \).  
The sublimation temperature at 5 mbar is \( T_{\text{sub}} = -20^\circ\text{C} \).  
The initial temperature \( T_i \) is set 10 K above \( T_{\text{sub}} \), resulting in \( T_i = -10^\circ\text{C} \).  

A graph is drawn showing pressure (\( p \)) on the y-axis and temperature (\( T \)) on the x-axis. The graph includes the following labeled regions:  
- "Triple point" at the intersection of the solid, liquid, and gas phase boundaries.  
- "Gas" region above the sublimation curve.  
- "Sublimation" curve separating the solid and gas phases.  
- "Solid" region below the sublimation curve.  
- "Frost fusion" curve separating the solid and liquid phases.  
The graph also includes arrows indicating the direction of sublimation and isothermal processes.

---