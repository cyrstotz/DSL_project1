The temperature \( T_i \) would decrease because the pressure \( p \) is less than the triple point pressure (\( p < p_{\text{triple}} \)).