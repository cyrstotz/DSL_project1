The mass of the gas (\( m_g \)) is calculated using the ideal gas law:  
\[
m_g = \frac{p V}{R T}
\]  
The specific gas constant (\( R \)) is derived from the universal gas constant (\( \bar{R} \)) divided by the molar mass (\( \mu \)):  
\[
R = \frac{\bar{R}}{\mu} = \frac{8.314 \, \text{J/(mol·K)}}{50 \, \text{kg/kmol}}
\]  
This simplifies to:  
\[
R = 0.16628 \, \text{J/(g·K)} = 166.28 \, \text{J/(kg·K)}
\]  
Substituting into the equation, the mass of the gas is calculated as:  
\[
m_g = 0.003922 \, \text{kg} = 3.922 \, \text{g}
\]

The specific heat capacity at constant volume is given as:  
\[
c_V = 0.633 \, \frac{\text{kJ}}{\text{kg·K}}
\]  
The molar mass of the gas is:  
\[
M = 50 \, \frac{\text{kg}}{\text{kmol}}
\]  

The task is to determine the gas pressure \( p_{g,1} \) and mass \( m_g \).  

The ideal gas law is written as:  
\[
pV = mRT
\]  

The total mass acting on the piston is:  
\[
0.1 \, \text{kg} + 32 \, \text{kg}
\]  

A sketch of the cylinder is drawn, showing a piston resting atop the ice-water mixture.  

The force exerted by the gas is given as:  
\[
mg = F
\]  

The pressure is calculated as:  
\[
\frac{mg}{0.05 \, \text{m}^2} = p = 40.094 \, \text{bar}
\]  

The total pressure in state 1 is the sum of atmospheric pressure and the pressure exerted by the piston:  
\[
p_{1,g} = p_{\text{atm}} + p_{\text{piston}} = 1.40094 \, \text{bar}
\]  

The final pressure is rounded to:  
\[
1.40 \, \text{bar}
\]