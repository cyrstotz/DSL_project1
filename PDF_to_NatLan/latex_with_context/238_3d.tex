For the ice-water system:  
\[
m_{\text{gas}} \cdot (u_2, \text{eis} - u_1, \text{gas}) = -Q
\]  

Given:  
\[
u_1, \text{gas} = u_{\text{flüssig}}(1 \, \text{bar}) + x_{\text{eis}} \cdot (u_{\text{fest}} - u_{\text{flüssig}})
\]  
\[
u_{\text{flüssig}}(1 \, \text{bar}) = -0.045 \, \frac{\text{kJ}}{\text{kg}}
\]  
\[
u_{\text{fest}}(1 \, \text{bar}) = 333.458 \, \frac{\text{kJ}}{\text{kg}}
\]  
\[
u_2, \text{eis} = -200.0928 \, \frac{\text{kJ}}{\text{kg}}
\]  

The heat transfer is calculated:  
\[
m_{\text{gas}} \cdot u_2, \text{eis} = -18.875 \, \text{kJ}
\]  
\[
u_2, \text{eis} = -185.75 \, \frac{\text{kJ}}{\text{kg}}
\]  

The final ice fraction \( x_{\text{eis},2} \) is determined:  
\[
x_{\text{eis},2} = \frac{u_2, \text{eis} - u_{\text{flüssig}}(1 \, \text{bar})}{u_{\text{fest}}(1 \, \text{bar}) - u_{\text{flüssig}}(1 \, \text{bar})} = 0.566
\]  

**Conclusion:**  
The final ice fraction is \( x_{\text{eis},2} = 0.566 \).