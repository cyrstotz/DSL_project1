The energy balance equation is written as:  
\[
\Delta E = \Delta U, \quad Q_{R,12} = -35 \, \text{MJ}
\]  
From this, the following relationship is derived:  
\[
U_2 - U_1 = \Delta m_{12} \cdot h(20^\circ\text{C}) + Q_{R,12}
\]  
The total mass in state 1 is given as:  
\[
m_{\text{ges,1}} = 5755 \, \text{kg} = m_1
\]  
The total mass in state 2 is expressed as:  
\[
m_{\text{ges,2}} = m_{\text{ges,1}} + \Delta m_{12} = m_2
\]  
Substituting into the energy balance:  
\[
m_2 U_2 - m_1 U_1 = \Delta m_{12} \cdot h(20^\circ\text{C}) + Q_{R,12}
\]  
Rearranging for \( \Delta m_{12} \):  
\[
\Delta m_{12} = \frac{m_1 (U_1 - U_2) + Q_{R,12}}{U_2 - h(20^\circ\text{C})}
\]  

From Table A-2, the following values are used:  
\[
h(20^\circ\text{C}) = h_f(20^\circ\text{C}) = 83.96 \, \text{kJ/kg} \quad \text{(saturated liquid)}
\]  
\[
u_2 = u_f(70^\circ\text{C}) = 292.95 \, \text{kJ/kg}
\]  
\[
u_1 = u_f(100^\circ\text{C}) = 418.94 \, \text{kJ/kg}
\]  

Substituting these values into the equation:  
\[
\Delta m_{12} = \frac{5755 \cdot (418.94 - 292.95) + (-35 \cdot 10^3)}{292.95 - 83.96}
\]  
The result is:  
\[
\Delta m_{12} = 3302 \, \text{kg}
\]  

---