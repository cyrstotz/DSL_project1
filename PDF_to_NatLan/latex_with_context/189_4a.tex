A graph is drawn showing the freeze-drying process in a pressure-temperature (\( p \)-\( T \)) diagram. The graph includes the following points:  
- Point 1 is at the top right, representing the initial state.  
- Point 2 is horizontally aligned with Point 1 but at a lower pressure.  
- Point 3 is vertically below Point 2, representing the sublimation pressure at 5 mbar.  
- The triple point is marked on the graph, and the pressure axis includes labels for "5 mbar" and "Triple Point."  
The temperature axis is labeled as \( T \), and the pressure axis is labeled as \( p \).

The process is described as reversible and adiabatic, which implies that \( U_4 = U_1 \).  

For isobaric condensation, the pressure at state 4 equals the pressure at state 3:  
\[
p_4 = p_3 = 8 \, \text{bar}
\]  

The specific internal energy at state 4 is equal to the specific internal energy at state 3:  
\[
U_4 = U_3 = U_g (\text{from tables}) = 0.0255 \, \text{kJ/kg}
\]