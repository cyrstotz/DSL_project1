To determine \( T_i \):  
The pressure is 5 mbar below the triple point, and the temperature is 10 K above the sublimation temperature.  
Thus, \( p = 1 \, \text{mbar} \) and \( T_i = -10^\circ\text{C} = 263 \, \text{K} \).  

The evaporator temperature is given as:  
\[
T_{\text{Verdampfer}} = 257 \, \text{K}
\]  

The enthalpy values are:  
\[
h_2 = h_{2g}, \quad h_3 = h_{3,f} \, (\text{fully compressed at 8 bar})
\]  
From the table (A.11):  
\[
h_{3,f} = h_3 = 93.42 \, \text{kJ/kg}
\]  

---

The vapor quality \( x_1 \) is to be determined.  

The mass flow rate of R134a is given as:  
\[
\dot{m}_{\text{R134a}} = 4 \, \frac{\text{kg}}{\text{h}}
\]  
The temperature \( T_2 \) is specified as:  
\[
T_2 = -22^\circ\text{C}
\]  

If \( x_1 = 0 \), the refrigerant is completely liquid.  

The formula for vapor quality \( x_1 \) is:  
\[
x_1 = \frac{s_1 - s_{1,f}}{s_{1,g} - s_{1,f}}
\]  

The process is adiabatic, with pressure equal to the ambient pressure.  

Using the entropy balance:  
\[
\dot{m} \, (s_4 - s_1) + \sum \frac{\dot{Q}_i}{T_i} + \dot{S}_{\text{gen}} = 0
\]  
Here, the term \(\sum \frac{\dot{Q}_i}{T_i}\) is zero for an adiabatic process.