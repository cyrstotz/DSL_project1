The pressure \( p_{g,1} \) is given as:  
\[
p_{g,1} = 1.5 \cdot 10^5 \, \text{Pa}, \quad m_g = 3.6 \, \text{kg} \, \text{(as stated in the problem)}
\]  

The temperature of the ice-water mixture remains constant:  
\[
T_{\text{EW}} = T_g \quad \text{(thermal equilibrium)}
\]  

The change in internal energy is expressed as:  
\[
\Delta U_{\text{EW}} = Q_{12}, \quad \Delta U_g
\]  
The energy balance for the ice-water mixture is:  
\[
U_{2,\text{EW}} - U_{1,\text{EW}} = u_{2,\text{EW}} - u_{1,\text{EW}}
\]  
\[
u_{2,\text{EW}} + 2000 \cdot x_{2,\text{EW}} = c_u (T_1 - T_2)
\]  

For the gas:  
\[
p_{g,2} = p_{g,1} \quad \text{and} \quad T_{g,2} = T_1 \left(\frac{p_{g,2}}{p_{g,1}}\right)^{\frac{1}{n}}
\]  
The temperature \( T_{g,2} \) is calculated as:  
\[
T_{g,2} = \frac{p_{g,2} \cdot U_{g,2}}{m_g \cdot R}
\]  

The specific internal energy of the ice-water mixture is expressed as:  
\[
u_{2,\text{EW}} = x \cdot u_{\text{ice}} + (1-x) \cdot u_{\text{water}}
\]  
\[
u_{2,\text{EW}} = 2000 \cdot x_{2,\text{EW}} + 3 \cdot c_u
\]  

The change in internal energy for the gas is calculated using the integral:  
\[
u_{g,2} - u_{g,1} = \int_{T_1}^{T_2} c_v \, dT = c_v (T_1 - T_2)
\]  

The polytropic exponent \( n \) is derived as:  
\[
n = \frac{c_p}{c_v} = \frac{c_p - R}{c_v} = \frac{c_u + R}{c_v}
\]  
Substituting values:  
\[
n = 1.66269
\]  

---  
No diagrams or graphs are present on this page.