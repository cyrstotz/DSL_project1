The pressure of the gas in state 1, \( p_{g,1} \), is calculated using the equation:  
\[
p_{g,1} = \frac{m_{g,1} \cdot g}{A} + \frac{m_K \cdot g}{A} + p_0
\]  
where \( A = \frac{\pi D^2}{4} = 0.05 \pi \, \text{m}^2 \).  

Substituting the values:  
\[
p_{g,1} = \frac{0.1 \, \text{kg} \cdot 9.81 \, \text{m/s}^2}{A} + \frac{32 \, \text{kg} \cdot 9.81 \, \text{m/s}^2}{A} + 10^5 \, \text{Pa}
\]  
\[
p_{g,1} = 140,037.9 \, \text{Pa} = 1.4 \, \text{bar}
\]  

Next, the mass of the gas, \( m_g \), is determined using the ideal gas law:  
\[
m_g = \frac{p V_{g,1}}{R T_{g,1}}
\]  
where \( R = \frac{8.314 \, \text{J/mol·K}}{50 \, \text{kg/kmol}} = 166.28 \, \frac{\text{J}}{\text{kg·K}} \).  

Substituting the values:  
\[
m_g = \frac{140,037.9 \, \text{Pa} \cdot 3.14 \cdot 10^{-3} \, \text{m}^3}{166.28 \, \frac{\text{J}}{\text{kg·K}} \cdot 773.15 \, \text{K}}
\]  
\[
m_g = 0.00342 \, \text{kg} = 3.42 \, \text{g}
\]