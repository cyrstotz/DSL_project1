The energy balance for determining \( \Delta m_{12} \) is derived using the first law of thermodynamics:  
\[
\Delta U = m_2 u_2 - m_1 u_1 - \dot{m} h_{\text{in}} - Q_{R,12}
\]  
Where:  
- \( h_{\text{in}} = h_f(20^\circ\text{C}) = 83.96 \, \text{kJ/kg} \)  
- \( Q_{R,12} = 35 \times 10^6 \, \text{J} \)  

The mass \( m_2 \) is expressed as:  
\[
m_2 = m_1 + m_{\text{ges},1}
\]  
Substituting into the energy balance:  
\[
\dot{m} (u_2 - h_{\text{in}}) = m_{\text{ges},1} (u_1 - u_2) - Q_{R,12}
\]  
Rearranging for \( \Delta m \):  
\[
\Delta m = \frac{m_{\text{ges},1} (u_1 - u_2) - Q_{R,12}}{u_2 - h_{\text{in}}}
\]  
Substituting values:  
- \( u_2 = u_f(70^\circ\text{C}) = 292.515 \, \text{kJ/kg} \)  
- \( u_1 = u_f(100^\circ\text{C}) = 418.9 \, \text{kJ/kg} \)  

The calculation yields:  
\[
\Delta m = \frac{3302 \, \text{kg} \cdot (418.9 - 292.515) - 35 \times 10^6}{292.515 - 83.96}
\]  
Resulting in:  
\[
\Delta m = 330 \, \text{kg}
\]  

---