The cross-sectional area of the cylinder is calculated as:  
\[
A = 5 \, \text{cm}^2 \cdot \pi = 0.00785 \, \text{m}^3
\]  

The force exerted by the piston is determined using the equation:  
\[
F = m \cdot g = 32 \, \text{kg} \cdot 9.81 \, \text{m/s}^2 = 314.9 \, \text{N}
\]  

The pressure exerted by the piston is:  
\[
p_e = \frac{F}{A} = \frac{314.9 \, \text{N}}{0.00785 \, \text{m}^2} = 0.84 \, \text{bar}
\]  

The total pressure in the system is:  
\[
P = 1 \, \text{bar} + 0.84 \, \text{bar} = 1.84 \, \text{bar}
\]  

The molar mass of the gas is given as:  
\[
M = 50 \, \frac{\text{kg}}{\text{kmol}}
\]  

The mass of the gas is calculated using the ideal gas law:  
\[
m = \frac{p \cdot V}{R \cdot T} = \frac{1.84 \, \text{bar} \cdot 3.14 \cdot 10^{-3} \, \text{m}^3}{766.28 \, \frac{\text{J}}{\text{kg·K}} \cdot 773.75 \, \text{K}} = 0.0034 \, \text{kg} = 3.4 \, \text{g}
\]  

The temperature is:  
\[
T = 773.75 \, \text{K}
\]  

The pressure is converted to Pascals:  
\[
\phi = 1.84 \, \text{bar} = 1.84 \cdot 10^3 \, \text{Pa}
\]  

The specific gas constant is calculated as:  
\[
R = \frac{R_{\text{universal}}}{M} = \frac{8.314 \, \frac{\text{kJ}}{\text{kmol·K}} \cdot 10^3}{50 \, \frac{\text{kg}}{\text{kmol}}} = 766.28 \, \frac{\text{J}}{\text{kg·K}}
\]  

The volume of the gas is:  
\[
V = 3.14 \cdot 10^{-3} \, \text{m}^3
\]  

---