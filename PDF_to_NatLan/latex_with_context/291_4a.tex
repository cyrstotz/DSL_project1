The phase diagram is drawn with pressure (\( p \)) on the y-axis and temperature (\( T \)) on the x-axis. The diagram includes the following features:  

1. **Triple Point**: Labeled at \( T = 0^\circ\text{C} \), where solid, liquid, and gas phases coexist.  
2. **Solid Region**: Below the triple point temperature and higher pressures.  
3. **Liquid Region**: Between the solid and gas regions, above the triple point temperature and pressures.  
4. **Gas Region**: At lower pressures and higher temperatures.  
5. **Sublimation Line**: A dotted line extending from the triple point to lower pressures, indicating sublimation at \( p = 5 \, \text{mbar} \).  
6. **Key Points**:  
   - \( T_i = 9^\circ\text{C} \) corresponds to \( T_i = 282.15 \, \text{K} \).  
   - \( T_{\text{sub}} @ 5 \, \text{mbar} = -1^\circ\text{C} \).  
   - \( T_K = T_i - 6 \, \text{K} \).  

The diagram visually represents the transitions between solid, liquid, and gas phases under varying pressure and temperature conditions.