The goal is to determine \( \Delta m_{12} \), the mass of saturated liquid water added to the reactor, so that the reactor temperature \( T_{R,2} \) reaches \( 70^\circ\text{C} \).  

Given data:  
\[
T_{\text{in,12}} = 20^\circ\text{C}, \quad Q_{R,12} = 35 \, \text{MJ}
\]  

An energy balance for the isolated system is applied:  
\[
\Delta E = m_2 u_2 - m_1 u_1 = \Delta m_{12} h_{\text{ein}} + Q_{R,12} - Q_{\text{out},12}
\]  

Where:  
\[
m_2 = m_1 + \Delta m_{12}, \quad u_2 = u(70^\circ\text{C}, \text{saturated liquid}), \quad m_1 = m_{\text{total},1}, \quad u_1 = u(100^\circ\text{C}, \text{saturated liquid}), \quad h_{\text{ein}} = h(20^\circ\text{C}, \text{saturated liquid})
\]  

Substituting into the energy balance:  
\[
(m_1 + \Delta m_{12}) u_2 - m_1 u_1 = \Delta m_{12} h_{\text{ein}} + Q_{R,12}
\]  

Rearranging:  
\[
\Delta m_{12} (u_2 - h_{\text{ein}}) = m_1 (u_1 - u_2) + Q_{R,12}
\]  

Solving for \( \Delta m_{12} \):  
\[
\Delta m_{12} = \frac{m_1 (u_1 - u_2) + Q_{R,12}}{u_2 - h_{\text{ein}}}
\]  

Substituting numerical values:  
\[
m_1 = 5755 \, \text{kg}, \quad u_1 = 418.94 \, \frac{\text{kJ}}{\text{kg}}, \quad u_2 = 252.51 \, \frac{\text{kJ}}{\text{kg}}, \quad h_{\text{ein}} = 83.96 \, \frac{\text{kJ}}{\text{kg}}, \quad Q_{R,12} = 35,000 \, \text{kJ}
\]  

\[
\Delta m_{12} = \frac{5755 \, \text{kg} \cdot (418.94 - 252.51) \, \frac{\text{kJ}}{\text{kg}} + 35,000 \, \text{kJ}}{252.51 - 83.96 \, \frac{\text{kJ}}{\text{kg}}}
\]  

Calculations:  
\[
\Delta m_{12} = \frac{3465.44 \, \text{kJ} + 167.43 \, \text{kJ}}{363.58 \, \frac{\text{kJ}}{\text{kg}}}
\]  

\[
\Delta m_{12} = 3636.88 \, \text{kg}
\]