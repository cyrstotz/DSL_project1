**Energy Balance:**  
The energy balance equation is given as:  
\[
0 = \dot{m} \left[ h_{\text{in}} - h_{\text{out}} \right] + \dot{Q}_R - \dot{Q}_{\text{out}}
\]  
Rearranging for \(\dot{Q}_{\text{out}}\):  
\[
\dot{Q}_{\text{out}} = \dot{m} \left[ h_{\text{in}} - h_{\text{out}} \right] + \dot{Q}_R
\]  
From the water tables:  
\[
h_{\text{in}} = h_f(70^\circ\text{C}) = 292.98 \, \frac{\text{kJ}}{\text{kg}} \quad (\text{Table A-2})
\]  
\[
h_{\text{out}} = h_f(100^\circ\text{C}) = 419.04 \, \frac{\text{kJ}}{\text{kg}} \quad (\text{Table A-2})
\]  
Substituting values:  
\[
\dot{Q}_{\text{out}} = 0.3 \, \frac{\text{kg}}{\text{s}} \left( 292.98 \, \frac{\text{kJ}}{\text{kg}} - 419.04 \, \frac{\text{kJ}}{\text{kg}} \right) + 100 \, \text{kW}
\]  
\[
\dot{Q}_{\text{out}} = 62.182 \, \text{kW}
\]  

---

The graph depicts a pressure-temperature (\(P\)-\(T\)) diagram. The curve represents the phase boundary between the liquid and gaseous states. Key points labeled on the graph include:  
- "fuel" near the liquid region,  
- "gas" near the gaseous region,  
- "T_{mix}" marked along the curve, indicating a mixing temperature.  

The horizontal axis is labeled \(T\) (temperature), and the vertical axis is labeled \(P\) (pressure).