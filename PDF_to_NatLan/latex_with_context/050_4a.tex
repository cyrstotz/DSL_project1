Two diagrams are drawn to represent the freeze-drying process in a pressure-temperature (\(p\)-\(T\)) diagram.  

1. The first diagram shows labeled regions:  
   - "Compressed" region on the left.  
   - "ND" (likely indicating the nozzle/dry region) in the center.  
   - "SH" (superheated) region on the right.  
   - A curved line separates the compressed and superheated regions, with a point labeled "St1" (state 1).  
   - An arrow points from the compressed region toward the superheated region.  

2. The second diagram shows a similar layout but includes additional curves and arrows:  
   - A curved line labeled "Compressed" leads into the "ND" region.  
   - Another curved line separates the "ND" region from the "SH" region.  
   - Arrows indicate transitions between states.  

Both diagrams visually describe the freeze-drying process and the transitions between different thermodynamic states.  

---

The diagrams depict the freeze-drying process in a pressure-temperature (\(P\)-\(T\)) diagram.  

1. The first graph shows the compressed gas phase, with a curve labeled "compressed" and "gas phase." The axes are pressure (\(P\)) and temperature (\(T\)).  
2. The second graph shows a phase diagram with labeled regions. The pressure decreases below the sublimation point, and the temperature is held constant. The axes are pressure (\(P\)) and temperature (\(T\)).  
3. The third graph illustrates the refrigeration cycle. It includes labeled states:  
   - State 1: after adiabatic expansion  
   - State 2: isobaric evaporation  
   - State 3: reversible adiabatic compression  
   - State 4: isobaric condensation  
   The axes are pressure (\(P\)) and temperature (\(T\)).