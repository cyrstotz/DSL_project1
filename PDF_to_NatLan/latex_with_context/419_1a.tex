The heat flow \( \dot{Q}_{\text{out}} \) is determined using the energy balance equation:  
\[
\frac{dE}{dt} = \sum \dot{m}_i h_i + \dot{Q} - \sum \dot{m}_e h_e
\]  
where \( \dot{m}_{\text{in}} = \dot{m}_{\text{out}} = 0.3 \, \text{kg/s} \).  

The enthalpy values are calculated as follows:  
\[
h_{\text{in}} = h(70^\circ\text{C}) = 343.47 \, \text{kJ/kg}
\]  
\[
h_{\text{out}} = h(100^\circ\text{C}) = 430.32 \, \text{kJ/kg}
\]  

Using the water tables, the enthalpy at \( 70^\circ\text{C} \) and \( 100^\circ\text{C} \) is determined.  

The heat flow removed by the coolant is calculated as:  
\[
\dot{Q}_{\text{out}} = \dot{m}(h_{\text{out}} - h_{\text{in}}) + \dot{Q}_R
\]  
Substituting the values:  
\[
\dot{Q}_{\text{out}} = 0.3(430.32 - 343.47) + 100 \, \text{kW}
\]  
\[
\dot{Q}_{\text{out}} = 62.286 \, \text{kW}
\]  
\[
\dot{Q}_{\text{out}} = 62.3 \, \text{kW}
\]  

---

The equation for energy balance is written as:  
\[
Q = \dot{m} \cdot (h_b - h_a - T \cdot c)
\]  

No additional explanation or diagrams are present.

The mean temperature \( \bar{T}_{g,2} \) is calculated as:  
\[
\bar{T}_{g,2} = 0.023^\circ \text{C}
\]  

The heat transfer \( Q_{12} \) is considered.  

The differential energy balance is expressed as:  
\[
\frac{d}{dt} \left( u \right) = \dot{m} \cdot \dot{u} + Q - W
\]  

The energy balance for the mass \( m \) is given as:  
\[
m \cdot (u_2 - u_1) = Q
\]  

Substituting values:  
\[
m \cdot c_v \cdot (T_2 - T_1) = Q
\]  
\[
= 3.42 \cdot 10^{-3} \cdot 0.633 \cdot (-273.4741 - 273.15)
\]  
\[
= -2.2657 \, \text{kJ}
\]  
\[
= -2265.8 \, \text{J}
\]