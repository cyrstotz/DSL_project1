The reactor operates with the following given parameters:  
- Mass flow rate at the inlet: \( \dot{m}_{\text{in}} = 0.3 \, \text{kg/s} \)  
- Inlet temperature: \( T_{\text{in}} = 70^\circ\text{C} \)  
- Steam quality: \( x_D = 0.005 \)  

To calculate the heat flow \( \dot{Q}_{\text{out}} \), the energy balance equation is applied:  
\[
0 = \dot{m}_{\text{in}} h_{\text{in}} - \dot{m}_{\text{out}} h_{\text{out}} + \dot{Q}_R - \dot{Q}_{\text{out}} - \dot{W}
\]  
Since there is no work (\( \dot{W} = 0 \)), the equation simplifies to:  
\[
\dot{Q}_{\text{out}} = \dot{m} (h_{\text{in}} - h_{\text{out}}) + \dot{Q}_R
\]  

Using water tables (Table A-2), the enthalpy values are:  
- \( h_{\text{in}} = h_f(70^\circ\text{C}) = 292.88 \, \text{kJ/kg} \)  
- \( h_{\text{out}} = h_f(100^\circ\text{C}) = 419.04 \, \text{kJ/kg} \)  

The mass flow rate is constant: \( \dot{m} = 0.3 \, \text{kg/s} \).  
The heat released by the chemical reaction is \( \dot{Q}_R = 100 \, \text{kW} \).  

Substituting these values:  
\[
\dot{Q}_{\text{out}} = \dot{m} (h_{\text{in}} - h_{\text{out}}) + \dot{Q}_R
\]  
\[
\dot{Q}_{\text{out}} = 0.3 \, \text{kg/s} \cdot (292.88 - 419.04) \, \text{kJ/kg} + 100 \, \text{kW}
\]  
\[
\dot{Q}_{\text{out}} = 62.182 \, \text{kW}
\]