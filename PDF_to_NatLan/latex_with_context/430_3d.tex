The calculations begin with determining the specific volume \( V_{g2} \) using the equation:  
\[
V_{g2} = \frac{mRT}{p} = \frac{3.429 \times 10^{-3} \, \text{kg} \cdot 0.16628 \, \frac{\text{m}^3}{\text{kg}} \cdot 273.15 \, \text{K}}{1.4 \times 10^2 \, \frac{\text{N}}{\text{m}^2}}
\]  
This results in:  
\[
V_{g2} = 1.103 \times 10^{-3} \, \text{m}^3
\]  

Next, the specific volume per unit mass is calculated:  
\[
v_{g2} = \frac{V_{g2}}{m} = \frac{1.103 \times 10^{-3} \, \text{m}^3}{3.429 \times 10^{-3} \, \text{kg}} = 0.324364 \, \frac{\text{m}^3}{\text{kg}}
\]  

The work done \( W_{12} \) is calculated using the integral of \( p \, dv \):  
\[
W_{12} = \int_{1}^{2} p \, dv = R T \cdot p (v_2 - v_1) = 1.4 \times 10^2 \, \frac{\text{N}}{\text{m}^2} \cdot (0.324364 \, \frac{\text{m}^3}{\text{kg}} - 0.948837 \, \frac{\text{m}^3}{\text{kg}})
\]  
This gives:  
\[
W_{12} = -83.164 \, \frac{\text{J}}{\text{kg}}
\]  

The total work \( W_{12} \cdot m \) is then calculated:  
\[
W_{12} \cdot m = -83.164 \, \frac{\text{J}}{\text{kg}} \cdot 3.429 \times 10^{-3} \, \text{kg} = -0.28434 \, \text{J}
\]  

The change in internal energy \( U_2 - U_1 \) is calculated using the equation:  
\[
U_2 - U_1 = m c_V (T_2 - T_1) = 0.683 \, \frac{\text{J}}{\text{kg·K}} \cdot (273.15 \, \text{K} - 773.15 \, \text{K}) \cdot 3.429 \times 10^{-3} \, \text{kg}
\]  
This results in:  
\[
U_2 - U_1 = -710.827 \, \text{J}
\]  

The heat transfer \( Q_{12} \) is calculated using the energy balance:  
\[
Q_{12} = U_2 - U_1 + W_{12} = -710.827 \, \text{J} + (-0.28434 \, \text{J}) = -7366.4 \, \text{J}
\]  

The absolute value of \( Q_{12} \) is noted:  
\[
|Q_{12}| = 7366.4 \, \text{J}
\]  

No diagrams or figures are present on this page.