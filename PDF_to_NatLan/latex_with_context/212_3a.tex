The problem involves determining the gas pressure \( p_{g,1} \) and mass \( m_g \) in state 1.  

The ideal gas law is applied:  
\[
p \cdot V = m \cdot R \cdot T
\]  
Rearranging for \( m_g \):  
\[
m_g = \frac{p \cdot V}{R \cdot T}
\]  
Given values:  
- \( R = \frac{R}{M} = \frac{166.28 \, \text{J/(kg·K)}}{0.050 \, \text{kg/mol}} \)  
- \( T = 738.15 \, \text{K} \)  
- \( V = 0.00314 \, \text{L} \)  

The calculated mass is:  
\[
m_g = 2.687 \, \text{g}
\]  

The pressure \( p_g \) is determined using the equation:  
\[
p_g \cdot A = p_0 \cdot A + m_g \cdot g
\]  
Where:  
- \( A = (0.1 \, \text{m})^2 \cdot \pi \)  
- \( p_g = 1.100 \, \text{bar} \).  

---