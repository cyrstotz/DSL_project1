The task involves determining the gas pressure \( p_{g,1} \) in equilibrium. A diagram is drawn showing a rectangular cylinder with arrows indicating forces acting on the piston: atmospheric pressure \( p_0 \), the weight of the piston \( m_K g \), and the weight of the ice-water mixture \( m_{\text{EW}} g \).  

The equilibrium pressure is calculated as:  
\[
p_{g,1} = p_0 + \frac{m_{\text{EW}} g}{A} + \frac{m_K g}{A}
\]  
Where:  
- \( p_0 = 10^5 \, \text{Pa} \)  
- \( m_{\text{EW}} = 0.14 \, \text{kg} \)  
- \( g = 9.81 \, \text{m/s}^2 \)  
- \( m_K = 32 \, \text{kg} \)  
- \( A = \pi r^2 = \pi (0.05 \, \text{m})^2 = 0.00785 \, \text{m}^2 \)  

Substituting values:  
\[
p_{g,1} = 10^5 + \frac{0.14 \cdot 9.81}{0.00785} + \frac{32 \cdot 9.81}{0.00785}
\]  
\[
p_{g,1} = 140144.78 \, \text{Pa}
\]  

The student notes to "calculate with other units where necessary."  

---

Next, the mass of the gas \( m_g \) is determined using the ideal gas law:  
\[
p V = m R T
\]  
Rearranging:  
\[
m_g = \frac{p_{g,1} V_{g,1}}{R T_{g,1}}
\]  

The gas constant \( R \) is calculated as:  
\[
R = \frac{R_u}{M} = \frac{8.314 \, \text{J/mol·K}}{50 \, \text{kg/kmol}} = 166.28 \, \text{J/kg·K}
\]  

Substituting values:  
\[
m_g = \frac{140144.78 \cdot 3.14 \cdot 10^{-3}}{166.28 \cdot (273.15 + 500)}
\]  
\[
m_g = 3.422 \, \text{g}
\]  

The final result for the gas mass is boxed:  
\[
m_g = 3.422 \, \text{g}
\]  

No additional diagrams or figures are present beyond the initial sketch of the cylinder.