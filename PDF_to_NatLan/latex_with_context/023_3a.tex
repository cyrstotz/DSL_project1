The problem begins by determining the gas pressure \( p_{g,1} \) and mass \( m_g \) in state 1.  

The cross-sectional area of the cylinder is calculated as:  
\[
A = \left(\frac{D}{2}\right)^2 \pi = 0.00785 \, \text{m}^2
\]  

The pressure is given as:  
\[
p = \frac{F}{A}
\]  
where \( F \) is the force exerted by the piston and atmospheric pressure.  

The force is calculated as:  
\[
F = (m_K \cdot g) + (p_{\text{amb}} \cdot A) = 3491.9 \, \text{N}
\]  

The gas pressure is then determined:  
\[
p_{g,1} = \frac{F}{A} + p_{\text{amb}} = 140114.8 \, \text{Pa} = 1.40114 \, \text{bar}
\]  

The gas mass is calculated using the ideal gas law:  
\[
m = \frac{p_{g,1} V_{g,1}}{R T_1}
\]  
where \( R \) is the specific gas constant:  
\[
R = \frac{\bar{R}}{M} = \frac{8.314 \, \text{kJ/kmol·K}}{50 \, \text{kg/kmol}} = 0.16628 \, \text{kJ/kg·K}
\]  

Substituting the values, the gas mass is:  
\[
m = 3.49 \, \text{g}
\]  

---