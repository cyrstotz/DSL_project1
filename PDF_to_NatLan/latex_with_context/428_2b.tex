The task involves determining the outlet velocity \( w_6 \) and temperature \( T_6 \).  

**Text and calculations:**  
The gas is modeled as an ideal gas with the following properties:  
- \( c_{p,\text{air}} = 1.006 \, \text{kJ/kg·K} \)  
- \( \kappa = 1.4 \)  

The ambient pressure \( p_0 \) is given as \( 0.191 \, \text{bar} \).  

For the process from state 5 to state 6, it is stated that the process is reversible and adiabatic, implying isentropic behavior. Using the isentropic relation:  
\[
\frac{T_6}{T_5} = \left( \frac{p_6}{p_5} \right)^{\frac{\kappa - 1}{\kappa}}
\]  

Substituting values:  
\[
\frac{T_6}{T_5} = \left( \frac{0.191 \, \text{bar}}{0.5 \, \text{bar}} \right)^{\frac{1.4 - 1}{1.4}}
\]  
\[
T_6 = 431.9 \, \text{K} \times \left( \frac{0.191}{0.5} \right)^{0.286}
\]  
\[
T_6 = 328.074 \, \text{K}
\]  

For \( w_6 \), the velocity can be calculated using the energy equation, but no explicit calculation is shown here.  

Additional notes:  
The specific gas constant \( R \) is calculated as:  
\[
R = c_{p,\text{air}} - c_{v,\text{air}}
\]  
\[
R = 1.006 \, \text{kJ/kg·K} - 0.7186 \, \text{kJ/kg·K} = 0.2874 \, \text{kJ/kg·K}
\]  

No further calculations for \( w_6 \) are visible.

The pressure \( p_0 \) is calculated as:  
\[
p_0 = \frac{\dot{m} \cdot R \cdot T_0}{A_0 \cdot w_0}
\]  
Substituting values:  
\[
p_0 = \frac{0.158 \, \text{kg/s} \cdot 0.2874 \, \text{kJ/kg·K} \cdot 243.15 \, \text{K}}{1 \, \text{m}^2 \cdot 2.7332 \, \text{m/s}}
\]  
\[
p_0 = 2.7332 \, \text{bar}
\]  

Similarly, the pressure \( p_6 \) is calculated as:  
\[
p_6 = \frac{\dot{m} \cdot R \cdot T_6}{A_6 \cdot w_6}
\]  
Substituting values:  
\[
p_6 = \frac{0.158 \, \text{kg/s} \cdot 0.2874 \, \text{kJ/kg·K} \cdot 293.15 \, \text{K}}{1 \, \text{m}^2 \cdot 2.0217 \, \text{m/s}}
\]  
\[
p_6 = 2.0217 \, \text{bar}
\]