The vapor quality \( x_1 \) of the refrigerant at state 1 after expansion is determined.  

Given:  
\[
x_1 = 0
\]  
At state 4:  
\[
p_4 = 8 \, \text{bar}
\]  
The throttling process is isenthalpic and adiabatic, with no work done:  
\[
h_4 = h_1
\]  

From the refrigerant table:  
\[
h_f(p_{\text{bar}}) = 93.42 \, \frac{\text{kJ}}{\text{kg}}
\]  

Using the enthalpy relation:  
\[
h_1 = h_f + x(h_g - h_f)
\]  
At \( 1 \, \text{bar} \), \( T_{\text{bar}} \), and \( p_{\text{amb}} \), the enthalpy values are referenced from the refrigerant table.

The vapor quality \( x_1 \) is calculated under the assumption that the throttling process is isenthalpic, with no work done and adiabatic conditions.  

The pressure at state 1 is \( p_1 = p_{\text{bar}} \).  
The enthalpy at state 1 is given as:  
\[
h_1 = h_u = h_f(p_{\text{bar}}) = 93.42 \, \text{kJ/kg}
\]  
The enthalpy at state 1 can also be expressed as:  
\[
h_1 = h_f + x(h_g - h_f) \quad \text{at } T_1
\]  
Solving for \( x \):  
\[
x = \frac{h_1 - h_f}{h_g - h_f} = \frac{93.42 \, \text{kJ/kg} - 27.72 \, \text{kJ/kg}}{234.08 \, \text{kJ/kg} - 27.72 \, \text{kJ/kg}}
\]  
\[
x = 0.337
\]