The gas pressure \( p_{g,1} \) and mass \( m_g \) in state 1 are calculated as follows:  

The perfect gas properties are given:  
\[
M_g = 50 \, \text{kg/kmol}, \quad c_V = 0.633 \, \text{kJ/kg·K}, \quad V_{g,1} = 0.00514 \, \text{m}^3
\]  
The cylinder diameter is \( D = 0.1 \, \text{m} \), and the area is calculated as:  
\[
A = \frac{\pi D^2}{4} = 0.00785 \, \text{m}^2
\]  
The volume is:  
\[
V = D \cdot A = 0.00514 \, \text{m}^3
\]  
The gas pressure is determined using the ideal gas law:  
\[
p_{g,1} = \frac{m_g R T_{g,1}}{V_{g,1}}
\]  
where \( R = \frac{8.314}{M_g} = 166.28 \, \text{J/kg·K} \).  

The force exerted by the piston is:  
\[
F = m_K \cdot g = 32.1 \, \text{kg} \cdot 9.81 \, \text{m/s}^2 = 315 \, \text{N}
\]  
The pressure due to the piston is:  
\[
p = \frac{F}{A} = \frac{315}{0.00785} = 1.1 \, \text{bar}
\]  
Thus, \( p_{g,1} = 1 \, \text{bar} + 1.1 \, \text{bar} = 2.1 \, \text{bar} \).  

The gas mass is calculated as:  
\[
m_g = \frac{p_{g,1} V_{g,1}}{R T_{g,1}} = \frac{2.1 \cdot 0.00514}{166.28 \cdot 500} = 0.002687 \, \text{kg}
\]  

---