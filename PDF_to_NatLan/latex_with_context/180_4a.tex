A pressure-temperature (\(p\)-\(T\)) diagram is drawn. The diagram includes the following features:  
- A curve labeled "flüssig" (liquid) representing the boundary between liquid and two-phase regions.  
- A region labeled "2-phasen" (two-phase) below the curve.  
- A vertical line labeled "sublim" (sublimation) intersecting the curve.  
- A horizontal line labeled "tripel" (triple point) crossing the vertical sublimation line.  
- The axes are labeled \(p\) (pressure) on the vertical axis and \(T\) (temperature) on the horizontal axis.  
- A point labeled "i" is marked within the diagram.