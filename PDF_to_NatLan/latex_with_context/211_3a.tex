The cross-sectional area of the cylinder is calculated as:  
\[
A = \pi \left(\frac{D}{2}\right)^2 = \pi \left(\frac{0.1 \, \text{m}}{2}\right)^2 = 7.854 \times 10^{-3} \, \text{m}^2
\]  

The forces acting on the system are:  
\[
F_{\text{piston}} = m_K \cdot g = 32 \, \text{kg} \cdot 9.81 \, \text{m/s}^2 = 313.92 \, \text{N}
\]  
\[
F_{\text{EW}} = m_{\text{EW}} \cdot g = 0.1 \, \text{kg} \cdot 9.81 \, \text{m/s}^2 = 0.981 \, \text{N}
\]  

The pressure is calculated as:  
\[
P = P_{\text{amb}} + \frac{F_{\text{piston}}}{A} + \frac{F_{\text{EW}}}{A} = 1.401 \, \text{bar}
\]  

The mass of the gas is determined using the ideal gas law:  
\[
\rho V = m_g R T \implies m_g = \frac{\rho V}{RT} = \frac{M_g \cdot P \cdot V_g}{R \cdot T_g}
\]  
Substituting values:  
\[
m_g = \frac{50 \, \text{kg/kmol} \cdot 1.401 \times 10^5 \, \text{Pa} \cdot 3.14 \times 10^{-3} \, \text{m}^3}{8.314 \, \text{J/(mol·K)} \cdot 773.15 \, \text{K}} = 3.42 \, \text{kg}
\]  

---