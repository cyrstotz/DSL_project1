The equation for the outlet velocity \( w_6 \) is derived using the energy balance:  
\[
2(h_1 - h_5) - w_5^2 = -w_6^2
\]  
Substituting values:  
\[
-208 + \frac{220^2}{2} = -w_6^2
\]  
\[
w_6 = 481.6 \, \text{m/s}
\]  

---

The steady-state energy balance for the nozzle is written as:  
\[
w_6 = \sqrt{w_5^2 + 2(h_5 - h_6)}
\]  
where \( w_6 \) is the outlet velocity, \( w_5 \) is the inlet velocity, and \( h_5 \) and \( h_6 \) are the specific enthalpies at states 5 and 6, respectively.  

The reversible adiabatic temperature relation is given as:  
\[
\frac{T_G}{T_S} = \left( \frac{p_G}{p_S} \right)^{\frac{\kappa - 1}{\kappa}}
\]  
where \( \kappa = 1.4 \), \( T_G \) is the gas temperature, \( T_S \) is the stagnation temperature, \( p_G \) is the gas pressure, and \( p_S \) is the stagnation pressure.  

The temperature \( T_6 \) is calculated as:  
\[
T_6 = T_S \cdot \left( \frac{p_G}{p_S} \right)^{\frac{\kappa - 1}{\kappa}}
\]  
Substituting values:  
\[
T_6 = 431.5 \cdot \left( \frac{p_G}{p_S} \right)^{\frac{1.4 - 1}{1.4}} = 320 \, \text{K}
\]  

The enthalpy difference \( h_5 - h_6 \) is calculated using the specific heat capacity \( c_p \):  
\[
h_5 - h_6 = c_p (T_5 - T_6)
\]  
Substituting values:  
\[
h_5 - h_6 = 1.006 \cdot (431.5 - 320) = 104 \, \text{kJ/kg}
\]