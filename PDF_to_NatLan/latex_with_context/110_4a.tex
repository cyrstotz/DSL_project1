The diagram is a pressure-temperature (\( p \)-\( T \)) graph illustrating the freeze-drying process. It includes the following features:  
- The \( p \)-axis is labeled with pressure in bar, ranging from approximately 0.3 bar to 8 bar.  
- The \( T \)-axis is labeled with temperature in Kelvin (\( K \)), starting at 0 and increasing to \( T_i \), which is marked as constant.  
- Three regions are labeled: "gas" at the top right, "liquid" in the middle, and "solid" at the bottom left.  
- A curve separates the "solid" and "liquid" regions, and another curve separates the "liquid" and "gas" regions.  
- The triple point is marked where the three phases meet.  

Additional notes are written below the graph:  
- \( T_i = \text{sublimation temperature} + 10 \, \text{K} \).  
- \( x = 0 \): liquid = vapor.  
- \( x = 1 \): \( p \) and \( T \) are solid?  

The graph visually represents the phase transitions and conditions relevant to the freeze-drying process.