The diagram is a pressure-temperature (\(P\)-\(T\)) graph. It shows the freeze-drying process with labeled states.  
- The vertical axis represents pressure in bar (\([ \text{bar} ]\)).  
- The horizontal axis represents temperature in Kelvin (\([ \text{K} ]\)).  
- The curve depicts the phase boundaries.  
- States \(1\), \(2\), \(3\), and \(4\) are marked along the process path.  
- The process transitions from state \(1\) to \(2\), then to \(3\), and finally to \(4\).  
- State \(3\) is labeled at 8 bar, indicating the pressure after compression.

The coefficient of performance \( \epsilon_K \) is defined as:  
\[
\epsilon_K = \frac{\dot{Q}_{\text{zu}}}{\dot{W}_K} = \frac{\dot{Q}_{\text{zu}}}{\dot{Q}_{\text{ab}} - \dot{Q}_{\text{zu}}}
\]  
This equation relates the heat input \( \dot{Q}_{\text{zu}} \) to the work \( \dot{W}_K \) and the difference between the heat rejected \( \dot{Q}_{\text{ab}} \) and the heat input.