The energy balance equation is given as:  
\[
\Delta E = Q - W
\]  

The work \( W \) is calculated as:  
\[
W = p_2 V_2 - p_1 V_1
\]  

The volume \( V \) is expressed as:  
\[
V = p_2 V_2 - p_1 V_1
\]  

Using the ideal gas law:  
\[
p_2 V_2 = m_g R_g T_{g2}
\]  

The volume \( V_2 \) is given as:  
\[
V_2 = 0.0011 \, \text{m}^3
\]  

The work \( W \) is then calculated:  
\[
W = p(V_2 - V_1)
\]  
\[
W = -254.3 \, \text{J}
\]  

The change in energy \( \Delta E \) is expressed as:  
\[
\Delta E = m(u_2 - u_1)
\]  

For a perfect gas, the change in internal energy \( u_2 - u_1 \) is given by:  
\[
u_2 - u_1 = c_V (T_2 - T_1)
\]  

Substituting into the energy balance:  
\[
m c_V (T_2 - T_1) = Q - W
\]  

Rearranging for \( Q \):  
\[
Q = m c_V (T_2 - T_1) + W
\]  

The specific heat capacity \( c_V \) is:  
\[
c_V = 633 \, \text{J/kg·K}
\]  

The temperatures are:  
\[
T_2 = 0^\circ\text{C} = 273.15 \, \text{K}, \quad T_1 = 500^\circ\text{C}
\]  

Finally, the heat \( Q \) is calculated as:  
\[
Q = -1867 \, \text{J}
\]  

This result is underlined in the solution.