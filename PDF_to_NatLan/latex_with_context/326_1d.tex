The mass \( \Delta m_{12} \) required to reduce the reactor temperature from \( 100^\circ\text{C} \) to \( 70^\circ\text{C} \) is calculated using an energy balance:  
\[
U_{\text{Reaktor}} = m_g U_r = 5755 \, \text{kg} \left( (1-x) U_f + x U_g \right)
\]  
From water tables:  
\[
U_f = 418.94 \, \frac{\text{kJ}}{\text{kg}}, \quad U_g = 2506.5 \, \frac{\text{kJ}}{\text{kg}}
\]  
At \( 70^\circ\text{C} \):  
\[
U_r(70^\circ\text{C}) = m_g U_f(70^\circ\text{C}) = 292.35 \, \frac{\text{kJ}}{\text{kg}} \cdot 5755 \, \text{kg} = 1.65 \cdot 10^6 \, \text{kJ}
\]  
The heat released:  
\[
-28.37 \cdot 10^6 \, \text{kJ} \text{ in the form of added water.}
\]

The energy balance is used to calculate the mass \( \Delta m \) of saturated liquid water added to the reactor during cooling. The specific internal energy difference is calculated as:  
\[
\Delta U_w = U_f(70^\circ\text{C}) - U_f(20^\circ\text{C}) = -83 \, \text{kJ/kg} + 252.95 \, \text{kJ/kg}
\]  
The heat released during cooling is given as \( Q_{R,12} = 35 \, \text{MJ} \). Using the energy balance:  
\[
\Delta m \cdot \Delta U_w + 35 \, \text{kJ} = 28.31 \cdot 10^6 \, \text{kJ}
\]  
Rearranging for \( \Delta m \):  
\[
\Delta m = \frac{28.31 \cdot 10^6 \, \text{kJ} - 35 \cdot 10^3 \, \text{kJ}}{\Delta U_w}
\]  
Substituting \( \Delta U_w = 208 \, \text{kJ/kg} \):  
\[
\Delta m = \frac{28.31 \cdot 10^6 \, \text{kJ} - 35 \cdot 10^3 \, \text{kJ}}{208 \, \text{kJ/kg}}
\]  
\[
\Delta m = 135.29 \cdot 10^3 \, \text{kg}
\]