The first law of thermodynamics is applied to the system, represented by the equation:  
\[
0 = \dot{m}_{\text{R134a}} \left[ h_1 - h_2 \right] + \dot{Q}_K
\]  
where \( h_1 \) and \( h_2 \) are enthalpies at states 1 and 2, respectively.  

The enthalpy at state 1 is calculated as:  
\[
h_1 = -\frac{\dot{Q}_K}{\dot{m}_{\text{R134a}}} + h_2
\]  

The pressure at state 1 is given as \( p_1 = p_2 = 3.3756 \, \text{bar} \), referencing Table A10 for R134a properties.  

The vapor quality \( x_1 \) at state 1 is determined using the enthalpy relation:  
\[
h_1 = h_f + x_1 \left( h_g - h_f \right)
\]  
Solving for \( x_1 \):  
\[
x_1 = \frac{h_1 - h_f}{h_g - h_f}
\]

The entropy at state 2 is equal to the entropy at state 3:  
\[
s_2 = s_3
\]  
From the saturated liquid data:  
\[
s_3 = s_f + x_3 (s_g - s_f)
\]  
The vapor quality at state 3 is calculated as:  
\[
x_3 = \frac{s_3 - s_f}{s_g - s_f}
\]  

The enthalpy at state 3 is given by:  
\[
h_3 = h_f + x_3 (h_g - h_f)
\]