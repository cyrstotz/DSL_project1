The pressure \( p_{g,1} \) is calculated as the sum of the ambient pressure \( p_{\text{amb}} \) and the pressure exerted by the piston \( p_{\text{piston}} \):  
\[
p_{g,1} = p_{\text{amb}} + p_{\text{piston}}
\]  
The piston pressure is determined using the formula:  
\[
p_{\text{piston}} = \frac{F}{A} = \frac{m_K \cdot g}{\pi \left(\frac{D}{2}\right)^2}
\]  
Substituting the values:  
\[
p_{\text{piston}} = \frac{32 \, \text{kg} \cdot 9.81 \, \text{m/s}^2}{\pi \left(\frac{10 \, \text{cm}}{2}\right)^2} = 0.900 \, \text{bar}
\]  
Thus:  
\[
p_{g,1} = 1.0 \, \text{bar} + 0.9 \, \text{bar} = 1.9 \, \text{bar}
\]  

The ideal gas law is used to calculate the gas mass \( m_g \):  
\[
p_{g,1} \cdot V_1 = m_g \cdot R \cdot T_1
\]  
The specific gas constant \( R \) is calculated as:  
\[
R = \frac{\bar{R}}{M} = \frac{8.314 \, \text{J/mol·K}}{50 \, \text{kg/kmol}} = 166.28 \, \text{J/kg·K}
\]  
Rearranging for \( m_g \):  
\[
m_g = \frac{p_{g,1} \cdot V_1}{R \cdot T_1} = \frac{1.9 \, \text{bar} \cdot 0.00372 \, \text{m}^3}{166.28 \, \text{J/kg·K} \cdot 773.15 \, \text{K}}
\]  
Converting \( \text{bar} \) to \( \text{Pa} \):  
\[
m_g = \frac{1.9 \cdot 10^5 \, \text{Pa} \cdot 0.00372 \, \text{m}^3}{166.28 \, \text{J/kg·K} \cdot 773.15 \, \text{K}} = 0.00392 \, \text{kg} = 3.92 \, \text{g}
\]  

---