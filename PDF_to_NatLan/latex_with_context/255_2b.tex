The process from state 5 to state 6 is described as isentropic. The temperature at state 6, \( T_6 \), is calculated using the following formula:  
\[
T_6 = T_5 \left( \frac{p_6}{p_5} \right)^{\frac{\kappa - 1}{\kappa}} = 328.02797 \, \text{K}
\]

The energy balance for the system is given by the steady-state form of the first law of thermodynamics:  
\[
0 = \dot{m} \left( h_5 - h_6 + \frac{w_5^2 - w_6^2}{2} \right) + \dot{Q} - \dot{W}
\]  
Since heat transfer (\( \dot{Q} \)) and work (\( \dot{W} \)) are zero, the equation simplifies to:  
\[
2(h_6 - h_5) = w_5^2 - w_6^2
\]  
Rearranging for \( w_6^2 \):  
\[
w_6^2 = w_5^2 - 2(h_6 - h_5)
\]  
Substituting values:  
\[
w_6 = 220 \, \text{m/s}
\]

The enthalpy difference \( h_6 - h_5 \) is calculated using:  
\[
h_6 - h_5 = c_p (T_6 - T_5) = -106.45 \, \text{kJ/kg}
\]

---