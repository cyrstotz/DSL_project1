The pressure and mass of the gas in the cylinder are calculated using the following initial conditions:  
\[
T_{g,1} = 500^\circ\text{C}, \quad V_{g,1} = 3.14 \, \text{L}, \quad m_{\text{EW}} = 0.1 \, \text{kg}, \quad T_{\text{EW},1} = 0^\circ\text{C}, \quad x_{\text{ice},1} = 0.6
\]  

The force exerted by the piston is derived as follows:  
\[
F = F_{\text{gas}}
\]  
The piston is in equilibrium, so:  
\[
F = p_{\text{gas}} \cdot A
\]  
The area of the piston is calculated as:  
\[
A = \pi \cdot \left( \frac{D}{2} \right)^2 = \pi \cdot \left( \frac{10 \, \text{cm}}{2} \right)^2 = 7.8539 \cdot 10^{-3} \, \text{m}^2
\]  

The pressure of the gas is determined using the equilibrium condition:  
\[
p_{\text{gas}} = \frac{F}{A}
\]  
Substituting values:  
\[
F = (m_{\text{EW}} + m_K) \cdot g = (0.1 \, \text{kg} + 32 \, \text{kg}) \cdot 9.81 \, \text{m/s}^2 = 314.3 \, \text{N}
\]  
\[
p_{\text{gas}} = \frac{314.3 \, \text{N}}{7.8539 \cdot 10^{-3} \, \text{m}^2} = 4.000 \cdot 10^4 \, \text{Pa} = 4.000 \, \text{bar}
\]