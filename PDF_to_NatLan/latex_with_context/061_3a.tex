To determine the gas pressure \( p_{g,1} \) and mass \( m_g \) in state 1:  

The pressure \( p \) is calculated as:  
\[
p = p_{\text{amb}} + \frac{m_K \cdot g}{A} + \frac{m_{\text{EW}} \cdot g}{A}
\]  
where \( p_{\text{amb}} \) is the ambient pressure, \( m_K \) is the piston mass, \( m_{\text{EW}} \) is the mass of the ice-water mixture, \( g \) is the gravitational acceleration, and \( A \) is the cross-sectional area of the cylinder.  

Substituting the values:  
\[
p = 1 \, \text{bar} + \frac{(32 \, \text{kg} + 0.1 \, \text{kg}) \cdot 9.81}{\pi \cdot (0.05 \, \text{m})^2}
\]  
\[
p = 1 \, \text{bar} + \frac{314.907 \, \text{N}}{\pi \cdot 0.0025 \, \text{m}^2} = 1 \, \text{bar} + 40,394.44 \, \text{Pa}
\]  
\[
p = 0.401 \, \text{bar} \quad \Rightarrow \quad p_{g,1} = 7.407 \, \text{bar}
\]  

Next, the mass \( m_g \) is calculated using the ideal gas law:  
\[
p \cdot V = m \cdot R \cdot T
\]  
Rearranging for \( m_g \):  
\[
m_g = \frac{p \cdot V}{R \cdot T}
\]  

The specific gas constant \( R \) is given as:  
\[
R = \frac{\bar{R}}{M} = \frac{8.314}{50} = 0.16628 \, \text{kJ/kg·K}
\]  

Substituting the values:  
\[
m_g = \frac{7.407 \, \text{bar} \cdot 3.14 \, \text{L}}{0.16628 \cdot 773.15 \, \text{K}}
\]  
\[
m_g = \frac{7.407 \cdot 3.14}{0.16628 \cdot 773.15} = 3.42 \, \text{g}
\]  

---