Two diagrams are drawn to represent the freeze-drying process in a pressure-temperature (\( p \)-\( T \)) diagram.  

1. The first diagram shows phase regions with labeled points \( 1 \), \( 2 \), \( 3 \), and \( 4 \).  
   - Point \( 1 \) is near the triple point.  
   - Point \( 2 \) is horizontally aligned with \( 1 \), indicating isobaric evaporation.  
   - Point \( 3 \) is at a higher pressure, representing compression.  
   - Point \( 4 \) is horizontally aligned with \( 3 \), indicating isobaric condensation.  
   - The diagram includes shaded regions to indicate phase boundaries.  

2. The second diagram is simpler, showing the same points \( 1 \), \( 2 \), \( 3 \), and \( 4 \) connected in a rectangular cycle. The triple point is marked.  

The temperature \( T_i = 5^\circ\text{C} \) is noted.