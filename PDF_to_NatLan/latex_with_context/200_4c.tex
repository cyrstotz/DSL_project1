The process is illustrated with a graph showing the enthalpy \( h \) versus pressure \( p \).  
- At state 3: \( p = 8 \, \text{bar} \), \( h = 277.37 \, \frac{\text{kJ}}{\text{kg}} \).  
- At state 1: \( p = 1.5748 \, \text{bar} \), \( h = 93.42 \, \frac{\text{kJ}}{\text{kg}} \).  

The enthalpy at state 1 is determined as \( h_f(8 \, \text{bar}) = 93.42 \, \frac{\text{kJ}}{\text{kg}} \). The process is isenthalpic due to the throttling.

Using Table A-10 for \( p = 1.5748 \, \text{bar} \), the vapor quality \( x_1 \) is calculated:  
\[
x_1 = \frac{h_1 - h_f}{h_g - h_f} = \frac{93.42 - h_f}{h_g - h_f} = 0.308
\]  
This corresponds to a vapor quality of \( x_1 = 0.308 \).