Determine the gas pressure \( p_{g,1} \) and mass \( m_g \) in the cylinder:  

The specific heat capacity at constant volume is given as:  
\[
C_v = 0.633 \, \frac{\text{kJ}}{\text{kg·K}}
\]  

The gas pressure \( p_{g,1} \) is calculated using the equation:  
\[
p_{g,1} = p_{\text{amb}} + \left( \frac{m_K g}{A_{\text{cyl}}} \right) + \left( \frac{m_{\text{EW}} g}{A_{\text{cyl}}} \right)
\]  
Substituting values:  
\[
p_{g,1} = 1.05 \, \text{bar} + \frac{32 \, \text{kg} \cdot 9.81 \, \text{m/s}^2}{0.00785 \, \text{m}^2} + \frac{0.1 \, \text{kg} \cdot 9.81 \, \text{m/s}^2}{0.00785 \, \text{m}^2}
\]  
\[
p_{g,1} = 1.05 \, \text{bar} + 39,983.8 \, \text{Pa} + 125 \, \text{Pa}
\]  
\[
p_{g,1} = 1.1 \, \text{bar}
\]  

The cross-sectional area of the cylinder is calculated as:  
\[
A_{\text{cyl}} = \left( \frac{D}{2} \right)^2 \pi
\]  
\[
A_{\text{cyl}} = \left( \frac{0.1 \, \text{m}}{2} \right)^2 \pi = 0.05^2 \pi \, \text{m}^2
\]  

The gas mass \( m_g \) is determined using the ideal gas law:  
\[
p_{g,1} V_{g,1} = m_g R_g T_{g,1}
\]  
Rearranging:  
\[
m_g = \frac{p_{g,1} V_{g,1}}{R_g T_{g,1}}
\]  
Substituting values:  
\[
R_g = \frac{8.314 \, \text{J/mol·K}}{50 \, \text{kg/kmol}} = 166.28 \, \frac{\text{J}}{\text{kg·K}}
\]  
\[
m_g = \frac{1.46 \, \text{bar} \cdot 3.14 \cdot 10^{-3} \, \text{m}^3}{166.28 \, \frac{\text{J}}{\text{kg·K}} \cdot 773 \, \text{K}}
\]  
\[
m_g = 0.00342 \, \text{kg} = 3.42 \, \text{g}
\]  

The volume is converted as:  
\[
3.14 \, \text{L} = 3.14 \, \text{dm}^3 = 3.14 \cdot 10^{-3} \, \text{m}^3
\]  

The temperature is converted as:  
\[
500^\circ\text{C} = 773 \, \text{K}
\]  

---