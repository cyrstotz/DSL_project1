The gas pressure \( p_g \) is calculated as the sum of the ambient pressure \( p_{\text{amb}} \) and the pressure exerted by the piston \( p_w \):  
\[
p_g = p_{\text{amb}} + p_w
\]  
The pressure exerted by the piston is determined using the formula:  
\[
p_w = \frac{m_K g}{A}
\]  
where \( m_K = 32 \, \text{kg} \), \( g = 9.81 \, \text{m/s}^2 \), and \( A = \frac{\pi}{4} D^2 \). Substituting the values:  
\[
p_w = \frac{32 \cdot 9.81}{\frac{\pi}{4} \cdot (0.1)^2} = 39953.52 \, \text{Pa}
\]  
Thus:  
\[
p_g = 1 \, \text{bar} + 39953.52 \, \text{Pa} = 1.1 \, \text{bar}
\]  

The gas mass \( m_g \) is calculated using the ideal gas law:  
\[
m_g = \frac{p_g V_{g,1}}{R T_{g,1}}
\]  
Substituting \( p_g = 1.1 \, \text{bar} \), \( V_{g,1} = 3.14 \, \text{L} = 3.14 \cdot 10^{-3} \, \text{m}^3 \), \( R = 0.82 \, \text{kJ/kg·K} \), and \( T_{g,1} = 500^\circ\text{C} = 773.15 \, \text{K} \):  
\[
m_g = \frac{1.1 \cdot 10^5 \cdot 3.14 \cdot 10^{-3}}{0.82 \cdot 773.15} = 3.4 \, \text{g}
\]  

The specific gas constant \( R_g \) is calculated as:  
\[
R_g = \frac{R}{M_g} = \frac{0.82}{50} = 0.0164 \, \text{kJ/kg·K}
\]  

---