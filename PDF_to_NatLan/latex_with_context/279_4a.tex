Two diagrams are drawn to represent the freeze-drying process in a pressure-temperature (\(P\)-\(T\)) diagram.  

1. **First Diagram**:  
   - The diagram shows phase regions labeled as "Eis" (ice), "Wasser" (water), and "Dampf" (vapor).  
   - The triple point is marked as "Tripelpunkt."  
   - A cycle is illustrated with four states labeled \(1\), \(2\), \(3\), and \(4\).  
   - The cycle transitions through the ice, water, and vapor regions, with arrows indicating the direction of the process.  

2. **Second Diagram**:  
   - Similar phase regions are labeled as "Eis," "Wasser," and "Dampf."  
   - The cycle is drawn again with states \(1\), \(2\), \(3\), and \(4\).  
   - Additional annotations include a vertical line labeled \(b_2\) and arrows showing the transitions between states.  

Both diagrams represent the freeze-drying process qualitatively, with clear phase boundaries and transitions between ice, water, and vapor phases.

---