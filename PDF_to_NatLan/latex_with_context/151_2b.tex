The outlet velocity \( w_6 \) and temperature \( T_6 \) are calculated as follows:  

Given:  
\[
T_5 = 431.9 \, \text{K}, \quad p_5 = 0.5 \, \text{bar}, \quad w_5 = 220 \, \text{m/s}, \quad p_6 = p_0 = 0.151 \, \text{bar}
\]  

The isentropic relation is used:  
\[
\frac{T_6}{T_5} = \left( \frac{p_6}{p_5} \right)^{\frac{\kappa - 1}{\kappa}}
\]  
Substituting values:  
\[
T_6 = T_5 \cdot \left( \frac{p_6}{p_5} \right)^{\frac{\kappa - 1}{\kappa}} = 431.9 \cdot \left( \frac{0.151}{0.5} \right)^{\frac{0.4}{1.4}} \approx 328.07 \, \text{K}
\]  

Energy conservation is applied:  
\[
\Delta E = 0 = \dot{Q} + \dot{W} + \Delta KE + \Delta PE
\]  
Since heat transfer (\( \dot{Q} \)) and potential energy (\( \Delta PE \)) are negligible:  
\[
\Delta KE = -\dot{W}
\]  

The kinetic energy difference is calculated:  
\[
\Delta KE = c_p \cdot (T_6 - T_5) = c_p \cdot (328.07 - 431.9) = -74.605 \, \text{kJ/kg}
\]  

The velocity difference is derived:  
\[
\Delta w = \sqrt{2 \cdot 74.605} = 12.215 \, \text{m/s}
\]  
Thus, the outlet velocity is:  
\[
w_6 = w_5 + \Delta w = 220 + 12.215 = 232.215 \, \text{m/s}
\]