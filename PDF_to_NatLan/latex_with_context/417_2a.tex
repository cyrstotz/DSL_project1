The page contains a graph labeled "2a" that appears to be a qualitative temperature-entropy (\( T \)-\( s \)) diagram. The graph includes a curve with arrows indicating the direction of the process. The axes are labeled as follows:  
- The vertical axis is labeled \( T \) (temperature).  
- The horizontal axis is labeled \( s \) (entropy).  

The curve represents a thermodynamic process, with points marked along the curve to indicate specific states.  

---

The enthalpy \( h_2 \) is calculated as follows:  
\[
h_2 = h(328.075 \, \text{K}) = h(320 \, \text{K}) + 8.075 \cdot \frac{h(350 \, \text{K}) - h(320 \, \text{K})}{10}
\]  
Substituting the values:  
\[
h_2 = 320.25 \, \text{kJ/kg} + 8.075 \cdot \frac{320.36 \, \text{kJ/kg} - 320.25 \, \text{kJ/kg}}{10}
\]  
\[
h_2 = 328.405 \, \text{kJ/kg}
\]  

The outlet velocity \( w_6 \) is given as:  
\[
w_6 = w_5 = 220 \, \text{m/s}
\]  

The nozzle exit velocity \( w_2 \) is calculated using the equation:  
\[
w_2 = \sqrt{2(h_1 - h_2) + w_6^2}
\]  
Substituting the values:  
\[
w_2 = \sqrt{2(h_1 - h_2) + 220^2}
\]  
\[
w_2 = 220.47 \, \text{m/s}
\]