The task involves determining the outlet velocity \( w_6 \) and temperature \( T_6 \).  

The process from point 5 to point 6 is described as a reversible adiabatic (isentropic) process. The following equations and calculations are provided:  

1. The temperature \( T_6 \) is calculated using the isentropic relation:  
\[
\frac{T_6}{T_5} = \left( \frac{p_6}{p_5} \right)^{\frac{k-1}{k}}
\]  
Substituting values:  
\[
T_6 = T_5 \cdot \left( \frac{p_6}{p_5} \right)^{\frac{k-1}{k}}
\]  
\[
T_6 = 328.07 \, \text{K}
\]  

2. The outlet velocity \( w_6 \) is calculated using the energy balance:  
\[
w_6^2 = 2 \cdot c_p \cdot k \cdot (T_5 - T_6)
\]  
Substituting values:  
\[
w_6 = \sqrt{2 \cdot c_p \cdot k \cdot (T_5 - T_6)}
\]  
\[
w_6 = 540.8 \, \text{m/s}
\]  

Additional notes:  
- \( k = \frac{c_p}{c_v} \), and \( c_v = c_p \cdot k \).  

No further content is visible.