The problem involves determining the final ice fraction \( x_{\text{ice},2} \) in state 2 using the given data and equations.  

Initial conditions:  
\[
T_{\text{EW},2} = 0.003^\circ\text{C}, \quad p_{g,1} = 1.5 \, \text{bar}, \quad m_g = 3.6 \, \text{g}, \quad m_{\text{EW}} = 0.1 \, \text{kg}, \quad T_1 = 0^\circ\text{C}, \quad x_{\text{FW},1} = 0.6
\]  
Heat transferred:  
\[
Q_{12} = 1500 \, \text{J}
\]  

The heat transfer equation is given as:  
\[
Q_{12} = m_{\text{EW}} \cdot \Delta u
\]  
Substituting values:  
\[
Q_{12} = 0.1 \cdot \left[ -333.442 + (0.4) \cdot (-0.033) \right] = 200.0828
\]  

Specific volumes are calculated as:  
\[
v_1 = 0.6 \cdot (-333.442) + (1 - x) \cdot (-0.033)
\]  
\[
v_2 = x \cdot (-333.442) + (1 - x) \cdot (-0.033)
\]  

Final equation for \( Q_{12} \):  
\[
1500 = 0.1 \cdot \left( 200.0828 + x \cdot 333.442 - 0.033 + 0.033x \right)
\]  

Solving for \( x \):  
\[
x = 0.5549441 \approx 0.555 \quad \text{(rounded to 3 significant figures)}
\]  

Final ice fraction:  
\[
x_{\text{ice},2} = 0.555
\]  

No diagrams or figures are present on this page.