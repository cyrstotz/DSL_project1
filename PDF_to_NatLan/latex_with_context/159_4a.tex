Two diagrams are drawn to represent the freeze-drying process.  

1. **First diagram**:  
   - The graph is a pressure-temperature (\( p \)-\( T \)) diagram.  
   - The phase regions are indicated with curves, showing transitions between solid, liquid, and vapor phases.  
   - Two points are labeled:  
     - \( (i) \): Corresponds to the initial state, with pressure at approximately 5 mbar.  
     - \( (ii) \): Corresponds to the sublimation step, with pressure reduced further below the triple point of water.  
   - The axes are labeled \( p \) (pressure) and \( T \) (temperature).  

2. **Second diagram**:  
   - The graph is a pressure-volume (\( p \)-\( V \)) diagram.  
   - The process transitions are shown as lines, with two states labeled:  
     - \( (i) \): Initial state at 5 mbar.  
     - \( (ii) \): Final state after sublimation.  
   - The axes are labeled \( p \) (pressure) and \( V \) (volume).