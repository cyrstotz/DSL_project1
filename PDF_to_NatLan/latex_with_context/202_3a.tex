The gas pressure \( p_{g,1} \) is calculated using the formula:  
\[
p_{g,1} = \frac{F}{A} + p_{\text{amb}}
\]  
where \( F = (m_K + m_{\text{EW}}) \cdot g \) and \( A = \pi \cdot \left(\frac{D}{2}\right)^2 \).  

Substituting values:  
\[
A = \pi \cdot \left(\frac{0.1}{2}\right)^2 = 0.00785 \, \text{m}^2
\]  
\[
F = (32 \, \text{kg} + 0.1 \, \text{kg}) \cdot 9.81 = 314.901 \, \text{N}
\]  
\[
p_{g,1} = \frac{314.901}{0.00785} + 10^5 = 14014.8 \, \text{Pa}
\]  

The number of moles \( n \) is calculated using the ideal gas law:  
\[
p \cdot V = n \cdot R \cdot T \quad \implies \quad n = \frac{p \cdot V}{R \cdot T}
\]  

Substituting values:  
\[
T = 500^\circ\text{C} = 773.15 \, \text{K}, \quad p = 14014.8 \, \text{Pa}, \quad V = 0.00314 \, \text{m}^3
\]  
\[
R = \frac{R_u}{M} = \frac{8.314}{50} = 0.16628 \, \frac{\text{J}}{\text{mol·K}}
\]  
\[
n = \frac{14014.8 \cdot 0.00314}{0.16628 \cdot 773.15} = 3.42 \, \text{g}
\]