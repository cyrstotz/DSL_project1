The problem begins with determining the gas pressure \( p_{g,1} \) and mass \( m_g \) in state 1.  

The equation of state for an ideal gas is given as:  
\[
pV = mRT
\]  

A force balance is applied to calculate \( p_{g,1} \):  
\[
p = p_{\text{amb}} + \frac{0.1}{2} \cdot g + 10^5 + \frac{0.1}{2} \cdot \frac{32 \cdot g}{\pi \cdot \frac{0.1}{2}^2}
\]  
Substituting values:  
\[
p_{g,1} = 0.1 \cdot g + 10^5 + \frac{0.1}{2} \cdot \frac{32 \cdot g}{\pi \cdot \frac{0.1}{2}^2} = 1.40 \, \text{bar}
\]  
where \( g = 9.81 \, \text{m/s}^2 \).  

Next, the gas mass \( m_g \) is calculated using the ideal gas law:  
\[
m_g = \frac{pV}{RT}
\]  
Substituting values:  
\[
m_g = \frac{1.40 \cdot 10^5 \cdot 3.14 \cdot 10^{-3}}{(500 + 273.15) \cdot \frac{50}{10^3}} = 3.422 \, \text{g}
\]  

---