The temperature \( T_{g,2} \) in state 2 is equal to \( T_{\text{EW},2} \), as thermal equilibrium is reached.  
\[
T_{g,2} = T_{\text{EW},2} = 273.15 \, \text{K}
\]  

The pressure \( p_{g,2} \) is calculated using the formula:  
\[
p_{g,2} = p_{\text{amb}} + \frac{m_K \cdot g}{A}
\]  
where \( p_{\text{amb}} = 1 \, \text{bar} \), \( m_K = 32 \, \text{kg} \), \( g = 9.81 \, \text{m/s}^2 \), and \( A = 0.00785 \, \text{m}^2 \).  
The calculation yields:  
\[
p_{g,2} = 1.769 \, \text{bar}
\]  

The explanation states that the temperature remains constant because the gas transfers heat to the ice-water mixture, melting the ice. The pressure changes due to the piston maintaining equilibrium.