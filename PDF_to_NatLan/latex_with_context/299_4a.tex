Two graphs are drawn to represent the freeze-drying process in a \( p \)-\( T \) diagram.  

1. **First graph**:  
   - The vertical axis is labeled \( p \) (pressure), and the horizontal axis is labeled \( T \) (temperature).  
   - The graph shows phase regions labeled "ice," "water," and "vapor."  
   - A curve separates the "ice" and "water" regions, and another curve separates "water" and "vapor."  
   - A rectangular box is drawn in the "vapor" region, labeled with state points corresponding to the refrigeration cycle.  

2. **Second graph**:  
   - Similar axes are used: \( p \) (pressure) on the vertical axis and \( T \) (temperature) on the horizontal axis.  
   - The graph shows the refrigeration cycle with states labeled \( 1 \), \( 2 \), \( 3 \), and \( 4 \).  
   - The cycle is represented as a closed loop, with arrows indicating the direction of the process.