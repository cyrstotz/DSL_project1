The pressure \( p_{g,1} \) in state 1 is calculated as:  
\[
p_{g,1} = p_{\text{amb}} + \frac{m_K \cdot g}{A}
\]  
Substituting the values:  
\[
p_{g,1} = p_{\text{amb}} + \frac{32.5 \cdot 9.81 \, \text{m/s}^2}{\left(0.1 \, \text{m}\right)^2 \cdot \pi}
\]  
\[
p_{g,1} = 1 \, \text{bar} + 3.59568 \, \text{Pa} \approx 1.4 \, \text{bar}.
\]  

The gas is modeled as a perfect gas with \( M_g = 50 \, \text{kg/kmol} \).  

The specific gas constant \( R \) is calculated as:  
\[
R = \frac{\bar{R}}{M} = \frac{8.314 \, \text{J/(mol·K)}}{50 \, \text{kg/kmol}} = 166.28 \, \text{J/(kg·K)}.
\]  

The mass \( m_g \) of the gas is determined using:  
\[
m_g = \frac{p \cdot V}{R \cdot T}.
\]  
Substituting the values:  
\[
m_g = \frac{3.14 \, \text{L} \cdot 1.4 \, \text{bar}}{166.28 \, \text{J/(kg·K)} \cdot 773.15 \, \text{K}} = 3.14 \, \text{g}.
\]  

No diagrams or figures are present on the page.