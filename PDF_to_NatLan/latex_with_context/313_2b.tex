The following calculations are performed for the outlet temperature \( T_6 \):  
\[
p_0 = p_6
\]
\[
T_6 = T_5 \left( \frac{p_0}{p_5} \right)^{\frac{n-1}{n}}
\]
Where \( n = \kappa = 1.4 \), \( p_0 = 15100 \, \text{Pa} \), and \( p_5 = 50000 \, \text{Pa} \). Substituting values:  
\[
T_6 = 323.07 \, \text{K}
\]

The energy balance equation is written:  
\[
0 = h_c - h_a + \frac{w_c^2 - w_a^2}{2}
\]
The specific heat equation is used:  
\[
c = c_p (T_c - T_a) + \frac{w_c^2 - w_a^2}{2}
\]
The work output is calculated:  
\[
w_c = 507.23 \, \frac{\text{m}}{\text{s}}
\]

It is noted that the **adiabatic nozzle produces no work**.