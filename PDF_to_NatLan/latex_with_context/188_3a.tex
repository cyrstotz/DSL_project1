The cross-sectional area of the cylinder is calculated as:  
\[
A = (0.05 \, \text{m})^2 \pi = 7.854 \times 10^{-3} \, \text{m}^2
\]  
The pressure exerted by the piston is determined using:  
\[
p_{g,1} = 1 \, \text{bar} + \frac{32 \, \text{kg} \cdot 9.81 \, \text{m/s}^2}{7.854 \times 10^{-3} \, \text{m}^2} + \frac{0.1 \, \text{kg} \cdot 9.81 \, \text{m/s}^2}{7.854 \times 10^{-3} \, \text{m}^2}
\]  
This results in:  
\[
p_{g,1} = 140000 \, \text{Pa} = 1.4 \, \text{bar}
\]  

Using the ideal gas law \( pV = mRT \), the mass of the gas is calculated as:  
\[
m_g = \frac{pV}{RT}
\]  
Substituting values:  
\[
m_g = \frac{140000 \, \text{Pa} \cdot 0.00314 \, \text{m}^3}{8.1663 \, \frac{\text{J}}{\text{kg·K}} \cdot 773.15 \, \text{K}} = 3.42 \, \text{g}
\]  

The specific gas constant \( R \) is calculated as:  
\[
R = \frac{\bar{R}}{M} = \frac{8.314 \, \frac{\text{kJ}}{\text{kmol·K}}}{50 \, \frac{\text{kg}}{\text{kmol}}} = 0.1663 \, \frac{\text{kJ}}{\text{kg·K}}
\]  

---