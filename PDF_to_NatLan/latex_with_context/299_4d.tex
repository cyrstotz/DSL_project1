No explicit calculation for the coefficient of performance \( \epsilon_K \) is visible, but the general formula is implied:  
\[
\epsilon_K = \frac{\dot{Q}_K}{\dot{W}_K}
\]  

No further details are provided.

The coefficient of performance \( \epsilon_K \) is defined as:  
\[
\epsilon_K = \frac{\lvert \dot{Q}_K \rvert}{\lvert \dot{W}_K \rvert}
\]  
Given \( T_i = -6^\circ \, \text{C} \) and \( T_e = -22^\circ \, \text{C} \), the heat capacity \( c_p \) is calculated using a constant value:  
\[
c_p = 7.934 \cdot 10^3 \, \text{J/kg·K}
\]  

The energy balance for \( \dot{Q}_K \) is expressed as:  
\[
\dot{Q}_K = \dot{m}_{\text{R134a}} \cdot (h_1 - h_2) + \dot{Q}_K
\]  
Substituting the enthalpy difference:  
\[
\dot{Q}_K = -\dot{m}_{\text{R134a}} \cdot (T_i - T_e) \cdot c_p = -91,796 \, \text{W}
\]  

The coefficient of performance is then calculated as:  
\[
\epsilon_K = \frac{-91,796 \, \text{W}}{-28 \, \text{W}} = -3,277.7
\]