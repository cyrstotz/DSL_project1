To determine the gas pressure \( p_{g,1} \) and mass \( m_g \) in state 1:  

The following values are given:  
- \( c_V = 0.633 \, \text{kJ/kg·K} \)  
- \( M_g = 50 \, \text{kg/kmol} \)  
- \( T_{g,1} = 500^\circ\text{C} = 773.15 \, \text{K} \)  
- \( V_{g,1} = 3.14 \, \text{L} = 0.00314 \, \text{m}^3 \)  

The ideal gas law is used:  
\[
p = \frac{mRT}{V}
\]  

The cross-sectional area of the cylinder is calculated as:  
\[
A = \pi \left(\frac{D}{2}\right)^2 = \pi \left(\frac{0.05 \, \text{m}}{2}\right)^2 = 7.854 \times 10^{-3} \, \text{m}^2
\]  

The pressure exerted by the piston is:  
\[
p = p_{\text{amb}} + \frac{F}{A}
\]  
where \( F = m_K \cdot g \). Substituting values:  
\[
p = 10^5 \, \text{Pa} + \frac{32 \, \text{kg} \cdot 9.81 \, \text{m/s}^2}{7.854 \times 10^{-3} \, \text{m}^2} = 10^5 \, \text{Pa} + 39469.1 \, \text{Pa} = 1.3947 \times 10^5 \, \text{Pa}
\]  
Thus, \( p_{g,1} = 1.3947 \times 10^5 \, \text{Pa} \).  

The gas constant is calculated as:  
\[
R = \frac{R_u}{M} = \frac{8.314}{50} = 0.1663 \, \text{kJ/kg·K}
\]  

The mass of the gas is determined using:  
\[
m = \frac{pV}{RT}
\]  
Substituting values:  
\[
m = \frac{1.3947 \times 10^5 \, \text{Pa} \cdot 0.00314 \, \text{m}^3}{0.1663 \, \text{kJ/kg·K} \cdot 773.15 \, \text{K}} = 3.418 \times 10^{-3} \, \text{kg}
\]  
Thus, \( m_g = 3.418 \, \text{g} \).  

---