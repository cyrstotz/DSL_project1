The problem begins with the given values:  
\[
T_{g,1} = 500^\circ\text{C}, \quad V_{g,1} = 3.14 \times 10^{-3} \, \text{m}^3
\]  

The gas pressure \( p_{g,1} \) is calculated using the ideal gas law:  
\[
p_{g,1} = \frac{mRT}{V}
\]  

The gas constant \( R \) is determined as:  
\[
R = \frac{R_M}{M} = \frac{8.314 \, \text{kJ/(kmol·K)}}{50 \, \text{kg/kmol}} = 0.16628 \, \text{kJ/(kg·K)}
\]  

The force \( F \) exerted by the piston is calculated as:  
\[
F = \frac{A}{A} = \frac{32 \, \text{kg} \cdot 9.81 \, \text{m/s}^2 + 0.1 \, \text{kg} \cdot 9.81 \, \text{m/s}^2}{2 \cdot \pi \cdot (0.05)^2} + 1 \, \text{bar} + 405 \, \text{Pa}
\]  

The pressure \( p \) is then calculated:  
\[
p = 8.00999 \, \text{bar} = 190099.94006 \, \text{Pa}
\]  

The mass \( m \) of the gas is calculated using the ideal gas law:  
\[
m = \frac{pV}{RT} = \frac{190099.94 \, \text{Pa} \cdot 3.14 \times 10^{-3} \, \text{m}^3}{0.16628 \, \text{kJ/(kg·K)} \cdot (500 + 273.15) \, \text{K}} = 3.927 \, \text{kg}
\]