The mass of saturated liquid water added to reduce the temperature from \(100^\circ\text{C}\) to \(70^\circ\text{C}\) is calculated using the energy balance:  

The heat released during cooling is:  
\[
Q_{R,12} = Q_{\text{out},12} = 35 \, \text{MJ}
\]

The energy balance equation is:  
\[
m_2 h_2 + m_1 h_1 = \Delta m h_f(20^\circ\text{C}) + Q_{R,12}
\]

Rearranging:  
\[
(m_1 + \Delta m) u_2 - m_1 u_1 = \Delta m h_f(20^\circ\text{C})
\]

Solving for \( \Delta m \):  
\[
\Delta m = \frac{m_1 u_1 - m_2 u_2}{h_{f,12} - h_f}
\]

Substituting values:  
\[
\Delta m = 3.668088 \, \text{kg}
\]

The specific internal energy values used are:  
\[
u_1 = 419.94 \, \text{kJ/kg}
\]
\[
u_2 = 292.58 \, \text{kJ/kg}
\]
\[
h_f = 83.96 \, \text{kJ/kg}
\]