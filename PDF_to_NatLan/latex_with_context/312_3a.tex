The gas pressure \( p_{G,1} \) is calculated using the equation:  
\[
p_{G,1} = p_{\text{amb}} + \frac{M_{\text{Kolben}} \cdot g}{\pi \left(\frac{D}{2}\right)^2} + \frac{m_{\text{EW}} \cdot g}{\pi \left(\frac{D}{2}\right)^2}
\]  
The cross-sectional area of the cylinder is determined as:  
\[
A = \pi \left(\frac{D}{2}\right)^2 = 0.00785 \, \text{m}^2
\]  
Substituting the values:  
\[
p_{G,1} = \frac{32 \, \text{kg} \cdot 9.81 \, \text{m/s}^2}{0.00785 \, \text{m}^2} + \frac{0.1 \, \text{kg} \cdot 9.81 \, \text{m/s}^2}{0.00785 \, \text{m}^2} = 140090 \, \text{Pa} = 1.40 \, \text{bar}
\]  

The mass of the gas \( m_{G,1} \) is calculated using the ideal gas law:  
\[
p_{G,1} \cdot V_{G,1} = m_{G,1} \cdot R_G \cdot T_{G,1} \quad \Rightarrow \quad m_{G,1} = \frac{p_{G,1} \cdot V_{G,1}}{R_G \cdot T_{G,1}}
\]  
The specific gas constant \( R_G \) is given as:  
\[
R_G = \frac{R}{M_G} = 166.3 \, \text{J/(kg·K)}
\]  
The gas volume \( V_{G,1} \) is converted from liters to cubic meters:  
\[
V_{G,1} = 3.14 \, \text{L} = 0.00314 \, \text{m}^3
\]  
Substituting the values:  
\[
m_{G,1} = 0.00342 \, \text{kg} = 3.42 \, \text{g}
\]