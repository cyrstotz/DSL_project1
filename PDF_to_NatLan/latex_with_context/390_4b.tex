The first law of thermodynamics is applied to a stationary flow process, neglecting kinetic and potential energy changes:  
\[
0 = \dot{m}_{\text{R134a}} \left[ h_2 - h_3 \right] + \dot{Q}_K - \dot{W}_t
\]  
Rearranging for the mass flow rate of refrigerant:  
\[
\dot{m}_{\text{R134a}} = \frac{\dot{W}_t}{h_2 - h_3}
\]  
Using data from Tables A-10 to A-12:  
- \( h_2 = h_g(-22^\circ\text{C}) = 234.02 \, \frac{\text{kJ}}{\text{kg}} \)  
- \( h_3 = h_g(8 \, \text{bar}) = 264.13 \, \frac{\text{kJ}}{\text{kg}} \)  
- \( \dot{W}_t = -\dot{Q}_K = -20 \, \frac{\text{kJ}}{\text{s}} = -20 \times 10^3 \, \frac{\text{W}}{\text{s}} \)  

Substituting values:  
\[
\dot{m}_{\text{R134a}} = \frac{-20 \times 10^3}{234.02 - 264.13}  
\]  
\[
\dot{m}_{\text{R134a}} = 0.000665 \, \frac{\text{kg}}{\text{s}}  
\]  
\[
\dot{m}_{\text{R134a}} = 2.39 \, \frac{\text{kg}}{\text{h}}  
\]