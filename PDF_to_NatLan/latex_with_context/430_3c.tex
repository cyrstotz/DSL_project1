The energy balance equation is written as:
\[
\frac{dE}{dt} = \dot{Q}_{12} - \sum \dot{W}_n
\]
Here, the term \( \sum \dot{W}_n \) is zero due to negligible volume work.

The relationship for the energy change is expressed as:
\[
U_2 - U_1 = Q_{12}
\]

The internal energy \( U_1 \) is calculated using:
\[
U_1 = U_f + x \cdot (U_g - U_f)
\]
where \( x \) is the ice mass fraction, \( U_f \) is the internal energy of saturated liquid, and \( U_g \) is the internal energy of saturated vapor.

From Table 1:
\[
U_f = -0.045 \, \frac{\text{kJ}}{\text{kg}}, \quad U_g = 0.6 \cdot (-33.3 \, \frac{\text{kJ}}{\text{kg}} + 0.045 \, \frac{\text{kJ}}{\text{kg}})
\]
\[
U_1 = -200.09 \, \frac{\text{kJ}}{\text{kg}}
\]

The internal energy of the ice-water mixture is:
\[
U_1 = U_{\text{in EW}} = -200.09 \, \frac{\text{kJ}}{\text{kg}}
\]

The internal energy \( U_2 \) is calculated as:
\[
U_2 = Q_{12} + U_1 = 1366.4 \, \text{kJ} - 200.1 \, \text{kJ} = 1166.3 \, \text{kJ}
\]

The ice mass fraction \( x \) is determined using:
\[
x = \frac{U_2 - U_f}{U_g - U_f}
\]
Substituting values:
\[
x = \frac{-1166.3 \, \text{kJ} + 0.045 \, \frac{\text{kJ}}{\text{kg}}}{0.6 \cdot (-33.3 \, \frac{\text{kJ}}{\text{kg}} + 0.045 \, \frac{\text{kJ}}{\text{kg}})}
\]
\[
x = 0.5587
\]

The mass of ice in equilibrium is calculated as:
\[
m_{2,\text{ice}} = x_{2,\text{ice}} \cdot m_{\text{EW}} = 0.5587 \cdot 0.1 \, \text{kg} = 0.05587 \, \text{kg}
\]

An energy balance is applied to the system:  
\[
\frac{dE}{dt} = \sum_j \dot{Q}_j - \sum_n \dot{W}_n
\]
This simplifies to:  
\[
U_2 - U_1 = Q_{12} - W_{12}
\]

Since there is no friction, the work \( W_{12} \) is reversible:  
\[
W_{12} = \int p \, dV
\]

The gas mass is calculated as:  
\[
m_g = \frac{V_{g,1}}{v_{g,1}}
\]
Substituting values:  
\[
m_g = \frac{3.14 \cdot 10^{-3} \, \text{m}^3}{3.429 \cdot 10^{-3} \, \text{m}^3/\text{kg}} = 0.9168397 \, \text{kg}
\]

---

Description of Diagram:  
A small sketch shows a system with labeled heat transfer (\( Q_{12} \)) and work (\( W_{12} \)) arrows. The system boundary is indicated, representing the energy balance applied to the gas and ice-water mixture.