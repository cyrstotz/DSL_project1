The first law of thermodynamics is applied to an EW (ice-water mixture) container. The change in internal energy is expressed as:  
\[
\Delta U = Q - W = c_{l2} - c_{l1}
\]

The internal energy at state 2 (\( U_2 \)) is calculated as:  
\[
U_2 = \frac{Q}{m_{\text{EW}}} + u_1 \implies U_2 = u_f + x_2 (u_{fg} - u_f)
\]  
Using the values:  
\[
u_f = -333.455 \, \frac{\text{kJ}}{\text{kg}}, \quad u_{fg} = 0.6 \cdot (-0.0452 + 333.458) \, \frac{\text{kJ}}{\text{kg}}
\]  
This results in:  
\[
u_2 = -133.41 \, \frac{\text{kJ}}{\text{kg}} \quad \text{(from Table 7 at \( 0^\circ\text{C} \))}.
\]

Next, the calculation continues:  
\[
U_2 = \frac{7500 \, \text{J}}{0.1 \, \text{kg}} + (-333.47 \, \frac{\text{kJ}}{\text{kg}})
\]  
\[
U_2 = 748.47 \, \frac{\text{kJ}}{\text{kg}} - 178.47 \, \frac{\text{kJ}}{\text{kg}}
\]

The ice mass fraction at state 2 (\( x_2 \)) is determined using:  
\[
x_2 = \frac{u_2 - u_s}{u_f - u_s}
\]  
Substituting values:  
\[
x_2 = \frac{-178.47 \, \frac{\text{kJ}}{\text{kg}} - (-333.442 \, \frac{\text{kJ}}{\text{kg}})}{-0.033 \, \frac{\text{kJ}}{\text{kg}} - (-333.442 \, \frac{\text{kJ}}{\text{kg}})}
\]  
\[
x_2 = 0.6449 \quad \text{(from Table 7 at \( 0.003^\circ\text{C} \))}.
\]

The student notes that this value should correspond to state 2 and questions whether it is consistent with the expected result:  
"Should this be higher than in state 1?"  

No diagrams or graphs are present on this page.