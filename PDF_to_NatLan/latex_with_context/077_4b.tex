The refrigerant mass flow rate (\(\dot{m}_{\text{R134a}}\)) is to be determined.  

The state transition from 2 to 3 is described as isentropic because it is reversible and adiabatic.  

The energy balance equation for the process is given as:  
\[
\dot{m}_{\text{R134a}} (h_e - h_a) + \sum \dot{Q}_j - \sum \dot{W}_j = 0
\]  
where \(h_e\) and \(h_a\) represent specific enthalpies, \(\dot{Q}_j\) is the heat transfer rate, and \(\dot{W}_j\) is the work rate.  

Additional notes:  
- The piston is described as mechanical.  
- Energy balance for the compressor is highlighted.  

The energy balance for compression is expressed as:  
\[
\dot{m} (h_2 - h_3) = 28 \, \text{W}
\]

The mass flow rate \( \dot{m} \) is calculated using the equation:  
\[
\dot{m} = \frac{28 \, \text{W}}{h_2 - h_3}
\]  
where \( h_2 - h_3 \) represents the enthalpy difference.  

At \( x_2 = 1 \), the refrigerant is fully vaporized (saturated vapor).  

---