Given: \( p_{g,1} \), \( m_g \)  

The gas constant \( R \) is calculated as:  
\[
R = \frac{\bar{R}}{M} = \frac{8.314 \, \text{kJ/(kmol·K)}}{50 \, \text{kg/kmol}} = 0.1663 \, \text{kJ/(kg·K)}
\]  

The specific heat capacities are:  
\[
c_V = 0.633 \, \text{kJ/(kg·K)}  
\]  
\[
c_P = R + c_V = 0.7993 \, \text{kJ/(kg·K)}
\]  

The pressure \( p_{g,1} \) is the pressure exerted from above:  
\[
p_{g,1} = 1 \, \text{bar} + \frac{g \cdot (m_K + m_{\text{EW}})}{A}
\]  
where \( A = (10 \, \text{cm})^2 \cdot \pi \).  

Substituting values:  
\[
p_{g,1} = 1 \, \text{bar} + \frac{9.81 \, \text{m/s}^2 \cdot (32 \, \text{kg} + 0.1 \, \text{kg})}{\frac{\pi}{100} \, \text{m}^2} = 1 \, \text{bar} + 0.1 \, \text{bar} = 1.1 \, \text{bar}
\]  

Since the gas is ideal, the equation of state applies:  
\[
pV = mRT
\]  

Thus:  
\[
m_g = \frac{p_{g,1} \cdot V_{g,1}}{R \cdot T_{g,1}} = \frac{1.1 \, \text{bar} \cdot 3.14 \, \text{L}}{0.1663 \, \text{kJ/(kg·K)} \cdot 500^\circ\text{C}}
\]  

Converting units and solving:  
\[
m_g = 2.686 \, \text{g}
\]  

---