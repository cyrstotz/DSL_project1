The gas pressure \( p_{g,1} \) is calculated using the ideal gas law:  
\[
p = \frac{mRT}{V}
\]  
Here, \( p_{g,1} \) represents the counterpressure against the ice-water mixture and the surroundings.  

The pressure is given by:  
\[
p_{g,1} = p_{\text{amb}} + \frac{m_K g}{A}
\]  
Where:  
- \( p_{\text{amb}} = 100,000 \, \text{Pa} \)  
- \( m_K = 32 \, \text{kg} \)  
- \( g = 9.81 \, \text{m/s}^2 \)  
- \( A = r^2 \pi \), with \( r = 0.1 \, \text{m} \), so \( A = 0.031416 \, \text{m}^2 \).  

Substituting values:  
\[
p_{g,1} = 100,000 + \frac{32 \cdot 9.81}{0.031416} = 100,000 + 10,000 \, \text{Pa} = 110,000 \, \text{Pa}
\]  

Next, the gas mass \( m_g \) is calculated:  
\[
m_g = \frac{p V}{R T}
\]  
Where:  
- \( p = 110,000 \, \text{Pa} \)  
- \( V = 0.00314 \, \text{m}^3 \)  
- \( R = \frac{8,314 \, \text{J/(kmol·K)}}{50 \, \text{kg/kmol}} = 166.28 \, \text{J/(kg·K)} \)  
- \( T = 500 + 273.15 = 773.15 \, \text{K} \).  

Substituting values:  
\[
m_g = \frac{110,000 \cdot 0.00314}{166.28 \cdot 773.15} = 0.002687 \, \text{kg}
\]  

Thus, \( m_g = 2.687 \, \text{g} \).  

---