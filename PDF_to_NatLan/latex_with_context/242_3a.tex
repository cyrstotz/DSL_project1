The ideal gas law is applied:  
\[
PV = mRT
\]  
The gas constant \( R \) is calculated as:  
\[
R = \frac{8314 \, \text{J/(kmol·K)}}{50 \, \text{kg/kmol}} = 166.28 \, \text{J/(kg·K)}
\]  
The cross-sectional area \( A \) of the cylinder is determined:  
\[
A = \pi \left(\frac{D}{2}\right)^2 = \pi \left(\frac{0.1}{2}\right)^2 = 7.854 \times 10^{-3} \, \text{m}^2
\]  

The pressure exerted by the piston (\( p_{\text{cis}} \)) is calculated:  
\[
p_{\text{cis}} = \frac{m_K g}{A} + p_{\text{amb}} = \frac{32 \, \text{kg} \cdot 9.81 \, \text{m/s}^2}{7.854 \times 10^{-3} \, \text{m}^2} + 1 \, \text{bar} = 1.43716 \, \text{bar} \approx 1.44 \, \text{bar}
\]  

The gas pressure (\( p_g \)) is determined:  
\[
p_g = p_{\text{cis}} - \frac{m_B g}{A} = 1.44 \, \text{bar} - p_{g,1}
\]  

The mass of the gas (\( m_g \)) is calculated using the ideal gas law:  
\[
m_g = \frac{PV}{RT} = \frac{p_{g,1} V_{g,1}}{R T_{g,1}} = \frac{1.44 \, \text{bar} \cdot 3.14 \, \text{L}}{166.28 \, \text{J/(kg·K)} \cdot 773.15 \, \text{K}} = 3.517 \times 10^{-3} \, \text{kg}
\]  

---