The task involves the freeze-drying process for food, specifically focusing on the behavior of water under different pressure and temperature conditions. A table is provided with three states:

1. \( p_1 = p_2 \), \( T > T_i \)  
2. \( p_2 = p_{\text{amb}} \), \( T = T_i \)  
3. \( p_3 < T(p) \), \( T = T_i \)  

Two diagrams are drawn to illustrate the process:

1. **First Diagram (p-T Diagram):**  
   - The vertical axis represents pressure (\( p \)) in mbar, ranging from 0.1 to 10 mbar.  
   - The horizontal axis represents temperature (\( T \)) in degrees Celsius.  
   - The diagram includes the following regions:  
     - "Dampf" (vapor) at the top.  
     - "Wasser" (water) in the middle.  
     - "Eis" (ice) at the bottom.  
   - The triple point is marked, and lines separate the phases.  
   - A trajectory is drawn, starting in the vapor region (point 1), passing through the triple point, and ending in the ice region.  

2. **Second Diagram (Zoomed p-T Diagram):**  
   - The vertical axis represents pressure (\( p \)) in mbar, ranging from 1 to 5 mbar.  
   - The horizontal axis represents temperature (\( T \)) in degrees Celsius, ranging from 0°C to 10°C.  
   - The diagram includes the following:  
     - "Flüssig" (liquid) and "Dampf" (vapor) regions.  
     - The triple point is marked.  
     - An isothermal line is drawn at \( T_i \).  
     - A trajectory is shown, starting in the desublimation phase (point 1), passing through the triple point (point 2), and ending in the vapor phase (point 3).  

Both diagrams visually represent the freeze-drying process and the transitions between phases under varying pressure and temperature conditions.