To determine the ice fraction \( x_{\text{ice},2} \) in state 2, the following formula is used:  

\[
x_{\text{ice},2} = \frac{U - U_{\text{Fest}}}{U_{\text{Flüssig}} - U_{\text{Fest}}}
\]  

Where \( U \) is defined as:  

\[
U = \frac{Q}{\dot{m}_{\text{EW}}}
\]  

Substituting values:  

\[
U = \frac{10882}{1.08} = 10.82 \, \frac{\text{kJ}}{\text{kg}}
\]  

Since the system is not fully liquid, the temperature remains at \( 0^\circ\text{C} \), and the values for \( U_{\text{Fest}} \) and \( U_{\text{Flüssig}} \) are taken from the table:  

\[
U_{\text{Fest}} = -333.458 \, \frac{\text{kJ}}{\text{kg}}, \quad U_{\text{Flüssig}} = -0.645 \, \frac{\text{kJ}}{\text{kg}}
\]  

Substituting these values into the formula:  

\[
x_{\text{ice},2} = \frac{(10.82 \, \frac{\text{kJ}}{\text{kg}}) - (-333.458 \, \frac{\text{kJ}}{\text{kg}})}{(-0.645 \, \frac{\text{kJ}}{\text{kg}}) - (-333.458 \, \frac{\text{kJ}}{\text{kg}})}
\]  

Simplifying:  

\[
x_{\text{ice},2} = \frac{344.278 \, \frac{\text{kJ}}{\text{kg}}}{332.813 \, \frac{\text{kJ}}{\text{kg}}} = 1.032
\]  

Thus, the ice fraction in state 2 is \( x_{\text{ice},2} = 1.032 \).