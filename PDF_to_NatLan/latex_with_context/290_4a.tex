Two diagrams are drawn to represent the freeze-drying process.  

1. **First Diagram**:  
   - The graph is labeled with axes: pressure \( P \) (in bar) on the vertical axis and specific volume \( v \) (in \( \text{m}^3/\text{kg} \)) on the horizontal axis.  
   - The diagram shows phase regions, including saturated liquid and saturated vapor.  
   - Four states are marked: \( T_i \), \( T_4 \), \( T_3 \), and \( T_1 \).  
   - Processes are labeled as "isobaric" and "isentropic."  
   - The student writes "sorry, falsch gelesen" (sorry, misread) with a sad face.  

2. **Second Diagram**:  
   - The graph is labeled with axes: pressure \( P \) (in bar) on the vertical axis and temperature \( T \) (in Kelvin) on the horizontal axis.  
   - Phase regions are depicted, including saturated liquid and saturated vapor.  
   - Four states are marked: \( x = 0 \) at state 4 and \( x = 1 \) at state 1.  
   - Processes are labeled as "isobaric" and "isentropic."