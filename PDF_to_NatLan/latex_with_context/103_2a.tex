The page contains a table and calculations related to thermodynamic properties and processes.  

### Table Description:  
The table includes the following columns:  
- \( p \) (pressure in bar)  
- \( T \) (temperature in Kelvin)  
- \( w \) (velocity in \( \text{m/s} \))  
- Flow designation (e.g., \( \dot{m}_{\text{in}} \), \( \dot{m}_K \), etc.)  

The rows are labeled with states (1, 2, 3, 4, 5, 6, and 0).  
- At state 5:  
  \( p = 0.5 \, \text{bar} \), \( T = 431.9 \, \text{K} \), \( w = 220 \, \text{m/s} \), flow designation \( \dot{m}_{\text{in}} \).  

- At state 0:  
  \( p = 0.191 \, \text{bar} \), \( T = 243.15 \, \text{K} \), \( w = 200 \, \text{m/s} \).  

The table also includes a note indicating "isentrop" (likely referring to an isentropic process).  

---

### Calculations:  
The following thermodynamic relationships are calculated:  

1. Specific heat at constant pressure:  
   \[
   c_p = 1.006 \, \frac{\text{kJ}}{\text{kg·K}}
   \]

2. Ratio of specific heats:  
   \[
   k = \frac{c_p}{c_v} = 1.4
   \]

3. Specific heat at constant volume:  
   \[
   c_v = \frac{c_p}{k} = \frac{1.006}{1.4} = 0.719 \, \frac{\text{kJ}}{\text{kg·K}}
   \]

4. Gas constant:  
   \[
   R = c_p - c_v = 1.006 - 0.719 = 0.287 \, \frac{\text{kJ}}{\text{kg·K}}
   \]  

No diagrams or graphs are present on this page.

The page contains calculations and derivations related to the jet engine process described in Task 2. The focus appears to be on determining work and temperature changes during the process from state 5 to state 6, considering isentropic and polytropic processes.

---

The process from state 5 to state 6 is described as isentropic, with \( S_5 = S_6 \).

The reversible work \( w_{\text{rev}} \) is expressed as:
\[
w_{\text{rev}} = -\int_1^2 v \, dp + \Delta \text{ke}
\]
where \( \Delta \text{ke} \) represents the change in kinetic energy:
\[
\Delta \text{ke} = \frac{w_6^2}{2} - \frac{w_5^2}{2}
\]

For a polytropic process (\( n \neq 1 \)), the work is derived as:
\[
w_{\text{rev}} = \int_1^2 v \, dp = \frac{R (T_6 - T_5)}{1 - n}
\]

The temperature \( T_6 \) is calculated using the polytropic relation:
\[
T_6 = T_5 \left( \frac{p_6}{p_5} \right)^{\frac{n-1}{n}}
\]
Substituting values:
\[
T_6 = 431.9 \, \text{K} \left( \frac{0.1 \, \text{bar}}{0.5 \, \text{bar}} \right)^{\frac{0.4}{1.4}}
\]
\[
T_6 = 328.1 \, \text{K}
\]

The work \( w_{\text{rev}} \) is then calculated:
\[
w_{\text{rev}} = 0.287 \, \frac{\text{kJ}}{\text{kg·K}} \left( 328.1 \, \text{K} - 431.9 \, \text{K} \right) \frac{1}{1 - 1.4}
\]
\[
w_{\text{rev}} = 79.5 \, \text{kW}
\]

Additional expressions for work and energy balance are provided:
\[
w_{\text{rev}} = \dot{m} w_{\text{rev}}
\]
\[
\dot{m} \left( h_5 - h_6 + \frac{w_5^2 - w_6^2}{2} \right) - w_{\text{rev}} = 0
\]
\[
\dot{m} c (T_5 - T_6) + \frac{w_5^2 - w_6^2}{2} - w_{\text{rev}} = 0
\]

---

No diagrams or graphs are present on the page. The calculations are consistent with the jet engine thermodynamic process described in Task 2.

The diagram is a qualitative \( T \)-\( S \) (temperature-entropy) plot.  

- The x-axis is labeled as \( S \) (entropy), and the y-axis is labeled as \( T \) (temperature).  
- The graph shows a process with multiple states labeled as 1, 2, 3, and 4.  
- The curve from state 1 to state 2 appears to represent an increase in entropy.  
- The curve from state 2 to state 3 is downward, indicating a decrease in temperature and entropy.  
- The curve from state 3 to state 4 is vertical, suggesting an isentropic process (constant entropy).  
- The label "isentrop" is written near the vertical line between states 3 and 4.  

No additional textual explanation is provided.