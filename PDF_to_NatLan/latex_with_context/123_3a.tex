The table outlines the initial and final states of the system:  

**State 1:**  
- \( V_g = 3.14 \, \text{L} \)  
- \( T_g = 500^\circ\text{C} = 773.15 \, \text{K} \)  
- \( m_{\text{EW}} = 0.1 \, \text{kg} \)  
- \( T_{\text{EW}} = 0^\circ\text{C} \)  
- \( x_{\text{ice}} = 0.6 \)  

**State 2:**  
(No values provided in the table for state 2.)  

Additional constants and parameters are listed:  
- \( m_K = 32 \, \text{kg} \)  
- \( p_{\text{amb}} = 1 \, \text{bar} \)  
- \( D = 0.1 \, \text{m} \)  
- \( c_V = 0.633 \, \text{kJ/kg·K} \)  
- \( M_g = 50 \, \text{kg/kmol} \)  

The task is to determine the gas pressure \( p_{g,1} \) and mass \( m_g \) in state 1.  

The following formulas and calculations are used:  
1. **Force and Area Relationship:**  
   \[ F = p \cdot A \]  
   The area of the piston is calculated using:  
   \[ A = \frac{d^2 \pi}{4} \]  

2. **Pressure Calculation:**  
   The pressure \( p_{g,1} \) is derived from the equilibrium of forces acting on the membrane:  
   \[ p_{g,1} = p_0 + \frac{m_K g}{A} + \frac{m_{\text{EW}} g}{A} \]  
   Substituting values:  
   \[ p_{g,1} = 1.05 \cdot 10^5 + \frac{32 \cdot 9.81}{\frac{0.1^2 \pi}{4}} + \frac{0.1 \cdot 9.81}{\frac{0.1^2 \pi}{4}} \]  
   \[ p_{g,1} = 78538.38 + 343302.0458 + 1100.694 \, \text{Pa} \]  
   \[ p_{g,1} = 440941.12 \, \text{Pa} = 4.41 \, \text{bar} \]  

3. **Gas Mass Calculation:**  
   Using the ideal gas law:  
   \[ p_{g,1} V_g = m_g R T_g \]  
   Rearranging for \( m_g \):  
   \[ m_g = \frac{p_{g,1} V_g}{R T_g} \]  
   Substituting values:  
   \[ m_g = \frac{440941.12 \cdot 3.14 \cdot 10^{-3}}{166.28 \cdot 773.15} \]  
   \[ m_g = 3.422 \cdot 10^{-3} \, \text{kg} \]  

Final results:  
- \( p_{g,1} = 4.41 \, \text{bar} \)  
- \( m_g = 3.422 \, \text{g} \)  

No diagrams or figures are present on this page.