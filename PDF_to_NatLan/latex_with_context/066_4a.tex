The table lists the states of the freeze-drying process with the following parameters: temperature \( T \), pressure \( p \), heat \( Q \), work \( W \), enthalpy \( h \), entropy \( s \), and vapor quality \( x \).  

State descriptions:  
1. \( T_i - 6 \, \text{K} \), \( p_1 \), \( Q_K \), \( x_1 = 0.934 \) (initial vapor quality).  
2. \( -9 \, \text{K} \), \( p_1 \), \( Q = 0 \), \( h = 249.53 \), \( s = 0.9267 \), \( x_2 = 1 \) (complete evaporation).  
3. \( 8 \, \text{bar} \), \( Q = 0 \), \( x_3 = 0 \) (fully condensed refrigerant).  
4. \( 8 \, \text{bar} \), \( Q = 0 \), \( x_4 = 0 \) (final state).  

For the adiabatic expansion (throttle), the enthalpy remains constant: \( h_1 = h_4 \).  

Step ii:  
The temperature \( T_i \) remains constant. The pressure \( p_i \) is reduced to \( 1 \, \text{mbar} \), \( -6 \, \text{mbar} \), and \( -5 \, \text{mbar} \) below the triple point. Sublimation occurs.  

The temperature \( T \) is calculated as:  
\[
T = 10 \, \text{K} \, \text{above sublimation point}, \quad T_i = 10 \, \text{K} + 273.15 \, \text{K} = 283.15 \, \text{K}.
\]

---

A graph is drawn showing a pressure-temperature (\( p-T \)) diagram. The curve represents the phase boundary between liquid and vapor regions. A point is marked on the curve, indicating a specific state. The axes are labeled as \( p \) (pressure) on the vertical axis and \( T \) (temperature) on the horizontal axis.