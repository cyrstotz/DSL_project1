The following calculations are shown to determine the temperature \( T_6 \):  
- Given values:  
  \( w_5 = 220 \, \text{m/s} \), \( p_5 = 0.5 \, \text{bar} \), \( T_5 = 431.9 \, \text{K} \).  
  The process from state 5 to 6 is labeled as "isentropic."  

- The temperature ratio is calculated using the isentropic relation:  
  \[
  \frac{T_c}{T_5} = \left( \frac{p_c}{p_5} \right)^{\frac{\kappa - 1}{\kappa}}
  \]  
  Substituting values:  
  \[
  \frac{T_c}{T_5} = \left( \frac{0.191 \, \text{bar}}{0.5 \, \text{bar}} \right)^{\frac{0.4}{1.4}}
  \]  
  \[
  T_c = 328.07 \, \text{K}
  \]  

- An energy balance equation is written:  
  \[
  0 = \dot{m} \left( h_5 - h_6 + \frac{w_5^2 - w_6^2}{2} \right) + \dot{Q} - \dot{W}
  \]  
  Additional terms are included to account for enthalpy differences and velocity changes.