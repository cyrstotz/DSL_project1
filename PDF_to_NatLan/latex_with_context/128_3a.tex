The pressure exerted by the gas is calculated using the equation:  
\[
p_g = \frac{m_K g}{A} + p_0 + \frac{m_{\text{EW}} g}{A}
\]  
where:  
- \( m_K = 32 \, \text{kg} \) (mass of the piston),  
- \( m_{\text{EW}} = 0.1 \, \text{kg} \) (mass of the ice-water mixture),  
- \( g = 9.81 \, \text{m/s}^2 \) (gravitational acceleration),  
- \( A = \pi \left(\frac{D}{2}\right)^2 = \pi \left(\frac{0.05 \, \text{m}}{2}\right)^2 \) (cross-sectional area of the cylinder),  
- \( p_0 = 1 \, \text{bar} \) (ambient pressure).  

Substituting values:  
\[
p_g = \frac{32 \, \text{kg} \cdot 9.81 \, \text{m/s}^2}{\pi \left(0.05 \, \text{m}/2\right)^2} + 1 \, \text{bar} + \frac{0.1 \, \text{kg} \cdot 9.81 \, \text{m/s}^2}{\pi \left(0.05 \, \text{m}/2\right)^2}
\]  
This results in:  
\[
p_g = 1.49 \, \text{bar}.
\]  

The mass of the gas \( m_g \) is calculated using the ideal gas law:  
\[
p_g V_g = \frac{m_g R T_g}{M_g},
\]  
where:  
- \( p_g = 1.49 \, \text{bar} = 1.49 \cdot 10^5 \, \text{Pa} \),  
- \( V_g = 3.14 \, \text{L} = 3.14 \cdot 10^{-3} \, \text{m}^3 \),  
- \( R = 8.314 \, \text{J/mol·K} \),  
- \( T_g = 773 \, \text{K} \),  
- \( M_g = 50 \, \text{kg/kmol} \).  

Rearranging for \( m_g \):  
\[
m_g = \frac{p_g V_g M_g}{R T_g}.
\]  

Substituting values:  
\[
m_g = \frac{1.49 \cdot 10^5 \cdot 3.14 \cdot 10^{-3} \cdot 50}{8.314 \cdot 773}.
\]  

This results in:  
\[
m_g = 3.42 \, \text{kg}.
\]  

---