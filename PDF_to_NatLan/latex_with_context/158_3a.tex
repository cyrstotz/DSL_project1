The gas pressure \( p_{g,1} \) and mass \( m_g \) in state 1 are determined as follows:

Given:  
- \( T_{g,1} = 500^\circ\text{C} = 773.15 \, \text{K} \)  
- \( V_{g,1} = 3.14 \, \text{L} = 0.00314 \, \text{m}^3 \)  
- \( m_{\text{EW}} = 0.1 \, \text{kg} \)  

The pressure exerted by the piston is calculated using:  
\[
p_{\text{EW}} = p_{\text{amb}} + \frac{m_K \cdot g}{A}
\]  
Where:  
- \( p_{\text{amb}} = 100 \, \text{kPa} \)  
- \( m_K = 32 \, \text{kg} \)  
- \( g = 9.81 \, \text{m/s}^2 \)  
- \( A = \pi r^2 = \pi \cdot (0.05 \, \text{m})^2 = 0.007854 \, \text{m}^2 \)  

Substituting values:  
\[
p_{\text{EW}} = 100 \, \text{kPa} + \frac{32 \cdot 9.81}{0.007854} = 100 \, \text{kPa} + 355.65 \, \text{kPa} = 455.65 \, \text{kPa}
\]  

The total pressure is:  
\[
p_{g,1} = p_{\text{EW}} = 455.65 \, \text{kPa}
\]  

To find the gas mass \( m_g \), use the ideal gas law:  
\[
m_g = \frac{p V}{R T}
\]  
Where:  
- \( R = \frac{8.314 \, \text{kJ/kmol·K}}{M_g} = \frac{8.314}{18.02} = 0.4614 \, \text{kJ/kg·K} \)  
- \( p = 455.65 \, \text{kPa} \)  
- \( V = 0.00314 \, \text{m}^3 \)  
- \( T = 773.15 \, \text{K} \)  

Substituting values:  
\[
m_g = \frac{455.65 \cdot 0.00314}{0.4614 \cdot 773.15} = 0.001232 \, \text{kg}
\]  

Final results:  
\[
p_{g,1} = 455.65 \, \text{kPa}, \quad m_g = 0.001232 \, \text{kg}
\]  

No diagrams or graphs are present on this page.