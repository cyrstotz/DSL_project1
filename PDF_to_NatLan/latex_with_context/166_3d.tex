The calculation begins with determining the final ice fraction \( x_{\text{ice},2} \) in state 2.  

The formula used is:  
\[
x_{\text{ice},2} = \frac{U_2 - U_{\text{EW}}(0^\circ\text{C})}{U_{\text{ice}}(0^\circ\text{C}) - U_{\text{EW}}(0^\circ\text{C})}
\]

Substituting values:  
\[
x_{\text{ice},2} = \frac{U_2 - U_{\text{EW}}(0^\circ\text{C})}{U_{\text{ice}}(0^\circ\text{C}) - U_{\text{EW}}(0^\circ\text{C})}
\]

Breaking down further:  
\[
x_{\text{ice},2} = \frac{U_2 - U_{\text{EW}}(0^\circ\text{C})}{U_{\text{ice}}(0^\circ\text{C}) - U_{\text{EW}}(0^\circ\text{C})}
\]

Final calculation:  
\[
x_{\text{ice},2} = 0.57 \, \text{or} \, 57\%.
\]  

No diagrams or additional figures are present.

Using the first law of thermodynamics for a closed system:  
\[
\Delta U = Q + W
\]  
The heat transfer \( Q \) is equal to the change in internal energy:  
\[
Q = \Delta U = m_u \cdot u_2 - m_u \cdot u_1
\]  

The internal energy \( u_1 \) is calculated as:  
\[
u_1 = u_F + x \cdot (u_E - u_F) = -200.1168 \, \frac{\text{kJ}}{\text{kg}}
\]  

The mass \( m_u \) is determined as:  
\[
m_u = m_u + \Delta m
\]  
Substituting values:  
\[
m_u = 1500 \, \text{kg} + 0.15 \, \text{kg} \cdot (-200.1168 \, \frac{\text{kJ}}{\text{kg}})
\]  
Resulting in:  
\[
m_u = 441.882 - 28.588 \, \text{kJ}
\]