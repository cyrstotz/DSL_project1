Determine \( p_{g,1} \) and \( m_g \):  

The pressure \( p_{g,1} \) is calculated using the equilibrium condition:  
\[
A_{\text{Zyl}} = 0.05^2 \pi = \frac{\pi}{100} = 7.85 \cdot 10^{-3} \, \text{m}^2
\]  
The forces acting on the piston include the weight of the piston and the atmospheric pressure:  
\[
p_1 = \frac{32.1 \cdot 9.81}{0.05^2 \pi} + 1 \, \text{bar} = 40094.1 \, \text{Pa} + 1 \, \text{bar} = 0.40094 \, \text{bar} + 1 \, \text{bar}
\]  
Thus:  
\[
p_1 = 1.40094 \, \text{bar}
\]  

To determine the mass \( m_g \), the ideal gas law is applied:  
\[
M = \frac{p V}{R T} = \frac{1.40094 \cdot 10^5 \cdot 3.14 \cdot 10^{-3}}{8.314 \cdot \frac{50 \cdot 10^{-3}}{1} \cdot (500 + 273.15)} = 5.4 \cdot 10^{-3} \, \text{kg}
\]  
The mass of the gas is:  
\[
m_g = 0.0054 \, \text{kg}
\]  

---