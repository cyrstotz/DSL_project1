The pressure exerted by the piston \( P_{\text{EW}} \) is calculated using the formula:  
\[
P_{\text{EW}} = \frac{m_{\text{piston}} \cdot g}{A}
\]  
where \( m_{\text{piston}} = 32 \, \text{kg} \), \( g = 9.81 \, \text{m/s}^2 \), and \( A = 0.005 \, \text{m}^2 \). Substituting these values:  
\[
P_{\text{EW}} = \frac{32 \cdot 9.81}{0.005} = 62,784 \, \text{Pa} = 12.7 \, \text{kPa}.
\]  

The atmospheric pressure \( P_{\text{atm}} \) is given as \( 100 \, \text{kPa} \).  

The pressure of the gas \( P_{\text{g,1}} \) is calculated as:  
\[
P_{\text{g,1}} = P_{\text{atm}} + P_{\text{EW}} + \Delta P,
\]  
where \( \Delta P = 0.127 \, \text{kPa} \). Substituting the values:  
\[
P_{\text{g,1}} = 100 + 12.7 + 0.127 = 189.98 \, \text{kPa} = 1.89 \, \text{bar}.
\]  

The mass of the gas \( m_g \) is determined using the ideal gas law:  
\[
pV = mRT \quad \Rightarrow \quad m = \frac{pV}{RT}.
\]  
Substituting \( p = 189.98 \, \text{kPa} \), \( V = 0.00314 \, \text{m}^3 \), \( R = 50 \, \text{J/(kg·K)} \), and \( T = 773.15 \, \text{K} \):  
\[
m_g = \frac{189.98 \cdot 0.00314}{50 \cdot 773.15} = 0.01634 \, \text{kg}.
\]