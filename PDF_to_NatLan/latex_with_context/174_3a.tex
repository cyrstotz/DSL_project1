The pressure \( p_{g,1} \) and mass \( m_g \) are calculated as follows:  

The cross-sectional area of the cylinder is determined using the formula:  
\[
A = \left(\frac{D}{2}\right)^2 \pi = 7.85 \times 10^{-3} \, \text{m}^2
\]  

The pressure \( p_{g,1} \) is expressed as:  
\[
p_{g,1} A = m_K g + m_{\text{EW}} g + p_{\text{amb}} A
\]  
Rearranging for \( p_{g,1} \):  
\[
p_{g,1} = g \frac{m_K + m_{\text{EW}}}{A} + p_{\text{amb}}
\]  
Substituting values:  
\[
p_{g,1} = g \frac{32 \, \text{kg} + 0.1 \, \text{kg}}{7.85 \times 10^{-3} \, \text{m}^2} + 1 \, \text{bar}
\]  
\[
p_{g,1} = 1.401 \, \text{bar}
\]  

The ideal gas law is used to calculate the mass \( m_g \):  
\[
p V = m R T \quad \Rightarrow \quad m_g = \frac{p V}{R T}
\]  
Substituting values:  
\[
m_g = \frac{1.4 \, \text{bar} \cdot 3.14 \times 10^{-3} \, \text{m}^3 \cdot 50 \, \text{kg/kmol}}{8314 \, \text{J/(kmol·K)} \cdot 773.15 \, \text{K}}
\]  
\[
m_g = 0.0084 \, \text{kg}
\]  

The total mass of the gas is:  
\[
m_g = 3.42 \, \text{g}
\]  

---