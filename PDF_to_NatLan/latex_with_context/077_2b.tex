The task involves determining the outlet velocity \( w_6 \) and temperature \( T_6 \).  

The temperature ratio \( \frac{T_c}{T_5} \) is calculated using the pressure ratio:  
\[
\frac{T_c}{T_5} = \left( \frac{p_c}{p_5} \right)^{\frac{\kappa - 1}{\kappa}}
\]  
Substituting values:  
\[
p_c = p_0 = 0.191 \, \text{bar}, \quad p_5 = 0.5 \, \text{bar}, \quad T_5 = 431.9 \, \text{K}, \quad \kappa = 1.4
\]  
\[
\frac{T_c}{T_5} = \left( \frac{0.191}{0.5} \right)^{\frac{0.4}{1.4}} = 0.286
\]  
\[
T_c = T_5 \cdot \frac{0.191}{0.5} = 328 \, \text{K} \, (327.33 \, \text{K})
\]  

The equation for \( w_6 \) is crossed out and not used further:  
\[
w_6 = - \int v \, dp + \Delta ke
\]  

A stationary energy balance is applied:  
\[
0 = \dot{m} \left( h_e - h_a + \frac{w_5^2 - w_6^2}{2} \right)
\]  

Rearranging for \( w_6 \):  
\[
\sqrt{2(h_5 - h_6)} + w_5^2 = w_6
\]  

The enthalpy difference \( h_5 - h_6 \) is calculated using:  
\[
h_5 - h_6 = c_p (T_2 - T_1)
\]  
Substituting \( c_p = 1.006 \, \text{kJ/kg·K} \).  

No further numerical values are provided for \( w_6 \).  

No diagrams or graphs are present on the page.