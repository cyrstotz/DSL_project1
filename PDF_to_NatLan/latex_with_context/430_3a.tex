First, the pressure in the gas chamber must be calculated:  

The pressure consists of the pressure from the ice-water mixture (EW), the pressure from the weight, and the pressure from the atmosphere.  

\[
p_L = \frac{F}{A} = \frac{32 \, \text{kg} \cdot 9.81 \, \frac{\text{m}}{\text{s}^2}}{\pi \cdot (5 \cdot 10^{-2} \, \text{m})^2} = 39969.36 \, \frac{\text{N}}{\text{m}^2}
\]  

\[
p_{\text{EW}} = \frac{F}{A} = \frac{0.1 \, \text{kg} \cdot 9.81 \, \frac{\text{m}}{\text{s}^2}}{\pi \cdot (5 \cdot 10^{-2} \, \text{m})^2} = 124.304 \, \frac{\text{N}}{\text{m}^2}
\]  

\[
p_{\text{amb}} = 10^5 \, \frac{\text{N}}{\text{m}^2}
\]  

The total pressure in the gas chamber is:  
\[
p_{g,1} = p_L + p_{\text{EW}} + p_{\text{amb}} = 140093.44 \, \frac{\text{N}}{\text{m}^2}
\]  

The specific gas constant \( R \) is calculated as:  
\[
R = \frac{\bar{R}}{M} = \frac{8.314 \, \frac{\text{J}}{\text{mol·K}}}{50 \, \frac{\text{kg}}{\text{kmol}}} = 0.16628 \, \frac{\text{J}}{\text{kg·K}}
\]  

The pressure in the gas chamber is:  
\[
p_{g,1} = 1.4 \, \text{bar}
\]  

The mass of the gas \( m_g \) is calculated as:  
\[
m_g = \frac{p_{g,1} \cdot V_{g,1}}{R \cdot T_{g,1}} = \frac{1.4 \cdot 10^5 \, \frac{\text{N}}{\text{m}^2} \cdot 3.14 \cdot 10^{-3} \, \text{m}^3}{0.16628 \, \frac{\text{J}}{\text{kg·K}} \cdot 773.15 \, \text{K}} = 3.419 \, \text{g}
\]  

The final result for the gas mass is underlined:  
\[
m_g = 3.419 \, \text{g}
\]