The gas pressure \( p_{g,1} \) and mass \( m_g \) are calculated as follows:  

The total force exerted by the piston is the sum of the piston weight, atmospheric pressure, and the pressure exerted by the ice-water mixture.  
\[
F_g = m_K \cdot g = 32.169 \, \text{kg} \cdot 9.81 \, \text{m/s}^2 = 319.907 \, \text{N}
\]  
The pressure exerted by the gas is given by:  
\[
p_g = p_0 + \frac{F_g}{A} = p_{\text{atm}} + \frac{319.907}{0.05 \, \text{m}^2}
\]  
Substituting values:  
\[
p_g = 1.401 \, \text{bar}
\]  

To calculate the mass of the gas \( m_g \), the ideal gas law is used:  
\[
m_g = \frac{p_g \cdot V_{g,1}}{R_g \cdot T_{g,1}}
\]  
With \( R_g = \frac{8.314 \, \text{m}^3 \, \text{kPa}/\text{mol·K}}{50 \, \text{kg/kmol}} = 0.16623 \, \text{J/g·K} \):  
\[
m_g = \frac{1.401 \cdot 10^5 \cdot 0.00314}{0.16623 \cdot (500 + 273.15)} = 3.422 \, \text{g}
\]  

---