The problem setup includes the following parameters:  
- Airspeed \( w_{\text{air}} = 200 \, \text{m/s} \)  
- Ambient pressure \( p_0 = 0.191 \, \text{bar} \)  
- Ambient temperature \( T_0 = -30^\circ\text{C} \)  
- Heat added per unit mass \( q_B = 1195 \, \text{kJ/kg} \)  
- Mean temperature during combustion \( \bar{T}_B = 1289 \, \text{K} \)  
- Air is modeled as a perfect gas.  

A qualitative \( T \)-\( s \) diagram is drawn to represent the jet engine process.  
- The diagram shows labeled states (0, 1, 2, 3, 4, 5, 6).  
- The process begins at state 0 and progresses through compression (state 1 to state 2), combustion (state 3), turbine expansion (state 4), mixing (state 5), and nozzle exit (state 6).  
- Isobars are indicated with red lines, and the entropy axis is labeled as \( s \) in \( \text{kJ/kg·K} \).  
- The temperature axis is labeled as \( T \) in \( \text{K} \).  
- The pressure at state 2 is equal to the pressure at state 3 (\( p_2 = p_3 \)), and the pressure at state 6 equals the ambient pressure (\( p_6 = p_0 \)).