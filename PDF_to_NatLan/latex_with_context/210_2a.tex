The first diagram is a temperature-entropy (\( T \)-\( S \)) diagram. It qualitatively represents the thermodynamic processes of a jet engine. The axes are labeled as follows:  
- The vertical axis is temperature (\( T \)) in Kelvin (\( [K] \)).  
- The horizontal axis is entropy (\( S \)) in \( \left[\frac{\text{kJ}}{\text{kg·K}}\right] \).  

The diagram includes several curves:  
- A dome-shaped curve representing phase boundaries.  
- Lines crossing the dome, likely representing isobars and other thermodynamic paths.  

The second diagram is also a temperature-entropy (\( T \)-\( S \)) diagram, but it is more detailed and specific to the jet engine process. The axes are labeled similarly:  
- The vertical axis is temperature (\( T \)) in Kelvin (\( [K] \)).  
- The horizontal axis is entropy (\( S \)) in \( \left[\frac{\text{kJ}}{\text{kg·K}}\right] \).  

This diagram includes labeled states (0, 1, 2, 3, 4, 5, and 6), representing different points in the jet engine cycle.  
- The process between states 0 and 1 is marked as \( S \neq \text{const} \), indicating a non-isentropic process.  
- The process between states 1 and 2 is marked as \( S = \text{const} \), indicating an isentropic process.  
- States 3, 4, and 5 show additional thermodynamic transitions, with state 6 returning to ambient conditions (\( p_0 \)).  

The student has added a note below the second diagram:  
"Zustand 4 liegt nicht per se auf der p5 Isobare," which translates to "State 4 does not necessarily lie on the \( p_5 \) isobar."