The following calculations are performed to determine the outlet velocity \( w_6 \) and temperature \( T_6 \) for the jet engine:

### Given Data:
- \( c_p = 1.006 \, \frac{\text{kJ}}{\text{kg·K}} \)  
- \( \kappa = 1.4 \)  
- \( p_e = p_0 = 0.191 \, \text{bar} = 19100 \, \text{Pa} \)  
- \( w_5 = 220 \, \frac{\text{m}}{\text{s}} \)  
- \( p_5 = 0.5 \, \text{bar} \)  
- \( T_5 = 431.9 \, \text{K} \)  

---

### Step 1: Calculate \( T_c \) using the adiabatic coefficient  
The temperature \( T_c \) is calculated as:  
\[
T_c = T_5 \left( \frac{p_c}{p_5} \right)^{\frac{\kappa - 1}{\kappa}}
\]  
Substituting values:  
\[
T_c = 431.9 \, \text{K} \left( \frac{0.191 \, \text{bar}}{0.5 \, \text{bar}} \right)^{\frac{0.4}{1.4}}
\]  
\[
T_c = 431.9 \, \text{K} \cdot \left( 0.382 \right)^{0.2857} = 328.07 \, \text{K}
\]  

---

### Step 2: Apply steady flow energy equation for \( w_6 \)  
The steady flow process is modeled as an adiabatic, reversible process. The energy equation is:  
\[
O = \dot{m} \left[ h_5 - h_6 + \frac{(w_5)^2 - (w_6)^2}{2} + p_e \cdot O \right]
\]  
Simplifying:  
\[
O = h_5 - h_6 + \frac{w_5^2 - w_6^2}{2}
\]  

---

### Step 3: Relate enthalpy difference to temperature difference  
Using \( h_5 - h_6 = c_p \Delta T = c_p (T_5 - T_6) \):  
\[
c_p (T_5 - T_6) = \frac{w_5^2 - w_6^2}{2}
\]  
Rearranging:  
\[
c_p (T_5 - T_6) + \frac{w_6^2}{2} = \frac{w_5^2}{2}
\]  
\[
w_6^2 = 2 c_p (T_5 - T_6) + w_5^2
\]  

---

### Step 4: Solve for \( w_6 \)  
Substitute values:  
\[
w_6^2 = 2 \cdot 1.006 \cdot (431.9 - 328.1) + (220)^2
\]  
\[
w_6^2 = 2 \cdot 1.006 \cdot 103.8 + 48400
\]  
\[
w_6^2 = 208.7 + 48400 = 48608.7
\]  
\[
w_6 = \sqrt{48608.7} \approx 507.2 \, \frac{\text{m}}{\text{s}}
\]  

---

### Final Results:
- Outlet velocity:  
\[
w_6 \approx 507.2 \, \frac{\text{m}}{\text{s}}
\]  
- Outlet temperature:  
\[
T_6 = T_c = 328.07 \, \text{K}
\]  

No diagrams or graphs are present on the page.