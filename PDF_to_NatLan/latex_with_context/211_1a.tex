The problem involves determining the heat flow \( \dot{Q}_{\text{ab}} \) removed by the coolant in a reactor system. The reactor operates between two states:  
- State 1: \( T_1 = 70^\circ\text{C} \)  
- State 2: \( T_2 = 100^\circ\text{C} \)  

The energy balance equation is given as:  
\[
0 = \dot{m}(h_2 - h_1) + \dot{Q}_{\text{ab}} \implies \dot{Q}_{\text{ab}} = \dot{m}(h_2 - h_1) = \dot{Q}_{\text{aus}} + \dot{Q}_R
\]  
where \( \dot{Q}_R \) remains in the system.  

The enthalpy values are calculated as follows:  
- \( h_2 \): Only the liquid portion is considered at \( T = 100^\circ\text{C} \), giving \( h_2 = 418.04 \, \text{kJ/kg} \).  
- \( h_1 \): At \( T = 70^\circ\text{C} \), \( h_1 = 292.88 \, \text{kJ/kg} \).  

Substituting into the equation:  
\[
\dot{Q}_{\text{ab}} = \dot{m}(h_2 - h_1) = 0.3 \, \text{kg/s} \cdot (418.04 - 292.88) \, \text{kJ/kg} = 107.818 \, \text{kW}
\]  

---