The cross-sectional area of the cylinder is calculated as:  
\[
A = \left(\frac{0.1}{2}\right)^2 \cdot \pi = 7.8538 \cdot 10^{-3} \, \text{m}^2
\]  

The problem asks to determine \( p_{g,1} \) and \( m_g \). The given values are:  
\[
T_{g,1} = 773.15 \, \text{K}, \quad V_{g,1} = 3.14 \cdot 10^{-3} \, \text{m}^3, \quad m_K = 32 \, \text{kg}, \quad m_{\text{EW}} = 0.1 \, \text{kg}
\]  
The ideal gas law is used:  
\[
p V = m R T
\]  
The specific gas constant is calculated as:  
\[
R = \frac{1}{M_g} \cdot \bar{R} = \frac{1}{50} \cdot 8314 = 166.28 \, \text{J/kg·K}
\]  

The pressure \( p_{g,1} \) is derived using the equilibrium of forces:  
\[
p_{g,1} = p_{\text{amb}} + \frac{m_K + m_{\text{EW}}}{A} \cdot g
\]  
Substituting values:  
\[
p_{g,1} = 100000 + \frac{32 + 0.1}{7.8538 \cdot 10^{-3}} \cdot 9.81 = 140058.8573 \, \text{Pa}
\]  

The mass \( m_g \) is calculated using the ideal gas law:  
\[
m_g = \frac{p_{g,1} \cdot V}{R \cdot T_{g,1}}
\]  

---