The temperature decreases from \( T_1 = 100^\circ\text{C} \) to \( T_2 = 70^\circ\text{C} \), resulting in \( \Delta T = 30 \, \text{K} \). The heat released during this cooling process is given as \( Q_R = Q_{\text{out}} = 35 \, \text{MJ} \).  

The specific internal energy values are calculated using water tables:  
\[
u_f(100^\circ\text{C}) = 419.94 \, \text{kJ/kg}, \quad u_f(70^\circ\text{C}) = 293.25 \, \text{kJ/kg}
\]  

The energy balance equation is written as:  
\[
Q_{\text{out}} = m_{\text{total},1} \cdot u_f(70^\circ\text{C}) + m_{\text{added}} \cdot u_f(20^\circ\text{C}) - m_{\text{total},1} \cdot u_f(100^\circ\text{C})
\]  

Rearranging for \( m_{\text{added}} \):  
\[
m_{\text{added}} = \frac{Q_{\text{out}} - m_{\text{total},1} \cdot (u_f(70^\circ\text{C}) - u_f(100^\circ\text{C}))}{u_f(20^\circ\text{C}) - u_f(70^\circ\text{C})}
\]  

Substituting values:  
\[
u_f(20^\circ\text{C}) = 83.95 \, \text{kJ/kg}, \quad m_{\text{total},1} = 5755 \, \text{kg}
\]  

\[
m_{\text{added}} = \frac{35,000 - 5755 \cdot (293.25 - 419.94)}{83.95 - 293.25}
\]  

The calculated mass \( m_{\text{added}} \) is approximately \( 3589.106 \, \text{kg} \).  

---