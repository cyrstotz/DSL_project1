Two diagrams are drawn.  

1. The first diagram is a pressure-volume (\(P\)-\(v\)) graph showing a cycle with labeled states. The cycle includes compression, expansion, and heat transfer processes. The states are marked as 1, 2, and 3, with arrows indicating the direction of the cycle.  
2. The second diagram is an enthalpy-pressure (\(h\)-\(P\)) graph showing the refrigerant cycle. States 1, 2, and 3 are labeled, and the processes include isobaric heat transfer and adiabatic compression.

Two diagrams are drawn to represent thermodynamic processes:  

1. The first diagram is a pressure-enthalpy (\(p\)-\(h\)) graph. It shows two isobaric lines labeled \(p_1\) and \(p_2\), with enthalpy increasing along the x-axis. The diagram includes points labeled \(1\), \(2\), and \(A\), connected by curves. The process transitions from \(1\) to \(2\) along an isobaric line, and from \(2\) to \(A\) along another isobaric line.  

2. The second diagram is a temperature-enthalpy (\(T\)-\(h\)) graph. It shows two isothermal lines labeled \(T_1\) and \(T_2\), with enthalpy increasing along the x-axis. Points \(1\), \(2\), and \(A\) are connected by curves, representing transitions between states.