The pressure \( p_{g,1} \) and mass \( m_g \) are calculated as follows:  

The pressure \( p_{g,1} \) is determined using the equation:  
\[
p_{g,1} \cdot A = m_g \cdot g + m_{\text{EW}} \cdot g + p_{\text{amb}} \cdot A
\]  
Rearranging:  
\[
p_{g,1} = \frac{1}{A} \left( m_g \cdot g + m_{\text{EW}} \cdot g \right) + p_{\text{amb}}
\]  
Substituting values:  
\[
A = \pi \cdot (0.1)^2 \quad \text{(area of the cylinder cross-section)}  
\]  
\[
p_{\text{amb}} = 100,000 \, \text{N/m}^2  
\]  
\[
p_{g,1} = 110,023.6 \, \text{N/m}^2  
\]  

The mass \( m_g \) is calculated using the ideal gas law:  
\[
m = \frac{p \cdot V}{R \cdot T}
\]  
Substituting values:  
\[
T = 273.15 \, \text{K}  
\]  
\[
R = \frac{8.314}{50} = 0.16628 \, \text{kJ/(kg·K)} = 166.28 \, \text{J/(kg·K)}  
\]  
\[
V = 3.14 \cdot 10^{-3} \, \text{m}^3  
\]  
\[
m = 0.0027 \, \text{kg}  
\]  

---