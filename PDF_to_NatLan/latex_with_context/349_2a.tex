The page contains two diagrams related to thermodynamic processes.

1. **First Diagram**:  
   - The graph is labeled with \( T \, [K] \) on the vertical axis and \( S \, [\frac{\text{kJ}}{\text{kg·K}}] \) on the horizontal axis.  
   - It appears to be a qualitative representation of a thermodynamic cycle on a temperature-entropy (\( T \)-\( S \)) diagram.  
   - The cycle includes numbered points (1, 2, 3, 4) connected by arrows indicating the direction of the process.  
   - The graph includes curved and straight lines, suggesting different thermodynamic processes such as isobaric or adiabatic transitions.  

2. **Second Diagram**:  
   - This graph is also labeled with \( T \, [K] \) on the vertical axis and \( S \, [\frac{\text{kJ}}{\text{kg·K}}] \) on the horizontal axis.  
   - It depicts a closed thermodynamic cycle with five numbered points (1, 2, 3, 4, 5) connected by arrows.  
   - Two segments are explicitly labeled as "isobar," indicating constant pressure processes.  
   - The cycle forms a polygonal shape, representing a sequence of thermodynamic states.  

Both diagrams are qualitative and do not include numerical values or specific equations. They visually describe the thermodynamic processes occurring in the system.

The exergy balance equation is expressed as:  
\[
E_{x,\text{vol}} = \dot{m} \left[ h_e - h_a - T_0 (s_e - s_a) + \delta h_e \right] + \sum_i \left( 1 - \frac{T_0}{T_i} \right) \dot{Q}_j = e_{x,\text{vol}} \cdot \dot{m}
\]

Setting the equation to zero:  
\[
0 = \dot{m} \left[ h_e - h_a + \delta h_e \right] + \sum_j \dot{Q}_j
\]

From this, the exergy destruction term can be derived:  
\[
E_{x,\text{vol}} = \dot{m} \left[ -T_0 (s_e - s_a) \right] + \sum_i \left( 1 - \frac{T_0}{T_i} \right) \dot{Q}_j - \sum_j \dot{Q}_j
\]

Simplifying further:  
\[
e_{x,\text{vol}} = -T_0 (s_e - s_a) + \frac{\sum_i \left( 1 - \frac{T_0}{T_i} \right) \dot{Q}_j}{\dot{m}} - \frac{\sum_j \dot{Q}_j}{\dot{m}}
\]