The coefficient of performance \( \epsilon_K \) is calculated as:  
\[
\epsilon_K = \frac{\dot{Q}_K}{\dot{W}_K}
\]  
This involves the heat transfer \( \dot{Q}_K \) and work \( \dot{W}_K \).  

The work \( W_K \) is expressed as:  
\[
W_K = \dot{m} \cdot \left( h_3 - h_4 \right) - \dot{Q}_{AB}
\]  
Thus, \( \epsilon_K \) can be rewritten as:  
\[
\epsilon_K = \frac{\dot{Q}_K}{\dot{m} \cdot \left( h_3 - h_4 \right) - \dot{Q}_{AB}}
\]

The coefficient of performance \( \epsilon_K \) is calculated using the formula:  
\[
\epsilon_K = \frac{\dot{Q}_{\text{cool}}}{\dot{W}_K} = \frac{\dot{W}_K + \dot{m}_{\text{R134a}} (h_{2} - h_{1})}{\dot{m}_{\text{R134a}} (h_{2} - h_{1})}
\]  
Substituting the values:  
\[
\epsilon_K = \frac{128 \, \text{J/s} + 1.3333 \, \text{g/s} \cdot (264.25 - 93.42) \, \text{J/g}}{1.3333 \, \text{g/s} \cdot (264.25 - 93.42) \, \text{J/g}}
\]  
\[
\epsilon_K = \frac{128 + 1.3333 \cdot 170.83}{1.3333 \cdot 170.83}
\]  
\[
\epsilon_K = 4.228
\]