The diagram is a pressure-temperature (\( p \)-\( T \)) graph. It shows the phase regions for the freeze-drying process. The curve represents the phase boundary, with points labeled \( i \) and arrows indicating transitions. The x-axis is labeled \( T(\text{K}) \), and the y-axis is labeled \( p(\text{N/m}^2) \).

No content found.

A graph is drawn representing the freeze-drying process in a pressure-temperature (\(p\)-\(T\)) diagram. The axes are labeled as follows:  
- The vertical axis represents pressure (\(p\)), with arrows indicating increasing pressure.  
- The horizontal axis represents temperature (\(T\)).  

The graph includes curves that depict phase transitions and regions corresponding to different phases (solid, liquid, and vapor). The curves are qualitatively drawn to show the relationships between pressure and temperature during the freeze-drying process.

---