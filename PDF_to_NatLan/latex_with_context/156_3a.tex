The pressure \( p_{g,1} \) and mass \( m_g \) of the gas in state 1 are determined using the following equations:  

\[
(m_K + m_{\text{EW}}) g + p_{\text{amb}} A = p_{g,1} A
\]

The area \( A \) is calculated as:  
\[
A = \left(\frac{D}{2}\right)^2 \pi - \frac{\pi}{400}
\]

Substituting values, the pressure \( p_{g,1} \) is found to be:  
\[
p_{g,1} = \frac{(m_K + m_{\text{EW}}) g}{A} + p_{\text{amb}} = 1.401 \, \text{bar}
\]

The ideal gas law is used to calculate the mass \( m_g \):  
\[
p_{g,1} V_{g,1} = m_g R T_{g,1}
\]

The gas constant \( R \) is calculated as:  
\[
R = \frac{\bar{R}}{M_g} = \frac{166.28}{50} = 3.3256 \, \text{J/(g·K)}
\]

Rearranging for \( m_g \):  
\[
m_g = \frac{p_{g,1} V_{g,1}}{R T_{g,1}}
\]

Substituting values:  
\[
m_g = \frac{1.401 \cdot 3.14}{3.3256 \cdot 773.15} = 3.427 \, \text{g}
\]

---