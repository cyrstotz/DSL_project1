The reactor operates with the following parameters:  
- Inlet temperature: \( T_{\text{in}} = 70^\circ\text{C} \)  
- Outlet temperature: \( T_{\text{out}} = 100^\circ\text{C} \)  
- Coolant inlet temperature: \( T_{\text{KF,in}} = 288.15 \, \text{K} \)  
- Coolant outlet temperature: \( T_{\text{KF,out}} = 298.15 \, \text{K} \)  
- Reactor mass: \( m_{\text{ges}} = 5755 \, \text{kg} \)  
- Steam quality: \( x_D = 0.005 \)  
- Heat released by the chemical reaction: \( \dot{Q}_R = 100 \, \text{kW} \)  

The pressure in the cooling jacket is constant, and the coolant is modeled as an ideal liquid. Water tables are referenced for calculations.  

The task is to determine the heat flow \( \dot{Q}_{\text{out}} \) removed by the coolant.  

Using the steady-state energy balance:  
\[
0 = \dot{m}_{\text{in}} (h_e - h_a) + \dot{m}_{\text{out}} (h_e - h_a) + \dot{Q}_R + \dot{Q}_{\text{out}} - \dot{W}
\]  
Since \( \dot{m}_{\text{in}} = \dot{m}_{\text{out}} \), this simplifies to:  
\[
\dot{Q}_{\text{out}} = \dot{m}_{\text{in}} (h_e - h_a) + \dot{Q}_R
\]  

Substituting values:  
\[
\dot{Q}_{\text{out}} = 2 \cdot 0.3 \, \text{kg/s} \cdot (292.98 - 419.04) \, \text{kJ/kg} + 100 \, \text{kW}
\]  
\[
\dot{Q}_{\text{out}} = -26.76 \, \text{kW} + 100 \, \text{kW} = 26.36 \, \text{kW}
\]