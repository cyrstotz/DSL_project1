The page contains a description and diagrams related to the jet engine exergy problem.  

### Text Description:  
- **Stages of the process:**  
  - **0–4:** Irreversible compression.  
  - **1–2:** Reversible compression.  
  - **2–3:** Isobaric combustion.  
  - **3–4:** Turbine operation, irreversible.  
  - **4–5:** Mixing chamber at \( 0.5 \, \text{bar} \).  
  - **5–6:** Nozzle operation, expanding from \( 0.5 \, \text{bar} \) to \( 0.191 \, \text{bar} \), isobaric.  

### Graphs:  
1. **First Graph:**  
   - A qualitative \( T \)-\( s \) diagram is drawn.  
   - The graph shows multiple curves representing different processes.  
   - The temperature \( T \) is plotted on the vertical axis, and entropy \( s \) is plotted on the horizontal axis.  
   - The curves represent the thermodynamic processes of the jet engine, including compression, combustion, turbine operation, and nozzle expansion.  

2. **Second Graph:**  
   - Another \( T \)-\( s \) diagram is shown with labeled states \( 0, 1, 2, 3, 4, 5, 6 \).  
   - Isobars are drawn and labeled as \( p_0 = p_6 \), \( p_1 \), \( p_2 = p_3 \), \( p_4 = p_5 \).  
   - The diagram visually represents the thermodynamic processes, including compression, combustion, turbine operation, mixing, and nozzle expansion.  

### Additional Notes:  
- The text mentions "irreversible" and "reversible" processes for compression and turbine operation.  
- The mixing chamber and nozzle expansion are described as isobaric processes.  

No further numerical calculations or detailed explanations are visible on the page.

The outlet velocity \( w_6 \), heat \( Q \), and temperature \( T_6 \) are being calculated.  

Given:  
\[
w_5 = 220 \, \text{m/s}, \quad p_5 = 0.5 \, \text{bar}, \quad T_5 = 431.9 \, \text{K}
\]  

The relationship between \( T_6 \) and \( T_5 \) is derived using the isentropic relation:  
\[
\frac{T_6}{T_5} = \left( \frac{p_6}{p_5} \right)^{\frac{\kappa-1}{\kappa}}
\]  
Substituting values:  
\[
T_6 = \left( \frac{p_0}{p_5} \right)^{\frac{\kappa-1}{\kappa}} = 328.07 \, \text{K}
\]  

The outlet velocity \( w_6 \) is calculated using the energy balance:  
\[
h_5 + \frac{w_5^2}{2} = h_6 + \frac{w_6^2}{2}
\]  
Rearranging:  
\[
\frac{w_6^2}{2} = h_5 - h_6 + \frac{w_5^2}{2}
\]  
Substituting \( h_5 - h_6 = c_p(T_5 - T_6) \):  
\[
\frac{w_6^2}{2} = c_p(T_5 - T_6) + \frac{w_5^2}{2}
\]  
\[
w_6 = \sqrt{2 \left[ c_p(T_5 - T_6) + \frac{w_5^2}{2} \right]}
\]  
Final result:  
\[
w_6 = 507.25 \, \text{m/s}
\]  

---