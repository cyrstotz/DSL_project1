The following equations and calculations are provided:

1. **Gas Pressure Equation:**  
   \[
   p_{\text{gas}} = \frac{m_{\text{gas}} \cdot g}{\left(\frac{D}{2}\right)^2 \cdot \pi} + p_{\text{amb}}
   \]  
   Substituting values, \( p_{\text{gas}} = 1.40 \, \text{bar} \).

2. **Specific Gas Constant:**  
   \[
   R_{\text{gas}} = \frac{R}{M_g} = 0.166 \, \frac{\text{kJ}}{\text{kg·K}}
   \]

3. **Gas Mass Calculation:**  
   \[
   m_g = \frac{p_{\text{gas}} \cdot V_1}{R_{\text{gas}} \cdot T_1}
   \]  
   Substituting values, \( m_g = 3.419 \, \text{g} \).

---

Additional Notes:  
- A small annotation defines \( p_a = \frac{N}{m^2} \), which is the unit of pressure.  
- The calculations are clear and consistent with the problem setup.

Since all the heat \( Q_{12} \) flows into the process of melting ice and some ice remains, it follows that \( T_{2,\text{ice}} = 0^\circ\text{C} \). Consequently, \( T_{2,\text{gas}} = 0^\circ\text{C} \) as well (due to thermal equilibrium). The pressure remains constant in this configuration:  
\[
p_{g,2} = p_{\text{amb}} = 1.4 \, \text{bar}.
\]

---