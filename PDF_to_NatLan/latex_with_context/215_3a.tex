The ideal gas law is given as:  
\[
p \cdot V = R \cdot T
\]  
where \( R = \frac{\bar{R}}{M} \), with \( \bar{R} = 8.314 \, \text{J/(mol·K)} \) and \( M = 50 \, \text{kg/kmol} \). Substituting, \( R = 0.16628 \, \text{J/(K·g)} \).  

The equation can also be expressed as:  
\[
p \cdot V = m \cdot R \cdot T
\]  
where \( m = \frac{p \cdot V}{R \cdot T} \).  

To calculate pressure from force, the formula is:  
\[
p = \frac{F}{A}
\]  
The area \( A \) is calculated as:  
\[
A = \pi r^2 = \pi \cdot (0.05)^2 = 0.0078 \, \text{m}^2
\]  

The force \( F \) is given by:  
\[
F = m \cdot g
\]  
Substituting \( m = 32 + 0.1 = 32.1 \, \text{kg} \), we find:  
\[
F = 32.1 \cdot 9.81 = 314.9 \, \text{N}
\]  
Thus, the pressure is:  
\[
p = \frac{F}{A} = \frac{314.9}{0.0078} = 40 \, \text{kPa}
\]  

The total gas pressure is:  
\[
p_{\text{gas}} = p_{\text{amb}} + p = 1.4 \, \text{bar}
\]  

---