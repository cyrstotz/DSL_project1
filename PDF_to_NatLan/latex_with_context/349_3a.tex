The gas pressure \( p_{1,g} \) is calculated using the ideal gas law:  
\[
p_{1,g} = \frac{m_g \, R \, T_{g,1}}{V_{g,1}}
\]  

The gas constant \( R \) is determined as:  
\[
R = \frac{\bar{R}}{M} = \frac{8.314 \, \text{J/mol·K}}{166.28 \, \text{g/mol}}
\]  

The mass of the gas \( m_g \) is calculated as:  
\[
m_g = \frac{p_{1,g} \, V_{g,1}}{R \, T_{g,1}}
\]  

The pressure \( p_{1,g} \) is also expressed as:  
\[
p_{1,g} = \frac{\pi}{4} D^2 \cdot (m_K + m_{\text{EW}}) + p_{\text{amb}}
\]  
Substituting values, \( p_{1,g} = 1.00 \, \text{bar} \).  

The mass of the gas is found to be:  
\[
m_g = 5.84 \, \text{g}
\]