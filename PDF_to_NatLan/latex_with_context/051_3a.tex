The universal gas constant \( R \) is calculated as:  
\[
R = \frac{\bar{R}}{M} = \frac{8.314 \, \text{J/(mol·K)}}{50 \, \text{kg/mol}} = 0.1663 \, \text{J/(g·K)} = 166.29 \, \text{J/(kg·K)}.
\]

The gas pressure \( p_{g,1} \) is determined using the formula:  
\[
p_{g,1} = p_{\text{amb}} + \frac{m_K g}{A}.
\]

The total mass exerting pressure is:  
\[
m_{\text{total}} = m_K + m_{\text{EW}} = 32 \, \text{kg} + 0.1 \, \text{kg} = 32.1 \, \text{kg}.
\]

The cross-sectional area of the cylinder is calculated as:  
\[
A = \pi r^2 = \pi (0.05 \, \text{m})^2 = 0.007854 \, \text{m}^2.
\]

Substituting values:  
\[
p_{g,1} = 1.10 \, \text{bar} + \frac{32.1 \, \text{kg} \cdot 9.81 \, \text{m/s}^2}{0.007854 \, \text{m}^2} = 1.910 \, \text{bar}.
\]

The mass of the gas \( m_g \) is calculated using the ideal gas law:  
\[
m_g = \frac{p V}{R T}.
\]

Substituting values:  
\[
m_g = \frac{1.9 \cdot 10^5 \, \text{Pa} \cdot 3.14 \cdot 10^{-3} \, \text{m}^3}{166.29 \, \text{J/(kg·K)} \cdot 773.15 \, \text{K}} = 0.00394 \, \text{kg}.
\]

Thus, the mass of the gas is:  
\[
m_g = 3.94 \, \text{g}.
\]

A small diagram is drawn showing a circle labeled with a radius \( r = 5 \, \text{cm} \) and diameter \( d = 0.05 \, \text{m} \).

---