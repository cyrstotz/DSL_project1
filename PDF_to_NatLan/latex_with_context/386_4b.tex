The enthalpy at state \(q\) is given as:  
\[
h_q = 93.92 \, \frac{\text{kJ}}{\text{kg}} \quad (\text{from Table A-11})
\]  
The enthalpy after throttling remains constant:  
\[
h_1 = h_q
\]  
This is noted with the explanation: "Throttling is isenthalpic."  

The inlet temperature \(T_i\) is determined from the diagram as:  
\[
T_i = -10^\circ\text{C}
\]  
The outlet temperature \(T_{w2}\) is calculated as:  
\[
T_{w2} = T_i - 6 \, \text{K} = -16^\circ\text{C}
\]  

The enthalpy at state \(h_2\) is calculated using tabulated values:  
\[
h_2 = 49.54 \, \frac{\text{kJ}}{\text{kg}} + 293.66 \, \frac{\text{kJ}}{\text{kg}} - 269.15 \, \frac{\text{kJ}}{\text{kg}} = 237.71 \, \frac{\text{kJ}}{\text{kg}}
\]  

The entropy at state \(s_2\) is calculated as:  
\[
s_2 = 6.2258 \, \frac{\text{kJ}}{\text{kg·K}} - s_3
\]

The work rate \( \dot{W}_K \) is calculated using:  
\[
\dot{W}_K = \dot{m} \cdot (h_3 - h_2)
\]  
The mass flow rate \( \dot{m} \) is determined as:  
\[
\dot{m} = \frac{\dot{W}_K}{h_3 - h_2} = \frac{0.000839 \, \frac{\text{kg}}{\text{s}}}{5} = 3.0035 \, \frac{\text{kg}}{\text{h}}
\]