The ice fraction \( x_1 \) is given as \( x_1 = 0.6 \) at \( T = 0^\circ \text{C} \).  
The specific internal energy \( u_1 \) is calculated using the equation:  
\[
u_1 = -0.045 + 0.6 \cdot (-333.458 + 0.045) = -200.0828 \, \frac{\text{J}}{\text{kg}}
\]

Using the energy balance equation:  
\[
\Delta u_{12} = \frac{Q_{12}}{m} = \frac{1076.9}{0.1} = 10769 \, \frac{\text{J}}{\text{kg}}
\]

The specific internal energy at state 2, \( u_2 \), is calculated as:  
\[
u_2 = u_1 + \Delta u_{12} = -189.324 + 10769 = -189.324 \, \frac{\text{J}}{\text{kg}}
\]

The ice fraction \( x_2 \) is determined using the equation:  
\[
x = \frac{u_2 - u_w}{u_i - u_w}
\]

Substituting values:  
\[
x = \frac{-189.324 + 0.045}{-333.458 + 0.045} = 0.5677
\]

Thus, the final ice fraction is:  
\[
x_2 = 0.568
\]