The pressure \( p_1 \) is calculated as the sum of the piston pressure, the force due to the piston weight, and the force due to atmospheric pressure acting on the piston area.  

The piston area \( A \) is determined as:  
\[
A = 0.05 \, \text{m}^2 \cdot \pi = 23 \, \text{cm}^2
\]  

The pressure \( p_1 \) is calculated as:  
\[
p_1 = 1 \, \text{bar} + \frac{32 \, \text{kg} \cdot 9.81 \, \text{m/s}^2}{23 \cdot 10^{-4} \, \text{m}^2} + \frac{0.1 \, \text{kg} \cdot 9.81 \, \text{m/s}^2}{23 \cdot 10^{-4} \, \text{m}^2}
\]  
\[
p_1 = 1 \, \text{bar} + 35966.39 \, \text{N/m}^2 + 724.83 \, \text{N/m}^2
\]  
\[
p_1 = 1.490 \, \text{bar}
\]  

The mass of the gas \( m_g \) is calculated using the ideal gas law:  
\[
p V = m R T
\]  
Rearranging for \( m_g \):  
\[
m_g = \frac{p_1 V_{g,1}}{R T_{g,1}}
\]  

Substituting values:  
\[
m_g = \frac{1.490 \cdot 10^5 \, \text{Pa} \cdot 3.14 \cdot 10^{-3} \, \text{m}^3}{287.05 \, \text{J/(kg·K)} \cdot 773.15 \, \text{K}}
\]  
\[
m_g = 0.003479 \, \text{kg} = 3.479 \, \text{g}
\]