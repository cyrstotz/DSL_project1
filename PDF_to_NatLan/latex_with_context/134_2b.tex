The task involves determining \( w_6 \) and \( T_6 \).  

1. **Isentropic Relation**:  
   Using the isentropic relation:  
   \[
   u = h = 1.4 \quad \Rightarrow \quad T_c = T_5 \left( \frac{p_c}{p_5} \right)^{\frac{1}{\kappa}}
   \]  
   Substituting values:  
   \[
   T_c = T_5 \left( \frac{5}{0.5} \right)^{\frac{1-1.4}{1.4}} = 328.675 \, \text{K}
   \]  

2. **Energy Balance**:  
   The energy balance equation is written as:  
   \[
   \dot{m} \cdot \text{input} \cdot (T_5 - T_c) + \frac{w_5^2 - w_c^2}{2} + R(T_2 - T_n) = 0
   \]  

3. **Gas Constant \( R \)**:  
   The relation between \( c_p \) and \( c_v \) is given:  
   \[
   \frac{c_p}{c_v} = 1.4 \quad \Rightarrow \quad c_v = \frac{c_p}{1.4} = 0.718 \, \text{kJ/kg·K}
   \]  
   The gas constant \( R \) is calculated as:  
   \[
   R = c_p - c_v = 0.287 \, \text{kJ/kg·K}
   \]  

4. **Final Relation**:  
   The task ends with a placeholder for \( T_R \):  
   \[
   T_R: \text{in?} = 
   \]  

No further calculations or conclusions are visible.  

No additional diagrams or figures are present beyond the \( T \)-\( s \) diagram described above.