The diagram at the top of the page represents a reactor system with labeled inlet and outlet streams. The horizontal lines indicate temperature levels, with "ein" (inlet) and "sieden" (boiling) marked.  

The reactor operates under the following conditions:  
- Mass flow rate at the inlet: \( \dot{m}_{\text{in}} = 0.3 \, \text{kg/s} \).  
- Inlet temperature: \( T_{\text{in}} = 70^\circ\text{C} \), which corresponds to saturated liquid.  
- Total reactor mass: \( m_{\text{total},1} = 5755 \, \text{kg} \).  
- Steam quality: \( x_D = 0.005 \).  
- Reactor temperature: \( T = 100^\circ\text{C} \), which is also saturated liquid.

The mean temperature difference \( \bar{T} \) is expressed as:  
\[
\bar{T} = \int_{s_e}^{s_a} T \, ds = \frac{\dot{Q}_{\text{out}}}{\dot{m}}
\]  
This simplifies to:  
\[
\bar{T} = \frac{T_2 - T_1}{\ln\left(\frac{T_2}{T_1}\right)}
\]  
where \( T_2 \) and \( T_1 \) are the temperatures at the respective states. This assumes ideal fluid behavior.