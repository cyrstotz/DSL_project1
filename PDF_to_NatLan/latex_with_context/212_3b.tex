The problem involves determining \( T_{g,2} \), \( p_{g,2} \), and justifying the equilibrium condition.  

Given values:  
- \( p_0 = 1.56 \, \text{bar} \)  
- \( m_g = 3.6 \, \text{g} \)  

The ideal gas law is applied again:  
\[
p \cdot V = m \cdot R \cdot T
\]  

It is stated that \( p_2 = p_1 \), and the pressure must remain constant because the system is in equilibrium and the weight of the piston does not change.  

The temperature \( T \) is calculated using:  
\[
T = \frac{p \cdot V}{m \cdot R}
\]  

Additional explanation: "The pressure remains constant because the system is in equilibrium, and the weight of the piston does not change after the air pressure is adjusted."  

No further numerical results are provided for \( T_{g,2} \) or \( p_{g,2} \).  

---  
No diagrams or graphs are present on the page.