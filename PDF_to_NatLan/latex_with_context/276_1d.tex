Energy balance for the reactor system is derived as follows:  

\[
\Delta E = \Delta U = \Delta m_{12} \cdot h_{\text{ein}} + Q_{R,12}
\]

Expanding the equation:  
\[
\Delta m_{12} \cdot h_{\text{ein}} + Q_{R,12} = m_{\text{ges}} \cdot U_2 - m_{\text{ges}} \cdot U_1
\]

Rearranging to solve for \( \Delta m_{12} \):  
\[
\Delta m_{12} = \frac{m_{\text{ges}} \cdot (U_1 - U_2) + Q_{R,12}}{U_2 - h_{\text{ein}}}
\]

From Table A-2, the following values are used:  
\[
U_1 (100^\circ\text{C}, x=0) = 418.29 \, \frac{\text{kJ}}{\text{kg}}
\]  
\[
U_2 (70^\circ\text{C}, x=0) = 292.35 \, \frac{\text{kJ}}{\text{kg}}
\]  
\[
h_{\text{ein}} (20^\circ\text{C}, x=0) = 83.96 \, \frac{\text{kJ}}{\text{kg}}
\]

The total reactor mass is given as:  
\[
m_{\text{ges}} = 5755 \, \text{kg}
\]

Substituting these values into the equation:  
\[
\Delta m_{12} = \frac{m_{\text{ges}} \cdot (U_1 - U_2) + Q_{R,12}}{U_2 - h_{\text{ein}}}
\]

---