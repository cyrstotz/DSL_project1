The problem involves calculating the force \( F \), pressure \( p_1 \), and mass \( m_g \) in state 1 using the given parameters.  

### Force Calculation  
The total force \( F \) exerted by the piston and the ice-water mixture is calculated as:  
\[
F = (m_K + m_{\text{EW}}) \cdot g
\]  
Substituting the values:  
\[
F = (32 \, \text{kg} + 0.1 \, \text{kg}) \cdot 9.81 \, \frac{\text{N}}{\text{kg}}
\]  
\[
F = 314.9 \, \text{N}
\]  

### Membrane Force  
The force exerted by the membrane \( F_{\text{mem}} \) is calculated as:  
\[
F_{\text{mem}} = p_{\text{amb}} \cdot A
\]  
Substituting the values:  
\[
F_{\text{mem}} = 10^5 \, \frac{\text{N}}{\text{m}^2} \cdot \pi \cdot (0.05 \, \text{m})^2
\]  
\[
F_{\text{mem}} = 1.5 \, \frac{\text{N}}{\text{m}^2} \cdot \pi \cdot (0.05 \, \text{m})^2
\]  
\[
F_{\text{mem}} = 7.85 \, \text{N}
\]  

### Pressure Calculation  
Using the equilibrium condition:  
\[
F + F_{\text{mem}} = F_1
\]  
The pressure \( p_1 \) is then calculated as:  
\[
p_1 = \frac{F + F_{\text{mem}}}{A}
\]  
Substituting the values:  
\[
p_1 = \frac{314.9 \, \text{N} + 7.85 \, \text{N}}{\pi \cdot (0.05 \, \text{m})^2}
\]  
\[
p_1 = 1.40 \, \text{bar}
\]  

### Ideal Gas Law  
The mass \( m_g \) of the gas is calculated using the ideal gas law:  
\[
p \cdot V = m \cdot R \cdot T
\]  
Rearranging for \( m_g \):  
\[
m_g = \frac{p_1 \cdot V_1}{R \cdot T_1}
\]  

### Gas Constant  
The specific gas constant \( R \) is calculated as:  
\[
R = \frac{R_u}{M}
\]  
Substituting the values:  
\[
R = \frac{8.314 \, \frac{\text{J}}{\text{mol} \cdot \text{K}}}{50 \, \frac{\text{kg}}{\text{kmol}}}
\]  
\[
R = 0.166 \, \frac{\text{kJ}}{\text{kg} \cdot \text{K}}
\]  

Final calculations for \( m_g \) are partially visible but incomplete.  

### Diagram Description  
The diagram shows a horizontal membrane with arrows indicating forces acting on it. The forces include \( F \) acting downward, \( p_{\text{amb}} \) acting upward, and \( p_1 \) acting downward. The membrane is labeled with its area \( A \).

The mass of the gas \( m_g \) is calculated using the formula:  
\[
m_g = \frac{V_g \cdot p_g}{R \cdot T_g} = \frac{3.14 \cdot 5}{0.166 \cdot 500} = 3.43 \, \text{g}
\]  
This is converted to kilograms:  
\[
m_g = 3.43 \, \text{g} = 3.43 \cdot 10^{-3} \, \text{kg}
\]  

The volume of the gas is given as:  
\[
V_g = 3.14 \, \text{L} = 3.14 \cdot 10^{-3} \, \text{m}^3
\]  

The temperature of the gas is:  
\[
T_g = 500^\circ\text{C} = 773.15 \, \text{K}
\]  

---