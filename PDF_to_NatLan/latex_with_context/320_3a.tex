The table describes the properties of the upper and lower chambers of the cylinder in states \( Z1 \) and \( Z2 \).  

**Upper Chamber (EW mixture):**  
- **State \( Z1 \):**  
  - Temperature (\( T \)): \( 0^\circ\text{C} \)  
  - Volume (\( V \)): \( V_1 \)  
  - Mass (\( m \)): \( 0.1 \, \text{kg} \)  
  - Phase: Ice-water mixture with ice mass fraction:  
    \[
    x = \frac{m_{\text{ice}}}{m_{\text{EW}}} = 0.6
    \]  

- **State \( Z2 \):**  
  - Volume (\( V \)): \( V_2 \)  
  - EW remains in equilibrium.  

**Lower Chamber (Perfect Gas):**  
- **State \( Z1 \):**  
  - Temperature (\( T \)): \( 500^\circ\text{C} \)  
  - Volume (\( V \)): \( 0.00314 \, \text{m}^3 \)  

- **State \( Z2 \):**  
  - Temperature (\( T \)): \( 0.003^\circ\text{C} \)  

Additional parameters for the perfect gas:  
- Specific heat capacity at constant volume (\( c_V \)): \( 0.633 \, \text{kJ/kg·K} \)  
- Molar mass (\( M_g \)): \( 50 \, \text{kg/kmol} \)  

**Piston Properties:**  
- Mass (\( m_K \)): \( 32 \, \text{kg} \)  
- Diameter (\( D \)): \( 10 \, \text{cm} = 0.1 \, \text{m} \)  
- Ambient pressure (\( p_{\text{amb}} \)): \( 1 \, \text{bar} \)  

No further calculations or explanations are provided beyond the labeled values.  

No diagrams or graphs are present on this page.

The equation for the gas pressure \( p_{g,1} \) is derived using the ideal gas law:  
\[
p_{g,1} V_{g,1} = m_g R T_1
\]  
The forces acting on the piston are balanced as follows:  
\[
m_{\text{EW}} g + m_K g + p_{\text{amb}} A = p_{g,1} A
\]  
Where:  
- \( A = \pi r^2 = \pi \left( \frac{D}{2} \right)^2 = 0.007854 \, \text{m}^2 \)  

Substituting values:  
\[
p_{g,1} = \frac{m_{\text{EW}} g}{A} + \frac{m_K g}{A} + p_{\text{amb}} = \frac{0.1 \, \text{kg} \cdot 9.81 \, \text{m/s}^2}{0.007854 \, \text{m}^2} + \frac{32 \, \text{kg} \cdot 9.81 \, \text{m/s}^2}{0.007854 \, \text{m}^2} + 10^5 \, \text{Pa} = 1.4 \, \text{bar}
\]  

The gas mass \( m_g \) is calculated using:  
\[
m_g = \frac{p_{g,1} V_{g,1}}{R T_1}
\]  
Where:  
\[
R = \frac{\bar{R}}{M} = \frac{8.314 \, \text{kJ/kmol·K}}{50 \, \text{kg/kmol}} = 166.28 \, \text{J/kg·K}
\]  

Substituting values:  
\[
m_g = \frac{1.4 \cdot 10^5 \, \text{Pa} \cdot 0.00314 \, \text{m}^3}{166.28 \, \text{J/kg·K} \cdot (500 + 273.15) \, \text{K}} = 3.449 \, \text{g}
\]