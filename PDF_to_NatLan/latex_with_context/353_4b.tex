The mass flow rate of the refrigerant R134a is given as:  
\[
\dot{m}_{\text{R134a}} = 4 \, \frac{\text{kg}}{\text{h}}, \quad T_2 = -22^\circ\text{C}.
\]  

The pressure at state 2 is calculated as:  
\[
p_2 = p(-22^\circ\text{C}) = 1.2192 \, \text{bar}.
\]  

Since \(T_1 = T_2\), it follows that:  
\[
h_1 = h_2.
\]  

The enthalpy at state 4 is determined using the isenthalpic throttling process:  
\[
h_4 = h_1.
\]  

From the refrigerant tables:  
\[
h_{f}(8 \, \text{bar}) = 93.42 \, \frac{\text{kJ}}{\text{kg}}.
\]  

The vapor quality at state 1 is calculated as:  
\[
x_1 = \frac{h_4 - h_f(-22^\circ\text{C})}{h_{fg}(-22^\circ\text{C})} \approx 0.337.
\]

The initial temperature is given as \( T_1 = -20^\circ\text{C} \). After expansion, the temperature becomes \( T_2 = -16^\circ\text{C} \). The specific entropy at state 2 is \( s_2 = 0.9298 \, \text{kJ/kg·K} \).  

The specific enthalpy at state 2 is calculated using the saturated vapor enthalpy at \( T_2 = -16^\circ\text{C} \):  
\[
h_2 = h_g(-16^\circ\text{C}) = 237.74 \, \text{kJ/kg}
\]  

The specific enthalpy at state 3 is calculated using the following equation:  
\[
h_3 = h(8 \, \text{bar}, 40^\circ\text{C}) - s_2 \cdot (8 \, \text{bar}, 31.33^\circ\text{C}) + h(8 \, \text{bar}, 31.33^\circ\text{C})
\]  
Substituting the values:  
\[
h_3 \approx 271.31 \, \text{kJ/kg}
\]  

The energy balance for the compressor is given as:  
\[
\dot{m}_{\text{R134a}} = \frac{-28 \, \text{kW}}{h_2 - h_3} \approx 3 \, \text{kg/h}
\]  

---