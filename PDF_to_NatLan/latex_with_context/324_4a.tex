The graph is a qualitative representation of the freeze-drying process in a pressure-temperature (\( p \)-\( T \)) diagram. The axes are labeled as follows:  
- The vertical axis represents pressure (\( p \)).  
- The horizontal axis represents temperature (\( T \)).  
No specific values or phase regions are marked on the graph.

---

A table is provided with labeled columns: \( z_{\text{ut}} \), \( P \), \( V \), \( T \), and \( x \). The rows correspond to states 1 through 4, with the following details:  
- State 1: \( T = T_i \), \( x = 1 \). This represents isobaric evaporation at 6 K below \( T_i \).  
- State 2: \( P = 6 \, \text{bar} \), \( x = 1 \). This corresponds to isentropic compression.  
- State 3: \( P = 8 \, \text{bar} \), \( x = 0 \). This represents isobaric condensation.  
- State 4: \( P = 8 \, \text{bar} \), \( x = 0 \), \( T = T_i \). This corresponds to adiabatic expansion.  

A sketch is included showing the process flow, with arrows indicating transitions between states. The transitions are labeled as "isobaric evaporation, 6 K" and "adiabatic throttle."

---