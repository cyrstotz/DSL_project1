The area \( A \) of the cylinder cross-section is calculated as:  
\[
A = \pi \frac{D^2}{4} = 25\pi = 78.54 \, \text{cm}^2 = \pi \frac{(0.1 \, \text{m})^2}{4} = 0.00785 \, \text{m}^2
\]  

The pressure \( p_{g,1} \) is determined using the equation:  
\[
p_{g,1} = p_{\text{amb}} + \frac{m_K g}{A} + m_{\text{EW}} g
\]  
Substituting values:  
\[
p_{g,1} = 10^5 \, \text{Pa} + \frac{32 \, \text{kg} \cdot 9.806 \, \text{m/s}^2}{0.00785 \, \text{m}^2} + 0.1 \, \text{kg} \cdot 9.806 \, \text{m/s}^2
\]  
\[
p_{g,1} = 140078.09 \, \text{Pa} = 1.4 \, \text{bar}
\]  

The gas volume \( V_{g,1} \) is given as:  
\[
V_{g,1} = v_{g,1} \cdot m_g
\]  

The gas mass \( m_g \) is calculated using the ideal gas law:  
\[
m = \frac{p V}{R T}
\]  
Substituting values:  
\[
m = \frac{140000 \cdot V_{g,1}}{R \cdot T_{g,1}}
\]  

The gas constant \( R \) is calculated as:  
\[
R = c_V \cdot (k - 1) = \frac{R}{M}
\]  
\[
R = \frac{8.314 \, \text{J/mol·K}}{50 \, \text{kg/kmol}} = 166.28 \, \text{J/kg·K}
\]  

Given:  
\[
V_{g,1} = 3.14 \, \text{L} = 3.14 \cdot 10^{-3} \, \text{m}^3
\]  
\[
T_{g,1} = 500 + 273.15 = 773.15 \, \text{K}
\]  

Substituting into the mass equation:  
\[
m = \frac{140000 \cdot 3.14 \cdot 10^{-3}}{166.28 \cdot 773.15} = 0.003432 \, \text{kg} = 3.4 \, \text{g}
\]  

---