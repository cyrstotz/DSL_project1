The initial temperature of the gas is \( T_{g,1} = 500^\circ\text{C} \), and the initial volume is \( V_{g,1} = 3.14 \, \text{L} \).  

The ideal gas law is applied:  
\[
P \cdot V = m \cdot R \cdot T
\]  

The specific gas constant \( R \) is calculated as:  
\[
R = \frac{\bar{R}}{M} = \frac{8314.5 \, \text{J/kmol·K}}{50 \, \text{kg/kmol}} = 0.16628 \, \text{kJ/kg·K}
\]  

The specific volume \( v \) is defined as:  
\[
v = \frac{V}{m}
\]  

A sketch of the system is provided, showing a cylinder divided into two chambers. The top chamber contains the ice-water mixture (EW), and the bottom chamber contains the gas. A piston separates the chambers, exerting pressure \( p_{\text{piston}} \) downward on the EW, while the gas pressure \( p_{\text{gas}} \) acts upward.  

The pressure of the gas is calculated as:  
\[
p_{\text{gas}} = p_0 + \frac{32 \, \text{kg} \cdot 9.81 \, \text{m/s}^2}{A}
\]  
where \( A = \frac{\pi \cdot D^2}{4} \). Substituting \( D = 0.1 \, \text{m} \):  
\[
A = \frac{\pi \cdot (0.1)^2}{4} = 7.85 \cdot 10^{-3} \, \text{m}^2
\]  

Thus:  
\[
p_{\text{gas}} = 1 \, \text{bar} + \frac{32 \cdot 9.81}{7.85 \cdot 10^{-3}} = 1 \, \text{bar} + 3.99 \, \text{bar} = 4.99 \, \text{bar}
\]