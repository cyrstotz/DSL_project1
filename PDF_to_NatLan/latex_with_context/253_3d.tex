The first law of thermodynamics is applied to the ice-water mixture (EW):  
\[
\Delta U_{12} = Q_{12} - W_{12}
\]  
Since the volume is constant (\( V = \text{const} \)), the work term \( W_{12} \) is zero:  
\[
\Delta U_{12} = Q_{12}
\]  
The internal energy change for the EW system is expressed as:  
\[
(U_2 - U_1)_{\text{EW}} = Q_{12} = 1.5 \, \text{kJ}
\]  
The final internal energy \( U_2 \) is calculated as:  
\[
U_2 = \frac{1.5 \, \text{kJ}}{m_{\text{EW}}} + U_1
\]  
The initial internal energy \( U_1 \) is determined using:  
\[
U_1 = U_f(p = 1.6 \, \text{bar}) + x_{\text{ice},1}(U_f - U_f')
\]  
Where \( U_f(p = 1.6 \, \text{bar}) \approx -133.45 \, \text{kJ/kg} \).

The internal energy at state 2 is given as:  
\[
U_2 = -118.145 \, \text{kJ/kg}
\]  
The pressure at state 2 is equal to the pressure at state 1:  
\[
p_2 = p_1 = 1.4 \, \text{bar}
\]  

The ice mass fraction at state 2 is calculated using the formula:  
\[
x_2 = \frac{U_2 - U_{\text{fest}}}{U_{\text{flüssig}} - U_{\text{fest}}}
\]  

Substituting the values:  
\[
x_2 = 0.645 = x_{\text{Eis},2}
\]  

No diagrams or additional figures are present.