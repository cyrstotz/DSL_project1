The pressure \( p_{g,1} \) is calculated using the formula:  
\[
p_{g,1} = p_{\text{amb}} + \frac{m_K \cdot g}{A} + \frac{m_{\text{mem}} \cdot g}{A}
\]  
The area \( A \) is determined as:  
\[
A = \pi \cdot \left(\frac{D}{2}\right)^2 = \pi \cdot \left(\frac{0.1 \, \text{m}}{2}\right)^2 = 0.0079 \, \text{m}^2
\]  

Substituting values:  
\[
p_{g,1} = 1.00 \, \text{bar} + \frac{32 \, \text{kg} \cdot 9.81 \, \text{m/s}^2}{0.0079 \, \text{m}^2} + \frac{0.1 \, \text{kg} \cdot 9.81 \, \text{m/s}^2}{0.0079 \, \text{m}^2}
\]  
\[
p_{g,1} = 1.4 \, \text{bar}
\]  

The ideal gas law \( pV = mRT \) is used to calculate the mass \( m_g \):  
\[
p_{g,1} \cdot V_{g,1} = m_g \cdot \frac{R}{M_g} \cdot T_{g,1}
\]  
Rearranging for \( m_g \):  
\[
m_g = \frac{p_{g,1} \cdot V_{g,1} \cdot M_g}{R \cdot T_{g,1}}
\]  

The specific gas constant \( R \) is calculated as:  
\[
R = \frac{\bar{R}}{M_g} = \frac{8.314 \, \text{J/mol·K}}{50 \, \text{kg/kmol}} = 0.166 \, \text{kJ/kg·K}
\]  

Substituting values:  
\[
m_g = \frac{1.40 \, \text{bar} \cdot 100 \, \text{kPa/bar} \cdot 0.00314 \, \text{m}^3}{0.166 \, \text{kJ/kg·K} \cdot 773.15 \, \text{K}}
\]  
\[
m_g = 3.422 \, \text{g} = 0.00342 \, \text{kg}
\]  

No diagrams or figures are present on the page.