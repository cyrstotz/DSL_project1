The following variables are given: \( w_5 \), \( p_5 \), \( T_5 \), and \( p_6 \). The process is described as adiabatic and reversible (isentropic).  

The temperature \( T_6 \) is calculated using the formula:  
\[
T_6 = T_5 \left( \frac{p_6}{p_5} \right)^{\frac{\kappa - 1}{\kappa}}
\]  
Substituting the values:  
\[
T_6 = 431.9 \, \text{K} \left( \frac{0.191 \, \text{bar}}{0.5 \, \text{bar}} \right)^{\frac{0.4}{1.4}} = 328.0 \, \text{K}
\]  

The energy balance equation is written as:  
\[
0 = \dot{m} \left( h_c - h_a + \frac{\omega_c^2 - \omega_a^2}{2} + g(z_c - z_a) \right) + \sum \dot{Q}_i - \sum \dot{W}_i
\]  
Neglecting potential energy and heat transfer terms, this simplifies to:  
\[
0 = \dot{m} \left( h_c - h_a + \frac{\omega_c^2 - \omega_a^2}{2} \right)
\]  

The enthalpy difference \( h_c - h_a \) is expressed as:  
\[
h_c - h_a = c_p (T_5 - T_6)
\]  

The outlet velocity \( w_6 \) is equal to \( w_5 \).