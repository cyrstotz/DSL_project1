The diagram is a qualitative representation of the jet engine process in a \( T \)-\( s \) diagram. The axes are labeled as follows:  
- The vertical axis represents temperature \( T \) in Kelvin \([K]\).  
- The horizontal axis represents entropy \( s \) in kilojoules per kilogram Kelvin \([kJ/kg·K]\).  

The process is depicted with six states labeled \( 0, 1, 2, 3, 4, 5, \) and \( 6 \).  
- State \( 0 \) is the ambient condition.  
- State \( 1 \) is the inlet air condition.  
- States \( 2 \) and \( 3 \) correspond to the compression process, with \( 3 \) being at the highest temperature and pressure.  
- States \( 4 \) and \( 5 \) represent the combustion and turbine processes, respectively.  
- State \( 6 \) is the nozzle exit condition.  

The diagram includes three isobaric lines:  
- \( p_2 = p_3 \), representing the compression process.  
- \( p_4 = p_5 \), representing the combustion process.  
- \( p_0 \), representing the ambient pressure.  

Dashed lines indicate the isobaric processes, and solid lines represent the transitions between states.

---