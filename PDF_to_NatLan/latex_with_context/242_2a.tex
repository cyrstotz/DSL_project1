The page contains two diagrams representing thermodynamic processes in a jet engine. Both diagrams are plotted on \( T \) (temperature) versus \( s \) (specific entropy) axes.  

### Diagram 1 (Top):  
- The diagram begins at point \( 0 \), representing ambient conditions (\( p_0 = 0.191 \, \text{bar} \)).  
- A vertical line from \( 0 \) to \( 2 \) indicates an adiabatic compression process.  
- The curve from \( 2 \) to \( 3 \) represents isobaric heat addition during combustion.  
- The process continues from \( 3 \) to \( 4 \) as an adiabatic expansion in the turbine.  
- The final curve from \( 4 \) to \( 5 \) represents mixing and heat exchange, ending at \( p = 0.5 \, \text{bar} \).  

### Diagram 2 (Bottom):  
- This diagram is more detailed and includes additional points.  
- The process starts at \( 0 \), representing ambient conditions.  
- A vertical line from \( 0 \) to \( 2 \) indicates adiabatic compression.  
- The curve from \( 2 \) to \( 3 \) represents isobaric heat addition during combustion.  
- The process continues from \( 3 \) to \( 4 \) as an adiabatic expansion in the turbine, ending at \( p = 0.5 \, \text{bar} \).  
- A mixing process occurs from \( 4 \) to \( 5 \), followed by a nozzle expansion from \( 5 \) to \( 6 \).  
- The label "clear axial gas flows" is written near the nozzle expansion.  

Both diagrams are qualitative representations of the jet engine cycle, showing the thermodynamic states and processes involved.

The page contains two diagrams related to thermodynamic processes:

1. **Top Diagram**:
   - The graph is a temperature-entropy (\( T \)-\( s \)) diagram.
   - The curve includes labeled points: 0, 1, 2, 3, 4, and 6.
   - The process appears to represent a thermodynamic cycle, with transitions between states marked by curved and straight lines.
   - The lines connecting the points suggest isobaric and adiabatic processes, typical of jet engine cycles.

2. **Bottom Diagram**:
   - The graph is also a temperature-entropy (\( T \)-\( s \)) diagram.
   - The curve includes labeled points: 0, 1, 2, 4, 5, and 6.
   - This diagram appears to depict a more detailed thermodynamic cycle, with additional points and transitions compared to the top diagram.
   - The lines connecting the points suggest similar thermodynamic processes, such as compression, heat addition, and expansion.

No additional textual content or equations are visible on the page.