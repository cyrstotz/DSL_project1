The problem begins with the assumption that kinetic energy (\( KE \)) and potential energy (\( PE \)) are negligible, so \( KE = PE = 0 \).  

The process is described as a stationary flow process with mass flow. The energy balance equation is given as:  
\[
0 = \dot{m} \left( h_{\text{in}} - h_{\text{aus}} \right) + \dot{Q}_R - \dot{W}_t
\]  

The enthalpy values are calculated using water tables:  
- For the outlet enthalpy (\( h_{\text{aus}} \)), the equation is:  
\[
h_{\text{aus}} = h_f + x_D \left( h_g - h_f \right)
\]  
Where:  
  - \( h_f \) is the enthalpy of saturated liquid at \( T = 100^\circ\text{C} \):  
    \[
    h_f = 419.04 \, \frac{\text{kJ}}{\text{kg}}
    \]  
  - \( h_g \) is the enthalpy of saturated vapor at \( T = 100^\circ\text{C} \):  
    \[
    h_g = 2676.1 \, \frac{\text{kJ}}{\text{kg}}
    \]  
  - Substituting these values, \( h_{\text{aus}} \) is calculated as:  
    \[
    h_{\text{aus}} = 430.33 \, \frac{\text{kJ}}{\text{kg}}
    \]  

- For the inlet enthalpy (\( h_{\text{in}} \)), the equation is:  
\[
h_{\text{in}} = h_f \quad \text{(saturated liquid at \( T = 70^\circ\text{C} \))}  
\]  
Where:  
  - \( h_f \) is the enthalpy of saturated liquid at \( T = 70^\circ\text{C} \):  
    \[
    h_f = 232.38 \, \frac{\text{kJ}}{\text{kg}}
    \]