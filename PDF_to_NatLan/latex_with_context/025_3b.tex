To determine \( T_{g,2} \) and \( p_{g,2} \), the following calculations are performed:  

The energy balance for the system is expressed as:  
\[
0 = m_g h_5 + m_6 \left( w_6^2 - w_5^2 \right) / 2
\]  

The entropy balance under steady-state conditions is considered.  

The temperature \( T_6 \) is calculated using the equation:  
\[
T_6 = T_5 + \frac{c_p}{T_5} \left( T_5 - T_6 \right)
\]  
where \( T_5 = 328.0 \, \text{K} \).  

The enthalpy \( h_5 \) is determined as:  
\[
h_5 = -104.453 \, \text{J}
\]  

The velocity \( w_6 \) is calculated using the energy equation:  
\[
w_6^2 = w_5^2 + \frac{2}{m} \left( h_5 - h_6 \right)
\]  

The mass \( m_6 \) is determined as:  
\[
m_6 = \frac{1}{2} \left( 1 + \frac{w_6^2}{w_5^2} \right)
\]  

Finally, the pressure \( p_{g,2} \) is calculated using the ideal gas law and the given conditions.  

No diagrams or figures are present on the page.

The pressure \( p_{g,2} \) is calculated as:  
\[
p_{g,2} = p_{\text{atm}} + m_K g
\]  
where \( p_{\text{atm}} \) is the atmospheric pressure, \( m_K \) is the mass of the piston, and \( g \) is the acceleration due to gravity.  

The pressure exerted by the piston is calculated as:  
\[
p_{\text{piston}} = \frac{(0.5 \cdot 10^{-2} \, \text{m})^2 \pi}{7.150 \cdot 10^{-3} \, \text{m}^2}
\]  
This results in:  
\[
p_{\text{piston}} = 7.150 \cdot 10^{-3} \, \text{N}
\]  

The total pressure is then:  
\[
p_{g,2} = (0.1 \, \text{bar}) + (0.1 \, \text{bar}) = 0.1 \, \text{bar}
\]  

Crossed-out content is ignored.

The temperature of the gas at state 2, \( T_{g,2} \), is equal to the temperature of the ice-water mixture, \( T_{\text{EW}} \), at state 2. This is due to thermal equilibrium between the gas and the ice-water mixture.