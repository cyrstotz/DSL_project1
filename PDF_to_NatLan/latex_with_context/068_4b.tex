The process involves sublimation. The pressure \( p \) is reduced to \( 1 \, \text{mbar} \) or \( 1.2 \times 10^{-3} \, \text{bar} \), which corresponds to \( 2.0 \times 10^{-3} \, \text{kg/m}^3 \). Sublimation occurs at this pressure.  

From Table A-6, interpolation is performed to determine the temperature \( T \) at \( 0.21 \, \text{kPa} \). The formula used is:  
\[
T(0.21 \, \text{kPa}) = T(0.20388 \, \text{kPa}) + \frac{T(0.1635 \, \text{kPa}) - T(0.0883 \, \text{kPa})}{0.1635 - 0.0883} \times (0.21 - 0.0883)
\]  
This yields \( T \approx -22^\circ\text{C} \approx 20.385^\circ\text{C} \).  

The sub-point temperature \( T_2 \) is calculated as \( T_2 = T_{\text{sub-point}} + 20 \, \text{K} \), which is \( 20.385^\circ\text{C} + 20^\circ\text{C} \).

The required refrigerant mass flow rate \( \dot{m}_{\text{R134a}} \) is calculated using the following energy balance equation:  
\[
\dot{m}_R = \frac{\dot{Q}_K - W_K}{h_3 - h_4}
\]  
Rearranging,  
\[
\dot{Q}_K = \dot{m}_R (h_3 - h_4) + W_K
\]  
Substituting into the equation:  
\[
\dot{m}_R \left( 1 + \frac{h_3 - h_4}{h_2 - h_1} \right) = \frac{W_K}{h_2 - h_1}
\]  
Simplifying further:  
\[
\dot{m}_R = \frac{W_K}{h_2 - h_1 + h_3 - h_4}
\]  
Numerical substitution:  
\[
\dot{m}_R = \frac{28 \, \text{kW}}{(232.62 - 16.82 + 93.42 - 264.25) \, \text{J/g}}
\]  

From Table A-20:  
At \( T_2 = T_{i,2} = 260^\circ\text{C} \),  
\[
h_g = h_2 = 232.62 \, \text{J/g}, \quad h_2 = h_f = 16.82 \, \text{J/g}
\]  

From Table A-12:  
At \( p_3 = \text{8 bar, saturated state} \),  
\[
h_3 = h_g = 264.25 \, \text{J/g}, \quad h_4 = h_f = 93.42 \, \text{J/g}
\]  

Final calculation:  
\[
\dot{m}_R = 0.6352 \, \text{g/s} \quad \text{or} \quad 4 \, \text{kg/h}
\]