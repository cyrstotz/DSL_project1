The equilibrium of forces is considered for the piston resting on the ice-water mixture (EW).  

The forces acting on the piston are:  
\[
F_T = F_{\text{EW}} = F_g
\]  
where \( F_T \) is the total force, \( F_{\text{EW}} \) is the force exerted by the EW, and \( F_g \) is the gravitational force.  

The pressure \( p \) is calculated using:  
\[
p = \frac{F}{A} \quad \text{(in \(\text{Pa}\))}
\]  
where \( A \) is the area of the piston.  

The piston radius is given as \( r = 0.05 \, \text{m} \), and the area is calculated as:  
\[
A = \pi r^2 = \pi (0.05)^2 = 7.854 \times 10^{-3} \, \text{m}^2
\]  

The gravitational force acting on the piston is:  
\[
F_g = m_K \cdot g = 32 \, \text{kg} \cdot 9.81 \, \text{m/s}^2 = 313.92 \, \text{N}
\]  

The total force \( F_T \) is:  
\[
F_T = F_g + F_{\text{amb}} \cdot A
\]  
where \( F_{\text{amb}} \) is the force due to atmospheric pressure.

The gas pressure \( p_{g,1} \) is calculated using the ideal gas law:  
\[
pV = mRT \quad \Rightarrow \quad T = T_{\text{gas}}
\]  
Given \( T_{\text{EW},1} = 0^\circ\text{C} \), the temperature of the gas is also \( T_{\text{gas}} = 0^\circ\text{C} \).  

The pressure is determined as:  
\[
p = 1.1 \, \text{bar} \quad \text{(from Table 1)}  
\]

The equation for \( U_2 \) is given as:  
\[
U_2 = (x_2 \cdot U_{\text{f,ice}}) + x_2 \cdot U_{\text{f,water}}
\]  

The value of \( U_{\text{f,ice}} \) is calculated as:  
\[
U_{\text{f,ice}} = -200.0328 \, \text{kJ/kg}
\]  

The equation for \( U_2 \) is expanded as:  
\[
U_2 = x_2 \cdot U_{\text{f,ice}} + Q_{12}
\]  

Substituting values:  
\[
U_2 = -185.0328 \, \text{kJ/kg}
\]  

The temperature \( T_2 \) is given as:  
\[
T_2 = 0.003^\circ \text{C}
\]  

The heat transfer \( Q_{12} \) is calculated as:  
\[
Q_{12} = 1.75 \, \text{MJ}
\]