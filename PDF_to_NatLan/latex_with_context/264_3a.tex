The ideal gas law is used to calculate pressure and mass. The equation is:  
\[
pV = m \frac{R}{M} T
\]  
where \( R = \frac{8.314 \, \text{kJ/(kmol·K)}}{50 \, \text{kg/kmol}} = 0.16628 \, \text{kJ/(kg·K)} \).  

The specific gas constant is calculated as:  
\[
R = 0.16628 \, \text{kJ/(kg·K)} = 0.16628 \, \text{kNm/(kg·K)}
\]  

The area of the piston is determined using:  
\[
A = \pi r^2 = \pi (5 \times 10^{-2} \, \text{m})^2 = 7.853981 \times 10^{-3} \, \text{m}^2
\]  

The pressure exerted by the piston is calculated as:  
\[
p_{s,1} = \frac{F}{A} + p_{\text{amb}} = \frac{32 \, \text{kg} \cdot 9.81 \, \text{m/s}^2}{7.853981 \times 10^{-3} \, \text{m}^2} + 100,000 \, \text{N/m}^2
\]  
\[
= 139,969.538 \, \text{N/m}^2 = 1.3967 \, \text{bar}
\]  

The mass of the gas is calculated using:  
\[
m_g = \frac{pV}{RT} = \frac{(139,969.538 \, \text{Pa})(3.14 \times 10^{-3} \, \text{m}^3)}{(0.16628 \, \text{kJ/(kg·K)})(500 + 273.15) \, \text{K}}
\]  
\[
= 3.418687423 \, \text{kg}
\]  
\[
m_g = 3.41879 \, \text{kg}
\]  

---