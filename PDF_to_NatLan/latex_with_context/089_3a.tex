The problem asks to determine the gas pressure \( p_{g,1} \) and the mass \( m_g \) of the gas in state 1, assuming the gas is a perfect gas.  

The equation for \( p_{g,1} \) is given as:  
\[
p_{g,1} = \frac{m_g R T_{g,1}}{V_{g,1}}
\]  

A diagram is drawn showing the forces acting on the piston, including \( m_g g \), \( m_{\text{EW}} g \), and \( p_{\text{amb}} \cdot A \), with \( p_{g,1} \cdot A \) acting upward.  

The total pressure \( p_{g,1} \) is calculated using the force balance:  
\[
p_{g,1} = \frac{m_g g + m_{\text{EW}} g}{A} + p_{\text{amb}}
\]  
The area \( A \) is determined as:  
\[
A = \pi \left(\frac{D}{2}\right)^2 = 7.854 \cdot 10^{-3} \, \text{m}^2
\]  

Substituting values, the pressure is calculated as:  
\[
p_{g,1} = 40.0948 \, \text{kPa} \, \text{N/m}^2 + 1 \, \text{bar}
\]  

The gas constant \( R \) is given as:  
\[
R = \frac{\bar{R}}{M_g} = \frac{8.314}{50} = 166.28 \, \frac{\text{J}}{\text{kg·K}}
\]  

The mass \( m_g \) is calculated using the ideal gas law:  
\[
m_g = \frac{p_{g,1} V_{g,1}}{R T_{g,1}}
\]  
Substituting values, the result is:  
\[
m_g = 2.4512 \cdot 10^{-2} \, \text{kg}
\]  

---