At state 2:  
\[
h_2 = h_g(-16^\circ\text{C}), \quad x = 1
\]

From table A10 at \( -16^\circ\text{C} \):  
\[
h_2 = 237.74 \, \frac{\text{kJ}}{\text{kg}}, \quad s_2 = 0.9298 \, \frac{\text{kJ}}{\text{kg·K}}
\]

At state 3:  
\[
h_3 = h_f(8 \, \text{bar}) + s_2 - s_f(8 \, \text{bar}) \cdot \frac{h_g(8 \, \text{bar}) - h_f(8 \, \text{bar})}{s_g(8 \, \text{bar}) - s_f(8 \, \text{bar})}
\]

Using table A11 at \( 8 \, \text{bar} \):  
\[
h_3 = h_f(8 \, \text{bar}) + s_2 - s_f(8 \, \text{bar}) \cdot \frac{h_g(8 \, \text{bar}) - h_f(8 \, \text{bar})}{s_g(8 \, \text{bar}) - s_f(8 \, \text{bar})}
\]

Substituting values:  
\[
h_3 = 93.42 + 0.9298 - 0.3459 \cdot \frac{269.15 - 93.42}{0.9374 - 0.2666}
\]

Final calculation:  
\[
h_3 = 273.31 \, \frac{\text{kJ}}{\text{kg}}
\]

The enthalpy at state 4 is equal to \( h_4 \), which corresponds to the saturated liquid at 8 bar. Since the vapor quality \( x = 0 \), the enthalpy is:
\[
h_4 = h_f(8 \, \text{bar}) = 93.42 \, \frac{\text{kJ}}{\text{kg}}
\]

---