The subtask involves determining the outlet temperature \( T_6 \) and velocity \( w_6 \).  
The following calculations are provided:  
- \( p_5 = 0.5 \, \text{bar} \)  
- \( T_5 = 431.9 \, \text{K} \)  
- \( p_6 = 0.191 \, \text{bar} \)  

The temperature \( T_6 \) is calculated using the formula:  
\[
T_6 = T_5 \left( \frac{p_6}{p_5} \right)^{\frac{k-1}{k}}
\]  
where \( k \) is the specific heat ratio.  

The energy balance for the flow process is written as:  
\[
W_t = \dot{m} \left[ h_e - h_a + \frac{w_2^2 - w_1^2}{2} \right]
\]  
where \( W_t \) is the work done, \( \dot{m} \) is the mass flow rate, \( h_e \) and \( h_a \) are specific enthalpies, and \( w_1 \), \( w_2 \) are velocities.  

An integral expression for work is provided:  
\[
W_t = \int_5^6 v \, dp = \int_5^6 \frac{R T}{p} \, dp
\]  
where \( v \) is specific volume, \( R \) is the gas constant, \( T \) is temperature, and \( p \) is pressure.  

Additional notes clarify relationships:  
\[
v = \frac{RT}{p}
\]  
\( dp = p_6 - p_5 \).