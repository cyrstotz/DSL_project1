Using the ideal gas law:  
\[
m_g = \frac{p_{g,1} \cdot V_{g,1}}{R_g \cdot T_{g,1}}
\]  
The specific gas constant \( R_g \) is calculated as:  
\[
R_g = \frac{\bar{R}}{M_g}
\]  
Substituting values:  
\[
R_g = \frac{8.314 \, \text{kJ/(kmol·K)}}{50 \, \text{kg/kmol}} = 166.3 \, \text{J/(kg·K)}
\]  

Now, substituting into the mass equation:  
\[
m_g = \frac{139970 \, \text{Pa} \cdot 3.14 \cdot 10^{-3} \, \text{m}^3}{166.3 \, \text{J/(kg·K)} \cdot 773.15 \, \text{K}}
\]  
This results in:  
\[
m_g = 3.42 \, \text{g}
\]  

No diagrams or graphs are present on the page.

The pressure \( p_{g,2} \) must remain constant due to force equilibrium.  
(The force compressing the system remains unchanged.)  

The temperature \( T_{g,2} \) must remain constant due to thermodynamic equilibrium.  

---