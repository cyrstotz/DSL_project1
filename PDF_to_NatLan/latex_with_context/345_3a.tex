The gas pressure \( p_{g,1} \) and mass \( m_g \) in state 1 are calculated as follows:  
\[
p_{g,1} = \frac{m_g R T_{g,1}}{V_{g,1}}
\]  
Given:  
\[
R = \frac{8314}{M_g}, \quad M_g = 50 \, \text{kg/kmol}, \quad T_{g,1} = 773.15 \, \text{K}, \quad V_{g,1} = 3.14 \, \text{L}
\]  
\[
R = 166.28 \, \text{J/(kg·K)}
\]  
\[
p_{g,1} = \frac{m_g \cdot 166.28 \cdot 773.15}{3.14 \times 10^{-3}} = 1.40 \, \text{bar}
\]  
\[
m_g = \frac{p_{g,1} \cdot V_{g,1}}{R \cdot T_{g,1}} = 3.479 \times 10^{-3} \, \text{kg}
\]