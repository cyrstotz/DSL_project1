The energy balance equation is written as:  
\[
\dot{m}_{\text{ges}} \left( h_5 - h_6 + \frac{w_5^2}{2} - \frac{w_6^2}{2} \right) = 0
\]  
This equation accounts for the enthalpy difference (\(h_5 - h_6\)) and the kinetic energy terms (\(w_5^2/2\) and \(w_6^2/2\)).  

The following derivations are shown:  
1. The polytropic relation for the temperature ratio:  
\[
\frac{T_6}{T_5} = \left( \frac{p_6}{p_5} \right)^{\frac{n-1}{n}}
\]  
From this, \(T_6\) is calculated as:  
\[
T_6 = \left( \frac{p_6}{p_5} \right)^{\frac{n-1}{n}} T_5 = 328.0747 \, \text{K}
\]  

2. The enthalpy difference and kinetic energy terms are used to find \(w_6\):  
\[
h_5 - h_6 + \frac{1}{2} w_5^2 - \frac{1}{2} w_6^2 = 0
\]  
Rearranging for \(w_6\):  
\[
w_6 = \sqrt{2 \left( h_5 - h_6 \right) + w_5^2}
\]  
Substituting values:  
\[
w_6 = \sqrt{2 c_p \left( T_5 - T_6 \right) + w_5^2}
\]  
\[
w_6 = \sqrt{2 \cdot 1.006 \cdot (403.8 - 328.0747) + 220^2}
\]  
\[
w_6 = 507.25 \, \text{m/s}
\]  

This calculation determines the outlet velocity \(w_6\) as \(507.25 \, \text{m/s}\).