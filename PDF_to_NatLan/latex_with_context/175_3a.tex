A sketch of a cylinder is drawn, showing a piston at the top and arrows indicating forces acting downward due to gravity (\( F_g \)) and atmospheric pressure (\( F_{\text{atm}} \)), as well as upward forces due to internal pressure (\( F_p \)). The cylinder diameter is labeled as \( D = 10 \, \text{cm} \).  

The force due to gravity is calculated:  
\[
F_g = m \cdot g = 31.319 \, \text{N}
\]  

The atmospheric force acting on the piston is calculated using the piston area:  
\[
F_{\text{atm}} = p \cdot \left( 0.005 \, \text{m} \right)^2 \cdot \pi = 7.854 \, \text{N}
\]  

Using force equilibrium (\( F_p = F_g + F_{\text{atm}} \)), the internal pressure \( p_1 \) is determined:  
\[
p_1 = \frac{F_g + F_{\text{atm}}}{\left( 0.005 \, \text{m} \right)^2 \cdot \pi} = 40.96 \cdot 10^5 \, \text{Pa}
\]  
\[
p_1 = 40.96 \, \text{bar}
\]  

The molar mass of the gas is calculated:  
\[
m_g = \frac{R}{M} = \frac{8.314 \, \text{J/mol·K}}{50 \, \text{kg/kmol}} = 0.16628 \, \text{kJ/kg·K}
\]  

Using the ideal gas law \( p V = m R T \), the mass of the gas is calculated:  
\[
m = \frac{p_1 \cdot V_1}{R T_1} = \frac{40.96 \cdot 10^5 \, \text{Pa} \cdot 3.114 \cdot 10^{-3} \, \text{m}^3}{0.16628 \, \text{kJ/kg·K} \cdot 773.15 \, \text{K}} = 99.127 \, \text{kg}
\]  

A note is added: "Since I thought my pressure was incorrect, I continued calculating with the given values."