Two graphs are drawn:  
1. The first graph is a pressure-temperature (\(p\)-\(T\)) diagram. It shows a dome-shaped curve representing phase regions. The x-axis is labeled \(T(K)\), and the y-axis is labeled \(p\). There are lines crossing the dome, likely indicating isobaric or isothermal processes.  
2. The second graph is also a pressure-temperature (\(p\)-\(T\)) diagram, but it focuses on a single curve. The x-axis is labeled \(T(K)\), and the y-axis is labeled \(p\). A point \(z\) is marked on the curve, with an arrow pointing downward from \(T\) to \(z\).