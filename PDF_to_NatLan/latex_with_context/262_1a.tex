The goal is to determine \( \dot{Q}_{\text{out}} \), the heat flow removed by the coolant.

### Cooling Power  
The problem involves calculating the cooling power using an energy balance.

### Energy Balance  
The system is stationary, and the energy balance equation is written as:  
\[
\frac{dE}{dt} = \sum \dot{m} \left[ h + \text{ke} + \text{pe} \right] + \sum \dot{Q} - \sum \dot{W}
\]  
Since the system is stationary, \( \frac{dE}{dt} = 0 \). Therefore:  
\[
0 = \dot{m} \left( h_{\text{in}} - h_{\text{out}} \right) + \dot{Q}_R - \dot{Q}_{\text{out}}
\]  
Rearranging gives:  
\[
\dot{Q}_{\text{out}} = \dot{m} \left( h_{\text{in}} - h_{\text{out}} \right) + \dot{Q}_R
\]

### Reference Data from Table A-2  
For pure water as a saturated liquid:  
\[
h_f(70^\circ\text{C}) = 292.98 \, \frac{\text{kJ}}{\text{kg}}
\]  
\[
h_f(100^\circ\text{C}) = 419.04 \, \frac{\text{kJ}}{\text{kg}}
\]  

### Given Values  
\[
\dot{m} = 0.3 \, \frac{\text{kg}}{\text{s}}, \quad \dot{Q}_R = 100 \, \text{kW}
\]

### Calculation of \( \dot{Q}_{\text{out}} \)  
Substituting the values:  
\[
\dot{Q}_{\text{out}} = 0.3 \, \frac{\text{kg}}{\text{s}} \left[ h_f(70^\circ\text{C}) - h_f(100^\circ\text{C}) \right] + \dot{Q}_R
\]  
\[
\dot{Q}_{\text{out}} = 0.3 \, \frac{\text{kg}}{\text{s}} \left[ 292.98 - 419.04 \right] + 100
\]  
\[
\dot{Q}_{\text{out}} = 0.3 \, \frac{\text{kg}}{\text{s}} \cdot (-126.06) + 100
\]  
\[
\dot{Q}_{\text{out}} = -37.818 + 100
\]  
\[
\dot{Q}_{\text{out}} = 62.182 \, \text{kW}
\]

### Final Result  
\[
\dot{Q}_{\text{out}} = 62.182 \, \text{kW}
\]