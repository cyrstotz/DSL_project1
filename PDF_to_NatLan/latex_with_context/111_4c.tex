The process described is an isenthalpic throttling (adiabatic). The enthalpy remains constant during throttling:  
\[
h_u = h_1
\]  
The enthalpy at state 1 is calculated using the formula:  
\[
h_{g,1} = h_f + x_1 (h_g - h_f)
\]  
where \( x_1 \) is the vapor quality, \( h_g \) is the enthalpy of the vapor phase, and \( h_f \) is the enthalpy of the liquid phase.  

Given:  
- \( p_2 = 1.2192 \, \text{bar} \)  
- \( T_2 = -22^\circ\text{C} \) (state 2 is in two-phase region)  

From the tables:  
\[
h_f = 211.77 \, \frac{\text{kJ}}{\text{kg}}, \quad h_g = 234.08 \, \frac{\text{kJ}}{\text{kg}}
\]  

The vapor quality \( x_1 \) can be calculated as:  
\[
x_1 = \frac{h_u - h_f}{h_g - h_f}
\]