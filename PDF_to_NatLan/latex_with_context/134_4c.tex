Another \(p\)-\(V\) diagram is shown with points labeled "1", "2", and "3". The process transitions between these points, with arrows indicating the direction of the process.  

Additional notes:  
- \(T_i = 16 \, \text{K}\) above the sublimation temperature.  
- Chamber pressure is \(5 \, \text{mbar}\) below the triple point of water.

The vapor quality \( x_1 \) is determined using the equation for an adiabatic throttle, which is an isentropic process:  
\[
s_1 = s_u \implies x_1 = \frac{s_u - s_{f4}}{s_{g4} - s_{f4}}
\]  
At \( s_u \) (8 bar, \( x_u = 0 \)):  
\[
s_u = s_f \quad \text{and} \quad s_f (8 \, \text{bar}) = 0.3459 \, \text{kJ/kg·K}
\]  
The temperature \( T \) is calculated as \( T = 31.33^\circ\text{C} \).

---