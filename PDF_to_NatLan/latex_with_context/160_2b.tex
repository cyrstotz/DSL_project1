A graph is drawn representing the jet engine process on a \( T \)-\( s \) diagram. The axes are labeled as follows:  
- The vertical axis represents temperature (\( T \)) in \( ^\circ\text{C} \).  
- The horizontal axis represents entropy (\( s \)) in \( \frac{\text{kJ}}{\text{kg·K}} \).  

Key features of the graph:  
- The process begins at state \( 0 \) and progresses through states \( 1 \) to \( 6 \).  
- States \( 2 \) and \( 3 \) are connected by an isentropic line (constant entropy).  
- States \( 3 \) to \( 4 \) follow an isobaric line (constant pressure).  
- States \( 4 \) to \( 5 \) are connected by another isentropic line.  
- States \( 5 \) to \( 6 \) follow an isothermal line (constant temperature).  
- The pressure at state \( 5 \) is labeled as \( 0.5 \, \text{bar} \).  

The graph visually represents the thermodynamic processes occurring in the jet engine, including isentropic and isothermal transitions.

The outlet velocity \( w_5 \) is given as \( 220 \, \text{m/s} \), the pressure \( p_5 = 0.5 \, \text{bar} \), and the temperature \( T_5 = 431.9 \, \text{K} \).  

The steady-flow energy equation is written as:  
\[
0 = \dot{m} \left[ h_5 - h_4 + \frac{w_5^2 - w_4^2}{2} \right] + \dot{Q} - \dot{W}_{\text{rev}}
\]  

The enthalpy difference \( h_5 - h_4 \) is expressed as:  
\[
h_5 - h_4 = c_p (T_5 - T_4)
\]  

The reversible work term is given as:  
\[
W_{\text{rev}} = -\dot{m} \int v dp + \Delta ke
\]  

Using the ideal gas law:  
\[
pv = RT \quad \text{and} \quad v = \frac{RT}{p}
\]  

Thus, \( W_{\text{rev}} \) simplifies to:  
\[
W_{\text{rev}} = -\dot{m} (RT)
\]  

---