The graph represents a qualitative \( T \)-\( s \) diagram for the jet engine process. The temperature \( T \) is plotted on the vertical axis (in Kelvin), and the specific entropy \( s \) is plotted on the horizontal axis (in \( \frac{\text{kJ}}{\text{kg·K}} \)).  

Key features of the diagram:  
- The process begins at state \( 0 \), with entropy increasing as the air is compressed to state \( 2 \).  
- States \( 2 \) and \( 3 \) are connected by an isobaric combustion process, where temperature increases significantly.  
- The entropy decreases from state \( 3 \) to state \( 4 \) during the turbine process.  
- States \( 5 \) and \( 6 \) represent the mixing and nozzle exit, respectively, with entropy decreasing further.  
- The isobars are labeled as \( p_2 = p_3 \) and \( p_4 = p_5 = p_6 \).  
- The ambient pressure \( p_0 = p_6 \) is indicated.  

Dashed lines represent isobaric processes.  

---