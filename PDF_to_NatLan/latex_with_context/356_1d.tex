The mass change \( \Delta m_{12} \) is to be determined.  

The inlet temperature is given as:  
\[
T_{\text{in,12}} = 20^\circ\text{C}
\]

The heat released during cooling is expressed as:  
\[
Q_{\text{abgeführt}} = m_{\text{ges,1}} \cdot c_w \cdot \Delta T
\]

The temperature difference is calculated as:  
\[
\Delta T = (T_{\text{Reactor,1}} - T_{\text{Reactor,2}}) = 30 \, \text{K}
\]

The specific internal energy values are used to calculate the change in internal energy:  
\[
u_1 = u_{f,70} + x \cdot (u_{g,70} - u_{f,70}) = 429.378 \, \frac{\text{kJ}}{\text{kg}}
\]

\[
u_2 = u_{f,100} + x_D \cdot (u_{g,100} - u_{f,100}) = 803.83325 \, \frac{\text{kJ}}{\text{kg}}
\]

From this, the change in internal energy for the existing water is:  
\[
\Delta u = 125.54455 \, \frac{\text{kJ}}{\text{kg}} \quad (\text{from Table A-2})
\]

For the new water added, the change in internal energy is:  
\[
\Delta u = u_f(20) - u_f(0) = -209 \, \frac{\text{kJ}}{\text{kg}} \quad (\text{from Table A-2})
\]

Using the energy balance, the mass \( m_{12} \) is calculated as:  
\[
m_{12} = \frac{125.54455 \cdot m_{\text{ges}}}{-209}
\]

Substituting values:  
\[
m_{12} = 3456.9849 \, \text{kg}
\]