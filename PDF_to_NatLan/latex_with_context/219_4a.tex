The page contains two diagrams related to the freeze-drying process and the refrigeration cycle.  

1. **First Diagram (p-T Diagram):**  
   - The diagram shows pressure (\( p \)) on the vertical axis and temperature (\( T \)) on the horizontal axis.  
   - The phase regions are labeled:  
     - "unterkühlte Flüssigkeit" (subcooled liquid) is shown below the saturation curve.  
     - "überhitzter Dampf" (superheated vapor) is shown above the saturation curve.  
     - "NQ-Gebiet (Nassdampf)" (wet steam region) is between the liquid and vapor saturation lines.  
   - The triple point is marked where the three phases (solid, liquid, and vapor) coexist.  

2. **Second Diagram (p-T Diagram for Solid-Liquid-Gas Transition):**  
   - This diagram also has pressure (\( p \)) on the vertical axis and temperature (\( T \)) on the horizontal axis.  
   - The regions are labeled:  
     - "Fest" (solid) is at lower temperatures.  
     - "Flüssig" (liquid) is at intermediate temperatures.  
     - "Gas" (gas) is at higher temperatures.  
   - The triple point is marked where solid, liquid, and gas phases coexist.