The goal is to determine the gas pressure \( p_{g,1} \) and mass \( m_g \) in state 1.  

The temperature of the gas in state 1 is given as \( T_{g,1} = 500^\circ\text{C} = 773.15 \, \text{K} \).  
The volume of the gas is \( V_{g,1} = 3.14 \, \text{L} = 0.00314 \, \text{m}^3 \).  

The specific gas constant is calculated as:  
\[
R_g = \frac{\bar{R}}{M} = 0.1663 \, \frac{\text{kJ}}{\text{kg·K}}
\]  

The gas equation is used:  
\[
p_{g,1} V_{g,1} = m_g R_g T_{g,1}
\]  

A diagram is drawn showing the forces acting on the piston. The piston area is \( A = 5 \, \text{cm diameter} \), which gives:  
\[
A = 0.00785 \, \text{m}^2
\]  

The pressure balance is written as:  
\[
p_{\text{amb}} A + (m_K + m_{\text{EW}}) g = p_{1} A
\]  
\[
p_{\text{amb}} + \frac{g}{A} (m_K + m_{\text{EW}}) = p_{1} = 1.4 \, \text{bar} = p_{g,1}
\]  

The mass of the gas is calculated using:  
\[
m_g = \frac{p_{g,1} V_{g,1}}{R_g T_{g,1}}
\]