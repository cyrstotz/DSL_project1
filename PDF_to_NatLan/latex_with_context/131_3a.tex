The gas pressure \( p_{g,1} \) is calculated using the equation:  
\[
p_{g,1} V_{g,1} = m_g R T_{g,1}
\]  
where \( R = \frac{\bar{R}}{M_g} \).  

The force exerted by the piston is calculated as:  
\[
p_{g,1} = \text{Kraft} = 32 \, \text{kg} \cdot 9.81 \, \text{m/s}^2 + 1 \, \text{bar} = 32 \, \text{kg} \cdot 9.81 \, \text{m/s}^2 + 1.105 \, \text{bar}
\]  
\[
= \frac{\text{Kraft}}{\text{Fläche}} = \frac{32 \, \text{kg} \cdot 9.81 \, \text{m/s}^2}{5 \, \text{cm}^2 \cdot \pi}
\]  
\[
= 35565.5 \, \text{Pa} + 100000 \, \text{Pa} = 135565.5 \, \text{Pa}
\]  

The mass \( m_{g,1} \) is then calculated as:  
\[
m_{g,1} = \frac{p_{g,1} V_{g,1}}{R T_{g,1}} = \frac{135565.5 \, \text{Pa} \cdot 3.141 \cdot 10^{-3} \, \text{m}^3}{\frac{8.314 \, \text{J/mol·K}}{50 \, \text{kg/kmol}} \cdot 773.15 \, \text{K}}
\]  
\[
= 3.449 \cdot 10^{-3} \, \text{kg} = 0.003449 \, \text{kg}
\]  

---