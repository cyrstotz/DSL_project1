The diagram is a pressure-temperature (\( p \)-\( T \)) graph illustrating the phase regions of a substance. The graph includes the following features:  
- The solid phase ("Fest") is on the left, the liquid phase ("Flüssig") is in the middle, and the gas phase ("Gas") is on the right.  
- The triple point is marked where the solid, liquid, and gas phases coexist.  
- An isobaric evaporation process is shown, labeled as "isobar verdampfen."  
- An isobaric pressure reduction process is depicted, labeled as "isobar druckabnahme."  
- The critical point is labeled at the upper end of the gas-liquid boundary curve.  
- The axes are labeled: pressure (\( p \)) in bar or Pascal on the vertical axis, and temperature (\( T \)) in Kelvin on the horizontal axis.