The goal is to determine the gas pressure \( p_{g,1} \) and mass \( m_g \) in state 1.  

**Given:**  
\[
T_{g,1} = 500^\circ\text{C} = 773.15 \, \text{K}, \quad V_{g,1} = 3.14 \, \text{L} = 0.00314 \, \text{m}^3
\]  

A diagram is drawn showing a cylinder with atmospheric pressure \( p_0 \), the piston mass \( m_K \), and the membrane exerting pressure \( p_1 \).  

The pressure \( p_1 \) is calculated using the formula:  
\[
p_1 \cdot A = p_0 \cdot A + m_K \cdot g + m_{\text{EW}} \cdot g
\]  
Rearranging:  
\[
p_1 = p_0 + \frac{m_K \cdot g}{A} + \frac{m_{\text{EW}} \cdot g}{A}
\]  

The cross-sectional area \( A \) is determined as:  
\[
A = \frac{\pi D^2}{4}
\]  

Substituting values:  
\[
p_1 = 100000 \, \text{Pa} + \frac{32 \, \text{kg} \cdot 9.81 \, \text{m/s}^2}{\frac{\pi \cdot (0.1 \, \text{m})^2}{4}} + \frac{0.1 \, \text{kg} \cdot 9.81 \, \text{m/s}^2}{\frac{\pi \cdot (0.1 \, \text{m})^2}{4}}
\]  
\[
p_1 = 5.01 \, \text{bar}
\]  

The gas mass \( m_g \) is calculated using the ideal gas law:  
\[
m_g = \frac{p \cdot V}{R \cdot T}
\]  
Substituting values:  
\[
m_g = \frac{5.01 \cdot 0.00314}{287 \cdot 773.15} = 0.255 \, \text{kg}
\]  

---