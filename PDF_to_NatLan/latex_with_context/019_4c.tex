The temperature at state 2 is given as \( T_2 = -22^\circ\text{C} \). The mass flow rate of the refrigerant is calculated as:  
\[
\dot{m} = \frac{4 \, \text{kg/h}}{3600 \, \text{s}} = \frac{4}{3600} \, \text{kg/s}.
\]

The process is described as an adiabatic process, assumed to be isenthalpic.  

The enthalpy at state 4 is \( h_4 = h_1 \).  

For \( p_4 = 8 \, \text{bar} \), the temperature is \( T_4 = 31.33^\circ\text{C} \), and the enthalpy is:  
\[
h_4 = 93.12 \, \text{kJ/kg}.
\]

From Table A-12, the pressure \( p_2 \) is calculated as:  
\[
p_2 = 1 \, \text{bar} + \frac{0.46 \, \text{bar}}{7.63^\circ\text{C}} \cdot (-22^\circ\text{C} - (-26.48^\circ\text{C})) = 1.23 \, \text{bar}.
\]

Thus, \( p_1 = p_2 = 1.23 \, \text{bar} \).  

The vapor quality \( x \) is determined using the formula:  
\[
x = \frac{h_1 - h_f}{h_g - h_f}.
\]

From the tables:  
\[
h_f = 21.82 + \frac{4.15}{0.2} \cdot (1.23 - 1.2) = 21.6875 \, \text{kJ/kg}.
\]  
\[
h_g = 233.86 + \frac{2.18}{0.2} \cdot (1.23 - 1.2) = 234.187 \, \text{kJ/kg}.
\]

Substituting into the formula:  
\[
x = \frac{93.12 - 21.6875}{234.187 - 21.6875} = \frac{71.4325}{212.4995} = 0.337.
\]  

The vapor quality at state 1 is \( x_1 = 0.337 \).