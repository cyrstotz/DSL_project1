Since the reactor remains at \( 100^\circ\text{C} \), the following equation applies:  
\[
\dot{Q}_{\text{out}} + \dot{m} \cdot (h_{\text{out}} - h_{\text{in}}) = \dot{Q}_R \quad \Rightarrow \quad \dot{Q}_{\text{out}} = \dot{Q}_R - \dot{m} \cdot (h_{\text{out}} - h_{\text{in}})
\]  
We can determine \( h_{\text{out}} \) and \( h_{\text{in}} \) using values from Table A-2:  
\[
h_{\text{out}} = x_D \cdot h_g(T=100) + (1 - x_D) \cdot h_f(T=100) = 730.72 \, \text{kJ/kg}
\]  
\[
h_{\text{in}} = x_D \cdot h_g(T=70) + (1 - x_D) \cdot h_f(T=70) = 709.6581 \, \text{kJ/kg}
\]  
Substituting into the equation:  
\[
\dot{Q}_{\text{out}} = 100 \, \text{kW} - 0.3 \cdot (730.72 - 709.6581) = 62.289 \, \text{kW}
\]