The goal is to determine the outlet velocity \( w_6 \) and temperature \( T_6 \).  

Given data:  
\[
w_5 = 220 \, \text{m/s}, \quad p_5 = 0.5 \, \text{bar}, \quad p_0 = 0.191 \, \text{bar}, \quad T_5 = 431.9 \, \text{K}.
\]  

The process is described as adiabatic and reversible, implying it is isentropic.  

Using the ideal gas relation for isentropic processes:  
\[
\frac{T_6}{T_5} = \left( \frac{p_6}{p_5} \right)^{\frac{\kappa - 1}{\kappa}},
\]  
where \( \kappa = 1.4 \).  

Substituting values:  
\[
T_6 = 431.9 \cdot \left( \frac{0.191}{0.5} \right)^{\frac{1.4 - 1}{1.4}} = 328.075 \, \text{K}.
\]  

Next, the energy balance is applied:  
\[
0 = \dot{m} \left( h_2 - h_1 \right) + \frac{w_2^2 - w_1^2}{2} + \dot{Q} - \dot{W}.
\]  

For steady flow, the specific work \( w_a \) is calculated using:  
\[
w_a = \dots
\]  
(Note: The final expression for \( w_a \) is partially obscured and cannot be fully transcribed.)