The page contains two diagrams related to the freeze-drying process described in Task 4.  

1. **First Diagram**:  
   - The graph is labeled as a pressure-temperature (\(p\)-\(T\)) diagram.  
   - The axes are marked with \(p\) (pressure) on the vertical axis and \(T\) (temperature in Kelvin) on the horizontal axis.  
   - Four states are labeled: \(1\), \(2\), \(3\), and \(4\).  
   - The process between states \(1\) and \(2\) is marked as "sotrop" (likely indicating an isentropic process).  
   - The curve transitions between states, with arrows indicating the direction of the process.  

2. **Second Diagram**:  
   - The graph is labeled as a pressure-enthalpy (\(p\)-\(h\)) diagram.  
   - The axes are marked with \(p\) (pressure) on the vertical axis and \(h\) (enthalpy) on the horizontal axis.  
   - Four states are labeled: \(1\), \(2\), \(3\), and \(4\).  
   - The process transitions between states, with curved arrows indicating the direction of the process.

The diagram is a pressure-temperature (\( p \)-\( T \)) graph illustrating the freeze-drying process. The axes are labeled as follows:  
- The vertical axis represents pressure (\( p \)) in unspecified units.  
- The horizontal axis represents temperature (\( T \)) in Kelvin (\( \text{K} \)).  

The graph includes the following features:  
- A curved line representing the phase boundary.  
- Four states labeled as 1, 2, 3, and 4.  
- A process between states 1 and 2 labeled as "isentrop," indicating an isentropic process.  
- A process between states 3 and 4 labeled as "isobar," indicating an isobaric process.  
- The region labeled "ND" (possibly indicating a no-disturbance or non-phase-change region) on the temperature axis.  

The diagram visually represents the thermodynamic transitions involved in the freeze-drying process, with clear distinctions between isentropic and isobaric processes.