The energy balance equation is written as:  
\[
0 = \dot{m} \cdot (h_{\text{in}} - h_{\text{out}}) + \dot{Q}_R - \dot{Q}_{\text{AB}}
\]  
The heat flow removed by the coolant is calculated as:  
\[
\dot{Q}_{\text{AB}} = \dot{Q}_R + \dot{m}_{\text{in}} \cdot (h_{\text{in}} - h_{\text{out}})
\]  
Substituting the given values:  
\[
\dot{Q}_{\text{AB}} = 100 \, \text{kW} + 0.3 \, \frac{\text{kg}}{\text{s}} \cdot (292.98 \, \frac{\text{kJ}}{\text{kg}} - 419.04 \, \frac{\text{kJ}}{\text{kg}})
\]  
\[
\dot{Q}_{\text{AB}} = 62.18 \, \text{kW}
\]  

The enthalpy values are:  
\[
h_{\text{in}}(70^\circ\text{C}) = 292.98 \, \frac{\text{kJ}}{\text{kg}}
\]  
\[
h_{\text{out}}(100^\circ\text{C}) = 419.04 \, \frac{\text{kJ}}{\text{kg}}
\]

The process steps for the jet engine are described as follows:  
- \( 0 \rightarrow 1 \): Adiabatic compression.  
- \( 1 \rightarrow 2 \): Isentropic compression.  
- \( 2 \rightarrow 3 \): Isobaric heat addition.  
- \( 3 \rightarrow 4 \): Adiabatic and irreversible expansion.  
- \( 4 \rightarrow 5 \): Mixing, isobaric.  
- \( 5 \rightarrow 6 \): Isentropic expansion.  

The pressure at state 5 is given as \( p_5 = p_6 = p_0 = 0.5 \, \text{bar} \).  

A graph is drawn representing the process in a \( T \) vs. \( s \) diagram:  
- The x-axis is labeled \( s \, [\text{kJ/kg·K}] \).  
- The y-axis is labeled \( T \, [\text{K}] \).  
- The diagram shows six states connected by curves and lines, with the following transitions:  
  - \( 0 \rightarrow 1 \): Adiabatic compression.  
  - \( 1 \rightarrow 2 \): Isentropic compression.  
  - \( 2 \rightarrow 3 \): Isobaric heat addition.  
  - \( 3 \rightarrow 4 \): Adiabatic and irreversible expansion.  
  - \( 4 \rightarrow 5 \): Mixing, isobaric.  
  - \( 5 \rightarrow 6 \): Isentropic expansion.  

Dashed lines indicate isobars for \( p_3 = p_2 \), \( p_5 = p_6 \), and \( p_4 \).  

---