The task begins with labeled states and thermodynamic properties for the freeze-drying process using R134a.  

State 1:  
\[
T_1 = \text{unknown}, \quad p_1 = \text{unknown}, \quad h_1 = 93.42 \, \text{kJ/kg}
\]  

State 2:  
\[
T_2 = \text{unknown}, \quad p_2 = p_1, \quad x_2 = 1
\]  

State 3:  
\[
T_3 = \text{unknown}, \quad p_3 = 8 \, \text{bar}
\]  

State 4:  
\[
T_4 = 31.33^\circ \text{C}, \quad p_4 = 8 \, \text{bar}, \quad x_4 = 0, \quad h_4 = 93.42 \, \text{kJ/kg}
\]  

The equation for the work done by the compressor is given as:  
\[
\dot{m}_{\text{R134a}} \cdot (h_2 - h_3) = \dot{W}_K
\]  

---

The page contains two diagrams illustrating phase regions in a pressure-temperature (\( p \)-\( T \)) diagram.  

1. **First Diagram**:  
   - The axes are labeled as \( p \) (pressure in bar) on the vertical axis and \( T \) (temperature in \( ^\circ \text{C} \)) on the horizontal axis.  
   - Three distinct regions are marked: "solid," "fluid," and "gas."  
   - The "solid" region is located at low temperatures and high pressures, while the "gas" region is at high temperatures and low pressures. The "fluid" region lies between them.  
   - A curve labeled "Tripel" (triple point) separates the three phases.  
   - Two points, labeled "1" and "2," are marked along the curve.  

2. **Second Diagram**:  
   - Similar axes are used: \( p \) (pressure in bar) on the vertical axis and \( T \) (temperature in \( ^\circ \text{C} \)) on the horizontal axis.  
   - Three regions are again identified: "solid," "fluid," and "gas."  
   - The "solid" region is at low temperatures and high pressures, while the "gas" region is at high temperatures and low pressures. The "fluid" region lies between them.  
   - A curve separates the phases, and two points, labeled "1" and "2," are marked along the curve.  

Both diagrams visually represent phase transitions and equilibrium points in the \( p \)-\( T \) space.