The problem involves determining the gas pressure \( p_{g,1} \) and mass \( m_g \) in state 1.  

A table is partially filled with the following data:  
\[
\begin{array}{|c|c|c|c|c|}
\hline
p \, [\text{bar}] & T \, [^\circ\text{C}] & h \, [\frac{\text{kJ}}{\text{kg}}] & s \, [\frac{\text{kJ}}{\text{kg·K}}] & \text{Other} \\
\hline
1 & 500 & & & V_{g,1} = 3.14 \, \text{L} \\
\hline
\end{array}
\]

Ambient pressure \( p_{\text{amb}} = 1 \, \text{bar} \), and piston mass \( m_K = 32 \, \text{kg} \).  

### Calculations:  
1. **Gas mass \( m_g \):**  
   Using the ideal gas law:  
   \[
   m_g = \frac{p_{g,1} V_{g,1}}{R T}
   \]  
   Given:  
   \[
   V_{g,1} = 3.14 \, \text{L} = 0.00314 \, \text{m}^3
   \]

2. **Pressure \( p_{g,1} \):**  
   Pressure is calculated using force equilibrium:  
   \[
   p_{g,1} \cdot A = p_{\text{amb}} \cdot A + m_K \cdot g + m_{\text{EW}} \cdot g
   \]  
   Rearranging:  
   \[
   p_{g,1} = p_{\text{amb}} + \frac{g \cdot (m_K + m_{\text{EW}})}{A}
   \]  
   Area \( A \) is calculated as:  
   \[
   A = \pi \left( \frac{D}{2} \right)^2 = \pi \left( \frac{0.1 \, \text{m}}{2} \right)^2 = 0.00785 \, \text{m}^2
   \]  
   Substituting values:  
   \[
   p_{g,1} = 1 \, \text{bar} + \frac{9.81 \, \frac{\text{N}}{\text{kg}} \cdot 32.1 \, \text{kg}}{0.00785 \, \text{m}^2} = 1.3986 \, \text{bar} \approx 1.4 \, \text{bar}
   \]

3. **Gas mass \( m_g \):**  
   Substituting into the ideal gas law:  
   \[
   m_g = \frac{1.4 \cdot 10^5 \, \text{Pa} \cdot 0.00314 \, \text{m}^3}{8.314 \, \frac{\text{J}}{\text{mol·K}} \cdot 773.15 \, \text{K} \cdot \frac{50 \, \text{kg}}{\text{kmol}}}
   \]  
   Result:  
   \[
   m_g = 3.42 \, \text{g}
   \]  

### Final Results:  
- Gas pressure: \( p_{g,1} = 1.4 \, \text{bar} \)  
- Gas mass: \( m_g = 3.42 \, \text{g} \)  

No diagrams or additional figures are present on the page.