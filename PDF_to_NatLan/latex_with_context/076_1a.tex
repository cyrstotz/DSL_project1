The goal is to calculate \( \dot{Q}_{\text{out}} \), the heat flow removed by the coolant.  

Given data:  
\[
\dot{Q}_R = 100 \, \text{kW}, \quad m_{\text{total},1} = 5755 \, \text{kg}, \quad \dot{m} = 0.3 \, \frac{\text{kg}}{\text{s}}, \quad x_0 = 0.005
\]  

Energy balance (steady-state):  
\[
0 = \dot{m} (h_e - h_a) + \dot{Q}_R - \dot{Q}_{\text{out}}
\]  

The enthalpy at the inlet (\( h_e \)) is calculated as:  
\[
h_e = h_f(70^\circ\text{C}) + x_0 \cdot \left( h_g(70^\circ\text{C}) - h_f(70^\circ\text{C}) \right)
\]  
Using water table values:  
\[
h_f(70^\circ\text{C}) = 293.38 \, \frac{\text{kJ}}{\text{kg}} \quad (\text{Table A-2}), \quad h_g(70^\circ\text{C}) = 2626.8 \, \frac{\text{kJ}}{\text{kg}} \quad (\text{Table A-2})
\]  
\[
h_e = 304.648 \, \frac{\text{kJ}}{\text{kg}}
\]  

The enthalpy at the outlet (\( h_a \)) is calculated as:  
\[
h_a = h_f(100^\circ\text{C}) + x_0 \cdot \left( h_g(100^\circ\text{C}) - h_f(100^\circ\text{C}) \right)
\]  
Using water table values:  
\[
h_f(100^\circ\text{C}) = 418.94 \, \frac{\text{kJ}}{\text{kg}} \quad (\text{Table A-2}), \quad h_g(100^\circ\text{C}) = 2676.1 \, \frac{\text{kJ}}{\text{kg}} \quad (\text{Table A-2})
\]  
\[
h_a = 430.33 \, \frac{\text{kJ}}{\text{kg}}
\]  

Substituting into the energy balance equation:  
\[
\dot{Q}_{\text{out}} = \dot{m} \cdot (h_e - h_a) + \dot{Q}_R
\]  
\[
\dot{Q}_{\text{out}} = 0.3 \cdot (304.648 - 430.33) + 100
\]  
\[
\dot{Q}_{\text{out}} = 62.3 \, \text{kW}
\]  

---