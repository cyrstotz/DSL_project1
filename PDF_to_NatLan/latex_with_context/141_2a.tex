The page contains a graph labeled as a \( T \)-\( s \) diagram. The axes are marked as follows:  
- The vertical axis represents temperature \( T \) in Kelvin (K).  
- The horizontal axis represents entropy \( s \) in \( \frac{\text{kJ}}{\text{kg·K}} \).  

The graph shows several curves and points:  
1. A curve labeled "0.191 bar" representing an isobaric process at a pressure of 0.191 bar.  
2. Another curve labeled "0.5 bar" representing an isobaric process at a pressure of 0.5 bar.  
3. Points labeled "1", "2", "3", and "4" are marked along the curves, indicating different states in the process.  
4. A dashed line labeled "0.16" appears near state 1, possibly indicating a specific entropy value.  

The diagram qualitatively illustrates the thermodynamic process, with isobars and states clearly labeled.

The diagram is a qualitative representation of the jet engine process on a \( T \)-\( s \) diagram. It includes labeled isobars and key states (0, 1, 2, 3, 4, 5, 6). The axes are labeled as follows:  
- \( T \) (temperature) on the vertical axis, in Kelvin (\( \text{K} \)).  
- \( s \) (specific entropy) on the horizontal axis, in \( \text{kJ}/\text{kg·K} \).  

The process transitions are shown with arrows connecting the states:  
- State 0 to 1 represents compression.  
- State 1 to 2 is the bypass flow.  
- State 2 to 3 represents combustion.  
- State 3 to 4 is turbine expansion.  
- State 4 to 5 is mixing.  
- State 5 to 6 represents nozzle expansion.  

Key annotations:  
- The process is described as "qualitative" and "from the solution sheet in the graph phase."  
- The student notes to "transfer to another sheet" for further elaboration.  

---