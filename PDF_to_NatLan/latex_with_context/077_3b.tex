The following questions are posed:  
- \( x_{\text{ice},2} > 0 \)  
- \( T_{g,2} \) is unknown.  
- \( p_{g,2} \) is unknown.  

Key assumptions and observations:  
- \( R \) and \( m \) are constant.  
- \( V_{\text{ist}} \) is smaller because \( p_{g,2} \) does not depend on the volume but rather on the pressure and mass.  

The temperature \( T_2 \) is calculated using the polytropic temperature relationship:  
\[
T_2 = T_1 \left( \frac{p_2}{p_1} \right)^{\frac{n-1}{n}}
\]  
Where:  
- \( C_p = R + C_v = 0.1663 \, \frac{\text{kJ}}{\text{kg} \cdot \text{K}} + 0.633 \, \frac{\text{kJ}}{\text{kg} \cdot \text{K}} = 0.7993 \, \frac{\text{kJ}}{\text{kg} \cdot \text{K}} \)  

Additional notes mention that \( T \) is smaller due to the polytropic process.  

No diagrams or figures are present on this page.

The ratio of specific heats \( \kappa \) is calculated as:  
\[
\kappa = \frac{c_p}{c_v} = \frac{0.7933}{0.633} = 1.263
\]  

The volume \( V \) becomes smaller, the pressure \( p \) becomes larger, and since the gas loses heat, the temperature \( T \) decreases.  

The temperature \( T_2 \) is calculated as:  
\[
T_2 = 773 \, \text{K} \left( \frac{p_x}{1.5} \right) = 773 \, \text{K} \left( \frac{0.26}{1.26} \right)
\]  

---