The pressure \( p_{g,1} \) and mass \( m_g \) of the gas in state 1 are calculated as follows:  

The cross-sectional area of the cylinder is given as \( A = 0.05 \, \text{m}^2 \).  

The pressure \( p_{g,1} \) is determined using the force balance equation:  
\[
p_{g,1} = \frac{32 \cdot 9.81 + 0.1 \cdot 9.81 + 10.5 \cdot 0.05 \cdot \pi}{0.05 \cdot \pi}
\]  
After solving, \( p_{g,1} \approx 1.4 \, \text{bar} \).  

The mass \( m_g \) is calculated using the ideal gas law:  
\[
pV = mRT
\]  
Rearranging for \( m_g \):  
\[
m_g = \frac{p_{g,1} V_{g,1}}{R T_{g,1}}
\]  
The gas constant \( R \) is given as:  
\[
R = \frac{R_u}{M_g} = \frac{8.314 \, \text{J/mol·K}}{50 \, \text{kg/kmol}} = 0.166 \, \text{J/g·K}
\]  
Substituting values:  
\[
m_g = \frac{1.4 \cdot 10^5 \cdot 3.14 \cdot 10^{-3}}{0.166 \cdot 773.15} = 0.0639 \, \text{kg} = 63.9 \, \text{g}
\]  

---