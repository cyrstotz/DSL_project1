The final ice fraction \( x_{\text{ice},2} \) is determined as follows:  

Given:  
\[
m_{\text{EW},1} = 0.6 \, \text{kg}, \quad m_e = 0.06 \, \text{kg}, \quad m_{\text{water}} = 0.04 \, \text{kg}
\]  

The energy balance is:  
\[
\Delta U = Q = U(T_2) - U(T_1) = m_{EW} c_V (T_2 - T_1)
\]  

\[
T_2 - T_1 = \frac{Q}{m c_V} \quad \text{where} \quad c_V = 0.633 \, \text{kJ/kg·K}
\]  

Substituting values:  
\[
T_2 - T_1 = \frac{1500}{0.1 \cdot 0.633} = 23.7 \, \text{K}
\]  

Thus:  
\[
T_2 = T_1 + (T_2 - T_1) = 273.15 + 23.7 = 296.846 \, \text{K}
\]  

The change in internal energy is:  
\[
\Delta u = u_{FI} + x (u_{FE} - u_{FI})
\]  

Solving for \( x \):  
\[
x = \frac{\Delta u - u_{FI}}{u_{FE} - u_{FI}} = \frac{15 - 0.045}{333.458 - 0.045} = 0.045
\]  

The final ice fraction is \( x_{\text{ice},2} = 0.045 \).