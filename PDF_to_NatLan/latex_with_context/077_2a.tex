The process is represented in a \( T \)-\( s \) diagram, where temperature \( T \) is plotted against entropy \( s \). The diagram includes labeled states (1, 2, 3, 4, 5, and 6) connected by lines representing different thermodynamic processes.  

- The vertical axis represents temperature \( T \), and the horizontal axis represents entropy \( s \).  
- The diagram includes isobars (lines of constant pressure) labeled \( p_0 \), \( p_7 \), and \( p_7p_0 \), as well as an isotherm (line of constant temperature) at \( -30^\circ\text{C} \).  
- The processes between states are described as follows:  
  1. From state 1 to state 2: Isentropic process (\( s \)-constant), pressure decreases (\( p \downarrow \)).  
  2. From state 2 to state 3: Isobaric process (\( p \)-constant), temperature increases (\( T \uparrow \)).  
  3. From state 3 to state 4: Non-isentropic process.  
  4. From state 4 to state 5: Isobaric process (\( p \)-constant).  
  5. From state 5 to state 6: Isentropic process (\( s \)-constant).  

Additionally, a smaller diagram is drawn in the corner, showing a simplified schematic of the thermodynamic cycle with states 1 through 6 connected in sequence.  

This representation qualitatively illustrates the thermodynamic processes occurring in the jet engine system.