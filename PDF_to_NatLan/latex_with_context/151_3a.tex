To determine the gas pressure \( p_{g,1} \) and mass \( m_g \) in state 1:  

The given values are:  
\[
T_{g,1} = 500^\circ\text{C} = 773 \, \text{K}
\]
\[
V_{g,1} = 3.14 \, \text{L} = 0.00314 \, \text{m}^3
\]
\[
M_g = 50 \, \text{kg/kmol}
\]

The ideal gas law is applied:  
\[
pV = nRT
\]
Where \( n \) is the number of moles, \( R \) is the specific gas constant, and \( m \) is the mass of the gas.  

The specific gas constant is calculated as:  
\[
R = \frac{\bar{R}}{M_g} = \frac{8.314}{50} = 0.166 \, \text{kJ/kg·K}
\]

The mass of the gas is expressed as:  
\[
m = \frac{pV}{RT}
\]

The pressure is derived as:  
\[
p = \frac{mRT}{V}
\]

Some calculations and derivations are crossed out and not legible.  

---