The system is modeled as stationary and adiabatic, with the following energy balance:  
\[
\dot{E}_{\text{in}} - \dot{E}_{\text{out}} = \frac{\partial E}{\partial t} = 0
\]  
This implies that the heat transfer (\(Q\)) and work (\(W\)) must balance.  

The heat transfer during isobaric evaporation of R134a is calculated using:  
\[
\dot{Q}_K = \dot{m}_{\text{R134a}} \cdot \left( h_{g}(T_i) - h_{f}(T_i) \right)
\]  
From the tables (A-10) for R134a at \(T_i = -20^\circ\text{C}\):  
\[
h_{g}(T_i) = 245.94 \, \frac{\text{kJ}}{\text{kg}}, \quad h_{f}(T_i) = 24.17 \, \frac{\text{kJ}}{\text{kg}}
\]  
\[
\dot{Q}_K = \dot{m}_{\text{R134a}} \cdot (245.94 - 24.17) = \dot{m}_{\text{R134a}} \cdot 219.67 \, \frac{\text{kJ}}{\text{kg}}
\]

The heat transfer during isobaric condensation of R134a at \(T_{\text{w}} = -26^\circ\text{C}\) is calculated similarly:  
\[
\dot{Q}_{\text{cond}} = \dot{m}_{\text{R134a}} \cdot \left( h_{g}(T_{\text{w}}) - h_{f}(T_{\text{w}}) \right)
\]  
From the tables (A-10) for R134a at \(T_{\text{w}} = -26^\circ\text{C}\):  
\[
h_{g}(T_{\text{w}}) = 242.43 \, \frac{\text{kJ}}{\text{kg}}, \quad h_{f}(T_{\text{w}}) = 16.25 \, \frac{\text{kJ}}{\text{kg}}
\]  
\[
\dot{Q}_{\text{cond}} = \dot{m}_{\text{R134a}} \cdot (242.43 - 16.25) = \dot{m}_{\text{R134a}} \cdot 226.18 \, \frac{\text{kJ}}{\text{kg}}
\]

The mass flow rate of R134a (\(\dot{m}_{\text{R134a}}\)) is determined using the energy balance:  
\[
\dot{m}_{\text{R134a}} = \frac{\dot{Q}_K}{h_{g}(T_i) - h_{f}(T_i)}
\]  
From the tables:  
\[
h_{g}(T_i) = 245.94 \, \frac{\text{kJ}}{\text{kg}}, \quad h_{f}(T_i) = 24.17 \, \frac{\text{kJ}}{\text{kg}}
\]  
\[
\dot{m}_{\text{R134a}} = \frac{\dot{Q}_K}{245.94 - 24.17} = \frac{\dot{Q}_K}{219.67 \, \frac{\text{kJ}}{\text{kg}}
\]