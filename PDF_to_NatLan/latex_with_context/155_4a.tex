The diagram is a pressure-temperature (\( p \)-\( T \)) graph labeled with the points 1, 2, 3, and 4.  

- The vertical axis represents pressure (\( p \)) in bar.  
- The horizontal axis represents temperature (\( T \)) in Kelvin (\( \text{K} \)).  
- The graph shows a rectangular cycle with the following transitions:  
  - From point 1 to point 2: a horizontal line indicating constant pressure.  
  - From point 2 to point 3: a diagonal line indicating a decrease in pressure and temperature.  
  - From point 3 to point 4: a horizontal line indicating constant pressure.  
  - From point 4 to point 1: a vertical line indicating an increase in pressure at constant temperature.  

This graph qualitatively represents the freeze-drying process in the \( p \)-\( T \) diagram, with labeled phase transitions between states.