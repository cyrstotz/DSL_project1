The problem involves determining the gas pressure \( p_{g,1} \) and mass \( m_g \) in state 1 using the ideal gas law.  

The ideal gas law is expressed as:  
\[
pV = nRT
\]  
where \( R = \frac{\bar{R}}{M} \).  

Constants and values used:  
\[
\bar{R} = 8.314 \, \frac{\text{J}}{\text{mol·K}}, \quad M = 50 \, \frac{\text{kg}}{\text{kmol}}, \quad R = 0.16628 \, \frac{\text{J}}{\text{kg·K}}
\]  

The temperature is converted to Kelvin:  
\[
T_{g,1} = 500^\circ\text{C} = 773.15 \, \text{K}
\]  

The volume is converted to cubic meters:  
\[
V_{g,1} = 3.14 \, \text{L} = 0.00314 \, \text{m}^3
\]  

The pressure is calculated using:  
\[
p_{g,1} = \frac{mRT}{V}
\]  
Substituting values:  
\[
p_{g,1} = \frac{0.0036 \, \text{kg} \cdot 0.16628 \, \frac{\text{J}}{\text{kg·K}} \cdot 773.15 \, \text{K}}{0.00314 \, \text{m}^3} = 197,333 \, \text{Pa}
\]  

The mass \( m_g \) is calculated using:  
\[
m_g = \frac{pV}{RT}
\]  
Substituting values:  
\[
m_g = \frac{1.5 \cdot 10^5 \, \text{Pa} \cdot 0.00314 \, \text{m}^3}{0.16628 \, \frac{\text{J}}{\text{kg·K}} \cdot 773.15 \, \text{K}} = 3.664 \, \text{g}
\]