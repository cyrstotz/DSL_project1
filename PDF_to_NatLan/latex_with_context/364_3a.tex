To determine the gas pressure \( p_{g,1} \) and mass \( m_g \) in state 1:  

The cross-sectional area of the cylinder is calculated as:  
\[
A_c = \pi \left( \frac{D}{2} \right)^2 = \pi \left( 0.05 \, \text{m} \right)^2 = 0.00785 \, \text{m}^2
\]  

The pressure is given by:  
\[
p = \frac{F}{A}
\]  
where \( F \) is the force exerted by the piston, which includes contributions from atmospheric pressure (\( F_{\text{amb}} \)), the piston weight (\( F_k \)), and any additional external force (\( F_E \)).  

The force balance is expressed as:  
\[
p_{g,1} \cdot A_c = p_{\text{amb}} \cdot A_c + m_k \cdot g + m_E \cdot g
\]  
Substituting values:  
\[
p_{g,1} = \frac{m_k \cdot g + m_E \cdot g}{A_c} + p_{\text{amb}}
\]  
\[
p_{g,1} = \frac{32 \, \text{kg} \cdot 9.81 \, \text{m/s}^2 + 0.1 \, \text{kg} \cdot 9.81 \, \text{m/s}^2}{0.00785 \, \text{m}^2} + 1.05 \, \text{bar}
\]  
\[
p_{g,1} = 1.3886 \, \text{bar} \approx 1.4 \, \text{bar}
\]  

To calculate the gas mass \( m_g \), the ideal gas law is used:  
\[
p \cdot V = m_g \cdot R_g \cdot T
\]  
Rearranging:  
\[
m_g = \frac{p_{g,1} \cdot V_1}{R_g \cdot T_1}
\]  
Substituting values:  
\[
m_g = \frac{1.4 \, \text{bar} \cdot 3.14 \cdot 10^{-3} \, \text{m}^3}{8.314 \, \text{J/(mol·K)} \cdot 773.15 \, \text{K}}
\]  
\[
m_g = 3.42 \, \text{g}
\]  

---