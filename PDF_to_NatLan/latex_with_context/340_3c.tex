The energy balance equation is used to calculate the heat transferred \( Q_{12} \):  
\[
Q_{12} = \Delta E_{\text{int}} + W
\]  
where \( \Delta E_{\text{int}} = m \cdot c_V \cdot (T_2 - T_1) \) and \( W = \int p \, dV = p(V_2 - V_1) \).  

The work \( W \) is calculated as:  
\[
W = p \cdot (V_2 - V_1) = 1.769 \cdot 10^5 \cdot (0.00695 - 0.00314) \, \text{m}^3
\]  
\[
W = 6.36 \, \text{kJ}
\]  

The heat transfer \( Q_{12} \) is then calculated:  
\[
Q_{12} = 32 \cdot 0.633 \cdot (273.15 - 773.15) + 6.36
\]  
\[
Q_{12} = -30,496.96 + 6.36 = -30,490.6 \, \text{kJ}
\]  

The final result is:  
\[
Q_{12} = 7,719.257 \, \text{J}
\]  

No diagrams or additional figures are described beyond the cylinder sketch.

The energy balance equation is written as:  
\[
\frac{dE}{dt} = \dot{Q} + \dot{W} - \dot{m} \cdot \Delta h
\]  
This simplifies to:  
\[
m \cdot \frac{du}{dt} = Q
\]  

The heat transfer \( Q \) is calculated using the logarithmic mean temperature difference:  
\[
u_2 = \frac{Q}{m \cdot u_{\text{ln}}}
\]  
Substituting values:  
\[
u_2 = \frac{7.79 \cdot 257 - 7.79 \cdot 0.14}{0.14} + (-2060.614 \, \text{kJ/kg})
\]  
This results in:  
\[
u_2 = -192.300 \, \text{kJ/kg}
\]