To determine \( p_{g,1} \) and \( m_g \):  

The gas pressure \( p_{g,1} \) is calculated as the sum of the atmospheric pressure and the pressure exerted by the piston on the ice-water mixture.  
\[
p_{g,1} = p_{\text{amb}} + \frac{m_K \cdot g}{\pi \cdot r^2}
\]  
Substituting values:  
\[
p_{g,1} = 100{,}000 \, \text{Pa} + \frac{32 \, \text{kg} \cdot 9.81 \, \text{m/s}^2}{\pi \cdot (0.05 \, \text{m})^2} = 141{,}105 \, \text{Pa}
\]  

The gas mass \( m_g \) is calculated using the ideal gas law:  
\[
m_g = \frac{p \cdot V}{R \cdot T}
\]  
Substituting values:  
\[
m_g = \frac{141{,}105 \, \text{Pa} \cdot 0.003144 \, \text{m}^3}{\frac{8.314 \, \text{J/(mol·K)}}{50 \, \text{kg/kmol}} \cdot (500 + 273.15) \, \text{K}} = 0.003429 \, \text{kg} = 3.429 \, \text{g}
\]  

---