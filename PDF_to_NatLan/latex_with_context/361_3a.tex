The pressure \( p_a \) is equal to \( p_g \), and the membrane is considered.  

The pressure \( p_a \) is calculated as:  
\[
p_a = p_{\text{amb}} + \frac{m_g \cdot g}{A} + \frac{m_{\text{EW}} \cdot g}{A}
\]  
Substituting values:  
\[
p_a = 10^5 + 32 \cdot 9.81 + \frac{0.1 \cdot 9.81}{0.05} = 140039 \, \text{Pa} = 1.4 \, \text{bar}
\]  

The cross-sectional area \( A \) is calculated as:  
\[
A = \frac{D^2}{4} \cdot \pi
\]  

The ideal gas law is applied:  
\[
pV = mRT
\]  
Rearranging for \( m \):  
\[
m = \frac{pV}{RT}
\]  
Substituting values:  
\[
m = \frac{1.4 \cdot 10^5 \cdot 3.14 \cdot 10^{-3}}{8.314 \cdot 500} = 0.05349 \, \text{kg} = 53.49 \, \text{g}
\]