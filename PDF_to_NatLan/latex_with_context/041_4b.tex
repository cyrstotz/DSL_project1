The process from state 2 to state 3 is described as isentropic and adiabatic.  

The energy balance equation is written as:  
\[
\frac{dE}{dt} = \dot{m}_i [h_i] + \sum \dot{Q}_i - \sum \dot{W}_i
\]  
For an adiabatic process, \(\sum \dot{Q}_i = 0\), and the equation simplifies to:  
\[
0 = \dot{m}_{\text{R134a}} [h_2 - h_3] + \dot{W}_K
\]  

The refrigerant mass flow rate is calculated as:  
\[
\dot{m}_{\text{R134a}} = \frac{\dot{W}_K}{h_2 - h_3}
\]  

Additional relationships are provided:  
\[
h_2 = h_g(T_2)
\]  
\[
h_3 = h_f(8 \, \text{bar}) + x \cdot [h_g(8 \, \text{bar}) - h_f(8 \, \text{bar})]
\]  
where \(x\) is the vapor quality, which needs to be determined.  

For an isentropic process:  
\[
s_2 = s_3
\]  
\[
s_2 = s_g(T_2)
\]  

The explanation mentions that work is performed on the system ("weil Arbeit am System verrichtet wird").