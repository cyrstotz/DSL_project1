The equation of state for the gas is given as:  
\[
pV = nRT
\]  

The forces acting on the membrane are balanced. The area of the membrane is calculated as:  
\[
A = \pi \cdot (0.05 \, \text{m})^2 = 3.926 \cdot 10^{-3} \, \text{m}^2
\]  

The force exerted by the piston is:  
\[
F_g = m_K \cdot g = 32 \, \text{kg} \cdot 10 \, \text{m/s}^2 = 311.90 \, \text{N}
\]  

The pressure in state 1 is calculated as:  
\[
p_{g,1} = p_{\text{atm}} + \frac{F_g}{A} = 1 \, \text{bar} + \frac{311.90 \, \text{N}}{3.926 \cdot 10^{-3} \, \text{m}^2} = 1.508 \, \text{bar}
\]  

Using the ideal gas law:  
\[
\frac{pV}{RT} = n
\]  

The number of moles is calculated as:  
\[
n = \frac{p_{g,1} \cdot V_{g,1}}{R \cdot T_{g,1}} = \frac{1.7 \cdot 10^5 \, \text{Pa} \cdot 3.14 \cdot 10^{-3} \, \text{m}^3}{8.314 \, \text{J/(mol·K)} \cdot 773.15 \, \text{K}} = 0.1662 \, \text{mol}
\]  

The mass of the gas is then:  
\[
m_g = n \cdot M_g = 0.1662 \, \text{mol} \cdot 50 \, \text{kg/kmol} = 8.31 \cdot 10^{-3} \, \text{kg}
\]