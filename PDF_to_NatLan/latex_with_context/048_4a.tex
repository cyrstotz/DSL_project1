A graph is drawn showing the phase regions of a substance on a pressure-temperature (\(p\)-\(T\)) diagram. The axes are labeled as follows:  
- The vertical axis represents pressure (\(p\)) in arbitrary units.  
- The horizontal axis represents temperature (\(T\)) in degrees Celsius (\(^\circ\text{C}\)).  

The graph includes the following features:  
- A line separating the gas phase from the liquid phase.  
- A line separating the liquid phase from the solid (ice) phase.  
- A point labeled "Triple Point" where the gas, liquid, and solid phases coexist.  
- The regions are labeled as "Gas," "Liquid," and "Ice."