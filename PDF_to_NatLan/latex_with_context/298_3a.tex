The gas pressure \( p_{g,1} \) and mass \( m_g \) are determined using the following relationships:  

1. The forces acting on the piston are balanced, as shown in the diagram. The piston is fixed at point A, and the forces include the weight of the piston and the pressure exerted by the gas and the ice-water mixture.  

\[
p_{g,1} = g \left( m_K + m_{\text{EW}} \right) + p_{\text{amb}}
\]

The cross-sectional area of the cylinder is calculated as:  
\[
A = \pi \frac{D^2}{4}
\]

Substituting into the equation for \( p_{g,1} \):  
\[
p_{g,1} = \frac{g \left( m_K + m_{\text{EW}} \right)}{\pi \frac{D^2}{4}} + p_{\text{amb}}
\]

2. The gas mass \( m_g \) is calculated using the ideal gas law:  
\[
p_A V_A = m_g R T_A
\]

Rearranging:  
\[
m_g = \frac{p_A V_A}{R T_A}
\]

The specific gas constant \( R \) is derived as:  
\[
R = \frac{\bar{R}}{M_g} = \frac{8.314 \, \text{kJ/kmol·K}}{50 \, \text{kg/kmol}} = 0.166 \, \text{kJ/kg·K}
\]