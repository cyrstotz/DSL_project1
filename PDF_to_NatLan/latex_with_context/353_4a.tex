A pressure-temperature (\(p\)-\(T\)) diagram is drawn to represent the freeze-drying process. The diagram includes the following features:  
- The vertical axis is labeled as \(p\) in mbar.  
- The horizontal axis is labeled as \(T\) in \(^\circ\text{C}\).  
- Three regions are marked: "Fest" (solid), "Flüssig" (liquid), and "Gas" (gas).  
- The sublimation point is indicated on the curve separating the solid and gas regions.  
- The triple point is labeled, with a pressure of 5 mbar below it.  
- Two steps are shown:  
  - Step I involves a vertical line indicating cooling at constant pressure.  
  - Step II involves a horizontal line indicating sublimation at constant temperature, 10 K above the sublimation temperature.