The page contains a graph and a table related to thermodynamic processes, likely corresponding to Task 2 (Jet Engine Exergy). Below is the transcription and description:

---

**Graph Description:**  
The graph is a qualitative \( T \)-\( s \) diagram showing thermodynamic processes. It includes three distinct curves representing isobars (constant pressure lines). Key points are labeled as follows:  
- \( 0 \): Initial state  
- \( 1 \): Compression process  
- \( 2 \): Combustion process  
- \( 3 \): Expansion process  
- \( 4 \): Mixing process  
- \( 5 \): Final state after mixing  

Vertical lines connect points \( 0 \) to \( 1 \), \( 2 \) to \( 3 \), and \( 4 \) to \( 5 \), indicating transitions between states. The curves are annotated with approximate temperature values:  
- \( 1097 \, \text{K} \)  
- \( 1450 \, \text{K} \)  

The graph visually represents the thermodynamic cycle of the jet engine, with entropy (\( s \)) increasing along the horizontal axis and temperature (\( T \)) increasing along the vertical axis.

---

**Table Description:**  
The table contains numerical data for various states in the thermodynamic process. Columns are labeled as follows:  
- \( \dot{m} \): Mass flow rate  
- \( V \): Specific volume  
- \( T \): Temperature  
- \( p \): Pressure  

Rows correspond to states \( 0 \) through \( 5 \), with values provided for each parameter. Some entries include:  
- \( \dot{m} = 0.175 \, \text{kg/s} \)  
- \( V = 0.02225 \, \text{m}^3/\text{kg} \)  
- \( T = 1450 \, \text{K} \)  
- \( p = 0.5 \, \text{bar} \)  

The table summarizes the thermodynamic properties at each state in the jet engine cycle.

---

No additional textual explanation is visible on the page.

The following equations and expressions are written:  

1. The outlet velocity \( w_6 \) is calculated as:  
\[
w_6 = \sqrt{2 \cdot \left( h_5 - h_6 \right) + w_5^2}
\]  

2. The mass-specific increase in flow exergy is expressed as:  
\[
\Delta ex_{\text{flow}} = \left( h_6 - h_0 \right) - T_0 \cdot \left( s_6 - s_0 \right) + \frac{w_6^2}{2} - \frac{w_0^2}{2}
\]  

3. The mass-specific exergy destruction \( ex_{\text{dest}} \) is given by:  
\[
ex_{\text{dest}} = \Delta ex_{\text{flow}} - \left( h_5 - h_0 \right) - T_0 \cdot \left( s_5 - s_0 \right)
\]

The outlet velocity \( w_6 \) is calculated using the following equation:  
\[
w_6^2 = w_x^2 + \frac{2(h_4 - h_6)}{h_4 - h_6}
\]

The enthalpy balance is expressed as:  
\[
(h_5 - h_6)w_x + h_4 = h_6
\]

The exergy destruction \( ex_{\text{dest}} \) is determined using:  
\[
ex_{\text{dest}} = ex_5 - ex_6 = (c_p T_5 \ln \frac{T_5}{T_6} + c_p T_6 - T_5)
\]

The mass-specific increase in flow exergy is given by:  
\[
\Delta ex_{\text{flow}} = ex_{\text{flow},6} - ex_{\text{flow},0}
\]