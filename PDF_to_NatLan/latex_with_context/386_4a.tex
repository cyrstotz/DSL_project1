A pressure-temperature (\(p\)-\(T\)) diagram is drawn, showing the phase regions of a substance. The diagram includes the following labeled features:  
- The "Fest" (solid) region on the left.  
- The "Flüssig" (liquid) region in the middle.  
- The "Gas" (gas) region on the bottom right.  
- The triple point is marked as "Tripel."  
- Two states, labeled as "1" and "2," are connected by a horizontal line within the liquid region.  
- The diagram is shaded to indicate phase boundaries and transitions.