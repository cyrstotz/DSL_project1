Determine \( p_{g,1} \) and \( m_g \).  

The gas pressure \( p_{g,1} \) is calculated as:  
\[
p_{g,1} = \frac{\text{Force exerted from above (due to equilibrium)}}{\text{Area of the piston}}  
\]

The force \( F \) is given by:  
\[
F = m_K \cdot g + p_{\text{amb}} \cdot A  
\]

Where:  
- \( m_K = 32 \, \text{kg} \) (mass of the piston)  
- \( g = 9.81 \, \text{m/s}^2 \) (gravitational acceleration)  
- \( p_{\text{amb}} = 1 \, \text{bar} \) (ambient pressure)  

The area \( A \) of the piston is calculated using the diameter \( D = 10 \, \text{cm} \):  
\[
A = \pi \cdot \left(\frac{D}{2}\right)^2 = \pi \cdot \left(\frac{0.1}{2}\right)^2 = 7.853981 \cdot 10^{-4} \, \text{m}^2  
\]

Substituting values:  
\[
F = 32 \cdot 9.81 + 1 \cdot 7.853981 \cdot 10^{-4} = 314.301 \, \text{N}  
\]

Thus, the gas pressure is:  
\[
p_{g,1} = \frac{F}{A} + p_{\text{amb}} = \frac{314.301}{7.853981 \cdot 10^{-4}} + 1.4 \, \text{bar}  
\]

---

Using the ideal gas law to determine \( m_g \):  
\[
pV = mRT  
\]

Rearranging for \( m \):  
\[
m = \frac{pV}{RT}  
\]

Where:  
- \( R = \frac{R_u}{M_g} = 0.166285242 \, \text{kJ/kg·K} \) (specific gas constant)  
- \( V = 3.14 \, \text{L} = 3.14 \cdot 10^{-3} \, \text{m}^3 \)  
- \( T = 500^\circ\text{C} = 773.15 \, \text{K} \)  
- \( p = p_{g,1} \)  

Substituting values:  
\[
m = \frac{pV}{RT} = \frac{1.4 \cdot 3.14 \cdot 10^{-3}}{0.166285242 \cdot 773.15} = 3.418241 \, \text{g}  
\]