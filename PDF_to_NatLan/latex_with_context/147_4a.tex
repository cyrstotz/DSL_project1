The graph is a pressure-temperature (\( p \)-\( T \)) diagram labeled with the following features:  

- The vertical axis represents pressure (\( p \)) in bar.  
- The horizontal axis represents temperature (\( T \)) in degrees Celsius (\( \text{°C} \)).  
- The diagram shows a cycle with four distinct states labeled as \( 1 \), \( 2 \), \( 3 \), and \( 4 \).  

Description of the cycle:  
- State \( 1 \) begins at low pressure and low temperature.  
- The process moves upward to state \( 2 \), indicating an increase in pressure and temperature.  
- From state \( 2 \), the curve peaks and transitions to state \( 3 \), where pressure decreases while temperature remains relatively high.  
- The cycle completes as the process moves back to state \( 4 \), with both pressure and temperature decreasing.  

Arrows indicate the direction of the cycle, suggesting a closed-loop process.