The page begins with a graph labeled as a \( T \)-\( s \) diagram. The diagram qualitatively represents the thermodynamic process of a jet engine. The axes are labeled:  
- The vertical axis is \( T \) (temperature) in Kelvin.  
- The horizontal axis is \( s \) (specific entropy) in \( \frac{\text{kJ}}{\text{kg·K}} \).  

The process includes the following states:  
- State \( 0 \): Ambient conditions, \( p = p_0 \).  
- State \( 1 \): Compression to \( p = p_1 \).  
- State \( 2 \): Further compression to \( p = p_2 \).  
- State \( 3 \): Combustion at constant pressure \( p = p_2 \).  
- State \( 4 \): Expansion through the turbine, \( p' = p_4 \).  
- State \( 5 \): Mixing chamber, \( p = p_5 \).  
- State \( 6 \): Nozzle exit, \( p = p_0 \).  

Dashed lines represent isobars for \( p_0 \), \( p_1 \), \( p_2 \), \( p_4 \), and \( p_5 \). The process paths are curved, indicating changes in entropy and temperature during compression, combustion, and expansion.  

---