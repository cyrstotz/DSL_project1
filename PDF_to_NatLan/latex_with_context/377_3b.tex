The student checks whether the conditions for state 2 are consistent:  
- \( p_{g,2} = 1.5 \, \text{bar} \)  
- \( m_{\text{g,2}} = 3.6 \, \text{g} \)  
- The volume \( V_{\text{gas}} \) remains constant at \( 3.14 \cdot 10^{-3} \, \text{m}^3 \).  
- The mass of EW remains unchanged: \( m_{\text{EW,1}} = m_{\text{EW,2}} \).  

The pressure of the gas remains constant:  
\[
p_g = \text{constant}
\]  

Crossed-out content is ignored.

The gas pressure \( p_{g,2} \) is \( 1.5 \, \text{bar} \), and the gas mass \( m_g \) is \( 3.6 \, \text{g} \).  

In state 2, water is incompressible, and the relationship between the volumes is given as:  
\[
V_w = V_{w,2}
\]  

The pressure exerted by the gas does not change in state 2 and remains constant:  
\[
p_{g,2} = p_{g,1} = 1.5 \, \text{bar}
\]  

The gas follows the ideal gas law:  
\[
pV = mRT
\]  

The temperature of the gas in state 2 is calculated as:  
\[
T_{g,2} = \frac{p_{g,2} V_2}{m_g R}
\]  

The ratio of specific heat capacities is:  
\[
K = \frac{c_p}{c_v} = \frac{0.7988}{0.633} = 1.262
\]  

Using the isentropic relation for pressure and volume:  
\[
\left( \frac{p_2}{p_1} \right)^K = \frac{V_1}{V_2}
\]  

This equation is rearranged to express the relationship between \( V_1 \) and \( V_2 \):  
\[
\frac{V_1}{V_2} = \left( \frac{p_2}{p_1} \right)^{\frac{1}{K-1}}
\]