The graph is a \( T \)-\( s \) diagram illustrating the thermodynamic process of the jet engine.  

- The vertical axis represents temperature (\( T \)) in \( ^\circ\text{C} \), ranging from \( -30^\circ\text{C} \) to approximately \( 67^\circ\text{C} \).  
- The horizontal axis represents entropy (\( s \)) in unspecified units.  

Key features of the graph:  
1. State 0 is marked at the bottom left, corresponding to ambient conditions (\( T = -30^\circ\text{C} \), \( p = 0.191 \, \text{bar} \)).  
2. State 1 is labeled near state 0, indicating the mixing process.  
3. State 2 is slightly higher on the graph, showing an increase in temperature.  
4. State 3 is marked further up, representing the compression process (isentropic).  
5. State 5 is labeled near the top right, corresponding to the mixing chamber conditions (\( T = 431.9 \, \text{K} \), \( p = 0.5 \, \text{bar} \)).  
6. State 6 is marked at the far right, showing the nozzle exit conditions (\( T = 2340 \, \text{K} \), \( p = 0.191 \, \text{bar} \)).  

Arrows indicate the direction of the process flow, with labels such as "isentropic" for the compression process and "p = 0.191" for the nozzle exit.

The entropy difference between states 5 and 6 is calculated using the following relationship:  

\[
s_6 - s_5 = 0 = s^\circ(T_6) - s^\circ(T_5) - R \cdot \ln\left(\frac{p_6}{p_5}\right)
\]

Rearranging for \( s^\circ(T_6) \):  
\[
s^\circ(T_6) = s^\circ(T_5) + R \cdot \ln\left(\frac{p_6}{p_5}\right)
\]

From the reference table (TAB A-22), the entropy at \( T_5 = 431.9 \, \text{K} \) is:  
\[
s^\circ(T_5) = 2.06533 \, \frac{\text{kJ}}{\text{kg·K}}
\]

Interpolating between values in the table:  
\[
s^\circ(T_5) = 2.06533 + \frac{(2.08870 - 2.06533)}{(440 - 430)} \cdot (431.9 - 430) \, \frac{\text{kJ}}{\text{kg·K}}
\]

This gives:  
\[
s^\circ(T_5) = 2.06997742 \, \frac{\text{kJ}}{\text{kg·K}}
\]

Using TAB A-1 for the gas constant \( R \):  
\[
R = \frac{8.314 \, \frac{\text{kJ}}{\text{kmol·K}}}{28.97 \, \frac{\text{kg}}{\text{kmol}}}
\]

Substituting into the equation:  
\[
s^\circ(T_6) = 2.06997742 \, \frac{\text{kJ}}{\text{kg·K}} + R \cdot \ln\left(\frac{0.1}{0.5}\right)
\]

After calculation:  
\[
s^\circ(T_6) = 1.793645 \, \frac{\text{kJ}}{\text{kg·K}}
\]

Finally, the temperature \( T_6 \) is interpolated using entropy values:  
\[
T_6(s^\circ(T_6)) = 325 \, \text{K} + \frac{(330 - 325) \, \text{K}}{(1.79783 - 1.78249) \, \frac{\text{kJ}}{\text{kg·K}}} \cdot (1.793645 - 1.78249) \, \frac{\text{kJ}}{\text{kg·K}}
\]

This results in:  
\[
T_6 = 328.62 \, \text{K}
\]