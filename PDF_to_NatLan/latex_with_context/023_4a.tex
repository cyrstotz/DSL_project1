Two graphs are drawn:  

1. The first graph is a pressure-temperature (\( p \)-\( T \)) diagram. It shows phase regions with a curve representing the phase boundary. The curve includes oscillations, labeled "ik," likely indicating a process within the phase boundary.  
2. The second graph is also a \( p \)-\( T \) diagram. It includes labeled regions for "Gas/Fest" (gas-solid) and "Flüssig" (liquid). A line separates these regions, and a point labeled \( T_i \) is marked on the diagram.  

Below the diagrams, the following information is written:  
\[
T_i = -10^\circ\text{C}, \, p_i = 1 \, \text{mbar}
\]  
Additionally, "Temp. Lockpunkt = -16^\circ\text{C}" is noted, indicating a temperature locking point of \(-16^\circ\text{C}\).  

---