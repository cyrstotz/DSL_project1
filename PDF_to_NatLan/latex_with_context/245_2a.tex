The diagram is a \( T \)-\( s \) plot (temperature vs. entropy) illustrating the thermodynamic processes in a jet engine.  
- The y-axis is labeled \( T [K] \) (temperature in Kelvin).  
- The x-axis is labeled \( s [\frac{\text{kJ}}{\text{kg·K}}] \) (specific entropy).  
- The diagram includes the following points and processes:  
  - Point \( 0 \): Ambient conditions.  
  - Point \( 2 \): After adiabatic, reversible compression.  
  - Point \( 3 \): After adiabatic, irreversible combustion.  
  - Point \( 4/5 \): Mixing chamber.  
  - Point \( 6 \): Nozzle exit.  
- The processes are labeled as "adiabatic, rev." (reversible adiabatic) and "adiabatic, irrev." (irreversible adiabatic).  
- Isobaric lines are shown connecting points.