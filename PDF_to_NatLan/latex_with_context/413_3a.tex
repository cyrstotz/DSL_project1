To calculate the gas pressure \( p_{g,1} \) and mass \( m_g \) in state 1:  

The ideal gas law is used:  
\[
p V = m R T
\]  
Given:  
\[
T = 500^\circ\text{C}, \quad V = 3.14 \times 10^{-3} \, \text{m}^3, \quad R = \frac{33.14}{50} \, \text{kJ/kg·K}
\]  

The pressure \( p_g \) is calculated as:  
\[
p_g = p_0 + \frac{m_K g}{A} + \frac{0.1 \, \text{kg} \cdot g}{A}
\]  
where \( A = \pi \cdot (0.05)^2 \, \text{m}^2 \).  

Substituting values:  
\[
p_g = 1.15 \, \text{bar}
\]  

The mass \( m_g \) is calculated as:  
\[
m = \frac{p V}{R T} = 0.00385 \, \text{kg} \approx 0.0039 \, \text{kg}
\]  

---