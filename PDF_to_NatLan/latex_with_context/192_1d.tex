**State 1:**  
The reactor has the following properties:  
- Mass \( m_1 = 5755 \, \text{kg} \)  
- Temperature \( T_1 = 100^\circ\text{C} \)  
- Steam quality \( x_1 = 0.005 \)  

**State 2:**  
After cooling, the reactor properties are:  
- Mass \( m_2 = m_1 + \Delta m \)  
- Temperature \( T_2 = 70^\circ\text{C} \)  
- Steam quality \( x_2 = 0.005 \)  

**Energy balance using the first law of thermodynamics:**  
The first law is applied as:  
\[
m_2 u_2 - m_1 u_1 + \Delta KE + \Delta PE = \Delta m h_{\text{m}} + \sum Q - W
\]  
Here, kinetic and potential energy changes (\( \Delta KE \), \( \Delta PE \)) are negligible, and heat input equals heat output (\( Q_{\text{in}} = Q_{\text{out}} \)).  

Rearranging for \( \Delta m \):  
\[
\Delta m = \frac{m_2 u_2 - m_1 u_1}{h_{\text{m}}}
\]  

**Internal energy calculations:**  
For state 1:  
\[
u_1 = u_f + x \cdot (u_g - u_f)
\]  
From water tables:  
\[
u_f = 418.94 \, \frac{\text{kJ}}{\text{kg}}, \quad u_g = 2506.5 \, \frac{\text{kJ}}{\text{kg}}
\]  
Substituting:  
\[
u_1 = 418.94 + 0.005 \cdot (2506.5 - 418.94) = 428.38 \, \frac{\text{kJ}}{\text{kg}}
\]  

For state 2:  
\[
u_2 = u_f(70) + x \cdot (u_g(70) - u_f(70))
\]  
From water tables:  
\[
u_f(70) = 292.95 \, \frac{\text{kJ}}{\text{kg}}, \quad u_g(70) = 2498.6 \, \frac{\text{kJ}}{\text{kg}}
\]  
Substituting:  
\[
u_2 = 292.95 + 0.005 \cdot (2498.6 - 292.95) = 303.8 \, \frac{\text{kJ}}{\text{kg}}
\]  

**Specific enthalpy of added mass (\( h_{\text{m}} \)):**  
For \( T_{\text{in}} = 20^\circ\text{C} \), \( x = 0.005 \):  
\[
h_{\text{m}} = h_f(20) + x \cdot (h_g(20) - h_f(20))
\]  
From water tables:  
\[
h_f(20) = 83.96 \, \frac{\text{kJ}}{\text{kg}}, \quad h_g(20) = 2538.1 \, \frac{\text{kJ}}{\text{kg}}
\]  
Substituting:  
\[
h_{\text{m}} = 83.96 + 0.005 \cdot (2538.1 - 83.96) = 96.23 \, \frac{\text{kJ}}{\text{kg}}
\]  

**Diagram description:**  
The diagram shows a rectangular reactor with inlet and outlet streams labeled. The inlet stream is marked with \( \Delta m \), and both inlet and outlet heat flows are labeled as \( 35 \, \text{MJ} \). The reactor is labeled as operating under constant volume (\( V = \text{const.} \)).

The mass \( \Delta m_{12} \) is calculated using the energy balance equation:  
\[
\Delta m_{12} = \frac{(m_1 + \Delta m_{12}) u_2 - m_1 u_1}{u_{\text{in}}}
\]  
Rearranging terms:  
\[
\Delta m_{\text{in}} = m_1 u_2 + \Delta m_{12} u_2 - m_1 u_1
\]  
Simplifying further:  
\[
\Delta m (u_{\text{in}} - u_2) = m_1 u_2 - m_1 u_1
\]  
Finally:  
\[
\Delta m_{12} = \frac{m_1 (u_2 - u_1)}{u_{\text{in}} - u_2}
\]  
The calculated value is:  
\[
\Delta m_{12} = 3452 \, \text{kg}
\]  

---