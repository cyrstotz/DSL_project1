The gas pressure \( p_{g,1} \) is calculated using the formula:  
\[
p_{g,1} = \frac{m_K \cdot g}{A} + \frac{m_{\text{EW}} \cdot g}{A} + p_{\text{amb}}
\]  
Substituting values:  
\[
p_{g,1} = \frac{32 \, \text{kg} \cdot 9.81 \, \text{m/s}^2}{(0.05 \, \text{m})^2 \cdot \pi} + \frac{0.1 \, \text{kg} \cdot 9.81 \, \text{m/s}^2}{(0.05 \, \text{m})^2 \cdot \pi} + 1 \, \text{bar}
\]  
\[
p_{g,1} = 140099.4408 \, \text{Pa} \quad \text{or} \quad 1.4 \, \text{bar}
\]  

The gas mass \( m_g \) is calculated using:  
\[
p_{g,1} \cdot V_{g,1} = m_g \cdot R \cdot T_{g,1}
\]  
Rearranging:  
\[
m_g = \frac{p_{g,1} \cdot V_{g,1}}{R \cdot T_{g,1}}
\]  
Substituting values:  
\[
m_g = \frac{140000 \cdot 0.00314}{77.3 \cdot (500 + 273.15)} \cdot \frac{1}{50 \, \text{kg/kmol}}
\]  
\[
m_g = 0.00341943 \, \text{kg}
\]  

---