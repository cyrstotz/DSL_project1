The problem begins by determining the gas pressure \( p_{g,1} \) and mass \( m_g \) in state 1.  

The given values are:  
\[
V_{g,1} = 3.14 \, \text{L} = 3.14 \times 10^{-3} \, \text{m}^3
\]
\[
T_{g,1} = 500^\circ\text{C} = 773.15 \, \text{K}
\]

The ideal gas law is used:  
\[
p V = m R T
\]

The piston pressure \( p_{g,1} \) is calculated as:  
\[
p_{g,1} = \frac{F_{\text{piston}}}{A} + p_{\text{amb}} + m_{\text{EW}} \cdot g
\]

Where:  
- \( F_{\text{piston}} = m_K \cdot g \) (force due to piston mass)  
- \( A = \pi \left(\frac{d}{2}\right)^2 \) (area of the piston)  
- \( p_{\text{amb}} = 1 \times 10^5 \, \text{Pa} \) (ambient pressure)  
- \( m_{\text{EW}} = 0.1 \, \text{kg} \) (mass of ice-water mixture)  

The piston area is calculated as:  
\[
A = \pi \left(\frac{10 \times 10^{-2}}{2}\right)^2 = \pi \cdot \frac{1}{400} \, \text{m}^2
\]

Substituting values:  
\[
p_{g,1} = \left(32.1 \, \text{kg} \cdot 9.81 \, \text{m/s}^2 \cdot \frac{\pi}{400} \, \text{m}^2\right) + 1 \times 10^5 \, \text{Pa} = 100002.42 \, \text{Pa}
\]

Next, the gas mass \( m_g \) is calculated using:  
\[
m = \frac{p V}{R T}
\]

The gas constant \( R \) is determined as:  
\[
R = \frac{R_{\text{universal}}}{M} = \frac{8.314 \, \text{J/mol·K}}{50 \, \text{kg/kmol}} = 0.16628 \, \text{J/kg·K}
\]

Substituting values into the ideal gas law:  
\[
m = \frac{100002.42 \cdot 3.14 \times 10^{-3}}{0.16628 \cdot 773.15} \approx 1.5 \, \text{kg}
\]  

The final gas mass is approximately \( 1.5 \, \text{kg} \).  

No diagrams or additional figures are present.

The mass of the gas \( m \) is calculated using the ideal gas law:  
\[
m = \frac{p V}{R T}
\]  
Substituting the given values:  
\[
R = 162.28 \, \text{J/(kg·K)}, \, T = 773.15 \, \text{K}, \, p = 1 \, \text{bar} = 10^5 \, \text{Pa}, \, V = 3.14 \times 10^{-3} \, \text{m}^3
\]  
The calculation yields:  
\[
m = 0.00244 \, \text{kg} \quad \text{or} \quad m = 0.1402 \, \text{kg} \, \text{(crossed out)}
\]