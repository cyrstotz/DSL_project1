The gas constant \( R \) is calculated as:  
\[
R = \frac{R_u}{M} = \frac{8.374 \, \text{kJ/(kmol·K)}}{50 \, \text{kg/kmol}} = 0.1663 \, \text{kJ/(kg·K)}
\]

The cross-sectional area \( A \) of the cylinder is determined using the diameter \( D = 0.1 \, \text{m} \):  
\[
A = \pi \left(\frac{D}{2}\right)^2 = \pi \left(\frac{0.1}{2}\right)^2 = 0.00785 \, \text{m}^2
\]

The pressure \( p_{g,1} \) in state 1 is calculated as:  
\[
p_{g,1} = p_0 + \frac{g}{A} (m_K + m_{\text{EW}}) = 1 \, \text{bar} + \frac{9.81 \, \text{m/s}^2}{0.00785 \, \text{m}^2} (32 \, \text{kg} + 0.1 \, \text{kg}) = 1.401 \, \text{bar}
\]

The ideal gas law is applied:  
\[
pV = mRT
\]
Given \( V_{g,1} = 0.00314 \, \text{m}^3 \) and \( T_{g,1} = 773.15 \, \text{K} \), the gas mass \( m_g \) is calculated as:  
\[
m_g = \frac{p_{g,1} V_{g,1}}{R T_{g,1}} = \frac{1.401 \, \text{bar} \cdot 0.00314 \, \text{m}^3}{0.1663 \, \text{kJ/(kg·K)} \cdot 773.15 \, \text{K}} = 3.4275 \cdot 10^{-3} \, \text{kg}
\]

---