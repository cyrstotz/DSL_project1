The pressure \( p_{g,1} \) is calculated using the formula:  
\[
p_{g,1} = p_{\text{amb}} + \frac{m_K \cdot g}{A} + \frac{m_{\text{EW}} \cdot g}{A}
\]  
Substituting the values:  
\[
p_{g,1} = 100000 \, \text{Pa} + \frac{32 \, \text{kg} \cdot 9.81 \, \text{m/s}^2}{0.007854 \, \text{m}^2} + \frac{0.1 \, \text{kg} \cdot 9.81 \, \text{m/s}^2}{0.007854 \, \text{m}^2}
\]  
\[
p_{g,1} = 140084 \, \text{Pa}
\]  

The cross-sectional area of the cylinder is calculated as:  
\[
A = \pi \cdot (0.05 \, \text{m})^2 = 0.007854 \, \text{m}^2
\]  

The mass of the gas \( m_g \) is determined using the ideal gas law:  
\[
m_g = \frac{p_g \cdot V_g}{R_g \cdot T_g}
\]  
Substituting the values:  
\[
R_g = \frac{R}{M_g} = \frac{8.314 \, \text{J/mol·K}}{50 \, \text{kg/kmol}} = 166.28 \, \text{J/kg·K}
\]  
\[
m_g = \frac{140084 \, \text{Pa} \cdot 0.00314 \, \text{m}^3}{166.28 \, \text{J/kg·K} \cdot 773 \, \text{K}}
\]  
\[
m_g = 0.000003428 \, \text{kg}
\]