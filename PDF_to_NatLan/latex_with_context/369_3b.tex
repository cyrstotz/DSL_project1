Since the ice has not completely melted, the heat transfer \( Q \) is zero, and the temperature remains constant.  

The temperature and pressure must decrease because heat flows from the warm gas to the cold ice-water mixture.  

In thermodynamic equilibrium, the gas and ice must have the same temperature, which is \( 0^\circ\text{C} \). If the temperature were higher, no ice would remain, and equilibrium would not be achieved.  

Using the equation:  
\[
\frac{p_1}{T_1} = \frac{p_2}{T_2}
\]  
Substituting values:  
\[
\frac{500 \, \text{K}}{0.49 \, \text{bar}} = \frac{273.15 \, \text{K}}{p_2}
\]  
This gives:  
\[
p_2 = 0.49 \, \text{bar}
\]

The pressure \( p_2 \) is equal to \( p_1 \), as the pressure is determined by \( p_0 \) (ambient pressure), and the mass does not change.  

The temperature ratio \( \frac{T_2}{T_1} \) is used to calculate the pressure ratio:  
\[
\frac{p_2}{p_1} = \left( \frac{T_2}{T_1} \right)^{\frac{R}{c_v}}
\]  
where \( R \) is the gas constant and \( c_v = 0.633 \, \text{kJ/kg·K} \).  

The specific heat ratio \( n \) is calculated as:  
\[
n = \frac{c_p}{c_v} = 7.2637
\]