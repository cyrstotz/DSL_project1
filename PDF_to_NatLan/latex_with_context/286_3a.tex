The pressure \( p \) is calculated using the formula:  
\[
p = p_{\text{atm}} + \frac{m_K \cdot g}{A}
\]  
where \( p_{\text{atm}} = 1 \times 10^5 \, \text{Pa} \), \( m_K = 32 \, \text{kg} \), \( g = 9.81 \, \text{m/s}^2 \), and \( A = \pi r^2 \) with \( r = 0.05 \, \text{m} \).  

Substituting the values:  
\[
p = 1 \times 10^5 + \frac{32 \cdot 9.81}{\pi \cdot (0.05)^2} = 411005.41 \, \text{Pa} = 4.11 \, \text{bar}
\]  

The gas pressure \( p \) is determined to be \( 4.11 \, \text{bar} \).  

Next, the gas mass \( m_g \) is calculated using the ideal gas law:  
\[
p \cdot V = n \cdot R \cdot T
\]  
Rearranging for \( n \):  
\[
n = \frac{p \cdot V}{R \cdot T}
\]  
Substituting \( p = 4.11 \, \text{bar} = 411005.41 \, \text{Pa} \), \( V = 3.14 \, \text{L} = 3.14 \times 10^{-3} \, \text{m}^3 \), \( R = 8.314 \, \text{J/mol·K} \), and \( T = 773.15 \, \text{K} \):  
\[
n = \frac{411005.41 \cdot 3.14 \times 10^{-3}}{8.314 \cdot 773.15} = 6.839 \times 10^{-2} \, \text{mol}
\]  

The molar mass of the gas is \( M_g = 50 \, \text{kg/kmol} = 50 \, \text{g/mol} \). Therefore, the mass \( m_g \) is:  
\[
m_g = n \cdot M_g = 6.839 \times 10^{-2} \cdot 50 = 3.42 \, \text{g}
\]