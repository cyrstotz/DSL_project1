A graph is drawn representing the process in a temperature-entropy (\( T \)-\( s \)) diagram. The axes are labeled as follows:  
- The vertical axis is labeled \( T [K] \), representing temperature in Kelvin.  
- The horizontal axis is labeled \( s [\frac{\text{kJ}}{\text{kg·K}}] \), representing specific entropy in kilojoules per kilogram per Kelvin.  

The diagram shows six states labeled \( 1 \), \( 2 \), \( 3 \), \( 4 \), \( 5 \), and \( 6 \), connected by curves and lines.  
- States \( 1 \) and \( 2 \) are connected by a curve.  
- States \( 2 \) and \( 3 \) are connected by a vertical line.  
- States \( 3 \) and \( 4 \) are connected by a curve.  
- States \( 4 \) and \( 5 \) are connected by a diagonal line.  
- States \( 5 \) and \( 6 \) are connected by a curve.  

The pressures \( p_0 \), \( p_3 \), \( p_4 \), and \( p_5 \) are marked along the curves, indicating isobars (lines of constant pressure).  

---