The weight of the piston is given as \( 32 \, \text{kg} \). The gas pressure \( p_{g,1} \) is calculated using the formula:  
\[
p_{g,1} = \frac{m_K \cdot g}{\pi \cdot \left(\frac{D}{2}\right)^2} + \frac{m_{\text{EW}} \cdot g}{\pi \cdot \left(\frac{D}{2}\right)^2} + p_{\text{amb}}
\]  
Substituting the values:  
\[
p_{g,1} = \frac{32 \cdot 9.81}{\pi \cdot \left(\frac{0.1}{2}\right)^2} + \frac{0.1 \cdot 9.81}{\pi \cdot \left(\frac{0.1}{2}\right)^2} + 1.0 \cdot 10^5
\]  
\[
p_{g,1} = 140094.4 \, \text{Pa} = 1.401 \, \text{bar}
\]  

The mass of the gas \( m_g \) is calculated using the ideal gas law:  
\[
m_g = \frac{p_1 \cdot V_1}{\frac{M_g}{R} \cdot T}
\]  
Substituting the values:  
\[
m_g = \frac{1.401 \cdot 10^5 \cdot 3.14 \cdot 10^{-3}}{\frac{50}{8.314} \cdot 773.15}
\]  
\[
m_g = 3.429 \, \text{kg}
\]