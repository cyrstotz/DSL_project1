The transition from state 5 to state 6 is described as an isentropic expansion process.  

The following formulas and values are provided:  
1. The kinetic energy per unit mass is given as:  
\[
ke = \frac{w^2}{2}
\]  
where \( w \) is the velocity.  

2. The relationship for temperature \( T_6 \) is derived using the isentropic relation:  
\[
T_6 = T_5 \cdot \left( \frac{p_5}{p_6} \right)^{\frac{k-1}{k}}
\]  
where \( k = 1.4 \) is the specific heat ratio.  

Given values:  
\[
p_0 = p_6 = 0.191 \, \text{bar}, \quad T_5 = 431.9 \, \text{K}, \quad p_5 = 0.5 \, \text{bar}
\]  

Substituting into the formula:  
\[
T_6 = T_5 \cdot \left( \frac{p_5}{p_6} \right)^{\frac{k-1}{k}} = 431.9 \cdot \left( \frac{0.5}{0.191} \right)^{\frac{1.4-1}{1.4}}
\]  
\[
T_6 = 568.58 \, \text{K}
\]  

The final temperature at state 6 is calculated as \( T_6 = 568.58 \, \text{K} \).

The energy balance for the steady-state flow process (SFP) is written as:  
\[
0 = \dot{m}_g \left[ h_5 - h_6 + \frac{w_5^2 - w_6^2}{2} \right] + 0
\]  
This equation accounts for the enthalpy difference and kinetic energy terms. The term "Schaufelrad IG" (likely referring to a turbine or impeller) is noted.  

Rewriting the energy balance:  
\[
0 = \dot{m}_g \left[ \int c_p (T_5 - T_6) + \frac{w_5^2 - w_6^2}{2} \right]
\]  

From this, the velocity term is isolated:  
\[
\frac{w_6^2 - w_5^2}{2} = \dot{m}_g \int c_p (T_5 - T_6)
\]  

The outlet velocity \( w_6 \) is derived as:  
\[
w_6^2 = 2 \left( c_p (T_5 - T_6) \right) - w_5^2
\]  

Finally, solving for \( w_6 \):  
\[
w_6 = \sqrt{2 \left( c_p (T_5 - T_6) \right) - w_5^2}
\]  

---