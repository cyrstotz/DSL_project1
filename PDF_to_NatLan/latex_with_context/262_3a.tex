The task involves determining the gas pressure \( p_{g,1} \) and mass \( m_g \) in state 1.  

### Explanation and Derivations:  
1. **Pressure Calculation**:  
   The pressure \( p_{g,1} \) in state 1 is derived using the force balance on the piston:  
   \[
   p_{g,1} = \frac{m_{\text{ew}} \cdot g}{A} + \frac{m_K \cdot g}{A} + p_{\text{amb}}
   \]  
   Here:  
   - \( g = 9.81 \, \text{m/s}^2 \) is the gravitational acceleration.  
   - \( A \) is the cross-sectional area of the cylinder.  
   - \( m_{\text{ew}} \) is the mass of the ice-water mixture.  
   - \( m_K = 32 \, \text{kg} \) is the mass of the piston.  
   - \( p_{\text{amb}} = 1 \, \text{bar} \) is the ambient pressure.  

2. **Area of Cylinder**:  
   The cross-sectional area \( A \) is calculated as:  
   \[
   A = \pi r^2 = \pi \left( \frac{D}{2} \right)^2 = \pi \left( \frac{0.1}{2} \right)^2 = 0.007854 \, \text{m}^2
   \]  

3. **Final Pressure**:  
   Substituting values into the pressure equation:  
   \[
   p_{g,1} = 170094.98 \, \text{Pa} = 11.407 \, \text{bar}
   \]  

---

4. **Mass of Gas \( m_g \)**:  
   The mass of the gas is calculated using the ideal gas law:  
   \[
   pV = mRT
   \]  
   Rearranging for \( m_g \):  
   \[
   m_g = \frac{pV}{RT}
   \]  
   Substituting known values:  
   - \( p = 11.407 \, \text{bar} = 1140700 \, \text{Pa} \)  
   - \( V = 3.14 \, \text{L} = 3.14 \times 10^{-3} \, \text{m}^3 \)  
   - \( R = \frac{R_{\text{univ}}}{M_g} = \frac{8.314 \, \text{J/mol·K}}{50 \, \text{kg/kmol}} = 0.16628 \, \text{m}^3 \cdot \text{Pa}/\text{K·kg} \)  
   - \( T = 500^\circ\text{C} = 773.15 \, \text{K} \)  

   Calculation:  
   \[
   m_g = \frac{1140700 \cdot 3.14 \times 10^{-3}}{0.16628 \cdot 773.15} = 0.003921 \, \text{kg}
   \]  

---

### Diagram Description:  
The diagram shows a cylinder containing gas at the bottom, with a piston resting on top. The piston is subjected to three forces:  
- The weight of the piston (\( m_K \cdot g \)).  
- The weight of the ice-water mixture (\( m_{\text{ew}} \cdot g \)).  
- Ambient pressure (\( p_{\text{amb}} \)).  
The gas pressure \( p_{g,1} \) counteracts these forces.  

No additional content found beyond this derivation and explanation.