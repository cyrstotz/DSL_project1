The system is described as steady-state, with no work (\( W = 0 \)), negligible kinetic energy, and no potential energy changes. The energy balance is expressed as:  
\[
\dot{Q} = \dot{m} \cdot (h_{\text{out}} - h_{\text{in}})
\]  
where \( \dot{Q} \) is the heat flow, \( \dot{m} \) is the mass flow rate, and \( h_{\text{out}} \) and \( h_{\text{in}} \) are the specific enthalpies at the outlet and inlet, respectively.  

From Table A2, the properties of saturated liquid water are used:  
- At \( T_{\text{in}} = 70^\circ\text{C} \):  
  \[
  h_{\text{in}} = 292.98 \, \text{kJ/kg}, \quad \rho_{\text{in}} = 0.9779 \, \text{kg/L}
  \]  
- At \( T_{\text{out}} = 100^\circ\text{C} \):  
  \[
  h_{\text{out}} = 419.04 \, \text{kJ/kg}, \quad \rho_{\text{out}} = 0.9584 \, \text{kg/L}
  \]  

The heat flow removed by the coolant is calculated as:  
\[
\dot{Q} = \dot{m} \cdot (h_{\text{out}} - h_{\text{in}}) = 0.3 \, \text{kg/s} \cdot (419.04 \, \text{kJ/kg} - 292.98 \, \text{kJ/kg}) = 37.81 \, \text{kW}
\]