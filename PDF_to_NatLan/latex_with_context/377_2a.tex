A graph is drawn representing the thermodynamic process in a jet engine on a \( T \)-\( s \) diagram. The axes are labeled as follows:  
- The vertical axis is \( T \) (temperature) in Kelvin (\( \text{K} \)).  
- The horizontal axis is \( s \) (specific entropy) in \( \frac{\text{kJ}}{\text{kg·K}} \).  

The process includes the following states and transitions:  
1. State 0: Ambient conditions (\( p_0 \), \( T_0 \)).  
2. State 1: Isentropic compression.  
3. State 3: Combustion process (isobaric).  
4. State 4: Expansion in the turbine (adiabatic).  
5. State 5: Mixing chamber.  
6. State 6: Nozzle exit (isentropic).  

Key annotations:  
- The curve from state 0 to state 1 is labeled "isentropic."  
- The curve from state 3 to state 4 is labeled "polytrope."  
- State 5 is labeled with \( p_5 = 0.5 \, \text{bar} \).  
- State 6 is labeled with \( p_6 = p_0 = 0.191 \, \text{bar} \).