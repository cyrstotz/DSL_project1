The pressure \( p_{g,1} \) is equal to the pressure exerted by the piston, \( p_{\text{piston}} \).  

The formula for the pressure exerted by the piston is given as:  
\[
p_{\text{piston}} = p_{\text{amb}} + \frac{(m_K + m_{\text{EW}}) \cdot g}{A}
\]  
Where:  
- \( A \) is the cross-sectional area of the cylinder, calculated as:  
\[
A = \pi \cdot r^2 = \pi \cdot (5 \, \text{cm})^2 = \pi \cdot 0.05^2 \, \text{m}^2
\]  

Substituting values:  
\[
p_{\text{piston}} = 10^5 \, \text{Pa} + \frac{32 \, \text{kg} \cdot 9.81 \, \text{m/s}^2}{\pi \cdot 0.05^2 \, \text{m}^2} = 1.048 \, \text{bar}
\]  

The mass of the gas \( m_g \) is calculated using the ideal gas law:  
\[
m_g = \frac{p \cdot V}{R \cdot T}
\]  
Substituting values:  
\[
m_g = \frac{1.04 \cdot 10^5 \, \text{Pa} \cdot 3.14 \cdot 10^{-3} \, \text{m}^3}{8.314 \, \text{J/mol·K} \cdot 773.15 \, \text{K}}
\]  
\[
m_g = \frac{327.77}{50 \, \text{kg/kmol}} = 2.54 \, \text{g} = 0.00254 \, \text{kg}
\]