The pressure consists of the weight of the fluid, the weight of the piston, and the ambient pressure:  
\[
p_{g,1} = \frac{m_g g}{A} + \frac{m_{\text{EW}} g}{A} + p_{\text{amb}}
\]  
Given:  
- Diameter \( D = 10 \, \text{cm} = 0.1 \, \text{m} \)  
- Area \( A = \pi \left(\frac{D}{2}\right)^2 = \pi \left(\frac{0.1}{2}\right)^2 = 0.0314 \, \text{m}^2 \)  

Substituting values:  
\[
p_{g,1} = \frac{32 \cdot 9.81}{0.0314} + \frac{0.1 \cdot 9.81}{0.0314} + 1 \, \text{bar}
\]  
\[
p_{g,1} = 1.1002 \, \text{bar}
\]  

The mass of the gas is calculated using the ideal gas law:  
\[
pV = mRT
\]  
Rearranging:  
\[
m = \frac{pV}{RT}
\]  
Substituting values:  
\[
m = \frac{1.9022 \cdot 10^5 \, \text{Pa} \cdot 0.00314 \, \text{m}^3}{8.314 \cdot 50 \cdot 773.15 \, \text{K}}
\]  
\[
m = 0.0027 \, \text{kg} \approx 2.69 \, \text{g}
\]  

---