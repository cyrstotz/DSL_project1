The goal is to determine the outlet temperature \( T_6 \) and outlet velocity \( w_6 \).  

### Temperature Calculation:  
The temperature \( T_6 \) is calculated using the ideal gas relation for an adiabatic process:  
\[
\left( \frac{T_6}{T_5} \right) = \left( \frac{p_6}{p_5} \right)^{\frac{n-1}{n}}
\]  
where \( n = \kappa = 1.4 \), \( p_6 = p_0 = 0.191 \, \text{bar} \), \( p_5 = 0.5 \, \text{bar} \), and \( T_5 = 431.9 \, \text{K} \).  

Substituting values:  
\[
T_6 = T_5 \cdot \left( \frac{p_6}{p_5} \right)^{\frac{n-1}{n}}
\]  
\[
T_6 = 431.9 \cdot \left( \frac{0.191}{0.5} \right)^{\frac{1.4-1}{1.4}}
\]  
\[
T_6 = 328.07 \, \text{K}
\]  

### Velocity Calculation:  
The outlet velocity \( w_6 \) is determined using the energy balance for an adiabatic and reversible process.  

Energy balance:  
\[
0 = \dot{m} \left[ h_5 - h_6 + \frac{w_5^2 - w_6^2}{2} + g(z_5 - z_6) \right]
\]  
Since \( g(z_5 - z_6) \) is negligible, the equation simplifies to:  
\[
0 = \dot{m} \left[ h_5 - h_6 + \frac{w_5^2 - w_6^2}{2} \right]
\]  

Rewriting:  
\[
h_5 - h_6 + \frac{w_5^2 - w_6^2}{2} = 0
\]  

Further derivations include terms for \( h_5 - h_6 \) and the specific gas constant \( R \):  
\[
h_5 - h_6 + \frac{w_5^2 - w_6^2}{2} = \frac{n \cdot R \cdot (T_6 - T_5)}{n-1}
\]  

Final expression for \( w_6 \):  
\[
\dot{m} \cdot \left( \frac{n \cdot R \cdot (T_6 - T_5)}{n-1} \right) = W_t
\]  

Additional notes:  
- The integral \( \int_{p_1}^{p_2} v \, dp \) is referenced for work calculations.  
- \( W_t \) represents turbine work, and the derivation includes terms for \( \dot{m} \cdot R \cdot (T_6 - T_5) \).  

No diagrams are present on this page.

The outlet velocity \( w_6 \) is calculated using the following equation:  
\[
w_6 = \sqrt{n \cdot R \cdot (T_6 - T_s) \cdot \frac{1}{n-1} + h_c - h_s + w_s^2} \cdot 2
\]  
where \( R = \frac{\bar{R}}{M_{\text{luft}}} \), and the values are:  
\[
\bar{R} = 0.296 \, \text{kJ/kg·K}, \quad M_{\text{luft}} = 28.96 \, \text{kg/kmol}, \quad c_p = 1.006 \, \text{kJ/kg·K}, \quad w_s = 220 \, \text{m/s}
\]  

The resulting outlet velocity is:  
\[
w_6 = 310.61 \, \text{m/s}
\]  

---

The task involves calculating the outlet velocity \( w_6 \) and temperature \( T_6 \) for the jet engine, as well as determining the exergy destruction.

The stationary exergy balance equation is written as:  
\[
0 = \dot{m} \left( h_e - \ldots \right) + \dot{Q} e_{x,Q} - W_{\text{turbine}} - \dot{e}_{x,\text{fuel}}
\]  
where \( e_{x,\text{fuel}} \) represents the exergy destruction due to fuel combustion.  

The exergy destruction is expressed as:  
\[
e_{x,\text{fuel}} = \Delta e_{x,\text{stream},6} + e_{x,Q,6} - W_{\text{turbine}}
\]  
Simplifying further:  
\[
e_{x,\text{fuel}} = \Delta e_{x,\text{stream},6} - W_{\text{turbine}}
\]  

For the turbine, the work \( W_{\text{turbine}} \) is calculated using:  
\[
W_{\text{turbine}} = -n \cdot R \cdot \frac{T_6 - T_5}{1 - n}
\]  
where \( n \) is the efficiency factor, \( R \) is the specific gas constant, \( T_6 \) is the outlet temperature, and \( T_5 \) is the inlet temperature.  

The value of \( W_{\text{turbine}} \) is given as:  
\[
W_{\text{turbine}} = -109260.9 \, \frac{\text{kJ}}{\text{kg}}
\]  

Finally, the exergy destruction is calculated as:  
\[
e_{x,\text{fuel}} = 100 \, \frac{\text{kJ}}{\text{kg}} + 109260.9 \, \frac{\text{kJ}}{\text{kg}}
\]  
Resulting in:  
\[
e_{x,\text{fuel}} = 204.26 \, \frac{\text{kJ}}{\text{kg}}
\]  

No diagrams or graphs are present on the page.