From the problem statement, we know that the coolant is modeled as an ideal liquid. The energy balance equation is given as:  
\[
\frac{dE}{dt} = \sum_i \dot{m}_i \left[ h_i + \text{ke}_i + \text{pe}_i \right] + \sum_j Q_j - \sum_k W_k
\]  
Here, kinetic energy (\( \text{ke} \)) and potential energy (\( \text{pe} \)) are negligible, and the system is adiabatic to the surroundings. Therefore, the equation simplifies to:  
\[
0 = \dot{m}_e \left[ h_{\text{in}} - h_{\text{out}} \right] + \dot{Q}_{\text{ab}}
\]  

The heat flow removed by the coolant (\( \dot{Q}_{\text{ab}} \)) is calculated as:  
\[
\dot{Q}_{\text{ab}} = \dot{m}_{\text{in}} \left[ h_{\text{out}} - h_{\text{in}} \right]
\]  

From the water tables (Table A-2):  
\[
h_{\text{in}} = h_f(70^\circ\text{C}) = 292.98 \, \text{kJ/kg}  
\]  
\[
h_{\text{out}} = h_f(100^\circ\text{C}) = 419.04 \, \text{kJ/kg}  
\]  

Substituting the values:  
\[
\dot{Q}_{\text{ab}} = 0.3 \, \text{kg/s} \left[ 419.04 \, \text{kJ/kg} - 292.98 \, \text{kJ/kg} \right] = 37.82 \, \text{kW}
\]