An energy balance is performed around the jet engine.  
The following equations and relationships are derived:  
1. The temperature at state 6 (\( T_6 \)) is calculated using the isentropic relation:  
\[
T_6 = T_5 \left( \frac{p_6}{p_5} \right)^{\frac{\kappa - 1}{\kappa}}
\]  
where \( \kappa = 1.4 \).  
The result is \( T_6 = 325.07 \, \text{K} \).  

2. The energy balance equation is written as:  
\[
0 = \dot{m} \left( h_0 - h_1 \right) + q_B - \dot{Q}
\]  
\[
0 = \dot{m} \left( h_1 - h_B \right) + \left( w_e^2 - w_a^2 \right) + Q
\]  
\[
0 = 2 \left( h_1 - h_B \right) + \left( w_e^2 - w_a^2 \right)
\]  
\[
2 \left( h_1 - h_B \right) = Q + \left( w_e^2 - w_a^2 \right)
\]  

The equations are used to calculate the outlet velocity \( w_6 \) and other thermodynamic properties.  

A block diagram is included to visually represent the energy flow and key states in the jet engine process. The diagram shows the inlet and outlet streams, heat addition \( q_B \), and the relationship between pressures and temperatures at various states.  

Additional notes:  
- \( n = 1.4 \) is used for the isentropic exponent.  
- Intermediate calculations for \( T_6 \) and other variables are shown.  

No further content is visible.

The equation for energy conservation in the jet engine process is written as:  
\[
0 = \left[ h_5 - h_6 + \frac{w_5^2}{2} - \frac{w_6^2}{2} \right] + \dot{Q}^o - \dot{W}^o
\]  

The velocity \( w_6 \) is derived using the following expression:  
\[
w_6^2 = 2 \cdot z \cdot (h_5 - h_6) + w_5^2
\]  

Substituting the enthalpy difference and temperature terms:  
\[
w_6^2 = 2 \cdot c_{p,\text{air}} \cdot (T_5 - T_6) + w_5^2
\]  

The final formula for \( w_6 \) is:  
\[
w_6 = \sqrt{2 \cdot 1.006 \, \text{kJ/kg·K} \cdot (431.9 \, \text{K} - 320.97 \, \text{K}) + 220^2 \, \text{m}^2/\text{s}^2}
\]  

No diagrams or additional figures are present on this page.