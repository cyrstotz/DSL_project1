Given data:  
\[
p_5 = 0.5 \, \text{bar}, \quad p_6 = 0.197 \, \text{bar}, \quad T_5 = 437.9 \, \text{K}, \quad w_5 = 220 \, \text{m/s}
\]  

The temperature at state 6 (\( T_6 \)) is calculated using the isentropic relation:  
\[
\frac{T_6}{T_5} = \left( \frac{p_6}{p_5} \right)^{\frac{\kappa - 1}{\kappa}}  
\]  
Substituting values:  
\[
T_6 = T_5 \cdot \left( \frac{p_6}{p_5} \right)^{\frac{\kappa - 1}{\kappa}} = 437.9 \, \text{K} \cdot \left( \frac{0.197 \, \text{bar}}{0.5 \, \text{bar}} \right)^{\frac{0.4}{1.4}}  
\]  
\[
T_6 = 328.075 \, \text{K}  
\]  

Using the first law of thermodynamics for the nozzle:  
\[
\dot{Q} = \dot{m} \left[ h_6 - h_5 + \frac{w_6^2 - w_5^2}{2} \right]  
\]  
Since the nozzle is adiabatic (\( \dot{Q} = 0 \)):  
\[
h_6 - h_5 = -\frac{w_6^2 - w_5^2}{2}  
\]  

The velocity at state 6 (\( w_6 \)) is calculated using:  
\[
w_6^2 = 2 \cdot c_p \cdot (T_5 - T_6) + w_5^2  
\]  
Substituting values:  
\[
w_6 = \sqrt{2 \cdot c_p \cdot (T_5 - T_6) + w_5^2} = \sqrt{2 \cdot 1.006 \, \text{kJ}/\text{kg·K} \cdot (437.9 \, \text{K} - 328.075 \, \text{K}) + (220 \, \text{m/s})^2}  
\]  
\[
w_6 = \sqrt{2 \cdot 1.006 \cdot 109.825 + 220^2} = \sqrt{2 \cdot 1006 \cdot 109.825 + 48400} = 507.29 \, \text{m/s}  
\]  

Final results:  
\[
T_6 = 328.075 \, \text{K}, \quad w_6 = 507.29 \, \text{m/s}  
\]