The total mass flow rate of the refrigerant (\( \dot{m}_{\text{ges}} \)) is calculated using the following equations:  

1. The inlet temperature of the evaporator is given as \( T_{\text{in}} = T_i - 6 \, \text{K} = 3^\circ \text{C} \).  
2. The first law of thermodynamics around the compressor is applied:  
   \[
   0 = \dot{m} \left[ h_2 - h_3 \right] + \dot{W}_K
   \]  
   Rearranging for \( \dot{m} \):  
   \[
   \dot{m} = \frac{\dot{W}_K}{h_2 - h_3}
   \]  

3. \( h_3 \) is obtained from Table A-12 at 8 bar using interpolation:  
   \[
   y = x - x_1 \cdot \frac{y_2 - y_1}{x_2 - x_1} + y_1
   \]  
   Where:  
   - \( x_1 = -6^\circ \text{C} \), \( x_2 = 2^\circ \text{C} \)  
   - \( y_1 = 234.08 \, \text{kJ/kg} \), \( y_2 = 247.23 \, \text{kJ/kg} \)  

   After interpolation, \( h_3 = 247.23 \, \text{kJ/kg} \).  

4. \( h_2 \) is determined at \( T_2 = -22^\circ \text{C} \), where the refrigerant is fully evaporated:  
   \[
   h_2 = 234.08 \, \text{kJ/kg}
   \]  

Descriptions of interpolations and calculations are provided to determine the enthalpy values at specific states.  

---

No additional content found beyond these tasks.