The energy balance equation for the reactor is written as:  
\[
\frac{dE}{dt} = \dot{m}_{\text{in}} h_{\text{in}}(t) + \dot{Q}_R - \dot{Q}_{\text{out}} - \dot{m}_{\text{out}} h_{\text{out}}(t) = 0
\]  

The heat flow removed by the coolant, \( \dot{Q}_{\text{out}} \), is expressed as:  
\[
\dot{Q}_{\text{out}} = \dot{Q}_R + \dot{m}_{\text{in}} \left( h_{\text{in}} - h_{\text{out}} \right)
\]  

Here, \( \Delta h = h_{\text{in}} - h_{\text{out}} \).  

Substituting values:  
\[
\dot{Q}_{\text{out}} = 100,000 \, \text{W} + 0.3 \, \frac{\text{kg}}{\text{s}} \cdot \Delta h
\]

The diagram is a pressure-temperature (\(P\)-\(T\)) graph. It shows three distinct states labeled as \(1\), \(2\), and \(3\).  

- State \(1\) is at a lower pressure and temperature.  
- State \(2\) is at a higher temperature but similar pressure to \(1\).  
- State \(3\) is at a higher pressure and temperature compared to \(1\) and \(2\).  

The graph includes curved lines connecting the states, indicating transitions between them. The axes are labeled:  
- The vertical axis represents pressure (\(P\)).  
- The horizontal axis represents temperature (\(T\)).  

No additional textual explanation or equations are provided.