The goal is to determine the gas pressure \( p_{g,1} \) and mass \( m_g \) in state 1.  

### Given Data:  
- \( T_{g,1} = 500^\circ\text{C} \)  
- \( V_{g,1} = 3.14 \, \text{L} \)  

### Formula Used:  
The ideal gas law:  
\[
pV = mRT
\]  

### Calculations:  
The gas constant \( R \) is calculated as:  
\[
R = \frac{8.314 \, \text{kJ}}{\text{kmol·K}} \div 50 \, \text{kg/kmol} = 0.1663 \, \text{kJ/kg·K}
\]  

Substituting into the ideal gas law:  
\[
p_{g,1} \cdot 3.14 \times 10^{-3} \, \text{m}^3 = m_g \cdot 0.1663 \, \text{kJ/kg·K} \cdot 773.15 \, \text{K}
\]  

### Determining \( p_{\text{EW}} \):  
The pressure exerted by the piston and atmospheric pressure is calculated as:  
\[
p_{\text{EW}} = p_{\text{amb}} + \frac{m_K \cdot g}{A}
\]  
Substituting values:  
\[
p_{\text{EW}} = 1 \, \text{bar} + \frac{32 \, \text{kg} \cdot 9.81 \, \text{m/s}^2}{\frac{\pi}{4} \cdot (0.1 \, \text{m})^2}
\]  
\[
p_{\text{EW}} = 1 \, \text{bar} + 0.39969 \, \text{bar} = 1.3997 \, \text{bar}
\]  

### Determining \( p_{g,1} \):  
\[
p_{g,1} = p_{\text{EW}} + \frac{m_{\text{EW}} \cdot g}{A}
\]  
Substituting values:  
\[
p_{g,1} = 1.3997 \, \text{bar} + \frac{0.1 \, \text{kg} \cdot 9.81 \, \text{m/s}^2}{\frac{\pi}{4} \cdot (0.1 \, \text{m})^2}
\]  
\[
p_{g,1} = 1.401 \, \text{bar}
\]  

### Determining \( m_g \):  
Rearranging the ideal gas law:  
\[
m_g = \frac{p_{g,1} \cdot V_{g,1}}{R \cdot T_{g,1}}
\]  
Substituting values:  
\[
m_g = \frac{1.401 \times 10^5 \, \text{N/m}^2 \cdot 3.14 \times 10^{-3} \, \text{m}^3}{0.1663 \, \text{kJ/kg·K} \cdot 773.15 \, \text{K}}
\]  
\[
m_g = 3.4215 \, \text{g}
\]  

### Final Results:  
- \( p_{g,1} = 1.401 \, \text{bar} \)  
- \( m_g = 3.4215 \, \text{g} \)