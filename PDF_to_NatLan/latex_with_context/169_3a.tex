The gas pressure \( p_{g,1} \) is calculated using the formula:  
\[
p_{g,1} = p_{\text{amb}} + \frac{(m_K + m_{\text{EW}}) \cdot g}{A}
\]  
Substituting values:  
\[
p_{\text{amb}} = 1.10^5 \, \text{Pa}, \quad A = \pi \cdot r^2 = \pi \cdot (0.052)^2 \approx 0.0079 \, \text{m}^2
\]  
\[
p_{g,1} = 1.10^5 \, \text{Pa} + \frac{(32 \, \text{kg} + 0.1 \, \text{kg}) \cdot 9.81 \, \text{m/s}^2}{0.0079 \, \text{m}^2}
\]  
\[
p_{g,1} \approx 140094 \, \text{Pa} \approx 1.4 \, \text{bar}
\]  

The gas mass \( m_{g,1} \) is calculated using the ideal gas law:  
\[
pV = mRT \quad \Rightarrow \quad m_{g,1} = \frac{RT_{g,1}}{p_{g,1} V_{g,1}}
\]  
The gas constant \( R \) is derived as:  
\[
R = \frac{R_u}{M_g} = \frac{8.314 \, \text{kJ/kmol·K}}{50 \, \text{kg/kmol}} \approx 0.166 \, \text{kJ/kg·K}
\]  
Substituting values:  
\[
m_{g,1} = \frac{1.4 \cdot 10^5 \, \text{Pa} \cdot 3.14 \cdot 10^{-3} \, \text{m}^3}{0.166 \, \text{kJ/kg·K} \cdot (500 + 273.15) \, \text{K}}
\]  
\[
m_{g,1} \approx 2.92 \, \text{g}
\]