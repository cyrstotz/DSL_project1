The task involves determining the final ice fraction \( x_{\text{ice},2} \) in state 2 using the solid-liquid equilibrium table. The following calculations and equations are presented:

1. The energy balance equation is written as:  
\[
x_{\text{ice},2} \cdot m_{\text{EW},2} - m_{\text{EW},1} = Q_{12} + \Phi
\]  
where \( Q_{12} \) represents the heat transferred and \( \Phi \) is an additional term (possibly entropy or work-related).

2. The mass of the ice-water mixture \( m_{\text{EW},2} \) is calculated as:  
\[
m_{\text{EW},2} = m_{\text{EW}} \cdot \left( 1 - x_{\text{ice},2} \right) - x_{\text{EW}}
\]  
where \( m_{\text{EW}} = 0.1 \, \text{kg} \).

3. The internal energy \( u_2 \) is expressed as:  
\[
u_2 = u_{\text{flüssig}} + x \cdot \left( u_{\text{fest}} - u_{\text{flüssig}} \right) \quad \text{at } T = 0^\circ\text{C}.
\]  
This equation accounts for the energy contributions from liquid and solid phases.

4. Substituting values:  
\[
u_2 = u_{\text{flüssig}} + 0.6 \cdot \left( -333.458 + 0.045 \right) \quad \text{at } T = 0^\circ\text{C}.
\]  
The result is:  
\[
u_2 = -20.01.
\]

5. Further calculations for \( m_{\text{EW},2} \):  
\[
m_{\text{EW},2} = 0.1 \, \text{kg} \cdot \left( 1 - x_{\text{ice},2} \right) - x_{\text{EW}} = 0.1 \cdot \left( 1 - x_{\text{ice},2} \right).
\]

6. The internal energy \( u_2 \) is recalculated as:  
\[
u_2 = -0.045 + x_{\text{ice},2} \cdot \left( -333.4 \right).
\]

7. The final equation is:  
\[
m_{\text{EW},2} = Q_{12} + m_{\text{EW},2}.
\]  
This equation is used to solve for \( x_{\text{ice},2} \).

No diagrams or graphs are present on the page.