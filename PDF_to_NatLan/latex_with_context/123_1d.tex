The problem involves cooling the reactor contents from \( T_1 = 100^\circ\text{C} \) to \( T_2 = 70^\circ\text{C} \). A mass \( \Delta m_{12} \) of saturated liquid water at \( T_{\text{in,12}} = 20^\circ\text{C} \) is added, and the heat released during cooling is \( Q_{12} = 35 \, \text{MJ} \).  

The energy balance equation is written as:  
\[
W = 0 \quad \text{(no work is done)}
\]  
\[
Q_{\text{in}} - Q_{\text{out}} = Q_{12}
\]  
\[
m_1 c_V T_1 - m_2 c_V T_2 = Q_{12}
\]  

Using the enthalpy values from Table A2:  
\[
u_1 (x = 0.005) = 428.38 \, \frac{\text{kJ}}{\text{kg}}
\]  
\[
u_2 (x = 1 / 2001) = 2402 \, \frac{\text{kJ}}{\text{kg}}
\]  

Substituting into the energy balance:  
\[
m_1 u_1 - m_2 u_2 = Q_{12}
\]  
\[
m_2 = \frac{m_1 u_1 - Q_{12}}{u_2}
\]  
\[
m_2 = \frac{5755 \cdot 428.38 - 35,000}{2402} = 10.14 \, \text{kg}
\]  

Thus, the added mass is:  
\[
m_{12} = 10.14 \, \text{kg}
\]  

---