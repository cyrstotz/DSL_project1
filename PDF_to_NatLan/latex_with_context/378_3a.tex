The pressure \( p_g \) is calculated using the equation:  
\[
p_g = \frac{m_g \cdot R_g \cdot T_g}{V_g}
\]  
where \( R_g = \frac{R}{M_g} = 166.28 \, \text{J/kg·K} \).  

The mass of the gas \( m_g \) is determined as:  
\[
m_g = \frac{p_g \cdot V_g}{R_g \cdot T_g}
\]  

The area \( A_2 \) is calculated as:  
\[
A_2 = \pi \cdot r^2 = \pi \cdot \frac{D^2}{4} = \frac{\pi \cdot D^2}{400}
\]  

The pressure \( p \) is given by:  
\[
p = \frac{m_E \cdot g}{A_2} + \frac{32 \, \text{kg} \cdot g}{A_2} + p_{\text{amb}} = 140 \, \text{kPa} \quad \text{or} \quad 1.4 \, \text{bar}
\]  

The mass \( m_g \) is calculated as:  
\[
m_g = 3.4217 \, \text{g}
\]  

---