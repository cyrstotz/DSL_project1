The first law of thermodynamics for the steady-state adiabatic process is written as:  
\[
0 = \dot{m} \left[ h_5 - h_6 + \frac{w_5^2}{2} - \frac{w_6^2}{2} \right] + Q - W_e
\]  
For an adiabatic nozzle, \(Q = 0\) and \(W_e = 0\), simplifying to:  
\[
0 = \dot{m} \left[ h_5 - h_6 + \frac{w_5^2}{2} - \frac{w_6^2}{2} \right]
\]  

The entropy balance is expressed as:  
\[
0 = \dot{m} \left[ s_5 - s_6 \right]
\]  

The entropy at state 5 (\(s_5\)) is equal to the entropy at state 6 (\(s_6\)):  
\[
s_5 = s_6
\]  

To determine \(s_5\), interpolation is performed using Table A-22 at a temperature of \(T_5 = 431.9 \, \text{K}\):  
\[
h_5 = \text{interpolation in Table A-22: } 440 - 430 = 433.86 \, \text{kJ/kg}
\]  

Additional calculations for \(s_5\) are crossed out and marked as incorrect.

The work done by the turbine is expressed as:  
\[
w_t = -\int_5^6 p \, dV
\]  
where \( p \) is the pressure and \( V \) is the specific volume.  

The relationship between pressure and volume is given by:  
\[
p = p_5 \left(\frac{V_5}{V}\right)^\kappa
\]  
where \( \kappa = \frac{c_p}{c_v} \).  

The integral for work is then:  
\[
w_t = -\int_5^6 p_5 \left(\frac{V_5}{V}\right)^\kappa \, dV
\]  

Further simplifications or numerical integration may be required to compute the exact value.  

No diagrams or additional figures are present on this page.