The outlet velocity \( w_6 \) and temperature \( T_6 \) are calculated.  

Given:  
\[
w_5 = 220 \, \text{m/s}, \quad p_5 = 0.5 \, \text{bar}, \quad T_5 = 431.9 \, \text{K}, \quad p_6 = 0.191 \, \text{bar}
\]  

The equation for \( T_6 \) is derived using the isentropic relation:  
\[
\frac{T_6}{T_5} = \left( \frac{p_6}{p_5} \right)^{\frac{\kappa - 1}{\kappa}}
\]  

Substituting values:  
\[
T_6 = T_5 \left( \frac{p_6}{p_5} \right)^{\frac{\kappa - 1}{\kappa}}
\]  
\[
T_6 = 431.9 \left( \frac{0.191}{0.5} \right)^{\frac{0.4}{1.4}}
\]  
\[
T_6 = 328.075 \, \text{K}
\]  

This calculation is boxed in the handwritten solution.  

Additionally, the entropy at state 6 is noted as \( S_6 = S_5 \), justified by the isentropic process.  

The general energy equation for the outlet velocity \( w_6 \) is written as:  
\[
Q = \dot{m}_{\text{gas}} \left( h_5 - h_6 + \frac{w_5^2}{2} - \frac{w_6^2}{2} \right) - \dot{W}_{\text{rev}}
\]  

No further calculation for \( w_6 \) is shown explicitly.

The outlet velocity \( w_6 \) is calculated using the following equation:  
\[
w_6 = \sqrt{2 \, c_p \, (T_5 - T_6) + \frac{w_5^2}{2}}
\]  
Substituting the values:  
\[
w_6 = \sqrt{2 \, (c_p) \, (T_5 - T_6) + \frac{w_5^2}{2}} = 507.2438 \, \frac{\text{m}}{\text{s}}
\]  

---