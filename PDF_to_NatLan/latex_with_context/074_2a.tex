A table is presented with columns labeled \( p \) (pressure) and \( T \) (temperature). The rows correspond to different states in the jet engine process:  

- **Ambient conditions (0):**  
  \( p = 0.191 \, \text{bar} \), \( T = 30^\circ\text{C} \)  

- **State 1:**  
  \( p = p_2 \), \( T \) not specified  

- **State 2:**  
  \( p = p_2 \), \( T \) not specified  

- **State 3:**  
  \( p = p_2 \), \( T \) not specified  

- **State 4:**  
  \( p = 0.5 \, \text{bar} \), \( T \) not specified  

- **State 5:**  
  \( p = 0.5 \, \text{bar} \), \( T = 431.9 \, \text{K} \)  

- **State 6:**  
  \( p = 0.191 \, \text{bar} \), \( T \) not specified  

---

Below the table, a graph is drawn representing a temperature-entropy (\( T \)-\( s \)) diagram for the jet engine process.  

- The vertical axis is labeled \( T \, (\text{K}) \), and the horizontal axis is labeled \( s \, \left( \frac{1}{\text{kg·K}} \right) \).  
- The process is depicted with six states connected by lines:  
  - **State 0 to 1:** Isobaric process.  
  - **State 1 to 2:** Isentropic compression.  
  - **State 2 to 3:** Isobaric heat addition.  
  - **State 3 to 4:** Isentropic expansion.  
  - **State 4 to 5:** Isobaric mixing.  
  - **State 5 to 6:** Isobaric process.  

Each segment is labeled appropriately with "isobar" or "isentropic" to indicate the type of thermodynamic process.