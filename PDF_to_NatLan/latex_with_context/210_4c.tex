The vapor quality \( x \) after the throttle is being determined.  

A schematic diagram of a throttle is drawn, showing an inlet at \( p_1 = 8 \, \text{bar} \) and an outlet at \( p_2 = 0 \, \text{bar} \).  

The enthalpy after the throttle remains constant:  
\[
h_4 = h_1
\]  

The vapor quality \( x \) is calculated using the formula:  
\[
\phi = \phi_f + x (\phi_g - \phi_f)
\]  

For an isenthalpic process:  
\[
x = \frac{h_4 - h_f}{h_g - h_f}
\]  

Values are substituted:  
\[
h_4 = h_1 = 264.15 \, \text{kJ/kg}
\]  
\[
h_g(p_2) = 732.02 \, \text{kJ/kg}, \, h_f(p_2) = 253.13 \, \text{kJ/kg}
\]  

The vapor quality is calculated as:  
\[
x = \frac{264.15 - 253.13}{732.02 - 253.13}
\]  

The enthalpy at \( 8 \, \text{bar} \) is referenced from a table:  
\[
h_g \, \text{from table at} \, 8 \, \text{bar} = 264.15 \, \text{kJ/kg}
\]