The pressure exerted by the piston is calculated as:  
\[
p_{\text{K}} = \frac{F_{\text{K}}}{A}
\]  
The cross-sectional area of the cylinder is determined using:  
\[
A = \left(\frac{D}{2}\right)^2 \cdot \pi = 7.85 \cdot 10^{-3} \, \text{m}^2
\]  
The force exerted by the piston is:  
\[
F_{\text{K}} = 32 \, \text{kg} \cdot 9.81 \, \text{m/s}^2
\]  
Substituting values, the piston pressure is:  
\[
p_{\text{K}} = 0.4 \, \text{bar}
\]  

The total pressure in the gas chamber is:  
\[
p_{\text{tot}} = p_{\text{amb}} + p_{\text{K}} = 1 \, \text{bar} + 0.4 \, \text{bar} = 1.4 \, \text{bar} = p_{\text{g,1}}
\]  

The mass of the gas is calculated using the ideal gas law:  
\[
m_{\text{g,1}} = \frac{p_{\text{g,1}} \cdot V_{\text{g,1}}}{R \cdot T_{\text{g,1}}}
\]  
Substituting values:  
\[
m_{\text{g,1}} = \frac{1.4 \cdot 10^5 \, \text{Pa} \cdot 3.14 \cdot 10^{-3} \, \text{m}^3}{8.314 \, \frac{\text{J}}{\text{mol·K}} \cdot 773.15 \, \text{K}}
\]  
\[
m_{\text{g,1}} = 0.16629 \, \text{mol} \cdot \frac{50 \, \text{kg}}{\text{kmol}} = 3.414 \, \text{g}
\]