The page contains diagrams and annotations related to thermodynamic processes in a jet engine, specifically addressing the entropy-temperature (\(T\)-\(S\)) diagram for the engine cycle. Below is the transcription and description:

---

### Diagram Description:
Three \(T\)-\(S\) diagrams are drawn, each illustrating the thermodynamic processes in the jet engine cycle. The diagrams include labeled states (0, 1, 2, 3, 4, 5, 6) and annotations describing the nature of the processes between these states. The processes are as follows:

1. **First Diagram**:
   - The curve begins at state 0 and proceeds through states 1, 2, 3, 4, 5, and 6.
   - The process between states is labeled:
     - \(0 \rightarrow 1\): Compression (\(p_0 \rightarrow p_1\)).
     - \(1 \rightarrow 2\): Isentropic compression (\(p_1 \rightarrow p_2\)).
     - \(2 \rightarrow 3\): Isobaric heat addition (\(p_2 = p_3\)).
     - \(3 \rightarrow 4\): Isentropic expansion (\(p_4 \rightarrow T\)).
     - \(4 \rightarrow 5\): Isobaric mixing (\(p_4 \rightarrow p_5\)).
     - \(5 \rightarrow 6\): Isentropic expansion (\(p_6 = p_0\)).

2. **Second Diagram**:
   - Similar to the first diagram but includes additional annotations and refined curves.
   - The processes are labeled as isentropic or isobaric, with arrows indicating the direction of entropy change.

3. **Third Diagram**:
   - A detailed \(T\)-\(S\) diagram with annotations:
     - \(0 \rightarrow 1\): Isentropic compression (\(p_0, p_1\)).
     - \(1 \rightarrow 2\): Isentropic compression (\(p_1, p_2\)).
     - \(2 \rightarrow 3\): Isobaric heat addition (\(p_2, p_3\)).
     - \(3 \rightarrow 4\): Isentropic expansion (\(p_4\)).
     - \(4 \rightarrow 5\): Isobaric mixing (\(p_4, p_5\)).
     - \(5 \rightarrow 6\): Isentropic expansion (\(p_6, p_0\)).

---

### Annotations:
- The initial conditions are noted:
  - \(T_0 = -30^\circ\text{C}\), \(p_0 = 0.191 \, \text{bar}\).
- Each process is described with corresponding pressure and temperature changes:
  - Compression, isentropic compression, isobaric heat addition, isentropic expansion, and mixing.

---

### Mathematical Expressions:
- The processes are described using pressure and temperature relationships:
  - \(p_0 \rightarrow p_1\), \(p_1 \rightarrow p_2\), \(p_2 = p_3\), \(p_4 \rightarrow p_5\), \(p_6 = p_0\).

No additional textual explanations or numerical calculations are present on the page. The diagrams visually represent the thermodynamic cycle of the jet engine as described in the problem setup.

To calculate the mass-specific increase in flow exergy \( \Delta ex_{\text{flow}} \):  

The general expression for flow exergy is:  
\[
ex_{\text{flow},6} = h_6 - h_0 - T_0 \cdot (s_6 - s_0)
\]  

Using the simplified relation:  
\[
\Delta ex_{\text{flow}} = c_p \cdot (T_6 - T_0) - T_0 \cdot c_p \cdot \ln \left( \frac{T_6}{T_0} \right)
\]  

Substituting values:  
\[
\Delta ex_{\text{flow}} = 1.006 \, \frac{\text{kJ}}{\text{kg·K}} \cdot (328.075 - 243.15) - 243.15 \cdot 1.006 \cdot \ln \left( \frac{328.075}{243.15} \right)
\]  

Simplifying:  
\[
\Delta ex_{\text{flow}} = 1.006 \cdot 84.925 - 243.15 \cdot 1.006 \cdot \ln \left( 1.348 \right)
\]  

Further calculations yield:  
\[
\Delta ex_{\text{flow}} = 85.45 - 42.45 = 42.45 \, \frac{\text{kJ}}{\text{kg}}
\]  

Thus, the mass-specific increase in flow exergy is \( 42.45 \, \frac{\text{kJ}}{\text{kg}} \).  

---