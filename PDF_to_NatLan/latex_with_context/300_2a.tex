A graph is drawn with temperature \( T \) (in Kelvin) on the vertical axis and entropy \( S \) (in \( \text{kJ}/\text{kg·K} \)) on the horizontal axis. The graph represents a thermodynamic process with labeled states: 0, 1, 2, 3, 4, 5, and 6.  
- State 0: \( T_0 = -30^\circ\text{C} \).  
- State 1: Adiabatic process. \( S_1 > S_0 \), \( T_1 > T_0 \).  
- State 2: Adiabatic and reversible process. \( S_2 = S_1 \), \( T_2 > T_1 \).  
- State 3: Isobaric process. \( T_3 > T_2 \), \( S_3 >> S_2 \).  
- State 4: \( T_4 < T_3 \), \( S_4 > S_3 \), irreversible process.  
- State 5: \( S_5 < S_4 \), \( T_5 \neq T_4 \).  
- State 6: \( S_6 = S_5 \), \( T_6 < T_5 \).

The mass flow rate is constant:  
\[
\dot{m}_{\text{in}} = \rho_5 A_5 w_5 = \rho_6 A_6 w_6 \quad \rightarrow \quad A_5 = A_6 \quad \text{(assumption, no further idea provided)}
\]  

The temperature at state 5 is:  
\[
T_5 = 431.9 \, \text{K}
\]  

The velocity at state 5 is:  
\[
V_5 = 2.42 \, \frac{\text{m}^3}{\text{kg}}
\]  

The temperature at state 6 is:  
\[
T_6 = 322.07 \, \text{K}
\]  

The velocity at state 6 is:  
\[
V_6 = 4.93717 \, \frac{\text{m}^3}{\text{kg}}
\]  

The outlet velocity \( w_6 \) is calculated using:  
\[
w_6 = \frac{p_5}{\rho_6} w_5
\]  
or alternatively:  
\[
w_6 = \frac{V_6}{V_5} w_5
\]  

The final outlet velocity is:  
\[
w_6 = 437.49 \, \text{m/s}
\]