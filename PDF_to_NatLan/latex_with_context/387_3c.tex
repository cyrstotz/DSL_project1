The maximum heat transfer \( Q_{\text{max}} \) is calculated using the enthalpy difference:  
\[
Q_{\text{max}} = m_{\text{EW}} \cdot (u_f(145) - u_i(145)) = -20 \, \text{kJ}
\]

The heat transfer \( Q_{12} \) is calculated using the energy balance equation:  
\[
\Delta U = Q - W
\]  
The work \( W_{12} \) is expressed as:  
\[
W_{12} = \frac{R(T_2 - T_1)}{1 - n}
\]  
where \( n = \frac{c_p}{c_v} \), and \( c_p \) is calculated as:  
\[
c_p = R + c_v = 799 \, \frac{\text{J}}{\text{kg·K}}
\]  
Thus, \( n = \frac{c_p}{c_v} = 1.2666 \).  

The work \( W_{12} \) is determined to be:  
\[
W_{12} = -1082.4 \, \text{J}
\]  

The internal energy change \( \Delta U \) is calculated as:  
\[
\Delta U = c_v (T_2 - T_1) m = 866.33 \, \text{J}
\]  

Finally, the heat transfer \( Q_{12} \) is:  
\[
Q_{12} = 1,945.4 \, \text{J}
\]  

---