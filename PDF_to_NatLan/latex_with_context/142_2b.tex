The problem asks to determine the outlet velocity \( w_6 \).  

Given data:  
\[
w_{\text{air}} = 200 \, \text{m/s}, \quad p_5 = 0.5 \, \text{bar}, \quad T_5 = 431.9 \, \text{K}, \quad p_0 = 0.191 \, \text{bar}, \quad T_0 = -30^\circ\text{C}, \quad \kappa = 1.4
\]  

The following equations are used:  

1. The temperature at state 6 is calculated using the isentropic relation:  
\[
\frac{T_6}{T_5} = \left( \frac{p_6}{p_5} \right)^{\frac{\kappa - 1}{\kappa}}
\]  
Substituting values:  
\[
T_6 = T_5 \cdot \left( \frac{p_6}{p_5} \right)^{\frac{\kappa - 1}{\kappa}} = 431.9 \cdot \left( \frac{0.191}{0.5} \right)^{\frac{0.4}{1.4}} = 328.07 \, \text{K}
\]  

2. The outlet velocity \( w_6 \) is determined using the energy balance for an adiabatic, reversible process:  
\[
w_6^2 = w_5^2 + 2 \cdot c_p \cdot (T_5 - T_6)
\]  
Substituting values:  
\[
w_6 = \sqrt{w_5^2 + 2 \cdot c_p \cdot (T_5 - T_6)}
\]  
Where \( w_5 = 220 \, \text{m/s} \), \( c_p = 1.006 \, \text{kJ/kg·K} \):  
\[
w_6 = \sqrt{220^2 + 2 \cdot 1.006 \cdot (431.9 - 328.07)} = 507.25 \, \text{m/s}
\]