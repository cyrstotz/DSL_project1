Energy balance for the heat exchanger in the refrigeration cycle is described.  

A diagram is drawn showing a rectangular heat exchanger with arrows indicating heat flow \( \dot{Q}_K \) and the refrigerant R134a flowing through it. The heat exchanger is outlined in orange, and the heat flow is labeled with a blue arrow pointing downward.  

The energy balance equation is written as:  
\[
SFP: 0 = \dot{m}_R [h_1 - h_2] + \dot{Q}_K
\]  

Further explanation of energy transfer in the heat exchanger is provided.  

Another diagram is drawn, similar to the first, showing heat flow \( \dot{Q}_K \) and labeled arrows.  

The equation for heat transfer is given as:  
\[
Q = m(s_2 - s_A) = \frac{\dot{Q}_K}{T_i} - \frac{\dot{Q}_K}{T_{ii}}
\]  

The heat flow \( \dot{Q}_K \) is expressed as:  
\[
\dot{Q}_K = m(s_2 - s_A) \cdot (-T_{ii} + T_{i})
\]  

A crossed-out equation is visible but ignored per instructions.  

The refrigerant mass flow rate \( \dot{m}_R \) is calculated as:  
\[
\dot{m}_R = \frac{-\dot{Q}_K}{h_1 - h_2}
\]  

No additional content or figures are present.

The page contains two diagrams related to the freeze-drying process in a \( p \)-\( T \) diagram.

---

**First Diagram Description:**  
The first diagram is a \( p \)-\( T \) plot with the pressure \( p \) on the vertical axis (labeled in bar) and temperature \( T \) on the horizontal axis (labeled in Kelvin). The diagram shows a phase curve separating different regions:  
- On the left side of the curve, the region is labeled "unterkühlte Flüssigkeit" (subcooled liquid).  
- On the right side of the curve, the region is labeled "überhitzter Dampf" (superheated vapor).  
- The area under the curve is labeled "Nassdampf" (wet vapor).  

Two points are marked on the diagram:  
- Point 1 is located on the right side of the curve, in the superheated vapor region.  
- Point 2 is located on the left side of the curve, in the subcooled liquid region.  

The curve represents the phase boundary between liquid and vapor phases.

---

**Second Diagram Description:**  
The second diagram is also a \( p \)-\( T \) plot with pressure \( p \) on the vertical axis (labeled in bar) and temperature \( T \) on the horizontal axis (labeled in Celsius). The diagram features a similar phase curve, with two points marked:  
- Point 1 is located near the peak of the curve.  
- Point 2 is slightly to the left of Point 1, also near the peak.  

The horizontal axis includes labels \( T_i \) and \( T_i - 6 \, \text{K} \), indicating temperature values relevant to the freeze-drying process.  

This diagram visually represents the freeze-drying process steps, including the isobaric evaporation and temperature differences.