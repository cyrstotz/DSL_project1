The mass flow rate at the inlet is given as:  
\[
\dot{m}_{\text{in}} = \dot{m}_{\text{out}} = 0.3 \, \text{kg/s}
\]

To calculate the heat flow removed by the coolant (\( \dot{Q}_{\text{out}} \)), the steady-state energy balance for the reactor is expressed as:  
\[
0 = \dot{m} \left( h_{\text{e}} - h_{\text{a}} \right) + \dot{Q}_R - \dot{Q}_{\text{out}}
\]
where:  
- \( \dot{Q}_R = 100,000 \, \text{W} \)  
- \( T_R = 100^\circ\text{C} = 373.15 \, \text{K} \)  
- \( m_{\text{ges}} = 5755 \, \text{kg} \)  
- \( x_D = 0.005 \)

The page contains a sketch of a phase diagram with pressure (\( p \)) on the vertical axis and temperature (\( T \)) on the horizontal axis. The diagram includes the following labeled regions:  

- **Flüssig** (Liquid): This region is to the left of the curve.  
- **Gasförmig** (Gaseous): This region is to the right of the curve.  
- **Nass-Dampf** (Wet Steam): This region is under the curve.  

The curve represents the phase boundary between liquid, wet steam, and gaseous states.  

Additional annotations:  
- The horizontal axis is labeled \( T \) for temperature.  
- The vertical axis is labeled \( p \) for pressure.  
- Some text near the diagram is crossed out and ignored.  

No further mathematical expressions or textual explanations are visible.