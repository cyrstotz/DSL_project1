The page contains two diagrams related to freeze-drying and refrigeration processes.  

**Diagram 1 (Top):**  
This is a phase diagram with temperature \( T \) on the horizontal axis and an unspecified variable (likely pressure or enthalpy) on the vertical axis. The diagram shows the phase transitions of ice and water.  
- A curve labeled "Eis" (ice) transitions into "Wasser" (water) as temperature increases.  
- A horizontal line labeled "Tripel" (triple point) indicates the equilibrium between ice, water, and vapor phases.  
- The curve appears to include a sublimation region, with arrows indicating the direction of phase change.  

**Diagram 2 (Bottom):**  
This is another phase diagram with temperature \( T \) on the horizontal axis and pressure \( p \) on the vertical axis.  
- The diagram includes a curve labeled "Tripel" (triple point) and an arrow pointing to a region labeled "1000 mbar."  
- Another arrow points to "200 mbar," indicating a pressure reduction.  
- The curves suggest transitions between different phase regions, likely related to the freeze-drying process.  

Both diagrams visually represent the thermodynamic behavior of the system during freeze-drying and refrigeration cycles.