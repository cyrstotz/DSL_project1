The given problem involves determining the gas pressure \( p_{g,1} \) and mass \( m_g \) in state 1 using the ideal gas law.  

The universal gas constant \( R \) is calculated as:  
\[
R = \frac{8.314 \, \frac{\text{J}}{\text{mol·K}}}{50 \, \frac{\text{kg}}{\text{kmol}}} = 166.28 \, \frac{\text{J}}{\text{kg·K}}
\]  

The specific heat capacity at constant pressure \( c_p \) is calculated using the relation \( c_p = R + c_v \):  
\[
c_p = 166.28 \, \frac{\text{J}}{\text{kg·K}} + 0.633 \, \frac{\text{kJ}}{\text{kg·K}} = 0.79928 \, \frac{\text{kJ}}{\text{kg·K}}
\]  

The mass \( m_g \) is determined using the ideal gas law:  
\[
m = \frac{p V}{R T}
\]  
Substituting the values:  
\[
m = \frac{1.5 \, \text{bar} \cdot 3.14 \, \text{L}}{166.28 \, \frac{\text{J}}{\text{kg·K}} \cdot 723.15 \, \text{K}}
\]  
Converting units and solving:  
\[
m = 3.664 \, \text{kg}
\]