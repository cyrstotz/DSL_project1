The ideal gas equation is used to determine the gas pressure \( p_{g,1} \) and mass \( m_g \) in state 1.  

The temperature of the gas is given as \( T_{g,1} = 500^\circ\text{C} \), and the volume is \( V_{g,1} \).  

The gas constant \( R_g \) is calculated using:  
\[
R_g = \frac{R}{M_g} = \frac{8.314 \, \text{J/(mol·K)}}{50 \, \text{kg/kmol}} = 166.28 \, \text{J/(kg·K)}
\]  

The relationship between \( c_p \) and \( c_v \) is given as \( c_p - c_v = R \). Substituting values:  
\[
c_p = 166.28 \, \text{J/(kg·K)} + 0.633 \, \text{kJ/(kg·K)} = 825.28 \, \text{J/(kg·K)}
\]  

The ideal gas law is expressed as:  
\[
pV = mRT
\]  

The pressure of the gas \( p_{g,1} \) is calculated using the equation:  
\[
p_{g,1} + p_{\text{EW}} = p_{\text{amb}} + \frac{m_g g}{A_{\text{piston}}}
\]  

The pressure of the EW is determined as \( p_{\text{EW}} = 1.4 \, \text{bar} \) (from Table X).  

---