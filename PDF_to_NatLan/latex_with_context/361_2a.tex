A graph is drawn representing the process in a \( T \)-\( s \) diagram. The graph includes labeled states \( 0 \), \( 2 \), \( 5 \), and \( 6 \), with arrows indicating transitions between these states. The process includes isobars and curves, with dashed lines showing the transitions.  

Below the graph, the following values are provided:  
\[
T_0 = -30^\circ\text{C}, \quad p_0 = 0.191 \, \text{bar}, \quad p_1 > p_0, \quad p_2 > p_1
\]

---

The heat transfer equation is written as:  
\[
Q_{\text{EW}} = Q = m \cdot (u_2 - u_1)
\]  
where \( u_1 \) is calculated using:  
\[
u_1 = u_{\text{ref}} + x \cdot (u_{\text{ice}} - u_{\text{ref}})
\]  
and \( x = 0.6 \), \( u_{\text{ref}} = -333.91 \, \text{kJ/kg} \), \( u_{\text{ice}} = -178.44 \, \text{kJ/kg} \).  

The specific internal energy \( u_2 \) is determined using:  
\[
u_2 = \frac{Q}{m} + u_1 = \frac{7.5}{0.1} + (-333.91) = -258.41 \, \text{kJ/kg}
\]  

The vapor quality \( x \) is calculated as:  
\[
x = \frac{u_2 - u_{\text{ref}}}{u_{\text{ice}} - u_{\text{ref}}} = \frac{-258.41 + 333.91}{-178.44 + 333.91} = 0.645
\]  

The temperature is noted as \( 0.003^\circ\text{C} \).  

No diagrams or graphs are present on the page.