The given parameters for the jet engine are:  
- Air velocity \( w_{\text{air}} = 200 \, \text{m/s} \)  
- Ambient pressure \( p_0 = 0.191 \, \text{bar} \)  
- Ambient temperature \( T_0 = -30^\circ\text{C} \)  
- Heat added per unit mass \( q_B = 1195 \, \text{kJ/kg} \)  
- Combustion temperature \( T_B = 1289 \, \text{K} \)  

A qualitative \( T \)-\( s \) diagram is drawn, showing the thermodynamic process of the jet engine. The diagram includes labeled isobars at \( 0.5 \, \text{bar} \) and \( 0.191 \, \text{bar} \). The states are marked as follows:  
- State 1: Initial condition  
- State 2: Compression  
- State 3: Combustion  
- State 4: Expansion in the turbine  
- State 5: Mixing chamber  
- State 6: Nozzle exit  

The axes are labeled:  
- \( T \) (temperature in Kelvin) on the vertical axis  
- \( s \) (specific entropy in \( \text{kJ}/\text{kg·K} \)) on the horizontal axis