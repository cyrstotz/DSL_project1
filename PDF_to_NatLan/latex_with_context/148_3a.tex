The pressure \( p_1 \) is calculated using the formula:  
\[
p_1 = p_0 + \frac{32 \, \text{kg} \cdot 9.81}{(0.1)^2 \cdot \pi} + \frac{0.1 \cdot 9.81}{(0.1)^2 \cdot \pi}
\]  
Substituting values, the result is:  
\[
p_1 = 110023.6 \, \text{Pa} = 1.1 \, \text{bar}
\]  

The ideal gas law is applied:  
\[
pV = mRT
\]  
Where \( R = 0.16628 \).  

Substituting values:  
\[
p \cdot 0.00314 \, \text{m}^3 = m \cdot 0.16628 \cdot 128.56
\]  
Rearranging to solve for \( m \):  
\[
m = \frac{p \cdot 0.00314}{128.56}
\]  
The calculated mass is:  
\[
m = 0.0000268 \, \text{kg} = 0.0269 \, \text{g}
\]