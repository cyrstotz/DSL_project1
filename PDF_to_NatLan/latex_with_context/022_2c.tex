The mass-specific increase in flow exergy is calculated using the following equation:  
\[
\Delta ex_{\text{flow}} = h_6 - h_0 - T_0 \cdot (s_6 - s_0) + \frac{w_6^2 - w_0^2}{2}
\]  
where:  
- \( h_6 \) and \( h_0 \) are the specific enthalpies at states 6 and 0, respectively.  
- \( T_0 \) is the ambient temperature.  
- \( s_6 \) and \( s_0 \) are the specific entropies at states 6 and 0, respectively.  
- \( w_6 \) and \( w_0 \) are the velocities at states 6 and 0, respectively.  

Given:  
\[
w_6 = 570 \, \text{m/s}, \quad T_0 = 243.15 \, \text{K}, \quad T_6 = 380 \, \text{K}
\]  

The entropy difference is calculated as:  
\[
\Delta s = s_6 - s_0 = s^0(T_6) - s^0(T_0) - R \ln \left(\frac{p_6}{p_0}\right)
\]  
where \( R \) is the specific gas constant.  

Interpolating \( s^0(T_6) \) and \( s^0(T_0) \) from the tables:  
\[
s^0(T_6) = 1.59117 \, \text{kJ/kg·K}, \quad s^0(T_0) = 1.47824 \, \text{kJ/kg·K}
\]  
\[
\Delta s = 1.59117 - 1.47824 = 0.11293 \, \text{kJ/kg·K}
\]