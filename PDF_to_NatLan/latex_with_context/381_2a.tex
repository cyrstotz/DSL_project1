The page contains a table and additional notes related to the jet engine problem. The table is structured with columns labeled \( T \), \( P \), and \( h \), representing temperature, pressure, and enthalpy, respectively. The rows correspond to different states (1 through 6).  

### Table Content:
- **State 1**:  
  \( T = -30^\circ\text{C} \)  
  \( P_2 = P_4 \)  

- **State 2**:  
  \( P_3 = P_4 \)  

- **State 3**:  
  \( P_4 = P_5 \)  

- **State 5**:  
  \( T = 431.9 \, \text{K} \)  
  \( P = 0.5 \, \text{bar} \)  

- **State 6**:  
  \( P = 0.151 \, \text{bar} \)  

### Notes:
- \( \eta_{\text{comp}} < 1 \)  
- \( w_{\text{air}} = 200 \, \frac{\text{m}}{\text{s}} \)  
- \( w_5 = 220 \, \frac{\text{m}}{\text{s}} \)  
- \( c_p = 1.006 \, \text{kJ/kg·K} \)  
- \( \frac{\dot{m}_M}{\dot{m}_K} = 5.293 \)  

### Description:
The table outlines the thermodynamic states of the jet engine process, including temperature and pressure values for specific states. The notes provide additional parameters such as the airspeed at the inlet (\( w_{\text{air}} \)), the velocity at state 5 (\( w_5 \)), and the specific heat capacity (\( c_p \)). The ratio of bypass to core mass flow rates (\( \frac{\dot{m}_M}{\dot{m}_K} \)) is also specified.  

No diagrams or graphs are present on this page.

The diagram represents a qualitative \( T \)-\( s \) (temperature-entropy) plot for the jet engine process. The axes are labeled as follows:  
- The vertical axis is \( T \) (temperature in Kelvin).  
- The horizontal axis is \( s \) (specific entropy in \( \text{kJ}/\text{kg·K} \)).  

The process includes the following states and transitions:  
1. Starting at state 1, an isentropic compression occurs, moving vertically upwards to state 2.  
2. From state 2 to state 3, heat is added isobarically, increasing entropy.  
3. From state 3 to state 4, an isentropic expansion occurs, moving vertically downwards.  
4. From state 4 to state 5, heat is removed isobarically, decreasing entropy.  
5. The final state (state 6) is marked as an isobaric process at \( p_6 = p_0 \).  

The diagram also includes annotations for isentropic and isobaric processes, with specific pressure values labeled (e.g., \( 0.5 \, \text{bar} \), \( 1.5 \, \text{bar} \)).  

---