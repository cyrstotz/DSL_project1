The table lists thermodynamic properties for four states in the refrigeration cycle:  

\[
\begin{array}{|c|c|c|c|c|}
\hline
\text{State} & p & T & s & x_1 \\
\hline
1 & 1.7792 \, \text{bar} & -22^\circ\text{C} & 0.9351 & x_1 = 1 \\
2 & 1.2192 \, \text{bar} & -22^\circ\text{C} & 0.9351 & x_1 = 1 \\
3 & 8 \, \text{bar} & 31.33^\circ\text{C} & 0.9851 & x_3 \\
4 & 8 \, \text{bar} & 31.33^\circ\text{C} & 0.3242 & x = 0 \\
\hline
\end{array}
\]

The page contains two diagrams related to the freeze-drying process described in Task 4.

1. **First Diagram**:  
   - The graph is a pressure-temperature (\( p \)-\( T \)) diagram.  
   - The x-axis represents temperature (\( T \)), and the y-axis represents pressure (\( p \)).  
   - The diagram shows a cycle with four states labeled as 1, 2, 3, and 4.  
   - The pressure at state 3 is marked as "8 bar".  
   - The cycle includes arrows indicating the direction of the process between states:  
     - From state 1 to state 2, the process moves diagonally upward.  
     - From state 2 to state 3, the process moves horizontally at constant pressure.  
     - From state 3 to state 4, the process moves diagonally downward.  
     - From state 4 to state 1, the process moves horizontally at constant pressure.  

2. **Second Diagram**:  
   - The graph is a pressure-volume (\( p \)-\( T \)) diagram.  
   - The x-axis represents temperature (\( T \)) in Kelvin (\( K \)), and the y-axis represents pressure (\( p \)).  
   - The diagram shows a cycle with four states labeled as 1, 2, 3, and 4.  
   - The cycle includes arrows indicating the direction of the process between states:  
     - From state 1 to state 2, the process moves horizontally.  
     - From state 2 to state 3, the process moves upward along a curve.  
     - From state 3 to state 4, the process moves horizontally.  
     - From state 4 to state 1, the process moves downward along a curve.  

These diagrams visually represent the freeze-drying process steps, including isobaric and adiabatic transitions.