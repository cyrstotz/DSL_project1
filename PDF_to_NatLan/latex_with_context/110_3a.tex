The problem begins with determining the gas pressure \( p_{g,1} \) and mass \( m_g \) in state 1.  

The cross-sectional area \( A \) of the cylinder is calculated using the formula for the area of a circle:  
\[
A = \pi r^2 = \pi \left(\frac{d}{2}\right)^2 = \pi \left(\frac{0.1 \, \text{m}}{2}\right)^2 = 0.00785 \, \text{m}^2
\]  

A force balance is applied to the piston, considering atmospheric pressure \( p_0 \), piston weight \( m_K g \), and the weight of the ice-water mixture \( m_{\text{EW}} g \):  
\[
p_0 A + m_K g + m_{\text{EW}} g = p_{g,1} A
\]  
Rearranging for \( p_{g,1} \):  
\[
p_{g,1} = \frac{p_0 A + m_K g + m_{\text{EW}} g}{A}
\]  

Substituting values:  
\[
p_{g,1} = \frac{1.05 \, \text{N/m}^2 \cdot 0.00785 \, \text{m}^2 + 32 \, \text{kg} \cdot 9.81 \, \text{m/s}^2 + 0.1 \, \text{kg} \cdot 9.81 \, \text{m/s}^2}{0.00785 \, \text{m}^2}
\]  
\[
p_{g,1} = 128853.00737 \, \text{Pa}
\]  
Converting to bar:  
\[
p_{g,1} = 1.2885 \, \text{bar}
\]  

Next, the mass \( m_g \) of the gas is determined using the ideal gas law:  
\[
p V = m R T
\]  
Rearranging for \( m_g \):  
\[
m_g = \frac{p V}{R T}
\]  

The gas constant \( R \) is calculated:  
\[
R = \frac{R_u}{M_g} = \frac{8.314 \, \text{J/mol·K}}{50 \, \text{kg/kmol}} = 0.16628 \, \text{J/g·K}
\]  

Substituting values:  
\[
m_g = \frac{1.2885 \, \text{bar} \cdot 3.14 \, \text{L}}{0.16628 \, \text{J/g·K} \cdot (500 + 273.15) \, \text{K}}
\]  
Converting units and simplifying:  
\[
m_g = \frac{128850 \, \text{Pa} \cdot 0.00314 \, \text{m}^3}{0.16628 \, \text{J/g·K} \cdot 773.15 \, \text{K}}
\]  
\[
m_g = 3.1471 \, \text{g}
\]