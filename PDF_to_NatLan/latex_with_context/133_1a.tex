The first law of thermodynamics is applied:  
\[
0 = \dot{m}(h_{\text{in}} - h_{\text{out}}) + \dot{Q}_{\text{out}} + \dot{Q}_R
\]  

From Table A-2:  
- \( h_{\text{in}}(70^\circ\text{C}, \text{saturated liquid}) = 292.984 \, \text{kJ/kg} \)  
- \( h_{\text{out}}(100^\circ\text{C}, \text{saturated liquid}) = 419.044 \, \text{kJ/kg} \)  

The heat flow removed by the coolant is calculated as:  
\[
\dot{Q}_{\text{out}} = \dot{m}(h_{\text{out}} - h_{\text{in}}) + \dot{Q}_R
\]  
Substituting values:  
\[
\dot{Q}_{\text{out}} = 62.182 \, \text{kW}
\]  

---

The temperature \( T_2 \) is calculated as:  
\[
T_2 = -10^\circ\text{C} - 6 \, \text{K} = -16^\circ\text{C}
\]

The entropy \( S_2 \) is given as:  
\[
S_2 = 0.9855 \, \text{kJ/kg·K} = S_3
\]

The entropy \( S_3 \) corresponds to values in Table A-9.  

The interpolation for \( T_3 \) is performed as follows:  
\[
T_3 = 30 + \frac{0.9855 - 0.9449}{0.9874 - 0.9449} (35 - 30) = 30^\circ\text{C}
\]