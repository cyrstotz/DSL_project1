The gas pressure \( p_{g,1} \) is calculated using the equation:  
\[
p_{g,1} = p_0 + \frac{m_K g}{A} + \frac{m_{\text{EW}} g}{A}
\]  
where \( A = \pi \left(\frac{D}{2}\right)^2 = 0.007854 \, \text{m}^2 \). Substituting values:  
\[
p_{g,1} = 10^5 \, \text{Pa} + \frac{32 \, \text{kg} \cdot 9.81 \, \text{m/s}^2}{0.007854 \, \text{m}^2} + \frac{0.1 \, \text{kg} \cdot 9.81 \, \text{m/s}^2}{0.007854 \, \text{m}^2}
\]  
\[
p_{g,1} = 140.1 \, \text{kPa}
\]  

The gas volume is calculated using the ideal gas law:  
\[
p V = R T
\]  
where \( R = \frac{\bar{R}}{M} = \frac{8.314 \, \text{J/mol·K}}{50 \, \text{kg/kmol}} = 0.16628 \, \text{kJ/kg·K} \).  
\[
V_{g,1} = \frac{R T_{g,1}}{p_{g,1}} = \frac{0.16628 \cdot 773.15}{140.1} = 0.9176 \, \text{m}^3/\text{kg}
\]  

The gas mass is then calculated:  
\[
m = \frac{V_{g,1}}{v_{g,1}} = \frac{0.00314 \, \text{m}^3}{0.9176 \, \text{m}^3/\text{kg}} = 0.003422 \, \text{kg}
\]  
\[
m_{g,1} = 3.422 \, \text{g}
\]  

---