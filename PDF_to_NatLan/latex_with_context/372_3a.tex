The ideal gas law is used:  
\[
p V = m R T
\]  
The mass of the gas \( m_g \) is calculated using the pressure \( p_{g,1} \), volume \( V_{g,1} \), and temperature \( T_{g,1} \).  

The radius \( r \) of the cylinder is given as:  
\[
r = 0.05 \, \text{m}
\]  
The cross-sectional area \( A \) is calculated as:  
\[
A = \pi r^2 = 0.00785 \, \text{m}^2
\]  

The pressure \( p_{g,1} \) is determined using the equilibrium of forces:  
\[
p_{g,1} \cdot A = m_{\text{EW}} \cdot g + m_K \cdot g + A \cdot p_{\text{amb}}
\]  
Substituting values:  
\[
p_{g,1} = 1.401 \, \text{bar}
\]  

The mass of the gas \( m_g \) is calculated using:  
\[
p_{g,1} V_{g,1} = m_g R T_{g,1}
\]  
Rearranging:  
\[
m_g = \frac{p_{g,1} V_{g,1}}{R T_{g,1}} = 3.422 \, \text{g}
\]  

The specific gas constant \( R_g \) is calculated as:  
\[
R_g = \frac{R}{M} = 0.16628 \, \frac{\text{kJ}}{\text{kg·K}}
\]  

---