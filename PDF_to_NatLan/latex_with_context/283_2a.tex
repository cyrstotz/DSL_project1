The page contains two diagrams labeled as part of "Aufgabe 2" (Task 2), specifically subtask "a)". Both diagrams are temperature-entropy (\( T \)-\( S \)) plots.

---

**First Diagram:**  
- The axes are labeled as \( T \, [K] \) (temperature in Kelvin) on the vertical axis and \( S \, [\frac{\text{kJ}}{\text{kg·K}}] \) (entropy in kilojoules per kilogram per Kelvin) on the horizontal axis.  
- The diagram shows two distinct processes starting from point "0".  
- The first process is a curved line moving upwards, labeled "1".  
- The second process starts at a higher entropy value and moves upwards, labeled "2".  
- Both processes are marked with arrows indicating the direction of change.  
- The label "isobar" appears near the second process, suggesting it occurs at constant pressure.

---

**Second Diagram:**  
- The axes are similarly labeled as \( T \, [K] \) and \( S \, [\frac{\text{kJ}}{\text{kg·K}}] \).  
- The diagram shows a more complex process with multiple points labeled "1", "3", "4", "5", and "6".  
- The process begins at point "1" and moves upwards along a curved line.  
- Point "3" is at the peak of the curve, indicating the highest temperature.  
- The label "isohoe" (likely intended to mean "isobar") appears near the curve between points "1" and "3".  
- The process continues downward from point "3" to "4", then to "5", and finally to "6".  
- A dashed horizontal line is drawn at point "5", with the label \( T_5 = 431.9 \, \text{K} \), indicating the temperature at this state.

---

No additional text or calculations are visible on the page.