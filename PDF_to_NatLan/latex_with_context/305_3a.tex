The gas pressure \( p_{g,1} \) is calculated as the sum of the ambient pressure \( p_{\text{amb}} \) and the pressure exerted by the piston \( p_{\text{piston}} \):  
\[
p_{g,1} = p_{\text{amb}} + p_{\text{piston}}
\]  
The piston pressure is determined using the formula:  
\[
p_{\text{piston}} = \frac{32 \, \text{kg} \cdot 9.81 \, \text{m/s}^2}{\left(0.05 \, \text{m}\right)^2 \cdot \pi} = 35.965 \, \text{kPa}
\]  
Thus, the total gas pressure is:  
\[
p_{g,1} = 100 \, \text{kPa} + 35.965 \, \text{kPa} = 135.97 \, \text{kPa}
\]  
The gas mass \( m_g \) is calculated using the ideal gas law:  
\[
m_g = \frac{V_{g,1} \cdot p_{g,1}}{R \cdot T_{g,1}}
\]  
Substituting the values:  
\[
m_g = \frac{0.00314 \, \text{m}^3 \cdot 135.97 \, \text{kPa}}{8.314 \, \text{kJ}/\text{kmol·K} \cdot 773.15 \, \text{K}} \cdot \frac{50 \, \text{kg}}{\text{kmol}} = 3.41 \, \text{g}
\]