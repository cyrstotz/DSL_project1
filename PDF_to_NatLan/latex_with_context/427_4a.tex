A pressure-temperature (\( p \)-\( T \)) diagram is drawn. The vertical axis represents pressure in bar (\( p \, [\text{bar}] \)), and the horizontal axis represents temperature in Kelvin (\( T \, [\text{K}] \)). The diagram shows four states labeled as 1, 2, 3, and 4.  

- State 1 is at low pressure and temperature.  
- State 2 is at higher pressure and temperature, connected to State 1 by an upward sloping line.  
- State 3 is at high pressure and lower temperature, connected to State 2 by a horizontal line.  
- State 4 is at the same pressure as State 3 but at lower temperature, connected to State 3 by a downward sloping line.  

The pressure at State 3 and State 4 is \( p_3 = p_4 = 8 \, \text{bar} \).  
The temperature difference between State 1 and State 4 is indicated as \( T = 6 \, \text{K} \).  
The vapor quality at State 2 is \( x_2 = 1 \), and at State 4, \( x_4 = 0 \).