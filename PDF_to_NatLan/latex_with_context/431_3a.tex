The pressure \( p_{g,1} \) is calculated using the equation:  
\[
p_{g,1} = p_{\text{amb}} + \frac{m_K g}{A}
\]  
where \( A = \frac{\pi D^2}{4} \). Substituting values:  
\[
A = \frac{\pi}{4} \cdot (0.1 \, \text{m})^2 = \frac{\pi}{400} \, \text{m}^2
\]  
\[
p_{g,1} = 1 \, \text{bar} + \frac{32 \, \text{kg} \cdot 9.81 \, \text{m/s}^2}{\frac{\pi}{400} \, \text{m}^2} = 1.46 \, \text{bar}
\]  

The ideal gas law \( PV = mRT \) is used to calculate the gas mass \( m_g \):  
\[
R = \frac{R_u}{M} = \frac{8314 \, \text{J/(kmol·K)}}{50 \, \text{kg/kmol}} = 166.28 \, \text{J/(kg·K)}
\]  
\[
m_g = \frac{p_{g,1} V_{g,1}}{R T_1} = \frac{1.46 \, \text{bar} \cdot 3.14 \, \text{L}}{166.28 \, \text{J/(kg·K)} \cdot 773.15 \, \text{K}}
\]  
Converting units:  
\[
p_{g,1} = 1.46 \cdot 10^5 \, \text{Pa}, \quad V_{g,1} = 3.14 \cdot 10^{-3} \, \text{m}^3
\]  
\[
m_g = \frac{1.46 \cdot 10^5 \cdot 3.14 \cdot 10^{-3}}{166.28 \cdot 773.15} = 3.62 \cdot 10^{-3} \, \text{kg} = 3.62 \, \text{g}
\]  

---