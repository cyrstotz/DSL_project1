Two diagrams are drawn to represent the freeze-drying process in a pressure-temperature (\(p\)-\(T\)) diagram.  

1. **First Diagram**:  
   - The diagram shows phase regions with labeled isobars and states.  
   - States 1, 2, 3, and 4 are marked along the process path.  
   - The isobar at \(p = 5 \, \text{bar}\) is drawn, with the vapor quality (\(x\)) transitioning from \(x = 0\) (saturated liquid) to \(x = 1\) (saturated vapor).  
   - The process path includes evaporation, compression, condensation, and expansion.  

2. **Second Diagram**:  
   - A more detailed view of the phase region is shown with an isobar at \(p = 8 \, \text{bar}\).  
   - States 1, 2, and 3 are labeled, with transitions between \(x = 0\) and \(x = 1\).  
   - The temperature axis (\(T\)) and pressure axis (\(p\)) are labeled.