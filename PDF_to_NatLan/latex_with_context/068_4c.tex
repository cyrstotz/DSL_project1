The process from state 1 to state 2 involves compression. The temperature difference \( \Delta T \) is \( 6 \, \text{K} \), resulting in \( T_i = 26^\circ\text{C} \).  

For the stationary equilibrium process, the equation is:  
\[
0 = \dot{m}_{\text{R134a}} \left( h_1 - h_2 + \frac{v_2^2}{2} - \frac{v_1^2}{2} \right) + \dot{Q}_K
\]  
This simplifies to:  
\[
\dot{m}_{\text{R134a}} = \frac{\dot{Q}_K}{h_e - h_a}
\]  
where \( h_a \) and \( h_e \) are enthalpies.

The vapor quality \( x_1 \) of the refrigerant at state 1 after expansion is determined as:  
\[
x_1 = \frac{h_{1,\text{actual}}}{h_g}
\]  
Substituting values:  
\[
x_1 = 0.6352 \, \text{g/s}
\]

The work done by the refrigerant is calculated using the integral:  
\[
W_{4 \to 1} = \int_{4}^{1} p \, dV = m_{\text{R134a}} \cdot \frac{R}{M} \cdot (T_4 - T_1)
\]  
Here, \( R \) is the specific gas constant, \( M \) is the molar mass, and \( T_4 \) and \( T_1 \) are the temperatures at states 4 and 1, respectively.  

Additional notes:  
- The term \( \frac{1}{k} \) is referenced, likely related to the isentropic process.  
- The refrigerant used is R134a.  

No diagrams or figures are present on the page.