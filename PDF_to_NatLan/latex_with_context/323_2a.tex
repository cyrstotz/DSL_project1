Two diagrams are drawn to represent the thermodynamic processes in a jet engine.  

1. **First diagram**:  
   - The axes are labeled as \( T \) (temperature) and \( S \) (entropy).  
   - The process starts at point 1 and follows an isentropic path to point 2.  
   - From point 2, the process moves to point 3 along an isobaric path.  
   - The process then transitions to point 4 via an adiabatic path.  
   - From point 4, the process moves to point 5 along another isobaric path.  
   - Finally, the process transitions to point 6 along an isentropic path.  
   - The diagram includes annotations for "isentrop" (isentropic) and "isobar" (isobaric) processes.  

2. **Second diagram**:  
   - The axes are labeled as \( T \) (temperature) and \( S \) (entropy).  
   - The process starts at point 1 and follows an isentropic path to point 2.  
   - From point 2, the process moves to point 3 along an isobaric path.  
   - The process transitions to point 4 via an adiabatic path.  
   - From point 4, the process moves to point 5 along another isobaric path.  
   - Finally, the process transitions to point 6 along an isentropic path.  
   - The diagram includes annotations for "isentrop" (isentropic) and "isobar" (isobaric) processes.  

---