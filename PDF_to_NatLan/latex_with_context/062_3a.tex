The universal gas constant \( R \) is calculated as:  
\[
R = \frac{8.314 \, \text{kJ}}{\text{kmol·K}} \cdot \frac{1}{50 \, \text{kg/kmol}} = 0.166 \, \text{kJ/kg·K}.
\]

The pressure of the gas in state 1, \( p_{g,1} \), is determined using the following relationships:  
The cross-sectional area \( A \) of the cylinder is:  
\[
A = \pi \left( \frac{D}{2} \right)^2.
\]

The pressure \( p_{g,1} \) is expressed as:  
\[
p_{g,1} = p_{\text{EW},1} + \frac{m_K \cdot g}{A} + p_{\text{amb}},
\]
where \( p_{\text{EW},1} \) is the pressure exerted by the ice-water mixture, \( m_K \) is the mass of the piston, \( g \) is the gravitational acceleration, and \( p_{\text{amb}} \) is the ambient pressure.

Substituting values:  
\[
p_{g,1} = p_{\text{EW},1} + \frac{32 \, \text{kg} \cdot 9.81 \, \text{m/s}^2}{\pi \left( \frac{0.1 \, \text{m}}{2} \right)^2} + 1.013 \, \text{bar}.
\]

After calculation, \( p_{g,1} \) is approximately \( 1.40 \, \text{bar} \).

The gas mass \( m_g \) is calculated using the ideal gas law:  
\[
pV = mRT.
\]

Given:  
\[
T_{g,1} = 773.15 \, \text{K}, \quad p_{g,1} = 1.40 \cdot 10^5 \, \text{Pa}, \quad V_{g,1} = 3.14 \cdot 10^{-3} \, \text{m}^3, \quad R = 0.166 \, \text{kJ/kg·K}.
\]

Substituting:  
\[
m_g = \frac{p_{g,1} \cdot V_{g,1}}{R \cdot T_{g,1}} = \frac{1.40 \cdot 10^5 \cdot 3.14 \cdot 10^{-3}}{0.166 \cdot 10^3 \cdot 773.15} = 3.43 \, \text{g}.
\]

---

The heat transferred \( Q_{12} \) is given as:  
\[
Q_{12} = 1300 \, \text{J}
\]

The volume of the ice-water mixture remains constant:  
\[
V_{\text{EW}} = V_{2,\text{EW}} \quad \text{and} \quad v_{1,\text{EW}} = v_{2,\text{EW}}
\]

The specific volume of the ice-water mixture is calculated using the ice fraction \( x_{\text{ice}} \):  
\[
v_{\text{EW}} = 0.6 \cdot v_g(0^\circ\text{C}) + (1 - 0.6) \cdot v_f(0^\circ\text{C}) = 125.98 \, \text{m}^3/\text{kg}
\]