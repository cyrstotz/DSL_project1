The table summarizes the states in the refrigeration cycle:  

\[
\begin{array}{|c|c|c|c|}
\hline
\text{State} & T \, (\text{K}) & P \, (\text{bar}) & \text{Process Description} \\
\hline
1 & 31.33 & 1 & \text{Adiabatic process: Isentropic expansion, } h_4 = h_1 \\
2 & 220 & 1 & \text{Isobaric evaporation, } \dot{Q}_K \\
3 & 31.33 & 5 & \text{Isentropic compression, } s_2 = s_3 \\
4 & 31.33 & 8 & x_4 = 0 \\
\hline
\end{array}
\]

The enthalpy at state 4 is equal to the enthalpy at state 1:  
\[
h_4 = h_1 = h_f(5 \, \text{bar}) \approx 93.42 \, \frac{\text{kJ}}{\text{kg}} \quad \text{(from Table A-11)}  
\]

The entropy at state 2 is equal to the entropy at state 3:  
\[
s_2 = s_3 = s_g(8 \, \text{bar}) \approx 0.9066 \, \frac{\text{kJ}}{\text{kg·K}} \quad \text{(from Table A-11)}  
\]

For the isentropic process between states 2 and 3:  
\[
s_2 = s_3 = 0.9066 \, \frac{\text{kJ}}{\text{kg·K}}  
\]

Using the first law of thermodynamics for the compressor:  
\[
\frac{dE}{dt} = \dot{m}_{\text{R134a}} (h_2 - h_3) + \dot{Q} - \dot{W}_K  
\]

Rearranging to find the mass flow rate of the refrigerant:  
\[
\dot{m}_{\text{R134a}} = \frac{\dot{W}_K}{h_2 - h_3}  
\]

---