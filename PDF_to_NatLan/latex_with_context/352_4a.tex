A pressure-temperature (\( p \)-\( T \)) diagram is drawn to represent the freeze-drying process. The diagram includes the following features:  
- A curve labeled "Gas" indicating the phase boundary.  
- A horizontal line labeled "Tripel" marking the triple point of water.  
- Three states labeled \( 1 \), \( 2 \), and \( 3 \):  
  - State \( 1 \) is at the beginning of the process.  
  - State \( 2 \) is after step (i), following isobaric cooling.  
  - State \( 3 \) is after step (ii), following isothermal pressure reduction.  
- Vertical arrows indicate transitions between states, labeled "Kühlen" (cooling) and "Druckabl." (pressure reduction).  
- The region above the curve is labeled "Fest" (solid), and the region below is labeled "Flüssig" (liquid).