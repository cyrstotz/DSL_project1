The equation for the coefficient of performance \( \epsilon_K \) is written as:  
\[
\epsilon_K = \frac{\dot{Q}_K}{\dot{W}_K}
\]  

Below this, there is a table with columns labeled \( T \), \( p \), \( x \), \( v \), \( h \), and \( s \). The rows contain numerical values, but some entries are unclear or illegible. The table appears to represent thermodynamic properties at different states, possibly for R134a.  

A graph is drawn below the table. It is a pressure-temperature (\( p-T \)) diagram with labeled regions:  
- The "Triple Point" is marked as a specific point on the graph.  
- The curve separates the "Solid" and "Liquid" regions.  
- The "Solid + Gas" region is labeled above the curve.  
- The "Liquid + Gas" region is labeled below the curve.  

Some parts of the graph are crossed out and illegible.  

No additional content is visible.