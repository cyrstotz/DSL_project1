The energy balance equation is written as:  
\[
\dot{m} \cdot (h_{\text{in}} - h_{\text{out}}) + \dot{Q}_R = 0
\]  
where \( h_{\text{in}} \) and \( h_{\text{out}} \) represent the specific enthalpies at the inlet and outlet, respectively.  

The enthalpy values are calculated using the water tables:  
\[
h_{\text{in}} = h_{\text{sat, liquid}}(70^\circ\text{C}) = 293.85 \, \text{kJ/kg}
\]  
\[
h_{\text{out}} = h_{\text{sat, liquid}}(100^\circ\text{C}) = 419.17 \, \text{kJ/kg}
\]  

Substituting into the energy balance:  
\[
\dot{Q}_{\text{out}} = \dot{m} \cdot (h_{\text{out}} - h_{\text{in}}) - \dot{Q}_R
\]  
\[
\dot{Q}_{\text{out}} = 0.3 \, \text{kg/s} \cdot (419.17 - 293.85) \, \text{kJ/kg} - 100 \, \text{kW}
\]  
\[
\dot{Q}_{\text{out}} = -0.0226 \, \text{kW}
\]

The page contains a table and some calculations related to jet engine thermodynamics. Below is the transcription of the content:

### Table Description:
The table lists various states (labeled as "Zustände") from 0 to 6, with corresponding values for pressure \( P \, [\text{bar}] \), temperature \( T \, [^\circ\text{C}] \), velocity \( w \, [\text{m/s}] \), and specific enthalpy \( h \, [\text{kJ/kg}] \).  

| Zustand | \( P \, [\text{bar}] \) | \( T \, [^\circ\text{C}] \) | \( w \, [\text{m/s}] \) | \( h \, [\text{kJ/kg}] \) |
|---------|-------------------------|----------------------------|------------------------|--------------------------|
| 0       | 0.191                   | -30                        | 200                    |                          |
| 1       |                         |                            |                        |                          |
| 2       |                         |                            |                        |                          |
| 3       |                         |                            |                        |                          |
| 4       |                         |                            |                        |                          |
| 5       | 0.5                     | 431.9                      | 220                    |                          |
| 6       | 0.5                     |                            |                        |                          |

### Notes and Calculations:
1. \( w_6 = \sqrt{2 \cdot \left( h_5 - h_6 \right) + w_5^2} \)  
   This formula calculates the velocity \( w_6 \) at state 6 based on enthalpy differences and the velocity at state 5.  

2. \( h_6 = h_5 - 0.297 \, \text{kJ/kg} \)  
   The enthalpy at state 6 is determined by subtracting a specific value from the enthalpy at state 5.  

3. \( h_5 = 657.4 \, \text{kJ/kg} \)  
   The specific enthalpy at state 5 is given as \( 657.4 \, \text{kJ/kg} \).  

4. \( w_5 = 220 \, \text{m/s} \)  
   The velocity at state 5 is explicitly stated as \( 220 \, \text{m/s} \).  

5. \( T_0 = -30^\circ\text{C} \), \( P_0 = 0.191 \, \text{bar} \)  
   Ambient conditions are noted for state 0.  

### Additional Notes:
- The table and calculations appear to be part of a solution for determining the outlet velocity \( w_6 \) and other thermodynamic properties of the jet engine process.  
- The handwritten notes suggest the use of energy balance equations and specific enthalpy values to compute the desired quantities.  

No diagrams or graphs are present on this page.

The page contains a diagram illustrating a thermodynamic process on a \( T \)-\( s \) diagram. The axes are labeled as follows:  
- The vertical axis represents temperature (\( T \)) in Kelvin (\( K \)).  
- The horizontal axis represents entropy (\( s \)) in \( \frac{\text{kJ}}{\text{kg·K}} \).  

The diagram includes several curves and points labeled with numbers:  
1. Point 0 is the starting state.  
2. Point 1 is connected to point 0 via an isobaric compression curve.  
3. Point 2 is connected to point 1 via an isobaric combustion curve.  
4. Point 3 is connected to point 2 via an adiabatic expansion curve.  
5. Point 4 is connected to point 3 via a mixing process.  
6. Point 5 is connected to point 4 via an isobaric cooling curve.  

The curves are annotated with descriptions such as "isobaric compression," "isobaric combustion," "adiabatic expansion," and "isobaric cooling."  

This diagram qualitatively represents the thermodynamic process of a jet engine, as described in the problem setup for Task 2.