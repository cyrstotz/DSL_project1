The handwritten page includes a table and multiple diagrams related to Task 4, which concerns the freeze-drying process with the R134a cycle.

### Table Description:
The table lists four states of the R134a cycle with corresponding values for \( p \) (pressure in bar), \( x \) (vapor quality), and notes about the processes at each state:
1. \( p = 1 \, \text{bar}, x = \text{C} \): Heat \( \dot{Q}_K \) is removed at \( T_i \).  
2. \( p = 1 \, \text{bar}, x = 1 \): Isentropic compression occurs.  
3. \( p = 8 \, \text{bar}, x = \text{V} \): \( \dot{W}_K = 28 \, \text{kW} \).  
4. \( p = 8 \, \text{bar}, x = 0 \): Heat \( \dot{Q}_K \) is added.

### Diagram Descriptions:
1. **First Diagram (p-T Diagram)**:
   - The diagram shows the pressure-temperature relationship for the freeze-drying process.  
   - States 1, 2, 3, and 4 are labeled along the curve, with transitions between phases (fluid, vapor, and liquid).  
   - The triple point is marked, and the process includes decompression and phase changes.

2. **Second Diagram (p-T Diagram)**:
   - A more detailed view of the freeze-drying process, showing phase regions (fluid, vapor, and liquid).  
   - States 1, 2, 3, and 4 are labeled, with arrows indicating transitions between states.  
   - The triple point is labeled, and decompression is noted.