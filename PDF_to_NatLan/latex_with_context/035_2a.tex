The task involves drawing the process qualitatively in a \( T \)-\( s \) diagram with labeled isobars and units on the axes.  

### Diagram Description:  
The diagram is a \( T \)-\( s \) plot with entropy \( s \) on the horizontal axis (units: \( \text{kJ/kg·K} \)) and temperature \( T \) on the vertical axis. The process is represented by several distinct segments:  
1. From point \( 0 \) to \( 1 \), entropy \( s \) decreases while temperature \( T \) decreases.  
2. From point \( 1 \) to \( 2 \), the process is an isentropic compression (\( s_1 = s_2 \)).  
3. From point \( 2 \) to \( 3 \), the process is isobaric with increasing temperature \( T \).  
4. From point \( 3 \) to \( 4 \), entropy \( s \) increases.  
5. From point \( 4 \) to \( 5 \), pressure \( p_4 = p_5 \).  
6. From point \( 5 \) to \( 6 \), entropy remains constant (\( s_5 = s_6 \)).  

### Process Notes:  
- \( 0 \rightarrow 1 \): Entropy \( s \) decreases, temperature \( T \) decreases.  
- \( 1 \rightarrow 2 \): Isentropic compression (\( s_1 = s_2 \)).  
- \( 2 \rightarrow 3 \): Isobaric process with increasing temperature \( T \).  
- \( 3 \rightarrow 4 \): Entropy \( s \) increases.  
- \( 4 \rightarrow 5 \): Pressure remains constant (\( p_4 = p_5 \)).  
- \( 5 \rightarrow 6 \): Entropy remains constant (\( s_5 = s_6 \)).  

No additional numerical values or equations are provided.