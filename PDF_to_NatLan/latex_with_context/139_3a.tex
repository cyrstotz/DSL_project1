The ideal gas law is applied to calculate the pressure and mass of the gas in state 1:  
\[
\rho V = mRT
\]  
The mass of the gas is calculated as:  
\[
m_{\text{gas}} = \frac{p V_{g,1}}{R T_{g,1}} = \frac{\frac{F_K}{A} \cdot V_{g,1}}{R T_{g,1}}
\]  
Substituting the values:  
\[
m_{\text{gas}} = \frac{\frac{32.1 \, \text{kg} \cdot 9.81 \, \text{m/s}^2}{0.005 \, \text{m}^2} \cdot 3.14 \, \text{L}}{50 \cdot (500 + 273.15)} = 9.982539 \, \text{g}
\]  
Approximated as:  
\[
m_{\text{gas}} \approx 9.983 \, \text{g}
\]  

The pressure is calculated as:  
\[
p = \frac{F_K + F_{\text{EW}}}{A} = \frac{4087.059939 \, \text{N}}{0.005 \, \text{m}^2} \approx 4087.059939 \, \text{Pa}
\]