The thermodynamic mean temperature of the coolant is expressed as:  
\[
T_{\text{FK}} = \frac{\int T \, dS}{\int dS}
\]  
The final value is left incomplete in the handwritten text.

The outlet temperature \( T_6 \) is calculated as follows:  
The process \( 5 \rightarrow 6 \) is reversible and adiabatic, meaning it is isentropic. Thus, \( s_5 = s_6 \).  

Using the isentropic relation:  
\[
T_6 = T_5 \left( \frac{p_6}{p_5} \right)^{\frac{\kappa - 1}{\kappa}}
\]  
Substituting values:  
\[
T_6 = 431.9 \, \text{K} \left( \frac{0.491}{0.5} \right)^{\frac{0.4}{1.4}} = 328.07 \, \text{K}
\]  

Next, the outlet velocity \( w_6 \) is determined using the steady-flow energy equation:  
\[
0 = h_0 - h_c + \frac{w_6^2 - w_c^2}{2} + \dot{q}_{\text{out}} + \dot{W}_{\text{in}} + 90 \, m_c
\]  

Rewriting for \( w_6^2 \):  
\[
w_6^2 = 2 \, c_{p,\text{air}} (h_0 - h_c) + w_c^2 = 2 \, c_{p,\text{air}} (T_0 - T_6) + w_c^2
\]  

Substituting values:  
\[
w_6 = \sqrt{2 \cdot 1.006 \, \text{kJ/kg·K} \cdot (243.15 \, \text{K} + 328.07 \, \text{K}) + (200 \, \text{m/s})^2}
\]  
\[
w_6 = 459.2 \, \text{m/s}
\]  

Additional notes:  
- \( m_g = m_c + m_k \cdot (1.5/5.293) \).  
- \( w_m = \frac{w_k}{6.293} \).