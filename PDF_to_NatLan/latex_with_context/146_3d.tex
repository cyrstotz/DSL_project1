The ice fraction in state 2 is calculated as follows:  

The temperature difference between the ice and the surrounding environment is given as:  
\[
T_{\text{ice},2} - T_{\text{EW},2} = 0.003^\circ\text{C}
\]  

The energy balance is expressed as:  
\[
\Delta E = E_2 - E_1 = Q_{12} = 1500 \, \text{J}
\]  

The change in internal energy is calculated using:  
\[
\Delta U_{12} = 1500 \, \text{J} = m \cdot (u_2 - u_1) = m \cdot \left[ (x_2 \cdot u_{2,\text{ice}} + (1 - x_2) \cdot u_{2,\text{fusion}}) - (x_1 \cdot u_{1,\text{ice}} + (1 - x_1) \cdot u_{1,\text{fusion}}) \right]
\]  

Rewriting the equation for \( x_2 \):  
\[
x_2 = \frac{\Delta U_{12}}{m} + \left[ x_1 \cdot u_{1,\text{ice}} + (1 - x_1) \cdot u_{1,\text{fusion}} \right] - u_{2,\text{fusion}} \, \bigg/ \, (u_{2,\text{ice}} - u_{2,\text{fusion}})
\]  

Substituting the values, the final ice fraction is determined to be:  
\[
x_2 = 0.555
\]