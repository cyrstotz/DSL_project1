Zuerst muss der Druck im Gasbehälter berechnet werden:  
Druck EW + Druck vom Gewicht + Druck Umgebung  

P L = F über A = 32 kg mal 9,81 Newton pro Quadratmeter über Pi mal (2 mal 10 hoch minus 2 Meter) hoch 2 = 39,96,8 Newton pro Quadratmeter  

P EW = F über A = 0,124 mal 9,81 Newton pro Quadratmeter über Pi mal (5 mal 10 hoch minus 2 Meter) hoch 2 = 724,304 Newton pro Quadratmeter  

P am6 = 10 hoch minus 5 Newton pro Quadratmeter  

P g,1 = p L + p EW + p am6 = 7400594,44 Newton pro Quadratmeter  

R = R über M = 8,314 Joule pro Mol Kelvin über 50 Gramm pro Mol = 0,16628 Joule pro Kilogramm Kelvin  

P g,1 = 1,4 bar  

m g = P g,1 mal V g,1 über R mal T g,1 = 1,4 mal 10 über 3 mal 1 über Quadratmeter mal 3,14 mal 10 hoch minus 3 Kubikmeter über 0,16628 Joule pro Kilogramm Kelvin mal 7,93,15 Kelvin = 3,419 Gramm