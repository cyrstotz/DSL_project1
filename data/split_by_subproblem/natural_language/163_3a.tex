Energy mass balance 2 to 3

Zero equals mass flow rate times h two minus h three plus zero times x zero plus rho times h two minus h three plus W i k

W i k over h two minus h three equals mass flow rate times rho one three four a

Assumption rho one equals rho two

T K equals T j; Assumption one x greater than zero

Throttle isenthalpic TAB A11

rho four 18 bar; h four 18 bar equals h four 18 bar prime equals 83.142 kilojoules per kilogram

h one equals h four equals 83.142 kilojoules per kilogram

T two from solution; T two equals minus 72 degrees Celsius

To table A10

h two equals h g minus 72 degrees equals 734.08 kilojoules per kilogram

h three S two equals S three 8 bar

S two equals S three equals S g minus 72 degrees equals 0.8351 kilojoules per kilogram Kelvin

Linear interpolation table A10 to A12

For 8 bar superheated steam

h three equals 0.93551 kilojoules per kilogram equals h zero 0.8374 minus h one 0.80666 kilojoules per kilogram divided by 0.8374 minus 0.80666 kilojoules per kilogram times 0.93551 minus 0.80666 kilojoules per kilogram

h three equals 727.95 kilojoules per kilogram

A diagram labeled 'Pgm' shows a box with arrows pointing down labeled 'Fuz' and arrows pointing up labeled 'Fgm'. 

Pgm equals pamb plus m times g times s divided by AK plus mEW times g times s divided by Am. AK equals Am equals D squared divided by four times pi equals 0.14 meters squared times pi equals 0.00785 square meters.

Pgm equals one bar plus thirty-two kilograms times 9.81 meters per second squared times 10 to the power of minus five bar per pascal divided by 0.00785 square meters plus 0.1 times 9.81 meters per second squared times 10 to the power of minus five bar per pascal divided by 0.00785 square meters equals 1.1401 bar.

Ideal Gas Law

pV equals mRT leads to m equals pV1 divided by RT1 equals 1.1401 bar times 10 to the power of five pascal times 0.34 cubic meters divided by 50 kilograms per mole times 166.3 joules per kilogram times (273.15 plus 500) Kelvin.

m equals 0.0034 kilograms equals 3.42 grams.

The pressure remains as in state 1 because the system expands upwards against the lower pressure, thus the mass remains the same. P2 equals 1.1401 bar.

Since the system is in thermodynamic equilibrium, T2 equals TE equals T(P2) equals 0 degrees Celsius equals TZGI equals TEZW.

See TE(P2) from Table 1.

One hundred thirty-six thousand seven hundred sixty-three

X sub two equals open parenthesis omega max minus two hundred point zero eight divided by i sub omega times ten to the power of three plus zero point zero three three divided by i sub omega times ten to the power of three close parenthesis divided by open parenthesis negative three hundred thirty-three point four five eight plus zero point zero three three close parenthesis times ten to the power of three divided by i sub omega

X sub two equals zero point one five five eight nine approximately equal to fifty-five point eight nine percent

equals fifty-five point eight percent

equals zero point one five five eight