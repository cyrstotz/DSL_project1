Adiabatic reversible substitution implies s sub c equals s sub c. Indicates gas, polytropic equation: T sub 6 over T sub 5 equals (p sub 0 over p sub 5) to the power of (n minus 1 over n), implies T sub 6 equals T sub 5 times (p sub 6 over p sub 5) to the power of (n minus 1 over n).

e equals 43,1 joules per kilogram, u sub 5 equals 429,1 kilojoules per kilogram, (p sub 6 plus 1) over p sub 6 equals 328,07 Kelvin equals T sub 6.

EB un substituted: 0 equals m dot s times (w sub 2 squared minus w sub 6 squared over 2 plus h sub 5 minus h sub 6) equals q dot out minus w dot s c.

p three equals p four equals eight bar

h two equals h four

h two equals

EB um Verdichter

0 equals m R two three four times (h two minus h three) minus W K

equals m R two three four times W K divided by (h two minus h three)

Q K equals Q ab minus W K