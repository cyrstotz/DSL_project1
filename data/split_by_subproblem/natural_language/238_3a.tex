p one g equals p ambient plus Fg over A

Fg equals 32 kilograms times 9.81 meters per second squared equals 313.92 N

A equals pi times (P one two) squared equals 0.007853982 square meters

p one g equals 1 bar plus Fg over A equals 10 to the power of 5 pascal plus 39966.59 pascal equals 1.4 bar

To determine the mass of the gas, we can now apply the ideal gas law:

m g equals p one g V one g over R T one g equals 1.4 bar times 3.14 times 10 to the power of minus 3 cubic meters over 0.16625 cubic meters per kilogram times 473.15 Kelvin equals 0.00342 kilograms

R equals R over m gas equals 0.16625 cubic meters per kilogram

m g equals 3.42 grams