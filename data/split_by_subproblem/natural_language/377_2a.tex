U sub A equals U sub P (100 degrees) plus x times (U sub G (100) plus U sub L (100))

418.94 plus 0.005 times (419.54 minus 418.94)

U sub A equals 418.94 kilojoules per kilogram

A sub 2 U sub 2 equals U (70 degrees) equals U sub G (70 degrees) equals 292.95 kilojoules per kilogram

gesamte Masse steckt

delta M equals 5755 times 418.94 kilojoules per kilogram minus 35 times 10 to the power of 3 kilojoules

divided by 292.95 minus 83.36

delta M equals 11368.9 kilograms

A graph is drawn with the vertical axis labeled as 'T' in square brackets 'K' and the horizontal axis labeled as 's' in square brackets 'kilojoule per kilogram'. The graph shows a curve with points labeled 1, 2, 3, 4, 5, and 6. The curve is marked with different sections: 'isotrope', 'adiatope', 'isobare', and 'isotrope'. Near the curve, there are labels 'Punktnummer', 'p equals c, q isobar', and 'p equals o'. A small sketch in the bottom left shows a curve labeled 'isotherm', 'adiabat', and 'isobare'.