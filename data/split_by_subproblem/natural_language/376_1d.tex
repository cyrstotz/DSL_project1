u r equals u f at ninety degrees Celsius equals two hundred ninety-two point five kilojoules per kilogram

u n equals u f at one hundred degrees Celsius plus x d times u g at one hundred degrees Celsius minus u f at one hundred degrees Celsius equals three hundred three point eight three three kilojoules per kilogram

h ein equals h f at twenty degrees Celsius equals eighty-three point nine six kilojoules per kilogram

Sum n equals one hundred thirty-two point two seven five kilograms

Q sub R, n = 35 kN = 35,000 kg

Halboffenes System

m dot 2 u 2 minus m dot 1 u 1 = delta m dot n h ein plus Q aus

u 2 = u wasser (70 degrees Celsius) siedende Flüssigkeit x sub D = 0

u 1 = u wasser (100 degrees Celsius) implies x sub D = 0.005

h ein = h (20 degrees Celsius) siedende Fl.

m dot 2 = m dot 1 plus sum m dot n

(m dot 1 plus sum m dot n) u 2 minus m dot 1 u 1 = sum m dot n h ein plus Q

m dot 1 (u 2 minus u 1) plus sum m dot n (u 2 minus h ein) = Q

Q minus m dot n (u 2 minus u 1) divided by (u 2 minus h ein) = sum m dot n