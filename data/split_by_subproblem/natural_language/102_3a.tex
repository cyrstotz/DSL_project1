a) P one and V one to gas in cylinder one

Condition: T gas one equals five hundred degrees Celsius, V gas one equals three point seven four liters equals three point seven four times ten to the power of negative three cubic meters

m E one equals rho E one times V E one, rho E one equals x times rho M one equals m E one divided by V E one, x equals F E one divided by m E one, x equals F E one divided by m E one

Diagram: 
A horizontal line with a force F pointing vertically upwards. There is a label "Hebelarm" and a horizontal line extending to the right labeled "F gas". The force F is labeled with an arrow pointing upwards.

Below the diagram: 
Durch des gases
Da mit gleichgewicht F equals F gas

P one, A equals F gas divided by A equals ten to the power of five divided by thirty-two point one kilograms times nine point eight one meters per second squared plus thirty-two kilograms times nine point eight one meters per second squared divided by A equals two point five times ten to the power of negative two square meters

F equals four point nine times ten to the power of negative two

P equals one point two six times ten to the power of five divided by V in cubic meters equals one point four times ten to the power of five bar

Below the diagram: 
Durch des gases

F equals F gas mit gleichgewicht

F gas equals P gas one minus A

P gas one divided by F gas one equals F divided by A equals ten to the power of five divided by A equals thirty-two point one kilograms times nine point eight one meters per second squared plus thirty-two kilograms times nine point eight one meters per second squared divided by two point eight five three five times ten to the power of negative three square meters

equals one point four times ten to the power of five Newtons per square meter equals one point four times ten to the power of five bar

Below the diagram: 
F equals force divided by A times (m E one plus m M one) times five point nine five meters per second squared

Hebelarm

Area A equals pi times radius squared equals pi times (five times ten to the power of negative two meters) squared equals pi times (five times ten to the power of negative two meters) squared

equals seven point eight five three nine times ten to the power of negative three square meters

equals one point four times ten to the power of five Newtons per square meter equals one point four times ten to the power of five bar