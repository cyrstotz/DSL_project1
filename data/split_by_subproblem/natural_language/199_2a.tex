m dot times h sub e minus m dot times h sub a plus Q dot minus V dot sub o. h sub e, h sub a, h sub e, h sub a, h sub e, h sub a, h sub e, h sub a, h sub e, h sub a. Q dot minus V dot sub o. 0 equals m dot times h sub e minus h sub a plus Q dot minus V dot sub o.

h sub e equals h sub e at seventy degrees Celsius (Tabelle A-2) equals two hundred ninety-two point nine eight kilojoules per kilogram.

h sub a equals h sub a at two hundred seventy degrees Celsius (Tab. A-2) equals four hundred ninety-six point zero four kilojoules per kilogram.

Q dot equals m dot times h sub e minus h sub a equals zero point three times four hundred ninety-six point zero four kilojoules per kilogram minus two hundred ninety-two point nine eight kilojoules per kilogram equals thirty-seven point eight one kilowatts.

minus Q dot sub aus equals Q dot minus Q dot sub R equals thirty-seven point eight one kilowatts minus one hundred kilowatts equals minus sixty-two point one eight kilowatts. Q dot sub aus equals sixty-two point one eight kilowatts. Q dot sub aus equals sixty-two point seven eight kilowatts.

A diagram is drawn with a vertical axis labeled 'T [K]' and a horizontal axis labeled 's [kJ/kgK]'. The diagram features a series of connected points labeled '1', '2', '3', '4', '5', and '6', with arrows indicating direction between these points. Points are connected by lines, and labels 'v1', 'v2', 'v3', 'v4', 'v5', 'v6' are placed near the respective points.