a) Qaus

1HS stationär: U = m dot times (he minus ha) plus sum of Q

Q = m dot times (ha minus he)

Qaus = m dot times [c times (T2 minus T1) plus V times (P2 minus P1)]

equals zero to constant

Tab A1: Q = m dot times (haus minus hein)

Qab = minus 23.04 kilojoules

hein = 2333.8

haus = 2957.0

b) T KF equals integral from sa to se of T dS equals delta R CV over sa minus se equals [f times (T2 minus T1) plus V times (P2 minus P1) over Q times ln (T2 over T1)]

equals zero to constant

T KF equals Taus minus Tein equals 294.57 Kelvin

c) Serz zu Reaktor und Kühlmittel.

Entropie, stationär: zero equals m dot times (se minus sa) plus sum of Q plus Serz

zero equals m dot times ns e minus m dot times KF sa plus sum of Qab over T KT plus Serz

Serz equals m dot times (sa minus se) minus Gab over T KT

Serz equals m dot times c times (Taus over Tein) minus Qab over T KT

Serz equals minus Qab over T KT equals minus 0.165 kilowatts

Delta m12  
Halboffenes System  
dE/dt = m [he - ha] + Sigma Q [ami chi]  

mzu2 - mu1 = Qzu12 + am12 ha12  

mR [u2R - u1R] - QzuR = am12  
ha12  

Tab A2:  
z1: TR = 100 degrees Celsius  
z2: TR = 70 degrees Celsius  

Tein12 = 20 degrees Celsius  

ua2 = u1 8.94  
u2R = 252.95  

ha12 = 83.96  

Delta m1,2 = -5052.79 grams nicht möglich

d) muss beregen ex, verl  
Ex, verl = T0 S ex  
Ex, verl = T0 [m dot (sa - se) - Q dot j]  
= T0 [m dot (sb - se) - QB over TB]  
= T0 [m dot cp ln (T6 over T0) - QB over TB]  
Ex, verl = T0 [cp ln (T6 over T0) - QB over TB]  
= -152.15 W  

b) delta E = mec2 - tm1u1 + delta KE + delta PE  
m2 g s (u2 - u1) = wc2 squared over 2 - wo2 squared over 2  
wc2 squared - 2 m2 g s (u2  

0 = m dot (h0 - h6 + wc2 squared over 2 - wo2 squared over 2) + t z (I dot w)  
QB = m dot (h0 - h6 + wc2 squared over 2)  
m dot cp (T0 - T6)  
wc2 squared = 2 Q dot j over m + h0 - h6 + wo2 squared over 2  
wc2 squared = 2 Q dot j over m + 2 (h0 - h6) + wo2 squared  
= 2 Q dot j over m + 2 cp (T0 - T6) + wo2 squared

p-T Diagram Section id ii

A graph is drawn with the y-axis labeled 'p (bar)' and the x-axis labeled 'T (K)'. The graph shows a curve with points labeled 1, 2, 3, and 4. Point 2 is marked on the curve, and point 4 is marked on a straight line extending from the curve. There is a note 'T2 =' written near the graph.

Ek equals Qzw divided by wt equals Qk divided by wk equals Qzw divided by Qab minus Qzw equals Qk divided by Qab minus Qk.

Qk equals m times R3u times (h2 minus h1).

Qk equals 205.79 kilojoules.

Qab equals m times R34 times (hu minus h3).

Qab equals 156.29.

Ek equals Qk divided by wt equals 7349.64.

Tab Delta 10.

h2 equals 234.08.

h1 equals ne plus x times (hg minus ne1) minus 48.87.

n1 equals ne plus (hg minus ne1) equals 48.87.

h2 equals 234.08.

hu equals Tab AM.

hu equals ne1 equals 93.92.