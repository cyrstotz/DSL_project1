Equilibrium of forces leads to p g one times D divided by two squared times pi equals p amb times D divided by two squared times pi plus m k times g.

Leads to p g one equals p amb plus m k times g equals 100,000 pascal plus 32 kilograms times 9.81 newton per kilogram divided by (0.05 meters) squared times pi equals 1.3997 bar equals p g one.

According to the ideal gas law applies

m g equals p g one times V g one divided by R g times T g one leads to R g equals R divided by M g equals 8.314 kilojoule divided by kilogram times kelvin divided by 50 kilograms per kilomole equals 166.3 joules per kilogram.

Leads to m g equals 139970 pascal times 3.19 times 10 to the power of negative 3 cubic meters divided by 166.3 joules per kilogram times 773.15 kelvin equals 3.42 grams equals m g.

Refers to EW Table 1.  

x equals negative one hundred eighty-five point zero nine kilojoules per kilogram plus zero point zero three three kilojoules per kilogram divided by negative three hundred thirty-three point four nine two plus zero point zero three three kilojoules per kilogram.  

Equals zero point five five five equals x.