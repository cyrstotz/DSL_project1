A graph is drawn with an x-axis labeled 'S' and a y-axis labeled with 'T'. Several points and lines are marked on the graph:
- Point 1 is labeled '243.15'.
- An arrow points upwards from point 1 labeled 'P1 = P2'.
- A line from point 1 to point 2 is labeled 'c1 = c2', 'P = P1', and 'P2 = P3'.
- Another line goes from point 2 to point 3 labeled '0.5 bar'.
- A line from point 3 to point 4 is labeled 'P4 = P5'.
- A diagonal line from point 4 to point 5 is labeled 'c1, 9 isar'.
- The line from point 5 to point 6 is labeled 'a=5 bar'.
- An arrow points from the x-axis to the top of the graph labeled 'T=1.9'.

Below the graph, a table is drawn with columns labeled 'P', 'V', 'T', 'Tc', 'Q', and 'S'. The first row of the table has values: '0.191', '30', and an arrow pointing to the right. The second row has 'P2 = P3'. The fourth row has '0.5', 'P4 = P5', and 'a 5'.

To the right of the table, it is noted: 'Wout = 200 m/s', 'N/Vs < 1'.

h6 minus h5 equals 220 squared minus W6 squared over 2.  
h6 minus h5 equals 220 squared minus W6 squared over 2.  
h5 plus thea equals 220 squared minus W6 squared over 2 minus 200 squared minus W6 squared over 2.  
5 to 6 isentropic.  
T7 over T6 equals P7 over P6 to the power of n minus 1 over n.  
431.8 Kelvin over T6 equals 0.5 over P0 to the power of k minus 1 over k.  
equals 0.111 over 0.5 to the power of 0.4 over 1.4 equals 0.759608.  
Therefore, T6 equals 0.759608 times 431.8 Kelvin equals 328.0747 Kelvin.  
Von(1): 0 equals 1.006 times (431.8 Kelvin minus 328.0747 Kelvin) plus 220 squared minus W6 squared over 2 equals 104.4882 plus 24200 minus W6 squared over 2.  
W6 squared over 2 equals 24304.  
W6 star equals square root of 24308.  
V equals 220.472 meters per second.