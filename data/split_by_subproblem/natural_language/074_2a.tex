|   | P            | T          |
|---|--------------|------------|
| Zustand | 0.1 bar     | 30°C      |
| 1 |                |            |
| 2 | p squared      |            |
| 3 | p squared      |            |
| 4 | 0.5 bar        |            |
| 5 | 0.15 bar       | 437.13 K   |
| 6 | 0.1 bar        |            |

A graph is drawn with axes labeled T (K) on the vertical axis and S (J/kg·K) on the horizontal axis. The graph consists of a closed loop with points labeled 1, 2, 3, 4, 5, and 6 connected by lines. The segments between the points are labeled as follows:
- From point 1 to point 2: isobar
- From point 2 to point 3: isobar
- From point 3 to point 4: isobar
- From point 4 to point 5: isobar
- From point 5 to point 6: isobar
- From point 6 to point 1: isobar

Zero equals m times the quantity h c minus h a plus the quantity omega c squared minus omega a squared divided by two.

h s, zero point zero zero five bar, four hundred thirty-seven point three k.

n equals one point four.

T six divided by T five equals the quantity p six divided by p five to the power of one point four minus one divided by one point four, implies T six equals T five times the quantity p six divided by p five to the power of zero point four divided by one point four equals four hundred thirty-one point nine k times the quantity zero point one three seven divided by zero point one five equals two hundred fifty-four point one k.

h s minus h c equals the integral from T five to T six of c p d T equals the quantity T five minus T six times c p, implies h s minus h c equals one point zero zero six kilojoules per kilogram times the quantity four hundred thirty-seven point three k minus two hundred fifty-four point one k equals one hundred thirty-eight point two kilojoules per kilogram.

Zero equals m times the quantity one hundred thirty-eight point two kilojoules per kilogram plus the quantity two hundred twenty-three squared minus omega squared divided by two.