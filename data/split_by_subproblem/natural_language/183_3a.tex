General equation: p times V equals m times R times T

The pressure is found with the equilibrium:

p_g,1 equals (m_W,1 times g over A) plus (m_K times g over A) plus p_amb

with m_W,1 equals 0.6 times 0.1 kilograms
minus 96.6 newtons

with A equals pi times (D over 2) squared

p_g,1 equals 1.6 bar

m_g,1 equals R times T_g,1 over V_g,1 times p_g,1

with R equals R over M_g less than 166.28 joules per kilogram times Kelvin

equals 292

m_g,1 equals p_g,1 times V_g,1 over R minus T_g,1

equals 3.42 grams

The same temperature as that of T1, because in thermodynamic equilibrium.

P2,g equals 1.4 bar, because the mass remains the same with the same pressure.