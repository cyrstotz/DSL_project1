a) Qaus?

1 HS: 0 = ṁ (he - ha) + ΣQ̇ - ΣQ̇

Q̇aus = ṁ (ha - he) = Q̇R

he = hf (10°C, x=0) = 252.38 kJ/kg

ha = hf (100°C, x=0) = 419.04 kJ/kg [Tab. A-2]

Q̇aus = 0.3 kg (419.64 - 252.38) kJ/kg - 100 kJ/s 

= -62.18 kJ/s (kW)

b)

T̅KF = ∫a^a Tds / sa - se

∫a^a Tds = ha - he (1. HS)

= ha - he / sa - se → KF = ideale Flüssigkeit

ha - he = cKF (T2 - T1)

sa - se = cKF ln (T2 / T1)

T̅KF = cKF (T2 - T1) / cKF ln (T2 / T1)

= (298.15 K - 288.15 K) / ln (298.15 / 288.15)

= 293.12 K