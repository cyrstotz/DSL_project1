1. a) Energy balance of the volume:

dE over dt equals m dot (h1 minus h2) plus Q dot minus W dot

Q dot equals m (h2 minus h1)

h1 equals T1 times c_p plus minus 292.95 kilojoules per kilogram

h2 equals T2 times c_p plus 77.04 kilojoules per kilogram

Q dot out equals c_p times m dot times (7.9 plus 6.9) minus 292.95 kilojoules per kilogram

equals 37.8 kilowatts

Q dot in equals c_p times m dot times h_in minus Q dot

equals 62.1 kilowatts

b) 

Q dot equals U dot

c) dU over dt, h_in, h_out, equals U equals m dot (e minus e_k) plus C dot in over T plus i dot out

i dot ext equals m dot (h1 minus h2) minus C dot in over T

equals 62.1 kilowatts over 295 Kelvin minus 0.37 kilowatts

equals minus 0.37 kilowatts

d) 

dE over dt equals m (h1) plus Q dot minus W dot

delta U equals m h_in plus Q dot

m2 h_out minus m1 h_in equals m delta h_in minus Q dot

delta m equals m2 h_out minus m1 h_in

m2 over m1 minus m1 over m2 equals m delta h_in plus Q dot

delta m equals m2 U over h_in minus m1

equals 595 times 6.9 plus 7.9 plus 292.95 minus 350.3

over 292.95 minus 292.95 over 5.9 plus 6.9

equals 338.9 kilojoules

e) 

Q dot equals g times h_out over dt plus U dot

equals S times m2 minus m1 h2

m2 equals m1 plus delta m equals 595 plus 350.1 c

equals 945.9 g times h_out

equals g times 595 times 6.9 plus 7.9 over 292.95 minus 595 times h_out

equals 792.7 times 2.6 times h_out over delta h_in

equals 6.95 times 6.9 times h_out

equals 6.95 times h_out

a) Diagram labeled with T and s axes. The curve is marked with numbers 1, 2, 3, 4, 5, 6 and labeled with terms such as 'isotherm', 'isobar', 'isenthalpic', 'isentropic', 'isochoric', 'expander', 'compressor', and 'heater'.

b) Equations and calculations involving thermodynamics. Includes terms like 'T zero', 'T', 'process', 'S', 'equilibrium', 'dE/dt', 'h one', 'h six', 'u squared minus v squared divided by 2', 'u minus v', 'enthalpy', 'heat', 'T zero divided by T', 'K divided by f', 'P divided by f', 'T zero equals seventy-five', 'P divided by f equals seven point thirty-nine', 'u squared plus v squared equals zero', 'R times T zero minus T divided by nine minus K', 'K equals c times p times v divided by nine minus K', 'u equals zero point zero two four times square root of three hundred and twenty-eight point five times ninety-one', 'equals seven point zero six six times nine point two', 'equals six point eight two seven kilojoules per kilogram', 'equals seventy-seven point six five nine kilojoules per kilogram', 'u zero equals square root of two times T zero minus T plus u squared minus two times K', 'equals square root of two times one hundred and six times nine point three one minus three hundred and twenty-eight point six times seventy-five times K times two hundred', 'equals thirty-six point zero eight times seven eight two five meters per joule equals u zero'.

dE over dt equals V times rho times d over dt of (...) plus Q dot minus W dot.

m times V bracket t equals Q.

Result: F_EW equals d times i times d times i times d times i.

F_EW equals M_A bracket t over A plus f equals f_1 times rho_1 over rho_2 plus c squared equals f_1 times rho_1 over rho_2.

u_2 equals u_4 bracket t equals u_4 bracket t plus rho times bracket H_2 minus H_1 bracket.

u_2 equals Q over ln u_4.

u_2 equals 79.257 times bracket minus 2.0 times 10 to the power of 4 bracket over 0.141 plus minus 2060.914 times bracket 1 over h bracket equals minus 192.300 kilojoules per kilogram.

equals minus 200.694.

Beta equals P_1 equals i times m_a times rho times h times k_B times t times g times i times b.

X_EW equals U_2 minus U_1 times bracket u_1 minus u_4 bracket.

X_EW equals minus 191.206 minus bracket minus 337.934 bracket over minus 0.085 minus bracket minus 1533.555 bracket equals 0.42333.

X_EW equals 1 minus X_FL.

equals 0.57663.

A graph is drawn with axes labeled 'p' and 'T'. Two lines are marked: i) 'disso' and ii) 'diffusion'.