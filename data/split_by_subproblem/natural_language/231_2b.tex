Fließprozess stationär:

O equals M dot multiplied by (he minus ha plus (wa squared minus wa2 squared divided by two)) plus the sum of Qi minus the sum of wo.

A term is crossed out: QB divided by M dot. 

The equation continues with he(T5) minus ha(T6) plus (wa squared minus wa2 squared divided by two).

T5 equals 431.9 K.

T6: T5 divided by T5 equals (P6 divided by P5) raised to the power of n minus one divided by n, which implies T6 equals 431.8 K multiplied by (0.10 bar divided by 0.5 bar) raised to the power of 1.4 minus one divided by 1.4.

T6 equals 328.947 K which approximates to 328.08 K.

A calculation for he(T5) is shown: h(490) plus 431.9 K minus 430 K divided by 490 K minus 430 K multiplied by (h(490) minus h(430)) equals 433.364 LD. 

Another calculation for ha(T6) is shown: h(330) plus 328.08 K minus 325 K divided by 330 K minus 325 K multiplied by (h(330) minus h(325)) equals 328.38 LD.

An equation is provided: wa equals wa2 squared minus (he(T5) minus ha(T6)) plus QB divided by M dot plus wa squared.

M dot equals question mark.