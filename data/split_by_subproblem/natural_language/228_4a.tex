A graph is shown with a y-axis labeled 'P [bar]' and an x-axis labeled 'T [K]'. The graph features a curve with a point labeled 'Tripelpunkt' at the peak. Two points are marked on the curve: point 1 on the left side and point 2 on the right side.

O minus m dot mass flow rate times (h subscript 2 minus h subscript 3) plus W dot.

Tab A 11

hf (8.06 cw, x equals 0) equals 83.42 kilojoules per kilogram equals h subscript 4
therefore T subscript 4 equals 31.33 degrees Celsius

c) x subscript 2 equals 1, T subscript 2 equals negative 22 degrees Celsius

therefore TAB A 10

P subscript 2 equals 1.2192

h subscript g equals 234.08 kilojoules per kilogram
s subscript g equals 0.8351 kilojoules per kilogram

T therefore isothermal therefore P subscript 2 equals P subscript x equals 1.2192

x subscript m equals (s minus s subscript f) over (s subscript g minus s subscript f) equals (s minus s subscript f) over (s subscript g minus s subscript f)