A graph is shown with the y-axis labeled 'T [K]' and the x-axis labeled 'S [kJ/kg K]'. The graph depicts a cycle with points labeled 1 to 6. The curve starts at point 1, moves to point 2, then to point 3, where P1 equals P3, then to point 4, then to point 5, where P4 equals P5, and finally to point 6, where P0 is indicated.

Zero equals h subscript s minus h subscript 6 plus w subscript s minus w subscript 6

equals c subscript p, Luft (T subscript s minus T subscript 6) plus w subscript s minus w subscript c

reversible, adiabatic state change

implies T subscript 6 over T subscript s equals (P subscript 6 over P subscript s) to the power of (one minus one over kappa)

implies T subscript 6 equals (P subscript 0 over P subscript s) to the power of (one minus one over kappa) minus T subscript s equals 328.07 Kelvin

implies h subscript 0 minus h subscript 6 equals negative 85.43 kilojoules per kilogram

implies w subscript 6 equals 510 meters per second

c) Delta e subscript ex, ishr equals (h subscript 6 minus h subscript 0 minus T subscript 0 (s subscript 6 minus s subscript 0) plus Delta ke) greater than zero

equals (c subscript p (T subscript 6 minus T subscript 0) minus T subscript 0 (c subscript p ln(T subscript 6 over T subscript 0) minus R ln(P subscript 6 over P subscript 0)) plus Delta ke)

w subscript 6 squared over two minus w subscript 0 squared over two

approximately 125.97 kilojoules per kilogram

Entropy balance at combustion chamber to calculate

O equals m dot sub K times bracket S two minus S three bracket plus Q dot sub B divided by T sub B plus S dot sub erz