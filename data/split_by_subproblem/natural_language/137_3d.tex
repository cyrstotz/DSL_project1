Calculate with (Q one) equals fifteen hundred joules further.

x one equals zero point six

x two equals (u two minus u s two) over (u g two minus u s two)

T two equals zero degrees Celsius

p two equals one point three eight five bar

u n equals u s n plus x one (u g one minus u s n)

Interpolation in table

u n equals

u two minus u n equals c v (T two minus T n)

u two equals c v (T two minus T n) plus u n

x two equals (u two minus u s two) over (u g two minus u s two)

Always with formula Y equals (Y t minus Y a) over (X t minus X a) times (x minus x a)

Values are drawn, insert

u g two from table interpolation

u g c

u g c equals Y equals negative three hundred fifty-three point six two six kilojoules per kilogram

u g two equals Y equals negative zero point zero six three kilojoules per kilogram

u g two equals Z equals negative zero point zero zero four nine five five kilojoules per kilogram

x one equals one point three nine nine five meters

x one equals one point three nine five meters

x two equals one point nine five meters

x two equals one point nine five zero meters

E ex equals Q dot zl divided by (Q dot ab plus Q dot zl) equals Q dot ul divided by (Q dot ab minus Q dot ul) equals Q dot ul divided by (Q dot minus W).

Q dot equals m dot times (h two minus h one) equals m dot times (h v2 minus h n), where W equals twenty-eight watts.