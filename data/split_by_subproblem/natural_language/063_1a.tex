a)

Solution KF:

O equals m dot KF times (h ein minus h aus) plus Q aus

Q aus equals m dot KF times (h aus minus h ein)

Q um equals m dot KF times c times (T aus minus T ein)

O equals m ein times (h ein minus h aus) plus Q e minus Q aus

Q um equals m ein times (h ein minus h aus) plus Q R

h aus equals h f (100 degrees Celsius) equals 419.04

h e equals h f (70 degrees Celsius) plus x times (h g (70 degrees Celsius) minus h f (70 degrees Celsius)) equals

b)

F minus integral from s e to s a of T d s over s 2 minus s e equals h a minus h e over s a minus s e equals L times (T a minus T e) over L times ln (T a over T e)

Integral from s e to T d s equals q

1HS: q equals h a minus h e

equals 298.15 K minus 288.15 K over ln (298.15 K over 288.15 K)

equals 293.12 K

c) 2.45: 

O equals Q out over T minus Q in over T plus 5 servz

5 servz equals Q out over T minus Q in over T

equals 65 kilowatts over 273.72 Kelvin minus 65 kilowatts over 373.75 Kelvin equals 97.5 O kilograms Kelvin

Diagram: A rectangular box labeled T at the top and Q at the middle, with an arrow pointing upwards labeled Q. The bottom is labeled 1000.

d) 1.45

o U equals m a u a minus m i u i minus o m [h ein] minus Q R12

h ein equals h f (200 degrees) equals 83.96 kilojoules per kilogram

Q R12 equals 35 times 10 to the power of 5

m 2 equals o m plus m ges 1

m n equals m ges 1

o m d 2 plus m ges o U 2 equals o m U h equals o m h ein minus Q R12

o m (u i minus h ein) equals m ges (u 1 minus u 2) minus Q R12

o m equals m ges (u 1 minus u 2) minus Q R12 over u i minus h ein equals 330 kilograms

u 2 equals u f (70 degrees) equals 292.515 kilojoules per kilogram

u 1 equals (u f (100 degrees)) equals 418.9 kilojoules per kilogram

a) o s equals m 2 s 2 minus m n s 1 equals 717.3 kilojoules per kilogram

m 2 equals m ges plus o m equals 3302 kilograms plus 5755 kilograms equals 9056.0 kilograms

m 1 equals 5755 kilograms

s 2 equals s f (70 degrees) equals 0.5958

s 1 equals s f (100 degrees) equals 1.3069