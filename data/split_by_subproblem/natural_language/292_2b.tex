w subscript zero equals question mark, T subscript zero equals question mark.

5 to 6: isentropic implies n equals u, P subscript 6 equals P subscript 0 equals 0.791 bar.

T subscript 6 divided by T subscript 5 equals (P subscript 6 divided by P subscript 5) raised to the power of k minus 1 over k implies T subscript 6 equals T subscript 5 times (P subscript 6 divided by P subscript 5) raised to the power of k minus 1 over k equals 437.9 Kelvin times (0.791 bar divided by 0.5 bar) raised to the power of 7.4 divided by 2.4 equals 328.075 Kelvin.

Delta E equals delta u plus delta KE plus delta P divided by E equals zero.

rho equals negative m times (u subscript i cubed times (T subscript 6) minus u subscript i cubed times (T subscript 5)) plus m subscript i times (w subscript 6 squared minus w subscript 5 squared) divided by 2.

rho equals psi times c subscript v times (T subscript 6 minus T subscript 5) plus one half psi times (w subscript 6 squared minus w subscript 5 squared).

rho equals negative 0.7182 kilojoules per kilogram Kelvin times (328.075 Kelvin minus 437.9 Kelvin) plus one half times psi times (w subscript 6 squared minus 210 squared meters squared per second squared).

One half times (w subscript 6 squared minus 220 squared meters squared per second squared) equals 0.7188 kilojoules per kilogram Kelvin times (328.075 Kelvin minus 437.9 Kelvin).

w subscript zero equals 219.66 meters per second.

c subscript v equals c subscript p minus R; c subscript p divided by k equals c subscript v.

c subscript v equals 0.718 times 57.9 divided by k subscript p.