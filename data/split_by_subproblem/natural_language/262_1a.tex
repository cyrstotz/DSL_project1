Q out = ?

Cooling capacity

Energy balance:

Stationary:

dE/dt = Σṁ [h_i + the + pe] + ΣQ - ΣW

0 = ṁ [h_in - h_aus] + Q dot R - Q dot out

Q dot out = ṁ [h_in - h_aus] + Q dot R =

TAB A-2:

Pure water, boiling liquid → h_f

h_f (70°C) = 292.98 kJ/kg

h_f (100°C) = 419.04 kJ/kg

Q dot R = 100 kW

Q dot out = 0.3 kg [h_f (70) - h_f (100)] + Q dot R

Q out = 62.182 kW

dS over dt equals E times Q over T plus Sez.

Sans minus Sein equals Qaus over T plus Sez.

Sez equals Sans minus Sein plus Qaus over T.

dS over dt equals E times Q over T plus Sez.

Sans minus Sein equals Qaus over T plus Sez.

Stationary Sez equals Sans minus Sein.

Entropy balance: O equals m times Sein minus Sans plus E times Q over T plus Sez from pure vapor.

Sez equals m times Sans minus Sein minus Q over TR plus Qaus over TKF.

TAB A-2

s of 70 equals sf of 70 equals 0.3579 kilojoules per kilogram Kelvin.

s of 100 equals sf of 100 equals 1.3069 kilojoules per kilogram Kelvin.

Sez equals 0.1056 kilojoules per Kelvin second minus 100 over 100 plus 273.15 plus Qaus over TKF.

Sez equals 49.95 watts per Kelvin.

m equals 0.3.

Exiverl

Stationary energy balance:

Zero equals m dot times (h_e minus h_i) plus sum of ex_i,Q minus W_bln minus E_x,verl

E_x,verl equals D_ex,strobl6 plus E_x,Q,ol6 minus W_bln

E_x,verl equals D_ex,strobl6 minus C_bln

Turbine: Aufgabe b

C_bln equals minus n times R times (T_6 minus T_8) over one minus n

D_ex,verl equals 100 kilojoules per kilogram plus 109260.9 joules per kilogram

equals 204.26 kilojoules per kilogram