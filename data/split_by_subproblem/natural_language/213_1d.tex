T2 equals 70 degrees Celsius.

Energiebilanz:
dE over dt equals delta m12 times h ein plus Q out plus Q R minus W to the power of 70.

(m0 plus delta m12) u2 minus m1 u1 equals delta m12 h e.

m12 minus m1 u2 equals delta m12 (h e minus u2).

m1 (u2 minus u1) over (h e minus u2) equals delta m12.

Fortsetzung

u sub 1 equals u sub f plus x times (u sub g minus u sub f) equals 918.94 plus 0.005 times (2506.5 minus 918.94)

equals 429.3778 kilojoules per kilogram

u sub 2 equals u sub f (70 degrees Celsius) equals 292.58 kilojoules per kilogram

h sub e equals h sub f (20 degrees Celsius) equals 83.96 kilojoules per kilogram

implies delta m sub 12 equals 3756.98 kilograms

e)

Delta S sub 12 equals m sub e times (m sub e plus delta m sub 12) times s sub 2 minus m sub 1 times s sub 1

s sub 1 equals 1.3069 kilojoules per kilogram Kelvin

s sub 2 equals 0.9549 kilojoules per kilogram Kelvin

implies delta S sub 12 equals 1561.65 kilojoules per Kelvin