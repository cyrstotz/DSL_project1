Upper section:
- T | V | Phase
- Z1: 0 degrees Celsius, Vn, 0.1 kilograms
- Z2: V2

Equation: x equals m water divided by m EW equals 0.6

Note: -EW omitted.

Lower section:
- T | V1
- Z1: 500 degrees Celsius, 0.0034 cubic meters
- Z2: 0.0039 cubic meters

Calculations:
- Pref equals 1 bar, Cv equals 0.633 liters per kilogram, V equals 50 liters per kilogram

Constants:
- Mass m equals 32 kilograms, Diameter D equals 10 centimeters equals 0.1 meters, Pressure p equals 1 bar

Note: Quiescent

Additional notes:
- 1 liter equals 10 to the power of minus 3 cubic meters
- 3.4 liters equals 0.0034 cubic meters

p subscript v times V subscript A equals m subscript v times R times T subscript 1.

A diagram is drawn with three downward arrows labeled m subscript EW times g, m subscript K times g, and p subscript o times A, all pointing to a horizontal line labeled p subscript A times A.

Equation: A equals pi r squared equals pi times (0.4 divided by 2 times u) squared equals 0.007854 meters squared.

Equation: KGLW: m subscript EW times g plus m subscript K times g plus p subscript o times A equals p subscript 1 times A.

Equation: p subscript 1 equals (0.4 kilograms times 9.85 meters per second squared divided by 0.007854 meters squared) plus (32 kilograms times 9.8 meters per second squared divided by 0.007854 meters squared) plus 10 to the power of 5 pascals equals 1.4 bar.

Crossed out text.

Equation: W subscript g equals (p subscript g times V subscript A3 divided by R times T subscript 1) equals (1.4 times 10 to the power of 5 pascals times 0.00314 meters cubed) divided by (166.25 joules per kilogram times (500 plus 273.15) kelvin).

Equation: R equals Q divided by m equals (8.314 joules per mole kelvin times 10 to the power of 3) divided by (50 kilograms divided by mole) equals 166.25 joules per kilogram kelvin.

Crossed out text.

Equals 3.499 grams.