dE/dt = m dot (h_ein minus h_aus) plus (u_ein pe) plus Q_ab minus Q_aus

Diagram: A rectangle with labels on each side. The left side is labeled "h_ein, t_ein", the right side is labeled "h_aus, t_aus", the top is labeled "Q_ab", and the bottom is labeled "m_ein, t_ein". An arrow labeled "Q_ab" points into the rectangle.

Q_ab = m dot (h_aus minus h_ein)

h_ein = h at (70 degrees Celsius, x equals 0) = h_f at (70 degrees Celsius) = 292.58 kJ/kg

h_aus = h at (100 degrees Celsius, x equals 0) = h_f at (100 degrees Celsius) = 419.94 kJ/kg

Therefore, Q_ab = 0.3 kg/s times (419.94 kJ/kg minus 292.58 kJ/kg) = 37.818 kW

Q_ab = Q_a minus Q_aus, therefore Q_aus = 62.928 kW

A diagram is shown with a rectangular shape and arrows pointing towards it labeled with Q dot A and Q dot B. An arrow pointing outwards from the rectangle is labeled Q dot A equals cooler.

Q dot B equals Q dot A equals 35 megajoules.

dE over dt equals m dot times (h equals water minus h equals warmer) plus Q dot A minus v squared over two.

Delta E equals sum (h equals exit minus h equals entrance) plus Q A, in.

Delta E equals Delta U equals m warmer times U1 minus U2.

h warmer equals h equals warmer equals h equals (70 degrees Celsius) equals 4.1945 kilojoules per kilogram.

h equals h equals (70 degrees Celsius) equals h equals 292.83 kilojoules per kilogram.

U1 equals m equals U1. U1 equals U f (120 degrees Celsius) plus (U g (120 degrees Celsius) minus U f (120 degrees Celsius)) times 0.005 equals 429.3785 kilojoules per kilogram.

Delta U equals negative 785.14 kilojoules.

m equals Delta U minus Q A, in over (h equals exit minus h equals entrance).

d) p times beta times A equals seven point four kilonewton, Q equals fifteen thousand

Delta E over dt equals m times u squared plus Q times u minus v squared

Delta E equals k times m times Q times u

Delta u times m times L equals Q times u equals fifteen thousand over h times g

Delta m times v equals delta m times c times delta t

x over m times v equals delta x minus delta u equals zero point zero zero four implies x equals x times v, x sub e equals zero point nine nine five zero equals one minus x times flowing

Diagrams:
Two diagrams depicting flow processes with labels "up" and "down" for each, and annotations "p times beta" and "p times flowing".