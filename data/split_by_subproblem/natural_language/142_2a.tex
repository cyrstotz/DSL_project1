A diagram is drawn with points labeled 1, 2, 3, 4, 5, and 6. Lines are connecting these points with arrows indicating direction. The diagram is labeled with various pressures and temperatures: p1, p2, p3, p4, p5, p6, T1, T2, T3, T4, T5, T6. The x-axis is labeled with 'in kJ' and the y-axis with 'in kJ/kg K'.

Below the diagram, the following text and equations are visible:

b) w specific equals 200 cubic meters per second, w6 equals question mark.

p5 equals 0.5 bar, T5 equals 437.0 Kelvin, p2 equals p20 minus 0.107 bar, T0 equals negative 30 degrees Celsius, n equals 1.4.

Theta over T5 equals (p6 over p5) to the power of (k minus 1 over n).

T6 equals T5 times (p6 over p5) to the power of (0.4 over 1.4) equals 328.07 Kelvin.

Adiabatic reversible, dS equals dQ.

Q equals m dot times (q0 minus qa) plus (w0 squared minus w2 squared over 2).

Adiabatic minus w2 squared equals (th minus tg) minus w0 squared.

Minus Vax equals square root of (2 times (cp times (T6 minus T5) times w0 squared)).

Vax equals square root of (cp times (T6 minus T5) times w6 squared).

w6 equals w5 equals 220 cubic meters per second.

Equals 507.25 meters per second.

h equals integral from T6 to T5 of Cp specific heat times dT.

h equals Cp times (T6 minus T5).

Cp specific heat equals 1.006 kJ per kg K.

q one equals q one equals five hundred degrees Celsius

V one equals three point one pi L

c L equals zero point six three three

d one equals d m over d p K

H equals fifty d p K wol

q one equals p one plus m K over pi d one squared plus m W over pi d two squared equals one over alpha der

phi equals V over m equals R T over V

m equals R T over V

M F equals V over R T equals

d q equals zero

one thousand g equals alpha Q equals alpha Q over (u two minus u one) times m

u two equals alpha Q over m plus u one equals two hundred thirty nine point six three four

u equals zero point seven one three equals u two minus u FL over u F minus u FL

u one equals two hundred thirty eight point five three equals fifty plus twelve

u T equals u FL plus s times (u F minus u FL) equals two hundred point zero nine two eight