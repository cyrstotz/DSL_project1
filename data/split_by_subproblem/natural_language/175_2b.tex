0 equals m dot times the sum of h sub o minus h sub 6 plus the quantity of (w sub e squared minus w sub a squared) divided by 2 times rho sub o plus Q dot j minus W dot t n  

Arrow pointing to 'nach aussen adiabatic', implying Q dot j equals 0  

W dot t n equals m times the quantity of h sub 5 minus h sub 6 plus (w sub e squared minus w sub a squared) divided by 2  

Bracketed expression: (W t n divided by m plus h sub a minus h sub f) times 2 equals w sub e squared minus w sub a squared  

w sub o squared equals w sub 5 squared minus W dot t n divided by m dot plus (h sub g minus h sub 6)  

v sub 5 equals m times R T sub 5 divided by P sub 5 equals 0.1286 joules per kilogram times 431.15 Kelvin divided by 0.5 times 10 to the power of 5 Pascals equals 8.314 joules per mole divided by 28.97 kilograms per kilomole equals 0.12567 joules per kilogram Kelvin equals 0.10024789 cubic meters per gram  

T sub 6 divided by T sub 5 equals (P sub 6 divided by P sub 5) to the power of (1.4 minus 1 divided by 1.4) implies T sub 6 equals T sub 5 times (P sub 6 divided by P sub 5) equals 328.107 Kelvin  

h sub 6 minus h sub 5 equals c sub p times (T sub 6 minus T sub 5)  

h sub 5 minus h sub x equals c sub p times (T sub 5 minus T sub 6) equals 1.006 kilojoules per kilogram times (431.13 Kelvin minus 328.107 Kelvin) equals 104.43 kilojoules per kilogram