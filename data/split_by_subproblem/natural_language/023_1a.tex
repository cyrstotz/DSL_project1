Q dot average equals question mark.

[Unreadable text] plus Q dot average.

[Unreadable text] (enthalpy).

In the boiler:

m dot d equals m gas times a times cosine x equals 28.775 kilograms.

m f equals m gas times (1 minus lambda) equals 5726.23 kilograms.

h in rein minus h out rein equals c kappa (T rein minus T out rein).

S ein equals Sf (78 degrees Celsius).

S aus equals Sf (100 degrees Celsius).

Integral T ds equals h out minus h ein equals Q dot over m dot equals 216.6.

c) S dot ex equals question mark.

A diagram with arrows labeled T R Feu, q dot, T R Feu, and m dot. T dot 2 equals 393.15 K. T dot 1 equals 373.15 K.

A formula: sigma equals m dot bracket S 2 minus S 1 bracket equals Q dot over T w plus S dot ex.

Another formula: Q dot in over T R Feu minus Q dot over T w equals S dot ex.

T w equals 100 degrees Celsius.

Equals 0.496 kilojoules per kilogram Kelvin.

A2: S of 70 degrees Celsius: s f of 70 degrees equals 0.9589 kilojoules per kilogram Kelvin. s g of 70 degrees equals 7.7558 kilojoules per kilogram Kelvin.

S of 100 degrees Celsius: s f of 100 degrees Celsius equals 1.3069 kilojoules per kilogram Kelvin. s g of 100 degrees Celsius equals 7.3599 kilojoules per kilogram Kelvin.

Q of 100 degrees Celsius.

S equals s f plus x times bracket s g minus s f bracket equals 1.33749 kilojoules per kilogram Kelvin.

d) T m equals 100 degrees Celsius equals 373.15 K. T 2 equals 393.15 K. Delta m equals question mark. T w equals 20 degrees Celsius equals 293.15 K. Q dot equals 35101.1 kilojoules.

Siedeliche Fließung equals lambda equals Q.

m 1 equals 575 kilograms.

Delta E equals m 2 u 2 minus m 1 u 1 equals delta m bracket h 2 bracket plus Q dot 12. m 2 equals bracket m 1 plus delta m bracket.

Bracket h 1 plus delta m bracket u f of 70 degrees Celsius minus m 1 bracket u f of 100 degrees Celsius bracket equals delta m bracket h f of 20 degrees Celsius bracket plus Q dot 12.

A2: u f of 70 degrees Celsius equals 292.95 kilojoules per kilogram. u f of 100 degrees Celsius equals 418.94 kilojoules per kilogram. u f of 20 degrees Celsius equals 83.96 kilojoules per kilogram.

Delta m equals bracket m 1 u f of 70 degrees Celsius minus m 1 u f of 100 degrees Celsius bracket minus Q dot 12 over u f of 20 degrees Celsius minus u f of 100 degrees Celsius equals 3637.9 kilograms.

e)

Delta S one two equals ?

Delta S one two equals m s two minus m s one equals Q over T plus S ex L

S ex L equals

s two equals s f ( 7.2 plus c ) equals 0.589 kilojoules per kilogram kelvin

s one equals s g ( 10.8 c ) equals 1.3069

a) Two graphs are drawn. The first graph has axes labeled 'p' and 'T' with a wavy line and a curve. The second graph has axes labeled 'p' and 'T' with lines labeled 'Gas, fest', 'flüssig', 'flüssig', and 'Gas, fest'. A point 'T' is marked on the line.

T1 equals P1 equals 1 mbar

T1 equals -10 degrees Celsius

Temperature at the boiling point equals -16 degrees Celsius

b) Energy balance 1 to 2

Q equals m dot times (h4 minus h2) plus Qk

h2 equals hfg at -22 equals 234.8 kilojoules per kilogram

Energy balance 2 to 3

Q equals m dot times (h2 minus h3) minus W dot k

c) h1 equals h4

h4 equals hf at 8 bar equals 93.42 kilojoules per kilogram

hf( ... )

hg( ... )

x equals (h4 minus hf) over (hg minus hf)