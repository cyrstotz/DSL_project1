A horizontal line with a circle and arrows is drawn above the equation: 
m gas equals m gas times (h six minus h five) minus w s.
M L equals 28.07 grams per mole.
m gas equals w L times v.
The equation s six equals s six is written. Then, s six minus s zero equals c p times ln (T six over T zero) minus R over M L times ln (p six over p zero).
An arrow points to the equation: 
ln (T six over T zero) minus R over M L c p times ln (p six over p zero) equals T zero over T six minus p zero over p six.
Another equation: 
T six over T s equals (p six over p five) to the power of k over k.
An arrow points to T six equals 328.07 Kelvin.
Theta equals m gas times (h six minus h five) minus (s five minus s six) minus (w five squared minus w six squared) over two.
w five squared over two equals h five over s five plus w five squared over two.
w six equals 507.24 meters per second.
Additional, reversible, and isentropic plus w five equals zero, Q five equals zero.