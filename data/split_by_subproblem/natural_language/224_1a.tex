Three values are obtained from the tasks such as Q two s, etc.

Theta equals m w ein times h w ein minus h w two s plus Q two s.

h w ein at 70 degrees Celsius equals 292.04 kilojoules per kilogram.

h w two s at 100 degrees Celsius equals 449.04 kilojoules per kilogram.

Q two s equals 62.482 kilowatts.

Theta equals m two times in brackets h two minus h one.

Isobar to h two equals h f. Bracket 8 bar equals 93.42 kilojoules per kilogram.

h two equals h f plus x times in brackets h g minus h f.

p two equals p one. Minus 22 degrees Celsius equals h g. Minus 22 degrees Celsius equals 234.08 kilojoules per kilogram.

h f equals 25.77 minus 9.32 divided by 1.4 minus 1.2.

1.21926 times 9.40 equals 1.2192 minus 1.2 plus 9.32 equals 21.472 kilojoules per kilogram.

h g equals 236.04 minus 233.86 divided by 1.4 minus 1.2.

1.2192 minus 1.2 plus 233.86 minus 234.09 kilojoules per kilogram.

x two equals 0.358.