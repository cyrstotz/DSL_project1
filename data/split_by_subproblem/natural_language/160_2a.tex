Cp equals one point zero zero six kilojoules per kilogram Kelvin  
n equals k equals one point four  
PE equals zero  

| p (bar) | T [°C] | s |  
|---------|--------|---|  
| 0       | 0.101  | -30°       |  
| 1       | p1 greater than p0 | T1 greater than T0 |  
| 2       | s1 equals s2 |  
| 3       | p3 equals p2 |  
| 4       | s3 less than s4 |  
| 5       | p4 equals p5 equals 0.5 | 4.37 kilojoules per kilogram Kelvin |  
| 6       | s5 equals s6 |  

a)  

Diagram:  
- The vertical axis is labeled as T [°C].  
- The horizontal axis is labeled as s [kJ/kg K].  
- Points labeled 0, 1, 2, 3, 4, 5 are connected by lines.  
- Point 0 is at the origin.  
- The curve from point 0 to point 1 is labeled Isotherm.  
- The curve from point 1 to point 3 is labeled Isobar.  
- The curve from point 3 to point 4 is labeled Isentrop.  
- The curve from point 4 to point 5 is labeled Isentrop, with an annotation of 0.5 bar.  
- The curve from point 5 back to point 0 is labeled Isotherm.

ex, verl equals T zero times s dot erz

zero equals in times s zero minus s one plus Q dot over T plus s dot erz

Adiabates Triebwerk

s dot erz equals s one minus s zero plus Q dot over T

s dot erz equals Cp times ln (T c over T zero) minus R times ln (P c over P zero)

ex, verl equals T zero times Cp times ln (T c over T zero)

T zero equals 243.15 K

ex, verl equals 243.15 K times 1.00 kJ over kg K times ln (34.0 K over 243.15 K) equals 82.009 kJ over kg