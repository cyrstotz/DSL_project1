weiter

Q̇aus = ṁ (h2 - h2) + Q̇R
= 0.3 (-125.67) kJ/kg + 100 kJ/s ⇒ Q̇aus = 62.29 kJ/s

T [W] | P [kW] | x  
1 | 343.14 | 0.005  
2 | 373.14 | 0.005  
3 | 288.15 | P13  
4 | 298.15 | P23  

Diagram:  
A rectangular box labeled "clam" with an arrow labeled "Qaus" pointing outwards. Inside the box, there is an arrow labeled "1" with "Qin" pointing towards the box, and an arrow labeled "2" pointing out of the box. Below the box, there is an arrow labeled "3" pointing left and an arrow labeled "4" pointing right with "mp". The text "m dot = const = 0.3 kg/s" and "nges = 577.5 (stationär)" is written near the diagram.

a) Qaus with energy balance I  
0 = m dot (h1 - h2) + Q2 = m dot r^0 - Qaus  

(hn - h2) wasser = h1 (T = 343.14) - h2 (T = 373.14)  

Dampftafel x1 = x2  
h2 = h1 + x1 (hg1 - hf1) hat Tn  

h2 = 304.649  
h2 = h12 + x2 (hg2 - h2)  
h2 = 430.82  

=> Seite 3  

Th1A = 292.98 h12 = 419.04  
hg1 = 262.68 hg2 = 2676 j m = x1 - x2  
e/kg  
m dot

a) E c zu Q punkt zu durch W r gleich Betrag Q punkt durch Betrag Q punkt minus W punkt

c) die Temperatur würde die bei das gleiche Temp wie das Kühlmedium im Gleichgewicht sein. (T gleich T i minus 6) Kalt kann wegen das Z H nicht zu warm übertragen werden. 

3 E S