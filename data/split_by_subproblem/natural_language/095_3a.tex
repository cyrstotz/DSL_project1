T gas equals five hundred degrees Celsius  
V gas x equals three comma one L  
g equals thirty-two comma nine eight one meters per second squared equals three hundred forty-three point five two N  

P gas equals question mark  
m gas equals question mark  

A equals pi times r squared, r equals D divided by two equals zero point zero six five meters  
A equals seven point eight five three times ten to the power of negative three meters squared  

p durch Gewicht equals three hundred forty-three point five two N divided by seven point eight five three times ten to the power of negative three meters squared equals zero point zero four four bar  

GSW, in EW Kammer, one point four bar, so muss in Gastkammer auch one point four bar  

P gas equals one point four bar  

m gas equals P gas times V gas divided by R gas times T gas  

m gas equals one point four times ten to the power of five pascal times three point one times ten to the power of negative three meters cubed divided by eight point three one four joules per mole Kelvin divided by seven hundred seventy-three point five Kelvin  

m gas equals three point four one five times ten to the power of negative three kilograms equals three point four one five grams