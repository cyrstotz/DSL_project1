T divided by T reference equals 295 Kelvin

O equals m dot (S exit minus S in) plus Q dot out divided by T reference m and S exit

S exit equals m dot (S out minus S in) plus Q dot out divided by T reference

S exit equals S f (70 degrees Celsius) equals 0.85 kilogram (kilo Joule per kilogram Kelvin)

S sum equals S f (100 degrees Celsius) equals 1.306 kilogram (kilo Joule per kilogram Kelvin)

S dot exit equals 0.3 kilogram per second (1.306 kilogram kilo Joule per kilogram Kelvin minus 0.85 kilogram kilo Joule per kilogram Kelvin) plus 62.182 kilo Joule per second squared divided by 295 Kelvin

equals 0.1316 kilo Joule per Kelvin

T sub 2 equals negative 22 degrees Celsius  
T sub 1 equals negative 16 degrees Celsius

Continuation Task

Delta m12

m2 u2 minus m1 u1 equals Delta m12 h12 plus Q12 minus Yo

m1 equals 5755 kg

m2 equals 5755 kg plus Delta m12

u2 equals u2f at 30 degrees Celsius equals 292.95 kilojoules per kilogram TAB A2

un equals uf plus xD times ug minus uf

equals bracket 418.94 plus 0.005 times bracket 2506.5 minus 418.94 closed bracket kilojoules per kilogram

equals 429.3378 kilojoules per kilogram

h12 equals hf at 20 degrees Celsius equals 83.96 kilojoules per kilogram TAB A2

(m1 plus Delta m12) u2 minus m1 u1 equals Delta m12 h12 plus Q12

Delta m12 equals m1 u2 minus m1 u1 minus Q12 divided by h12 minus u2

equals 3538.57 kilograms