A second graph is drawn with the y-axis labeled T (k) and the x-axis labeled S. This graph contains a curve with points labeled 1, 2, 3, 4, 5, and 6. The curve starts at point 1, goes through point 2, reaches point 3, moves to point 4, continues to point 5, and ends at point 6. There are lines connecting these points, with a line labeled P intersecting the curve between points 4 and 5. It is noted that n2 equals n3 and n5 equals n6.

Energy balance in the system:

O equals m times (h0 minus h6 plus u6 squared minus u0 squared divided by 2).

Two times (h6 minus h2) equals u2 squared minus u0 squared.

u6 minus h0 equals vu times uL minus two times (h6 minus h2).

(h0 minus h0) equals cp times (T6 minus T0).

T6 divided by T5 equals (P6 divided by P5) to the power of (k minus 1 divided by k) equals 328.07k.