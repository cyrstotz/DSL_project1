dE over dt equals m dot in times h in of t plus Q dot R minus Q dot aus minus k ins h aus of t equals zero.

Q dot aus equals Q dot R plus v dot ein times (h ein minus h aus), which is represented as delta h.

Equals seven hundred thousand L plus zero point three k s over three times delta h.

A diagram is drawn with a coordinate system labeled with 'P' on the vertical axis and 'T' on the horizontal axis. Three points are marked on the graph labeled as '1', '2', and '3'. There are curves connecting these points: a curve connects point '1' to point '2', and another curve connects point '2' to point '3'.