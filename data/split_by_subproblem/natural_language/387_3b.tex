due to length because external pressure must be the same

c) Q one two

see last calculation Q one two equals

delta U equals Q minus W

V one two equals R times T two minus T one divided by one minus n

n equals c p divided by c v equals one point two eight six equals six

c p equals R plus c v equals seven nine nine Joules divided by kilogram Kelvin squared

equals minus three one six point five Joules divided by kilogram

W one two equals minus one thousand eighty two Joules

delta U equals c v times T two minus T one times m equals eight six six point three Joules divided by kilogram

Q one two equals one thousand nine hundred forty five Joules

d) X ice two:

minus m e w zero point six U fresh R two plus m e w times X ice two U fresh R two

plus m m e w one minus X ice two U fossil R two minus Q one two times m e w U fossil R two

equals Q one two minus arrow zero