Delta S subscript 1 to 2 equals m subscript 2 S subscript 2 minus m subscript 1 S subscript 1 equals [illegible] 64.85 MJ/K

m subscript 1 equals m subscript 1, ges equals 57.55 kg

m subscript 2 equals Delta m subscript 2 plus m subscript ges equals 3.55 kg

S subscript 1 equals S subscript F (100 degrees Celsius) plus x subscript D (S subscript g (100 degrees Celsius) minus S subscript F) equals 1.33 kJ/kgK

TAB A2

S subscript F (100 degrees Celsius) equals 1.3069 kJ/kgK

S subscript g (100 degrees Celsius) equals 7.3543 kJ/kgK

S subscript 2 equals S subscript g (70 degrees Celsius) equals 7.7553 kJ/kgK

A graph is drawn with the y-axis labeled as T [K] and the x-axis labeled as S [h2 over hGK]. The graph contains six points labeled from 0 to 6. The points are connected with lines, and there are annotations next to some lines: "isobare bei p3 = p2," "isobare bei p4 = p5," and "isobare bei p1 = p6."