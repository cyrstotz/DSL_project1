a) Q equals m dot times (h1 minus h2) plus QR equals Qaus minus W dot.

h1, h2 equals TAB A2 saturated liquid. hf at 70 degrees Celsius equals 292.58 kilojoules per kilogram equals h1. hf at 100 degrees Celsius equals 419.09 kilojoules per kilogram equals h2.

W dot equals integral from 1 to 2 of v dp equals integral from 1 to 2 of v dp in. v equals v at 100 degrees Celsius plus v at 100 degrees Celsius over 2 equals 1.0333 cubic meters per kilogram.

equals minus v times 0.0333 times (p2 minus p1) times m dot equals minus 0.2153 cubic meters per second times 1000 kilograms per cubic meter equals minus 215.3 kilojoules.

Qaus equals m dot times (h1 minus h2) plus QR minus W dot equals 83.68 kilojoules.

b) Qaus rein implies reversed implies S dot equals 0.

Q equals m dot times (s1 minus s2) plus Qaus over T implies T bar dot times m dot times (s2 minus s1) plus Qaus.

Q equals m dot times (hEF minus hKF) plus Qaus.

minus Qaus over (hEF minus hKF) equals mUF minus Qaus over e to the power of i times (T1 minus T2).

T bar equals 1 over e to the power of i times (Qaus over e to the power of i times (T1 over T2 over T1)) times Qaus equals (T1 minus T2) over ln (T2 over T1) equals 293.1211.

c) Q equals Qaus over T reactor minus Qaus over T EF implies Entropy balance can be used.

S dot equals Qaus over T EF minus Qaus over T reactor equals 654 kilojoules over 285 Kelvin minus 654 kilojoules over 373.15 Kelvin equals 0.046615 kilojoules per Kelvin.

Results from testing used.

U sub 2 equals U sub test plus X sub zeiss multiplied by (U sub ausgang minus U sub test)