O equals m times (he minus ha) plus the sum of Q minus the sum of W (zero with a dot on top).
O equals m dot (he minus ha) plus Qe plus Qaus.
Qaus equals m dot (hu minus he) minus QR.
ha equals h2 equals hf (100 degrees Celsius) equals 419.04 kilojoules per kilogram.
he equals ha equals hf (70 degrees Celsius) equals 292.97 kilojoules per kilogram.
Qaus equals m dot (h2 minus h1) minus 10 kilowatts equals minus 621.82 kilowatts.
Since Qaus as

a) A graph is drawn with a curve labeled "gauförmig" and lines labeled "fest" and "flüssig". The y-axis is labeled "p" and the x-axis is labeled "T". Points on the graph are marked with numbers 1, 2, 3, and 4.

b) p2 equals 3.32 bar. T1 equals -16 degrees Celsius. pa equals 1.5748. A10 is noted. T2 equals -16 degrees Celsius. 

1. HS: m times (ha minus he) equals Q minus W. 
h2 equals h4 equals hg (kur). 
he equals hg (-16 degrees Celsius) equals 253.74 kilojoules per kilogram. 
W dot equals m dot times (h1 minus he). 
m dot equals W dot divided by (h1 minus he) equals 0.028 kilowatts divided by 1.0002 times 10 to the power of 3.

c) 1. HS: m dot times (ha minus he) equals W.

m dot equals 1.1 times (h1 minus he) equals m dot times (h2 minus ha) equals W dot. 
h2 equals ha. 
hg (1 bar) equals ha equals 43.24 kilojoules per kilogram. 
X1 equals hg (1 bar) minus hg (-16 degrees Celsius) divided by hg (1 bar) minus hg (-16 degrees Celsius) equals 0.3076 equals 30.76 percent.

d) 1. HS: m times (he minus ha) plus Q minus W equals 0.

Ek equals Q zu divided by (Q kalt minus Q zu). 
Q zu equals Q k equals m times (ha minus he) equals m times (hg (1 degree Celsius) minus hg (-16 degrees Celsius)) equals 4 kilograms times hg (-16 degrees Celsius) equals 533.8 kilojoules per kilogram.
Q gas equals m times (h1 minus he) equals m times hg (1 bar). 
Q gas equals 738.96 kilojoules per kilogram per hour. 
Ek equals 8.742.

u2 equals u4 (70 degrees Celsius) equals 258.29215 kilojoules per kilogram. A2

u1 equals u2 (1000 degrees Celsius) plus x (ug1 (4000 degrees Celsius) minus ug1 (1000 degrees Celsius)) A2

equals 421.38 kilojoules per kilogram

h2 equals hg (20 degrees Celsius) equals 38.26 kilojoules per kilogram. A2

Delta m equals 2290.88 kilograms

e) Closed system:

Delta r2 reg equals m2 s2 minus m1 s1

equals (Delta m2 plus m1) s2 minus m1 s1

s2 equals sg2 (70 degrees Celsius) equals 0.2544 kilojoules per kilogram Kelvin A2

s1 equals sg1 (100 degrees Celsius) equals 1.3069 kilojoules per kilogram Kelvin A2

Delta s12 equals (3600 kilograms plus 5755 kilograms) s2 minus 5755 kilograms s1

equals 1412 kilojoules per Kelvin

Cylinder has an area A equals pi times diameter divided by two squared equals seven point eight five four times ten to the power of minus three square meters. P one comma zero equals mRT divided by V. We do not know m and p. PV equals mRT.