T
ein sieden

m dot equals 0.3 kilograms per second  
Tein equals 70 degrees Celsius arrow sieden fluessig  
mges 1 equals 5.755 kilograms  
x0 equals 0.005  
T equals 100 degrees Celsius immer

Half-open system

Delta E equals m sub 2 u sub 2 minus m sub 1 u sub 1 equals delta m sub 12 u sub 12 plus Q minus W. v sub 2 equals m sub 1 plus m sub 2, m sub 1 equals 5.755 kilograms, minus 35000 kilojoules.

h sub 12 at 20 degrees Celsius equals u sub f at 20 degrees Celsius equals less than or equal to 3.96 kilojoules per kilogram.

u sub 2 at 70 degrees Celsius equals u sub f at 70 degrees Celsius equals 232.95 kilojoules per kilogram.

u sub 1 at 100 degrees Celsius equals 84 plus 415.94 plus 0.005 times (2506.5 minus 415.94) equals 429.377 kilojoules per kilogram.

Delta m sub 12 (u sub 2 minus h sub 12) plus m (u sub 2 minus u sub 1) equals Q.

Delta m sub 12 equals Q minus m sub 1 (u sub 2 minus u sub 1) over (u sub 2 minus u sub 12).

Delta m sub 2 equals 3589.346 grams.

A graph is drawn with axes labeled "T" on the horizontal axis and "P" on the vertical axis. The graph shows a curve labeled with points "1" (Fast), "2" (Gas), "Triple" (point), and "4" (Flüssig). 

Another graph is drawn with "P" on the vertical axis and "T" on the horizontal axis. 

Equations and notes include:
- T_i = 10 K
- T = 4 K (with a note "Ueberh.")

A table is drawn with columns labeled "T" and "P":
1. P1 = P2, labeled "flüssig"
2. Labeled "dampf"
3. s2 = s3, labeled "reversibel adiabt"
4. x4 = 0, labeled "flüssig"

Equation:
Q̇ = ṁ_R (h2 - h3)
Σ W_R = 0,258 kW

h2 (x2 = 1, s3)

h3 = 264,15 kJ/kg

h4 = h1 = (u4 @ 8 bar) 83,42 kJ/kg

x1 = (h1 - h_f (P1)) / (h_g - h_f)

J bar equals the integral from se to sa of Tds divided by (sa minus se) equals Tds equals Q divided by m dot equals T2 minus T1 equals cv ln (T2 divided by Td). Ideal flüssigkeit.

W sub c equals square root of two times m dot g times (c sub p times (T sub 3 minus T sub 0) plus omega sub 5 squared divided by two).

m dot g equals (scratched out text) m dot k plus m dot m equals m dot k times (one plus five point two five three).

O equals m dot k times (h sub 2 minus h sub 3) plus m dot k times q sub b.

minus q sub b equals (scratched out text) c sub p times (T sub 2 minus T sub 3) leads to solve T sub 2.

Turbine leads to O equals m dot k times c sub p times (T sub 1 minus T sub 2) plus Q sub el over U equals W sub t over k sub w.