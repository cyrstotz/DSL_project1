a) p sub gm minus m sub g

T sub gm equals five hundred degrees Celsius equals seven hundred seventy three point one five Kelvin

V sub gm equals three point one four two equals zero point zero zero three one four cubed

m sub g equals zero point one kilograms

p sub gm equals p sub out plus A

p sub out equals one bar equals one hundred kilopascals

A equals x squared pi equals five squared pi equals twenty five pi centimeters squared equals eighty one point five four centimeters squared

p sub out equals one hundred kilopascals plus thirty two kilograms per second times five point twenty one squared pi equals zero point zero zero seven eight five four squared equals zero point zero zero seven eight five four squared

three hundred thirty five point six five divided by m sub fifty equals three hundred thirty five point six five pascals

p sub out equals one hundred plus three hundred thirty five point six five five four four pascals

p sub out equals one hundred thirty five point nine five six five pascals

b) p sub sin

p sub sin equals one hundred thirty five point nine five six five pascals plus zero point one kilograms times five pi divided by zero point zero zero seven eight five four squared

p sub sin equals one hundred forty kilopascals

m sub g equals p V divided by R T

m sub g equals one hundred forty kilopascals times zero point zero zero three one four meters cubed divided by zero point four six one four kilograms per kilogram Kelvin times seven hundred seventy three point one five Kelvin

m sub g equals zero point zero zero one two three two kilograms

R equals eight point three one four liter times bar divided by mole Kelvin divided by eighteen point zero two kilograms per mole

R equals zero point four six one four kilojoules per kilogram Kelvin