One to two, Q ab. T one equals seventy degrees Celsius, T two equals one hundred degrees Celsius.

Zero equals m dot (h two minus h one) plus Q ab. Therefore, Q ab equals m dot (h two minus h one) equals Q aus plus Q e remains in the system, equals zero.

h two equals: only the liquid part goes out. h two (T equals one hundred degrees Celsius) equals four hundred forty-one point zero four kilojoules per kilogram.

h one: h one (T equals seventy degrees Celsius) equals two hundred ninety-two point eighty-eight kilojoules per kilogram.

Therefore, Q ab equals m dot (h two minus h one) equals Q e dot equals zero point three kilograms per second (four hundred forty-one point zero four minus two hundred ninety-two point eighty-eight) kilojoules per kilogram equals one hundred seven point eight eighteen kilowatts.

dE over dt equals plus Q dot implies delta V one two equals Q aus one two.

Delta m two (u two E u one E) plus m ges (u two gas minus u one gas) equals minus thirty-five megajoules.

Delta m two c (T two minus T ein one two) plus m ges c (T two minus T reactor one) equals minus thirty-five megajoules.

Delta m two c (T two minus T ein one two) equals minus thirty-five megajoules minus m ges c (T two minus T reactor one).

Delta m two equals minus thirty-five megajoules over c minus m ges over c (T two minus T reactor one) over T two minus T ein one two minus m ges (u two gas minus u one gas) over (u two k minus u one k).

u reactor one equals four hundred eighteen point ninety-four kilojoules per kilogram plus zero point zero zero five (two thousand five hundred six point five minus four hundred eighteen point ninety-four) kilojoules per kilogram equals four hundred twenty-three point thirty-eight kilojoules per kilogram.

u k one equals eighty-three point fifty-five kilojoules per kilogram.

u two equals