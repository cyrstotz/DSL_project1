Epsilon equals Q twenty divided by Q delta B equals Q twenty divided by V times H equals four point two nine nine.

Q twenty equals Q k equals m times h two minus h one equals approximately one hundred twenty point three six.

h one equals h two equals nine hundred thirty-four point two kilojoules per kilogram.

h two equals h two at negative sixty degrees Celsius equals two hundred thirty-seven point seven zero kilojoules per kilogram.

Table delta equals ten.

A graph is drawn with axes labeled. The y-axis has labels "h1", "h2", "h3", "h4", and "h5". The x-axis is labeled with temperatures: "-50°C", "-40°C", "-30°C", "-20°C", "-10°C", "0°C", and "T". A curve is drawn starting from the y-axis, going upwards and then flattening out. There is a point marked on the curve with "10K" and another point marked with "2a". An arrow indicates a direction from the curve to a point labeled "T_triple" and "5 mbar".