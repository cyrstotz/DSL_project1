c) Graph with axes labeled T in degrees Celsius and S in kilojoules per log-kilogram. The graph includes points labeled 0, 1, 2, 3, 4, 5, 6, and lines labeled isobar, isotherm, and isentrop. There are dashed lines indicating various transitions between the points.

b) Omega sub c equals 2. T sub 6 equals ?. S sub 5 equals S sub 6. Eta equals kappa equals gamma sub 4 isentrop. T sub 6 over T sub 5 equals P sub 6 over P sub 5 raised to the power of gamma minus n over n. T sub 6 equals T sub 5 minus P sub 0 over P sub 5 raised to the power of 0.4 over gamma sub 4 equals 328.076697 degrees Celsius.

Equation: q dot equals m dot g times h sub 5 minus h sub 6 plus omega sub c squared over 2 minus omega sub 6 squared over 2 plus q dot sub 56 minus q dot sub 56 equals zero.

Equation: q equals h sub 5 minus h sub 6 plus omega sub c squared over 2 minus omega sub 6 squared over 2 plus q sub 56 minus q sub 56 equals zero.

Equation: omega sub c squared over 2 equals h sub 5 minus h sub 6 plus omega sub c squared over 2 equals c sub p times T sub 5 minus T sub 6 plus omega sub c squared over 2.

Equation: omega sub 6 equals square root of 2 times c sub p times T sub 5 minus T sub 6 plus omega sub c squared equals 507.294 meters per second.