The equations are completely taken over with E one minus W one.

m E equals left parenthesis u two minus u one right parenthesis equals Q twelve

u two minus u one equals Q twelve over m E implies u two equals Q twelve over m E plus u one

with u one left parenthesis plus one right parenthesis equals x e two star u Fe plus left parenthesis one minus x e two star right parenthesis u Fe

equals negative two hundred comma one left parenthesis kJ over kg right parenthesis

Since T does not change, the process is isobaric.

This gives us u two equals x e two comma one u Fe plus left parenthesis one minus x e two comma one right parenthesis u Fe

u two equals two thousand five hundred kJ over kg plus u one equals crossed out equals negative one hundred eighty three comma one left parenthesis kJ over kg right parenthesis

x e two comma one equals left parenthesis u two minus u Fe right parenthesis over left parenthesis u Fe minus u Fe right parenthesis equals zero comma five five five

d) Sicherer Feuchtigkeit x2 = 0

Heizleistung System:

m2 u2 - m1 u1 = Delta m u + Qaus 12

u1 (20°C, x = 0) = 83.56 kilojoules per kilogram

m1 = 5375 kilograms

m2 = m1 + Delta m

[TAB-12]  
u2 (x = 0, 70°C) = 252.55 kilojoules per kilogram

u1 (x = 0.005, 100°C) = crossed out x (u1, g) + u1, f (1-x)

u1, f = 416.54  
crossed out = u2 = 42.8.3772 kilojoules per kilogram

u1, g = 2506.51

-Delta m u1 + (m1 u1 - u2 = Qaus 12 + m1 u2 - m1 u2

=> Delta m = Qaus 12 + m1 u2 - m1 u2  
u2 - u1

Delta m = 3924.311 kilograms

e)

Qzu 12 = 30000

Delta S 12 = m2 s2 - m1 s1 crossed out

[TAB-12]  
s2 (70°C, x = 0) = 0.5545 kilojoules per kilogram kelvin

s1 (100°C, x = 200) = crossed out 1.33774

s1, f = 1, 3065 kilojoules per kilogram kelvin

s1, g = 7, 3545 kilojoules per kilogram kelvin

u2 = u1 + Qzu 12 = 5355 kilograms

Delta S = 1237 kilojoules per kelvin

a) Q dot out is the sum of Q dot R equals 100 kilowatts and Q dot W from the SMO.

Continuity:
d m dot over d t equals 0. 

m dot times (h e minus h a) plus Q dot W plus Q dot out equals 0.

Q dot W equals m dot e (h e minus q e).

Given: saturated steam, pressure equals pressure a equals 0 equals pressure a.

Table 4:
h a equals 252.58 kilojoules per kilogram.
h e equals 417.04 kilojoules per kilogram.
Q dot W equals 37.878 kilowatts per kilogram (as seen above, because the corresponding line is missing).

Q dot sum equals Q dot R minus Q dot W equals 62.122 kilowatts.

b) Q dot sum equals 65 kilowatts.

T KF equals epsilon C times T DS equals 4 over S a minus S e minus h a minus h KF over S a minus S e.

With:
d L over d t equals m dot (h KF minus h Fa) plus Q dot sum plus W dot.

Q dot sum equals (h Fa minus h KF) times m dot KF.

Isobare:
Isotherm Recirculation:
h Fa minus h KF equals c i* (T 2 minus T 1) plus v i* (p 2 minus p 1).

S a minus S e equals c i* times ln (T 2 over T 1).

T KF equals ln (T 2 over T 1) equals 253.121 Kelvin.

c) Draw:
Q dot sum System

Entropiesraum?

Q dot sum equals 65 kilowatts.
T EF equals 255 Kelvin and T W equals 100 degrees Celsius plus 273.15 Kelvin.

Closed system (adiabatic envelope):
d S over d t equals Q dot sum over T EF plus Q dot sum over T W plus S dot F.

S dot EF equals Q dot sum times (1 over T W plus 1 over T KF).

S dot EF equals plus 0.04615 kilowatts per Kelvin.