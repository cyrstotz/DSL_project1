P times V equals m times R times T.  

R equals R over M times g equals eight point three one four kilojoules per kilogram per mole per Kelvin divided by fifty kilograms per mole equals zero point one six six two eight kilojoules per Kelvin.  

P sub g equals one bar plus (F over A) equals one bar plus thirty-two kilograms times nine point eight one meters per second squared divided by pi times (two over two) squared equals three point one plus zero point three nine six equals one point four five bar.  

mg equals (R times T) over (P times V) equals zero point one six six two eight kilojoules per Kelvin times five hundred degrees Celsius divided by one point four five bar times three point one four two equals two point nine two grams.