There are two graphs labeled (a) with axes marked. The first graph has a vertical axis labeled 'T' with units '[K]' and a horizontal axis labeled 'S' with units '[K/kylk]'. It shows several lines labeled 'isobar', 'isentrop', 'isochor', 'isotherm', and '6 greater than zero'. Points are marked as '1' and '2'.

The second graph has a vertical axis labeled '[K]' and a horizontal axis labeled 'S' with units '[t^2/log k]'. It shows lines labeled 'isobar', 'isotherm', 'isobar p greater than p', 'isochor', and 'isobar'. Points are marked as '1', '2', '3', '4', '5', and '6'.

There is a table with columns labeled 'P', 'T', and 'V'. It includes rows with entries:
- Row 1: '0.191', '-30 degrees Celsius', 's1 = s2'
- Row 2: 'p2 = p3'
- Row 3: 
- Row 4: '0.5'
- Row 5: '0.5', '43 i.9', 'w5 = 200 meters per second', 's5 = s6'
- Row 6: '0.191'

Arrows indicate 'mges' leading to 'm in' and 'm k'.