h_q equals 93.92 kilojoules per kilogram (delta minus delta 1).

h_1 equals h_q (Drossel ist isenthalp)

T_i equals negative 10 degrees Celsius (aus Diagramm)

T_x equals T_i minus 6 Kelvin equals negative 16 degrees Celsius

h_2 equals 492.54 kilojoules per kilogram minus 294.15 kilojoules per kilogram equals 237.7 kilojoules per kilogram

s_2 equals 6.2258 kilojoules per kilogram Kelvin minus s_3

h_3 equals 8 equals h_3 equals 264.15 kilojoules per kilogram plus (273.66 kilojoules per kilogram minus 269.15 kilojoules per kilogram) times (s_2 minus s_4 equals 1) divided by (s_4 (80 degrees Celsius) minus s_1st)

(auf nächster Seite weitergeführt)