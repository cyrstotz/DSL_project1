a)  
Q̇aus?  
QR = 100 kW  
mges,1 = 5.355 kg  
ṁ = 0.3 kg/s  
x0 = 0.005  

Energiebilanz: (Reaktor)  
0 = ṁ(he - ha) + Q̇R ± Q̇aus → 0  

he = hf(30°C) + x0 (hfg(30°C) - hf(30°C))  
he = 304.648 kJ/kg  

ha = hf(100°C) + x0 (hfg(100°C) - hf(100°C))  
ha = 430.33 kJ/kg  

Q̇aus = 0.3 kg/s (304.648 - 430.33) kJ/kg + 100 kW  

Q̇aus = 62.3 kW  

(Aus positiv definiert in Strömungsrichtung K.F.)  

hcf(30°C) = 252.98 kJ/kg (TAB A-2)  
hfg(30°C) = 2626.8 kJ/kg  
hcf(100°C) = 418.94 kJ/kg (TAB A-2)  
hfg(100°C) = 2676.0 kJ/kg  

b)  
T̅KF = ∫T2T1 ds / S2 - S1 = q̇12rev / S2 - S1 = h2 - h1 / S2 - S1  

T1 = 288.15  
T2 = 238.15  

i.p.: h2 - h1 = ∫T1T2 ċidT + vi̇ (p2 - p1) = ċi (T2 - T1)  

S2 - S1 = ċi ln(T2 / T1)  

T̅KF = T2 - T1 / ln(T2 / T1) = 238.15 - 288.15 / ln(238.15 / 288.15) = 293.12 K