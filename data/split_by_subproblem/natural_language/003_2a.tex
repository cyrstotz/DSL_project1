A graph is drawn with axes labeled with a circle containing a number one and a square containing a number two. The graph is a plot with a curve labeled 'NS' on the x-axis labeled 's'. The y-axis is labeled 'I'. Points on the graph are numbered 1 through 6. The curve has annotations such as 'reversible adiabatic', 'isentrope', 'p0 = p6', and 'u-formnummer'. There are arrows pointing to various sections of the graph.

Below the graph, an equation is written:

Partial derivative of E with respect to t equals the sum over i of m sub i of t times h sub i of t plus l sub e sub i of t plus q sub e of t plus the sum over j of q sub j of t minus the sum over n of w in of t.

An expression follows:

o equals m sub ges times the integral from T sub B to T sub S of c sub p of T dT equals c sub p times the quantity T sub S minus T sub B.

Below this, 'm dot ges' is written.