Qaus stationäre Flüssigkeit

a) Qaus

O = m in (he - ha + ke + pe) + Σj (Gj) - Σn (Vn fn)

= mein (he - ha) + QR - Qaus

Qaus = mein (he - ha) + QR

= 61,972 kW

c)

T kF = e Tols / sa se = Δh / sa se = ha - he / sa se = 360,11 K

d)

S erf = Σ Qj / T j durch Umwandlung

habs = 80,34

hats (70°C) = 48,70

se (70°C) = 0,1859

sa (70°C) = 0,3783

ha (70°C) = 104,84

he (45°C) = 62,91

se (45°C) = 0,2465

sa (45°C) = 0,354

b)

T kF = e Tols / sa se = Δh / sa se = ha - he / sa se = 293,2 OK

Tab A-2

he (70°C, x = 0) = 297,28 kj

sa (700, x = 0) = 178,09 kj/kg

ha (700°C, x = 0) = 499,91 kj/kg

se (70°C, x = 0) = 0,041 kj/kg K

w6 Q T6  
w6 = 720 meters per second  
p6 = 0.5 bar  
T5 = 537.9 Kelvin  

T5 over T6 equals (p6 over p5) to the power of k minus 1 over k  
T6 equals (p6 over p5) to the power of k minus 1 over k  
equals 329.07 Kelvin  

l5 plus w6 squared over 2 equals l6 plus w6 squared over 2  

w6 squared over 2 equals l5 minus l6 plus w6 squared over 2  
equals m dot cp (T5 minus T6) plus w6 squared over 2  

w6 equals square root of 2 (cp (T5 minus T6) plus w6 squared over 2)  
equals 507.25 meters per second