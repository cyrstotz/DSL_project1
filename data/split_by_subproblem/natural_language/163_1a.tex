W subscript K equals twenty-eight watts divided by h subscript two minus h subscript three (734.08 minus 27.926) h subscript 2 lin e equals 0.1720 kilograms per second equals m dot R subscript 234 a.

in one divided by h subscript 2 equals 7.58 h subscript 2.

T subscript 1 equals T subscript 2 equals minus 22 degrees Celsius, TAB A-11.

h subscript 1 equals h subscript g equals h subscript 1 (8 bar) equals 83.42 h subscript 1 lin e.

X subscript 1: h subscript 1 minus h subscript 1 equals 83.42 minus 24.726.

h subscript g minus h subscript 1 equals 235.31 minus 24.726.

X subscript 1 equals 0.3577 divided by 1 equals 35.77 percent.

d) E subscript K equals Q dot zu divided by W subscript K.

Q dot zu equals Q dot K equals m dot R subscript 234 a times h subscript 2 minus h subscript 1.

Q dot K equals 0.172 h subscript 2 lin e times 10 to the power of minus 3 times (734.08 minus 83.42 h subscript 2 lin e) equals 100.10 times 13 watts.

10.13 watts divided by 28 watts equals E subscript K equals 3.618.

e) The temperature will be raised to the temperature T subscript z.

b) A thermodynamic equilibrium is established.

a) Diagram: A cycle with four points labeled 1 through 4. Point 1 to 2 is labeled 'isochor', 2 to 3 is labeled 'isobar', 3 to 4 is labeled 'isochor', and 4 to 1 is labeled 'isobar'. The cycle is labeled 'TEK'.

U sub one EW equals U divided by open parenthesis U divided by open parenthesis P EW plus X sub one divided by U sub one divided by open parenthesis P EW minus U sub one divided by P close parenthesis close parenthesis

P EW equals P amps plus A K equals one thousand one hundred forty-four bar

U sub one EW equals negative zero point zero one four five divided by i sub omega squared plus zero point six zero divided by i sub omega squared times open parenthesis negative three hundred thirty-three point four five eight divided by i sub omega squared close parenthesis plus zero point zero one zero divided by i sub omega squared

equals negative two hundred point zero eight divided by i sub omega squared plus zero point zero one zero divided by i sub omega squared

Energy balance. Volume remains constant.

Zero equals m dot times [h sub in minus h sub aus plus delta e plus delta p] plus Q dot R plus Q dot aus minus W dot m equals zero.

TAB A-2

h sub in (700 degrees Celsius) equals h one (700 degrees Celsius) equals 2,892.88 kilojoules per kilogram.

h sub aus (1000 degrees Celsius) equals h one (1000 degrees Celsius) equals 4,148.04 kilojoules per kilogram.

Q dot aus equals Q dot R plus m dot times [h sub ein minus h sub aus].

Equals 100 kilowatts plus 0.3 kilograms per second times [2,892.88 minus 4,148.04] kilojoules per kilogram equals 62.19 kilowatts.

Therefore, Q dot aus equals minus 62.19 kilowatts.

Delta T sub KF

For a stationary flow process, it applies that Q dot aus equals Q dot ein KF equals Q dot aus.

T sub KF equals S aus KF minus S ein KF equals S aus KF minus S ein KF equals integral from T ein KF to T aus KF of C i over T dT equals 4.18 kilojoules per kilogram Kelvin times ln [2,893.15 over T 893.15].

C i (300 Kelvin, water) equals 4.18 kilojoules per kilogram Kelvin.

T sub KF equals 62.19 kilowatts over (0.1426 kilojoules per kilogram Kelvin) times m dot kilowatts equals 283.12 Kelvin.

m dot kilowatts equals Q dot ein KF over h two minus h one KF equals Q dot ein KF over c p times delta T equals 62.19 kilowatts over (4.18 kilojoules per kilogram Kelvin times 10 Kelvin) equals 1.488 kilograms per second.