a) All values from Table A-2:  
h sub ein equals h sub f (70 degrees Celsius) equals 252.98 kilojoules per kilogram  
h sub aus equals h sub f (100 degrees Celsius) equals 419.09 kilojoules per kilogram  
h sub R equals h sub f (100 degrees Celsius) plus x sub 0 times (h sub g (100 degrees Celsius) minus h sub f (100 degrees Celsius)) equals 430.33 kilojoules per kilogram  

Energy balance:  
For surrounding adiabatic leads to partial derivative of E with respect to t equals 0  

Q equals m dot sub ein times (h sub ein minus h sub R) plus m sub aus times (h sub R minus h sub aus) plus Q dot sub R plus Q dot sub aus  
equals m sub ein times (h sub ein minus h sub aus) plus Q dot sub R plus Q dot sub aus  
equals minus 37.82 kilowatts plus 100 kilowatts plus Q dot sub aus  
Q dot sub aus equals minus 62.182 kilowatts  

e) T bar equals integral from e to a of T ds divided by s sub a minus s sub e

phi