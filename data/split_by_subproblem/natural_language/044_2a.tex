A diagram is drawn with several labeled points and lines. Points are labeled as p six, p five, p four, p three, p two, p one, and an arrow pointing to 'sign hop'. The axis is labeled 'S' with a note: '(kappa) k divided by kappa times (kappa) k divided by kappa'. Additional notes are written: 'Pressure ratio, i.e., the pressure space'. The axis is labeled 'T' divided by 'T' s.

a) perfect gas

p1 = pamb plus 32 times g times z plus rho1 times g times z

equals 9.27 times 1.2 divided by 2

equals 1.40059 bar

ideal gas law:

m dot g equals p1 times V1 divided by R times T1

R equals R divided by mg equals 166.28 kilojoules per kilogram Kelvin

equals 1.40059 times 3.16 times 10 to the power of minus 2

equals 166.28 divided by (500 plus 273.15)

equals 3.4364 s