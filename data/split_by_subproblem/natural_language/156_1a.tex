a) given Q aus

0 = m dot (h1 - h2) + Q dot R + Q aus

Therefore, Q aus = -Q dot R + m dot (h2 - h1) = -62 kW

1) T1 = 70 degrees Celsius, h1 = hp, A2 = 282.38 kJ/kg
2) T2 = 100 degrees Celsius, h2 = hp, A2 = 419.04 kJ/kg

p1g3 equals p0 plus

p1g times Vgm equals mg times R times Tgm

unit R equals R divided by mg equals 166.28 divided by 18.028 equals 3 divided by kg K

mg equals p1g times V1.19 divided by R times Tgm equals 3.4277 g

T sub 12 equals 0.003 degrees Celsius.

c)
Find Q sub 12.

g:
U sub 2 minus U sub 1 equals Q sub 12.

Therefore, Q sub 12 equals m sub g times C sub V times (T sub Z minus T sub n) equals negative 1.0820 kilojoules.

a)
Find x sub Eis,12.

T sub EN,12 equals 0.003 degrees Celsius.

EW:
U sub 2 minus U sub 1 equals negative Q sub 12.

Therefore, U sub 2 equals U sub 1 minus Q sub 12.

U sub 2 equals U sub 1 minus Q sub 12 divided by m sub EW.

With x sub f equals 0.6, T sub EW equals 0 degrees.

Tab 1: U sub n equals U sub f plus x times (U sub g minus U sub f) equals negative 133.41 kilojoules per kilogram.

Therefore, U sub 2 equals U sub 1 minus Q sub 12 divided by m sub EW equals negative 122.5805 kilojoules per kilogram.

Tab 1: x sub 2 equals U sub 2 minus U sub f divided by U sub g minus U sub f equals 0.9888.