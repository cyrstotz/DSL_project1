A graph is drawn with T in parentheses k on the y-axis and s in parentheses k times joules divided by kilograms on the x-axis. The graph has a series of labeled points: 1, 2, 3, 4, 5, 6, and 0. There are curves connecting these points.

An equation is written as follows:
m dot times s times in parentheses h e minus h a plus one half w s squared equals one half w e squared close parentheses equals zero.

Another equation is:
h a minus h e equals c p times in parentheses T l minus T l close parentheses equals one thousand six times in parentheses T l minus T s close parentheses.

Below this, there is:
T 6 equals T 6 equals eight hundred sixty five.