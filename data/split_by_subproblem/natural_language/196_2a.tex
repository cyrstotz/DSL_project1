A graph is drawn with points labeled 0, 1, 2, 3, 4, 6. The y-axis is labeled with 'T' and '[K]', and the x-axis is labeled with 's' and '[kJ/kgK]'. The graph includes several annotations: 
- Between points 0 and 1, the path is labeled as 'isotrop'.
- Between points 1 and 2, the path is labeled as 'adiabat, reversibel, isotrop'.
- Between points 2 and 3, the path is labeled as 'isochore'.
- Between points 3 and 4, the path is labeled as 'dc, irreversibel, 0,5 bar'.
- Between points 4 and 6, the path is labeled as 'isochore'.

The question is about mass flow rate 'ṁ R134a = ?'. It mentions 'isobare Verdampfung R134a' (isobaric vaporization R134a) with 'Wk = 28 W'. The temperature of the evaporator 'T Verdampfer = T1 minus 6 K = 257.15 K'. The temperature 'T1 = ? = -10°C = 263.15'. It discusses 'Energiebilanz über verdichter' (energy balance over compressor): 'Q = ṁ (he - ha) + ΣQ = -0 adiabatic'. The equation 'ṁ = Wk / (he - ha) = -0.08 kW' is crossed out.

Below this, it states 'he = ? T = -16°C von R134a' (he = ? T = -16°C from R134a), 'h2 = h2', 'x2 = 1'. It refers to 'Tabelle 1.10' (Table 1.10) with 'T@ -16°C hg = 237.9 hfg'. It concludes 'h2 = he = he' and asks 'ha = ?'.

b) he minus ha equals cp times (431.3 Kelvin minus 328.07 Kelvin)

h5 minus h equals cp times (431.3 Kelvin minus 328.07 Kelvin)

equals 1.006 kilojoules per kilogram Kelvin times 104.498 kilojoules per kilogram

we equals 270 meters per second

wa equals square root of (2.104.488 kilojoules per kilogram plus (270 meters per second squared) divided by 100)

equals square root of (2.104.488 kilojoules per kilogram times 1000 plus (270 meters per second squared))

equals 507.294 meters per second

c) delta er,str equals ?

ex,str equals (h minus ho minus To times (s minus so) plus (ke plus pe))

ho equals

ex,str,G equals ex,str,O

ex,str,G equals (hG minus ho minus To times (sG minus so) plus (ke))

ex,str,O equals (ho minus ho minus To times (sO minus so) plus (ke)) equals keO

delta ex,str equals ex,str,G minus keO

equals

hG minus ho equals cp times (T6 minus To) equals 1.006 kilojoules per kilogram Kelvin times (328.07 Kelvin minus 293.15 Kelvin)

equals 35.428 kilojoules per kilogram

s6 minus so equals cp times ln(T6 divided by To) minus R times ln(n)

equals 1.006 kilojoules per kilogram Kelvin times ln(328.02 Kelvin divided by 293.15 Kelvin) minus 0.301335 kilojoules per kilogram Kelvin

T zero equals 283.15 Kelvin  

Delta exergy equals 85.923 kilojoules per kilogram minus 283.15 Kelvin times 0.303 kilojoules per kilogram Kelvin plus 507.294 meters per second squared divided by 2 minus 700 meters per second squared divided by 2  

Equals 85.923 kilojoules per kilogram minus 73.7272 kilojoules per kilogram plus 128.648 kilojoules per kilogram minus 257.296 kilojoules per kilogram minus 70 kilojoules per kilogram  

Equals minus 120.005 kilojoules per kilogram