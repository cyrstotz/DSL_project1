Gas: C subscript v equals 0.633  
M subscript g equals 50 kilogram per mole  

a) p subscript g comma 1, p V equals R T  

R equals R over M subscript g equals 8.314 joule per mole Kelvin over 50 kilogram per mole equals 0.16628 kilojoule per kilogram Kelvin  

V subscript g comma 1 equals 3.14 liter  
T subscript g comma 1 equals 500 degrees Celsius  
m subscript e w equals 0.1 kilogram  

Diagram labeled K G W showing:  
p subscript g comma 1 A equals p subscript amb A plus m subscript e w g plus m subscript e w g  
p subscript g comma 1 equals p subscript amb plus m subscript k g over A plus m subscript e w g over A  

equals 4.10 to the power of 5 newton per meter squared plus 32.981 newton over 0.00785388 meter squared plus 0.1 times 9.81 newton over 0.0079 meter squared equals 140'034.4406 newton per meter squared approximately equals 1.4 bar  

A equals D squared over 4 pi equals 100 times 10 to the power of minus 3 meter squared over 4 pi equals 0.00785388 meter squared  

p V equals m R T  

m subscript g comma 1 equals p subscript g V subscript g over R T subscript g equals 1.4 times 10 to the power of 5 pascal times 3.14 times 10 to the power of minus 3 meter cubed over 0.16628 times 10 to the power of 3 joule per kilogram Kelvin times [500 plus 273.15] Kelvin  

equals 0.00634 kilogram approximately equals 3.422 gram  

b) x subscript eis comma 2 greater than or equal to 0, x subscript eis comma 1 equals m subscript eis over m subscript e w equals 0.6  

m subscript eis equals 0.6 times m subscript e w equals 0.06 kilogram  

Gas und E W Thermodyn. G G W  

Weil Dichte von Eis und Wasser gleich sind, verändert sich die Masse (und Volumen) von Eiswasser nicht. Durch das K G W sieht man, dass p subscript g comma 2 equals p subscript g comma 1  

p subscript g comma 2 equals 1.4 bar