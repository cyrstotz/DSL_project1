A graph is drawn with an x-axis labeled as s over u subscript s times h. The y-axis is labeled T of u. The graph has several points labeled 0, 1, 2, 3, 4, 5, and 6. There are dashed and solid lines connecting these points. Several pressure levels are indicated: P subscript 2 equals P subscript 3, P subscript 4, P subscript 5, P subscript 1, P subscript 7, P subscript 6.

R equals C subscript p minus T subscript 0 over 1.4 equals 288 s over u subscript s times h.

5.) U subscript 0, T subscript 0

T subscript 0 equals T subscript s times the fraction of P subscript 0 over P subscript s to the power of k minus 1 over k equals 288.07 K.

w subscript 0 6 equals R times T subscript 0 minus T subscript 0 over n minus n equals minus 61.2 kJ.

c.) s subscript ex equals C subscript p times (T subscript s minus T subscript 0) minus T subscript 0 times (C subscript p times h of T subscript s over T subscript 0 minus R times h of P subscript s over P subscript 0) plus w subscript 0 squared over 2 minus U subscript 0 squared over 2.

s subscript ex equals square root of (T subscript s times u subscript 0 squared) over u subscript s.

A two is constant

A one two is y sub s sat
h sub three is F sub sat plus

S minus F sub sat (8 bar)
S (8) (u sub c) minus F (8 bar) sub s sat
times (h (8 bar) sub c a one) minus h sub s sat (8 bar)

h sub three is 27,31

U dot sub x equals m dot Q (h sub two minus h sub three)

m dot sub Q equals h sub x minus U dot sub x
h sub two minus h sub three
equals 9,8 kg per s