T two equals seventy degrees Celsius, x two equals zero.

Delta two:
h two equals u two (one thousand degrees) equals two hundred ninety-two point thirty-five kilojoules per kilogram.

(m one plus Delta m two) u two minus m one u two minus Delta m two u ein equals Q.

Delta m two u two minus Delta m two u ein equals Q minus m one u two plus m one u ein.

Delta m two equals Q minus m one u two plus m one u ein over u two minus u ein equals thirty-five point eighty-three kilograms.

A graph is drawn with the y-axis labeled as "T [K]" and the x-axis labeled as "S [H over m dot s]". The graph shows a curve with points labeled 0, 1, 2, 3, and 4. The curve is described as "isentrop" from point 0 to 1, "isobare" from point 1 to 2, and "isobare" from point 2 to 3. The term "prädor p=0.1966..." is noted near the curve.