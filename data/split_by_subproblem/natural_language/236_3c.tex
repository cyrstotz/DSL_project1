Exergy of a flow

E ex dot equals m dot ex equals m dot times bracket h minus h zero minus T zero times bracket s minus s zero bracket plus ke plus p squared bracket

Zustand 1: m dot ex equals m dot ges times bracket h one minus h zero minus T zero times bracket s one minus s zero bracket plus w zero squared divided by two

Zustand 0: m dot ex zero equals m dot ges times bracket h zero minus h zero minus T zero times bracket s zero minus s zero bracket plus ke

equals m dot ges times bracket ke equals m dot ges times w zero squared divided by two

e x zero equals w zero squared divided by two equals 20000 meters squared per second squared

Zustand 6: e x zero equals bracket h six minus h zero minus T zero times bracket s six minus s zero bracket plus w zero squared divided by two

h six minus h zero equals Cp e times bracket T six minus T zero bracket equals 97.431 kilojoules per kilogram

s six minus s zero equals Cp e times bracket ln T six divided by T zero bracket minus R times ln bracket p six divided by p zero bracket

R equals R divided by M equals 28.97 kilojoules per kilomole equals 0.287 kilojoules per kilogram

e x six equals 97.431 kilojoules per kilogram minus R times ln bracket p six divided by p zero bracket

p zero divided by p zero equals 1

s six minus s zero equals integral from T zero to T six Cp divided by T dt equals bracket Cp ln T bracket from T zero equals Cp bracket ln T six minus ln T zero bracket

s six minus s zero equals 0.337 kilojoules per kilogram Kelvin

e x six equals 97.431 kilojoules per kilogram minus 243.15 times 0.337 kilojoules per kilogram Kelvin plus 510 meters squared per second squared divided by two equals 130065.5 kilojoules per kilogram equals 130065.5

Delta E equals Q12 minus W12.  
pV equals mRT therefore p equals mRT over V equals RT over v.  
v equals V over m.  
v1 equals three point fourteen times ten to the power of negative three over three point four three times ten to the power of negative three.  
v1 equals zero point nine one five meters cubed per kilogram.  
Gas isobar: p2 equals RT over v1 equals forty-nine point fifty-six kilopascals.  
E equals U plus zero plus zero (negligible).  
Delta U perfect gas: Delta U equals Cv times (T2 minus T1) equals zero point six three three kilojoules per kilogram times (two hundred seventy-three point one five three kelvin minus seven hundred seventy-three point fifteen kelvin) equals negative three hundred sixteen point five kilojoules per kilogram.  
Q12 equals Delta U plus W12 equals W12 minus three hundred sixteen point five kilojoules per kilogram.

Stat. Fließprozess mit n Punkt Punkt Punkt

Q Punkt gleich m Punkt [Se Punkt Sa Punkt] plus Q Punkt durch t plus Se Punkt

ds durch dt gleich null gleich m Punkt ein Se ein minus m Punkt aus Se aus plus m Punkt rein Se ein minus m Punkt raus Se aus plus Q Punkt durch t plus Se Punkt

Daraus folgt Se Punkt gleich m Punkt ein [Sa aus minus Se ein] plus m Punkt kp [Sa aus minus Se ein]