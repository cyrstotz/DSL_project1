a) S i P T is greater than or equal to O equals m dot in times h equals m dot out times h plus Q dot out minus Q dot in minus Q dot aux.

Q aux equals m dot times (h e minus h out) plus Q R.

Steigende Flüssigkeit beide am Ein und Ausgang.

x e is greater than 0; x f is greater than 1 jeweils.

Am TAB K2: h e equals h f (at 700 degrees Celsius) equals 292.98 kilojoules per kilogram.

h out equals h f (at 200 degrees Celsius) equals 430.04 kilojoules per kilogram.

Q aux equals 0.3 times 2 kilograms per second times (292.98 minus 24.04) kilojoules per kilogram plus 200 kilojoules per second equals 62.482 kilowatts.

b) T equals integral from s a to s e of T d s divided by s a minus s e.

m dot e equals Q dot K minus W dot K divided by h three minus h four implies Q dot K equals m dot e times (h three minus h four) plus W dot K.

m dot e times (h two minus h three) plus W dot K divided by h two minus h one.

m dot e times (one plus h two minus h three divided by h two minus h one) equals W dot K divided by h two minus h one.

m dot e times (h two minus h three divided by h two minus h one) equals W dot K divided by h two minus h one.

m dot e equals W dot K divided by h two minus h one plus h two minus h three equals 28 kilograms per second equals (232.62 minus 16.82 plus 93.42 minus 264.25) Joules per kilogram.

From TAB A-20: at T two equals T one equals 260 degrees Celsius, h f equals h two equals 232.62 Joules per kilogram, h two equals h f equals 16.82 Joules per kilogram.

From TAB A-12: at p three equals p four equals 2 cubic, h three equals h g equals 264.25 Joules per kilogram, h two minus h three equals 93.42 Joules per kilogram.

m dot e equals 0.6352 kilograms per second equals 4 kilograms per hour.

m dot h two minus h three equals 0.6352 grams per second.

c) Yes 2: adiabatic reversible process efficiency equals lambda.