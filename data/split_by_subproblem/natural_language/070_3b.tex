v6 equals question mark, T6 equals question mark.

v5 equals 220 meters per second, T5 equals 437.5 Kelvin, p5 equals 0.5 bar.

p6 divided by p0 equals 0.197 bar.

isentropic, s5 equals s6.

Stationary process.

O equals m dot times (h5 minus h6 plus w5 squared minus w6 squared divided by 2) plus Q dot minus W dot.

Adiabatic, reversible.

h5 minus h6 equals cp times (T5 minus T6) equals 1496.2286 kilojoules per kilogram.

cp equals 1 kilojoule per kilogram Kelvin, cv equals 0.714 kilojoules per kilogram Kelvin.

R equals cp minus cv equals 0.9024 kilojoules per kilogram Kelvin.

Isentropic, s5 equals s6.

O equals cp times ln (T6 divided by T5) minus R times ln (p6 divided by p5).

R times ln (p6 divided by p5) equals cp times ln (T6 divided by T5) times e to the power of 1.

e to the power of R times ln (p6 divided by p5) equals e to the power of cp times ln (T6 divided by T5).

(p6 divided by p5) to the power of R equals (T6 divided by T5) to the power of cp.

T6 equals T5 times (p6 divided by p5) to the power of R divided by cp equals 328.075 Kelvin.

w6 squared equals w5 squared.

Equals 27933.4.

Equals 1729.93 meters per second.

Since under the given conditions nothing changes, there are still 32 kilograms plus 1 kilogram atmospheric pressure, therefore P_3,2 equals P_3,1 equals 1.297 bar.

Delta E equals 0 equals m_EV times (u_2g minus u_1g) plus m_g times (u_2 minus u_1) equals m_EV times (u_2g minus u_1g) plus m_g times (C_V times (T_2 minus T_1)).