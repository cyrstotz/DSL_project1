- wider
- fest
- Tripelpunkt
- 2 alpha
- flüssig
- dampf
- Isobar

A diagram is drawn with the following labels:

- The vertical axis is labeled as "T [K]".
- The horizontal axis is labeled as "s [J/kgK]".
- Points are marked as 1, 2, 3, 4, 5, and 6.
- Between points 1 and 2: "isentrop".
- Between points 2 and 3: "isobar".
- Between points 3 and 4: "adiabat irreversible".
- Between points 4 and 5: "isobar".
- Between points 5 and 6: "adiabat, reversibel → isentrop".
- The line from point 0 to point 1 is labeled "η < 1" and "adiabat, reversibel → isentrop".
- "To = 243.15 K" is noted near the origin.

A table is drawn below the diagram with the following columns: P, T, and Notes.

- Row 0: "0.65 bar", "243.15 K", "η < 1", "adiabat, reversibel → isentrop".
- Row 1: "1", "Isobar".
- Row 2: "2", "Isobar", "T = 273 K".
- Row 3: "3", "adiabat, irreversible".
- Row 4: "4", "Isobar".
- Row 5: "5", "0.5 bar", "TB", "31 K".
- Row 6: "6", "0.19 bar".

b) Water

m dot (h_a + h_e) = v_e/2 (w_e squared - w_a squared)

h_a - h_e = w_e squared - w_a squared / 2

w_e puff squared / 2 (h_a - h_b) = w_e squared / 2

w_e = square root (w_e puff squared - 2 (h_a - h_b))

w_0 = square root (w_e cuff squared + 2 c_p (t_0 - T_0))

w_0 = square root (200 - w_e squared / 5 - 7.1 times 10 to the power of 5 / 3 times 251.69 K - 203.15)

w_0 = 195.07 m/s

muss hier ein

w_a = w_0, w_0 = w_e (w puff)

perfect gas

h_a - h_0 = c_p (T_0 - T_0)