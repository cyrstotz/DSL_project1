b) F kf = ∫ a e δ T ds / s a - s e

s a - s e = c fl ln (T aus / T ein)

∫ a e T ds = q rev = h aus - h ein (1. HS)

= c fl [T aus - T ein] (ideale Flüssigkeit) (isobar)

T aus = 298.15 K

T ein = 288.15 K

T aus / T ein = 293.12 K

Wc, Tc

1. HS um Triebwerk:
stationär, dE = 0, keine Arbeit, adiabatic, q = 0

0 = m dot (ha - hc) + we squared minus wc squared over 2 equals 2 (ha - hc) plus we squared over 2 equals p0 over pc

minus 2 HA (ha - hc) plus we squared equals wc squared

Wo equals plus 2 HA (ha - hc) plus we squared equals

ha equals

hc equals (ho - hc equals cp is (To - Tc) equals minus 85.43

To equals minus 30 degrees

Tc:
1. HS um Schubdüse:
Tc equals Ts times (pc over ps) to the power of 0.4 over 1.4 equals 328.075 K equals 54.925 degrees Celsius

pc equals 0.19 bar, ps equals 0.85 bar

hs minus hc equals cp is (Ts minus Tc) equals 109.498

1 HS um Schubdüse:
0 equals m dot (hs - hc plus ws squared minus wc squared over 2)

hs minus hc equals ws squared over 2 minus wc squared over 2

therefore wc equals square root of 2 (hs - hc) plus ws squared equals 160 meters per second