p in bar, T in degrees Celsius, h in kilojoules per kilogram, s in kilojoules per kilogram Kelvin, and Anmerkung.

1: 500

Vg1 = 3.14 L

amb: p equals 1 bar, mk equals 32 kg

a) Pgg equals ?, mg equals ?

perfektes Gas, mg equals Vg over Vgg

Vg1 equals 3.14 L equals 0.00314 cubic meters

P via Kräfte GGW:

Pgg times A equals Pamb times A plus mk times 3.81 newtons per kilogram plus mew times 9.81 newtons per kilogram

Pgg equals Pamb plus 9.81 newtons per kilogram times (mz plus mew) over A

A equals pi times (1 over 2) squared equals pi times (0.1 meters over 2) squared equals 0.0079 square meters

Pgg equals 1 bar plus 9.81 newtons per kilogram times 32.1 kilograms over 0.0079 square meters equals 1.3986 bar

equals 1.4 bar

mg equals Pgg times Vgg minus 1.4 times 10 to the power of 5 pascals times 0.00314 cubic meters over R times T equals 8.314 joules per mol per Kelvin, 773.15 Kelvin

mg equals 3.42 grams