A graph is drawn with a horizontal axis labeled 't' and a vertical axis labeled 'e'. The graph depicts a curve with a peak, and a point labeled 'Tiefpunkt' is marked at the bottom of the curve. The point is indicated with coordinates (-1, 1). There is also a label 'Sattel' near the vertical axis.

hs divided by h0 plus ws squared divided by 2 equals η times (T6 minus Ts) equals ws squared divided by 2.

c) St. FD: Exergiebilanz

0 equals m dot times (he minus ha minus T0 times (se minus sa)) plus QKF plus Σ times (1 minus T0 divided by T1) times Q dot i minus Σ times w dot e,in minus Exvert

exist (FS)

equals he minus h0 minus T0 times (se minus s0) plus te plus PE

equals ex0 minus ex0,0 equals (he minus h0 minus T0 times (se minus s0) plus ke0)

kinetische energie vernachlässigt

equals (he minus h0 minus T0 times (se minus s0) plus 1 divided by 2 times (ce0 squared minus ce squared))

d) St. FD: Exergiebilanz?

0 equals m dot times (he minus ha minus T0 times (se minus sa)) plus QKF plus QPE

plus Σ times (1 minus T0 divided by T1) times Q dot i minus Σ times w dot e,in minus E invert

Energy balance for the half-open system:

m2 u2 minus m1 u1 plus sum of kinetic energy plus sum of potential energy equals zero.

Sum of hi plus kinetic energy plus potential energy equals sum of qi plus sum of wi.

m2 equals m1 plus delta m.

(m1 plus delta m) u2 minus m1 u1 equals zero. Hence, h in equals 209.

u2 equals u1 (300 degrees Celsius) plus x2 u f g equals u1 (700 degrees Celsius) because q2 equals zero.

u1 equals 118.9 kilojoules per kilogram plus 0.005 times 2006.5 kilojoules per kilogram.

Equals 130.17 kilojoules per kilogram.

u2 equals 280.95 kilojoules per kilogram (A-2).

h in equals 3883.96 kilojoules per kilogram.

delta sum equals m2 (u2 minus u1) divided by h in minus u2.

Equals 5735.6 (130.17 kilojoules per kilogram minus 707.05 kilojoules per kilogram).

Equals 2.92 divided by 35.96 minus 282.95 kilojoules per kilogram.

a) E tube reactor.  

St. FP:  

O mein [hein haas - KE / DE] + QR Qgas  

hein und haas von Tab A 2  

hein = h f (70 degrees Celsius) = 292.98 kJ / kg  

haas = hA (1000 degrees Celsius) = 119.01 kJ / kg  

Qgas = mein [hein - haas] + QR  

= 0.35 kg / s (292.98 kJ / kg - 119.01 kJ / kg) + 100 kW  

= 61.8 kW