A graph is drawn with an axis labeled 'T [K]' and another labeled 's [N/kgK]'. Several curves are plotted with points labeled 1, 2, 3, 4, 5, 6. The points have arrows indicating directions. Various pressures are marked on the graph: p0, p1, p2, pS. 

Below the graph, there is a table with columns labeled 'T', 's', and 'P'. Rows are numbered 1 to 6. The entries include:
1: -30°C, s1, p0
2: s3 greater than s1, p2 greater than p0
3: s2 greater than s2, p1 equals p2
4: s3 greater than s3, p4 less than p3
5: 98.3K, 0.5 bar equals p4 equals p5
6: s6 equals s5, p0

h subscript s equals nine hundred fifty-three point two six kilojoules per kilogram

Isentrop Zust and:
T subscript 6 equals T subscript s open bracket p subscript 6 over p subscript s close bracket to the power of n minus one over n

T subscript 6 equals three hundred twenty-eight point zero seven five Kelvin

h subscript 6 equals h open bracket three hundred twenty-five Kelvin close bracket plus b open bracket three hundred thirty Kelvin close bracket minus h open bracket three hundred fifty Kelvin close bracket over five K

h subscript 6 equals three hundred thirty-three point nine eight kilojoules per kilogram

zero equals m dot open bracket h subscript s minus h subscript 6 close bracket plus omega subscript s squared minus omega subscript 6 squared over two close bracket

m dot ges squared open bracket h subscript 6 minus h subscript s close bracket minus m dot ges omega subscript s squared minus omega subscript 6 squared over two, minus omega subscript 6 squared equals two open bracket h subscript 6 minus h subscript s close bracket minus omega subscript s squared

omega subscript 6 squared equals two open bracket h subscript s minus h subscript 6 close bracket plus omega subscript s squared, omega subscript 6 equals nine hundred ninety-eight point two six meters per second