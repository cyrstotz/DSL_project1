W dot k equals m dot times (h3 minus h4), therefore m dot equals W dot k divided by (h3 minus h4).

We now take h2 and h4 from the tables A-10, A-11, A-12.

h2 needs to be interpolated from h2 (T i equals 6), because T i is found with the p-T diagram at minus 20 degrees Celsius (table A-10).

h2,5 equals h2 (T equals minus 26 degrees) equals 231.62, and the entropy s2 equals s3 (T equals minus 26 degrees Celsius) equals 0.8390.

Since the process 2 to 3 is adiabatic reversible, delta S equals 0.

In table A-12, we find h3 equals h (s3) equals 273.66 plus (231.62 plus 273.66) divided by (891.1 minus 0.5396) times (0.8390 minus 0.5396) approximately equals 239.14.

Therefore, m dot equals W dot k divided by (h3 minus h4) equals 2E divided by 49.55 equals 0.658 kilograms per second equals 2.37 kilograms per hour.

T i was found with the diagram at minus 20 degrees Celsius, therefore T equals minus 20 minus 6 equals minus 26 degrees Celsius.