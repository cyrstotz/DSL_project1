Entropy balance at the real boundary:

O equals Q dot over T peak minus Q dot over T KF plus S dot enz

S dot enz equals Q dot anz abs times (1 over T KF minus 1 over T peak)

T peak equals 100 degrees Celsius equals 373.15 Kelvin

S dot enz equals 62.5 kilowatts times (1 over 293.12 minus 1 over 373.15) equals [scratched out] 0.570 watts per Kelvin

W subscript G equals 520 meters squared per second cubed  
T subscript 6 equals 340 Kelvin  

Delta e subscript exergy equals e subscript exergy b minus e subscript exergy 0  

e subscript exergy b equals phi subscript b minus h subscript n minus T subscript u (S subscript c minus S subscript u) plus W subscript 2 squared divided by 2  

e subscript exergy 0 equals h subscript 0 minus h subscript n minus T subscript 0 (S subscript 0 minus S subscript u) plus W subscript 0 squared divided by 2  

Delta e subscript exergy equals h subscript 6 minus h subscript 0 minus T subscript u (S subscript c minus S subscript 0) plus W subscript 2 squared minus W subscript 0 squared divided by 2  

T subscript u minus T subscript 0 equals 243.15 Kelvin  

h subscript 6 minus h subscript 0 equals C subscript p liquid times (T subscript 6 minus T subscript 0) (table figures 1G) arrow (page 90.1)  

S subscript c minus S subscript 0 equals C subscript p liquid times ln (T subscript 6 divided by T subscript 0) minus Q divided by p subscript 0  

Delta e subscript exergy equals C subscript p liquid times [(T subscript 6 minus T subscript 0) minus T subscript 0 times ln (T subscript 6 divided by T subscript 0)] plus W subscript 2 squared minus W subscript 0 squared divided by 2 plus S subscript 0 squared minus 2 omega subscript 0 squared divided by 2  

Delta e subscript exergy equals 1.006 kilojoules per kilogram Kelvin times [(340 minus 243.15) minus 243.15 times ln (340 divided by 243.15)] plus 125.62 kilojoules per kilogram  

Delta e subscript exergy equals crossed out text

T subscript 3,2 equals 0.003 degrees Celsius.

Q subscript 12 !

Energy balance around gas:

Delta E equals E subscript 2 minus E subscript 1 equals Q subscript 12 minus W subscript L.

Delta E equals Delta U plus Delta K E plus Delta P E.

Delta U equals m subscript g times (u subscript 2 minus u subscript 1) equals m subscript g times C subscript V times (T subscript 2 minus T subscript 1) equals 3.418 times 10 to the power of 3 joules per kilogram times (0.633 kilojoules per kilogram Kelvin times (0.003 minus 500)).

Delta U equals minus 1.082 kilojoules.

W subscript L

Q subscript 2 equals Q subscript 1

m subscript 2 equals m subscript 1

rho V equals m R subscript g T subscript 2

V subscript 2 over V subscript 1 equals m subscript g R subscript g T subscript 2 over p subscript 1 equals 3.418 times 10 to the power of 3 joules kilogram times 273.253 Kelvin over 7.4 bar equals 1.103 liters.

W subscript V equals p subscript 1 times (V subscript 2 minus V subscript 1) equals 7.4 bar times (crossed out: 1.108 minus 3.14) times 10 to the power of minus 3 meters cubed.

W subscript L equals minus 0.284 kilojoules.

Delta P E equals m subscript H subscript 2 O times g times (Delta V over Delta h) equals minus 91.47.

Delta U plus Delta P E equals Q subscript 12 minus W subscript L.

Q subscript 12 equals Delta U plus Delta P E plus W subscript L equals minus 1.082 minus 871.4 times 10 to the power of minus 3 minus 0.284 equals 796.5.

rho n equals rho 2.

h n minus h u equals h f ( p n over 8 bar ) equals 93.42 k J over kg. T equals a T equals n.

x n equals h n minus h f ( p 2 ) over h g ( p 2 ) minus h f ( p 2 ).