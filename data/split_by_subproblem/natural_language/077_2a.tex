A graph is depicted with a vertical axis labeled 'T' and a horizontal axis labeled 'S'. Various lines and points are marked on the graph:

- Points are labeled as 1, 2, 3, 4, 5, and 6.
- A line labeled 'isotherm' is drawn horizontally.
- Two lines labeled 'isobar' are drawn at different angles, with 'p greater than p0' and 'p less than p0' noted.
- A line labeled 'isentrop' is drawn vertically.
- The temperature '-30°C' is marked on the vertical axis.

Next to the graph, a smaller diagram is shown with points labeled 1, 2, 3, 4, 5, and 6.

Below the graph, transitions between states are described:

(1 -> 2) isentrop s - constant p decreases

(2 -> 3) isobar p - constant T increases

(3 -> 4) not isentrop

(4 -> 5) isobar p - constant

(5 -> 6) isentrop s - constant