Energy balance for a nozzle is written as follows:
u equals n with a dot multiplied by (h subscript S minus h subscript 6 plus (w subscript S squared minus w subscript 6 squared) divided by 2).

T subscript 0 using the Brayton equation:
T subscript 0 equals (P6 divided by P5) raised to the power of k minus 1 divided by k multiplied by T5.
Equals 828.075 Kelvin.

Ideal gas:
u subscript 6 equals square root of n with a dot multiplied by c subscript p multiplied by (T5 minus T6) plus w subscript S squared divided by 2 multiplied by 2.

Equals 220.47 meters per second.

Additional notes:
Section 5 to 6 is reversed.
P subscript c equals 0.5 bar adiabatic.
w subscript R equals 220 meters per second.
T subscript S equals 451.912.
P6 equals P6 equals 0.191 bar.
W subscript in equals 0.

d) Exergy, reversible equals Serz minus T0 over m gas

Entropy balance:

Serz, adiabatic equals m [sq minus sc]

Serz equals sc minus s0

sc minus s0 equals cp ln (Tc over T0)

Alles einsetzen equals exergy, reversible equals 82.01 kilojoules per kilogram.