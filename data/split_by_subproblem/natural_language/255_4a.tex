Graph with labels:
- Sublimieren
- Tripel
- kritischer Punkt
- Phasendiagramm
- fest
- flüssig
- gasförmig

b) pT equals p2, T equals six k equals T1 minus T2

7. HS über Verdichter, W punkt equals 29 W

O equals m Punkt R134 (h2 minus h3) plus l

m Punkt R134 equals minus W punkt over h2 minus h3

T1 equals 40 degrees Celsius

h2 equals 249.53

s2 equals s3 equals 0.9.765

s3 equals sf plus x3 (sg minus sf) implies x3 equals s3 minus sf over sg minus sf

h3 equals hf plus x3 (hg minus hf)

m Punkt R134 equals W punkt over h4 plus x3 (hg minus hf) minus h2

A graph is drawn with the y-axis labeled as "P (in bar)" and the x-axis labeled as "T (C)". The graph is divided into three sections labeled "fest" (solid), "flüssig" (liquid), and "gas" (gas). A curve is drawn starting from the origin, going upwards and then leveling off. A point on the curve is labeled "kritischer Punkt" (critical point). Another point on the curve is labeled "Tripel" (triple point).