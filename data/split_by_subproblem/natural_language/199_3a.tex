R equals R divided by M equals eight point three one four joules per mole Kelvin divided by fifty kilograms per kilomole equals zero point one six six two eight joules per kilogram Kelvin.

p gas equals p ambient plus F L divided by A plus F EV divided by A equals p ambient plus m L times g divided by open parenthesis pi divided by two squared close parenthesis times T plus m EV times g divided by open parenthesis pi divided by two squared close parenthesis times T equals one point four zero zero times ten to the power of negative five Newtons per square meter minus one point six bar.

p times V equals m times R times T. n equals p divided by V times R times T. m g equals p ambient times V g divided by R times T g one.

One point four times ten to the power of negative five Newtons per square meter times three point one four times ten to the power of negative three cubic meters divided by zero point one six six two eight joules per kilogram Kelvin times two seven three point one five Kelvin equals three point four one nine times ten to the power of negative three kilograms equals three point four two grams.