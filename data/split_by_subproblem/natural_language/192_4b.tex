c equals constant.

Therefore, T bar equals phi times (T2 minus T1) over phi times ln (T2 over T1).

T2 equals 298.15.

T1 equals 288.15.

T bar equals 293.12 Kelvin.

T i equals minus ten degrees Celsius

ϕ p T h s

1 1.3748 minus 16

x equals 1 2 p1 minus 16

3 8

x equals 0 4 8

T h 2 equals T i minus 6 Kelvin equals minus 16 degrees Celsius

h 2 equals h g (minus 16 degrees Celsius) implies A 10

h 2 equals 237.74 kilojoules per kilogram

s 2 equals s 3 equals s g (minus 16 degrees Celsius) q 0 equals 0.9288 kilojoules per kilogram Kelvin

p 3 equals 8 bar implies [crossed out]

via A-M sehe ich, wir sind im Dampf-gebiet

implies A-12

h 3 equals h sat plus (h (40) minus h sat) divided by s (40) minus s sat times (s 3 minus s sat)

h (40) equals 273.66

h sat equals 269.45

s sat equals 0.9066

s (40) equals 0.9374

h 3 equals 271.3 kilojoules per kilogram