Let's set up the energy balance.

Q equals m dot gs times (h2 minus h3) times I plus W dot x

W dot x equals m dot gs times (h2 minus h3)

Negative W dot x over h2 minus h3 equals m dot gs

h2 equals hg (277.15 K) equals hg (9 degrees Celsius) equals 249.53 kilojoules per kilogram (from Table A.10)

h3 minus h1 (8 bar) equals 269.15 kilojoules per kilogram

m dot gs equals 0.28 kilowatts

s2 minus s3 (9 degrees Celsius) equals 0.9169 kilojoules per kilogram Kelvin

s3 equals s2 equals 0.9169 kilojoules per kilogram

x3 equals h3 (8 bar, 0.9169 kilojoules per kilogram) equals 267.33 kilojoules per kilogram

h (8 bar, 0.3066 kilojoules per kilogram) equals 264.15 kilojoules per kilogram

h (8 bar, 0.3372 kilojoules per kilogram) equals 273.66 kilojoules per kilogram

y equals (x minus x1) over (x2 minus x1) times (y2 minus y1) plus y1 equals 267.33 kilojoules per kilogram (from Table A.12, interpolated)

W dot x over m dot gs equals h3 minus h2 equals 0.0157 kilograms per second