(with Q aus equals 65 kilowatts)
T equals integral T ds from s a to s e.

d h equals T d s plus V d p isobar.

d s equals d h over T.

Equals integral d h from s a to s e equals integral c p d T from T e to T a equals c p times (T a minus T e) over c p times ln(T a over T e).

Equals 298.1 minus 288.75 over ln(298.75 over 288.75) equals 293.72 Kelvin.

2 to 3 is isentropic  
s2 equals s3

A table is drawn with columns labeled: 'zustand', 'p [bar]', 'T', 'x', 's', 'h'. The rows are:

1 | | | | |  
2 | | | 1 | s2 equals s3 |  
3 | 8 | | |  
4 | 8 | 0 | | h equals h4 |

x3 equals

p equals 17: h7 equals hf (36 bar) equals 93.42 kilojoules per kilogram equals h1