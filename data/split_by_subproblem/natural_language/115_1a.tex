a) Given: Q dot out, with the assumption of a steady state process with the environment.

Zero equals m dot times bracket h2 minus h1 bracket plus sum Qi minus sum E Vi in minus KE and PE negligible.

m dot equals m dot in minus m dot out equals zero point three kilogram per second, sum E Vi in: water implies sum E Vi in equals zero.

Therefore, h2 equals h water, saturated (70 degrees Celsius), h1 equals h water, saturated (100 degrees Celsius).

Therefore, Q dot a equals sum Qi dot minus Q dot R minus Q dot out.

Therefore, Q dot out equals Q dot R plus m dot in times bracket h water, saturated (70 degrees Celsius) minus h water, saturated (100 degrees Celsius) bracket.

Table A2: h water, saturated (70 degrees Celsius): h2 equals 292.98 kilojoule per kilogram, h water, saturated (100 degrees Celsius): h1 equals 419.04 kilojoule per kilogram.

Therefore, Q dot out equals 100 kilowatt plus 0.3 kilogram per second times bracket 292.98 kilojoule per kilogram minus 419.04 kilojoule per kilogram bracket.

Minus 62.192 kilowatt equals Q dot out.

dot S_eq = M_dot times (s_a - s_e) minus dot Q_12 over T_12 minus dot Q_abs over T_F

equals 0.3 kilograms per second times (1.3009 minus 0.9549) kilojoules per kilogram Kelvin minus 100 kilowatts over 337.75 Kelvin minus (-62.782 kilowatts) over 793.122 Kelvin

equals 0.00497 kilowatts per Kelvin equals dot S_eq