A diagram is drawn with the y-axis labeled as T in Kelvin and the x-axis labeled as S in kilojoules per kilogram Kelvin. The diagram is a graph with points labeled 0, 2, 3, 4, 5, and 6 connected by lines. The line from point 0 to 2 is labeled mK, and the line from 2 to 3 is labeled as isobaric. The line from 3 to 4 is curved, and the line from 4 to 5 is labeled as isobaric. The line from 5 to 6 is straight. The line from 0 to 6 is labeled as mM, with an additional label of (HS) near the line. The pressure is marked as 0.191 bar.

Below the diagram is a table with columns labeled P, T, and W. The rows are numbered from 0 to 6.

Row 0: 
- P: 0.191 bar
- T: 243.15 Kelvin, -30 degrees Celsius
- W: 200 meters per second

Next to row 0, it is noted that Q0A equals 0, and mR over mM equals 5.235.

Row 5: 
- P: 0.5 bar
- T: 431.9 Kelvin
- W: 220 meters per second

Next to row 5, it is noted that qB equals 1.195 kilojoules per kilogram.

Row 6: 
- P: 0.191 bar

Next to row 6, it is noted that delta S equals 0.

Given w_c, T_b

T_c over T_5 equals (p_c over p_5) to the power of 0.4 over 1.4

T_6 equals T_5 times (p_6 over p_5) to the power of 0.4 over 1.4 equals 328.07 K

0 equals m_dot_ges times (h_5 minus h_6 plus (w_5 squared minus w_6 squared over 2)) plus d_adibant Q minus W_e_5z

equals c_p times (T_5 minus T_6)_s

w_c squared equals 2 times c_p times (T_5 minus T_6) plus w_5 squared equals 507.25 meters per second

O equals m dot parenthesis s zero minus s six parenthesis plus sum Q dot over T B plus S dot zero two

S zero two equals s six minus s zero minus ninety eight over T B equals c p natural logarithm T zero over T B minus ninety eight over T B equals negative six hundred twenty-five point seventy-three joules per kilogram rewrite positive sign

e x verl equals s zero two times T zero equals negative one hundred fifty-two point one four four kilojoules per kilogram