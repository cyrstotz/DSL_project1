First law of thermodynamics, reactor:
Zero equals mass flow rate times (enthalpy in minus enthalpy out) plus heat transfer rate plus work rate out.

Enthalpy in equals enthalpy at seventy degrees Celsius, pressure equals zero equals two hundred fifty-two point five eight kilojoules per kilogram.

Enthalpy out equals enthalpy at one hundred degrees Celsius equals four hundred forty-three point six four kilojoules per kilogram.

Work rate out equals mass flow rate times (enthalpy out minus enthalpy in) minus heat transfer rate.

Equals negative sixty-two point one eight two kilowatts.

1    2    delta 12  
T    100°C  70°C  20°C  
m    5755 kg  
x    0.005    0  

Q equals zero    W equals zero  

1. HS: minus m u2 minus m2 u2 minus m u1 equals delta m12 (h ein)  

u1 equals u f plus x (u s minus u f) equals 418.84 plus 0.005 (2506.5 minus 418.84) kilojoules per kilogram  
equals 423.38 kilojoules per kilogram  

u2 equals u f (70°C) equals 252.35 kilojoules per kilogram A2  
h ein equals h f (20°C) equals 83.36 kilojoules per kilogram A2  

M2 equals m1 plus delta m12  

delta m12 (h ein minus u2) equals m1 u2 minus m1 u1  

therefore, delta m12 (h ein minus u2) equals M1 u2 minus m1 u1  
therefore, delta m12 equals M1 u2 minus m u1 over (h ein minus u2)  
equals 1178 kilograms  

e) delta S12 equals M2 S2 minus m1 S1 equals delta m12 S ein plus S e12

The page contains a series of diagrams and equations related to thermodynamics, with labels in German. 

At the top of the page, there is a grid with numbers 1, 2, 3, 4, and letters T and P, followed by the numbers 8 bar and 8 bar, and the numbers 1 and 0.

Below is a phase diagram with pressure (p) on the vertical axis and temperature (T) on the horizontal axis. The diagram includes the following labels:
- "unterkühlte Flüssigkeit" (subcooled liquid)
- "Tripelpunkt" (triple point)
- "überhitzter Dampf" (superheated steam)
- "NΔGebiet (Nassdampf)" (wet steam region)

Further down, there is another diagram with pressure (p) on the vertical axis and temperature (T) on the horizontal axis, labeled with:
- "Fest" (solid)
- "Flüssig" (liquid)
- "Gas" (gas)
- "Tripelpunkt" (triple point)

Following the diagrams, there are two sections labeled b) and c).

b) 2 → 3:
Q equals m dot times (h2 minus h3) minus W dot K
h2 equals h2

c) E K equals Q dot K over W dot K
1 → 2: C equals m dot times (h1 minus h2) plus Q dot K