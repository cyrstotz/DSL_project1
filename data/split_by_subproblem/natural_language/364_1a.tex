E sub u equals absolute value of Q sub zu divided by absolute value of W sub e is greater than or equal to absolute value of Q sub a s minus absolute value of Q sub u divided by absolute value of W sub u

a) S1-FP: O equals m times che minus ha plus le dot plus pe dot plus sum of Q dot minus xi times v dot.

Q out equals m times (ha minus le).

Q out equals m ein times (h aus minus h ein).

h ein equals h water at seventy degrees Celsius equals two hundred thirty-two point eighty-eight kilojoules per kilogram.

h aus equals h water at one hundred degrees Celsius plus the energy four hundred nineteen point four kilojoules per kilogram.

Q out equals thirty-seven point nine kilowatts.

b) T bar equals integral of T ds over s a minus s e, keine Druckänderung, e integral of T ds equals q rev.

T bar equals q rev over s a minus s e equals c i dot times (T2 minus T1) over c i dot times ln (T2 over T1) equals T2 minus T1 over ln (T2 over T1) equals (two hundred eighty-eight point fifteen minus two hundred eighty-eight point fifteen) Kelvin over ln (two hundred eighty-eight point fifteen over two hundred eighty-eight point fifteen).

T bar equals two hundred ninety-three point twelve Kelvin.