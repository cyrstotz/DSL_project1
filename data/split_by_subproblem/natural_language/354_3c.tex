S sub T equals m sub 1 S sub f (400 degrees Celsius) plus m sub 2 S sub f (200 degrees Celsius)

Table A-2

S sub T equals m sub 1 S sub f (100 degrees Celsius) plus m sub 2 S sub f (200 degrees Celsius), 1 kilogram

equals m sub 1 (S sub f (100 degrees Celsius) plus x (S sub g 100 degrees Celsius minus S sub f (100 degrees Celsius)) plus m sub 2 S sub f (200 degrees Celsius)

equals m sub 1 (1.337 kilojoules per kilogram Kelvin) plus m sub 2 (0.2966 kilojoules per kilogram Kelvin)

S sub T (p) equals S sub f (T) equals 0.2966 kilojoules per kilogram Kelvin

S sub n equals 8.715 megajoules per kilogram

S sub 2 equals S sub f (700 degrees Celsius) equals 0.9599 kilojoules per kilogram Kelvin

S sub 2 equals m sub g S sub 2 equals 8.753 megajoules per kilogram

Delta S sub n2 equals S sub 2 minus S sub n equals 74 kilojoules per kilogram