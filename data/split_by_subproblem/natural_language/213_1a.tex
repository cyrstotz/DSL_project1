Energy balance:

m times (h sub e minus h sub a) plus Q sub R minus Q sub aus equals zero stationary

Q sub aus equals m times (h sub e minus h sub a) plus Q sub R

h sub e equals h sub f (70 degrees Celsius) equals 232.98 kilojoules per kilogram

h sub a equals h sub f (100 degrees Celsius) equals 419.04 kilojoules per kilogram

implies Q sub aus equals 0.3 times (h sub e minus h sub a) plus 100 kilowatts equals 62.182 kilowatts

Sign according to sketch,

b)

T dS equals S Q

T bar times (s sub 2 minus s sub a) equals q sub 12

a sub 12 equals h sub 2 minus h sub 1

T bar equals h sub 2 minus h sub a over s sub 2 minus s sub 1

h sub 2 minus h sub a equals c superscript if times (T sub 2 minus T sub 1)

s sub 2 minus s sub 1 equals c superscript if times natural log of (T sub 2 over T sub a)

implies T equals T sub 2 minus T sub a over natural log of (T sub 2 over T sub a) equals 283.1216 Kelvin