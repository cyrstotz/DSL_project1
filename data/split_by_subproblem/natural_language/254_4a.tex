b) h sub g equals h sub steam equals 140.73 kilojoule per kilogram, s sub g equals 0.2459 kilojoule per kilogram kelvin equals s sub 1, TAB A-11

W dot sub K equals m dot times (h sub 2 minus h sub 3) implies m dot equals W dot sub K over h sub 2 minus h sub 3

T sub 1 equals T sub 2 equals negative 10 degrees Celsius, h sub 2 equals 237.94 kilojoule per kilogram, TAB A-10

m dot equals 4.198 times 10 to the power of negative 4 kilogram per second

c) x sub 1 equals 0.3459 minus 0.1492 over 0.5298 minus 0.1492 equals 0.28, (Werte aus TAB A-10)

d) h sub 1 equals h sub f plus x sub 1 times (h sub g minus h sub f) equals 87.6822 kilojoule per kilogram, (Werte aus TAB A-10)

implies Q dot sub K equals m dot times (h sub 2 minus h sub 1)

h sub 2 equals h sub g equals 237.94 kilojoule per kilogram, TAB A-10

implies Q dot sub K equals 62.7 watt

E sub K equals Q dot sub K over W dot sub K equals 2.24