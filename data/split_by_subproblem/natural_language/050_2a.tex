A graph is depicted with an x-axis labeled 's' and a y-axis with no visible label. Several curves and lines are drawn on the graph:

- A line is marked from point '0' to point '1' with an annotation 'isobar'.
- Another line extends from point '2' to point '3' with the label 'adiabatic'.
- A vertical line is drawn from point '4' to point '5', labeled 'adiabatic'.
- Point '6' is annotated with '0.1871'.
- A note next to point '0' indicates '0 degrees Celsius'.
- A list of numbers from '0' to '6' is written next to the graph with corresponding annotations:
  - '0' is marked '0'.
  - '1' is marked '0'.
  - '2' is marked 'p2 equals p3'.
  - '3' is marked 'p4 less than p3'.
  - '4' is marked 'p4 equals p6'.
  - '5' is marked '0.5, 4, 3, 1.9 K'.
  - '6' is marked '0.1871'.

dE over dt equals zero equals m dot times (h zero minus h six plus (w zero squared minus w six squared over two)) plus q B minus W dot.  
Zero equals h zero minus h six plus (w zero squared minus w six squared over two) plus q B.  
w six equals square root of (2 times (h zero minus h six plus q B) plus w zero squared).  
equals square root of (2 times (c p times (T zero minus T six) plus q B) plus w zero squared).

q B equals 1195 kilojoules per kilogram.  
T zero equals 243.15 Kelvin.  
T six equals ....  
w zero equals 200 meters per second.

Diagram: A triangle with a point on the left labeled 'r' and an arrow pointing right labeled 'b'.  
Zero equals m dot times (s zero minus s six) plus q over T zero, ad.  
s zero minus s six minus zero equals c p times ln of (T S over T six) minus R times ln of (p S over p six) equals zero.  
T S over T six equals (p S over p six) to the power of ((k minus 1) over k).  
T six equals T S over (p S over p six) to the power of ((k minus 1) over k) times 328.1 Kelvin.

w b equals 20.