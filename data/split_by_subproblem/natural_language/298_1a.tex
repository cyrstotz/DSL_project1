a) Q aus?

1. HS um Wasser: dE/dt = m ein (h ein - h aus) + Q R + Q aus

=> Q aus = m ein (h aus - h ein) - Q R

h aus = h (Wasser, 100 degrees Celsius FL) = 443.04 kilojoules per kilogram (TAB A2)

h ein = h (Wasser, 70 degrees Celsius, 80 FL) = 292.88 kilojoules per kilogram (TAB A2)

Q aus = 0.2 kilograms per second (4.9 kilojoules per kilogram - 2.92 kilojoules per kilogram) - 1000 kilowatts

= -62.18 kilowatts (Vorzeichen negativ da h aus < h ein)

b) Redum 1: Q aus = 6.8 kilowatts weil

T m = integral from T a to T e of c ds over S a - S e

T d,1 = dH + p0 times V = h a - h e over S a - S e

Stoffmodell ideale FL: h a - h e = c times (T a - T e) + v (p 2 - p 1)

Stoffmodell '' : S a - S e = integral from T a to T e of c over T dt + 0.005

c ln (T a over T e)

= c times (T a - T e) over e to the power of c times ln (T a over T e)

= (288.15 Kelvin - 288.15 Kelvin) over ln (258.15 over 288.15) = -293.12 kilojoules

c) S setz im Wur zu rechnen und lichen, im manche flessich lin flusse

Erstprobieren im mache dT over dt = m über s2 + sum of Q i over s setz

=> S setz = -sum of Q über V = (Q aus over Reaktor) - (Q aus over TUK)

= -6 kilowatts (5 kilowatts) over 393.75 Kelvin (295 Kelvin) = 0.047 kilojoules per kilogram

T reaktor = konstant 100 degrees Celsius - T reaktor 1

Q aus ist negativ da Q i aus (less)

X E1,2

Gei, T5,2 equals 0.005 times L equals TEi,2

X Ei,1 equals mEi divided by mEW equals 0.6

Energy balance over the total system or given:

O equals sum m times (hKi) plus Q minus v squared divided by two kinetic energy

O equals delta E equals dU

O equals mEW times hK1 minus mEW times hK2 plus mg times (X2 minus X1 divided by XL)

u u u

O equals mEW times (hK1 minus hK2) plus mg times Cv times (T2,g minus T1,s)

ideal total

Ep equals R plus Cv equals Rg plus Cv equals 0.466 plus 0.033 equals 0.799 k divided by kg

hK1 equals UFL1 times XA times (UFL2 minus UFL)

hK2 equals UFL plus X2 times (UEi2 minus UFL)

minus O equals mEW times (UFL minus UA) times (UFL2 minus UFL) minus UFL minus X2 times (UFL2 minus UFL1) plus mg times Cv times (T2,g minus T1,g)

O equals mEW times (UFest minus UFL) times (X1 minus X2) plus mg times Cv times (T2,g minus T1,g)

O equals mEW times (UEi2 minus UFL) times X1 minus mCH times (UFL2 minus UFL) times X2 plus mg times Cv times (T2,s minus T1,s)

X2 equals X1 plus mg times Cv times (T2,s minus T1,s) divided by mEW times (UFest minus UFL)

UFEFL minus UFest times (0.003) equals minus 337.4 times UFL times (k divided by kg)

UFL1 equals UFL times (0.001) minus 0.007 times (k divided by kg)