sin | aus | kein | kann  
ṁ = 0.3 | 0.3  
T = 740 | 1000 | 288.15 K | 298.15 K  
x = 1 | 1  

a) ges: Qaus  
Energiebilanz  
h(x=1/100) = 292.18 kJ/kg  
h(x=1/1000) = 149.04 kJ/kg  
(Tafel 11c A-2)  

dE/dt = Σ ṁ hi + Q - Wi → (crossed out equation)  

0 = ṁ (he - ha) + QR - Qaus + Q → Qaus = ṁ (he - ha) + QR  

Qaus = 0.3 [292.18 - 149.04] + 100 kW = 62.18 kW  

b) ges: T̅KF  
∫ sa T ds = qrev = ha - he  

T̅KF = ∫ sa T ds / sa - se = ha - he / sa - se =  

(T2 - TA) / ln (T2/TA) = (p2/pA)  
sit. ln (T2/TA)  

T2 - TA / ln (T2/TA) = 298.15 - 288.15 / ln (298.15/288.15) =  

T̅UF = Δθ / ln (θ1/θ2) = 293.12  

c) ges: s̅se  
0 = ṁ [s̅se - s̅si] + Q̇ / T̅si + s̅se → s̅se = ṁ [s̅sa - s̅se] - Q̇aus / T̅j  

s̅se = 0.3 [7.35744 - 7.35533] = 62.18 / 293.12 = 0.121012 - 0.121232 = 0.33 kJ/kg  

s̅C1/2θ1 = 7.35533  
s̅C1/2θ2 = 7.35443 (Tafel A-2)  

mges = 37.55 kg  
xD = mP/mges = 0.005  
Treal = 100°C  
QD = 100 kW  
Qaus =

d) T1 equals one hundred degrees Celsius  
m1 equals five thousand seven hundred fifty-five kilograms  
Q12 equals thirty-five megajoules  

T2 equals seventy degrees Celsius  
T2,2ein equals twenty degrees Celsius (x equals A)  
Geist equals delta m12  

W equals zero  
Energiebilanz mit Q  

Q1 minus Q12,ein equals Q12  
m1 times cv times T1 minus m12 times cv times T2 equals Q12  

uvk (cv times T1 minus T2) verkuerzt minus u1 minus u2  

m1 times u1 minus m12 times u2 equals Q12  

m12 equals m1 times u1 minus Q12 divided by u2  
m12 equals five thousand seven hundred fifty-five times four thousand two hundred forty-three point eight three minus thirty-five thousand divided by two thousand four hundred two point eight  

m12 equals forty point fourteen kilograms  

u1 (x equals one thousand equals zero point zero zero five) equals four thousand two hundred forty-three point eight three kilojoules per kilogram  
Tabelle A2  

u2 (x equals one thousand two hundred one) equals two thousand four hundred two point eight kilojoules per kilogram  

u1 plus uF plus x times (u16 minus u61) equals u1 plus uF plus zero point zero zero five times (two thousand five hundred sixty-six point four minus four thousand eight hundred sixty-one) equals four thousand two hundred forty-nine point three seven seven  

Tabelle A2  

e) gesi equals Sesi  

m2 times S2 minus m1 times SA equals m12 times S12 plus Ses2

A graph labeled "Aufgabe 4" with axes labeled "p" (pressure) and "T" (temperature) is shown. The graph depicts a thermodynamic cycle with points 1, 2, 3, and 4 marked. There are lines labeled "isobar" and "isentrop" connecting these points. The x-axis is labeled "T (K)" and the y-axis is labeled "p (1/bar)". The cycle involves processes labeled as "isobar", "isentrop", and "isotherm/isotrop".

A diagram shows a cycle with components labeled 1, 2, 3, and 4. The process between 1 and 2 is labeled "isobar", between 2 and 3 is "adiabatic", and 3 to 4 is "isobar". The diagram also indicates "reversible" and shows work done W_k = 28 kW.