Energy balance around the reactor is stationary.

Zero equals Q dot R minus Q dot out plus m dot in times (h e minus h a).

Q dot out equals Q dot R plus m dot in times (h e minus h a).

h e equals h f at twenty degrees Celsius plus x D times (h g at seventy degrees Celsius minus h f at seventy degrees Celsius).

Table A-2

h e equals 292.98 kilojoules per kilogram plus 0.005 times (2626.8 kilojoules per kilogram minus 292.98 kilojoules per kilogram).

h e equals 304.65 kilojoules per kilogram.

Analogous applies for h a.

h a equals h f at one thousand degrees Celsius plus x D times (h g at one thousand degrees Celsius minus h f at one thousand degrees Celsius).

Table A-2

h a equals 419.04 plus 0.005 times (2676.1 minus 419.04) kilojoules per kilogram.

h a equals 430.325 kilojoules per kilogram.

Page 7.

Substitute into Q dot out equals Q dot R plus m dot in times (h e minus h a).

Equals 100 kilojoules per second plus 0.3 kilograms per second times (304.65 minus 930.325) kilojoules per kilogram.

Equals 62.3 kilojoules per second equals Q dot out.

s sub erz equals Q dot aus divided by T sub KF minus Q dot acs divided by T sub Reaktor equals sixty-two point three kilojoules per second times open parenthesis one divided by two hundred ninety-five Kelvin minus one divided by three hundred seventy-three point one five Kelvin close parenthesis equals forty-four point three joules per Kelvin equals s sub erz