Variables 'w6' and 'T6' are mentioned. An equation is written as:

T subscript 6 equals T subscript 5 times the fraction (P subscript 0 divided by P subscript 5) raised to the power (n minus 1 divided by n) equals 328 K

The value of n is given as 1.4.

An equation is written as:

O equals m dot subscript gas times (h subscript 5 minus h subscript 6 plus w subscript 5 squared divided by 2 minus w subscript 6 squared divided by 2)

Another equation is simplified to:

O equals h subscript 5 minus h subscript 6 plus w subscript 5 squared divided by 2 minus w subscript 6 squared divided by 2

An expression for w subscript 6 is given as:

w subscript 6 equals the square root of 2 times (h subscript 5 minus h subscript 6 plus w subscript 5 squared divided by 2)

Another expression for w subscript 6 is given as:

w subscript 6 equals the square root of 2 times (C subscript P times (T subscript 5 minus T subscript 6) plus w subscript 5 squared divided by 2) equals 507.4 meters per second