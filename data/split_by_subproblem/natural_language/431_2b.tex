c) h two minus h four, equation adiabatic Drossel  
h two equals h f (eight bar) minus ninety-three point four two kilojoules per kilogram  

Ki equals h one minus h f (forty degrees Celsius) divided by h f (four degrees Celsius) minus h f (forty degrees Celsius) equals zero point one three nine

5 to 6 implies isentropic, therefore delta S equals 0.

0 equals c sub p times the natural logarithm of T sub 6 over T sub 5 minus the natural logarithm of P sub c over P sub 5.

Eta equals c sub p over c sub v equals c sub p over 2 equals 0.7186 kilojoules per kilogram Kelvin.

R equals c sub p minus c sub v equals 0.2874 kilojoules per kilogram Kelvin.

T sub 6 equals T sub 5 times e to the power of R over c sub p times the natural logarithm of P sub c over P sub 5.

T sub 6 equals 328.084 Kelvin.

0 equals m dot times h sub f minus h sub e plus w sub 5 squared minus w sub 6 squared over 2 plus q dot minus phi dot equals 0.

w sub 6 equals the square root of 2 times h sub 5 minus h sub 6 plus w sub 5 squared minus the square root of 2 times c sub p times T sub 5 minus T sub 6 plus c sub w5 squared minus 507.23 meters per second.