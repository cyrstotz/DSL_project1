A graph is drawn with an x-axis labeled 's [kJ/kgK]' and an unlabeled y-axis. Points are marked on the graph as 0, 1, 2, 3, 4, 5, 6, and S. The graph includes various lines and segments with annotations:

- Between points 0 and 1: 's' followed by a crossed-out section, then 'T'.
- Between points 1 and 2: 'isentrope' with 's subscript 1 equals s subscript 2'.
- Between points 2 and 3: 'isobar' followed by 'T'.
- Between points 3 and 4: 's'.
- Between points 4 and 5: 'p subscript 4 equals p subscript 5'.
- Between points 5 and 6: 's subscript 5 equals s subscript 6' with a crossed-out section.

Additional annotations on the graph include:
- Near point 3: 'p equals p subscript u equals p subscript 5'.
- Near point 5: 'p equals p subscript 4 equals p subscript 5'.
- Near point S: 's equals const'.
- Near point G: 's equals const'.