Reactor

a) From the energy balance:

dE/dt = Σ mi [hi(1) + kei(1) + pei(1)] + Σ Q̇i(1) - Σ win(i) = 0.

ṁ (hein - hanu) + Q̇a + Q̇ans = 0.

Q̇ans = (-Q̇a + ṁ (hanu - hein)).

From the TAB A-3:
hein = 1267 kJ/kg = hf(70°C)
hanu = hf(100°C) = 1407.6 kJ/kg.

Q̇ans = (-100 kW) + 0.3 kg/s (1267 kJ/kg - 1407.6 kJ/kg) = 57.82 kW.

b) From the entropy balance:

0 = ṁ (se - san) + Σ Q̇j/Tj + i Sers.

From the exergy balance:

0 = ṁ (he - ha) + (z - T₀/T₁) = 0.

T̅ = ṁ (he - ha)/ṁ (be se - san) = cw (Tein - Tans)/cw ln (Tein/Tans).

= (288.15 - 298.15) K/ln (288.15 K/298.15 K) = 293.12 K.

Reaktor

c) Delta m12 equals three thousand six hundred kilograms.

Energy at the turbine

a) (T_k)

A diagram is drawn with a horizontal axis labeled 's (kJ/kg K)' and a vertical axis. The diagram is a cycle with six points labeled 1 to 6, connected by lines. The segments are labeled as follows:
- 1 to 2: 'isobar'
- 2 to 3: 'adiabat s = const reversibel'
- 3 to 4: 'isobar p2 = p3'
- 4 to 5: 'adiabat irreversibel'
- 5 to 6: 'isobar'
- 6 to 1: 'adiabat s = const reversibel'

The diagram also includes the following annotations:
- Between points 4 and 5: 'p5 = 0.5 bar'
- Between points 0 and 6: 'p0 = p6 = 0.1916 bar'

b) From the energy balance

0 = m dot [h_e - h_a + (W_e squared - W_a squared) / 2] I + sum over j Q dot j - sum over n W dot en = 0.

Q dot j is crossed out.

W dot t = 0

q = h dot s over h dot j

W_e = W_muft = 200 m/s.

(W_e squared - W_a squared) / 2 = h_a - h_e q dot

= Cp (T_a - T_e) q dot

5 to 6 isentropic

T_6 over T_5 = (p_6 over p_5) to the power of n minus 1 over n

= (p_0 over p_5) to the power of 0.4 over 1.4

= (0.191 over 0.5) to the power of 0.4 over 1.4

= 0.76

T_6 = T_5 times 0.76 = 0.76 times 4.3 times 1.916 = 328.07 K