A diagram is drawn showing a container with forces labeled: 'Patm', 'mg', 'pi', and 'F'. There is a dimension labeled 'D = 10 cm'.

Fg equals m times g equals 31.319 newtons.

Fatm equals p times (5 cm)^2 times pi equals p times (0.005 m)^2 times pi equals 7.854 newtons.

Fp equals pi times (0.005 m)^2 times pi.

Force equilibrium:

Fp equals Fg plus Fatm implies pi equals Fg plus Fatm over (0.005 m)^2 times pi equals 40.96 times 10^5 pascals.

pi equals 40.96 bar.

R over M equals 8.314 joules per mole-kelvin over 50 kilograms per kilomole equals 0.16628 joules per kilogram-kelvin.

p times Vi equals m times R times Ti.

m equals pi times Vi over R times Ti equals 40.96 times 10^5 pascals over 0.16628 joules per kilogram-kelvin times 773.15 kelvin equals 99.127 kilograms.

- Because I thought my pressure was wrong, I continue calculating with the given values.