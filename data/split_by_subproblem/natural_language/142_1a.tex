mass equals 1.5755 kilograms, velocity equals 0.3 meters per second, Q dot R equals 100 kilowatts

velocity equals 100 degrees Celsius, temperature in equals 70 degrees Celsius, q equals 292.99 kilojoules over kilograms

pressure equals 10 bar, pressure in equals 0.3713 bar, x p equals 0.005

reaction equals 400 degrees Celsius, L a equals 40.04 kilojoules per kilogram from 1 plus 2

temperature F in equals 293.15 Kelvin

temperature F out equals 293.15 Kelvin

a) The absorbed heat quantity stationary process

Q equals m dot times (q e minus q a) plus Q dot R minus Q dot out

Q out equals 62.182 kilowatts, which leads to delta k plus T, Q dot out over T

b) Thermodynamic mean temperature. q e plus s of T 2 minus s of T 1 equals integral from T 1 to T 2 of c over T dt

c) Q equals m dot times (q e minus q a) plus Q dot out over T plus S dot out

T one equals seventy degrees Celsius implies delta T equals thirty Kelvin.

Q R minus Q aus equals thirty-five megajoules.

U one four hundred degrees Celsius equals one hundred forty-eight point nine four kilojoules per kilogram.

U two four hundred degrees Celsius equals two thousand nine hundred sixty-six point five two kilojoules per kilogram.

C p equals psi plus zero point zero zero five times (U two minus U one) equals four point two five three seven three eight kilojoules per kilogram.

U two equals U one (seventy degrees Celsius) equals two hundred ninety-two point three five kilojoules per kilogram.

U three equals ninety-three point nine five kilojoules per kilogram.

Arrow to equation.

m gas two times U two plus U three m gas two plus U three minus U two equals Q aus.

Equals m gas two plus delta m times U two equals m gas one U one minus delta m U three equals Q aus.

Equals Q aus minus m gas two times U two plus m gas one U one equals zero times ln (U two minus U three).

Delta m equals Q aus minus m gas two U two plus m gas one U one divided by U two minus U three.

Delta m equals three thousand five hundred eighty-nine point one zero six kilograms.

m A (T equals one hundred degrees Celsius) equals m A plus x times (m B minus m F).

m B (T equals one thousand degrees Celsius) equals seventy-three point zero zero five kilograms.

m B (T equals seven hundred degrees Celsius) equals seven point three five zero zero kilograms.

m A (T equals seven hundred degrees Celsius) equals zero point five six zero kilograms per kilogram.

m gas equals m gas star times delta m.

m two equals m gas star.