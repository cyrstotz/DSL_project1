Q equals sigma e x sub str plus sigma e x sub q minus Z times W minus a v divided by P sub c times a q minus t k u r l

zero equals m times g times (S sub 0 minus S sub a) plus Q divided by T times g plus S times l times z

S sub e x z equals S sub G minus S sub 0, since ideal gas

equals S superscript circle times (T sub G) minus S superscript circle times (T sub c) minus R times ln (P sub G divided by P sub 0)

c, since P sub G equals P sub c

dot e sub r x sub vert equals T sub 0 times S sub e x z

s of T1 equals s of T2

s of 1240 Kelvin, s of 250 Kelvin, interpolate to s of 6243.15 Kelvin equals 243.17 kilojoules per kilogram Kelvin

s of 325 Kelvin, s of 330 Kelvin, interpolate to s of 328.716 Kelvin equals 328.33 kilojoules per kilogram Kelvin

s of zero equals 85.22 kilojoules per kilogram Kelvin

exergy equals T zero times s zero equals 20.723 megajoules per kilogram