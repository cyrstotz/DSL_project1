A square diagram is drawn with the following annotations:
- m2, 200 degrees Celsius, x equals 0
- T2 equals 100 degrees Celsius
- mg1 equals 97 kilograms
- x0 equals 0.005
- T1 equals 70 degrees Celsius
- L equals 0

Below the diagram, the following equation is noted:
Q equals 35 megajoules

Halboffenes System

m2 times u2 minus m1 times u1 equals delta m times hm plus Q minus V squared over 2

hm (TAB A2 bei 20 degrees Celsius x equals 0) equals 83.367 kilojoules per kilogram

u2 (TAB A2 bei 70 degrees Celsius x equals 0) equals 292.95 kilojoules per kilogram

u1 (TAB A2 bei 100 degrees Celsius x equals 0.005)

u1 equals 1.0435 plus 0.005 times (1.673 minus 1.0435) equals 1.04664 kilojoules per kilogram

m2 equals m1 plus delta m

(m1 plus delta m) times u2 minus m1 times u1 equals delta m times hm plus Q

m1 times u2 plus delta m times u2 minus m1 times u1 equals delta m times hm plus Q

delta m times u2 minus delta m times hm equals Q plus m1 times u1 minus m1 times u2

delta m equals Q plus m1 times u1 minus m1 times u2 divided by u2 minus hm equals (35000 kilojoules) plus 5755 kilojoules divided by 292.55 minus 83.36

delta m equals 8205.67 kilograms