Qaus

m times (hein minus ha) plus QR plus Qaus equals zero

Water  
hein 70 degrees Celsius  
A equals 2 at 70 degrees Celsius  
siedend arrow Nassdampf  
with x0 equals 0.005

he equals hf (70 degrees Celsius) plus x0 times (hg (70 degrees Celsius) minus hf (70 degrees Celsius))  
equals 292.98 kilojoules per kilogram plus 0.005 times (2626.8 kilojoules per kilogram minus 292.98 kilojoules per kilogram)  
equals 309.65 kilojoules per kilogram

ha equals hf (100 degrees Celsius) plus x0 times (hg (100 degrees Celsius) minus hf (100 degrees Celsius))  
equals 419.04 kilojoules per kilogram plus 0.005 times (2676.1 kilojoules per kilogram minus 419.04 kilojoules per kilogram)  
equals 430.33 kilojoules per kilogram

m dot equals 0.3 kilograms per second

0.3 kilograms per second times (309.65 kilojoules per kilogram minus 430.33 kilojoules per kilogram) plus 100 kilowatts plus Qaus equals zero

Qaus equals minus 62.296 kilowatts

Arrow: Wenn Pfeil anders herum zeigt  
Arrow: 62.30 kilowatts fliesst heraus  
equals Qaus

In der ganzen Aufgabe  
habe ich für den Begriff  
siedend mit einem dampfanteil  
x von 0.005 gerechnet

m1 equals five thousand seven hundred fifty-five kilograms  
X1 equals zero point zero zero five  
T1 equals one hundred degrees Celsius  
Tein equals twenty degrees Celsius  

m2 equals m1 plus delta m  
delta m  
T2 equals seventy degrees Celsius  

m2 u2 minus m1 u1 equals omhein plus Qums12  

A-Z at one hundred degrees Celsius  
u1 equals u4 at one hundred degrees Celsius plus Xb times (hg at one hundred degrees Celsius minus u4 at one hundred degrees Celsius)  

equals zero point zero two five times (two thousand four hundred sixty-nine point six minus four hundred twenty-nine point ninety-five)  

equals four hundred eighteen point ninety-nine kilojoules per kilogram plus zero point zero zero five times (two thousand five hundred sixty point five kilojoules per kilogram minus four hundred eighteen point ninety-nine kilojoules per kilogram)  
equals four hundred twenty-nine point thirty-eight kilojoules per kilogram  

at seventy degrees Celsius  
u2 equals two hundred ninety-two point ninety-five kilojoules per kilogram plus zero point zero zero five times (two thousand four hundred ninety-nine kilojoules per kilogram minus two hundred ninety-two point ninety-five kilojoules per kilogram)  
equals three hundred three point eighty-three kilojoules per kilogram  

hein at twenty degrees Celsius  
hein equals hf at twenty degrees Celsius plus X0 times (hg at twenty degrees Celsius minus hf at twenty degrees Celsius)  
equals seventy-nine point seventy-seven kilojoules per kilogram plus zero point zero zero five times (two thousand five hundred thirty-six point two kilojoules per kilogram minus seventy-four point seventy-seven kilojoules per kilogram)  
equals eighty-seven point zero eight kilojoules per kilogram  

m1 u2 minus m1 u1 plus delta m u2 minus delta m hein equals Qums  

delta m equals Qums plus (m1 times (u1 minus u2)) divided by (u2 minus he)  

equals three thousand four hundred ninety-five kilograms

A graph is drawn with the vertical axis labeled 'T' and the horizontal axis labeled 'S'. The curve starts from the lower right and moves upward to the left, with points marked '1', '2', '3', and '4'. The segment between points '3' and '4' is labeled 'isentrop'.