dS over dt equals the sum of m sub i S sub i plus the sum of Q sub i over T sub i plus S with a dot sub erz.

S with a dot sub erz equals m times (S sub aus minus S sub ein) plus Q sub ab over T sub job.

Equals m times c with a dot times ln(T sub aus over T sub ein) plus 65 kilowatts over 295 Kelvin.

S with a dot sub erz equals m with a dot times c with a dot times ln(288.15 Kelvin over 285.15 Kelvin) plus 0.22 kilojoules over kilograms Kelvin.

exergy stream equals bracket h minus h zero minus T zero bracket s minus s zero bracket plus kinetic energy plus potential energy bracket

exergy stream zero equals bracket

exergy stream c minus exergy stream zero equals bracket h zero minus h zero minus T zero bracket s e minus s c bracket plus omega squared divided by two minus omega zero squared divided by two bracket

h c minus h zero equals c p bracket T c minus T zero bracket equals one point zero zero six kilojoules per kilogram Kelvin bracket three hundred twenty-eight point zero seven Kelvin minus two hundred ninety-three point one five Kelvin bracket equals eighty-five point four two nine five kilojoules per kilogram

s c minus s c equals c p bracket ln bracket T c divided by T zero bracket minus R ln bracket p c divided by p zero bracket equals one point zero zero six kilojoules per kilogram Kelvin ln bracket three hundred twenty-eight point zero seven Kelvin divided by two hundred ninety-three point one five Kelvin bracket minus zero point two eight five six kilojoules per kilogram Kelvin ln bracket four point nine bar divided by one point zero bar bracket

M air equals twenty-eight point nine seven kilograms per kilomole (from table A-1)

R equals R divided by M air equals zero point two eight five six kilojoules per kilogram Kelvin

s c minus s c equals zero point three zero one three kilojoules per kilogram Kelvin

delta exergy stream equals exergy stream c minus exergy stream zero equals bracket eighty-five point four two five five kilojoules per kilogram minus two hundred ninety-three point one five Kelvin bracket zero point three zero one three kilojoules per kilogram Kelvin bracket plus bracket two hundred twenty point zero meters per second squared divided by two minus two hundred meters per second squared divided by two bracket

delta exergy stream equals minus four thousand three hundred fifteen point six seven kilojoules per kilogram

To solve this task, we need the energy balance for a closed system, where dE/dt equals the sum of (m_i times [h_i plus ke_i plus pe_i]) plus the sum of Q_i minus the sum of W_n.

System: (a drawing labeled "systemgrenze" is present)

mu_1 minus mu_2 equals Q_12

u_1 equals u_flüssig (1.4 bar) plus x_eis,1 times (u_fest (1 bar) minus u_flüssig (1.4 bar))

u_flüssig (1.4 bar) equals -0.045 kilojoules per kilogram

u_fest (1 bar) equals 333.458 kilojoules per kilogram

u_1 equals -200.0928 kilojoules per kilogram

u_2 equals

m times (u_2 minus u_1) equals Q dot (according to energy balance; considered with system gas)

u_2 gas minus u_1 equals c_v times (T_2 minus T_1) equals c_v times (0 degrees Celsius minus 500 degrees Celsius) equals 0.632 kilojoules per kilogram times (-500 Kelvin) equals -331.5 kilojoules per kilogram

m dot gas times (u_2 minus u_1) equals Q

Q dot equals -1.134 kilojoules (this is how much heat is extracted from the gas)

d e over d t equals the sum of m times h plus k times p e plus j plus the sum of Q j minus the sum of W n

Q k equals m times h two minus h one