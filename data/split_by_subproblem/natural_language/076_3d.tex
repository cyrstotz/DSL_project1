T peak 1 equals 30 degrees Celsius

Delta m 12, T equals 20 degrees Celsius

Q R 12 equals 35 MJ

Half-space system:

Delta E equals Delta U equals m 2 u 2 minus m 1 u 1 not equal Delta m 12 times (h q ein) does not equal Q ein 12

Assumption:

m 2 equals m 1 plus Delta m

u 1 equals u f (100 degrees Celsius) plus x 0 times (u g (20 degrees Celsius) minus u f (100 degrees Celsius))

u 2 equals u f (30 degrees Celsius) plus x 0 times (u g (30 degrees Celsius) minus u f (30 degrees Celsius))

h q ein equals u f (20 degrees Celsius) plus x 0 times (u g (120 degrees Celsius) minus u f (20 degrees Celsius))

Delta e equals minus Delta e subscript exergy plus q subscript b (1 minus T subscript 0 divided by T subscript 6) minus K subscript V superscript 0 minus e subscript x invol equals Lexi Q  

e subscript x invol equals minus Delta e subscript exergy plus q subscript b (1 minus T subscript 0 divided by T subscript 6)  

e subscript x invol equals minus 725.42 kilojoules per kilogram plus 1195 kilojoules per kilogram (1 minus 263.15 divided by 293) equals 849.76 kilojoules per kilogram  

m squared divided by s squared equals N times m  

N equals V subscript G times m divided by S subscript 2  

J equals V subscript G times m squared divided by S squared

X Eis 2 equals U 2 minus U F 1 (7.4 bar) divided by U Q (7.4 bar) minus U F (7.4 bar) subscript Fl.

U F (7.4 bar) equals negative 0.945 kilojoules per kilogram.

U G (7.6 bar) equals negative 333.585 kilojoules per kilogram.

Epsilon k equals absolute value of Q 2 u over absolute value of Q a 1 minus absolute value of Q 2 u equals absolute value of Q 2 u over absolute value of Q k.

Q 2 u equals Q k.

Epsilon k equals absolute value of Q k over absolute value of Q a 1 minus absolute value of Q k.