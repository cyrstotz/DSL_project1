a) Harrot stationary flow process

Q equals m dot times (h at seventy degrees Celsius minus h at one hundred degrees Celsius) plus Q dot Z plus Q dot loss

Q dot loss equals m dot times (h at one hundred degrees Celsius minus h at seventy degrees Celsius) minus Q dot Z

TAB A-2

equals zero point three kilograms per second times (418.04 kilojoules per kilogram minus 293.88 kilojoules per kilogram) minus 100 kilojoules per second

equals negative 62.18 kilowatts

b)

T equals integral from T1 to T2 of T dT divided by sA minus sL equals 9kW minus 2k divided by sL minus sC times ln of T2 over T1

T2 minus T1 over ln of T2 over T1

T2 equals 292.15 Kelvin

TA equals 293.15 Kelvin

equals 233.12 Kelvin

c)

S dot erz equals m dot times (sC minus sA) times Q dot R over 100 degrees Celsius

minus m dot times (zero) times Q dot R over 100 degrees Celsius

equals negative 0.268 kilowatts per Kelvin

Since we have done the calculation for water, we have to adjust the signs

S dot erz equals 0.268 kilowatts per Kelvin

Q two one equals m times (U two minus U one)

U two equals Q two one divided by m plus U one

equals 1500 joules divided by 0.4 kilograms plus (-133.1262 kilojoules per kilogram)

equals -118.1262 kilojoules per kilogram

U x equals -333.458 plus 0.6 times (-0.0015 minus (-333.458)) ENTA B A

equals -133.1262 kilojoules per kilogram

INTERPOLIEREN MIT EW TAB A MIT T equals 0.008 degrees Celsius

x equals U two minus U test divided by U flüssig minus U test equals 0.645