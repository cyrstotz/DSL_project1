Q dot out

implies Q dot gross minus m dot in (h out minus h in) equals Q dot n minus Q dot out

h out approximately equals h f (100 degrees Celsius) equals 419.04 kilojoules per kilogram

h in approximately equals h f (70 degrees Celsius) equals 293.59 kilojoules per kilogram (from TA-2)

implies Q dot out equals Q dot n plus m dot in (h in minus h out) equals 100 kilojoules plus 0.3 kilograms per second (293.59 minus 419.04) kilojoules per kilogram

equals 62.1 kilojoules

Delta S one two equals Delta M one two S W two zero degrees Celsius minus Q dot out one two over T W F  

equals S W two zero degrees Celsius equals zero point two five six kilojoules per kilogram per Kelvin (T A minus two)  

Delta S one two equals three one two seven kilograms times zero point two five six kilojoules per kilogram per Kelvin minus three five zero zero zero kilojoules over two nine three point one five Kelvin equals eight zero eight point one kilojoules per Kelvin