O equals m dot ein (h ein minus h aus) plus Q dot R minus U dot R zero plus Q dot aus.

h (70 degrees Celsius) equals h ein equals 292.98 kilojoules per kilogram, tab A-2.

h (100 degrees Celsius) equals h aus equals 419.04 kilojoules per kilogram, tab A-2.

Q dot aus equals m dot ein (h aus minus h ein) minus Q dot R equals 0.3 (419.04 minus 292.98) minus 400 equals minus 62.48 kilowatts.

D_ex,STR = 97,43 - 243,15 (0,3373) + 510 squared over 2 - 200 squared over 2

= 125469,5 Joules per kilogram = 125,47 kilojoules per kilogram

a) Diagram with axes labeled P and T.

b) S sub 2 equals S sub 3

O equals m dot R 134 a times (h sub 2 minus h sub 3) plus Q dot 23 minus L dot 23

L dot 23 equals L dot K equals minus 28 W

m dot R 134 a equals k sub A equals 7.111 times 10 to the power of negative 3

O equals m dot R 134 a times (h sub 2 minus h sub 1)

T sub 2 equals minus 22 degrees Celsius equals 251.15 Kelvin

S sub 3 equals S sub 2 equals S at T equals 22 degrees Celsius equals minus (0.9089 minus 0.9102) divided by (24 minus 22) plus 0.9089 equals 0.90955 kJ per kg Kelvin

TAB A equals 10