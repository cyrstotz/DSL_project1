Q̇aus equals question mark

Energy balance:

Zero equals ṁein (ḣein minus ḣaus) plus Q̇aus plus Q̇R minus U̇T

Q̇aus equals ṁein (ḣaus minus ḣein) minus Q̇R

ḣaus (100 degrees Celsius) equals ḣg equals 419.04 kilojoules per kilogram

ḣein (70 degrees Celsius) equals ḣf (70 degrees Celsius) equals 292.98 kilojoules per kilogram from Table A2

Q̇aus equals minus 62.182 kilowatts

Negative from reactor as given

d) m two u two plus delta m twelve u two minus m one u one equals delta m twelve h in. m two u two minus m one u one equals delta m twelve (h in minus u two). Delta m twelve equals m two (u two minus u one) divided by (h in minus u two). Equals three thousand seven hundred fifty-six point eight four kilograms.

h in equals h f (twenty degrees Celsius) equals eighty-three point six kilojoules per kilogram.

e) Delta m equals three thousand kilograms.

Delta s equals m two s two minus m one s one equals delta m twelve s i p plus Q R minus Q res divided by T plus s c res.

s m two equals (m two plus delta m twelve) s two minus m one s one.

s one equals s f g (one hundred degrees Celsius) plus x one (s g (one hundred degrees Celsius) minus s f g (one hundred degrees Celsius)) equals one point three three seven seven kilojoules per kilogram Kelvin.

s f g (one hundred degrees Celsius) equals one point three zero six kilojoules per kilogram Kelvin. s five equals seven point three three eight nine kilojoules per kilogram Kelvin.

s two equals s f (seventy degrees Celsius) equals zero point nine five eight nine kilojoules per kilogram Kelvin.

s m twelve equals seven thousand two hundred thirty-seven point eight five kilojoules per kilogram Kelvin.

A graph is labeled with 'P [E-P]' on the vertical axis and 'T [°K]' on the horizontal axis. There is a blue, four-sided figure drawn on the graph with points labeled 1, 2, 3, and 4. The segments between points 1 and 2, and points 3 and 4 are marked as 'isober'. Another smaller, similar shape is drawn to the right of the main graph.