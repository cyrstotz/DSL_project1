Adiabatic discharge is not possible, therefore h sub 3 equals h sub 2.

h sub p equals h sub 2 times A sub 17 minus h sub 2 equals 93.42 times q times kilojoules per kilogram equals h sub 1.

P sub 2 equals T sub 2. P sub 2 equals T sub 2 times A sub 10 equals T sub 2 minus 16 times q times kilojoules per kilogram equals 9.97 times q times kilojoules per kilogram.

h sub c equals (2.9 times q times kilojoules minus 2.9 times q times kilojoules) over (1.5 times 9.7 times q times kilojoules minus 1.7 times 2.5 times q times kilojoules) equals 2.9 times q times kilojoules.

h sub 7 equals 93.97 times q minus 32.05 times q over 7.6 times q minus 1.4 times q equals (1.5 times 9.7 times q minus 1.4 times q) over 2.7 times 7.263 times q equals 2.7 times 7.263 times q.

X sub 1 equals h sub r minus h sub 7 over h sub 7 minus h sub f. X sub 1 equals 93.62 minus 92.27 times q over 2.7 times 7.263 minus 2.9 times 2.8 times q equals 0.3677.