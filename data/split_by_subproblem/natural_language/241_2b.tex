A graph is drawn with an x-axis labeled "[L] over [kg] times [s]" and a y-axis labeled "[degrees Celsius]". The y-axis has markings at 67 and -30. The graph includes several points labeled:

- Point labeled "micst restrop" near -30 on the y-axis.
- Points labeled 1, 2, 3, 5, and 6 along two curves.
- Point 3 has a note "isentrope".
- Point 6 has a note "P equals 0.191".

S subscript 5 equals S subscript 6

S with T equals 431.9 Kelvin

S subscript 0 minus S subscript 5 equals 0 equals S degree with T subscript 6 minus S degree with T subscript 5 minus R times ln with P subscript 6 over P subscript 5

S degree with T subscript 6 equals S degree with T subscript 5 plus R over M times ln with P subscript 6 over P subscript 5

S degree with T subscript 5 with 431.9 Kelvin equals 21.06533 Joules over kilogram Kelvin

TAB A-22

equals 21.06977 Joules over kilogram Kelvin

S degree with T subscript 6 equals 21.06977 Joules over kilogram Kelvin

TAB A-1

equals 1.7936 Joules over kilogram Kelvin

T subscript 6 with S degree with T subscript 6 equals 325 Kelvin plus 330 minus 325 Kelvin over 1.7936 minus 1.78249 Joules over kilogram Kelvin times 1.7936 minus 1.78249 Joules over kilogram Kelvin

equals 328.62 Kelvin