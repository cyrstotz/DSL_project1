I assume m dot equals 4 kilograms per hour for safety equals 4 over 3600 kilograms per second  

0 equals m dot (h1 minus h2) plus Q dot K  

minus Q dot K over m dot plus h2 equals h1  

h3 equals 271.3 kilojoules per kilogram  

3 to 4 isobaric 2y: 8 bar x equals 0 h4 equals hf (8 bar)  

h4 plus m equals 93.42 kilojoules per kilogram  

m dot (h3 minus h4) plus Q dot equals 0  

Q dot equals 0.198 kilowatts  

Aufgabe Kreisprozess: W plus E dot ab equals 0

e sub x comma site number minus e sub x comma pstro equals h sub 6 minus h sub o minus T sub b divided by u S sub 6 plus T sub o divided by u S sub o plus k e sub 6 minus k e sub o.

h sub 6 minus h sub o equals C sub p multiplied by (T sub 6 minus T sub o).

S sub o minus S sub 6 equals C sub p multiplied by ln(T sub o divided by T sub 6) minus R ln(p sub o divided by p sub 6) with a crossed-out zero.

k e sub 6 minus k e sub o equals one-half multiplied by (W sub 6 squared minus W sub o squared).

S sub e equals C sub p multiplied by (T sub 6 minus T sub o) plus T sub o divided by u multiplied by C sub p ln(T sub o divided by T sub 6) plus one-half multiplied by (W sub 6 squared minus W sub o squared).

C sub p equals 1.006.

T sub 6 equals 328.1, T sub o equals 293.15 equals T sub u.

W sub 6 equals 507.2.

W sub o equals 200.

S sub e comma x comma tr equals 120.8 kJ divided by kg.