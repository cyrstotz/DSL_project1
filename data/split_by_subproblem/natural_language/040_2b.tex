The task is labeled 'ges: ω6, T6'. Below, there is a statement about 'isentrope Schutzbe, Us, ps, T6 bekannt'. An equation is crossed out: Q equals M dot (hs minus h8) plus (ωs squared minus ω6 squared divided by 2) equals ωv. 

Another equation follows: M dot equals M dot ges. 

An exergo-bilanz equation is written: Q equals M dot ges (hs minus h6 plus Δke), Δke equals ωs squared minus ω6 squared divided by 2. 

A reference to hs is given: hs to TAB Azz equals h (930K) plus h (990K) minus h (952K) divided by 10K, 1.9K.

Zeta subscript e x s t r o g equals e subscript x s t r o g minus e subscript x s t r o equals m dot open bracket h subscript 6 minus h subscript o minus T subscript o open bracket s subscript 6 minus s subscript o close bracket plus delta k o close bracket

h subscript 6 equals three hundred thirty-three point nine eight kilojoules per kilogram, h subscript o equals h open bracket two hundred sixty-five point one five Kelvin close bracket

s subscript 6 equals s open bracket T subscript 6 close bracket, s subscript o equals s open bracket T subscript o close bracket, r k o equals omega subscript 6 squared minus omega subscript o squared over two

s open bracket T subscript 6 close bracket approximately equals one point seven nine kilojoules per kilogram Kelvin, s subscript o approximately equals one point eight six kilojoules per kilogram Kelvin

h subscript o equals two hundred sixty-three kilojoules per kilogram

delta e subscript k s t r equals e subscript x s t r over m dot ges equals one hundred ninety-one point one nine kilojoules per kilogram, one hundred nineteen point zero nine kilojoules per kilogram