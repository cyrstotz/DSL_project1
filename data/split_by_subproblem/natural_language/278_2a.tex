Q equals m dot times (h in minus h out) plus Q dot R minus Q AB. Q dot AD equals Q dot R plus m in times (h in minus h out) equals 100 kilowatts plus 0.3 kilograms per second times (292.98 minus 419.04) kilojoules per kilogram equals 62.18 kilowatts. h in equals h at 70 degrees Celsius equals 292.98 kilojoules per kilogram. h out equals h at 70 degrees Celsius equals 419.04 kilojoules per kilogram.

0 to 1 adiabatic, Ventility  
1 to 2 isentrop  
2 to 3 isobar  
3 to 4 adiabatic, isentrop  
4 to 5 mixing, isobar  
5 to 6 isentrop  

q equals p u minus p a equals 0.5 kilowatt  

a)  

Diagram:  
- A graph labeled T L K T on the y-axis and s (kilojoule per kilogram Kelvin) on the x-axis.  
- Points marked from 0 to 6.  
- Arrows and lines connecting the points with labels:  
  - 0 to 1 labeled adiabatic  
  - 1 to 2 labeled isentrop  
  - 2 to 3 labeled isobar  
  - 3 to 4 labeled isentrop  
  - 4 to 5 labeled isobar  
  - 5 to 6 labeled isentrop  
- Dotted lines labeled p 3 equals p 2 and p 6 equals p 0  

b)  

m dot e, T 6 ?  

5 to 6 reversible and adiabatic to isentrop, s 5 equals s 6  

T 6 equals T 5 times (p 6 over p 5) to the power of kappa minus 1 over kappa equals 431.3 Kelvin times (0.491 over 0.5) to the power of 0.4 over 1.4 equals 328.07 Kelvin  

st. F P.  

O equals m dot e times (h o minus h c plus omega squared minus omega c squared over 2) plus q dot adiabatic plus W dot t plus 90 m c  

omega e squared equals q dot e plus 2 m dot b (h o minus h c) plus omega o squared equals 2 C p water (T o minus T c) plus v in plus 90 over omega dot e over 6.283  

omega e equals square root of 2 times 1.00 kilojoule per kilogram Kelvin (243.15 Kelvin plus 328.07 Kelvin) plus (200 meters per second) squared equals 459.2 meters per second