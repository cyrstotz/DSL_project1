Given data:
omega subscript 0, T subscript 0 gas.
omega subscript 0 equals 220 meters per second.
P subscript 5 equals 0.5 bar.
T subscript 5 equals 431.9 Kelvin.
P subscript c equals 0.141 bar.

Equation:
m dot squared (h subscript 5 minus h subscript 6 plus (omega subscript 5 squared minus omega subscript 6 squared) divided by 2) equals 0.

An expression is written:
C subscript L (T subscript 5 minus T subscript 0) plus (omega subscript 5 squared divided by 2 minus omega subscript 2 squared divided by 2) equals 0.

Additional calculations:
n equals 1.4.
Polytrop T subscript G equals T subscript 5 multiplied by (P subscript c divided by P subscript 5) raised to the power of 0.4 divided by 1.4, equals 328.07 Kelvin.

In a boxed section:
(omega subscript 5 squared divided by 2 equals (1.006 times (T subscript 5 minus T subscript 6) plus omega subscript 5 squared divided by 2). This results in values of 230.45 and 77.95. omega subscript 5 equals 220.7 meters per second divided by x subscript 4.