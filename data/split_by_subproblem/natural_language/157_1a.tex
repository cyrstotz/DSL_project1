m dot equals zero point three kilograms per second  
m gas equals five thousand seven hundred fifty-five kilograms  
Zero equals minus m dot times (h ein minus h aus) plus Q e plus Q aus minus arrow zero  
Q aus equals minus Q e plus m dot times (h aus minus h ein)  
h ein equals h f (seven hundred degrees Celsius) plus zero point zero zero five times (h g (seven hundred degrees Celsius) minus h f (seven hundred degrees Celsius)) equals three hundred four point six five kilojoules per kilogram  
h aus equals h f (one hundred degrees Celsius) plus zero point zero zero five times (h g f (one hundred degrees Celsius)) equals four hundred thirty point eight two five kilojoules per kilogram  
Thus Q aus equals minus sixty-two point two nine seven two kilojoules per second equals minus sixty-two point three kilojoules

Delta E equals m two u two minus m one u one equals Delta m two h 12 plus Q e 12 minus zero.

m two equals m one plus Delta m 12.

u two equals u w (70 degrees Celsius) equals 292.95 kilojoules per kilogram.

u one equals u w (100 degrees Celsius) equals 418.97 kilojoules per kilogram.

m one equals 5755 kilograms.

m two equals 5755 kilograms plus Delta m 12.

h 12 equals 83.96 kilojoules per kilogram.

From Table A2.

m one u two plus Delta m 12 u two minus m one u one equals Delta m 12 h 12 plus Q e 12.

Delta m 12 (u two minus u one) equals Q e 12 plus m one u one minus m one u two.

Delta m 12 equals 3500 kilograms.