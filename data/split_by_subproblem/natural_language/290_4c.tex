x nach Drossel?
P1 = P2, T2 = -16°C, P4 = 8 bar, x4 = 0
Δab = m (h4-h3) = 0.974 * 10 (93.42 - 266.497) = -1709 W
h4 (x = 0, p4 = 8 bar) = 95.42 [kJ/kg] (TAB-A11)
? wie auf T4 oder p4 kommen?

Q subscript 1 2 equals m subscript e v i n parentheses u subscript 1 minus u subscript 2 end parentheses equals 0.85 kilojoules.

Arrow u subscript 1 minus u subscript 2 equals 8.5 kilojoules per kilogram.

u subscript 1 equals u subscript e f parentheses 0 degrees Celsius end parentheses plus 0.6 parentheses u subscript e f superscript 0.045 plus 333.495 end parentheses equals negative 333.458 plus 0.6 parentheses negative 0.045 plus 333.495 end parentheses equals negative 133.4102 kilojoules per kilogram.

Arrow u subscript 2 equals negative 191.97 kilojoules per kilogram equals u subscript e f parentheses 0.005 degrees Celsius end parentheses plus x parentheses u subscript e f minus u subscript e g end parentheses.

Arrow x equals fraction bar negative 191.97 plus 333.458 divided by negative 0.053 plus 333.458 equals 0.574 in a box.