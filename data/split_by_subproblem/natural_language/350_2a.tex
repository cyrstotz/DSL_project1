The page contains several graphs and equations related to thermodynamics processes. 

Graph 1: A plot with temperature T on the vertical axis labeled in Kelvin [K] and entropy S on the horizontal axis labeled in kilojoules per kilogram Kelvin [kJ/kg·K]. The graph includes curves labeled with numbers 1, 5, and 6, and arrows indicating direction.

Graph 2: Another plot with temperature T on the vertical axis labeled in Kelvin [K] and entropy S on the horizontal axis labeled in kilojoules per kilogram Kelvin [kJ/kg·K]. This graph includes curves labeled as isobare 0.5 bar and isobare p0, with points labeled 1 and 5.

Graph 3: A third plot with temperature T on the vertical axis labeled in Kelvin [K] and entropy S on the horizontal axis labeled in kilojoules per kilogram Kelvin [kJ/kg·K]. This graph includes curves labeled p equals p2 equals p3 (isobaren), p equals 0.35 bar, and p0 equals 0.191 bar. Points are labeled 1, 2, and O. The graph mentions isentropen and wirkungsgrad.

A graph is depicted with the x-axis labeled as 'S' with units in joules per kilogram Kelvin (J/kg-K) and the y-axis labeled as 'T' with units in degrees Celsius (°C). 

The graph shows a cycle with points labeled 0, 1, 2, 3, 4/5, and 6. 

- Point 0 to 1 is labeled as 'Kompression' with pressures p_e and p_3 noted.
- Point 1 to 2 is labeled as 'isentrop' and 'adiabat' with pressures p_3 and p_5.
- Point 2 to 3 is labeled as 'Isobar'.
- Point 3 to 4/5 is labeled as 'inverses adiabatic'.
- Point 4/5 to 6 is labeled as 'adiabat, reversibel'.
- Point 6 to 0 is labeled as 'Isobar phantastrom'.

The area between points 0 and 1 is labeled as 'Isotherm'.