h f at 70 degrees Celsius equals 292.98 kilojoules per kilogram equals h e

h f at 100 degrees Celsius equals 419.04 kilojoules per kilogram equals h a

b)
T LKF equals question mark Kelvin

T bar equals integral from a to e of T dS over S a minus S e isobar equals du over S x minus S e equals d bracket T 2 minus T 1 end bracket over k ln bracket T 2 over T n end bracket equals 278.15 Kelvin minus 228.15 Kelvin over k ln bracket 278.15 Kelvin over 228.15 Kelvin end bracket

equals 293.72 Kelvin

c)
S dot erz equals question mark

0 equals m dot times bracket S e minus S a end bracket plus epsilon Q dot j over T j plus S dot erz

S dot erz equals m dot times bracket S a minus S e end bracket minus epsilon Q dot j over T j

equals 0.3 kilograms per second times bracket 1.3000 kilojoules per kilogram Kelvin minus 0.959 kilojoules per kilogram Kelvin end bracket minus 62.182 kilowatts plus 100 kilowatts over 293.72 Kelvin

equals 0.02344 kilojoules per Kelvin per second

S a: TAB 1.2.
S e: TAB A2.

A graph is drawn with an x-axis labeled 's' in kilojoules per kilogram Kelvin and a y-axis labeled 'T' in Kelvin. The graph includes points labeled 0, 1, 2, 3, 4, 5, and 6. There are lines connecting these points, with segments labeled 'isotrop' between 0 and 1, and between 1 and 2, 'isobar' between 2 and 3, 'isotrop' between 3 and 4, and 'isotrop' between 5 and 6.