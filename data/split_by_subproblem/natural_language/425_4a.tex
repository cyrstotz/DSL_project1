T E V V two equals T j two (das Gleichgewicht)  
(alle bei p equals n bar, T equals zero point zero zero zero two)  
X E i n one plus X E i s two equals u two minus u fest divided by u flüssig minus u fest equals zero point six four u  

Q n two equals delta U, da isochor  
u two minus u n equals Q n two divided by m g plus u n equals fifteen hundred joules divided by zero point zero zero three four kilograms plus (negative one hundred thirty seven u n kilojoules per kilogram) equals negative one hundred forty eight point nine u A kilojoules per kilogram  
alle bei p equals one point one bar, T equals zero degrees Celsius  

X E i s two equals zero point five five five

[Graph] The graph is a pressure (p in bar) versus temperature (T in degrees Celsius) diagram. It shows three phases: fest (solid), flüssig (liquid), and gasförmig (gaseous). The graph includes a line labeled (i) and a point labeled (ii) indicating the 'Tripel' (triple) point.