Zero equals mass flow rate times the enthalpy change plus two times heat transfer minus five times work input. Two times heat transfer equals six times heat out. Heat out equals heat rejected plus heat absorbed.

h with a subscript a equals four hundred seventy-nine point zero four plus zero point zero zero five times (two thousand six hundred seventy-six point one minus four hundred seventy-nine point zero four) equals four hundred thirty point three three kilojoules per kilogram.

h with a subscript b equals two hundred ninety-two point eight plus zero point zero zero five times (two thousand six hundred twenty-six point eight minus two hundred ninety-two point eight) equals three hundred four point six five kilojoules per kilogram.

Q out equals zero point three times (three hundred four point six five minus four hundred thirty point three three) plus four hundred equals positive sixty-two point three zero kilowatts.

T equals the integral from Sa to Se of T ds divided by Sa Se equals the integral from Ta to Se of T ds divided by the integral from Ten to T of c if T dt.

e sub k equals Q sub 2a divided by Q sub 0b minus Q sub 2a equals

c)

Q acc equals 65 kilowatts

T KF equals 295 kelvin

O equals m dot times open bracket S e minus S a close bracket plus xi times open bracket Q with dot above close bracket divided by F plus xi times S e r z

Balance im Reaktor

S e r z equals m dot times open bracket S a minus S e close bracket minus Q aus divided by T Reaktor

T Reaktor equals 100 degrees Celsius equals 373.15 kelvin

S a equals 1.3 times open bracket 69 plus 0.005 times open bracket 7.35 times 69 minus 1.30 times 69 close bracket equals 7.34 kilojoules per kilogram kelvin

S e equals 0.95 times 69 plus 0.005 times open bracket 7.7553 minus 0.95 times 69 close bracket equals 0.9 kilojoules per kilogram kelvin

S e r z equals 0.3 times open bracket 1.34 minus 0.9 close bracket minus 65 kilowatts divided by 373.15 kelvin equals 0.024 kilojoules plus watts per kilogram kelvin

a) Delta F equals the sum of m i (h i) plus one half q i equals m g2 d2 minus m ad1 equals m t plus Delta m n2 (d2 minus m n d1).

m ad1 plus Delta m n2 d2 minus m n d1 equals Delta m n2 (h n2) plus Q R n2.

Delta m n2 equals m n d2 minus m n d1 minus Q e n2 divided by h n2.

u1 equals 418.94 plus 0.005 (2566.5 minus 418.94) equals 429.38 kilojoules per kilogram.

a2 equals u (t 20 degrees Celsius) equals 292.85 kilojoules per kilogram.

h n2 equals h e (20 degrees Celsius) equals 83.96 kilojoules per degree.

Delta m n2 equals m n 5735 kilograms minus 292.85 kilojoules per kilogram minus 5735 kilograms 429.38 kilojoules per kilogram minus 35000 kilojoules divided by 83.96 kilojoules per kilogram minus 292.85 kilojoules per kilogram equals 33.46 kilograms.

e) Delta S equals m n2 s2 minus m s1.

Delta m n2 equals 3600 kilograms.

s1 equals 1.3069 plus 0.005 (7.3549 minus 1.3069) equals 1.34 kilojoules per kilogram per Kelvin.

s2 equals s (t 20 degrees Celsius) minus 1.0955 kilojoules per kilogram per Kelvin.

Delta S equals (5735 kilograms plus 3600 kilograms) 1.02 kilojoules per kilogram per Kelvin minus 5735 kilograms 1.34 kilojoules per kilogram per Kelvin equals 4830.4 kilojoules per kilogram per Kelvin.

E x level equals T zero S level

Theta equals m [S c minus S a] plus Q j over T j plus integral S e z

S e z over m equals S a minus S e minus Q B over T B equals integral from T two to T three c i p c T over T d T minus R ln (P B squared over P two squared)

equals c i p over T B minus Q B over T B equals minus 1295.81 kilojoules per Kelvin

E x level equals 243.13 kilojoules per Kelvin times 1295.81 kilojoules per Kelvin equals 315076.20 kilojoules