a) Energy balance around the reactor:

sum of h times p plus sum of Q minus sum of W equals zero.

m dot times (h in minus h out) plus Q two minus Q out equals zero.

Q out minus Q two times m dot times (h in minus h out) equals 62.182 kilojoules per kilogram.

Q two equals 100 kilowatts.

h in equals h one (700 degrees Celsius) equals 292.88 kilojoules per kilogram.

h out equals h one (100 degrees Celsius) equals 418.04 kilojoules per kilogram.

b) T average KF equals integral from s in to s out of ds divided by s out minus s in isobaric arrow T average KF equals h out minus h in divided by s out minus s in.

Ideal fluidity: T average KF equals integral from T one to T two of c p times dT divided by integral from T one to T two of c p divided by T times dT equals e to the power of (T two minus T one) divided by e to the power of (ln (T two divided by T one)).

T average KF equals T two minus T one divided by ln (T two divided by T one) equals 293.12 Kelvin.

c) Entropy balance around the reactor:

Q out divided by T average KF equals Q out divided by T out.

O equals m dot times (s two minus s one) plus Q divided by T plus S ex two.

Q out divided by T reactor minus Q out divided by T average KF plus S ex two equals zero.

S ex two equals Q out times (1 divided by T average KF minus 1 divided by T reactor).

S ex two equals 0.0455 kilojoules per kilogram Kelvin equals 45.5 joules per kilogram Kelvin.

T average KF equals 293.12 Kelvin.

T reactor equals 373.15 Kelvin.