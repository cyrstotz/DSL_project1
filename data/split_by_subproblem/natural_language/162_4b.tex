Zero equals m dot times h plus Q dot minus W dot.

Zero equals m dot times h plus Q dot minus W.

W equals m dot times the quantity h subscript 2 minus h subscript 3.

m dot equals W divided by the quantity h subscript 2 minus h subscript 3, equals.

h subscript 2 equals h subscript g times the quantity T subscript 2 minus phi, equals.

Isentropic: S subscript 3 equals S subscript 2.

The quantity P subscript 3 divided by P subscript 2 to the power of n minus 1 over n equals T subscript 3 divided by T subscript 2. Therefore, T subscript z equals T subscript 2 times the quantity P subscript 2 divided by P subscript 3 to the power of n minus 1 over 4.