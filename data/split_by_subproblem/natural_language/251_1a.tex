1. HS (Kühlerkreislauf): O equals m k f times (h k f ein minus h k f aus) plus Q aus
1. HS (Brenner): O equals m he times a h e times Q aus minus Q k equals Q trim times (h e minus h a) equals 62.482 kilowatts
h e equals h f (70 degrees Celsius) equals 292.98 kilojoules per kilogram (T A13 A2)
h a equals h f (100 degrees Celsius) equals 419.6 kilojoules per kilogram

A graph labeled "p [bar]" with a curve and a diagonal line is drawn. The diagonal line is labeled "Tripel" and has two points marked "i" and "ii". The graph is titled "Nach Wäg. 22-52h-45".