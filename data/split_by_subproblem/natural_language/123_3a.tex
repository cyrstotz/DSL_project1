| Zustand 1 | Zustand 2 |
|-----------|-----------|
| V₀ = 3 mL |           |
| T₀ = 500°C = 773.15 K |           |
| mₑₙ = 0.1 kg |           |
| Tₘₙ = 0°C |           |
| Xₑ = 0.6 |           |

mₖ = 32 kg / p₀ = 1 bar / D = 0.1 m / cᵥ = 0.633 kJ/kg·K / M₉ = 50 kJ/kmol

a) Given: p₃,₁ / m₉

Wasser interpretiert daher wie extra gewichtet!

F = p·A

A = (d²·π)/4

p₀·f / Tₘₖ / f₉,ₑ / Membran

p₀·A + m₉·g + mₑₙ·g = p₃,₁·A

→ p₃,₁ = (p₀·A + m₉·g + mₑₙ·g) / A = (1·0.05·0.1²·π/4 + 32·9.81 + 0.1·9.81) / (0.1²·π/4)

p₃,₁ = 785,383 + 343,62 + 0.981 = 1,140.984 Pa = 1.14 bar

p₃,₁·V₃ = m₉·R·T → m₉ = p₃,₁·V₃ / R·T = (1,140.984·3·10⁻³) / 166.28 - 773.15 = 3.422·10⁻³ kg

R = 166.28 J/kg·K

R = (50·10³ kJ/mol·K) / (8.314 J/mol·K)

m₉ = 3.422 g/s