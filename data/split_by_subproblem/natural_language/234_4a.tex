Absolute value of Q one two equals fifteen thousand joules  

Energy balance of the entire system (ice water):  
Delta E equals m e w multiplied by (U two minus U one) equals Q two  

U one equals U liquid (zero degrees Celsius) plus x ice multiplied by (U frozen (zero degrees Celsius) minus U liquid (zero degrees Celsius))  

equals minus zero point zero four five kilojoules per kilogram minus three hundred thirty-three point four five eight kilojoules per kilogram  

equals minus two hundred point zero nine two eight  

U two equals U final (ten degrees Celsius) plus x ice one two multiplied by (U frozen minus U final liquid)  

Therefore, Q one two divided by m e w plus U one equals U final plus x two multiplied by (U final minus U final liquid)  

Q one two divided by m e w plus U one minus U final divided by U final minus U final liquid equals x two equals zero point five five five

A graph is drawn with pressure (p) in millibar on the vertical axis and temperature (T) in degrees Celsius on the horizontal axis. The graph shows a curve labeled 'fest' at 0.5 millibar and another curve labeled 'gas'. There are three points labeled 1, 2, and 3, with point 1 above point 2, and point 3 below point 2. The point labeled 'Tripelpunkt' is on the curve. The label 'flüssig' is written near point 1.