Pg1 , mg

Perfect gas: p1 V1 = mg R T1

Force equilibrium: ΔPamb + mg g + mF g = P2 A incompressible

Pamb π os cm squared + 32 g = PFis π sum squared = Pg1 π sum squared

=> Pg1 = Pamb + (32 . 9.81 + 0.1 . 9.81) / π (5 . 10 squared)

= 1 . 10 to the power of 5 + (32 . 9.81 + 0.1 . 9.81) / π (5 . 10 squared)

= 4.0146 bar = 1.40094 bar

=> mg = p1 V1 / R T1 cos R = 8.314 / 50 . 10 to the power of -3

= 1.400 . 10 to the power of 5 / 3.14 . 10 to the power of -3

= 8.314 / 50 . 10 to the power of -3 . 773.15

= 3.421 g

Absolute value of Q one comma two equals absolute value of m g one c v delta T. This is equal to three point six times ten to the power of negative three times zero point six three three times open parenthesis five hundred degrees Celsius minus zero degrees Celsius close parenthesis. This equals one point one three nine kilojoules.