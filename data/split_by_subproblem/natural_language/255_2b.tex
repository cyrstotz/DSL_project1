Another graph is drawn below the first, with axes labeled "T" and "S" with units "kJ/kgK". This graph includes a series of points numbered from 0 to 6, connected by lines. The points are marked as follows: 0, 1, 2, 3, 4, 5, 6. There is a zigzag line between points 5 and 6. The graph also contains labels "Ps = Pu" and "Po" near the top right.

Five to six isentropic nu equals seven a.  
T sub b equals T sub five times P sub b over P sub five to the power of seven over seven minus one over seven equals three hundred twenty-eight point zero two nine seven K.

Seven HS Durch.  
Zero equals m times h sub five minus h sub six plus w sub five squared minus w sub six squared over two plus ZQ, zero is crossed out.  
Two times h sub six minus h sub five equals w sub six squared minus w sub five squared.  
Therefore, w sub six squared equals w sub five squared minus two times h sub six minus h sub five, therefore w sub six equals two hundred crossed out m squared over s squared.  
h sub six minus h sub five equals c sub p times T sub b minus T sub five equals minus one hundred zero point four five kJ over kg.