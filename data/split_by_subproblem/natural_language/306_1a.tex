a) First Law, stationary: Q out equals m dot times (h out minus h in) minus Q R

h out (200 degrees Celsius, x equals 0) equals 919.09 kilojoules per kilogram from A-2

h in (70 degrees Celsius, x equals 0) equals 292.98 kilojoules per kilogram from A-2

Q out equals 0.3 kilograms per second times (919.09 minus 292.98) equals 62.182 kilowatts

b) T R F equals integral from s a to s e of T ds over p equals constant equals integral from s a to s e of dH equals c p times (T a us W F minus T e in e r) over c p times ln (T a us W F over T e in e r)

equals 298.15 minus 288.15 over ln (298.15 over 288.15) equals 293.12 Kelvin

c) Second Law: S e r z equals m dot times (s a minus s e) minus Q out over T reactor minus Q out over T K F on the wall!

implies S e r z equals 1 over second times 65 kilowatts over 293.15 Kelvin plus 65 kilowatts over 293.12 Kelvin equals 0.0476 kilowatts per Kelvin equals 99.56 watts per Kelvin

d) First Law: half-open system

E2 minus E1 equals m dot times h a2 plus Q out

h a2 (200 degrees Celsius, x equals 0) equals 83.96 kilojoules per kilogram from A-2