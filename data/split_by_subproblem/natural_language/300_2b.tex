P5 times V5 equals R times T5.
R equals R bar divided by M equals 0.2867.
V5 equals R times T5 divided by P5.
V5 equals 2.48 cubic meters per kilogram.

A formula is written:
O equals m dot times (h2 minus h1 plus w2 squared minus w1 squared divided by 2).

The following equation is shown:
(P6 divided by P5) to the power of k minus 1 divided by k equals (V5 divided by V6) to the power of k minus 1 divided by k equals T6 divided by T5.

V5 divided by V6 equals the square root of T6 divided by T5.

Tb equals T5 times (P6 divided by P5) equals 328.07 K.

V6 equals 4.8377 cubic meters per kilogram.