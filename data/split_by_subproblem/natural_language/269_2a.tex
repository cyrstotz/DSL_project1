The page contains a graph with the label 'T (K)' on the vertical axis and 'S [kJ/kg]' on the horizontal axis. The graph depicts a cycle with three points labeled '2', '4', and '6'. The point '2' is marked with 'p2 = p6 = p5', point '4' is marked with 'p4', and point '6' is marked with 'p6'. There are arrows showing the direction of the cycle, and the curve is drawn connecting these points.

Below the graph, labeled section 'b' contains several equations:
- O equals m dot times (h5 minus h6) times t minus omega dot.
- T6 equals Ts times (p6 divided by p5) to the power of (n minus 1 divided by n), which equals 328.071 Kelvin.
- Omega dot e rev equals m dot times (h5 minus h6) equals m dot times cp times (Ts minus T6) equals 104.45 kilowatts.
- Omega dot e rev equals integral from 5 to 6 v dp plus (omega 6 squared divided by 2 plus omega 5 squared divided by 2), which implies an arrow.
- Integral from 5 to 6 v dp equals minus n times integral from 5 to 6 v du equals minus n divided by n minus 1 times R times (T6 minus Ts) equals minus 104.45 kilowatts.
- n equals cp divided by cv equals 1.4.
- cv equals 0.7186 kilojoules per kilogram Kelvin.
- R equals cp minus cv equals 0.287 kilojoules per kilogram Kelvin.