0 → 1 adiabatic R less than 1  
1 → 2 isentropic  
2 → 3 isobaric  
3 → 4 adiabatic R less than 1  
4 isochoric  
4 → 5 isentropic  
5 → 6  

Kernstrom

[Diagram of a T-s graph with labeled points and processes]
- y-axis labeled as T [K]
- x-axis labeled as s [J/kgK]
- Points labeled 0, 1, 2, 3, 4, 5, 6
- Processes labeled as isobaric, isochoric, isentropic, adiabatic
- P1 = P5 = 0.5 bar
- P0 = 0.15 bar

b) ω6 = ?  T6 = ?

t5 = 431.5 K, P3 = 0.5 bar, ω5 = 220 meters per second, P6 = P0 = 0.15 bar

n = κ = 1.4

(T2/T0) = (P2/P0) raised to the power of (n-1)/n  
(T6/T5) = (P6/P5) raised to the power of (n-1)/n  

T6 = T5 (P6/P5) raised to the power of (n-1)/n = 328.07 K

0 = m [h2 - h6 + ω2 squared over 2 - ω6 squared over 2]  
0 = h5 - h6 + ω5 squared over 2 - ω6 squared over 2  

h6 - h5 - ω5 squared over 2 = -ω6 squared over 2  
ω6 = square root of [ω5 squared + 2 (h5 - h6)]

A-21 (431.5 K - 450 K) (444.66 - 431.43) (kJ/kg) + 431.43 (kJ/kg) = 433.36 (kJ/kg)

h5 = (440 K - 450 K)  
h6 = A-21 (328.07 - 325 K) (330.34 - 325) (kJ/kg) + 325.34 (kJ/kg) = 328.41 (kJ/kg)

ω6 =