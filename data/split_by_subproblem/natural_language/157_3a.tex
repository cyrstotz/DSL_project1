D equals ten centimeters equals 0.1 meters  
R equals 0.05 meters  
A equals 0.007854 square meters equals R squared times pi  

Given:  

p gas 1 equals p ambient over A plus m K times g plus m E N times g  

p gas 1 equals p ambient plus A times g times (m K plus m E N)  

p gas 1 equals 100 kilopascals plus 0.007854 square meters times g times (32 kilograms plus 0.1 kilograms) equals  

p gas 1 times A equals p ambient times A plus g times (m K plus m E N)  

p gas 1 equals p ambient plus g over A times (m K plus m E N)  

p gas 1 equals 1.4 bar  

Therefore, p gas 1 times V gas 1 equals m gas times R over M g times T gas 1  

with V gas 1 equals 0.0314 cubic meters  
T gas 1 equals 773.15 Kelvin  

m gas equals p gas 1 times V gas 1 times M g over R times T gas 1 equals 3.422 grams