Diagram 1: A graph with pressure (p) in bar on the vertical axis and temperature (T) in Kelvin on the horizontal axis. The graph has a peak labeled "kritischer Punkt" (critical point) and a descending curve labeled "Nassdampf" (wet steam).

Diagram 2: A graph with pressure (p) in bar on the vertical axis and temperature (T) in Kelvin on the horizontal axis. The graph shows a line labeled "Gas" and a curve labeled "Flüssig" (liquid). There is a point labeled "Tripel" (triple) and a line labeled "I" with "Fes" (solid) next to it. The temperature is marked as 40°C.

Epsilon equals (Q dot abl divided by Phi dot zul) equals (Phi dot ic divided by V dot f) equals (Phi dot ic divided by (Q dot abl plus Phi dot zul)) equals (Phi dot ic divided by (Phi dot ic plus Q dot abl)).  

Q dot ic equals m dot times (h2 minus h3).  
Q dot abl equals m dot times (h4 minus h3).