m dot equals 0.3 kilograms per second, T in equals 70 degrees Celsius  
x p equals 0.005  

a) 1 HS O equals m dot h in minus h aus plus 100 kilowatts minus Q aus minus W  
Q aus equals m dot h in minus h aus plus Q R  

TAB A2  
h in equals h f (70 degrees Celsius) equals 292.88 kilojoules per kilogram  
h aus equals h f (100 degrees Celsius) equals 419.64 kilojoules per kilogram  
m in equals 0.3 kilograms  
m aus equals 0.3 kilograms minus m in  

therefore Q aus equals m dot (h in minus h aus) plus 100 kilowatts  
Q aus equals 62.182 kilowatts  

b) Exergy balance for cooling liquid  
O equals E dot x sr plus E dot x j minus W minus E dot x wi  

c) T KF equals 285 Kelvin  
2 HS O equals m dot (s e x sr minus s e x a) plus Q aus over T KF minus s e x z  

s e x z equals plus Q aus over T KF equals 62.182 kilowatts over 285 Kelvin equals 0.2107 kilowatts per Kelvin minus s e x z

Halboffenes System 1/15

M sub 2 U sub 2 minus M sub 1 U sub 1 equals delta M h ein plus Q minus CW

M sub 1 equals M ges

Q equals Q aus equals 35 megajoules

M sub 2 equals M ges plus M sub B

M sub B plus M sub 1 U sub 2 minus M sub B U sub 1 equals delta M h ein plus negative Q aus

M sub B U sub 2 plus M sub 2 U sub 2 minus M sub B U sub 1 equals delta M h plus gamma

delta M times (U sub 2 minus h ein) equals M sub B U sub 1 plus gamma

delta M equals M ges plus Q over U sub 2 minus h ein

M ges equals 5755 kilograms

h ein equals h (t ein) minus h e (20 degrees) equals h ein equals h f (20 degrees) equals 83.36 kilojoules per kilogram