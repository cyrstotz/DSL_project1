p1 equals 1.9 bar

m1 equals m2 equals 0.1 kilograms

Un equals U at 0 degrees Celsius, 1.9 bar equals Uf2 at 0 degrees Celsius, 1.16 bar plus x times (Ug at 0 degrees Celsius, 1.96 bar minus Uf at 0 degrees Celsius, 1.96 bar)

equals minus 133.4 kilojoules per kilogram

x2 equals U2 minus Ufest divided by Udüssig minus Ufest

U2g equals m times R times T2 divided by p equals 1.11 times 10 to the power of minus 3 cubic meters

W equals p1 times (V2 minus V1) equals 28.27 joules

m1 equals m2 equals mw

m2 times U2 minus mw times Un equals Q12 minus W12

U2 equals Q12 minus W12 divided by mw plus Un

equals minus 122.59 kilojoules per kilogram

x2 equals U2 minus Ufest divided by Udüssig minus Ufest equals 63.28 percent

F equals rho times A

Diagram 1:
A rectangle divided into three sections labeled E, w, P1, and P2. The top section contains a rectangle labeled m.

Diagram 2:
A rectangle labeled m with arrows pointing down labeled mg and arrows pointing up labeled P1. Above the rectangle is a label pamb.

Sum of forces equals rho ambient times pi times d squared divided by two plus mg equals P1 times pi times d squared divided by two.

P2 equals rho ambient plus four times mg divided by pi times d squared

P2 equals 1.20 bar

Diagram 3:
A rectangle divided into two sections labeled P1 and P2. Above the rectangle is a label m E w g with arrows pointing up and one arrow pointing down.

Sum of forces equals P1 times pi times d squared divided by two plus m E w g equals P2 times pi times d1 squared divided by two

P2 equals P1 plus four times m E w g divided by pi times d squared

P2 equals 1.40 bar

m g equals P2 times V divided by R times T specific

R equals R divided by M equals 166.28 Joules per kilogram Kelvin

m g equals 1.40 bar times 50 times 10 to the power of minus 3 cubic meters divided by 166.28 Joules per kilogram Kelvin times 773.15 Kelvin

equals 3.42 times 10 to the power of minus 3 kilograms