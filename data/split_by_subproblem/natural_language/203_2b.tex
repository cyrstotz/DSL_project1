p sub s equals 0.5 bar, p sub 6 equals 0.197 bar  
T sub s equals 437.9 K, w sub s equals 220 meters per second  

h sub 6 minus h sub 5 equals c sub p times (T sub 6 minus T sub 5)  

T sub 6 over T sub 5 equals (p sub 6 over p sub s) to the power of n minus 1 over n  

implies T sub 6 equals T sub s times (p sub 6 over p sub s) to the power of n minus 1 over n equals 437.9 K times (0.197 bar over 0.5 bar) to the power of 0.4 over 1.4 equals 328.075 K  

1. Hs über Schubdüse  

Q dot equals m dot times [h sub c minus h sub a plus (w sub c squared minus w sub a squared over 2)]  

O dot equals m dot times [h sub 5 minus h sub 6 plus (w sub 5 squared minus w sub 6 squared over 2)]  

w sub e squared equals 2 times (h sub s minus h sub c) plus w sub s squared  

w sub 6 equals square root of 2 times c sub p times (T sub s minus T sub 6) plus w sub s squared equals square root of 2 times 1000 joules per kilogram Kelvin times (437.9 K minus 328.075 K) plus (220 meters per second) squared equals 507.29 meters per second