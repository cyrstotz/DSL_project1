P g1 equals P amb plus P piston  
P piston equals thirty-two kilograms times eight point four meters squared divided by open parenthesis zero point zero zero five meters close parenthesis squared times pi equals thirty-five point nine six five kilopascals  
P g1 equals one hundred kilopascals plus thirty-five point nine six five kilopascals equals one hundred thirty-five point nine six five kilopascals  
m g1 equals V g1 times P g1 divided by R times T g1 equals zero point zero zero three one four cubic meters times one hundred thirty-five point nine seven kilopascals divided by eight point three one four kilojoules per kilomole Kelvin times three hundred seventy-three point one five Kelvin equals three point four one five grams