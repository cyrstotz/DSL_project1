There are two diagrams drawn. The first diagram has axes labeled as 'p' and 'T', with points marked as 1, 2, and 4. A line connects point 4 to point 1, labeled as 'sotrop'. The second diagram has axes labeled as 'p (max)' and 'T (eq)', with points marked as 1, 2, 3, and 4. There is a curved line connecting these points.

Below the diagrams, there is a table with the following columns: 'ze', 'P(low)', 'T', 'h', 'Q dot / w', 'x', and 'Notes'. The rows are filled as follows:

Row 1:
- P(low): 1
- T: (OC)
- h: h(c) [2]
- Q dot / w: Qk
- x: (empty)
- Notes: (empty)

Row 2:
- P(low): 2
- T: (OC)
- h: h(c) [2]
- Q dot / w: Wk = 28 W
- x: 1
- Notes: adiabatic reversible

Row 3:
- P(low): 8
- T: (OC)
- h: h(c) [2]
- Q dot / w: Qabs
- x: (empty)
- Notes: (empty)

Row 4:
- P(low): 8
- T: (OC)
- h: h(c) [2]
- Q dot / w: (empty)
- x: 0
- Notes: (empty)

The diagram is a graph with pressure labeled as 'P [wbar]' on the vertical axis and temperature labeled as 'T [k]' on the horizontal axis. The graph shows a curve with four points labeled 1, 2, 3, and 4. The section between points 1 and 2 is labeled 'isotrop', and the section between points 3 and 4 is labeled 'isobar'. There is a curve connecting these points, and the horizontal axis has an additional label 'ND'.