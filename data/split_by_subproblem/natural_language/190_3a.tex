Given: P_g1, m_g

R equals R over M equals eight point three one four kilojoules per kilomole Kelvin over fifty equals zero point one six six three kilojoules per kilogram Kelvin.

C_v equals zero point six three three kilojoules per kilogram Kelvin.

C_p equals R plus C_v equals zero point seven nine nine three kilojoules per kilogram Kelvin.

P_g1 equals pressure from above.

Therefore, P_g1 equals one bar plus g times (m_k plus m_EW) over A (z equals one) where A equals ten centimeters squared times pi.

Thus, P_g1 equals one bar plus nine point eight one meters per second squared times (thirty-two kilograms plus zero point one kilograms) over pi over one hundred meters squared equals one bar plus zero point one bar equals one point one bar.

Since the gas is perfect, it holds that:

pV equals mRT

Therefore:

m_g equals P_g1 times V_g1 over R times T_g1 equals one point one bar times three point one four liters over zero point one six six three kilojoules per kilogram Kelvin times five hundred degrees Celsius equals two point six eight six grams.