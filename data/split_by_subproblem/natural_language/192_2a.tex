h2 equals 237.74  
h3 equals 271.3  

2 to 3 1. HS  

0 equals m dot (h2 minus h3) plus Q dot (adiabat) minus W dot  

W dot over h2 minus h3 equals m dot  

m dot equals 0.834 grams per second

A graph is depicted with the y-axis labeled as T in Kelvin and the x-axis labeled as S in kilojoules per kilogram Kelvin. The graph shows a series of points and lines, labeled as follows:

- Point 0 at the bottom left.
- Point 1 above point 0, with a vertical line labeled "isentrope."
- Point 2 above point 1, with a line labeled "isentrope."
- Point 3 to the right of point 2, with a line labeled "n greater than 1."
- Point 4 below point 3, with a line labeled "n equals 1."
- Point 5 to the left of point 4, with a line labeled "isentrope."
- Point 6 below point 5.

The lines connecting these points form a cycle, with arrows indicating the direction of the process.

To the right of the graph, there are labels:
- "p0" at the bottom.
- "p1" above "p0."
- "pc" at the top, with a note "(hochdruckseite n much greater than p0)."

Below the graph, the following information is listed:

z0: 0.15 bar, -30 degrees Celsius

z1: p1 (as calculated, n less than 1)

z2: isentrop, p2

z3: isobar

p4 equals p1 equals p5