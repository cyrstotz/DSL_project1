a) Energy balance in reactor:

O equals m dot times (h in minus h out) plus Q R plus Q loss (Q loss is less than zero, because heat is removed).

h in (70 degrees Celsius) equals h p (70 degrees Celsius) equals 282.89 kilojoules per kilogram.

h out (100 degrees Celsius) equals h p (100 degrees Celsius) equals 419.04 kilojoules per kilogram.

Q loss equals minus m dot times (h in minus h out) plus Q e.

equals minus 62.18 kilowatts.

b) T KF equals integral from s a to s e of T d s divided by s a minus s e.

equals q rev divided by s a minus s e.

c) Entropy balance:

O equals m dot times (s e i minus s e out) plus Q loss divided by T KF plus S e y.

The energy balance in the reactor is:

m2 times u2 minus m1 times u1 equals Am12 times h dot ein,12 plus QR,12

with h ein,12 at 20 degrees Celsius plus hp at 20 degrees Celsius equals 83.86 kilojoules per kilogram

m1 equals 57.55 kilograms

m2 equals m1 plus Am12

u1 at 20 degrees Celsius equals up at 20 degrees Celsius equals 83.95 kilojoules per kilogram

u2 at 70 degrees Celsius equals up at 70 degrees Celsius equals 282.95 kilojoules per kilogram

Substitute in:

m1 times u2 plus Am12 times u2 minus m1 times u1 equals Am12 times h ein,12 plus QR,12

equals greater than Am12 equals (m1 times u1 minus m1 times u2 plus QR,12) times 1 over (u2 minus h ein,12)

Integral of W dot over m dot equals negative integral from 5 to 6 of v dp plus w6 squared minus w5 squared over 2.

Equals negative n times R times (T6 minus T5) over 1 minus n plus ln (w6 squared minus w5 squared over 2).

Equals n R times (T5 minus T5 over 1 minus n) plus (w5 squared minus w6 squared over 2) ln.

O equals h5 minus h6 plus w5 squared minus w6 squared over 2 minus n R (T6 minus T5) over 1 minus n.

So, I calculate 

W subscript K divided by h subscript 2 minus h subscript 3.