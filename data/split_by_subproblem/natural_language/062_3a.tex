R equals eight point three one four kilojoules per kilomole per Kelvin divided by fifty kilojoules per kilomole equals zero point one six six kilojoules per Kelvin.

a) p g 1, m g

A equals Pi times (D divided by 2) squared

p g 1

p ENV 1, A

p g 1,1

p g 1,1 equals p ENV 1 plus m g divided by A plus p ENV 1

thirty-two kilograms times nine point eight one meters per second squared divided by Pi times (D divided by 2) squared plus one point four times ten to the power of five Newtons per meter squared equals one point four zero bar

p V equals m R T

T g 1 equals seven hundred seventy-three point one five Kelvin

m g equals p g 1 divided by K g 1

one point four times ten to the power of five Newtons per meter squared divided by zero point one six six times ten to the power of three Joules per kilogram per Kelvin times four hundred seventy-three point one five Kelvin equals three point four three grams