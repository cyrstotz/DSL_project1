dS over dt equals sum of m dot i s i plus sum of Q dot over T plus S dot ex

One G, c sub p equals one point zero zero six joules per kilogram Kelvin, n greater than K equals one point four.

Given: w sub c and T sub s, p sub e equals p sub o equals zero point one nine one bar equals one hundred fifty-four kilopascals.

w sub s equals two hundred twenty meters per second, p sub s equals zero point five bar, T sub s equals four hundred thirty-one point eight Kelvin.

Therefore, T sub c via adiabatic coefficient:

T sub c over T sub s equals (p sub c over p sub s) to the power of n minus one over n, therefore T sub c equals T sub s times (p sub c over p sub s) to the power of zero point four over one point four.

T sub c equals four hundred thirty-one point eight Kelvin times (zero point one nine one bar over zero point five bar) to the power of zero point four over one point four equals three hundred twenty-eight point zero seven Kelvin.

w sub c: steady flow process an adiabatic reversible disc.

O equals m dot times [(h sub s minus h sub c) plus (w sub s squared minus (w sub c) squared) over two plus p sub e times O plus O squared minus O times O].

O equals h sub s minus h sub c plus w sub s squared minus w sub c squared over two.

One G: h sub s minus h sub c equals c sub p times delta T equals c sub p times (T sub s minus T sub c).

c sub p times (T sub s minus T sub c) equals (w sub s squared over two minus w sub c squared over two).

c sub p times (T sub s minus T sub c) plus (w sub c squared over two) equals w sub s squared over two.

w sub c squared equals two c sub p times (T sub s minus T sub c) plus (w sub s squared).

w sub c equals the square root of two times one point zero zero six times (four hundred thirty-one point eight Kelvin minus three hundred twenty-eight point one Kelvin) plus two hundred twenty squared meters per second.

w sub c equals approximately five hundred seven point two meters per second.

Q equals m times c sub v times delta T for 1 HS over on gas side

Q sub 12 equals m times g minus c sub v times (T)

p sub 1 equals p sub 2

1 HS at the piston

c sub p equals R plus c sub v

m dot times (U sub 2 minus U sub 1) equals Q tilde sub 12 minus W sub 12

W sub 12: since isochoric, h equals zero plus perfect gas

R times (T sub 2 minus T sub 1) equals 8.314 kilojoules divided by 50 Kelvin times (0.0028 minus 500 degrees Celsius)

equals 28 minus 83.14 kilojoules

implies Q sub 12 equals m times g times (c sub p minus 1)

Q sub 12 equals m times g times (c sub v times (T sub 2 minus T sub 1)) plus U sub 12

equals 0.0038 kilograms times 50

equals 0.0026 kilograms divided by 0.633 kilojoules per kilogram Kelvin times (449.987 Kelvin)

Q sub 12 equals m times g times (c sub v times (T sub 2 minus T sub 1)) plus W sub 12

equals 0.0026 kilograms times (0.633 times 449.997 kilojoules per kilogram) minus 83.14 kilojoules

Q sub 12 equals