Ideal gas law: pV = mRT  
Force equilibrium state 1: 

[Diagram showing a piston with arrows indicating forces: F_ew, F_k, F_g, and F_am]

F_g = F_ew + F_am + F_k  

A_z = cross-sectional area of cylinder = pi r squared  
r = 5 cm = 0.05 m  
Therefore, A_z = pi times 0.05 m squared = 7.85 times 10 to the power of -3 m squared  

F_g = p_g1 times A_z  

F_ew = m_ew times g = 0.1 kg times 9.81 m/s squared = 0.981 N  

F_k = m_k times g = 32 kg times 9.81 m/s squared = 313.92 N  

F_am = p_am times A_z = 100 kPa times A_z = 100,000 N/m squared times 7.85 times 10 to the power of -3 m squared = 785 N  

F_g = F_ew + F_k + F_am = 1099.901 N  

F_g = p_g1 times A_z  
Therefore, p_g1 = F_g / A_z = 1099.901 N / 7.85 times 10 to the power of -3 m squared = 140144.78 N/m squared  

p_g1 = 140144.78 Pa = 140.14 kPa = 1.4 bar  

Ideal gas law: pV = mRT  
Therefore, m = pV / RT  
Therefore, m_g = p_g1 V_g1 / R_g T_g1  

R_g = R / M = 8.314 kJ/kmol K / 50 kg/kmol = 0.166 kJ/kg K  

V_g1 = 3.14 l = 3.14 times 10 to the power of -3 m cubed  

T_g1 = 500 degrees Celsius = 773.15 K

One hundred forty point fourteen kilopascals times three point fourteen times ten to the power of negative three cubic meters.  
Mg equals zero point one six six kilojoules per kilogram times seven hundred seventy-three point fifteen kelvin equals three point four three times ten to the power of negative three kilograms equals three point four three grams.