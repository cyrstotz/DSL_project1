dE over dt equals the sum of m sub i times h sub i plus c squared times k sub i squared minus Q dot minus W dot.

Delta U equals delta m sub 12 times h sub ein plus Q dot.

m sub 2 times U sub 2 minus m sub 1 times U sub 1 equals delta m sub 12 times h sub ein plus Q dot.

m sub 2 equals m sub 1 plus delta m sub 12.

(m sub 1 plus delta m sub 12) times U sub 2 minus m sub 1 times U sub 1 equals delta m sub 12 times h sub ein plus Q dot.

minus Q dot plus m sub 1 times (U sub 2 minus U sub 1) equals delta m sub 12 times h sub ein minus delta m sub 12 times c sub 12.

minus Q dot plus m sub 1 times (U sub 2 minus U sub 1) equals delta m sub 12 times (h sub ein minus U sub 2).

Therefore, delta m sub 12 equals the fraction (m sub 1 times (U sub 2 minus U sub 1) minus Q dot) over (h sub ein minus U sub 2).

Boiling liquid implies x equals 0.5.

h sub ein equals 83.96 plus 0.5 times (2538.1 minus 83.96) equals 1311.03 kilojoules per kilogram.

TAB-A-2

U sub 1 equals 418.94 plus 0.005 times (2506.5 minus 418.94) equals 429.3778 kilojoules per kilogram.

TAB-A-2

U sub 2 equals 292.35 plus 0.5 times (2463.6 minus 292.35) equals 1381.275 kilojoules per kilogram.

TAB-A-2

Delta m sub 12 equals the fraction (5755 kilograms times (1381.275 kilojoules per kilogram minus 429.3778 kilojoules per kilogram) plus 35 times Q dot) over (1311.03 kilojoules per kilogram minus 1381.275 kilojoules per kilogram).