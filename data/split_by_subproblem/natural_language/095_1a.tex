a) Given: Heat over the surface at a liquid  

[Diagram of two vertical rectangles labeled: T_Focus and T_env with arrows pointing to Q_out and Q_in]  

Stationary FP with 1 mass flow  

G_eq equals m_dot_in times (h_in minus h_out) plus Q_out plus Q_R  

Q_out equals m_dot_in times (h_out minus h_in)  

Q_out equals m_dot_in times (h_in minus h_out) plus G_eq  

h_ein over h_ein equals h (70 degrees Celsius) over h_ein times h (100 degrees Celsius, liquid) equals 2.077 kilojoules per kilogram  

[Crossed out section]  

h_ein equals h (70 degrees Celsius) equals 2.092 kilojoules per kilogram  

h_auss minus h (100 degrees Celsius) equals 4.9104 kilojoules per kilogram  

Q_out equals 0.83 times (h_ein minus h_auss) minus 4.9104 kilojoules per kilogram plus 100 kilowatt  

Q_out equals 62.182 kilowatt  

b) Given: I_KF  

S1_FP: Q equals m_dot times (s_2 minus s_a) equals sigma Q over s_2 minus s_1 equals h_2 minus h_1 over s_2 minus s_1  

I_F equals T_2 minus T_1 over ln (T_2 over T_1) equals 298.15 Kelvin minus 288.15 Kelvin over ln (298.15 Kelvin over 288.15 Kelvin) equals 203.12 Kelvin

d d t equals m dot ein times bracket h ein minus h aus bracket plus delta m z times h inerte plus q dot

e) delta s a z equals m dot ein times bracket s z minus s 1 bracket