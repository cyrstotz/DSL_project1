a) Q̇aus ges.

Stat. EP

0 = ṁ (he - ha) + Q̇R + Q̇aus

Q̇aus = ṁ (ha - he) + Q̇R

(gesicherte Umgebung)

he = hF @ 100°C = 419.04 kJ/kg

ha = hF @ 70°C = 292.98 kJ/kg

= 0.5 kg/s (419.04 - 292.98) + 1000 kW

= 137.818 kW

(TAB +2)

b) T̅re

T̅ = ∫Tds

Sa - Se

Tds = x dx

Cp = konst.

p = konst.

Sa - Se

= Cp ln (T2/T1) - R ln (p2/p1)

(weil Cp = konst.)

F = Cp (T2 - T1)

Cp ln (T2/T1) = 10 K

= ln (293.15/283.15) = 243.12 K

= 19.97°C

c) Ṡerz

0 = ṁ (Sc - Sa) + Q̇R

Ṡerz = ṁ (sa - se) - Q̇aus/Tw

Ṡerz = (Q̇aus/Tw) * Serz

Tw = ~ 850°C = 3455.15 K

Su = 0.504 kJ/kg K

Se = 0.0544 kJ/kg K

Ṡerz = (0.278) ?

7. (h.w.)

e) Delta S twelve

Delta S equals M two S two minus M one S one

M two equals Delta M twelve plus M one

equals three thousand six hundred kilograms plus fifty-seven point five five kilograms

from sheet equals four thousand three hundred fifty-five kilograms

S two equals zero point nine four five four kilojoules per kilogram Kelvin

at one hundred degrees Celsius S f equals S f one plus x (S g minus S f)

equals one point eight five zero nine plus zero point zero zero five (seven point three five four nine minus one point three zero six nine)

equals one point three three seven one four kilojoules per kilogram Kelvin

Delta S equals four thousand three hundred fifty-five kilograms (zero point nine four five four kilojoules per kilogram Kelvin) minus

fifty-seven point five five kilograms (one point three three seven one four kilojoules per kilogram Kelvin)

equals

equals one thousand two hundred thirty-seven point eight four nine kilojoules per Kelvin