a) P3,1

Through jju P3,1 = Pgaussian

Pmaximum = Pmedium times (u plus MEW) times A divided by g

A = pi times r squared = pi times (5 cm) squared = pi times 0.05 squared

Therefore, Pmedium = 10 to the power of 5 pascal plus 32 kg times 0.16 g divided by pi times 5 squared cm squared times 381 m/s squared equals 1.046 ws

Mg = P times V divided by R times T1 = 1.04 times 10 to the power of 5 pascal times 3.14 times 10 to the power of negative 3 m squared equals 0.00254 kg

8.314 J divided by mol K divided by 50 K divided by 10 to the power of negative 3 equals 2.54 J equals 1 Mg

b) P3,2 = Pgaussian = 1.048 ws, because the mass or the usage has not changed.