The page contains three graphs, each with a pressure versus temperature diagram.

1. The first graph is labeled "R 134 a" at the top. It has axes labeled with pressure (P) in bar on the vertical axis and temperature (T) in Kelvin on the horizontal axis. The graph shows three regions labeled "fest" (solid), "flüssig" (liquid), and "gas" (gas). The lines separate these regions, meeting at a point labeled "Tripel" (triple point).

2. The second graph also has axes labeled with pressure (P) in bar and temperature (T) in Kelvin. It shows regions labeled "fest" (solid), "flüssig" (liquid), and "gasförmig" (gaseous). There are markings labeled "i" and "ii" with arrows pointing to a point labeled "Tripel" (triple point), and another point labeled "x" with an "o" next to it.

3. The third graph has axes labeled with pressure (P) in bar and temperature (T) in Kelvin. The graph shows intersecting lines, with no specific labels or annotations visible.