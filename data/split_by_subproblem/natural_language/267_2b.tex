P subscript y equals 2 times T subscript 5 implies y subscript 5 equals 2 times T subscript 5 divided by P subscript 5 equals 3.476 kilojoules per kilogram Kelvin.
mil P equals n times c subscript v minus c subscript v equals c subscript v times (n minus 1) approximately equals 40.2 kilojoules per kilogram.
T subscript 14 divided by T subscript 5 equals (P subscript 6 divided by P subscript 5) raised to the power of n minus 1 divided by n implies T subscript 1 equals T subscript 5 times (P subscript 6 divided by P subscript 5) raised to the power of n minus 1 divided by n approximately equals 328.07 Kelvin.
mil P subscript 6 equals P subscript 0.
1. HS: minus U equals c subscript p times T subscript 0 minus T equals v subscript 5 squared plus k subscript e subscript 6 minus k subscript e subscript 5 equals 0.
c subscript v times (T subscript 1 minus T subscript 5) equals 1/2 times (w subscript 2 squared minus w subscript 5 squared) equals 0.
w subscript 0 squared minus w subscript 5 squared minus 2 times c subscript v times (T subscript 1 minus T subscript 5) equals 50.7.244 meters per second.