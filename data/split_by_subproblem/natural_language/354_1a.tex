a) Q dot equals m dot times (h a minus h e)

(h a minus h e) equals C p times (T exhaust minus T e, ein)

From Table A12, one can now determine the enthalpy using the entropy. It is known from A10 that s2 equals s3 equals 6.5285 kilojoules per kilogram per Kelvin.

h3 equals open parenthesis h at 8 bar, 90 degrees Celsius minus h at 8 bar, T saturated close parenthesis over open parenthesis s3 minus s at 8 bar, T saturated close parenthesis plus h at 8 bar, T saturated.

Equals 271.37 kilojoules per kilogram per Kelvin.

Therefore, for the balance equation and the compressor, we assume adiabatic conditions.

0 equals m dot open parenthesis h2 minus h3 close parenthesis plus Q dot minus W dot.

m dot equals W dot over open parenthesis h2 minus h3 close parenthesis equals 8.34 times 10 to the power of negative 4 kilograms per second.

c) Adiabatic throttle - isenthalpic Table A11

h4 equals h3

h3 equals hf at 8.0 bar equals 93.92 kilojoules per kilogram

The evaporator operates in water vapor state and at minus 16 degrees Celsius, one can calculate the vapor content Table A10.

Xn equals open parenthesis h1 minus hf at negative 16 degrees Celsius close parenthesis over open parenthesis hg at negative 16 degrees Celsius minus hf at negative 16 degrees Celsius close parenthesis equals 30.76 percent.