a) A graph is drawn with axes labeled P (pressure) and T (temperature). Points are marked on the graph with labels such as "fest", "flüssig", and "gasförmig". The graph includes a phase boundary line and arrows indicating transitions between phases. 

- Phasengrenze
  1. im Zustandspunkt
  2. Zustandsgrößen bekannt
  3. Änderung auf überbar untersuchen
  4. Bestimmung der Wärme in den Kältemittel

b) dE/dt = Σ mi (hi + ke_i + pe_i) + Σ aj - Σ Wn - Σ Qj der inneren

dE/dt = mi (he - ha) + Qk - Wk
ΔE = oU + oKe + oPe

Wk - Qk = m R 134 a (he - ha)
Wk = 26 W (aus Aufgabe)

c) x = φe - φf / φg - φf

d) "ik = [Q̇zu] / [Ẇt] = [Q̇abl] - [Q̇zu] * ε = [Q̇K / ẆK]

Q̇k = "Energiebilanz"
Qk = Qzu (he - ha + we squared - wa squared / 2) eg (ze - za) / 2 Q̇k - Qzu

dE/dt = Σ Qi - Σ Wn
E = U - KE - PE → ΔE = U
ΔV = Q̇K =

e) Die Temperatur würde weiter fallen, bis der Nullpunkt erreicht ist, da immer Energie entzogen wird

→ intro = m = o, T2 / T1 (p0 / p1) squared = 1 → T1, 1 / p00 = T2