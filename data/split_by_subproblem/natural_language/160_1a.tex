Stat. Fließprozess

Zero equals m dot times (se minus sa) plus Q dot divided by T plus Sexe.  
Zero equals m dot times (he minus ha) plus Q dot aus minus p2 minus p1, da isobar.  
Q dot aus equals m dot times (ha minus he).  
ha minus he equals ci times (Ta minus Te) plus u squared divided by two times (p2 minus p1).  
p2 equals pa.

Diagram: A horizontal rectangle labeled Tke,e on the left and Tke,a on the right with an arrow labeled Q dot aus pointing right through the rectangle.

Q dot aus equals m dot times ci times (Tke,a minus Tke,e).

b) w sub s equals two hundred twenty meters per second, p sub s equals zero point five bar, T sub s equals four hundred thirty-one point zero Kelvin

Stat. Filippov

zero equals m dot (h sub s minus h sub t plus w sub s squared minus w sub t squared divided by two) plus q dot, a dieheat minus W dot sub t

h sub s minus h sub t equals C sub p (T sub s minus T sub t)

W dot rev equals minus m dot (integral from p to v dp plus delta l sub e)

p v equals R T

v equals R T divided by p

equals minus m dot (R T

c) niges extra equals m dot (h minus h sub o minus T sub o (s minus s sub o) plus k sub e plus p e squared)

e sub xtra equals h sub o minus h sub o minus T sub o (s minus s sub o) plus w sub e squared divided by two

e sub xtra, o equals h sub o minus h sub o minus T sub o (s sub o minus s sub o) plus w sub o squared divided by two

Delta e sub xtra equals h sub o minus h sub o minus T sub o (s sub o minus s sub o) plus w sub o squared divided by two minus w sub o squared divided by two

h sub o minus h sub o equals C sub p (T sub o minus T sub o)

s sub o minus s sub o equals C sub p ln (T sub o divided by T sub o) minus R ln (p sub o divided by p sub o)

Delta e sub xtra equals C sub p (T sub o minus T sub o) minus C sub p ln (T sub o divided by T sub o) plus w sub o squared divided by two minus w sub o squared divided by two

equals one point zero zero six kilojoules per kilogram Kelvin (three hundred forty Kelvin minus (minus thirty plus two hundred seventy-three point fifteen) Kelvin) minus (minus thirty plus two hundred seventy-three point fifteen) Kelvin times one point zero zero one kilojoules per kilogram Kelvin ln (three hundred forty divided by (minus thirty plus two hundred seventy-three point fifteen)) plus five hundred ten squared divided by two minus two hundred twenty squared divided by two

equals one hundred ten point zero six five kilojoules per kilogram equals one hundred ten point one megajoules per kilogram