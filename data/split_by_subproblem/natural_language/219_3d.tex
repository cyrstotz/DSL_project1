2. HS: 0 equals s subscript e minus s subscript a plus c subscript f plus s subscript e z

equals s subscript e z equals Delta s subscript o c equals c subscript p, left parenthesis ln left parenthesis T subscript c divided by T subscript o right parenthesis right parenthesis minus R left parenthesis ln left parenthesis p subscript c divided by p subscript o right parenthesis right parenthesis

equals 0.3014 kilojoules per kilogram Kelvin

e subscript y, vel equals T subscript c s subscript e z

equals 73.276 kilojoules per kilogram

Graph: A graph is drawn with T on the vertical axis and s on the horizontal axis. There is a curve labeled 1 to 2 with a point labeled o.

W subscript v, x, 2 equals m subscript s times rho subscript s dv equals rho subscript s times (V subscript 2 minus V subscript 1).
Equals minus 211.2 joules.
Therefore Q subscript 12 equals minus 1015.2 joules.

T constant