A graph is drawn with an upward curve labeled 'p5' and 'p0/p6'. The vertical axis is labeled 'T [k]' and the horizontal axis is labeled 'S (w2/y2*k)'. There is an arrow pointing upwards labeled 'isobaren'. Another arrow labeled 'adiabate kompressor' is drawn vertically between two points on the curve.

b) Given: omega 0 plus T6  
Due: Q equals 0, omega equals 0 implies h5 equals h6  
omega j equals omega 0 equals 220 meters per second

c) m dot ext equals m dot [h6 minus h0 minus T0 (s6 minus s0) plus ke plus pe]  
m ext equals m gas [cp (T6 minus T0) minus T0 (cp ln (T6 over T0) minus R ln (p6 over p0)) plus (200 meters per second squared minus (5 times 10 to the power of 5 meters per second squared)) over 2]  
m ext equals m gas [1.006 (340 k minus 243.15) minus 243.15 (1.006 (340 over 243.15))] minus  
R equals cp minus cv  
n equals k equals cp over cv implies cv equals cp over k

d) e x, vel equals T0 times S dot 0 over m dot gas

b) The substance is R134a. The process is a closed cycle process with an energy balance equation: 
0 = ṁ (h1 - h2) + Q̇k
Q̇k = ṁ (h2 - h3)
0 = ṁ (h2 - h3) - Wk

The mass flow rate equation is given as:
ṁ = Wk / (h2 - h3) = -28 kW / (234.08 - 593.39) kJ/kg = 0.083 kg/s

A table with columns labeled 1, 2, 3, and 4 is shown. It includes rows for pressure (p), quality (x), temperature (T), and enthalpy (h), with specific values filled in for each column.

Additional notes include:
p3 = p4
h1 = h4
T4 = T3

Interpolations are shown with calculations for h3 using given enthalpy values:
h3 = h(400) + (h(50) - h(400)) * s3 / (s(50) - s(400))
h3 = 293.66 + (284.33 - 293.66) * 0.3351 / (0.3344 - 0.3344) = 521.39 kJ/kg

Other calculations and references to tables (e.g., Tab A10) are noted.