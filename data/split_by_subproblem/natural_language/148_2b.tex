The equation is shown as "integral of w squared over 2 = m dot (h exit plus w exit squared over 2)". 

Below, there is a reference to "Tabelle A-2", and several calculations are shown:
- h exit minus h ambient = c p times (T2 minus T1) = 1.006 times (526.948 minus 431.9) = 95.58 kilojoules per kilogram.
- T1 equals 431.9 Kelvin.
- T2 is calculated as T1 times 0.22, which equals 526.948 Kelvin, therefore T2 equals T6.
- The equation "T2 over T1 = (rho2 over rho1) raised to (n minus 1) over n" is shown.
- Another equation shown is "95.58 kilojoules per kilogram plus 220 meters per second squared over 2 minus w ambient squared over 2 equals 0".
- The calculation "w ambient squared over 2 equals 95580 joules per kilogram plus 24200 joules per kilogram" is shown.

There is a large cross drawn over the bottom portion of the page.

T6 over T5 equals P6 over P5 to the power of n minus 1 over n.

T6 over T5 equals 0.191 over 0.5 to the power of 1 over 1.4.

T6 equals T5 times 0.759.

T6 equals 327.81 K.

he minus ha plus we squared over 2 minus wa squared over 2 equals 0.

he minus ha equals 104.71.

wa squared over 2 equals 104710 Joules per kilogram plus 24200 Joules per kilogram.

wa squared over 2 equals 128910 Joules per kilogram.

wa squared equals 257820 Joules per kilogram.

wa equals 507.76 meters per second.