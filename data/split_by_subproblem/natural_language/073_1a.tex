a) O equals m dot times (h ein minus h aus) plus G aus.

G aus equals m dot u times (h aus minus h ein)

equals m dot u times [c times (T aus minus T ein) plus v i squared times (p aus minus p ein)]

E dot total equals T zero S zero dot equals zero (da es reversibel)

O equals m dot (h S minus h G minus T C (S S minus S C) plus alpha k e) minus W T dot

W T equals (crossed out section)

W T dot equals R (T G minus T S) over one minus k (ideales Gas und isentrop)

W T equals m dot c S C

Therefore, O equals m dot (h S minus h C minus T C (S S minus S C) plus one half (omega zero squared minus omega S squared)) minus R (T G minus T S) over one minus k minus m dot eta

R (T G minus T S) over one minus k equals h S minus h C minus T C (S S minus S C) plus one half c S squared minus one half omega S squared

(S S minus S C) equals star C P ln (T S over T G) minus R ln (P S over P G)

(h C minus h S) equals C P (T G minus T S)

Therefore, one half omega G squared equals R (T G minus T S) over one minus k plus one half c S squared plus T C [C P ln (T S over T G) minus R ln (P S over P G)] plus C P (T C minus T S)

Therefore, omega G equals square root of [two R (T G minus T S) over one minus k minus omega S squared minus two T C [C P ln (T S over T G) minus R ln (P S over P G)] plus C P (T G minus T S)]