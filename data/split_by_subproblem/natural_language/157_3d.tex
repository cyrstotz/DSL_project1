T subscript 12 equals 0.003 degrees Celsius equals T subscript EW

m subscript EW equals 0.1 kilograms

Delta E equals Q subscript 12

U subscript 2 minus U subscript 1 equals i Q subscript 12

u subscript 1 equals minus 0.045 kilojoules per kilogram plus X subscript E1/2 times (minus 333.458 kilojoules per kilogram plus 0.045 kilojoules per kilogram) equals minus 200.09 kilojoules per kilogram

U subscript 1 equals u subscript 1 times m subscript n equals minus 20 kilojoules equals minus 20000 joules

U subscript 2 equals (Q subscript 12 plus U subscript 1) equals minus 17954.16435 joules equals minus 17.95 kilojoules

u subscript 2 equals U subscript 2 over m subscript EW equals minus 179.546 kilojoules per kilogram

u subscript 2 equals u subscript FI (0.003 degrees Celsius) plus

U subscript 2 equals u subscript FI (0.0016 degrees Celsius) plus X subscript Eis1/2 (u subscript Fest minus u subscript FI) implies X subscript Eis1/2 equals u subscript 2 minus u subscript FI over u subscript Fest minus u subscript FI

X subscript Eis1/2 equals 0.5384 approximately 0.538