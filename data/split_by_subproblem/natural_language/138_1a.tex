Set up the energy balance for the open system of the reactor:

E dot equals m dot e in plus P e plus Q out dot plus W dot over two minus m dot e out times h out plus u out squared over two plus P out over rho out equals zero for the stationary case.

The potential and kinetic energies are negligible, so:

E dot equals m dot e in plus Q out dot minus m dot e out times h out equals zero implies Q out dot equals m dot e out times h in minus h out.

According to the table for the corresponding enthalpy at 780 degrees Celsius or A equals two with x equals zero:

h in at 780 degrees Celsius equals 7,092.8 kilojoules per kilogram.
h out at 1000 degrees Celsius equals 4,789.04 kilojoules per kilogram.

Therefore:

Q out dot equals 196 kilojoules per second times 0.36 kilograms per second times (7,092.8 kilojoules per kilogram minus 4,789.04 kilojoules per kilogram) equals approximately 6,218.2 kilowatts.

v initial minus v final equals zero  
m liquid equals 5755 kilograms  
T initial equals 100 degrees Celsius  
x initial equals 0.005  
T ambient equals 20 degrees Celsius  
x D, 2 equals zero  

Delta m liquid equals ?  
T final, 12 equals 20 degrees Celsius  

=> We have a mass with m liquid with 0.005 mass fraction as steam and then as pure liquid at 100 degrees Celsius.  

=> Equilibrium of the closed system:  

E dot equals d by dt of KE plus d by dt of PE plus d by dt of U equals Q dot minus W dot mechanical => u dot equals Q dot  

Inner energy in state 4 to A-2:  

u steam (100 degrees Celsius) equals 250.05 kilojoules per kilogram  
u liquid (100 degrees Celsius) equals 41.98 kilojoules per kilogram  
u liquid (20 degrees Celsius) equals 83.93 kilojoules per kilogram  

=>  

Delta m liquid equals m liquid minus m liquid times u liquid (100 degrees Celsius) minus m liquid times (x initial times u steam plus (1 minus x initial) times u liquid (100 degrees Celsius)) divided by u liquid (20 degrees Celsius)  

=> m liquid times q liquid (100 degrees Celsius) minus (m liquid times x initial times u steam plus (1 minus x initial) times u liquid (100 degrees Celsius)) equals Q  

=> (m liquid plus Delta m liquid) times u liquid (20 degrees Celsius) minus (m liquid times x D, 0 plus (1 minus x D) times q liquid) equals Q final, 12  

approximately equals 2794.6 kilograms