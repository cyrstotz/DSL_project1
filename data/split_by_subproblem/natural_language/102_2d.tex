Q H two equals m times c p times (T two minus T one)

c p equals R plus c v equals R over M plus c v

known heat point

m gas equals 3.6 times 10 to the power of minus 3 kilograms

T two equals 0.003 degrees Celsius

Q H two equals 3.6 times 10 to the power of minus 3 kilograms over 50 kilograms per kilometer times 8.314 joules per mole Kelvin over 0.623 kilojoules per kilogram Kelvin times (0.003 degrees Celsius minus 350 degrees Celsius) equals minus 14.356 kilojoules