A table with columns labeled 'p bar', 'T', and 'X':
- Row 1: 1, T1, 1 (isobar)
- Row 2: p1, isentrop
- Row 3: 8
- Row 4: 8 bar, 0

Equation: m dot times (h2 minus h3) minus W dot k equals 0

T2 equals T1 equals Ti minus 6 Kelvin equals minus 16 degrees Celsius equals 257.15 Kelvin

Ti equals minus 10 degrees Celsius

h2: A10 at minus 16 degrees Celsius, X equals 1, h2 equals h g at minus 16 degrees Celsius

h2 equals 237.74 kilojoules per kilogram, S2 equals 0.9288 kilojoules per kilogram

h3: A11 at 8 bar, h3 equals hf at 8 bar plus S2 minus Sg at 8 bar times (h g at 8 bar minus hf at 8 bar)

h3 equals 93.42 kilojoules per kilogram plus 0.9288 minus 0.9459 times (269.15 kilojoules per kilogram minus 93.42 kilojoules per kilogram)

A12 at 8 bar

h3 equals h sat plus S2 minus S sat times (h sat minus h sat)

equals 269.15 plus 0.9288 minus 0.9064 times (273.6 minus 269.15) equals 271.31 kilojoules per kilogram