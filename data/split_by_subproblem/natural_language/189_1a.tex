a) Energy balance for the reaction mixture:  
dE/dt = Σiṁihi + Σjṁjkrj(Pi) + ΣjṁjRj + ΣiṁiEini

=> 0 = ṁin hin - ṁaus haus + Q̇R - Q̇aus

ṁin = ṁaus

=> Q̇aus = ṁein (hein - haus) + Q̇R

hein = hF (70°C) - hAUS = 292.88 kJ/kg → TAB-A2 @ 70°C  
haus = hE (100°C) - hAUS = 415.04 kJ/kg → TAB-A2 @ 110°C

=> Q̇aus = 62.192 kW

b) T̅ = ∫ T ds / (sA - sE)

Energy balance on reaction mixture

b) Delta E equals Delta m12 times (u2 minus u1) equals m total times h1 plus Q dot R minus Q dot out

implies Delta u m12 equals m total times h1 plus Q dot R minus Q dot out over u2 minus u1

h1: h1 equals h e(100 degrees Celsius) plus x p times (h g(100 degrees Celsius) minus h c(100 degrees Celsius))

equals 419.04 plus 0.005 times (2676.1 minus 419.04) equals 430.325 kilojoules per kilogram

u1: u1 equals u f(100 degrees Celsius) plus x p times (u g(100 degrees Celsius) minus u c(100 degrees Celsius))

equals (418.91 plus 0.005 times (2506.5 minus 418.91)) equals (429.377 kilojoules per kilogram)

u2: T2 equals 70 degrees Celsius

Direct adiabatic implies u four equals u one. Isobaric condensation implies p four equals p three equals eight bar. Therefore, u four equals u three equals u g (x equals zero) equals zero point zero two five kilojoules per kilogram.