w subscript 5 equals 220 meters per second

w subscript 6 equals 200 meters per second

m dot g equals

k v subscript s less than isentrop

m dot h equals

m dot u equals 5.293

eta subscript B equals Q dot subscript 0 divided by m dot equals 11.95 kT subscript wB

T subscript B equals 1289

P  T
0  0.194 bar  -30 degrees
5  0.5 bar  431.9 K

c subscript p equals 1.005

n equals k equals 1.4

a)

[Diagram of a graph with axes labeled T in kilojoules and s in kilojoules per kilogram Kelvin. The graph shows a cycle with points labeled 1, 2, 3, 4, 5, 6. Arrows indicate directions of the cycle and processes such as isentrop and isobar.]

b)

W subscript 6 equals ? T subscript 6 equals ?

w subscript rev equals integral from 0 to 6 of p d v equals p

T subscript 0 divided by T subscript 6 equals T subscript 6 divided by T subscript 5 times (P subscript 6 divided by P subscript 5) raised to the power of n minus 1 divided by n

Therefore, T subscript 6 equals T subscript 5 times (P subscript 0 divided by P subscript 5) raised to the power of n minus 1 divided by n

equals 431.9 K times (0.194 bar divided by 0.5 bar) raised to the power of 0.4 divided by 1.4 equals 328.075 K equals T subscript 6

[Arrow pointing to isentrop n equals k equals 1.4]

stat.

1 divided by m dot

O equals m dot times (h subscript 2 minus h subscript a) plus (w subscript 2 squared minus w subscript a squared) divided by 2 plus sum of Q dot minus sum of W dot subscript e

Diagram:  
A graph with labeled axes, p (bar) on the vertical axis and v (m cubed per kg) on the horizontal axis.  
Points labeled 1, 2, 3, with arrows indicating processes:  
1 to 2 is isotherm,  
2 to 3 is isentrop,  
3 to 1 is isobar.