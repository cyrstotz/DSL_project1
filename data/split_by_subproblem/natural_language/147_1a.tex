a) For steady flow process:
Q dot equals m dot KF times (h e minus h a plus (w e squared minus w a squared) divided by 2 plus g times (z e minus z a)) equals Q dot KF

Stationary reactor: (since system mass is constant, m in equals m out)
Q dot equals m dot in times (h e minus h a plus (w e squared minus w a squared) divided by 2 plus g times (z e minus z a)) equals Q dot out plus Q dot R

h e equals h f at 70 degrees Celsius equals 292.98 kilojoules per kilogram
h a equals h f at 800 degrees Celsius equals 419.04 kilojoules per kilogram

Q dot out equals 0.3 kilograms per second times (292.98 kilojoules per kilogram minus 419.04 kilojoules per kilogram) plus 100 kilojoules per second equals 62.182 kilowatts

W v,12 = integral from 1 to 2 of p dv

(p is constant)

= integral from 1 to 2 of 1.4 times 10^5 pascal times dv

= 1.4 times 10^5 pascal times (1.4 times 10^(-3) minus 3.4 times 10^(-3)) cubic meters

= -284.4 joules per meter = -873.16 joules

U2 - U1 plus V v,12 = Q12

(perfect gas)

Q12 = Cv times (T2 - T1) plus V v,12

= 0.653 joules per gram Kelvin times (273.15 Kelvin - 273.15 Kelvin) times 3.422 grams minus 873.16 joules

= -2056 joules

d) Closed system with water:

U2,w minus U1,w (change in energy) = 0 (incompressible, density difference negligible)

E2 - E1 = Q12 - W

U2 - U1 = Q12

P w = P ambient

m times (U2,w - U1,w) = -Q12

U2,w = Q12 over m plus U1,w

= 2056 joules minus 200.09 joules per gram

3.422 grams

= -400.73 joules per gram

P w = P ambient plus 30 kilograms per meter squared times gravity

= -1 bar plus 30 kilograms times 9.81 meters per second squared times pi over 400 cubic meters

= 1.33992 bar approximately equals 1.4 bar

U1 = U1, ambient plus x times (U liquid, ambient - U1, ambient)

= (0.045 plus 0.6 times 333.458 plus 0.045) kilojoules per gram

= -200.098 joules per gram

Continuation on another sheet!