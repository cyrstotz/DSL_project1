Q aus 
m punkt wasser equals 0.5 kilograms per second
Q aus equals m punkt (h wasser minus h ein)
A-2 h ein equals h f (80 degrees Celsius) equals 292.88 kilojoules per kilogram
h ein
A-2 (100 degrees Celsius) h two equals h f plus x (h fg minus h f) equals 430.25 kilojoules per kilogram
A-2 h aus equals h f (100 degrees Celsius) equals 418.94 kilojoules per kilogram
Q aus equals minus Q R plus m punkt (h aus minus h ein) equals minus 62.184 kilojoules per second

c) w subscript 6 equals five hundred ten meters per second, T subscript 6 equals three hundred forty Kelvin

b) w subscript 6 equals square root of two c p j times (T subscript 6 minus T bar) T bar equals fifteen times (zero point one eight one divided by zero point five) times (zero point four divided by one point four) equals three hundred twenty-eight Kelvin

w subscript c equals square root of two c p times j times (T subscript c minus T bar) plus w subscript c squared equals

c) e subscript x, str equals E dot subscript x, str divided by m dot equals h subscript 6 minus h subscript 0 minus T subscript 0 (s subscript 6 minus s subscript 0) plus (five hundred ten minus two hundred meters per second) squared divided by two

E dot subscript x, str divided by m dot equals c p subscript is j (T bar minus T subscript 0) minus T subscript 0 (c p subscript is j times ln (T bar divided by T subscript 1)) minus R ln (p subscript 1 divided by p subscript 0)

T bar equals minus thirty degrees Celsius equals two hundred seventy-three point fifteen minus thirty Kelvin equals two hundred forty-three point fifteen Kelvin

equals approximately fifteen kilojoules per kilogram