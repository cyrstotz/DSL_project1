m dot equals 0.3 kilograms per second

Q dot aus equals m dot times (h aus minus h ein) minus Q dot a

Table A2:

h ein equals h f at 70 degrees Celsius equals 292.88 kilojoules per kilogram

h aus equals h f at 100 degrees Celsius equals 419.04 kilojoules per kilogram

Q dot aus equals minus 62.187 kilowatts

c)

Zero equals m dot in times (se minus sa) plus Q dot aus divided by T KF plus Q dot R divided by T R plus Serz.

Serz equals m dot in times (sa minus se) plus Q dot aus divided by T KF equals Q dot R divided by T R.

Serz equals minus Q dot aus divided by T KF equals 0.212 kilojoules per kilogram Kelvin.

d)

m2 U2 minus m1 U1 equals delta m2 hein.

Table A-2 for u1f, u1g, u1 (70 degrees Celsius), hein.

U1f equals 418.94 kilojoules per kilogram, U1g equals 2506.4 kilojoules per kilogram.

U1 equals u1f plus x0 times (u1g minus u1f) equals 429.8 kilojoules per kilogram.

U2 equals u1 (70 degrees Celsius) equals 292.95.

hein equals 83.96 kilojoules per kilogram.

m2 equals delta m2 plus m1.

delta m2 U2 plus m1 (U2 minus U1) equals delta m2 hein.

m1 (U2 minus U1) divided by hein minus U2 equals delta m2.

delta m2 equals 3756.9 kilograms.

Delta S one two equals S two minus S one

Table A-2:

S one hundred degrees Celsius f equals one point three zero six kilojoules per kilogram Kelvin, S one hundred degrees Celsius g equals seven point three five eight kilojoules per kilogram Kelvin

S seventy degrees Celsius f equals zero point nine five five kilojoules per kilogram Kelvin

S one equals S f plus x zero point six S g minus S f equals one point three three seven kilojoules per kilogram Kelvin

S two equals zero point nine five five kilojoules per kilogram Kelvin

Delta S one two equals S two minus S one equals negative zero point three eight two kilojoules per kilogram Kelvin