Zero equals m dot times (h e minus h a plus (omega e squared minus omega a squared) divided by two) plus Q dot minus W dot

Isentrope Schubdüse

implies nu equals H equals 1.9

T2 divided by T1 equals (nu times tau divided by nu a) to the power of n minus one divided by n

hier equals T2 divided by T5 equals (nu divided by nu 5) to the power of K minus one divided by K

T6 equals T5 times (nu 0 divided by nu 5) to the power of K minus one divided by K equals 328.07 Kelvin

m dot R times (h2 minus h3) minus W dot K

x2 equals 1

h2 equals h g bei 110 degrees Celsius

Interval über 8 degrees and 110 degrees Celsius

259.03 minus 257.8 divided by h2 minus 8 times (110 degrees Celsius minus 8 degrees Celsius) plus 257.8 equals

h2 equals 252.396 kilojoules per kilogram

h3 implies 8 bar

Ti less than 40 degrees

h3 equals

m dot R, 190a equals W dot K divided by h2 minus h3

A diagram is present with an arrow pointing upwards labeled 'h' and an arrow pointing to the right labeled 'T'.