a) p_g,1 = ?

C_v = 0.633 kilojoule per kilogram Kelvin  
M_g = 50 kilograms per mole  
T = 500 degrees Celsius = 773 Kelvin  
V = 3.14 liters = 3.14 times 10 to the power of negative 3 cubic meters  

pV = mRT  
(unknown)  

1) Find p  
2) use m = pV over RT to find m  

p_1,g = ?  

p = p_amb + 32 kilograms per square second over A_0  

1 bar + 32 (kilograms per square meter) times 9.81 over square second times A_0 (square meter)  

A_0 = pi r squared = pi times (40 over 2) squared = 400 pi = 25 pi (square centimeter)  

25 pi square centimeter = 25 pi times 0.01 squared square meter  

p = 1 bar + 32 kilograms per square meter over 25 pi times 0.01 squared square meter  

p_1 = 1 bar + 32 times 9.81 over 25 pi times 10 to the power of negative 4  

1 bar + 39,971 pascal (newton per square meter) = 1 bar + 0.3997 bar =  

= 1.3997 bar

m = ?

m = (1.3997 bar) times (3.14 times 10 to the power of negative 3 cubic meters) divided by (R times 773 Kelvin)

R = (8.314 kilojoules per kilomole per Kelvin) divided by (50 kilograms per kilomole) = 0.1663 kilojoules per kilogram per Kelvin

m = (1.3997 times 10 to the power of 5 kilograms per meter squared per second squared) times (3.14 times 10 to the power of negative 3 cubic meters) times kilograms times Kelvin divided by (meters squared times seconds squared) times 0.1663 kilojoules times 773 Kelvin

m = (10 to the power of 3 kilograms per meter squared per second squared) implies m = 3.92 times 10 to the power of negative 3 kilojoules

m = 3.92 times 10 to the power of negative 3 kilojoules = 3.92 grams