T sub g,1 equals five hundred degrees Celsius minus seven hundred seventy-three point one five Kelvin.

T sub g,2 equals zero degrees Celsius equals two hundred seventy-three point one five Kelvin.

Delta T equals five hundred Kelvin.

v sub 2 equals m R T sub 2 over M P sub 2 equals three point four two five times eight point three seven four cubic meters per mole times two hundred seventy-three point one five Kelvin over fifty moles per cubic meter times one point four zero one bar.

v sub 2 equals zero point zero zero four one cubic meters.

v sub 1 equals one point one zero eight cubic meters.

W sub 12 equals integral from v sub 1 to v sub 2 of P sub e dV equals one point four zero one bar times (v sub 2 minus v sub 1).

Q sub 12 equals c sub v times m times Delta T equals zero point six three three kilojoules per kilogram Kelvin times three point four two five kilograms times five hundred Kelvin.

Q sub 12 equals one point zero eight kilojoules.