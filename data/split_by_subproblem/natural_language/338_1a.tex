a) Find Q out

[Diagram]
Q out
m in
V
m out
Q R equals 100 kW

O equals m (h in minus h aus) plus Q

h ein (70 °C)

h ein minus h aus equals c if (T ein minus T aus) plus v if (p ein minus p aus)

h ein minus h aus equals c if (288.15 K minus 298.15 K)

b) T equals integral from S a to S s of T d S over S a minus S e equals h

b) T equals integral from S a to S s of T d S over S a minus S e equals h a minus h e over S a minus S e

since reversible stationary process

h ein minus h aus equals c if (T ein minus T aus) plus v if (p ein minus p aus)

c) O equals m ein (s ein minus s aus) plus sum of Q i over T i plus S er z

d) d E over d t equals m w (h ein minus h aus) plus sum of Q minus W

e) Delta S 12 equals m (s 2 minus s 1) m 2 s 2 minus m 1 s 1 equals sum of Q i over T i plus S er z

O equals m sub ges delta e sub x sub s sub r sub o plus Z minus W minus E sub verl

i sub Wo equals

E sub verl equals m sub ges delta e sub x sub s sub r sub o minus W

Delta E twelve equals m times (u two minus u one) equals Q twelve minus W twelve  
u two minus u one equals minus three hundred sixteen point four nine eight kilojoules per kilogram  
W twelve equals minus zero point three eight six kilojoules  
m times (u two minus u one) plus W twelve equals Q twelve equals minus one thousand four hundred sixty-eight kilojoules  
m equals three point four two grams