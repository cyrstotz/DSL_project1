Plus delta e x i s t r equals h zero minus h six minus T zero times (s six minus s zero) plus delta k e equals (h zero minus h six minus T c times (s six minus s zero)) plus (omega two squared minus omega six squared) divided by two. Given from task two.

h zero minus h six equals c p times (T zero minus T six) minus one point zero zero zero kilojoules per kilogram times (243.15 Kelvin minus 340 Kelvin).

T c equals minus 30 plus 273.15 equals 243.15 Kelvin.

h zero minus h six approximately equals minus 97.43 kilojoules per kilogram.

s zero minus s six equals zero, because adiabatic and reversible.

Therefore, delta e x i s t r equals (h zero minus h six plus (omega two squared minus omega six squared) divided by two) equals (minus 97.43 kilojoules per kilogram plus (200 meters squared per second squared minus 510 meters squared per second squared) divided by two) approximately equals (minus 97.43 kilojoules per kilogram plus 200 squared meters per second squared divided by 2 minus 510 squared meters per second squared divided by kilogram). 

Approximately equals plus 207.48 kilojoules per kilogram.

h sub n equals h sub f plus x sub 1 times (h sub g minus h sub f)

x sub 1 equals (h sub n minus h sub f) over (h sub g minus h sub f)

T sub i equals 10 minus 273.15 minus 20 equals minus 283.15 degrees Celsius