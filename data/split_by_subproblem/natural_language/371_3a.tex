P three plus one equals open parenthesis m two v two squared over two close parenthesis plus m one v one squared over two close parenthesis over A plus p zero

equals open parenthesis zero point one kilogram three point eight meters per second squared over A close parenthesis plus thirty-two kilograms nine point eight meters per second squared over A close parenthesis plus ten to the power of five pascals

equals one hundred forty thousand thirty-seven point nine pascals equals one point four bar

m g Ideal Gas Equation

m g equals p V g one over R T g one

R equals eight point three four joules per mole kelvin over fifty kilograms per mole equals one hundred sixty-six point two eight joules per kilogram kelvin

therefore m g equals fourteen thousand one hundred seventy-nine pascals times three point four nine times ten to the power of minus three cubic meters over one hundred sixty-six point two eight joules per kilogram kelvin times two hundred seventy-three point one five kelvin

equals zero point zero zero three four two kilograms equals three point four two grams