P sub g equals P sub amb plus P sub hu plus P sub ew, P sub amb equals 1 bar.

P sub hu equals thirty-two kilograms times nine point eight one meters per second squared equals N over A equals thirty-nine thousand six hundred sixty-five pascals over (zero point zero five meters squared) pi.

P sub ew equals N over A equals zero point one kilograms times nine point eight one meters per second squared over (zero point zero five meters squared) pi equals one hundred twenty-five pascals.

Therefore, P sub g equals one hundred thousand pascals plus thirty-nine thousand six hundred sixty-five pascals plus one hundred twenty-five pascals equals one hundred forty thousand pascals equals one point four bar.

m sub g equals P sub g V sub g over R T sub g equals B over M sub g equals eight point three four five joules per mole kelvin over fifty kilograms per mole equals one hundred sixty-six point three joules per kilogram kelvin.

Therefore, m sub g equals one hundred forty thousand pascals times zero point zero zero three one four meters cubed over one hundred sixty-six point three joules per kilogram kelvin times seven hundred seventy-three point one five kelvin equals three point four grams.