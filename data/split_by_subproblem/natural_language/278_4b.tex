p two equals p two, p three equals p four equals three bar, implies p two equals p one minus p. h four equals h two equals h f (eighty bar) equals ninety-three point four two kilojoules per kilogram, s three equals s two. 

Q equals m dot times (h two minus h three) minus W dot k. 

m dot equals W k over (h two minus h three) equals negative twenty-eight kilowatts over (two hundred thirty-seven point seventy-four minus two hundred fifty-three point three one) kilojoules per kilogram equals one point eight zero kilograms per second. 

h two equals h g (two hundred fifty-seven point one five kilojoules) equals h g (negative sixteen degrees Celsius) equals two hundred thirty-seven point seventy-four kilojoules per kilogram. 

T two equals T two equals T i minus T k equals T two equals one hundred ten minus zero Kelvin minus two hundred seventy-three point one five Kelvin minus twenty Kelvin plus forty-six Kelvin equals two hundred fifty-seven point one five equals negative sixteen degrees Celsius. 

h three (eighty bar, s two) equals ninety-three point four two kilojoules per kilogram, equals two hundred fifty-three point thirty-two kilojoules per kilogram. 

s two equals s g (negative sixteen degrees Celsius) equals zero point nine two five eight kilojoules per kilogram Kelvin equals s three. 

h three (eighty bar, s two) equals two hundred forty-six point one five kilojoules per kilogram equals two hundred seventy-three point six six minus two hundred forty-six point one three over zero point nine three seven four minus zero point eight zero six six (zero point nine two five eight minus zero point eight zero six six) equals two hundred fifty-three point thirty-two kilojoules per kilogram.