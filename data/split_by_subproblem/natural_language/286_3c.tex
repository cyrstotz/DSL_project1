Weiter mit T K equals 285 Kelvin.

m times v equals five hundred ten cubic meters, T sub e equals three hundred forty Kelvin

Exergie eines Stroms

e sub x, ström equals m dot times (h sub e minus h sub o minus T sub o times (s sub e minus s sub o) plus ke plus pe)

implies e sub x, ström equals h sub e minus h sub o minus T sub o times (s sub e minus s sub o) plus ke

ke equals one divided by two times five hundred ten squared divided by two times seven hundred squared equals one hundred ten thousand fifty Joules divided by kilogram equals one hundred ten point zero five Joules divided by kilogram

h sub e minus h sub o equals integral from T sub o to T sub e of c sub p of T dT

implies h sub e minus h sub o equals one thousand point zero six Joules divided by kilogram Kelvin times three hundred forty Kelvin minus one thousand point zero six Joules divided by kilogram Kelvin times two hundred forty three point one five Kelvin equals eighty seven Joules divided by kilogram

s sub e minus s sub o equals c sub p times integral from T sub o to T sub e of dT divided by T minus R ln of p sub e divided by p sub o

implies s sub e minus s sub o equals one thousand point zero six Joules divided by kilogram Kelvin times ln of three hundred forty Kelvin divided by two hundred forty three point one five Kelvin minus eight point three four ln of one divided by one

implies s sub e minus s sub o equals zero point three three seven two Joules divided by kilogram Kelvin

implies e sub x, ström equals eighty seven Joules divided by kilogram plus two hundred forty three point fifteen minus zero point three three seven two Joules divided by kilogram Kelvin plus one hundred ten Joules divided by kilogram

e sub x, ström equals one hundred twenty five point zero six Joules divided by kilogram

p2 equals p1 times V2 over V1.  
T2 over T1 equals p2 over p1 to the power of n minus 1 over n.  
n equals Cp over Cv implies Cp minus Cp times R over M implies Cp equals 0.799 times 28.  
implies n equals 0.799 over 0.633 equals 1.263.  
implies p2 over p1 equals rho minus 1 over rho minus 1 times T2 over T1 implies 0.265 over p2 equals 0.379 over 9 bar implies p2 equals 9.67 times 10 to the power of minus 3 bar.