Q dot aus equals sixty-five kilowatts

T bar HF equals integral from sein to saus T ds equals q kv divided by saus minus sein

From energy balance: q kv equals h aus minus h ein equals c i F times (T aus minus T ein)

sa us minus sein equals integral from T ein to T aus of one over T c i F dT equals c i F times natural logarithm of (T aus divided by T ein)

Therefore, T bar HF equals (T aus minus T ein) divided by natural logarithm of (T aus divided by T ein) equals two hundred ninety-three point one two three Kelvin

pc equals p0 equals 0.191 bar

TG equals ? ; TS equals 431.9 K ; pS equals 0.5 bar ; k equals 1.4

→ isentropic equation:

TG divided by TS equals (pG divided by pS) raised to the power of k minus one divided by k → TG equals TS times (pG divided by pS) raised to the power of k minus one divided by k equals 328.07 K

Energy balance

0 equals m dot times (h0 minus h6) plus one half times (w0 squared minus w6 squared) plus sum of Qj dot minus sum of Wt,n dot

2 times (h6 minus h0) equals w0 squared minus w6 squared

w6 squared equals w0 squared plus 2 times (h0 minus h6)

w6 squared equals w0 squared plus 2 times cp ig, Luft times (T0 minus TG)

w0 squared equals 200 squared meters squared per second squared plus 2 times 1.006 kilojoules per kilogram times (293.15 minus 328.07) Kelvin

equals 200 squared meters squared per second squared plus 2 times 1.006 kilojoules per kilogram times 1 kilojoule per kilogram times 1,000

w6 equals 450 meters per second equals 458.19 meters per second