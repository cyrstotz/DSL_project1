A diagram with an arrow pointing up and labeled Q is drawn, depicting a house.

The equation dE over dt equals the sum of Q minus the sum of W is written.

Delta U equals Q. Delta u equals q.

Q12 equals 1500 J. q equals Q12 over m equals 1500 over 0.1 kilograms equals 15000 joules per kilogram equals 15 kilojoules per kilogram.

u2 minus u1 equals 15 kilojoules per kilogram. u2 equals 15 kilojoules per kilogram plus u1.

u1: p equals 1.1 bar, T1 equals 0 degrees Celsius, x1 equals 0.6.

u1 equals x1 times uFest plus (1 minus x1) times uFluessig equals 0.6 times (-333.458) plus 0.4 times (-3.045).

u1 equals -200.0928 kilojoules per kilogram.

u2 equals 15 kilojoules per kilogram plus 200.0928 kilojoules per kilogram equals -185.0928 kilojoules per kilogram.

xEis;2 equals xEis times uFest plus (1 minus xEis) times uFluessig equals -185.0928.

x1 times uP plus uFL minus x times uFL equals -155.

x times (uP minus uPL) minus 155 minus uFL equals -185.0928 minus uFluessig.

xEis equals -185.0928 plus 0.015 over uFest minus uFL equals -333.458 plus 0.015.

xEis;2 equals 0.555.

e subscript k equals Q dot subscript abl over W dot subscript k equals Q dot subscript abl over 28 W