k equals 1.4  
c sub p equals 1.006 kilojoules per kilogram Kelvin

Table:
- Row 0: P equals 0.49 bar, T equals 293.15 Kelvin
- Row 1: (empty)
- Row 2: (empty)
- Row 3: (empty)
- Row 4: (empty)
- Row 5: P equals 0.5 bar, T equals 431.9 Kelvin, W equals 220 meters cubed per second

Diagrams:
a) Left diagram: 
- Vertical axis labeled T (Kelvin)
- Horizontal axis labeled s (kilojoules per kilogram Kelvin)
- Curved lines with points labeled 0, 1, 2, 3, 4, 5
- Arrows indicating direction between points
- Additional labels: iso b, 6, 6a, 6b

Right diagram:
- Vertical axis labeled T (Kelvin)
- Horizontal axis labeled s (kilojoules per kilogram Kelvin)
- Curved lines with points labeled 0, 1, 2, 3, 4, 5
- Arrows indicating direction between points
- Additional label: isochore

b) W subscript G, T subscript G

W subscript S equals 220 meters per second

c) S to G

A equals m dot times the quantity h subscript S minus h subscript G plus W subscript S squared minus W subscript G squared divided by 2 minus W subscript close

W subscript G equals ...

T subscript G divided by T subscript S equals the quantity P subscript G divided by P subscript S raised to the power of k minus 1 over k

Therefore, T subscript G equals T subscript S times the quantity P subscript G divided by P subscript S raised to the power of k minus 1 over k equals 328.07 Kelvin

Ex B S to G

m dot times the quantity h subscript S minus h subscript G minus T subscript 0 times the quantity S subscript S minus S subscript G plus Q dot equals W subscript close

W subscript close equals m dot times the quantity C subscript P times the quantity T subscript S minus T subscript G minus T subscript 0 times the quantity C subscript P times ln of T subscript G divided by T subscript G minus R times ln of P subscript S divided by P subscript G.

wc equals 5 kilogram per second, Tg equals 39 degrees Celsius

mi equals

Sensible heat equals exshr.g minus ethwo

Sensible heat equals mi times [hG minus hs minus Ta (hG minus hS) plus (wG minus wL)] plus

Sensible heat equals cp (TG minus TA) minus Ta (cp ln (TG over TA) plus wG minus wL)

equals 335.42 kilojoules per kilogram