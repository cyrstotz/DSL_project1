absolute value of Q 12 equals absolute value of m times c times (T 2 minus T 1) equals 3.42 grams times 0.633 kilojoules per kilogram times 500 Kelvin equals 1.08 kilojoules equals 1082.45 divided by da Wärmefluss

U one ew plus Q two one equals U two ew

U one ew equals m times (one minus x) times U f passing plus x times U f rest, in brackets: one point four bar, zero degrees Celsius

equals zero point one kilograms times (zero point nine times minus zero point zero four five kilojoules per kilogram) plus zero point six times minus three hundred thirty-three point four five eight kilojoules per kilogram, equals minus twenty kilojoules

minus twenty kilojoules plus one thousand eighty-two point four five joules equals zero point one kilograms times (one minus x) times (minus zero point zero four five kilojoules per kilogram) plus x times minus three hundred thirty-three point four five eight kilojoules per kilogram

U two equals minus eighteen point five one seven six kilojoules per kilogram divided by zero point one kilograms equals minus one hundred eighty-five point one seven six kilojoules per kilogram

x two equals U two minus U f one divided by U f rest minus U f equals minus one hundred eighty-five point one seven six kilojoules per kilogram plus zero point zero four five kilojoules per kilogram divided by minus three hundred thirty-three point four five eight plus zero point zero four five kilojoules per kilogram equals zero point five six seven

p three equals p four equals eight bar  
u four equals u three equals h one eight minus six equals ninety-three point ninety-two kilojoules per kilogram  
h one equals three point ninety-two kilojoules per kilogram  
x one equals u two minus u f divided by u g minus u f  
h one equals h four  
day isenthalp drossel  
A minus ten  
p one equals one point five seven four eight bar (A minus ten, minus ten degrees Celsius)  
x one prime equals ninety-three point ninety-two kilojoules per kilogram minus twenty-nine point thirty kilojoules per kilogram divided by two hundred eight point forty-five kilojoules per kilogram equals zero point three zero eight