Given: p zero and p one

p zero equals p atmosphere plus F A zero divided by A.

Force divided by area equals pressure of the weight force. The mass m B equals 32 kg, 9.81 meters per second squared.

p zero equals 70 Pascal plus pi divided by 2 times radius squared for A zero equals (0.5 meters squared) times pi.

p one equals total mass. E equals 0.7 kg. Area remains the same.

Therefore, p one equals 70 Pascal plus (32 kg plus 0.7 kg) times 9.81 meters per second squared divided by (0.05 meters squared) equals 2273 Pascal.

p one approximately equals 3.74 bar.

Perfect gas: p times V equals m times R times T equals m times R divided by M times T.

p one equals 70 Pascal plus 32.7 kg times 9.81 meters per second squared divided by 0.05 meters squared equals 7.4 bar equals p one.

Perfect gas: p times V equals m times R times T equals m times R divided by M times T equals p one times V one times M divided by R times T one.

m equals 7.4 times 0.05 squared times 3.74 times 10 to the power of minus 3 times 3.50 times 10 to the power of minus 5 divided by 8.314 divided by 273.75 Kelvin equals 3.42 grams equals 3.42 times 10 to the power of minus 3 kg.