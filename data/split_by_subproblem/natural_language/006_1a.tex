a) 0 = m dot (h1 minus h2) plus Q dot R plus Q dot aus

h1: T1 equals 70 degrees Celsius, Fluss 8  
h1 equals 292.98 kilojoules per kilogram  

h2: T2 equals 100 degrees Celsius, Fluss 8  
h2 equals 419.04 kilojoules per kilogram  

minus Q dot aus equals 0.3 kilojoules per second (292.98 minus 419.04) plus 100 kilowatts

Q dot aus equals minus 62.182 kilowatts

b) T bar equals integral from sen to sa T ds over sa minus se equals T bar equals (298.15 plus 288.15) Kelvin over 2 equals 293.15 Kelvin

c) 0 equals m dot (se minus sa) plus Q dot aus over T bar plus S dot erz

S dot erz equals minus 0.3 kilograms per second (0.9549 minus 1.3069) kilojoules per kilogram Kelvin

62.182 kilowatts over 293.15 Kelvin equals 0.318 kilojoules per kilogram Kelvin

d) m2 u2 minus m1 u1 equals delta m2 h ein

equals 83.95 kilojoules per kilogram, delta T equals 20 degrees, Fluss 8

m2 arrow T2 equals 60 to 70 degrees Celsius, Fluss 8, and delta m2 equals 292.95 kilojoules per kilogram

m2 equals 5755 plus delta m2

u1 equals 418.94 plus 0.005 (2506.5 minus 429.94) equals 429.378 kilojoules per kilogram

m1 equals 5755 kilograms

(5755 kilograms plus delta m2) 100 292.95 kilojoules per kilogram minus 5755 kilograms 429.378 kilojoules per kilogram equals delta m2 83.95 kilojoules per kilogram

delta m2 292.95 kilojoules per kilogram minus delta m2 839.7 kilojoules per kilogram equals 78543.14 kilojoules per Kelvin

delta m2 equals 3756.67 kilograms