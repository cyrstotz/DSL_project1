Q dot out

Zero equals m dot times [h in minus h aus] plus Q dot aus

Zero equals m dot times [h (70 degrees Celsius) minus h (60 degrees Celsius)] plus Q dot aus

1 HS on cooling jacket

Zero equals m dot times [h (288.15 Kelvin) minus h (298.15 Kelvin)] plus Q dot aus

Q dot aus equals m dot KF times [h l (298.15 Kelvin) minus h (288.15 Kelvin)]

equals m dot KF times C p KF (298.15 Kelvin minus 288.15 Kelvin)

1 HS on reactor

Zero equals m dot times [h in minus h aus] plus Q dot R minus Q dot aus

h in equals 292.98 plus 0.005 times (2626.8 minus 292.98) TAB A equals 2

equals 304.64 kJ per kg

h aus equals 419.04 plus 0.005 times (2676.1 minus 419.04) TAB A equals 2

equals 430.325 kJ per kg

Q dot aus equals 0.3 kg per second times (crossed out terms) kJ per kg plus 100 kW

equals crossed out terms kW

62.2945 kW

A graph is drawn with the vertical axis labeled T in brackets K and the horizontal axis labeled S in brackets kJ over kg. The graph has several points labeled 0, 1, 2, 3, 4, 5, and 6. There are two pressure lines labeled P equals P2 and P equals P0. The temperature values on the vertical axis are marked as 1289, 431.9, 322.08, and 243.15.

A graph is drawn with two axes labeled. The vertical axis is labeled 'P [bar]', and the horizontal axis is labeled 'T [K]'. The graph contains a closed loop with four points labeled sequentially as 1, 2, 3, and 4. Arrows indicate the direction of the loop, starting from point 4 and moving to point 1, then to point 2, proceeding to point 3, and finally returning to point 4.