a) [Graph is drawn with axes labeled: vertical axis labeled as 'p' and horizontal axis labeled as 'T'. The graph shows a process with points labeled 1, 2, and 3, and lines connecting these points. The line from point 3 to 1 is horizontal. There are additional labels: 'p_tripel' and 'p_propen_isobar' on the graph.]

b) Energy balance on compressor

- Stationary flow process

E dot equals m dot R134a times (h2 minus h3) plus W dot c

m dot R134a equals W dot c over h3 minus h2

h2: s2 equals s1 isentrop equals s3 equals s2

T1 equals 20 degrees Celsius

- 1 to 2 isobar implies p2 equals pA

T2 equals 26 degrees Celsius

- h2 equals hg (minus 26 degrees Celsius) equals 231.62 kilojoules per kilogram implies TAD equals A70 at minus 26 degrees Celsius

h3: isentrop to s2: s2 equals sg (minus 26 degrees Celsius) equals 0.9390 implies TAD A70 at minus 26 degrees Celsius

h3 interpolieren aus TAB A-72 mit s3

s3 equals 0.9390 273.66 kilojoules per kilogram h3 0.9374 284.30 kilojoules per kilogram

h3 equals 273.66 0.9374 plus 284.30 minus 273.66 over (0.9390 minus 0.9374)

equals 274.74 kilojoules per kilogram

implies m dot R134a equals 0.0006478 kilograms per second equals 0.05298 kilograms per hour