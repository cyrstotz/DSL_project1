A graph is shown with the vertical axis labeled T in Kelvin and the horizontal axis labeled S in kilojoules per kilogram Kelvin. The graph depicts a series of processes labeled as follows: starting from a point labeled P6, moving upwards is an isentropic process to P2, followed by an isobar process to P3, then an isentropic process to P4. From P4, there is an isobar process labeled with pressure equal to 0.5 bar leading to P5. Finally, there is an isentropic process labeled as reversed back to P6.

c) Water with wc equals 510 meters per second, Tc equals 316 Kelvin.

Delta exergy, stream equals [hf minus h0 minus T0 (sc minus s0) plus ke c].

Equals

hf minus h0 equals cp [Tc minus T0]

sc minus s0 equals cp ln (Tc over T0) plus k ln (pc over p0), isobaric, pc equals p0

ke c equals one half wc squared

T0 equals minus 30 degrees Celsius equals 243.15 Kelvin

Delta exergy, stream equals cp [Tc minus T0] minus T0 ln (Tc over T0) plus one half wc squared.

Equals 13,084 kilojoules per kilogram.