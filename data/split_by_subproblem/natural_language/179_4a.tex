A graph is drawn with axes labeled P and T. The graph includes a curve labeled "Fest," a line labeled "isok," and another line labeled "Flüssig." An intersection point is marked as "Tripelpunkt," and a portion of the graph is labeled "gas."

Zero equals n times h zero minus h six plus v zero squared minus v six squared.

v zero squared divided by two equals h zero minus h six plus w zero squared divided by two.

w zero equals square root of v zero times h zero minus h six plus v zero squared equals one hundred ninety-three meters per second.

v zero equals square root of two times c p times T zero minus T six plus w zero squared.

T zero equals three hundred Kelvin.

T six equals three hundred twenty-three point zero four Kelvin.

c p equals one point zero zero six kilojoules per kilogram Kelvin.