Given: Wu equals 300 joules per kilogram, Cp equals 1,000 joules per kilogram Kelvin, n equals k equals 1.4.

|   | [T]  | [bar] | [kg] | Q | W | h | s |
|---|------|-------|------|---|---|---|---|
| Instruction |   |   |   |   |   |   |   |
| A | 30 | 0.5 |   | 0 |   | s4 |
| B |   |   |   |   |   | s4 |
| C |   |   |   |   |   | s4 |
| D |   |   |   |   |   | s4 |
| 1 | 0.5 |   |   |   |   | s4 |
| 2 | 439 Kelvin | 0.5 | 0 |   |   | s5 |
| 3 | 939 Kelvin | 9.5 |   |   |   | s5 |

Diagram:
- A graph with the x-axis labeled as 's' in kilojoules per kilogram Kelvin and the y-axis labeled as 'T' in Kelvin.
- Points are marked from 1 to 6 with lines connecting them.
- Additional notes: 0.5 bar, 0.9 bar plus 6 bar.

Q out equals m delta u plus 1 over V in. Equals m C V (T S minus T 1) plus p amb (V 2 minus V 1) equals 3.40 times 10 to the power of minus 5 kilogram times 0.653 kilojoule per kilogram Kelvin (373.153 Kelvin minus 773.15 Kelvin) plus 10 to the power of 6 cubic meter (c CO2 M1 times u1 minus c CO2 M2 times u2). Q out equals minus 204 plus 0.8 kilojoule.

A graph is shown with a curve plotted on a grid. The vertical axis is labeled P, and the horizontal axis is labeled T. There is a point marked on the curve.