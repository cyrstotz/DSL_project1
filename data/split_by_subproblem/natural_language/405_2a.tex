b) T6, w6

Ideal gas implies T6 equals TS times (p6 over p5) to the power of n minus one over n.

T6 equals 328.075 Kelvin.

Disc:

h'e plus w2 squared over 2 equals h'a plus w2 squared over 2.

h5 plus w5 squared over 2 equals h6 plus w6 squared over 2.

w6 equals the square root of 2 times (h5 minus h6) plus w3 squared.

w6 equals 507.24 meters per second.

h5 minus h6 equals the integral from T6 to T5 of cp dT.

h5 minus h6 equals cp times (T5 minus T6) equals 104647.95 Joules per kilogram.

c) ex,sta equals (h6 minus h0 minus T0 times (s6 minus s0) plus w6 squared over 2).

h6 equals cp times T6 equals 330063.95 Joules per kilogram.

h0 equals cp times T0 equals 296608.8 Joules per kilogram.

cv equals cp minus R over M.

R equals R over m equals 288.982 Joules per kilogram Kelvin.

cv equals 715.01 Joules per kilogram Kelvin.

s6 minus s0 equals the integral from T0 to T6 of cp over T dT minus R times ln(p6 over p0) equals cp times (ln(T6) minus ln(T0)) minus R times ln(p6 over p0).

s6 minus s0 equals 301.361 Joules per kilogram Kelvin.

ex,sta equals 156239.168 Joules per kilogram.

Delta e sub x, sub str, equals one hundred t sub three divided by t sub zero (according to specification)

e sub x, sub von, equals T sub zero sen z