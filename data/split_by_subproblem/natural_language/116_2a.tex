a) T-s diagram

Diagram:
- Vertical axis labeled as T in Kelvin [K].
- Points labeled 0, 1, 2, 3, 4, 5, 6. 
- Point 1.5 is marked between points 1 and 2.
- Line segments connecting points, with annotations such as "p2 = p" and "0.5 bar".
- Arrow indicating direction from point 0 to point 6.
- Annotation near point 5: "nach auf 0.5 bar".

Table:
- Columns labeled as z = 37, P, T [K], s, h, s [kJ/kgK].
- Rows labeled from 1 to 6, with an additional row labeled 0.
- Row 1: P = 0.5 bar.
- Row 2: s = s2.
- Row 4: P = 0.5 bar.
- Row 5: P = 0.5 bar, T = 43.9.
- Row 6: P = 0.19 bar, T = 328.075, s = s5.
- Row 0: P = 0.19 bar.

Equation:
- Ideal gas: Isentrope K = 1.4
- T6 = T5 times (P6/P5) raised to the power of (0.4/1.4).

c) O equals h6 minus h0 minus T0 times (s6 minus s0) plus Ahe

d) exvel equals delta exstr equals 100 kilojoules per kilogram