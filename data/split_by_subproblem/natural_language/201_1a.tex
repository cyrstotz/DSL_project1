a) GS: Q aus

=> O = m dot (he - hn) + Q dot e = Q dot aus

he(70°C, 1 bar) = hg(70°C) - 233.5 kJ/kg h f(70°C) = 292.98 kJ/kg  
hn(100°C, 1 bar) = hg(100°C) - 225.5 kJ/kg h f(100°C) = 419.04 kJ/kg  

(=> an offenes System -> atm Druck 1 bar  
siedende Flüssigkeit -> Flüssigkeitsvolumenzustand  
siedend -> Flüssig)

=> Q dot aus = m dot (hr - hn) + Q dot R = 76.96 kW 62.78 kW

b) GS: T KF, Q aus = 65 kW

T KF = (T KF aus + T KF ein) / 2 = 293.15 K  
arithmetisch

c) ΔS = Q aus / T KF + S erz

<=> S erz = ΔS - Q aus / T KF

JS minus GS over JS minus GS equals LX  
JS equals LS  
LS equals GS  
LX equals SY  

J, ZZ minus YI prime over Y divided by R4R minus VR4R times W (P)