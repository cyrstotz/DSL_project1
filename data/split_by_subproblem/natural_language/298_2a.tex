A graph is drawn with the vertical axis labeled as T in Kelvin and the horizontal axis labeled as S in kilojoules per kilogram Kelvin. The graph includes a line starting from point 1, moving through points 2, 3, 4, 5, 6, and ending at point O. 

Annotations on the graph include:
- Between points 1 and 2, it says "HD Seite wird reversible".
- Between points 3 and 4, it says "irreversible Entspannung".
- Between points 4 and 5, it says "reversible verlustlose Entsp.".
- Between points 5 and 6, it says "P6 = P0 - T6 = T0".

Below the graph, there are numbered steps:

1. T0, P0: adiabatic, irreversible, work less than L, compressed.
2. P1 greater than P0: mass flow rate m dot N, work in the mixture to the auxiliary mass flow rate m dot K, high pressure, adiabatic process, to P2, T2.
3. 2 to 3: steam chamber, work network.
4. 3 to 4: adiabatic, irreversible expansion, N dot L.
5. After mixing chamber: T5, W5, P5.
6. 6: reversible, adiabatic expansion to P0, with T0.

Q ex, vel = Ex, str, vel - O - Sum Ex, str + Sum Ex, Q - Sum E x W - Ex, vel

Q ex, vel = Ex, str, vel divided by m = Sum Ex, str divided by m

O - Sum Ex, str + Ex, Q - W t - Ex, vel

Q ex, vel = Ex, vel divided by k is q vel = Delta Ex, str + (Lambda minus 10 divided by T B) q B

Q ex, vel - Delta Ex, str + (1 minus 10 divided by T B) q B = 100 kJ divided by kg (1 minus 243.15 kJ divided by 1293 kJ) minus 145 kJ divided by kg

= 106.7 kJ divided by kg

P sub s, l, Masse m sub g

Diagram: A horizontal rectangle labeled with m sub k times g plus m sub w times g pointing downwards and P sub amb pointing right. An arrow pointing upwards labeled Fixe Punkt A.

Kraftgesetz am Kolben

1) P sub l times A equals P sub s, l

2) A equals pi times D squared divided by four equals pi times open parenthesis D divided by two close parenthesis squared equals pi times D squared divided by four

3) P sub s, l equals g times open parenthesis m sub k plus m sub w close parenthesis divided by pi times D squared divided by four plus rho sub amb b equals 

P sub A times V sub A equals m sub A times R times T sub A equals m sub r times g equals P sub r times V sub r divided by R times T sub r

Vorher geschrieben

equals P sub s, l times g divided by V sub r equals R sub g times T sub r times V

As m sub w: P sub s equals 1.15 bar, m sub g equals 3.6 g

R equals R divided by M sub g equals 8.314 kilogram meter squared divided by kilogram times second squared times Kelvin divided by 50 kilogram divided by kilomole equals 0.166 kilojoule divided by kilogram times Kelvin