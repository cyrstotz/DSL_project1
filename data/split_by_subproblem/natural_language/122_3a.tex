| Tj | Vj | pj |
|----|----|----|
| 1  | 500°C | 374°C | 1.4008 bar |
| 2  |       |       | 1.4008 bar |

a) pj = (R * T1) / (mj * V1)

Rj = R / mj = 0.166285944

[Diagram with labels: 32 kg, EW, On]

pj = pEW + (F / A) = pEw + (Fk / A) + (FEW / A)

Fk = 373.813 N

A = pi * r squared = pi * d squared / 4 = 0.00785 m squared

FEW = mEW * g = 0.381 N

pj = 140080.8163 Pa = 1.4008 bar

mj = (pj * Vj) / (Rj * T1) = 3.42 g

pj / pj2 = pj1 = 1.4008 bar, Tj2 = TEW2 = 0°C

Zustand 2: keine Wärmezufuhrung mehr -> xEis > 0 -> TEW2 = 0°C

Da immernoch Eis im EW ist, ist die Temperatur immernoch = 0°C. Der Druck verändert sich nicht im Gas, da die Stoffe beständig bleiben, die Temperatur des Gases nimmt die Temperatur des Wassers an, da kein Wärmeaustausch im Zustand 2 vorliegt.