A graph is drawn with an axis labeled 'T [K]' vertically and 's [kJ/K]' horizontally. The graph shows a cycle with points labeled 1, 2, 3, 4, 5, and 6. Several lines are drawn: one labeled 'isobar' between points 3 and 4, another labeled 'isentrop' between points 2 and 3, and another labeled 'isentrop' between points 5 and 6. There are lines labeled 'p2', 'p3 = p4', 'p5', and 'p0 (isobare)'.

w subscript b squared equals two times (h subscript 0 minus h subscript 6) plus w subscript 0 squared

w subscript 0 equals w subscript luft

T subscript 0 equals 273.15 K

w subscript 6 equals square root of 2 times (h subscript 0 minus h subscript 6) plus w subscript 0 squared

w subscript b squared equals two times integral from T subscript b to T subscript 0 of c subscript p, luft dT plus w subscript 0 squared equals two times c subscript p, luft times (T subscript 0 minus T subscript b) plus w subscript 0 squared

p two equals m divided by k times g divided by (d divided by two) squared equals two point two five six bar.

TAB:
U fest (two bar) equals minus three hundred thirty-three point four nine five kilojoules per kilogram.
U fest (two point two five six bar) equals U fest (two bar) plus two point two five six bar minus two bar divided by ten bar minus two bar times (U fest (ten bar) minus U fest (two bar)) equals minus three hundred thirty-three point five four one kilojoules per kilogram.

TAB:
U flüssig (two bar) equals minus zero point zero seven nine kilojoules per kilogram.
U flüssig (ten bar) equals minus zero point three three two kilojoules per kilogram.

U flüssig (two point two five six bar) equals U flüssig (two bar) plus two point two five six bar minus two bar divided by ten bar minus two bar times (U flüssig (ten bar) minus U flüssig (two bar)).