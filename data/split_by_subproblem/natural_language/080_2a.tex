A graph is drawn with axes labeled as T [K] on the vertical axis and S [kJ/kg-K] on the horizontal axis. The graph contains the following annotations:
- Point labeled (2), (3), (L2), (L1), and (c).
- Curves labeled as "isobare 2" and "isobare 1 = p5" and "isobare 0 = p0".
- A shaded area is marked with a zigzag pattern.

e x, ucel: Stationäre Fließprozesse → Exergie Bilanz um Teichenhülle.

O equals m dot times e x, str plus sum of (1 minus T zero over T) times q dot equals e x, ucel. e x, Q equals (1 minus 248.15 Kelvin over 1289 Kelvin) times 1115 equals 169.582 kilojoules per kilogram. m dot equals e x, a equals 967.582 kilojoules per kilogram.

Delta e x, ucel equals Delta e x, str plus e x, Q. e x, ucel equals minus Delta e x, str plus e x, Q equals 969.582 kilojoules per kilogram minus 60.21 kilojoules per kilogram equals 909.37 kilojoules per kilogram.

E sub k equals the absolute value of Q sub x divided by W sub total equals six divided by twenty-eight equals zero point two one four.