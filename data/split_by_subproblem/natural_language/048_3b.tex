P subscript g,2 equals P subscript amb greater than 1 bar due to thermodynamic equilibrium. The gas does not expand further.

P V equals n R T (crossed out section)

T subscript g,2 divided by T subscript g,1 equals open parenthesis P subscript g,2 divided by P subscript g,1 close parenthesis to the power of open parenthesis n minus 1 close parenthesis divided by n.

K equals C subscript p divided by C subscript v

C subscript p equals R plus c v equals 0.17 joules per kilogram kelvin plus 0.633 kilojoules per kilogram kelvin equals 0.80 kilojoules per kilogram kelvin. Therefore, k equals 0.80 kilojoules per kilogram kelvin divided by 0.633 kilojoules per kilogram kelvin equals 1.26.

T subscript g,2 equals T subscript g,1 open parenthesis P subscript g,2 divided by P subscript g,1 close parenthesis to the power of open parenthesis n minus 1 close parenthesis divided by n equals 446.13 degrees Celsius.