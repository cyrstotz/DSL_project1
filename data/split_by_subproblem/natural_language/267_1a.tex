a) E = 3 L: Ein - Aus (sLabl, isotherm, isochor)

0 = m rein (h ein - h aus) + Q aus + Q R

h ein = h f (70 degrees Celsius) equals 292.98 kilojoules per kilogram

h aus = h f (20 degrees Celsius) equals 419.04 kilojoules per kilogram

=> Q aus = 62.782 kilowatts

b) S = 3 L:

h f over T aus minus h f over T ein = c ln (T aus over T ein) + v (p aus over p ein)

=> 293.72 Kelvin

c) S = 3 L: Reaktionswand (stationär, kein Massenstrom)

delta S rev = Q aus over T aus = delta S equals 272.74 watts per Kelvin

d) E = 3 Lanz - 2 (offen, adiabatic da Q rev + Q aus = 0, isochor)

U = u2 (m2 + a m12) equals u1 m1 = a m12 h ein12

=> a m12 = u m2 - u m1 over h ein12 - u2 equals 3756.84 kilograms

u2 = v f (70 degrees Celsius) + x2 (v g (70 degrees Celsius) - v f (70 degrees Celsius)) equals 429.3778 kilojoules per kilogram

h ein12 = h f (20 degrees Celsius) equals 83.96 kilojoules per kilogram

e) 0 = S12 minus m2 s2 minus m1 s1 = m s1 = s2 (m2 + a m12) equals -1387.6 joules per Kelvin

s1 = s f + x0 (s g - s f) equals 7.33 724 joules per kilogram Kelvin

s2 = s f (70 degrees Celsius) equals 0.9549 joules per kilogram Kelvin

=> 0 = S12 minus m2 s2 minus m1 s1 = -1387.6 joules per Kelvin