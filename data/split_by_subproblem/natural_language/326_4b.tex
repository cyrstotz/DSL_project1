A series of calculations are shown:

- T1 equals negative 10 degrees Celsius.
- T2 equals negative 16 degrees Celsius.
- S2 equals S3 equals 0.8288 kilojoules per kilogram Kelvin.
- h2 equals h of negative 16 degrees Celsius equals 237.79 kilojoules per kilogram.
- A formula is given: m dot R equals m dot K times (h2 minus h3).
- A calculation for x3: x3 equals (S2 minus Sf at 8 bar) divided by (Sg minus Sf at 8 bar).
- A numerical result is given: 1.604... with a note 'nicht möglich' (not possible).
- A final formula: m dot R equals W dot K divided by (h2 minus h3).