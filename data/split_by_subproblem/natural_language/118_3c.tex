Delta E equals the sum of Q twelve minus the sum of W v twelve.

W equals the integral from one to two of p dV equals p gas times (U two minus U one).

U two equals mRT two divided by p equals three point forty-two times ten to the power of three times R times two hundred seventy-three point one five three divided by one point four times ten to the power of five equals zero point zero one one zero cubic meters approximately equals one point one one liters.

Therefore, W equals one point one one times ten to the power of five times (one point one one times ten to the power of negative three plus three point eleven times ten to the power of negative two) equals five hundred ninety-four point nine four three.

Delta E equals C v times (T two minus T one).

Equals zero point six three three kilojoules per kilogram Kelvin times (two hundred seventy-three point one five three minus seven hundred seventy-three point one five).

Equals negative three hundred ninety-six kilojoules per kilogram. Therefore, Delta E times m equals.

Therefore, Q twelve equals Delta E plus W.