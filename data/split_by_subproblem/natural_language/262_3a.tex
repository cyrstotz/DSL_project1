Pg1 Zustand 1
mg im Zylinder

Pg1:
[Diagram] A cylindrical container labeled "Pg1 Gas" with arrows indicating "mg", "pamb", and "mew g".

Pg1 = mew g over A plus mg over A plus pamb

Zylinder:
A = pi r squared = pi times (D over 2) squared
= pi times (0.11 over 2) squared
A = 0.0028549 square meters

Pg1 = 170094.99 pascals
= 11.407 bar

mg:
rho V = m R T
rho V1 over R T1 = mg = 0.003921 kilograms
= 3.921 grams

V1 = 3.4 liters
= 3.4 times 10 to the power of negative 3 cubic meters

T1 = 500 degrees Celsius equals 773.15 kelvin

Cv = 0.637 kilojoules per kilogram kelvin
Mg = 50 kilograms per kilomole

P = F over A
g = 9.81

R = R over Mg = 8.314 joules per mole kelvin over 50 kilograms per kilomole
0.16628 cubic meters per kilogram kelvin equals R

x sub E S 1, 2 is greater than 0

Since state 2 represents an equilibrium state and is still present as the final state, T g 12 must have the same temperature as T 1 5 as T equals 0 degrees Celsius. Since the piston is given as isolated, it neither absorbs nor releases temperature.

T g 12 equals 0 degrees Celsius

X sub Eis, 2

X sub Eis, 1 equals 0.6

From above, given Q ab, we take 1500 S from task

Energy:

Delta U sub 2 1 equals m i f times f plus Q 1 2 minus W sub n

Released heat: Q 1 2 equals minus Q 1 2

Delta U sub 2 1 equals minus Q 1 2

m sub E W times (u 2 minus u 1) equals minus Q 1 2

u 2 equals minus Q 1 2 over m sub E W plus u 1

u 1 equals u fest plus X times (u flüssig minus u fest)

u 1 equals minus 13.9402 kilojoules per kilogram

u 2 equals minus 1500 times 10 to the power of 3 over m sub E W plus u 1 equals minus 28.9402 kilojoules per kilogram

Values from the given table, x equals 0.6

p equals 1.76 bar

Interpolate with the given table

p 2 equals p 1

X sub Eis, 2 equals 0.085

X sub Eis, 2 equals 1 minus 0

X sub Eis, 2 equals minus 333.858 minus (-0.085)

(-28.9402 minus (-0.085)) plus 1.0085