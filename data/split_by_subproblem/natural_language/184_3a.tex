M sub G equals fifty kilograms per cubic meter.

C sub v gas equals zero point six three three kilojoules per kilogram Kelvin.

m equals E V - konstant.

p sub G one, m sub G?

p V equals m R T

Kritische Gleichgewicht: 

p equals P sub G one (zero point one divided by two) squared equals zero point one times g plus ten to the power of five times (zero point one squared divided by two) (pi times r squared plus thirty-two times g)

p sub G one equals zero point one times g plus ten to the power of five (zero point one squared divided by two) (pi times r squared plus thirty-two times g divided by zero point one squared divided by two times pi) equals one point four zero four bar

g equals nine point eighty-one meters per second squared

m sub G equals p V divided by R T equals (one point four zero four times ten to the power of five) times (three point one four times zero point five times ten to the power of negative three divided by (three hundred plus two hundred seventy-three point fifteen)) times (five point five times ten to the power of negative three divided by fifty times ten to the power of negative three) equals three point four two two grams