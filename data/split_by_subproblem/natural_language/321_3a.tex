P sub s1 = P sub ew + P sub anb + P sub u

P sub ew = m sub ew multiplied by g divided by pi times (0.05 squared)

equals 0.2 kilograms times 9.81 meters per second squared divided by pi times 0.0025 meters squared equals 124,405 pascals

P sub anb = 1 bar equals 100,000 pascals

P sub u = m sub u multiplied by g divided by pi times (0.05 squared) equals 39,984.5 pascals

P sub s1 = 140,004 pascals equals 1.4 bar

P sub s1 multiplied by V sub s1 equals m sub s1 multiplied by R times T sub s1

T sub s1 = 500 degrees Celsius equals 773.15 Kelvin

R equals R divided by M sub g equals 8.314 joules per mole Kelvin divided by 50 grams per mole equals 166.28 joules per kilogram Kelvin

m sub g equals P sub s1 multiplied by V sub s1 divided by R times T sub s1 equals 140,004 pascals times 3.14 times 10 to the power of negative 3 cubic meters divided by 166.28 joules per kilogram Kelvin times 773.15 Kelvin equals 3.4217 times 10 to the power of negative 3 kilograms equals 3.4217 grams