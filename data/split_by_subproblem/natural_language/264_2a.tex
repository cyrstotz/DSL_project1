A diagram is drawn with labeled points connected by lines. The points are labeled as follows:
1. 'isotherm verdichter' pointing downwards to point 1.
2. Point 2 is connected to point 3 with a line labeled 'isobare wärmezufuhr'.
3. Point 3 is connected to point 4 with a line labeled 'adiabat, irreversibler Turbine'.
4. Point 4 is connected to point 5 with a line labeled 'isobare Mischkammer'.
5. Point 5 is connected to point 6 with a line labeled 'isotherme schieber'.
The x-axis is labeled 's [kJ/kgK]' and the y-axis is labeled 'T'.

T sub b equals 431.4

(191,100 pascal, 50,000 pascal) raised to the power of (1.4 minus 1) over 1.4

equals 328.67469 Kelvin

equals 328.1 Kelvin

Interpolation h sub 6 equals 330 minus 325

equals 328.4031

equals 328.40 kilojoules per kilogram

w sub o equals w sub s equals square root of h sub s over t sub h sub c

equals 220 meters per second

equals 194.2051 meters per second

w sub 6 equals 194.2 meters per second

R equals the square root of R over M Luft.

S six minus S zero equals S zero six of b minus S zero six of zero minus R times the natural logarithm of P two over P one.

Interpolation:

S zero six equals 1.54191 minus 1.47824 over 250 minus 240 times (243.75 minus 240) plus 1.47824.

Equals 1.49118249.

Equals 1.49113.