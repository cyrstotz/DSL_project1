The average temperature T-bar is given by the ratio of the integral of temperature T with respect to S from S_a to S_e, over the integral of dS from S_a to S_e, which equals T_dS, according to the ideal gas theory. This implies that C times the natural logarithm of the ratio of the temperature at the outlet, T_K,out, to the temperature at the inlet, T_K,in, equals the isobaric condition, which is the difference between the enthalpy at the outlet, h_K,out, and the enthalpy at the inlet, h_K,in.

This is further simplified to the ratio of the difference in temperatures at the outlet and inlet over the natural logarithm of the ratio of these temperatures, which equals the difference between 298.15 Kelvin and 288.15 Kelvin divided by the natural logarithm of the ratio of 298.15 Kelvin to 288.15 Kelvin, approximately equal to 293.1 Kelvin.