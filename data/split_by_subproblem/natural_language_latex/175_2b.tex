Zero equals the mass flow rate times the quantity of enthalpy at point zero minus enthalpy at point six plus half the difference of the square of the exit velocity and the square of the approach velocity, plus the ratio of pressure at point zero to density at point zero, plus the heat flow rate minus the work done by the turbine.

- Adiabatic to the outside implies that the heat flow rate equals zero.

The work done by the turbine equals the mass times the quantity of enthalpy at point five minus enthalpy at point six plus half the difference of the square of the exit velocity and the square of the approach velocity.

The ratio of the work done by the turbine per mass plus the difference of enthalpy at point six minus enthalpy at point five, all multiplied by two, equals the difference of the square of the exit velocity and the square of the approach velocity.

The square of the velocity at point six equals the square of the velocity at point five minus the ratio of the work done by the turbine to the mass flow rate plus the difference of enthalpy at point five minus enthalpy at point six.

The velocity at point five equals the mass flow rate times the gas constant times the temperature at point five divided by the pressure at point five, which equals 0.1286 Joules per gram Kelvin times 431.13 Kelvin divided by 0.5 times 10 to the power of 5 Pascals, equals 0.10024789 cubic meters per gram.

The ratio of temperature at point six to temperature at point five equals the ratio of pressure at point six to pressure at point five raised to the power of (kappa minus one) divided by kappa, which implies that temperature at point six equals temperature at point five times the ratio of pressure at point six to pressure at point five raised to the power of four divided by 4.4, equals 328.107 Kelvin.

The difference of enthalpy at point six minus enthalpy at point five equals the specific heat at constant pressure times the difference of temperature at point six minus temperature at point five.

The difference of enthalpy at point five minus enthalpy at point four equals the specific heat at constant pressure times the difference of temperature at point five minus temperature at point six, equals 1006 kilojoules per kilogram Kelvin times the difference of 431.13 Kelvin minus 328.107 Kelvin, equals 104.143 kilojoules per kilogram.

The square of angular velocity at point six equals the square of angular velocity at point five minus the ratio of the work done by the engine to the mass flow rate plus the difference of enthalpy at point five minus enthalpy at point six.

The work done by the engine equals the gas constant times the difference of temperature at point six minus temperature at point five divided by one minus 1.4 times four, equals 0.12367 Joules per gram times the difference of 328.07 Kelvin minus 431.13 Kelvin divided by 0.4, equals negative 741.42 Joules per gram.

The angular velocity at point six equals the square root of the square of 200 meters per second plus the ratio of 741.42 Joules per gram to the total mass flow rate plus 104.45 square meters per second squared.