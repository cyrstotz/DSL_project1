T1 equals T0 minus 6 Kelvin.
T0 equals negative 10 degrees Celsius, which is 263.15 Kelvin.
T1 equals 257.15 Kelvin, which is negative 16 degrees Celsius.
From Table A-10:
The enthalpy hf at T equals negative 16 degrees Celsius is 293.30 kilojoules per kilogram.
h1 equals h4 equals 293.30 kilojoules per kilogram (throttling isenthalpic process).
P3 equals P4 equals 8 bar.
x4 equals 0 from Table A-11.
h4 equals hg at T4 equals 8 bar plus 263.15 kilojoules per kilogram.

The ratio of P1 over P4 equals the ratio of T1 over T4 raised to the power of n over n minus 1.

First law of thermodynamics at the compressor:
The mass flow rate times the difference in enthalpy h2 minus h3 equals the heat transfer rate minus the work rate by the compressor.
The work rate by the compressor equals the mass flow rate times the difference in enthalpy h2 minus h3.
The mass flow rate times the gas constant R times the change in entropy Delta s4a equals the ratio of the work by the compressor over the difference in enthalpy h2 minus h3.