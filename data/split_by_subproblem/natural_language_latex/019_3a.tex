The section is looking for \( P_{1a} \) and \( m_g \).

Given: \( T_{g1} = 500^\circ C = 773.15\,K \)

\( V_{g1} = 3\,l\,kg = 0.003\,m^3 \)

There are two tables presented:
- The first table contains \( P_0 \) with units \( \frac{mkg}{m^2s^2} \).
- The second table contains \( P_1 \).

The equation provided is:
\[ P_1 \cdot A = P_0 \cdot A + mkg \cdot \frac{m}{s^2} + \frac{m}{s^2} \]

This simplifies to:
\[ P_1 = P_0 + \frac{mkg \cdot \frac{m}{s^2}}{A} + \frac{\frac{m}{s^2}}{A} \]

The area \( A \) is given by:
\[ A = \frac{D^2}{4\pi} \]

Substituting the values, the equation for \( P_1 \) becomes:
\[ P_1 = 100000\,\frac{N}{m^2} \quad 1\,bar + \frac{32\,kg \cdot 9.81\,\frac{m}{s^2}}{\frac{D^2}{4\pi}} + \frac{0.1\,kg \cdot 9.81\,\frac{m}{s^2}}{\frac{D^2}{4\pi}} \]

This results in:
\[ P_1 = 5.01\,bar \]

The mass \( m_g \) is calculated using the formula:
\[ m_g = \frac{P \cdot V}{R \cdot T} = 0.255\,kg \]