The rate of heat output is denoted by Q dot subscript "aus".
The mass flow rate of water is 0.5 kilograms per second.
The rate of heat output, Q dot subscript "aus", equals the mass flow rate of water, m dot subscript "w", times the difference between the enthalpy at the outlet, h subscript "aus", and the enthalpy at the inlet, h subscript "ein".
A minus 2.
The enthalpy at state 1, h1, equals the enthalpy of fluid at 80 degrees Celsius, which is 292.88 kilojoules per kilogram, and this is equal to the enthalpy at the inlet, h subscript "ein".
The enthalpy at state 2, h2, is calculated as the enthalpy of fluid plus x times the difference between the enthalpy of gas and the enthalpy of fluid, which equals 430.23 kilojoules per kilogram.
A minus 2 at 100 degrees Celsius.
The enthalpy at the outlet, h subscript "aus", equals the enthalpy of fluid at 100 degrees Celsius, which is 418.94 kilojoules per kilogram.
The rate of heat output, Q dot subscript "aus", equals the negative of Q dot subscript "R" plus the mass flow rate of water, m dot subscript "w", times the difference between the enthalpy at the outlet, h subscript "aus", and the enthalpy at the inlet, h subscript "ein", which results in negative 62.184 kilojoules per second.