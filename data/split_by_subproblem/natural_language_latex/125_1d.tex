The heat flow rate into component c, denoted as Q dot subscript zu,c, equals the heat flow rate at 100 degrees Celsius for component c, represented by Q dot subscript 100,c. This is illustrated by a graph with a wavy line.

The product of the total mass flow rate at state 1, specific heat capacity, and the temperature at state 1, denoted as m dot subscript ges,1 times c times T subscript R,1, equals the product of the total mass flow rate at state 2, specific heat capacity, and the temperature at state 2, denoted as m dot subscript ges,2 times c times T subscript R,2.

This equation is repeated and illustrated by a graph with a wavy line.

The product of the total mass flow rate at state 1, specific heat capacity, and the temperature at state 1 equals 378.57 kilograms per second times the sum of 100 and 273.15 degrees Celsius times the specific heat capacity c. This is noted to refer to the lecture.

Continuing with the problem:

The change in mass from state 1 to state 2, denoted as Delta m subscript 12, equals the product of a factor f and the sum of 70 and 773.75 kilograms per k, plus the product of the mass of gas at state 1, factor f, and the ratio of the sum of 1000 and 773.75 kilograms per k.

This is further simplified to the sum of Delta m subscript 12 and the mass of gas at state 1, multiplied by the factor f and the ratio of the sum of 70 and 773.75 kilograms per k, plus the heat flow rate at state 12 minus the heat flow rate of the gas at state 12.

Finally, Delta m subscript 12 is calculated as the difference between the product of the mass of gas at state 1 and the sum of 70 and 773.75 kilograms per k, minus the sum of 1000 and 773.75 kilograms per k, all divided by the difference in the sums of 70 and 773.75 kilograms per k. The result is boxed and equals 3453 kilograms.