The mass flow rate, denoted as m-dot subscript x, r, t, equals m times the expression in brackets: h subscript 6 minus h subscript 0 plus half of the difference between s subscript 6 and s subscript 0, plus h subscript 6 times e.

h subscript 1 minus h subscript 6.

c equals c subscript p.

h subscript 0 minus h subscript 8.

c subscript 0 equals c subscript v.

T subscript 0 equals T subscript 8.

h subscript 6 minus h subscript 0 equals h subscript 6 minus h subscript 0 equals c subscript p times the difference between T subscript 6 and T subscript 0.

s subscript f equals c subscript p times the natural logarithm of the ratio of T subscript 6 to T subscript 0 minus R times the natural logarithm of the ratio of p subscript 1 to p subscript 0.

R equals c subscript p minus c subscript v equals c subscript p minus c subscript p times x equals 0.287.

c subscript v equals the reciprocal of gamma times c subscript p.

The expression for kinetic energy change, one half times the difference between w subscript 6 squared and w subscript 0 squared, equals 1.1 times 10 to the power of 4 Joules.