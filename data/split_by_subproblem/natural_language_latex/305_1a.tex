The change in energy, Delta E, equals Q minus W, which implies that the change in internal energy, Delta U, equals the mass entering, m subscript ein, times the specific internal energy upon entering, u subscript ein, minus the mass exiting, m subscript aus, times the specific internal energy upon exiting, u subscript aus.

The rate of heat transfer Q dot subscript 2 plus the rate of heat transfer Q dot subscript aus equals the mass flow rate, m dot, times the difference between the specific internal energy upon entering, u subscript ein, and the specific internal energy upon exiting, u subscript aus.

The specific internal energy upon entering, u subscript ein, at 70 degrees Celsius is 292.95 kilojoules per kilogram, labeled as A2.

The specific internal energy upon exiting, u subscript aus, at 100 degrees Celsius is 418.94 kilojoules per kilogram, labeled as A2.

The rate of heat transfer upon exiting, Q dot subscript aus, is calculated as 0.3 kilograms per second times the difference between 292.95 kilojoules per kilogram and 418.94 kilojoules per kilogram, plus 100 kilowatts.

This results in 62.203 kilowatts, which implies that the rate of heat transfer upon exiting, Q dot subscript aus, is negative 62.203 kilowatts.