The change in internal energy is represented by dU equals Q minus W.

The change in internal energy from state 1 to state 2 is equal to the specific heat at constant volume times the change in temperature, expressed as u2 minus u1 equals cv times (T2 minus T1).

The change in internal energy, dU, is given as 316.5 kilojoules per kilogram.

The heat added, Q, is equal to the change in internal energy, dU, plus the work done, W.

The work done per unit mass is given by the integral of pressure with respect to volume, expressed as W over m equals the integral of p dv.

The work done per unit mass from volume V1 to V2 is equal to negative the integral from V1 to V2 of p dv, which simplifies to negative p1 times the natural logarithm of (V2 over V1), and is calculated to be 0.972 kilojoules.

The volume at state 2, V2, is calculated using the formula V2 equals (m times R times T2) divided by pG2, resulting in 1.11 liters.

The total heat added, Q, is equal to the mass times the change in internal energy plus the work done, which totals 2052 Joules.

The product of the amount of substance and the change in internal energy, n times dU, is 1.08 kilojoules.