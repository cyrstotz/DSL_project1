The mass flow rate at the exit, denoted as m-dot subscript ext, equals m-dot times the expression in brackets: h subscript 6 minus h subscript 0 minus T subscript 0 times (s subscript 6 minus s subscript 0) plus kinetic energy plus potential energy.

The mass flow rate at the exit, m-dot subscript ext, equals the total mass flow rate, m-dot subscript ges, times the expression in brackets: c subscript p times (T subscript 6 minus T subscript 0) minus T subscript 0 times (c subscript p times the natural logarithm of T subscript 6 over T subscript 0 minus R times the natural logarithm of p subscript 6 over p subscript 0) plus the difference of the squares of 200 meters per second and 0.005 meters per second, all divided by 2.

The mass flow rate at the exit, m-dot subscript ext, equals the total mass flow rate, m-dot subscript ges, times the expression in brackets: 1.006 times (340 Kelvin minus 243.15 Kelvin) minus 243.15 times (1.006 times the natural logarithm of 340 over 243.15) minus (the expression is incomplete).

The gas constant R equals c subscript p minus c subscript v.

The polytropic index n equals kappa, which is the ratio of c subscript p to c subscript v, leading to the expression c subscript v equals c subscript p divided by kappa.