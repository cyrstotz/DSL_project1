The energy balance for SFP is given by the equation zero equals the product of the total mass flow rate and the change in specific energy across the threshold, denoted as delta e subscript x, thr.

The change in specific energy across the threshold, delta e subscript x, thr, is given by the expression inside the brackets: enthalpy at state 7 minus enthalpy at state 6 minus the product of the reference temperature T subscript 0 and the difference in entropy between state 7 and state 6, plus half the difference of the squares of velocity at state 7 and state 6.

The difference in enthalpy between state 7 and state 6 is equal to the product of the specific heat at constant pressure for water vapor, c subscript p superscript w, and the difference in temperature between state 7 and state 6.

The difference in entropy between state 7 and state 6 is given by the product of the specific heat at constant pressure, c subscript p, times the natural logarithm of the ratio of temperature at state 7 to temperature at state 6, minus the gas constant R times the natural logarithm of the ratio of pressure at state 7 to pressure at state 6.

A table is provided with various states and properties:
- At state 0, the pressure is 9797 Pascal per square meter, and the temperature is minus 20 degrees Celsius.
- State 1 is adiabatic.
- State 2 is marked as mechanically adiabatic and reversible.
- State 3 has a temperature of 298 Kelvin, is isobaric, and the heat transfer per unit mass is 1705 kilojoules per kilogram.
- State 4 is adiabatic and non-isotropic but isobaric.
- At state 5, the pressure is 0.56, the temperature is 4319, and the volume flow rate is 220 cubic meters per second.
- State 6 is isotropic.