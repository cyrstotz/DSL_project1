The change in heat, denoted as Delta Q, is equal to the mass of the substance, denoted as m subscript g, multiplied by the specific heat at constant pressure, denoted as C subscript p, multiplied by the change in temperature, denoted as Delta T. This is expressed as:

Delta Q equals m subscript g times C subscript p times Delta T.

Substituting the values, we have:

Delta Q equals 3.6 times 10 to the power of negative 3 times 0.8 times (T subscript 2 minus T subscript 1).

Further substituting the temperatures, we get:

Delta Q equals 3.6 times 10 to the power of negative 3 times 0.8 times (0.003 degrees Celsius minus 500 degrees Celsius).

This calculation results in:

Delta Q equals negative 1.4393 Joules.

For the specific heat at constant pressure, C subscript p, it is given by the formula:

C subscript p equals R over M plus C subscript v,

where R is the gas constant, M is the molar mass, and C subscript v is the specific heat at constant volume. Substituting the values, we have:

C subscript p equals 8.314 kilojoules over (50 kilograms per kilomole times Kelvin) plus 0.633.

This simplifies to:

C subscript p equals approximately 0.8 kilojoules per kilogram per Kelvin.