In the given text:

1. The values of \( w_6 \) and \( T_6 \) are questioned.
2. Under the isentropic condition, \( n \) equals \( k \) and \( P_6 \) equals \( P_0 \) which is 0.797 bar.
3. The ratio \( \frac{T_6}{T_5} \) is equal to \( \left( \frac{P_6}{P_5} \right)^{\frac{k-1}{k}} \). Substituting values, \( T_6 \) is calculated as 328.075 Kelvin using the formula \( T_6 = 437.9 \text{ K} \left( \frac{0.797 \text{ bar}}{6.5 \text{ bar}} \right)^{\frac{7.4-1}{7.4}} \).
4. In the energy balance section, the change in energy \( \Delta E \) which includes internal energy change \( \Delta u \), kinetic energy change \( \Delta KE \), and potential energy change \( \Delta PE \), sums up to zero. The equation simplifies to \( 0 = m \left( u_5 + \frac{T_6}{T_5} \right) - u_5 \left( \frac{T_7}{T_5} \right) + \frac{m}{2} \left( \frac{w_6^2 - w_5^2}{2} \right) \).
5. The specific heat at constant volume \( c_v \) is given by \( c_v = c_p - R \) and \( \frac{c_p}{k} = c_v \) where \( c_v \) is 0.718 kilojoules per kilogram Kelvin.
6. The equation \( 0 = 0.718 \frac{kJ}{kg \cdot K} \left( 328.075 \text{ K} - 437.9 \text{ K} \right) + \frac{1}{2} \left( w_6^2 - 220^2 \frac{m^2}{s^2} \right) \) is used to find \( w_6 \).
7. Solving for \( w_6 \), it is found to be 219.66 meters per second.