Energy balance:

Steady state fixed point: Q equals the mass flow rate times the quantity h_c minus h_a plus half of the change in potential energy plus the sum of Q dot minus the sum of W. This leads to the mass flow rate being equal to 1 times w_31 divided by h_2 minus h_3, which equals negative 28 Watts divided by 227.9 minus 276.27, resulting in negative 0.579 times 10 to the power of negative 3, which equals 2039.4 kilograms per hour.

A-12 interpolation:

h_3 equals the fraction 0.9456 minus 0.9374 over 0.9711 minus 0.9374 times the difference between h at s equals 0.9711 and 8 bar minus h at 0.9374 and 8 bar plus h at 0.9374 and 8 bar, which equals 276.27.

h_2 equals h_f at T_1 equals negative 32 degrees Celsius minus 9.52, which equals h_g at T_1 equals negative 32 degrees Celsius minus 227.90. h_1 equals h_f at p equals 8 bar minus 204.15 kilograms per hour, which equals h_f at 8 bar.

s_4 equals s_1, which equals s_g at 8 bar equals 0.9066. For adiabatic and reversible processes, s_2 equals s_3. s_2 equals s_f at T_1 equals negative 32 degrees Celsius plus 0.0401, which equals s_g at T_1 equals negative 32 degrees Celsius equals 0.9456. h_3 equals s_3 at 8 bar times p_3 at 8 bar, which equals 0.9456.

For section A.11, subpart c):

h_2 equals h_f at Blaul equals 93.42. s_4 equals 0.3459 equals s_1. s_f at 3 bar. T_1 equals negative 22 degrees Celsius. x equals the fraction s_1 minus s_f over s_g minus s_f, which equals approximately 0.303.

T_1 equals negative 22 degrees Celsius. s_f at negative 22 degrees Celsius equals 0.0897. s_g at negative 22 degrees Celsius equals 0.9351.