The heat transfer rate for A, subscript s, is denoted as Q dot subscript A, s.

The product of mass flow rate, m dot, and enthalpy, h, equals the heat transfer rate for R, Q dot subscript R, which is equal to the heat transfer rate for A, subscript s, Q dot subscript A, s.

The heat transfer rate for A, subscript s, is equal to the ratio of area A2 to area A1, multiplied by the product of h, c, and the temperature at 100 degrees Celsius. This calculation simplifies to four halves times h c at 100 degrees Celsius, resulting in 614.04 kilojoules per second.

The heat transfer rate for A, subscript s, at 70 degrees Celsius is calculated similarly, resulting in 292.58 kilojoules per second.

The heat transfer rate for A, subscript s, is also calculated as the heat transfer rate for R, Q dot subscript R, minus the product of mass flow rate, m dot, and enthalpy, h, which equals 621.82 kilojoules per second.

The change in entropy from state 1 to state 2, Delta S subscript 1-2, is equal to the product of mass at state 2, m2, and entropy at state 2, s2, minus the product of mass at state 1, m1, and entropy at state 1, s1, resulting in 1237.58 kilojoules per Kelvin.

The mass at state 1, m1, is 0.755 kilograms per second.

The change in time, Delta t, is 1 second.

The entropy at state 1, s1, is calculated as 1.150 kilojoules per kilogram Kelvin plus x subscript p times the difference between 7.355 kilojoules per kilogram Kelvin and 1.306 kilojoules per kilogram Kelvin, resulting in 1.337 kilojoules per kilogram Kelvin.

The entropy at state 2, s2, at 20 degrees Celsius, s subscript f, is 0.851 kilojoules per kilogram Kelvin.

The mass at state 2, m2, is the sum of mass at state 1, m1, and 36 cooks per second, resulting in 37.355.