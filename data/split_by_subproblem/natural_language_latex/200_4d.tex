The energy efficiency \( E_k \) is defined as the ratio of the absolute value of the heat input rate \( \dot{Q}_{zu} \) to the absolute value of the electrical work rate \( \dot{W}_{el} \), which is also expressed as the ratio of the absolute value of the heat input rate \( \dot{Q}_{zu} \) to the absolute value of the difference between the heat output rate \( \dot{Q}_{ab} \) and the heat input rate \( \dot{Q}_{zu} \).

The heat input rate \( \dot{Q}_{zu} \) is given by the product of the mass flow rate \( \dot{m} \) and the difference in enthalpy \( h_2 - h_1 \).

The heat output rate \( \dot{Q}_{ab} \) is given by the product of the mass flow rate \( \dot{m} \) and the difference in enthalpy \( h_a - h_3 \).

The specific enthalpies are given as:
- \( h_1 = 93.42 \) kilojoules per kilogram,
- \( h_2 = 237.74 \) kilojoules per kilogram,
- \( h_3 = 277.37 \) kilojoules per kilogram,
- \( h_a \) is crossed out and was initially \( 93.92 \) kilojoules per kilogram.

Substituting these values, the energy efficiency \( E_k \) is calculated as the ratio of the absolute value of \( (h_2 - h_1) \) to the absolute value of \( (h_a - h_3) - (h_2 - h_1) \), which simplifies to \( \frac{144.32}{77.99 - 144.32} \).

The final value of the energy efficiency \( E_k \) is boxed and equals approximately 4.299.