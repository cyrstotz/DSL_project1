x subscript 2 equals 1, x subscript c equals 0, T subscript k equals T subscript i minus 6K, T subscript f equals T subscript TOP.

T subscript i equals T subscript TOP under sublimation pressure.

T subscript i at 100 millibar.

T subscript i at 5 millibar.

The boiling temperature T subscript Sied at 5 millibar equals negative 1 degree Celsius.

T subscript i equals 9 degrees Celsius equals 282.15 Kelvin.

Graph Description:

The graph is a pressure-temperature (P-T) phase diagram. The vertical axis is labeled P in millibar and the horizontal axis is labeled T in degrees Celsius.

- The graph shows three distinct regions labeled "solid", "liquid", and "gas".
- The "solid" region is at the top left, the "liquid" region is at the top right, and the "gas" region is at the bottom.
- There is a curve separating the "solid" and "gas" regions, which starts from the top left and curves downwards to the right.
- The point where the "solid", "liquid", and "gas" regions meet is labeled "triple point" at 0 degrees Celsius.
- A horizontal dashed line at approximately 5 mbar extends from the "triple point" to the right, indicating the sublimation pressure.
- The temperature at the triple point is marked as T subscript Triple equals 0 degrees Celsius.
- The graph also shows a horizontal line at T subscript i at 5 millibar, indicating the initial temperature at 5 millibar.
- Another horizontal line is drawn at T subscript i equals 9 degrees Celsius.