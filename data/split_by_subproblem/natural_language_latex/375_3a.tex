The pressure \( p_{s1} \) is equal to the ambient pressure \( p_{amb} \) plus the term \( \frac{m \cdot g}{A} \) plus the term \( \frac{m_{2v} \cdot g}{A} \), which simplifies to:
\( p_{s1} \) equals 100000 Pascals plus \( \frac{32 \text{ kilograms} \cdot 9.81 \text{ meters per second squared}}{0.007854 \text{ square meters}} \) plus \( \frac{0.1 \cdot 8.81 \text{ kilograms per square meter per second squared}}{0.007854 \text{ square meters}} \).
This results in \( p_{s1} \) equaling 140084 Pascals.

The mass \( m_2 \) is calculated by the formula \( m_2 = \frac{p_{s1} \cdot V_6}{R_6 \cdot T_6} \), which results in:
\( m_2 \) equals \( \frac{140084 \text{ Pascals} \cdot 0.00314 \text{ cubic meters}}{166.28 \text{ Joules per kilogram Kelvin} \cdot 773 \text{ Kelvin}} \).
This results in \( m_2 \) equaling 0.000003428 kilograms.

The gas constant \( R_6 \) is calculated by \( R_6 = \frac{R}{\mu_6} \), which simplifies to:
\( R_6 \) equals \( \frac{8.314 \text{ Joules per mole Kelvin}}{\frac{56 \text{ kilograms}}{1000 \text{ moles}}} \), resulting in \( R_6 \) equaling 166.28 Joules per kilogram Kelvin.

Lastly, 3.14 liters is converted to cubic meters as:
3.14 liters equals 0.00314 cubic meters.