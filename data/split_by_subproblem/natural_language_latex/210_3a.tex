The table has six columns labeled as follows: blank, p in bar, T in degrees Celsius, h in kilojoules per kilogram, s in kilojoules per kilogram Kelvin, and Other. There are three rows numbered 1, 2, and 3, with all cells empty except for the temperature of 500 degrees Celsius in row 1.

The volume Vg1 is given as 3.14 liters.

Ambient conditions are specified with a pressure of 1 bar and a mass mk of 32 kilograms.

For a perfect gas, the mass mg is calculated as the ratio of Vg to Vg1.

Vg1 is converted to cubic meters as 0.00314 cubic meters.

The pressure pg1 is calculated using force equilibrium, where pg1 times area A equals the sum of pamb times A, mk times 3.81 Newtons per kilogram, and mew times 9.81 Newtons per kilogram.

The area A is calculated using the formula for the area of a circle with diameter d, resulting in 0.0079 square meters.

The pressure pg1 is then calculated to be 1.3986 bar, rounded to 1.4 bar.

The mass mg is then calculated using the ideal gas law formula, resulting in 3.42 grams.

In the handwritten student solution, the problem statement involves finding Tg2 and pg2 with a condition on X12.

The solution involves calculating heat Q using mass m, specific heat at constant volume cv, and temperature change Delta T for the first law of thermodynamics applied to the gas side.

A table with a single cell labeled "Gas" is shown next to the word "Container."

The equation for Q12 involves mg, cv, and temperature T, with pressures p1 and p2 being equal.

The first law of thermodynamics is applied to a piston, relating mass flow rate m dot, internal energies U2 and U1, heat Q12, and work W12.

Work W12 is specified for an isochoric process (constant volume), where h equals zero, and using the perfect gas law, the change in temperature is calculated.

Finally, Q12 is calculated using mg, the difference in specific heats cp and cv, and the change in temperature T2 minus T1, plus internal energy U12. The result involves calculations with specific values for mass, specific heat, and temperature change, leading to a final expression for Q12.