Area A equals pi times the square of diameter D divided by 4, which equals 25 pi, which equals 78.54 square centimeters.
This equals pi times the square of diameter d1 divided by 4, which equals 0.00785 cubic meters.

The pressure P_a1g equals the amplitude pressure P_amp plus the mass m_k divided by area A times the acceleration due to gravity g plus the mass m_ew divided by g.
This equals 140078.09 Pascals.
This equals 1.4 bar.

The volume V_g1 equals the velocity v_g1 times the mass m_g.

Implies that the mass m equals the pressure P_g times the volume V_a divided by the gas constant R times the temperature T_a, which equals 140094 times V_a divided by R times T_a, which equals 0.003432 kilograms, which equals 3.4 grams.

The gas constant R equals the heat capacity at constant volume C_v times (K inverse minus lambda). Question mark. R divided by the molar mass M equals 8.314 Joules per mole Kelvin divided by 50 Joules per kilomole, which equals 166.28.

The volume V_a equals 3.14 times the length l, which equals 3.14 times 10 to the power of negative 3 cubic meters.

The temperature T_g1 equals 500 plus 273.15.