The heat transfer rate, Q, is equal to the mass flow rate entering, denoted as m-dot with subscript "ein" (entering), multiplied by the difference in enthalpy between entering and exiting, h with subscript "ein" (entering) minus h with subscript "aus" (exiting), plus the heat transfer rate Q-dot with subscript R, if and only if it equals the heat transfer rate exiting, Q with subscript "aus".

The heat transfer rate exiting, Q with subscript "aus", is equal to the mass flow rate entering, m-dot with subscript "ein", multiplied by the difference in enthalpy between exiting and another state, h with subscript "aus" minus h with subscript "out", plus the heat transfer rate Q-dot with subscript R.

This equals 0.3 kilograms per second times the sum of 473.04 kilojoules per kilogram and 292.39 kilojoules per kilogram, plus the heat transfer rate Q-dot with subscript R, which equals 62.782 kilowatts.

The enthalpy entering, h with subscript "ein", is 292.39 kilojoules per kilogram.

The enthalpy exiting, h with subscript "aus", is 473.04 kilojoules per kilogram.