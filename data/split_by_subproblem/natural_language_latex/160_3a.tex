The equation \( p_{g1} \cdot V_1 = m_g \cdot R \cdot T \) is given.

The gas constant \( R \) is calculated as \( R = \frac{\mathcal{R}}{M_g} \), where \( \mathcal{R} \) is 8.314 kilojoules per kilomole Kelvin and \( M_g \) is 50 kilograms per kilomole, resulting in \( R = 0.16628 \) kilojoules per kilogram Kelvin.

The initial volume of the gas \( V_{g1} \) is 3.14 liters.

The initial temperature of the gas \( T_{g1} \) is 500 degrees Celsius.

The mass of the evaporated water \( m_{ew} \) is approximately 0.1 kilograms.

The equation \( p_{g1} \cdot A = p_{amb} \cdot A + m_k \cdot g + m_{ew} \cdot g \) is presented in a table format.

The pressure \( p_{g1} \) is calculated as \( p_{g1} = p_{amb} + \frac{m_k \cdot g}{A} + \frac{m_{ew} \cdot g}{A} \).

Substituting the values, \( p_{g1} \) is calculated as 140.0344406 Newtons per square meter, which is approximately 1.4 bar.

The area \( A \) is calculated using the formula \( A = \frac{D^2}{4} \pi \), where \( D \) is 100 millimeters, resulting in \( A = 0.00785398 \) square meters.

The ideal gas law \( pV = mRT \) is stated.

Finally, the mass of the gas \( m_{g1} \) is calculated using the formula \( m_{g1} = \frac{p_{g1} \cdot V_{g1}}{R \cdot T_1} \), resulting in approximately 3.422 grams.