a) The gas constant \( R \) is calculated as \( R = \frac{\bar{R}}{M} \), where \( \bar{R} \) is given as \( 8.314 \frac{\text{kJ}}{\text{kmol K}} \) and \( M \) is \( 50 \frac{\text{kg}}{\text{kmol}} \). This results in \( R = 0.16628 \frac{\text{kJ}}{\text{kg K}} \).

The gas pressure \( p_{\text{gas}} \) is the sum of the ambient pressure \( p_{\text{amb}} \), the force due to the liquid \( \frac{F_L}{A} \), and the force due to evaporation \( \frac{F_{EV}}{A} \).

The ambient pressure \( p_{\text{amb}} \) is calculated as \( p_{\text{amb}} \) plus the pressure contributions from the mass of the liquid \( m_L \) and the mass due to evaporation \( m_{EV} \), both divided by the cross-sectional area of the pipe \( \left(\frac{\pi d^2}{4}\right) \). With \( m_L = 32 \text{kg} \), \( m_{EV} = 0.1 \text{kg} \), \( g = 9.81 \frac{\text{m}}{\text{s}^2} \), and \( d = 0.1 \text{m} \), the calculations yield \( p_{\text{amb}} = 1.4 \text{bar} \).

The relationship \( p \cdot V = m \cdot R \cdot T \) is used to find the number of moles \( n \) as \( n = \frac{p \cdot V}{R \cdot T} \).

The mass of the gas \( m_g \) is calculated using the ambient pressure \( p_{\text{amb}} \), the initial volume \( V_{0,1} \), the gas constant \( R \), and the initial temperature \( T_{0,1} \). With \( p_{\text{amb}} = 1.4 \cdot 10^5 \frac{\text{N}}{\text{m}^2} \), \( V_{0,1} = 3.14 \cdot 10^{-3} \text{m}^3 \), \( R = 0.16628 \frac{\text{kJ}}{\text{kg K}} \), and \( T_{0,1} = 273.15 \text{K} \), the mass of the gas is calculated to be \( 3.42 \text{g} \).