Graph 1: The graph is a plot with the vertical axis labeled as p in bar and the horizontal axis labeled as T in Kelvin. The plot displays a wavy line that begins at the bottom left, rises and falls multiple times, and concludes at the bottom right. Additionally, there is a shaded region in the upper left corner of the graph.

The following points are noted:
- From point 1 to point 2: Temperature increases, pressure decreases.
- From point 2 to point 3: Entropy at point 2 equals entropy at point 3, pressure increases.
- From point 3 to point 4: The process is isobaric, temperature increases, pressure remains constant.
- From point 4 to point 1: The process is isenthalpic, enthalpy at point 4 equals enthalpy at point 1, pressure decreases, temperature remains constant.

Graph 2: The graph is a plot with the vertical axis labeled as p in bar and the horizontal axis labeled as T in Kelvin. The plot displays a closed loop with four points labeled 1, 2, 3, and 4. The path from point 1 to point 2 is curved and labeled as temperature constant. The path from point 2 to point 3 is straight and labeled as pressure constant. The path from point 3 to point 4 is curved and labeled as temperature constant. The path from point 4 to point 1 is straight and labeled as pressure constant.