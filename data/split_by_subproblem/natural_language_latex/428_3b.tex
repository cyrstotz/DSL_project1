The gas in the cylinder is considered as a perfect gas, with a molar mass \( M_g = 10 \, \frac{\text{kg}}{\text{kmol}} \).

The universal gas constant \( R^* \) is calculated as follows:
\[
R^* = \frac{R}{M} = \frac{8.314 \, \frac{\text{J}}{\text{mol} \cdot \text{K}}}{10 \, \frac{\text{kg}}{\text{kmol}}} = 0.8314 \, \frac{\text{J}}{\text{kg} \cdot \text{K}}
\]
which simplifies to
\[
R^* = 0.8314 \, \frac{\text{J}}{\text{kg} \cdot \text{K}}
\]

The mass \( M \) of the gas is calculated as:
\[
M = \frac{p \cdot V}{R \cdot T} = \frac{3.14 \, \text{L} \cdot 1.46 \, \text{bar}}{166.58 \, \frac{\text{J}}{\text{kg} \cdot \text{K}} \cdot 775.15 \, \text{K}} = 3.14 \, \text{g}
\]

- Initial condition: If the temperature of the ice \( T_{Eis} \) is greater than 0 degrees Celsius, it implies there is still ice in the ice-water mixture, hence \( T_{Eis,2} = 0^\circ C \).
- Consequently, the temperature \( T_{gz,2} = 0^\circ C \).
- Conclusion: Since the saturation with the piston remains unchanged, \( P_{S,1} = P_{S,2} \) which equals 7.4 bar.