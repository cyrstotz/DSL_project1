- \( \varepsilon_{K} \) refers to the efficiency of a refrigerator.
- The efficiency of the refrigerator, \( \varepsilon_{K} \), is given by the ratio of the heat input \( \dot{Q}_{zu} \) to the absolute value of the work input \( \dot{W}_{K} \), expressed as \( \varepsilon_{K} = \frac{\dot{Q}_{zu}}{\left| \dot{W}_{K} \right|} \).
- \( \dot{Q}_{zu} \) is equal to \( \dot{Q}_{K} \), which relates to the first law of thermodynamics for a stationary flow process, specifically from state 1 to state 2.
- The equation \( 0 = \dot{m} (h_{1} - h_{2}) + \dot{Q}_{K} - \dot{W}_{K} \) holds under isobaric conditions.
- \( \dot{Q}_{K} \) is calculated as \( \dot{m} (h_{2} - h_{1}) \), where \( h_{1} \) is given as 93.42 kJ/kg, see further details in the referenced section.