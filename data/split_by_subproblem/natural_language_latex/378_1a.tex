The derivative of E dot with respect to time equals the mass flow rate (m dot) times the difference between the enthalpy at the inlet (h subscript ein) and the enthalpy at the outlet (h subscript aus), plus the heat transfer rate out (Q dot subscript ab) minus the work rate (W dot).

A diagram is drawn here. It is a rectangular box with arrows pointing in and out. The left side of the box is labeled with h subscript ein and m dot subscript ein. The right side of the box is labeled with h subscript aus and m dot subscript aus. The top side of the box is labeled with Q dot subscript ab with an arrow pointing into the box. The bottom side of the box is labeled with W dot with an arrow pointing out of the box.

It implies that the heat transfer rate out (Q dot subscript ab) equals the mass flow rate (m dot) times the difference between the enthalpy at the outlet (h subscript aus) and the enthalpy at the inlet (h subscript ein).

The enthalpy at the inlet (h subscript ein) for a temperature of 70 degrees Celsius and quality x equals 0 is equal to the saturated liquid enthalpy at 70 degrees Celsius, which is 292.58 kilojoules per kilogram.

The enthalpy at the outlet (h subscript aus) for a temperature of 100 degrees Celsius and quality x equals 0 is equal to the saturated liquid enthalpy at 100 degrees Celsius, which is 419.04 kilojoules per kilogram.

It implies that the heat transfer rate out (Q dot subscript ab) equals 0.3 kilograms per second times the difference between 419.04 kilojoules per kilogram and 292.58 kilojoules per kilogram, which equals 37.818 kilowatts.

The heat transfer rate out (Q dot subscript ab) equals the heat transfer rate in (Q dot subscript zu) minus the heat transfer rate out (Q dot subscript aus), which implies that the heat transfer rate out (Q dot subscript aus) equals 62.982 kilowatts.