Energy balance around a closed system:

There is a matrix with the elements 'wo', 'wi', 'he', 'hc' each enclosed in boxes.

The equation for w0 is given as the rate of heating, which equals 200 Joules per gram.

The equation for the system is:
0 equals the mass flow rate times (h0 minus hc plus (the difference of the squares of the velocities wu and wc divided by 2) plus the rate of heat transfer minus the rate of work.

The difference between h0 and hc is calculated as h at T6 equals 243.19 Kelvin minus h at T equals Tc.

The equation simplifies to:
1 times the specific heat capacity times (T6 minus T0) minus 1.000 times (314.4 minus 243.19) equals 71.9 kilojoules per kilogram times (hc minus h0).

It is suggested to find Tc via the nozzle equation 5-26:
(T6 over T5) equals (P0 over P5) raised to the power of ((k-1) over k), which leads to T6 equals T5 times (P6 over P5) raised to the power of ((k-1) over k).

The calculation for Tc is:
1 times 314.4 times (0.199 over 0.5) raised to the power of (0.4 over 0.4) equals 314.4 Kelvin minus Tc.

Finding the total mass flow rate with the heating mass flow rate being 273 times the heating mass flow rate:
The heating mass flow rate equals 5.293.

Energy balance at the combustion chamber:
0 minus the mass flow rate times (h2 minus h3) plus the rate of heat transfer due to burning.

The mass flow rate at condition c is calculated as:
The rate of heat transfer due to burning divided by (h2 minus h3) equals 98 over 1296.7 equals 0.9215 kilograms per second.

The mass flow rate at condition 4 is:
5.293 times the mass flow rate at condition c equals 487.78 kilograms per second.

The difference between h2 and h3 is:
The specific heat capacity times (T2 minus T3) minus the specific heat capacity at T3 equals 1.000 times 1289 kilojoules per kilogram, which equals 1296.7 kilojoules per kilogram.

The net mass flow rate plus the heating mass flow rate plus another mass flow rate equals 5.799 kilograms per second, approximately 5.8 kilograms per second.

Plugging into the overall energy balance law:
2 times the mass flow rate times (h2 minus h0) equals the mass flow rate times w2 squared minus 2 times the mass flow rate times (h2 minus h0) equals w6 squared.

The calculation for w6 is:
The mass flow rate times (200 squared meters squared per second squared plus 2 times 5.8 times 7.9) divided by 2 equals 202.67 meters per second.

There is a note that the value for m might be wrong and is way too small.