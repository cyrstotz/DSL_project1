Item c) includes:

1. Graph 5: A graph with axes labeled P (vertical axis) and T (horizontal axis). The graph shows a curve starting from a point labeled 1, rising steeply, then curving downwards and leveling off towards the right.

2. Graph 6: A graph with axes labeled P (vertical axis) and T (horizontal axis). The graph shows a complex curve with points labeled 1, 2, and 3. The curve starts at point 1, moves to point 2, then to point 3, and finally loops back towards point 2.

The following equations are presented:

- The mass flow rate times the difference in enthalpy between h1 and h2.
- The mass flow rate times the difference in enthalpy between h2 and h3 plus the heat transfer rate minus the work rate is greater than zero.
- The enthalpy at state 2 when x equals 1 implies saturated vapor enthalpy.
- The enthalpy at state 3 at 8 bar is equal to the enthalpy at state 3 at 8 bar when x equals 2.
- The initial temperature Ti equals 16 hours above the subline implies the temperature T: 4.5 bar and triple point equals 0 degrees Celsius.
- The mass flow rate times the difference in enthalpy between h2 and h3 plus the heat transfer rate minus the work rate is greater than zero.
- The mass flow rate at temperature T equals 0 degrees Celsius.

In the student solution section:

The equation describes a process where x_n T_2 goes through an adiabatic throttle to an isotropic state at 20 bar. The entropy at state n equals the entropy at state u, and x_n is calculated as the ratio of the difference between entropy at state u and entropy at state 1t to the difference between the entropy at state g1 and entropy at state 1t.

Additionally, the entropy at state u (at 8 bar, x_u equals 0) equals the entropy at state 1t, and the entropy at state k (at 8 bar) equals 0.345 times the entropy at state g. The change in temperature Delta T equals 31.33 degrees Celsius.