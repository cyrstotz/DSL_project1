**Aircraft engine stationary and adiabatic**

The equation is:
Q equals m dot times the quantity h subscript e minus h subscript a plus the quantity w subscript e squared minus w subscript a squared all over 2 plus p subscript e superscript 0 divided by rho superscript 0, plus Q dot superscript 0 minus W dot.

**5-7-6 isentropic**

The equation is:
T subscript 0 equals T subscript 5 times the quantity p subscript 6 over p subscript 5 raised to the power of (n minus 1) over n.

The equation is:
T subscript 0 equals T subscript 5 times the quantity p subscript 6 over p subscript 5 raised to the power of (n minus 1) over n equals 328.1 Kelvin, defined as A.

**Energy balance around the jet nozzle**

The equation is:
0 equals m dot times the quantity h subscript c minus h subscript a plus the quantity w subscript c squared minus w subscript a squared all over 2 plus p subscript c superscript 0 divided by rho superscript 0, minus u dot, where m dot equals m dot.

The equation is:
w subscript 0 squared over 2 equals h subscript 5 minus h subscript 6 plus w subscript 5 squared over 2 minus w subscript rev,56, multiplied by 2.

The equation is:
w subscript 0 squared equals 2 times the quantity h subscript 5 minus h subscript 6 plus w subscript 5 squared minus 2 times w subscript rev,56, where h subscript 5 minus h subscript 6 equals c subscript p times the quantity T subscript 5 minus T subscript 6, and w subscript rev,56 equals R times the quantity T subscript 6 minus T subscript 5 over 1 minus n.

The equation is:
w subscript 0 squared equals 2 times c subscript p times the quantity T subscript 5 minus T subscript 6 plus w subscript 5 squared minus 2 times the quantity R times the quantity T subscript 6 minus T subscript 5 over 1 minus n.

The equation is:
w subscript 0 equals the square root of 329.11 meters per second, defined as C.

The equation is:
R equals R bar over M subscript Luft equals 28.849, defined as B, with units Joules per kilogram Kelvin.