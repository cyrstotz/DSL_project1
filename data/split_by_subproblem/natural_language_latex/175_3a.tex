a) 

A rectangular diagram is drawn with three horizontal lines inside it, dividing it into four sections. To the right of the diagram, there is a vertical arrow labeled "F". Above the diagram, there are several downward arrows labeled "Patm" and "mg". Below these arrows, there are several upward arrows labeled "p1". The width of the diagram is labeled "D equals 10 centimeters".

The gravitational force F_g is equal to mass m times gravitational acceleration g, which equals 31.319 Newtons.

The atmospheric force F_atm is equal to pressure p times the square of 5 centimeters times pi, which simplifies to pressure p times the square of 0.005 meters times pi, equaling 7.854 Newtons.

The force F_p1 is equal to pressure p1 times the square of 0.005 meters times pi.

In the force equilibrium section, it states that F_p1 equals F_g plus F_atm, which leads to p1 being equal to (F_g plus F_atm) divided by (the square of 0.005 meters times pi), resulting in 40.96 times 10 to the power of 5 Pascals.

p1 equals 40.96 bar.

The gas constant R is calculated as the universal gas constant R divided by molar mass M, which equals 8.314 Joules per mole Kelvin divided by 50 kilograms per kilomole, resulting in 0.16628 Joules per kilogram Kelvin.

The equation p1 V1 equals m R T1 is given.

The mass m is calculated as (p1 V1) divided by (R T1), which equals (40.96 times 10 to the power of 5 Pascals times 3.14 times 10 to the power of -3 cubic meters) divided by (0.16628 Joules per kilogram Kelvin times 773.15 Kelvin), resulting in 99.27 kilograms.

There is a note stating "since I thought my pressure was wrong, I continue calculating with the given values".

The mass m is recalculated with the given pressure p_geg as (p_geg V1) divided by (R T1), which equals (1.5 times 10 to the power of 5 Pascals times 3.14 times 10 to the power of -3 cubic meters) divided by (0.16628 Joules per kilogram Kelvin times 773.15 Kelvin), resulting in 3.635 grams.