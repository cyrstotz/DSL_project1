The rate of exergy change for component c is given by the equation:
The rate of exergy change for component c equals the mass flow rate of exergy for component c, which is equal to the mass flow rate times the expression in brackets. The expression includes the enthalpy at the exit minus the enthalpy at the inlet minus the product of the inlet temperature and the difference in entropy between the exit and inlet, plus the change in kinetic energy.

The kinetic energy for component c is given by:
The kinetic energy for component c equals one half times the mass of component c times the square of velocity, which is equal to one half times the mass of component c times the square of the inlet velocity.

The rate of exergy change for component o is given by:
The rate of exergy change for component o equals the mass flow rate of exergy for component o, which is equal to the mass flow rate times the enthalpy at the outlet.

The change in the rate of exergy is given by:
The change in the rate of exergy equals the rate of exergy change for component c minus the rate of exergy change for component o.

This simplifies to:
The change in the rate of exergy equals the mass flow rate times the expression in brackets. The expression includes the enthalpy at the outlet minus the enthalpy at the exit minus the product of the outlet temperature and the difference in entropy between the outlet and component c, plus the change in kinetic energy.

The change in kinetic energy is given by:
The change in kinetic energy equals one half times the mass of components o and c times the square of the difference in velocities between the outlet and component c, which simplifies to one half times the mass of components o and c times the square of the difference between 510 meters per second and 200 meters per second.

The enthalpy at the outlet, given conditions (243K, 0.10, omega), is calculated by:
The enthalpy at the outlet equals the ratio of the difference in enthalpies at 250.05 and 240.02 kilojoules per kilogram over the difference in temperatures at 250 and 240 Kelvin, multiplied by the difference in temperature from 250 to 243 Kelvin, plus 240.02 kilojoules per kilogram.

This results in:
The enthalpy at the outlet equals 247.04 kilojoules per kilogram.

The difference in entropy between the outlet and component c is given by:
The difference in entropy equals the entropy at the outlet temperature minus the entropy at the exit temperature minus the gas constant times the natural logarithm of the ratio of the pressures at the outlet and the exit.