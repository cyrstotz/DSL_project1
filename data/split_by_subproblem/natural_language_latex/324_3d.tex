1. For the first equation, x subscript 2 equals x.

2. The change in internal energy, Delta U, equals the heat transferred, Q subscript 12, minus the work done, W subscript 12.

3. The mass of the mixture, m subscript EW, times the difference in internal energy per unit mass from state 1 to state 2, (u subscript 2 minus u subscript 1), equals the heat transferred, Q subscript 12.

4. The internal energy per unit mass at state 2, u subscript 2, equals the heat transferred, Q subscript 12, divided by the mass of the mixture, m subscript EW, plus the internal energy per unit mass at state 1, u subscript 1.

5. The internal energy per unit mass at state 2, u subscript 2, equals negative 1,361.54 Joules.

6. The internal energy per unit mass at state 2, u subscript 2, equals 0.1 kilograms times negative 200.

7. The internal energy per unit mass at state 2, u subscript 2, equals negative 186.3836 kilojoules per kilogram.

8. The internal energy per unit mass at state 2 at temperature equals zero, u subscript 2 (T equals 0), equals negative 186.3836.

9. The internal energy per unit mass at state 2 at temperature equals zero, u subscript 2 (T equals 0), equals negative 186.3836, which is equal to the internal energy per unit mass of the liquid, u subscript Flüssig, plus x times the difference between the internal energy per unit mass of the saturated vapor, u subscript RST, and the internal energy per unit mass of the liquid, u subscript Flüssig.

10. x equals the difference between the internal energy per unit mass at state 2, u subscript 2, and the internal energy per unit mass of the liquid, u subscript Flüssig, divided by the difference between the internal energy per unit mass of the saturated vapor, u subscript RST, and the internal energy per unit mass of the liquid, u subscript Flüssig, which equals 0.553.

11. If all is gas then x equals 1.

12. The internal energy per unit mass at state 1 at temperature equals zero and x equals 0.6, u subscript 1 (T equals 0, x equals 0.6), equals the internal energy per unit mass of the liquid, u subscript Flüssig, plus x times the difference between the internal energy per unit mass of the saturated vapor, u subscript RST, and the internal energy per unit mass of the liquid, u subscript Flüssig.

13. Interpolate Table 1.

14. The internal energy per unit mass at point A, u subscript A, equals negative 0.045 plus 0.6 times the difference between negative 333.458 and 0.045.

15. The internal energy per unit mass at state 1, u subscript 1, equals negative 200 kilojoules per kilogram.

16. The internal energy per unit mass of the liquid, u subscript Flüssig, equals negative 0.045 kilojoules per kilogram.

17. The internal energy per unit mass of the saturated vapor, u subscript RST, equals negative 333.458 kilojoules per kilogram.