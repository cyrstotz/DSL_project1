The equations provided are as follows:

1. The pressure at equilibrium, denoted as \( p_{ew} \), is equal to the force of gravity per unit area, \( \frac{F_g}{A} \), plus the ambient pressure \( p_{amb} \).

2. The calculation for the force of gravity per unit area is given by \( \frac{m_g \cdot g}{(9.7 \cdot \pi)} \cdot 10^5 \, \text{Pa} \), which equals \( \frac{23.9 \cdot 0.1}{0.00785} \cdot 10^5 \) resulting in approximately 728792.6 Pa or about 7.3 bar.

3. The temperature at equilibrium, \( T_{ew} \), is \( 0^\circ \).

4. The internal energy \( U_A \) is calculated as the sum of the product of the mass of ice \( m_{eis} \) and its specific internal energy \( U_{Fe} \), plus the product of the mass of water \( m_w \) and its specific internal energy \( U_{FL} \). This results in \( 0.075 \cdot -333.4486 + 0.062 \cdot 0.0366 = -1753.05 \).

5. The mass of ice \( m_{eis} \) is 0.6 times the mass of water \( m_w \).

6. The total mass \( m_{ew} \) is the sum of the mass of ice \( m_{eis} \) and the mass of water \( m_w \), which equals \( 1.6 \cdot m_w \) and simplifies to \( 0.1 \cdot 1.6 = 0.062 \, kg \).

7. The mass of ice \( m_{eis} \) is recalculated as \( 0.6 \cdot m_w = 0.6 \cdot 0.062 = 0.0375 \, kg \).

8. The specific internal energy \( u_{Fe} \) is calculated using a formula involving values \( U(96a) \) and a fraction involving changes in \( u \) values over a range, resulting in \( -333.4486 \).

9. The specific internal energy \( u_{FL} \) is similarly calculated, resulting in \( -0.065 \).

10. The change in energy \( \Delta E \) is equal to the heat \( Q \) multiplied by \( U \), and \( Q \) is given as 1366.29. This is equal to the sum of the products of the masses and specific internal energies of ice and water, multiplied by the product of \( U_{CL} \), \( m_{ew} \), and \( U_{A} \).

11. The equation \( Q - U_A \) is equal to the sum of the products of the mass of ice and its specific internal energy, plus the product of the difference of 0.1 and the mass of ice, and the specific internal energy of water. This simplifies to \( m_{eis} \cdot (U_{Fe} - U_{FL}) \).

12. The mass of ice \( m_{eis} \) is then recalculated using the formula involving \( Q \), \( U_A \), and the specific internal energies.

13. The mass of water \( m_w \) is recalculated as \( 0.1 - m_{eis} \).

14. The fraction \( x \) is calculated as the ratio of the mass of ice to the sum of the mass of ice and the mass of water in equilibrium.