The description is of a graph with two axes. The vertical axis is labeled \( P(v) \) and the horizontal axis is labeled \( v \) with units of cubic meters per kilogram. The graph features two curves. The first curve begins at the origin, ascends steeply to a peak, and then declines gradually. The second curve also starts at the origin, ascends more gently, intersects the first curve at its peak, and then continues to ascend.