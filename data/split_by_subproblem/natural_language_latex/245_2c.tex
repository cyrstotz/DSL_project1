e subscript x, st6 equals e subscript x, st4 minus the quantity h subscript 6 minus h subscript 0 minus T subscript 0 times the quantity s subscript 6 minus s subscript 0 plus omega subscript 6 squared over 2, minus the quantity h subscript 5 minus h subscript 0 minus T subscript 0 times the quantity s subscript 5 minus s subscript 0 plus omega subscript 5 squared over 2. This equals the quantity h subscript 6 minus h subscript 5 minus T subscript 0 times the quantity s subscript 6 minus s subscript 5 plus omega subscript 6 squared minus omega subscript 5 squared over 2. This equals the integral of c subscript p dT minus T subscript 0 times the quantity integral of c subscript p over T dT minus the natural logarithm of the ratio P subscript 6 over P subscript 5 plus omega subscript 6 squared minus omega subscript 5 squared over 2.

R equals c subscript p minus c subscript v equals c subscript p over c subscript p minus c subscript v equals 0.28742 kilojoules per kilogram Kelvin.
T subscript 0 equals 293.15 Kelvin.

Delta e subscript x, st4 equals c subscript p times the quantity T subscript 6 minus T subscript 5 minus T subscript 0 c subscript p times the natural logarithm of the ratio T subscript 6 over T subscript 5 minus T subscript 0 times the natural logarithm of the ratio P subscript 6 over P subscript 5 plus omega subscript 6 squared minus omega subscript 5 squared over 2. This equals negative 109.95 minus 67.26 minus 67.255 plus 1 plus 74.437. This equals negative 167.97 kilojoules per kilogram.