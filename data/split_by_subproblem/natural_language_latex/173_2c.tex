T1 equals 773.15 Kelvin.
T2 equals 273.15 Kelvin.
p1 equals 101000 Pascals.
p2 equals 101000 Pascals.

The rate of change of internal energy with respect to time equals the sum of mass flow rate plus the sum of heat flow minus the sum of work done.

The change in internal energy equals heat added minus work done.

The heat transfer from state 1 to state 2 equals the change in internal energy plus the work done from state 1 to state 2.

The work done from state 1 to state 2 equals the integral of pressure with respect to volume from volume 1 to volume 2.

The product of pressure and volume equals mass times the gas constant times temperature.
Volume 2 equals mass times the gas constant times temperature 2 divided by pressure 2, which calculates to 9.474 times 10 to the power of minus 5 cubic meters.

The work done from state 1 to state 2 equals pressure times the integral of dV from volume 1 to volume 2, which simplifies to pressure times the difference between volume 2 and volume 1, resulting in -426.64 Joules.

The change in internal energy equals the specific heat at constant volume times the difference in temperatures, which calculates to -319.5 Joules.

The heat transfer from state 1 to state 2 equals the change in internal energy plus the work done from state 1 to state 2, resulting in -746.14 Joules.

Graphical Description:
There is a diagram of a piston-cylinder device. The cylinder is drawn as a rectangle with a horizontal line at the top representing the piston. Inside the cylinder, there is a shaded area labeled as V2. An arrow pointing upwards next to the piston indicates the work done by the gas, labeled as w.