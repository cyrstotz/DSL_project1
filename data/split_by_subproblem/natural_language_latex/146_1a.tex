a) The heat transfer rate, denoted as Q dot ums, equals what?

Zero equals the mass flow rate, m dot, times the difference between the enthalpy at exit, h sub e, and the enthalpy at entry, h sub c, plus the sum of Q dot sub i for all i.

This implies that the mass flow rate, m dot, times the difference between the enthalpy at point a, h sub a, and the enthalpy at exit, h sub e, minus Q dot sub i equals the heat transfer rate, Q dot ums, which is equal to 0.3 times the mass flow rate, m dot, times the difference between the enthalpy at point 2, h sub 2, and the enthalpy at point 1, h sub 1, minus 100 kilowatts.

The enthalpy at A-2 is 2626.8 kilojoules per kilogram, the enthalpy at A-2g is 2676.4 kilojoules per kilogram, the enthalpy at A,f is 252.9 kilojoules per kilogram, and the enthalpy at 2,f is 419.04 kilojoules per kilogram.

The enthalpy, h, equals the enthalpy at the fluid, h sub f, plus the quality, x, times the difference in enthalpy, h sub fg.

The enthalpy at point 1, h sub 1, is 304.65 kilojoules per kilogram, and the enthalpy at point 2, h sub 2, is 436.33 kilojoules per kilogram.

The heat transfer rate, Q dot ums, equals negative 62.3 kilowatts.