The rate of change of \( e_{x, \text{ist}} \) is equal to \( E \) divided by \( \dot{m} \).

Exreal equals \( T_0 \) times \( s_{erz} \).

\( s_{erz} \) equals \( \dot{m} \) times \( (s_2 - s_1) \) minus \( Q \) divided by \( T_J \), and this is labeled as 1:m.

\( s_{erz} \) equals \( \dot{m}_{05} \) times \( (s_8 - s_0) \) minus \( q \) divided by \( T_J \).

\( s_{erz} \) equals \( \dot{m} \) times \( cp \) times \( (T_2 - T_1) \) minus 110.050 kilojoules per kilogram.

\( m_{05} \) equals \( \dot{m} \) times 4.5 times 2.5.

\( m_{05} \) equals \( \dot{m}_1 \) divided by 5.293.

\( \dot{m}_{05} \) equals 3 GT 5 1 kilowatts per kilojoule.

\( \dot{E} \) equals 80.3 times 12 kilowatts per kilojoule.

\( m_{05} \) equals \( \dot{m}_1 \) times \( (1 + 5.293 nc) \).

At the top left of the page, there is a small circle with the letter "e" inside it. To the right of this circle, there is an arrow pointing to the right.

The change in \( S_{12} \) equals \( m^t \) times \( s_2 \) minus \( m^i \) times \( s_1 \).

Below the formula, there is a scribble or crossed-out section.
To the left of the next section, there is a circle with "T1" written inside it.
Below the circle, there is a label "TAZ".

\( T_1 \) equals 100.

\( s \) equals 1.33 megajoules per kilogram.

\( s_2 \) equals \( s_3 \) equals 7.7533.

The change in \( S_{12} \) equals 4.813 megajoules per kilogram minus 35 megajoules divided by 295 Kelvin.

Below the formula, there is a horizontal arrow pointing to the left.

The change in \( S_{12} \) equals 9.813 times \( t \).