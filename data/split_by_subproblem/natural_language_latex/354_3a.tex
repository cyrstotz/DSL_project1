The pressure \( p_1 \) is 1.9 bar.

The masses \( m_1 \) and \( m_2 \) are both 0.1 kg.

The internal energy \( U_1 \) at 0 degrees Celsius and 1.9 bar is calculated as \( U_{fg} \) at 0 degrees Celsius and 1.9 bar plus \( x \) times the difference between \( U_g \) and \( U_f \) at 0 degrees Celsius and 1.9 bar, which equals -133.4 kilojoules per kilogram.

The quality \( x_2 \) is calculated as the ratio of \( U_2 - U_{est} \) to \( U_{Düssig} - U_{est} \).

The volume \( U_{2g} \) is calculated using the formula \( \frac{m \cdot R \cdot T_2}{p} \), resulting in 1.11 times 10 to the power of -3 cubic meters.

The work \( W \) is calculated as \( p_1 \) times the difference between \( V_2 \) and \( V_1 \), resulting in 28.27 Joules.

The masses \( m_1 \), \( m_2 \), and \( m_w \) are equal.

The equation \( m_2 U_2 - m_w U_w = Q_{12} - W_{12} \) is given.

The internal energy \( U_2 \) is calculated as \( \frac{Q_{12} - W_{12}}{m_w} + U_w \), resulting in -122.5 kilojoules per kilogram.

The quality \( x_2 \) is calculated again and found to be 63.28%.

The force \( F \) is calculated as pressure \( p \) times area \( A \).

Graphical descriptions are provided for different setups involving forces and pressures, including a vertical rectangle divided into three sections with a mass \( M \), and a horizontal rectangle with a mass \( M \) on top and various pressures and forces acting on it.

The pressure \( P_1 \) is calculated as \( p_{amb} + \frac{4mg}{\pi d^2} \), resulting in 1.20 bar.

The pressure \( P_2 \) is calculated as \( P_1 + \frac{4 m_{EW} g}{\pi d^2} \), resulting in 1.40 bar.

The mass \( m_g \) of the gas is calculated using the ideal gas law, resulting in 3.42 times 10 to the power of -3 kilograms.