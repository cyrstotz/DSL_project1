R equals the ratio of R bar over M_g, which equals 8.314 Joules per mole Kelvin divided by 50 kilograms per kilomole, resulting in 0.1663 kilojoules per kilogram Kelvin.

V_g,1 equals 3.14 liters, which is equivalent to 3.14 times 10 to the power of negative 3 cubic meters.

The product of p_g,1 and V_g,1 equals R times T_g,1 times m_g,1. Therefore, p_g,1 equals the fraction of 0.1663 kilojoules per kilogram Kelvin times the sum of 500 and 273.15 Kelvin, all divided by 3.14 times 10 to the power of negative 3 cubic meters, resulting in 2863.6 kilopascals.

p_m,g times pi times D squared equals g times the sum of m_m and m_m,w plus p_amb. Thus, p_m,g equals the fraction of 9.81 meters per second squared times the sum of 0.14 kilograms and 32 kilograms, all divided by pi times the square of 0.1 meters, resulting in 10023 Pascals, which is approximately 11.1348 times 10 to the power of 6 Pascals.

m_g,1 equals the fraction of p_g,1 times V_g,1 divided by R times T_g,1, which equals the fraction of 2863.6 times 10 to the power of 3 Pascals times 3.14 times 10 to the power of negative 3 cubic meters, all divided by 0.1663 times 10 to the power of 3 Joules per kilogram Kelvin times 773.15 Kelvin, approximately equaling 2.445 times 10 to the power of negative 4 kilograms.