Given: \( p_1, T_1 \)

The mass \( m_2 \) is equal to \( \frac{m}{50 \, \text{grams per mole}} \).

The constant \( \Pi \) is calculated as \( \frac{8.314 \, \text{joules per mole kelvin}}{50 \, \text{grams per mole}} \) which equals \( 166.28 \, \text{joules per kilogram kelvin} \).

The gas constant \( R \) is equal to \( c_p - c_v \).

The specific heat at constant pressure \( c_p \) is \( R + c_v \) which equals \( 166.28 \, \text{joules per kilogram kelvin} + 0.633 \, \text{kilojoules per kilogram kelvin} \) resulting in \( 0.79928 \, \text{kilojoules per kilogram kelvin} \).