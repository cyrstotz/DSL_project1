T sub 5 equals 431.9 Kelvin.
p sub 5 equals 0.1 bar.
v sub 5 equals 220 cubic meters per kilogram.
For the ideal process from state 5 to 6, n equals k equals 1.4.
The ratio of T sub 6 to T sub 5 equals the ratio of p sub 6 to p sub 5 raised to the power of (n minus 1) divided by n.
This implies T sub 6 equals T sub 5 times the ratio of p sub 0 to p sub 5 raised to the power of 0.4 over 1.4.
This implies T sub 6 equals 431.9 Kelvin times the ratio of 0.194 bar to 0.1 bar raised to the power of 0.4 over 1.4.
This results in T sub 6 equals 328.075 Kelvin.

Zero equals m dot sub 12 times (h sub 5 minus h sub 6) plus the difference of w sub 5 squared minus w sub 6 squared over 2 plus Q dot minus W dot sub tn.
h sub 6 minus h sub 5 equals the difference of w sub 5 squared minus w sub 6 squared over 2.
c sub p times (T sub 0 minus T sub 5) equals the difference of w sub 5 squared minus w sub 6 squared over 2.
This implies w sub 0 equals 507.24 cubic meters per second.

Listed items:
- c sub p equals 1.006 kilojoules per kilogram Kelvin.
- T sub 6 equals 328.075 Kelvin.
- T sub 5 equals 431.9 Kelvin.
- w sub 5 equals 220 cubic meters per second.