The energy efficiency \( E_u \) is equal to the ratio of the heat input rate \( \dot{Q}_{zu} \) to the turbine work rate \( \dot{W}_t \), which is further expressed as the ratio of the absolute value of the heat input rate \( |\dot{Q}_{zu}| \) to the difference between the heat rejection rate \( \dot{Q}_{abl} \) and the absolute value of the heat input rate \( |\dot{Q}_{zu}| \).

It follows that the heat input rate \( \dot{Q}_{zu} \) is equal to the heat input rate \( \dot{Q}_i \), which is the product of the mass flow rate \( \dot{m} \) and the difference in enthalpy between states 2 and 1, \( h_2 - h_1 \).

It also follows that the heat rejection rate \( \dot{Q}_{ab} \) is equal to the product of the mass flow rate \( \dot{m} \) and the difference in enthalpy between states 4 and 3, \( h_4 - h_3 \).

The enthalpy at state 1, \( h_1 \), is a function of temperature \( T_1 \) and pressure \( p_1 \).

The enthalpy at state 2, \( h_2 \), for the temperature \( T_2 - 22^\circ C \), is given as 236.08.

The enthalpy at state 3, \( h_3 \), is a function of the pressure (8 bar) and temperature \( T_3 \).

The enthalpy at state 4, \( h_4 \), is the fluid enthalpy \( h_f \) at a pressure of 8 bar, given as 93.62.

In German, the text states that the temperature \( T_p \) would continue to decrease until it hypothetically reaches absolute zero.