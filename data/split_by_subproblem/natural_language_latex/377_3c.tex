In the section labeled "c)", the following equations and descriptions are presented:

1. The values for \(Q_{12}\), \(T_{2g}\), \(p_{2g}\), and \(m_g\) are given as \(0.003^\circ C\), \(1.5\) bar, and \(3.6\) grams respectively.

2. A diagram is described featuring a container with a piston. The piston is labeled \(Q_{12}\) and the container is filled with a substance.

3. The left-hand side (LHS) of an equation is given by the rate of change of energy \( \frac{dE}{dt} \) which equals the mass flow rate \( \dot{m} \) times \( h + \frac{k}{p} \) plus the heat flow rate \( \dot{Q} \) minus the work flow rate \( \dot{W} \), integrated over time \( \int dt \).

4. The change in internal energy \( \Delta U_{12} \) is equal to the heat flow rate \( \dot{Q} \) minus the work flow rate \( \dot{W} \).

5. The mass of the gas \( m_g \) times the change in specific internal energy \( u_2 - u_1 \) equals the heat flow rate \( \dot{Q} \). Assuming an ideal gas model, \( u_2 - u_1 \) is equal to \( c_v \) times the temperature difference \( T_2 - T_1 \).

6. The heat transfer \( Q \) is calculated as the mass of the gas \( m_g \) times the specific heat at constant volume \( c_v \) times the temperature difference \( T_{2g} - T_{1g} \). This results in a value of \(-1.133935\) kilojoules.

7. The absolute value of \( Q_{12} \) is underlined and given as \( 1133.39 \) Joules.

Additionally, there are two other equations presented:
- \( x_{EIS2} \)
- \( V_{EIN} = V_{ZEW} \) which states that the volume \( V_{EIN} \) is equal to the volume \( V_{ZEW} \).