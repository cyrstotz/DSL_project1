The volume V subscript g,2 is less than the volume V subscript g,1.

The equation P times V equals m times R times T.

The pressure p subscript g must be equal because the load on the piston is the same. If the pressure p subscript g,1 equals the pressure p subscript g,2 and the volume V subscript g,2 is less than the volume V subscript g,1, it implies that the temperature T subscript 2g is less than the temperature T subscript 1g because the gas is perfect.

The pressure p subscript 2g equals the pressure p subscript 1g equals 1.5 bar.

The temperature T subscript 2g equals 0 degrees Celsius.