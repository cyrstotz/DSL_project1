a) The rate of heat output, denoted as Q-dot-out-star, equals negative mass flow rate times the difference between the enthalpy at the exit and the enthalpy at the entrance.

The temperature at the inlet of the cooler, T-sub-K-comma-in, is 288.15 Kelvin, and the temperature at the outlet of the cooler, T-sub-K-comma-out, is 298.15 Kelvin.

The temperature at the inlet, T-sub-in, is 70 degrees Celsius, and the temperature at the outlet, T-sub-out, is 100 degrees Celsius (pure water).

The enthalpy at the entrance, h-sub-e, equals the enthalpy at 70 degrees Celsius from Table A-2, which is 292.98 kilojoules per kilogram.

The enthalpy at the exit, h-sub-ea, is calculated as the enthalpy of the fluid at 100 degrees Celsius plus x-sub-D times the difference between the enthalpy of the gas at 100 degrees Celsius and the enthalpy of the fluid at 100 degrees Celsius. This results in 419.04 plus 0.005 times (2676.1 minus 419.04) kilojoules per kilogram, which equals 430.33 kilojoules per kilogram.

The rate of heat output, Q-dot-out-star, is calculated as negative 0.3 kilograms per second times the difference between 292.98 kilojoules per kilogram and 430.33 kilojoules per kilogram, resulting in 41.2 kilowatts. Therefore, the heat rate Q-dot-R equals Q-dot-out-star minus Q-dot-out, where Q-dot-out is 58.8 kilowatts.