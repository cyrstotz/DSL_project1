The text describes a control volume with a diagram showing four arrows indicating the flow of heat and mass. The top arrow pointing into the control volume is labeled as the heat input, denoted by Q-dot-in equals 100 kilowatts. The left arrow pointing into the control volume is labeled as mass flow rate in, denoted by m-dot-in. The right arrow pointing out of the control volume is labeled as mass flow rate out, denoted by m-dot-out. The bottom arrow pointing out of the control volume is labeled as heat output, denoted by Q-dot-out.

The equations provided are as follows:

1. The steady-state energy equation for the control volume is given by m-dot times the sum of (h1 minus h2 plus v squared over 2 plus g times z) plus the sum of Q-dot minus W-dot equals zero.

2. The inequality 0 less than x_D less than 1 implies a non-azeotropic mixture.

3. At condition Z1, the temperature is 70 degrees Celsius and the quality, x_p, is 0.005.

4. The enthalpy h1 is calculated using the formula: h1 equals h_f at 70 degrees Celsius plus x_D times (h_g at 70 degrees Celsius minus h_f at 70 degrees Celsius).

5. The value of h1 is given as 304.65 kilojoules per kilogram, where h_f is 292.58 and h_g is 2626.8.

6. The enthalpy h2 is calculated using the formula: h2 equals h_f at 100 degrees Celsius plus x_D times (h_g at 100 degrees Celsius minus h_f at 100 degrees Celsius).

7. The value of h2 is given as 430.33 kilojoules per kilogram, where h_f is 419.04 and h_g is 2676.1.

8. The equation m-dot times (h1 minus h2) plus 100 kilowatts minus Q-dot-out equals zero.

9. The heat output, Q-dot-out, is calculated to be 62.3 kilowatts.