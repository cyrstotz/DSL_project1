The temperature \( T_{g1} \) is 500 degrees Celsius. The volume \( V_{g1} \) is 3.14 liters, which is equal to \( 3.14 \times 10^{-3} \) cubic meters, and also equal to 0.00314 cubic meters. The specific heat at constant volume \( C_v \) is 0.633 kilojoules per kilogram per Kelvin. The molar mass \( M_g \) is 50 kilograms per kilomole.

The equation \(-c \times R\) equals \(\frac{R}{M_g}\), which calculates to \(\frac{8.314 \, \text{kilojoules per kilomole per Kelvin}}{50 \, \text{kilograms per kilomole}}\) resulting in 166.28 kilojoules per kilogram per Kelvin.

The equation \(-p \times V = R \times T\) leads to \( p \times \frac{V}{m} = R \times T \) and then to \( p = \frac{R \times T \times m}{V} \).

The text "Druckbilanz" translates to "Pressure balance".