The rate of change of entropy \( S \) with respect to time \( t \) is given by the sum over \( i \) of the mass flow rate \( \dot{m}_{i}(t) \) times the entropy \( s_{i}(t) \), plus the integral from 6 to \( a \) of the heat transfer rate \( \dot{Q} \) over the temperature \( T_{a} \), plus the generated entropy \( \dot{S}_{\text{gen}} \).

This implies that the change in entropy \( \Delta S \) is equal to the sum over \( i \) of the change in mass \( \Delta m_{i} \) times the entropy \( s_{i} \), plus the integral from \( G \) to \( a \) of the heat \( Q \) over the temperature \( T_{a} \), plus the generated entropy \( S_{\text{gen}} \).

This further implies that the change in entropy \( \Delta S \) is equal to the change in mass \( \Delta m_{12} \) times the difference in entropy between 20 degrees Celsius and 70 degrees Celsius, plus the mass \( m_{\text{gen,1}} \) times the difference in entropy between 100 degrees Celsius and 70 degrees Celsius, plus the environmental entropy \( S_{\text{env}} \).

The entropy at 20 degrees Celsius is 0.2966 kilojoules per kilogram per Kelvin.

The entropy at 70 degrees Celsius is 0.8549 kilojoules per kilogram per Kelvin.

The entropy at 100 degrees Celsius is given by \( 1.3069 + x_{i} \times (7.3549 - 1.3069) \) kilojoules per kilogram per Kelvin, which equals 1.33714.

The environmental entropy \( S_{\text{env}} \) is zero, indicating an adiabatic process.