For subtask d), the maximum heat transfer, denoted as Q_max, is calculated as the product of mass m_e and the difference between u_f at 1.45 and u_e at 1.45, which equals negative 20 kilojoules.

For subtask e), it is stated that less heat Q is required to cool the ice to zero degrees Celsius than to melt the ice, thus the temperature remains at zero degrees Celsius. The final temperature T_g2 is zero degrees Celsius, and the final pressure P_g2 equals the initial pressure P_g1, which is calculated as 1.1 times 0.15.

In the Figures and Graphs section, there is a graph described with an x-axis labeled x and a y-axis labeled y. Three points are plotted on the graph: Point 1 at coordinates (1, 1), Point 2 at coordinates (2, 4), and Point 3 at coordinates (3, 9).

In the continuation of Problem A3.5, it is mentioned that the situation is long-lasting because the expression and mass remain the same.

In subtask d) titled "X_Eis 2:", the equation involves negative m_EW times 0.6 times the sum of U_frisch at T_2 and m_EW times X_Eis2 times U_frisch at T_2, plus m_MEW times (1 minus X_Eis2) times U_flüssig at T_2 minus 0.6 times m_EW times U_flüssig at T_2, which results in Q from 1 to 2 approaching zero.

In the continuation of A2b, the equations involve Q_12 plus M_EW times 0.6 times U_est plus 0.4 times M_EW times U_crisis minus M_EW times U_est, and M_EW times U_est minus M_EW times U_crisis, which equals K_E:3 times P divided by 2. Another equation for Q_12 involves Delta M_E:i times Delta U, and negative Delta M_E:i equals C_x divided by Delta U, which results in negative 0.0058 kilograms per second, equivalent to 5.8 grams per second. The final equation involves 0.6 times 0.1 plus Delta M_E:i divided by 0.1 times U, which equals 0.5615, denoted as x_E:i:2.