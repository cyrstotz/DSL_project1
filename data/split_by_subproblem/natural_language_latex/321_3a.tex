a) The pressure \( P_{s,1} \) is equal to the sum of \( P_{ew} \), \( P_{amb} \), and \( P_u \).

The pressure \( P_{ew} \) is calculated as:
\[
P_{ew} = \frac{m_{ew} \cdot g}{\pi (0.05)^2}
\]
Substituting the values, we get:
\[
= \frac{0.2 \, \text{kg} \cdot 9.81 \, \text{m/s}^2}{\pi \cdot 0.0025 \, \text{m}^2}
\]
\[
= 124.405 \, \text{Pa}
\]

The ambient pressure \( P_{amb} \) is:
\[
P_{amb} = 1 \, \text{bar} = 100000 \, \text{Pa}
\]

The pressure \( P_u \) is calculated as:
\[
P_u = \frac{m_u \cdot g}{\pi (0.05)^2} = 39404.5 \, \text{Pa}
\]

Summing these, the pressure \( P_{s,1} \) is:
\[
P_{s,1} = 140404 \, \text{Pa} = 1.4 \, \text{bar}
\]

The product of \( P_{s,1} \) and \( V_{s,1} \) is equal to the product of \( m_{s,1} \), \( R \), and \( T_{s,1} \):
\[
P_{s,1} \cdot V_{s,1} = m_{s,1} \cdot R \cdot T_{s,1}
\]

The temperature \( T_{s,1} \) is:
\[
T_{s,1} = 500^\circ \text{C} = 773.15 \, \text{K}
\]

The gas constant \( R \) is calculated as:
\[
R = \frac{\frac{R}{M_y}}{M_y} = \frac{8.314 \, \text{J/mol} \cdot \text{K}}{50 \, \text{kg/kmol}} = 166.28 \, \text{J/kg} \cdot \text{K}
\]

The mass \( m_y \) is calculated using the ideal gas law:
\[
m_y = \frac{P_{s,1} \cdot V_{s,1}}{R \cdot T_{s,1}} = \frac{140404 \, \text{Pa} \cdot 3.14 \cdot 10^{-3} \, \text{m}^3}{166.28 \, \text{J/kg} \cdot \text{K} \cdot 773.15 \, \text{K}} \approx 3.4247 \cdot 10^{-3} \, \text{kg}
\]
\[
= 3.4247 \, \text{g}
\]