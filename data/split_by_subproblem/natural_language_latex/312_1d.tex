The change in energy, denoted as Delta E, equals the crossed out Delta U, which is equal to m2 times u2 minus m1 times u1. This is further equal to Delta m12 times h_ein,12 minus Q_aus,12.

Delta m12 is equal to the fraction of (m2 times u2 minus m1 times u1 plus Q_aus,12) divided by h_ein,12.

Continuing, the equation (m1 plus Delta m12) times U2 minus m1 times U1 equals Delta m12 times h_ein,12 minus Q_aus,12. Rearranging gives Delta m12 times (U2 minus h_ein,12) equals m1 times U1 minus m1 times U2 minus Q_aus,12. Solving for Delta m12, it is equal to the fraction of (m1 times U1 minus m1 times U2 minus Q_aus,12) divided by (U2 minus h_ein,12), where m1 equals m_gas,1 equals 3755 kilograms.

The enthalpy at entry, h_ein,12, is equal to h at 20 degrees Celsius, x equals 0, which is equal to h_f at 20 degrees Celsius, equal to 83.36 kilojoules per kilogram. The internal energy U1 is equal to U at 100 degrees Celsius, x equals 0, which is equal to U_f at 100 degrees Celsius, equal to 418.94 kilojoules per kilogram. The internal energy U2 is equal to U at 70 degrees Celsius, x equals 0, which is equal to U_f at 70 degrees Celsius, equal to 292.95 kilojoules per kilogram.

The change in enthalpy, Delta h12, is equal to U1, which is U at 100 degrees Celsius, x equals 0.005, minus U_f at 100 degrees Celsius, plus x times (U_g at 100 degrees Celsius minus U_f at 100 degrees Celsius), equal to 429.38 kilojoules per kilogram.

Finally, Delta m12 is equal to 3588.4 kilograms.