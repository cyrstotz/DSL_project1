The temperature \( T_{3/2} \) is equal to 0.003 degrees Celsius.

The factorial of \( Q_{12} \) is denoted as \( Q_{12}! \).

Under the section titled "Regarding a Gas":

The change in energy \( \Delta E \) is equal to \( E_2 - E_1 \), which is \( Q_{12} - W_L \).

The change in energy \( \Delta E \) is also equal to the sum of changes in internal energy \( \Delta U \), kinetic energy \( \Delta KE \), and potential energy \( \Delta PE \).

The change in internal energy \( \Delta U \) is equal to the mass of the gas \( m_g \) times the difference in specific internal energy \( u_2 - u_1 \), which is further calculated as \( m_g \cdot (C_V \cdot (T_2 - T_1)) \) resulting in \( 3.418 \times 10^{-3} \) kilograms times \( (0.653 \) kilojoules per kilogram Kelvin times \( (0.003 - 500)) \).

The change in internal energy \( \Delta U \) is equal to -1.082 kilojoules.

\( Q_{12} \) is equal to \( Q_{A} \).

The mass at state 2 \( m_2 \) is equal to the mass at state 1 \( m_1 \).

The product of density \( \rho \), volume \( V \), is equal to the mass \( m \) times the gas constant \( R_g \) times the temperature \( T \).

The ratio of volume at state 2 to volume at state 1 \( \frac{V_2}{V_1} \) is equal to \( \frac{m g R_g T_2}{p_1} \), calculated as \( \frac{3.418 \times 10^{-3} \times 0.3 \times 273.153 \) Kelvin divided by \( 7.4 \) bar, approximately equal to 1.103 liters.

The work done by volume change \( W_V \) is equal to the initial pressure \( p_1 \) times the change in volume \( V_2 - V_1 \), calculated as \( 7.4 \) bar times \( (1.103 - 3.141) \times 10^{-3} \) cubic meters.

The work done by volume change \( W_V \) is equal to -0.284 kilojoules.

The change in potential energy \( \Delta PE \) is equal to the mass of water \( m_{H_2O} \) times the gravitational acceleration \( g \) times the ratio \( \frac{\Delta V}{\Delta h} \), resulting in -87.47.

The sum of the change in internal energy \( \Delta U \) and the change in potential energy \( \Delta PE \) is equal to \( Q_{12} - W_L \).

\( Q_{12} \) is calculated as \( \Delta U + \Delta PE + W_L \), which equals \( -1.082 - 87.47 \times 10^{-3} - 0.284 \), resulting in 796.5.