The temperature of the reactor, denoted as \( T_{\text{reaktor}} \), is 70 degrees Celsius.

The change in mass between state 1 and state 2 is denoted as \( \Delta m_{12} \), and the temperature at state 1, \( T_1 \), is 20 degrees Celsius.

The heat transfer for the process from state 1 to state 2, denoted as \( Q_{R,12} \), is 35 megajoules.

Under the section titled "Household Equation":
The change in energy, \( \Delta E \), is equal to the change in internal energy, \( \Delta U \), which is calculated as \( m_2 u_2 - m_1 u_1 \). This should be equal to the product of the change in mass \( \Delta m_{12} \) and the specific enthalpy at the inlet \( h_{a, \text{ein}} \), plus the heat transfer \( Q_{R,12} \).

Under the section titled "Constraint":
The mass at state 2, \( m_2 \), is equal to the mass at state 1, \( m_1 \), plus the change in mass \( \Delta m \).

The specific internal energy at state 1, \( u_1 \), is calculated as the specific internal energy at the saturated liquid state \( u_f \) at 100 degrees Celsius plus a fraction \( x_0 \) of the difference between the specific internal energy at the saturated vapor state \( u_g \) and \( u_f \) at 100 degrees Celsius.

The specific internal energy at state 2, \( u_2 \), is calculated as the specific internal energy at the saturated liquid state \( u_f \) at 70 degrees Celsius plus a fraction \( x_{10} \) of the difference between the specific internal energy at the saturated vapor state \( u_g \) and \( u_f \) at 70 degrees Celsius.

The specific enthalpy at the inlet, \( h_{a, \text{ein}} \), is calculated as the specific internal energy at the saturated liquid state \( u_f \) at 20 degrees Celsius plus a fraction \( x_0 \) of the difference between the specific internal energy at the saturated vapor state \( u_g \) and \( u_f \) at 20 degrees Celsius.