The total pressure \( p_{\text{tot}} \) is given by the sum of the pressure due to the mass of the body \( M_{\text{KG}} \) times gravity \( g \) divided by the area \( A \), plus the pressure due to the mass of the extra weight \( M_{\text{EWG}} \) times gravity \( g \) divided by the area \( A \), plus the ambient pressure \( p_{\text{amb}} \).

The area \( A \) is calculated as the square of 0.105 meters times pi, which equals 0.010785 square meters.

Substituting the values, the total pressure \( p_{\text{tot}} \) is calculated as the sum of \( \frac{32 \text{ kg} \times 9.81 \text{ m/s}^2}{A} \), \( \frac{0.1 \text{ kg} \times 9.81 \text{ m/s}^2}{A} \), and 400000 Pascals, which equals 1460000 Pascals.

The relationship between pressure \( p \), volume \( V \), mass \( m \), gas constant \( R \), and temperature \( T \) is given by \( p \cdot V = m \cdot R \cdot T \).

The mass \( m \) can be calculated from \( m = \frac{p \cdot V}{R \cdot T} \).

The specific gas constant \( R \) is calculated by dividing the universal gas constant \( 8.314 \text{ J/mol K} \) by the molar mass \( 50 \text{ kg/kmol} \), resulting in \( 166.28 \text{ J/kg K} \).

Finally, substituting the values, the mass \( m \) is calculated as \( \frac{1460000 \text{ Pa} \times 0.01007 \text{ m}^3}{166.28 \text{ J/kg K} \times (500 \degree \text{C} + 273.15 \text{K})} \), which equals 0.100342 kg.