The temperature at point C, denoted as T_C, is calculated using the formula where the ratio of pressure P2 to P1 raised to the power of (n-1) divided by n, multiplied by the temperature at point S, T_S. This results in a temperature of 328.075 Kelvin.

The velocity at point C, denoted as u_C, is calculated using the formula where the ratio of volume v6 to v5 multiplied by the velocity at point S, w_S, resulting in a velocity of 437.427 meters per second.

The volume at point 5, denoted as v_5, is calculated using the formula where the gas constant R multiplied by the temperature at point 5, T_5, divided by the pressure at point 5, P_5, resulting in a volume of 0.00248 cubic meters per kilogram.

The gas constant R is calculated as the difference between the specific heat at constant pressure, c_p, and the specific heat at constant volume, c_v, which equals 0.2874.

The ratio of specific heats, denoted as kappa, is the ratio of c_p to c_v. From this, c_v is calculated as c_p divided by kappa, resulting in a value of 0.7149.

The volume at point 6, denoted as v_6, is calculated using the formula where the gas constant R multiplied by the temperature at point C, T_C, divided by the pressure at point 6, P_6, resulting in a volume of 0.0049 kilograms per cubic meter.