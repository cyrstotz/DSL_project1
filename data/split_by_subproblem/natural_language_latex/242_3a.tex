The equations and expressions are as follows:

1. The product of pressure (P) and volume (V) equals the product of mass (m), the gas constant (R), and temperature (T):
   \[
   PV = mRT
   \]
   From this, pressure (P) can be expressed as:
   \[
   P = \frac{mRT}{V}
   \]

2. The gas constant (R) is given as 8.314 Joules per mole Kelvin, and the molar mass (M) is 30 kilograms per kilomole, leading to a new expression for R:
   \[
   R = \frac{8.314 \times 10^3}{30} \text{ Joules per kilogram Kelvin} = 166.28 \text{ Joules per kilogram Kelvin}
   \]

3. Lambda (λ) is calculated as the square of the ratio of length (l) to diameter (d):
   \[
   \lambda = \left( \frac{l}{d} \right)^2 = 7.854 \times 10^{-3} \text{ square meters}
   \]

4. The question about pressure, specifically the pressure of ice (p_eis), is posed:
   \[
   \text{Pressure? Pressure } p_{eis}:
   \]
   The pressure of ice (p_eis) is calculated as:
   \[
   p_{eis} = \frac{m_{eis} g}{A} + p_0 = 1.437 + 1.58 \text{ bar} = 1.44 \text{ bar}
   \]

5. This leads to the calculation of the pressure of gas (p_gas):
   \[
   p_{gas} = \frac{[m_{eis}]}{p_{eis}} \quad \Rightarrow \quad p_g = p_{eis} + \frac{m_B g}{A} = 1.44 \text{ bar} - p_{g1}
   \]

6. Finally, the mass (m) is calculated using the ideal gas law rearranged from the first equation:
   \[
   m = \frac{PV}{RT} = \frac{p_{g1} V_{g1}}{R T_{g1}} = 3.517 \times 10^{-3} \text{ kilograms} = mg
   \]