Required: The heat transfer rate, denoted as Q dot, equals the mass flow rate, denoted as m dot, times the difference between the entropy at the inlet, denoted as s subscript ein, and the entropy at the outlet, denoted as s subscript aus, plus the ratio of the power input, denoted as P dot subscript auf, to the mass flow rate of the cooling fluid, denoted as m dot subscript KF, plus the ratio of the sectional heat transfer rate, denoted as Q dot subscript sez, to the mass flow rate of the cooling fluid, denoted as m dot subscript KF.

The sectional heat transfer rate, denoted as Q dot subscript sez, equals two times the mass flow rate, denoted as m dot, times the difference between the entropy at the outlet, denoted as s subscript aus, and the entropy at the inlet, denoted as s subscript ein, plus the ratio of the power input, denoted as P dot subscript auf, to the mass flow rate of the cooling fluid, denoted as m dot subscript KF, minus the ratio of the heat transfer rate at the reference point, denoted as Q dot subscript RP, to the mass flow rate of the cooling fluid, denoted as m dot subscript KF.

The temperature at the power input, denoted as T subscript auf, is 1103 degrees Celsius.

The temperature at the reference point, denoted as T subscript RP, is 373.15 Kelvin.

The entropy at the outlet, denoted as s subscript aus, is found in Table 1 and is 1.3069 kilojoules per kilogram Kelvin.

The entropy at the inlet, denoted as s subscript ein, is found in Table 1 and is 0.9509 kilojoules per kilogram Kelvin.

The sectional heat transfer rate, denoted as Q dot subscript sez, is 26.9 kilojoules per Kelvin.