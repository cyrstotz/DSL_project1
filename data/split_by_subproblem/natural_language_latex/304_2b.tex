Reversible adiabatic nozzle - thus isentropic with n equals 1.4. The ratio of T_0 over T_5 equals the ratio of p_0 over p_5 raised to the power of (n minus 1) divided by n. Therefore, T_6 equals T_5 times the ratio of p_6 over p_5 raised to the power of (n minus 1) divided by n. T_6 equals 431.9 times the ratio of 0.49 over 0.5 raised to the power of 0.286, which equals 328.07 Kelvin.

Stationary flow process around the nozzle. Zero equals the mass flow rate times the quantity (enthalpy at e minus enthalpy at a) plus half the difference of the squares of velocity at e and velocity at a. Therefore, zero equals h_5 minus h_6 plus half the difference of the squares of velocity at 5 and velocity at 6. Therefore, half of w_6 squared equals c_p times (T_5 minus T_6) plus half of w_5 squared. w_6 equals the square root of 2 times c_p times (T_5 minus T_6) plus w_5 squared, which equals the square root of 2 times 1.006 kilojoules per kilogram Kelvin times (431.9 Kelvin minus 328.07 Kelvin) plus 220 meters squared per second squared, resulting in 507.25 meters per second.