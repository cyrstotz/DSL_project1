In the table, the columns are labeled as follows: the first two columns are under the heading "fully condensed," the third column is labeled "saturated," and the fourth column is labeled "s.v.k." The rows represent different properties measured at various conditions: temperature in degrees Celsius, pressure in bar, entropy in kilojoules per kilogram Kelvin, and enthalpy in kilojoules per kilogram. The specific values are given for temperature and pressure at certain conditions, with temperature being 37.1 degrees Celsius and 31.33 degrees Celsius under "saturated" and "s.v.k." conditions respectively, and pressure being 3.3765 bar, 3.3755 bar, and 8 bar under different conditions.

The equations provided are as follows:
- The temperature \( T_1 \) is equal to the temperature at Suno plus 10 Kelvin, which calculates to 10 degrees Celsius.
- The temperature \( T_{\text{verd}} \) is equal to \( T_2 \), which is 10 degrees Celsius minus 6 Kelvin, resulting in 4 degrees Celsius.
- The entropy \( s_2 \) at 4 degrees Celsius is 0.9165 kilojoules per kilogram Kelvin, as referenced from Table 3-A10.
- The temperature \( T_3 \) is calculated using a formula involving the temperatures and entropies at different states, resulting in 37.1 degrees Celsius.

In the second set of equations:
- The first equation represents the first law of thermodynamics applied between states 2 and 3, equating to zero and involving mass flow rate \( \dot{m}_{12} \), enthalpy difference \( h_2 - h_3 \), and work \( \dot{W}_k \).
- The mass flow rate \( \dot{m}_{12} \) is calculated as the work divided by the enthalpy difference.
- The enthalpy \( h_2 \) at 10 degrees Celsius is given as 249.53 kilojoules per kilogram.
- The enthalpy \( h_3 \) is not provided and is indicated with ellipses.