Subsection: Reactor Water

A table with columns labeled (from left to right) with no label, 'p', 'T', and no label:
- Row 1: no value under 'p', 70 degrees Celsius under 'T', and 'x equals 1'.
- Row 2: no value under 'p', 100 degrees Celsius under 'T', and 'x equals 1'.

Equations:
1. Zero equals the mass flow rate times the difference between inlet and outlet specific enthalpy plus the heat added in the reactor minus the heat removed, expressed as:
   0 = (mass flow rate) * (specific enthalpy at inlet - specific enthalpy at outlet) + heat added in reactor - heat removed.

2. The heat removed is equal to the mass flow rate times the difference between inlet and outlet specific enthalpy plus the heat added in the reactor, expressed as:
   Heat removed = (mass flow rate) * (specific enthalpy at inlet - specific enthalpy at outlet) + heat added in reactor.

3. The specific enthalpy at the inlet at 70 degrees Celsius and x equals 1 is 2626.8 kilojoules per kilogram.

4. The specific enthalpy at the outlet at 100 degrees Celsius and x equals 1 is 2676.1 kilojoules per kilogram.

5. The heat removed is 85.2 kilowatts, noted as positive according to convention.