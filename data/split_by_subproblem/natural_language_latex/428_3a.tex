The pressure \( P_{3,1} \) is equal to the ambient pressure \( P_{\text{amb}} \) plus the force \( F \) divided by the area \( A \). This is expressed as:
\[ P_{3,1} = P_{\text{amb}} + \frac{F}{A} \]
Substituting the values, we have:
\[ P_{3,1} = P_{\text{amb}} + \frac{8 \, \text{N}}{A} \]
Further calculation with the area \( A \) being the area of a circle with radius \( 0.7 \, \text{m} \), which is:
\[ A = (0.7 \, \text{m})^2 \cdot \pi \, \text{m}^2 \]
Substituting the area, we get:
\[ P_{3,1} = 1 \, \text{bar} + \frac{8 \, \text{N}}{(0.7 \, \text{m})^2 \cdot \pi \, \text{m}^2} \]
\[ P_{3,1} = 1 \, \text{bar} + 3.595865 \, \text{Pa} \]
Rounding off, the pressure \( P_{3,1} \) is approximately:
\[ P_{3,1} \approx 1.4 \, \text{bar} \]