The heat flow out, denoted as Q dot with subscript "aus."

This is a steady flow process involving:

The mass flow rate of water, denoted as m dot with subscript "Wasser," multiplied by the difference in enthalpy at points A and 2 for water, plus the heat flow rate denoted as Q dot with subscript R, minus the heat flow out, plus the mass flow rate m dot multiplied by the difference in enthalpy between points 1 and 2.

The enthalpy at point A for water, denoted as h with subscript A,w, equals the enthalpy of fluid at 90 degrees Celsius, described as boiling liquid.

Lambda minus 2.

The enthalpy of fluid at 90 degrees Celsius is 202.98 kilojoules per kilogram.

The enthalpy of fluid with subscript f,w at 20 degrees Celsius.

The enthalpy at point 2 for water at 100 degrees Celsius equals the enthalpy of fluid with subscript f,w at 100 degrees Celsius, which is 419.07 kilojoules per kilogram.

The mass flow rate of water, denoted as m dot with subscript w, multiplied by the difference in enthalpy between points 1 and 2, plus the heat flow rate denoted as Q dot with subscript R, equals the heat flow out, which is underlined as 62.782 kilowatts.

Graph Description:

The graph is a pressure-volume (P-V) diagram. The x-axis is labeled with v (specific volume) and the y-axis is labeled with p (pressure). The graph shows a dome-shaped curve representing the phase change of a substance. The left side of the dome is labeled "boiling liquid," and the right side is labeled "wet steam." The top of the dome represents the critical point where the liquid and vapor phases are indistinguishable.

The mean temperature, denoted as T with an overline and subscript MW, equals the integral from e to a of T ds divided by the difference s_a minus s_e, implying q with subscript "aus."

The mean temperature equals the mass flow rate m dot times q with subscript "aus" divided by the difference s_a minus s_e, which equals the heat flow out divided by the product of the mass flow rate and the difference s_a minus s_e.

The difference s_e double prime minus s_a double prime equals the integral from T_a to T_e of the specific heat at constant pressure divided by T dT.

The product of q and s dot with subscript a equals the heat flow rate Q dot divided by the mass flow rate m dot.

The mean temperature equals the integral from 1 to 2 of T ds divided by the difference s_a minus s_e, which equals the integral from 1 to 2 of T ds divided by the product of the specific heat at constant pressure and the natural logarithm of T_e over T_a, implying the ratio of s over m dot.

The ratio of the heat flow rate Q dot over the mass flow rate m dot times s over m dot equals the heat flow out divided by the product of the specific heat at constant pressure and the natural logarithm of T_e over T_a.