The equations are as follows:

1. \( p g_1 = m g \)
2. The mass \( M_g \) is 50 kilograms.
3. The equation \( \frac{p V}{p R T} = m g R T g \).
4. \( p g_1 \) equals the sum of \( \frac{m K \cdot g}{A} \), \( \frac{m E W \cdot g}{A} \), and the ambient pressure \( p \text{amb} \).
5. The area \( A \) is equal to \( R^2 \pi \) which is also equal to \( D^2 \frac{\pi}{4} \).
6. \( p g_1 \) equals the sum of \( \frac{m K \cdot g}{D^2 \frac{\pi}{4}} \), \( \frac{m E W \cdot g}{D^2 \frac{\pi}{4}} \), and the ambient pressure \( p \text{amb} \), which totals to 14.0094.
7. \( p g_1 \) is 14 bar.
8. \( m g \) equals \( \frac{p_1 V_1}{R T_1} \).
9. The gas constant \( R \) is calculated as \( \frac{R}{M_g} \) which equals \( \frac{8.314 \, \frac{l}{\text{mol} \cdot K}}{50 \, \frac{\text{kg}}{\text{kmol}}} \) resulting in \( 0.16628 \, \frac{l}{\text{kg} \cdot K} \).
10. \( m g \) is 0.003422 kilograms, which is equivalent to 3.422 grams.