c)

The work done by nitrogen gas is given by the integral of pressure with respect to volume, which is calculated from initial volume \( V_1 \) to final volume \( V_2 \) with the pressure being the sum of \( P_0 \) and \( \frac{mg}{A} \). This results in \( W_{N2 \, gas} = (P_0 + \frac{mg}{A})(V_2 - V_1) \).

The heat added to nitrogen, \( Q_{N2} \), is the product of the mass of the gas \( m_g \) and the change in specific internal energy \( (\dot{u}_{2g} - \dot{u}_{1g}) \) plus the work done by nitrogen gas. This is equal to \( C_v \) times the temperature difference \( (T_2 - T_1) \) and is calculated to be \(-233.586 \, \frac{kJ}{kg}\).

The specific heat at constant volume \( C_v \) is calculated as \( \frac{8}{R} - C_p \) and is equal to \( 0.46672 \, \frac{kJ}{kgK} \), which is also expressed as \(-2 \times 165\).

Verbal description of the diagram:

There is an arrow pointing to \( V_{2g} \) with the text next to it stating:
\( V_{2g} = \frac{m_g \cdot R \cdot T_{2g}}{P_0} = 0.01106 \, m^3 \).