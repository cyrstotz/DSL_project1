- Diagram: A horizontal cylinder with a piston in the middle. To the left of the piston is the pressure \( p_{\text{amb}} \), and to the right of the piston is the pressure \( p_{\text{gas}} \). A force \( mg \) acts downward on the piston and an area \( A \) acts upward.

The equation for the gas pressure is:
\[ p_{\text{gas}} = p_{\text{amb}} + \frac{mg}{A} \]

The gas pressure is approximately:
\[ p_{\text{gas}} = 1.399 \, \text{bar} \approx 1.4 \, \text{bar} \]

For a perfect gas, the relationship is given by:
\[ p_{\text{gas}} V_6 = m_{\text{gas}} RT \]

The mass of the gas can be calculated as:
\[ m_{\text{gas}} = \frac{p_{\text{gas}} \cdot V_6}{RT} \]

The calculated mass of the gas is:
\[ m_{\text{gas}} = \frac{3.14}{3.42} = 3.42 \, \text{g} \]

The following equations are also provided:
- The change in mass times the difference in internal energy equals the heat added:
  \[ \Delta m (u_2 - u_1) = Q \]
- The change in mass times the difference in temperature equals the heat added:
  \[ \Delta m (T_2 - T_1) = Q \]
- The change in mass times the specific heat at constant volume times the temperature difference equals the heat added:
  \[ \Delta m c_v (T_2 - T_1) = Q \]
- The heat transfer at constant volume is the negative product of mass and the change in specific volume:
  \[ Q_v = -m \Delta D \]