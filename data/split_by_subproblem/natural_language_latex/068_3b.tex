Sublimation:

Equation (1) leads to equation (9) implies that p equals 2 mbar, which is equal to 2 times 10 to the power of 3 bar, which is equal to 2 times 10 to the power of 3 times 10 to the power of negative 5 bar. Negative 20 degrees Celsius equals 0.1 bar, which is true during sublimation.

From Table A-6, using interpolation:
Temperature at 0.21 kPa equals the temperature at 0.20388 kPa plus the fraction of the temperature difference between 0.0883 kPa and 0.20388 kPa over the difference between 0.1635 and 0.0883, multiplied by 0.1 kPa. The difference between 0.1 kPa and 0.0883 kPa results in negative 22 degrees Celsius, approximately equal to 20.385 degrees Celsius.

Diagram description:
- There is a horizontal line labeled "T2 equals Sub-point".
- Below this line, there is a note "20K above the sublimation point".
- An arrow points from "20K" to "20.385 degrees Celsius".
- Another horizontal line labeled "T2 equals Sub-point equals 20.385 degrees Celsius, approximately 20 degrees Celsius, when read from the diagram".

Process 1-2: Evaporation
The evaporation temperature equals the liquefaction temperature plus 6 Kelvin, which equals 26 degrees Celsius.

Steady Flow Process:
0 equals the mass flow rate times the sum of enthalpy at point 1, half the velocity squared at point 1, and gravitational potential energy at point 1, minus the sum of enthalpy at point 2, half the velocity squared at point 2, and gravitational potential energy at point 2, plus the heat transfer rate minus the work rate. The heat transfer rate to the system equals the heat transfer rate from the cooler plus the heat transfer rate from the boiler. The heat transfer rate from the cooler equals the heat transfer rate to the system minus the heat transfer rate from the boiler.

Process 3-4:
The mass flow rate times the change in enthalpy equals the heat transfer rate to the system minus the heat transfer rate from the boiler, repeated multiple times.

Equation (II) implies that the mass flow rate at the exit equals the difference between the heat transfer rate from the cooler and the work rate from the cooler divided by the difference in enthalpy between points 3 and 4, which implies that the heat transfer rate from the cooler equals the mass flow rate at the exit times the difference in enthalpy between points 3 and 4 plus the work rate from the cooler.

Equation (III) leads back to equation (I) and implies that the mass flow rate at the exit equals the mass flow rate at the exit times the ratio of the difference in enthalpy between points 3 and 4 to the difference in enthalpy between points 2 and 1, plus the ratio of the work rate from the cooler to the difference in enthalpy between points 2 and 1.

This implies that the mass flow rate at the exit times the sum of 1 plus the ratio of the difference in enthalpy between points 3 and 4 to the difference in enthalpy between points 2 and 1 equals the ratio of the work rate from the cooler to the difference in enthalpy between points 2 and 1.

This implies that the mass flow rate at the exit equals the work rate from the cooler divided by the sum of the differences in enthalpy between points 2 and 1 and points 3 and 4.

The mass flow rate at the exit equals the work rate from the cooler divided by the sum of the differences in enthalpy between points 2 and 1 and points 3 and 4, which equals 28 kilojoules per second divided by the sum of 232.62 minus 16.82 plus 93.42 minus 264.25 kilojoules per kilogram.

This results in a mass flow rate at the exit of 0.6352 kilograms per second.

From Table A-20:
At temperatures T2 and T3 equal to 260 degrees Celsius, the enthalpy hg2 equals hg3 equals 232.62 kilojoules per kilogram.

The enthalpy at point 2 equals hf equals 16.82 kilojoules per kilogram.

From Table A-12:
At pressure p3 equals 3 bar and quality x equals 0.6,
The enthalpy at point 3 equals hf equals 264.25 kilojoules per kilogram, and the difference in enthalpy hfg equals hg2 minus hf2 equals 93.42 kilojoules per kilogram.

This results in a mass flow rate at the exit of 0.6352 kilograms per second or 4 kilograms per hour.

The total mass flow rate equals 0.6352 grams per second.