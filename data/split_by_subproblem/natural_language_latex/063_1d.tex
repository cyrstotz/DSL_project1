The equation is zero equals m subscript 2 times u subscript 2 minus m subscript 1 times u subscript 1 minus m dot times h subscript ein minus Q subscript R12.

h subscript ein equals h subscript f at 200 degrees equals 83.96 kilojoules per kilogram.

Q subscript R12 equals 35 times 10 to the power of 6 Joules.

m subscript 2 equals m dot plus m subscript ges,1.

m subscript 1 equals m subscript ges,1.

The equation is zero equals m dot times u subscript 2 plus m subscript ges,1 times u subscript 2 minus m dot times u subscript ein minus Q subscript R12.

m dot times (u subscript 2 minus h subscript ein) equals m subscript ges times (u subscript 1 minus u subscript 2) minus Q subscript R12.

m dot equals the fraction of (m subscript ges times (u subscript 1 minus u subscript 2) minus Q subscript R12) divided by (u subscript 2 minus h subscript ein), which equals 330 kilograms per second.

u subscript 2 equals u subscript f at 70 degrees equals 292.35 kilojoules per kilogram.

u subscript 1 equals u subscript f at 100 degrees equals 418.9 kilojoules per kilogram.