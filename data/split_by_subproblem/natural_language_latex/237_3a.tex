The pressure \( p_{g1} \) is equal to the product of the mass of the Earth's weight \( m_{EW} \) times gravity \( g \), times the ratio \( \frac{A}{A} \), plus the mass of the kilogram \( m_{kg} \) times gravity \( g \), times the ratio \( \frac{A}{A} \), plus the ambient pressure \( p_{amb} \). This results in \( p_{g1} \) being equal to 1.4 bar.

The product of pressure \( p \) and volume \( V \) equals the product of mass \( m \), gas constant \( R \), and temperature \( T \). The mass \( m_{g1} \) is calculated by dividing the product of \( p_{g1} \) and \( V_{12} \) by the product of \( R \) and \( T_{g1} \), resulting in \( m_{g1} \) being equal to 3.49 grams.

The area \( A \) is calculated as the square of half the diameter \( D \) times pi, which equals 0.0039 grams per square meter.

The gas constant \( R \) is calculated by dividing the universal gas constant \( \bar{R} \) by the molar mass \( M_g \), resulting in \( R \) being equal to 0.76628 kilojoules per kilogram Kelvin. The conversion of 3.19 liters to cubic meters results in 0.00319 cubic meters. The temperature \( T_1 \) is given as 773.15 Kelvin.