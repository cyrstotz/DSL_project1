The subsection includes the following mathematical expressions and calculations:

1. The terms \( p_{g,1} \) and \( m_{g} \) are presented without further context or calculation.

2. Under the bold heading "SSW":
   - The equation for steel \( p_{g,1} \) is given as \( p_{amb} + g \cdot \left( \frac{m_c + m_{ew}}{A} \right) \).
   - The result of the calculation is \( 9740.194 \, \text{Pa} \) which is also converted to \( 140.115 \, \text{kPa} \).
   - The area \( A \) is calculated using the formula \( \frac{\pi \cdot d^2}{4} \) and the result is \( 0.00785 \, \text{m}^2 \).
   - The volume \( V_{g,1} \) is given as \( 0.00314 \, \text{m}^3 \).

3. Under the bold heading "IG":
   - The ideal gas law is presented as \( m \cdot R \cdot T = p \cdot V \).
   - The mass \( m \) is calculated using the formula \( \frac{p \cdot V}{R \cdot T} \) and the result is \( 0.00392 \, \text{kg} \).
   - The gas constant \( R \) is calculated using \( \frac{R_m}{m} = \frac{8.314}{50} \) and the result is \( 0.1663 \, \frac{\text{kJ}}{\text{kg} \cdot \text{K}} \).
   - The temperature \( T \) is converted from \( 500^\circ \text{C} \) to Kelvin, resulting in \( 773.15 \, \text{K} \).