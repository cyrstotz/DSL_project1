Eta equals the ratio of P over P naught.

P naught equals P four.

P four equals P one.

P naught equals P one.

Graph Description:

The graph is a plot with the vertical axis labeled eta and the horizontal axis labeled s times the fraction of t over the square root of t.

- The vertical axis eta has the following points marked:
  - At the bottom, the point is labeled as 0.
  - A point higher up is labeled as 1.
  - Another point higher up is labeled as 2.
  - The topmost point is labeled as 3.

- The horizontal axis s times the fraction of t over the square root of t has the following points marked:
  - The point at the far right is labeled as 80.

There are three curves drawn on the graph:
- The first curve starts at the origin (0,0), passes through a point labeled 6, and ends at a point labeled 1. This curve is labeled eta less than 1.
- The second curve starts at the origin (0,0), passes through a point labeled 4, and ends at a point labeled 2.
- The third curve starts at the origin (0,0), passes through a point labeled 2, and ends at a point labeled 3.

There are dashed lines extending from the end points of the curves towards the right.