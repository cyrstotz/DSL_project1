The initial pressure \( p_0 \) is calculated as the sum of the force \( F \) plus the ambient pressure \( p_{\text{amb}} \) times the mass of the solid \( m_{g,s} \) plus the ambient pressure \( p_{\text{amb}} \) times the mass of the gas \( m_{g,a} \), all divided by the area \( A \). This simplifies to the ambient pressure \( p_{\text{amb}} \) times the sum of the masses of the solid and gas, divided by the area \( A \), which equals 41.63 Bar.

The total mass \( m_g \) is the sum of the mass of the evaporated water \( m_{\text{EW}} \) and the mass of the air \( m_{\text{a}} \), which equals 32.1 kilograms.

The pressure \( p_1 \) is announced to be 1.5 Bar.

The mass of the substance \( m_s \) is calculated using the formula \( \frac{p_1 V}{R T_1} \), which results in 3.663 grams.

The gas constant \( R \) is given by \( R \) divided by the molar mass \( M \), which equals 0.1663 Joules per gram Kelvin, and this is equivalent to 166.3 Joules per kilogram Kelvin.