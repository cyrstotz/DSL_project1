The temperature T_g is 500 degrees Celsius. The volume V_g1 is 3.1 liters. The conversion of 96.2 meters per second squared equals 32.981 kilogram meters per second squared, which is equal to 343.52 Newtons.

Diagram: A schematic drawing of a cylinder with two horizontal lines dividing the cylinder into three sections. The top section is labeled with 526. An arrow points from top to bottom through the cylinder.

The pressure p_g1 and the mass m_g are unknown. The area A is equal to pi times r squared, where r is half of the diameter D, which is 0.65 meters. Therefore, A equals 7.853 times 10 to the negative third square meters. The pressure due to weight is calculated as 343.52 Newtons divided by 7.853 times 10 to the negative third square meters, resulting in 0.04 bar.

Equilibrium state (GGW) in the first chamber is 1.4 bar, so the pressure in the gas chamber must also be 1.4 bar.

The pressure p_g1 is 1.4 bar. The mass m_g is calculated using the formula m_g equals the product of pressure p_g, volume V_g, divided by the product of the gas constant R_g and the temperature T_g in Kelvin. Substituting the values, m_g equals 1.4 times 10 to the fifth Newtons per square meter times 3.1 times 10 to the negative third cubic meters, divided by 8.314 Joules per mole Kelvin times 773.15 Kelvin. This results in m_g equals 3.045 times 10 to the negative third kilograms, which is 3.045 grams.