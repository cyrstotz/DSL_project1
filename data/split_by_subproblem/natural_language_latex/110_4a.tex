The section describes a graph with the following characteristics:

- The graph is a plot with the x-axis labeled as T in degrees Kelvin and the y-axis labeled as P in bar.
- There are three distinct regions on the graph labeled as "Solid", "Liquid", and "Gas".
- The graph features a line that begins at the origin (0,0) and moves upwards to the right, labeled as "Liquid".
- Another line also starts from the origin and moves upwards to the right at a steeper angle, labeled as "Gas".
- The point where these two lines intersect is labeled as \( T_i \).
- There is a horizontal dashed line at \( P = 5 \) bar.
- The x-axis has a point labeled as \( T_i \) and another point labeled as 0.

Additionally, there are three equations or statements:
- \( T_i \) is equal to the solid phase times the derivative of enthalpy with respect to temperature (\( \frac{dH}{dT} \)).
- At \( x = 0 \), the liquid phase is equal to the gas phase.
- At \( x = 1 \), the expression \( P/E \) is questioned if it pertains to the solid phase.