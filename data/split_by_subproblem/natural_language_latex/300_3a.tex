A equals pi times r squared equals pi times d squared over 4 equals 7.85 times 10 to the power of negative 3 cubic meters.
P subscript g,1 times A equals m subscript EW times g plus 32 times g plus rho subscript amb times A.
P subscript g,1 equals m subscript EW times g over A plus 32 times g over A plus rho subscript amb.
P subscript g,1 equals 9.4 bar.

R equals R over M equals 0.16622 kilojoules per kilogram Kelvin.
m subscript g equals V subscript g,1 over v subscript g,1 equals 3.42 times 10 to the power of negative 3 kilograms.
equals 3.42 grams.
V subscript g,1 equals 3.74 liters.
P subscript g,1 times V subscript 1 equals R times T subscript 1.
V subscript 1 equals R times T subscript 1 over P subscript g,1 equals 0.918 cubic meters per kilogram.