The magnitude of \( Q_{12} \) is equal to the magnitude of the product of mass flow rate (\( \dot{m} \)), specific heat at constant volume (\( C_v \)), and the temperature difference (\( T_2 - T_1 \)), which equals the magnitude of \( 3.42 \) kilograms per second times \( 0.633 \) kilojoules per kilogram Kelvin times \( 500 \) Kelvin. This results in \( 1.08 \) kilojoules per second, which is equal to \( 1082.4 \) kilojoules per second, indicating heat outflow.

For the equation \( U_{1\text{EW}} + Q_{1\rightarrow 2} = U_{2\text{EW}} \):
- \( U_{1\text{EW}} \) is calculated as \( m \) times the sum of \( (1-x) \) times \( U_{\text{liquid}} \) plus \( x \) times \( U_{\text{solid}} \) at \( 1.4 \) bar and \( 0°C \), which equals \( 0.1 \) kilogram times \( (0.9 \times -0.045 \) kilojoules per kilogram plus \( 0.6 \times -333.458 \) kilojoules per kilogram), resulting in \( -20 \) kilojoules.
- Adding \( 1082.45 \) joules to \( -20 \) kilojoules gives \( 0.1 \) kilogram times \( (1-x) \times (0 \times -0.045 \) kilojoules per kilogram plus \( x \times -333.458 \) kilojoules per kilogram).
- \( U_2 \) is calculated as \( -18.5 \times 176 \) kilojoules per kilogram divided by \( 0.1 \) kilogram, resulting in \( 1889.176 \) kilojoules per kilogram.
- \( x_2 \) is calculated as \( (U_2 - U_f) \) divided by \( (U_{\text{solid}} - U_f) \), which equals \( (-189.176 \) kilojoules per kilogram plus \( 0.045 \) kilojoules per kilogram) divided by \( (-333.458 + 0.045 \) kilojoules per kilogram), resulting in \( 0.567 \).