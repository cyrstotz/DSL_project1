Delta s subscript e, str equals the expression in square brackets: h subscript 0 minus h subscript 6 minus T subscript 0 times (s subscript 0 minus s subscript 6) plus the fraction (w subscript 0 squared minus w subscript 6 squared) over 2, plus the expression in parentheses: 1 minus the fraction T subscript 0 over T subscript B, times the fraction Q subscript B over m dot, minus psi subscript tn.

For the Ideal Gas:

h subscript 0 minus h subscript 6 equals c subscript p times (T subscript 0 minus T subscript 6), which equals 1.006 kilojoules per kilogram Kelvin times (-30 degrees Celsius minus 328.1 Kelvin plus 273.15 Kelvin), which equals -85.46 kilojoules per kilogram.

T subscript 0 times (s subscript 0 minus s subscript 6) equals T subscript 0 times c subscript p times the natural logarithm of the fraction T subscript 0 over T subscript 6 minus R times the natural logarithm of the fraction p subscript 0 over p subscript 6, which equals 243.15 Kelvin times 1.006 kilojoules per kilogram Kelvin times the natural logarithm of the fraction 243.15 Kelvin over 328.1 Kelvin, which equals -73.3 kilojoules per kilogram.

The fraction (200 meters squared per second squared minus 507.24 meters squared per second squared) over 2 equals -108.65 kilojoules per kilogram.

The expression in parentheses: 1 minus the fraction T subscript 0 over T subscript B, times Q subscript B equals (1 minus the fraction 243.15 Kelvin over 1783 Kelvin) times 1.855 kilojoules per kilogram, equals Q subscript B times 1.580 kilojoules per kilogram.

Delta s subscript e, str equals -85.46 plus 73.3 kilojoules per kilogram minus 108.65 kilojoules per kilogram plus Q subscript B times 1.580 kilojoules per kilogram, equals -84.81.

Therefore, Delta s subscript e, str equals negative Delta s subscript 0,6 equals -84.81.