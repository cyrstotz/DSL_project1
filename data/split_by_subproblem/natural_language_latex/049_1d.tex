d) From Table A-2 we can interpolate the internal energy:

The interpolated internal energy equals 0.005 times the internal energy at 40 degrees Celsius plus 0.995 times the internal energy at 100 degrees Celsius, which equals 423.37.

Values:

The internal energy when vaporized at 70 degrees Celsius equals 252.85.

The internal energy at 20 degrees Celsius equals 83.55.

This implies that the sum of m1 times the internal energy at f1 and m12 times the internal energy at f2 equals the sum of m1 plus m12 times the internal energy when vaporized, leading to a mixture.

The sum of m1 times the internal energy at f1 and m12 times the internal energy at f2 equals m1 times the internal energy when vaporized plus m12 times the internal energy when vaporized. This implies that m1 times the difference between the internal energy at f2 and f1 equals m12 times the difference between the internal energy when vaporized and the internal energy at f1.

Thus, m12 equals m1 times the ratio of the difference between the internal energy at f2 and f1 to the difference between the internal energy when vaporized and the internal energy at f1, which equals 5491.2 kilograms.