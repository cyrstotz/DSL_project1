Verbal description of the diagram:

The diagram consists of a horizontal arrow pointing to the right, labeled "Energy balance". Above the arrow, there is a label "Q_out". Below the arrow, there is a label "T_KF". To the right of the arrow, there is a circle labeled "Q_out". From this circle, there are two arrows pointing to the right. The upper arrow is labeled "Serz" and the lower arrow is labeled "Q_out". The upper arrow points to a label "T_Reactor, a" and the lower arrow points to a label "T_Reactor, n".

The equations are as follows:

Q equals the mass flow rate of water (m_dot_w) times the difference between the enthalpy at the inlet (h_in) and the enthalpy at the outlet (h_out) plus Q_out.

Q_out equals the mass flow rate of water (m_dot_w) times the difference between the enthalpy at the outlet (h_out) and the enthalpy at the inlet (h_in).

This can be expanded to Q_out equals the mass flow rate of water (m_dot_w) times the sum of the specific heat capacity (c) times the temperature difference between the outlet and the inlet (T_out minus T_in) plus the square of the velocity (v_i squared) times the pressure difference between the outlet and the inlet (p_out minus p_in).

Serz equals the ratio of Q_out to T_KF plus the ratio of Q_R to T_Reactor, a.

This simplifies to Serz equals the ratio of Q_out to T_KF plus the ratio of Q_R to T_Reactor, n.

Finally, substituting values, Serz equals 63 kW divided by 293 K plus 100 kW divided by 373.15 K, which equals 47.64 Watts per Kelvin.