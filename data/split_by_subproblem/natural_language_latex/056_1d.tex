The mass flow rate, denoted as m dot, equals zero.

The temperature T of H2 is 70 degrees Celsius.

The temperature T of C2 is 70 degrees Celsius.

The change in mass of H2, denoted as Delta m of H2, equals the input temperature of H2, which is 70 degrees Celsius.

The heat transfer rate Q for H2, denoted as Q subscript R, H2, is approximately equal to Q and is about 35 megajoules.

In the section titled "Half-open," the equations are as follows:

The change in energy, Delta E, equals m2 times u2 minus m1 times u1, which simplifies to t times Delta m2 times h of the input plus Sigma Q of the output, all terms involving Sigma Q and Delta m2 are canceled out in subsequent steps.

Delta m2 equals the fraction where the numerator is m2 times u2 minus m1 times u1 minus Sigma Q, and the denominator is h of the input.

Delta m2 also equals the fraction where the numerator is m2 plus m1 times u2 minus m1 times u1 minus Sigma Q, and the denominator is h of the input.

The input enthalpy, h of the input, at 20 degrees Celsius is 2538.1 kilojoules per kilogram.

m1 equals 5755 kilograms.

m2 equals Delta m2 plus m1.

Sigma Q equals negative 35 megajoules, which is negative Q of the output, kt.

Delta m2 equals the fraction where the numerator is u2 plus m1 times u2 minus m1 times u1, and the denominator is h of the input minus u2 plus Q of the output.

The mass flow rate of Delta m2, denoted as m dot of Delta m2, is approximately 17.1955 kilograms, with a note indicating uncertainty or clarification needed (???).

There is a figure described on the right side of the page, consisting of several equations and values boxed together:

u2 equals u of g at 70 degrees Celsius minus 2469.6.

u1 equals u of f plus x times (u of g minus u of f) at 100 degrees Celsius.

u of f equals 468.94.

u of g equals 2506.5.

u1 equals 429.38.