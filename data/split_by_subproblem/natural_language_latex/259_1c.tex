The heat accumulation rate, denoted as Q dot with subscript 'acc', equals 65 kilowatts, and the average temperature of the cooling fluid, denoted as T bar with subscript 'KF', equals 293 Kelvin.

The equation is zero equals the mass flow rate, denoted as m dot, times the difference between the entropy at the exit, denoted as S subscript e, and the entropy at the entrance, denoted as S subscript a, plus the heat flow rate, denoted as Q dot, divided by the average temperature, denoted as T bar, plus the generated entropy rate, denoted as S dot with subscript 'erz'. This represents the balance in the reactor.

The generated entropy rate, denoted as S dot with subscript 'erz', equals the mass flow rate, denoted as m dot, times the difference between the entropy at the entrance, denoted as S subscript a, and the entropy at the exit, denoted as S subscript e, minus the heat accumulation rate, denoted as Q dot with subscript 'acc', divided by the average temperature of the reactor, denoted as T bar with subscript 'Reaktor'.

The average temperature of the reactor, denoted as T bar with subscript 'Reaktor', equals 100 degrees Celsius, which is 373.15 Kelvin.

The entropy at the entrance, denoted as S subscript a, is calculated as 1.3 (with the last two digits crossed out) plus 0.005 times (7.35 (with the last two digits crossed out) minus 1.3069), which equals 1.34 kilojoules per kilogram Kelvin.

The entropy at the exit, denoted as S subscript e, is calculated as 0.95 (with the last two digits crossed out) plus 0.005 times (7.75 (with the last two digits crossed out) minus 0.95 (with the last two digits crossed out)), which equals 0.9 kilojoules per kilogram Kelvin.

The generated entropy rate, denoted as S dot with subscript 'erz', equals 0.3 times the difference between 1.34 and 0.9 minus 65 kilowatts divided by 373.15 Kelvin, which equals approximately 0. (with the last digit crossed out) 29 kilowatts per kilogram Kelvin.