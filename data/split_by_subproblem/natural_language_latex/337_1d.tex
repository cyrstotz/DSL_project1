The change in energy, denoted as Delta E, is equal to the heat added, Q, minus the work done by the system, W_v. This is expressed as the difference between the product of mass and internal energy at state 2 and state 1, mathematically represented as:

Delta E equals Q minus W_v equals m_2 times u_2 minus m_1 times u_1.

In the figure description, there is a depiction of a square with an arrow pointing to the right, which is labeled with:

Q_R equals 35 MegaJoules.

The equation provided next relates the change in internal energy due to the change in mass and the addition of heat at a specific temperature:

m_2 times u_2 minus m_1 times u_1 equals Delta m times h_f at 20 degrees Celsius plus Q_R.

It is also stated that the mass at state 2 is equal to the mass at state 1 plus the change in mass:

m_2 equals m_1 plus Delta m.

Lastly, at a condition marked with an '@' symbol, when x_2 equals 0 at 70 degrees Celsius, the internal energy u_4 at 70 degrees Celsius is given as 292.95 kiloJoules per kilogram, referenced from a source denoted as T & B number 2.