T sub 5 equals 431.9 Kelvin, p sub 5 equals 0.5 bar, w sub 5 equals 720 meters per second. The process from 5 to 6 is adiabatically reversible, hence isentropic with k equals 1.4 and p sub 0 equals p sub 6. T sub 6 is calculated using the polytropic relation, T sub 6 equals (p sub 6 over p sub 5) raised to the power of (k minus 1) over k, times T sub 5, which equals 328.1 Kelvin.

For the thrust nozzle: W equals 0, Q equals 0, and it is adiabatic. For the stationary condition: 0 equals h sub e minus h sub a plus (w sub e squared minus w sub a squared) over 2. Simplifying further, 0 equals h sub e minus h sub a plus w sub e squared over 2 minus w sub a squared over 2. Thus, w sub 6 squared over 2 equals h sub e minus h sub a plus w sub e squared over 2. Solving for w sub 6, w sub 6 equals the square root of 2 times (h sub e minus h sub a) plus w sub e squared, approximately equals 507.2 meters per second.

Additional Notes:
- c sub p is constant!
- c sub p ideal equals 1006 Joules per kilogram Kelvin.
- c sub p times (T sub 5 minus T sub 6) equals 104432 Joules per kilogram.