Phi equals P subscript g1 times V subscript g1, which equals M subscript g1 times R times T subscript g1.

P subscript g1 equals 1 bar plus M subscript E times g divided by A plus M subscript L times g divided by A, which equals 100000 Pascals plus 0.4 times 9.81 meters per second squared divided by 25 square centimeters plus 32 times 9.81 meters per second squared divided by 25 square meters per second squared.

P subscript g1 equals 1.40094 bar.

M subscript g1 equals P subscript g1 times V subscript g1 divided by R times T subscript g1, which equals 0.008422 kilograms.

R equals R divided by M.

V subscript g1 equals 3.74 liters, which equals 0.00374 cubic meters.

U subscript 2 equals U subscript fest plus x subscript 2,1 times (U subscript flussig minus U subscript fest).