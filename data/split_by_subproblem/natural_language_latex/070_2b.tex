b) What are v_6 and T_6?

Given values:
- v_5 equals 220 meters per second,
- T_5 equals 437.5 Kelvin,
- p_5 equals 0.5 bar,
- p_6 equals p_0 equals 0.197 bar,
- The process is isentropic, meaning s_5 equals s_6.

This is a stationary process, described by the equation:
0 equals the mass flow rate times (h_5 minus h_6 plus half the difference of the squares of w_5 and w_6) plus heat transfer rate minus work rate.

The process is adiabatic and reversible, leading to:
h_5 minus h_6 plus half the difference of the squares of w_5 and w_6 equals zero.

From this, we derive:
h_5 minus h_6 equals c_p times (T_5 minus T_6) equals 146,2286 kilojoules per kilogram.

The specific heat at constant pressure c_p is given as:
c_p equals 1 kilojoule per kilogram Kelvin, and it equals n times c_v, which is 1.4089 kilojoules per kilogram Kelvin.

Thus, the gas constant R is:
R equals c_p minus c_v equals 0.4024 kilojoules per kilogram Kelvin.

Reiterating the isentropic condition:
s_5 equals s_6.

This leads to the equation:
0 equals c_p times the natural logarithm of (T_6 over T_5) minus R times the natural logarithm of (p_6 over p_5).

Solving for the ratio of pressures and temperatures:
R times the natural logarithm of (p_6 over p_5) equals c_p times the natural logarithm of (T_6 over T_5).

Exponentiating both sides:
(p_6 over p_5) raised to the power R equals (T_6 over T_5) raised to the power c_p, leading to T_6 equals T_5 times (p_6 over p_5) raised to the power (R over c_p) equals 328.075 Kelvin.

For the velocity w_6:
w_6 squared equals w_5 squared plus 2 times (h_5 minus h_6).

Calculating w_6:
w_6 equals the square root of (w_5 squared plus 2 times 146,2286 kilojoules per kilogram).

Converting and calculating:
w_6 equals the square root of (220 squared plus 2 times 146,2286 times 1000) meters per second,
which equals 1729.93 meters per second.