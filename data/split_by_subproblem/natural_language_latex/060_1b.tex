The expression for \( T_{KEF} \) is given by the integral of \( \frac{\delta Q}{T} \), which equals the integral of \( \frac{\dot{Q}}{T} dt \).

It is stated that \( \frac{\delta Q}{T} \) equals \( \frac{\dot{Q}}{T} dt \), which implies \( \frac{\delta Q}{T} \) equals \( T_{KEF} \).

The integral from \( T_1 \) to \( T_2 \) of \( \frac{\delta Q}{T} \) equals the natural logarithm of \( \frac{T_2}{T_1} \), which further simplifies to the natural logarithm of \( \frac{298.15}{298.15} \) and results in 0.0347.

This leads to the conclusion that \( T_{KEF} \) equals the integral of \( \frac{\delta Q}{T} \), which simplifies to \( \frac{\delta Q}{0.3674} \) and results in 0.2245.

Finally, \( T_{KEF} \) is calculated as \( \frac{\delta Q}{0.3674} \), which simplifies further to \( \frac{\delta Q}{0.1425} \) twice, resulting in a final temperature of 373.86 Kelvin.