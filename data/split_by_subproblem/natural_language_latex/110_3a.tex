a) The area A is calculated as pi times the radius squared, which equals pi times the square of half the diameter. Substituting the diameter as 0.111 meters, the area A is calculated to be 0.00785 square meters.

In the diagram:
- There is a horizontal line labeled "po" with two downward arrows, one labeled "mkg" and the other labeled "mewg".
- Below this, there is another horizontal line labeled "D = 10 cm".

For the equilibrium condition, the equation is set up as the sum of the forces due to pressure (p0 times A), the weight of the mass in kilograms (mkg times g), and the weight of the mass in equivalent weight (mew times g) equals the force due to gauge pressure (pg times A). Substituting the values, the gauge pressure pg is calculated to be 128853.00737 Pascals, which is equivalent to 1.2885 bar.

To find the mass mg, the ideal gas equation pV equals mRT is used. The gas constant R is calculated by dividing the universal gas constant R by the molar mass mu_g, giving R as 0.16628 Joules per gram Kelvin. Substituting the values for pressure pg, volume Vg, gas constant Rg, and temperature Tg, the mass mg is calculated to be 3.4971 grams.