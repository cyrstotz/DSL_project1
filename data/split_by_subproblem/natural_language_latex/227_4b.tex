b)

The derivative of a function with respect to time t is equal to the sum of the mass flow rate of each component i, multiplied by the sum of the enthalpy of component i, plus half the square of the velocity of component i, plus the product of the gravitational acceleration and the height of component i, plus the heat transfer rate, minus the work rate.

Q equals the mass flow rate times the difference in enthalpy between the exit and the entrance, plus the heat transfer rate, minus the work rate.

The mass flow rate is equal to the difference between the work rate and the heat transfer rate, divided by the difference in enthalpy between the exit and the entrance.

A table with columns labeled density (rho), temperature (T), and quality (x), and rows for values:
- Row 1: density 1, temperature 0, quality 0
- Row 2: density 2, temperature 1, quality 1
- Row 3: density 3, temperature 8 bar, quality 1
- Row 4: density 4, temperature 8 bar, quality 0

Heat exchanger (translated from German "Wärmeübertrager")

From state 1 to state 2, the mass flow rate times the difference in enthalpy between the exit and the entrance, plus the heat transfer rate equals zero, which implies that the heat transfer rate equals the mass flow rate times the difference in enthalpy between the entrance and the exit.