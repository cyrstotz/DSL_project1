Pressure \( p_{g,1} \) and mass of gas \( m_g \):

1. EV Piston Atmos:
   - Down arrow, down arrow, down arrow
   - \( p_{g,1} \)

2. The equation \( p_{g,1} \times A_{zy1} = m_{ev} \times g + m_{k} \times g + p_{atmos} \times A_{zy1} \).

3. The area \( A_{zy1} \) is calculated as \( \pi r^2 = \pi (0.05 \, \text{m})^2 = \pi \times 7.853982 \times 10^{-3} \, \text{m}^2 \).

4. The pressure \( p_{g,1} \) is given by \( \frac{m_{ev} \times g + m_k \times g}{A_{zy1}} + p_{atmos} \).

5. Substituting the values, \( p_{g,1} = \frac{0.1 \, \text{kg} \times 9.81 \, \text{m/s}^2 + 32 \times 9.81 \, \text{m/s}^2}{7.853982 \times 10^{-3} \, \text{m}^2} + 10^5 \, \text{Pa} \).

6. The calculated pressure \( p_{g,1} \) is \( 140094.4137 \, \text{Pa} \).

7. Using the ideal gas law \( pV = mRT \) where \( T_1 = 500^\circ \text{C} = 773.15 \, \text{K} \).

8. The pressure \( p_1 = 140000 \, \text{Pa} \) and volume \( V_1 = 3.141 \times 0.0037 \, \text{m}^3 \).

9. The mass of gas \( m_g \) is calculated using \( m_g = \frac{p_1 V_1}{RT_1} \).

10. The gas constant \( R \) is \( \frac{R}{M} = \frac{8314 \, \text{J/(kmol K)}}{56 \, \text{kg/kmol}} = 0.16628 \, \text{KJ/(kg K)} \).

11. Substituting the values, the mass of gas \( m_g \) is \( \frac{0.0037 \times 140000}{0.16628 \times 773.15} = 0.2922 \, \text{kg} \).