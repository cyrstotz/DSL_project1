The pressure inside the patient (p_pa) is equal to the sum of the ambient pressure (p_amb), the pressure due to the column of liquid (p_k), and the pressure due to the weight of the equipment (p_EW). The ambient pressure (p_amb) is given as 1 bar. The pressure due to the column of liquid (p_k) is calculated using the formula p_k equals the mass of the column (m_k) times gravity (g) divided by the area, which is the radius squared times pi. Substituting the values, the mass of the column is 32 kilograms, gravity is 9.81 meters per second squared, and the diameter (d) is 0.10 meters, the calculated pressure p_k is 35969 Pascals or 0.400 bar. The pressure due to the weight of the equipment (p_EW) is calculated similarly and is found to be 124.9 Pascals or 0.00125 bar. Therefore, the total pressure inside the patient (p_pa) is 1.401 bar. 

The mass of the gas (m_g) is calculated using the formula m_g equals the pressure inside the patient (p_pa) times the volume (V_1) divided by the gas constant (R) times the temperature (T_1). Substituting the values, the mass of the gas is found to be 0.00342 kilograms or 3.42 grams.

The gas constant (R) specific to the gas is calculated by dividing the universal gas constant (8.314 Joules per mole Kelvin) by the molar mass (mu), which is 50 kilograms per kilomole, resulting in a specific gas constant of 166.28 Joules per kilogram Kelvin.