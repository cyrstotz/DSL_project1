d)

The value of \( x_{Eis,1} \) is 0.6.

The internal energy \( U_1 \) is calculated as:
0.6 times negative 333.658 plus (1 minus 0.6) times negative 0.005 kilojoules per kilogram,
which equals negative 200.1 kilojoules per kilogram.

The change in internal energy for ice, \( \Delta U_{Eis} \), is given by the heat \( Q_{12} \) divided by the mass of the ice-water mixture \( m_{EW} \).

Continued:

The output energy \( Ou_{E,5} \) is equal to the heat \( Q_{12} \) divided by the mass of the ice-water mixture \( m_{EW} \), which results in 13.676 Joules per gram.

The output energy \( Ou \) is the difference between \( U_2 \) and \( U_{12} \), calculated as 13.676 Joules per gram.

The internal energy \( U_{12} \) is calculated as 13.676 plus negative 200.7 Joules per gram, resulting in negative 186.469 kilojoules per kilogram.

The fraction \( X_{E,5,2} \) is the ratio of the difference between \( U_2 \) and \( U_{12} \) to the difference between the solid state internal energy \( U_{fest} \) and the liquid state internal energy \( U_{flüssig} \).

The equation \( X_{E,5,2} \times U_{fest} + (1 - X_{E,5,2}) \times U_{flüssig} = U_2 \) represents a balance of internal energies.

The values for \( U_{fest} \) and \( U_{flüssig} \) are still at a pressure of 1.8 bar since the process is isobaric.

Finally, \( X_{E,5,2} \) is calculated as the ratio of \( U_2 - U_{flüssig} \) to \( U_{fest} - U_{flüssig} \), resulting in 0.0552.