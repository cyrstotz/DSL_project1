The equation is m subscript 2 times u subscript 2 minus m subscript 1 times u subscript 1 equals delta m times h subscript zu.

Table A2 lists u subscript 2, u subscript 1f, u subscript 1g, and h subscript zu.

The values are given as follows:
- m subscript 2 equals 5755 kilograms,
- u subscript 2 equals 292.85 kilojoules per kilogram,
- m subscript 0 equals 5755 times 0.005 kilograms equals 28.775 kilograms,
- u subscript 1g equals 2506.5 kilojoules per kilogram,
- m subscript f equals 5755 times (1 minus 0.005) kilograms equals 5726.23 kilograms,
- u subscript 1f equals 418.94 kilojoules per kilogram,
- h subscript zu equals 83.96 kilojoules per kilogram.

The equation m subscript 2 times u subscript 2 minus m subscript 0 times u subscript 1g minus m subscript f times u subscript 1f equals delta m times h subscript zu.

Delta m is calculated as (m subscript 2 times u subscript 2 minus m subscript 0 times u subscript 1g minus m subscript f times u subscript 1f) divided by h subscript zu.

This results in (5755 kilograms times 292.85 kilojoules per kilogram minus 28.775 kilograms times 2506.5 kilojoules per kilogram minus 5726.23 kilograms times 418.94 kilojoules per kilogram) divided by 83.96 kilojoules per kilogram.

The final result is negative 9351.40 kilograms.