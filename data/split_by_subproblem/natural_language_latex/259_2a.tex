The graph description details a plot with a horizontal axis and a vertical axis, where the horizontal axis has an arrow pointing to the right and the vertical axis has an arrow pointing upwards. There are six points labeled from \( P1 \) to \( P6 \). Points \( P2 \) and \( P3 \) are on the same vertical level. Point \( P4 \) is positioned above \( P3 \), and \( P5 \) is above \( P4 \). Point \( P6 \) is located below \( P5 \) and to the right of \( P1 \). A curved line connects \( P1 \) to \( P6 \), passing through \( P2 \), \( P3 \), \( P4 \), and \( P5 \). Additionally, there is a straight line labeled "guideline" connecting \( P1 \) to \( P6 \). The area between the curved line and the straight line is shaded.

The equation \( E_{1, \text{real}} = T_0 \cdot \dot{S}_{e12} \) is given.

The equation \( 0 = m \left[ S_e - S_a \right] + \frac{Q_B}{T_B} + \int \dot{S}_{e12} \) is also provided.

It is followed by \( \frac{\dot{S}_{e12}}{m} = S_a - S_e - \frac{Q_B}{T_B} = \int_{T_2}^{T_3} \frac{c_p}{T} dT - R \ln \left( \frac{P_B^2}{P_2} \right) \).

This simplifies to \( c_p \ln \frac{T_3}{T_B} - \frac{Q_B}{T_B} = -1295.81 \frac{kJ}{K} \).

Finally, \( \dot{E}_{1, \text{real}} = 243.13 \frac{kJ}{K} \cdot -1295.81 \frac{kJ}{K} = 315076.20 \text{ kJ} \) is calculated.