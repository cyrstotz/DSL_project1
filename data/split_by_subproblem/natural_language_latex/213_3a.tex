The gas constant \( R_g \) is equal to the ratio of \( R \) over \( M_g \), which equals 166.289 joules per kilogram Kelvin.

The area \( A \) is equal to pi times the square of the radius \( \frac{D}{2} \), which simplifies to pi times \( \frac{D^2}{4} \), resulting in 78.5338 square centimeters, which is equivalent to 78.5338 times \( 10^{-4} \) square meters.

Force equilibrium:

The sum of the product of the mass of the evaporator \( m_{EW} \) and the mass of the condenser \( m_K \) with gravity \( g \), plus the product of the initial pressure \( p_0 \) and area \( A \), equals the product of the superheated vapor pressure \( p_{SU} \) and area \( A \).

Dividing the sum of the product of the mass of the evaporator and the mass of the condenser with gravity by area \( A \), plus the initial pressure \( p_0 \), results in the pressure \( p_{g,1} \), leading to the equation \( \frac{32.1 + 9.31}{A} + 1 \) bar equals 1.40094 bar.

Ideal Gas Law:

The product of the superheated vapor pressure \( p_{SU} \) and its volume \( V_{SU} \) equals the product of the mass of the gas \( m_g \), the gas constant \( R \), and the temperature \( T_{g,1} \).

Dividing the product of the superheated vapor pressure and its volume by the product of the gas constant and the temperature of the superheated vapor \( T_{SU} \) gives the mass of the gas \( m_g \), which is 3.4215 grams.