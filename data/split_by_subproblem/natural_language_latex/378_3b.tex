The equations and descriptions are as follows:

1. The set of equations starts with the derivative of energy with respect to time, represented as dE/dt, equals the sum of heat flow denoted by dot Q with some parameters, plus a constant heat flow dot Q2 with a superscript zero, minus the work done which is represented by dot W subscript u.

2. This implies that the change in energy, Delta E, equals the change in internal energy, Delta U, which equals negative work done, -Wu.

3. The derivative of energy with respect to time, dE/dt, is equal to a linear function, denoted by 'lin'.

4. The equation p subscript g12 minus p subscript Ev12 is also presented.

Graphical Description:
- A wavy line oscillates across the page from left to right, representing a function or process over time.
- Annotations along the wavy line indicate changes in energy or other quantities, specifically Delta E, Delta U, and Delta E subscript int equals 'cif'.
- A vertical double line to the right of the wavy line possibly represents a boundary or separation between different regions or phases.
- The phrase "for gas" is written next to the vertical double line, indicating the context of gas for the equations or processes described.

Student Solution:
- The equation p equals constant implies that T subscript 3,2 equals T subscript 3,1 times the ratio of V subscript 1g over V subscript 1f.
- The derivative with respect to time of V subscript 1g equals negative W subscript v divided by rho plus V subscript r.
- The derivative of energy with respect to time, dE/dt, equals dot m with some parameters plus dot Q subscript in minus dot W subscript v.
- The change in internal energy, Delta u subscript in, equals dot Q subscript in minus dot W subscript v, and similarly, Delta u subscript ing equals dot Q subscript in minus dot W subscript v, which implies that dot W subscript v equals an unspecified value.