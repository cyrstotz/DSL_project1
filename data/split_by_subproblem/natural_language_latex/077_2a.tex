The list describes different stages of a process:
- From point 1 to 2, the process is isentropic, meaning entropy (s) is constant, and pressure (p) decreases.
- From point 2 to 3, the process is isobaric, meaning pressure (p) is constant, and temperature (T) increases.
- From point 3 to 4, the process is not isentropic.
- From point 4 to 5, the process is isobaric, meaning pressure (p) is constant.
- From point 5 to 6, the process is isentropic, meaning entropy (s) is constant.

Graph Description:
The main graph is a plot with temperature (T) on the vertical axis and entropy (S) on the horizontal axis. The vertical axis is labeled with T and has an upward arrow. The horizontal axis is labeled with S and has a rightward arrow. The graph includes:
- A horizontal line at the bottom labeled minus 30 degrees Celsius.
- A horizontal line in the middle labeled "isentrop".
- A diagonal line labeled "isotherm".
- Several diagonal lines labeled p1, p2, p3, and p4.
Points are marked and connected by lines in the order from 1 to 6, with specific positions described for each point. There is also a smaller inset graph to the right of the main graph showing a simplified version of the main graph.

For the equations and calculations:
- The work done by gas (w_G) and the temperature at point c (T_c) are to be determined.
- The ratio of temperature at point c to temperature at point 5 is given by a power function of the ratio of pressure at point c to pressure at point 5 with an exponent of 0.286.
- The pressure at point 5 is 0.5 bar, and the temperature at point 5 is 431.9 Kelvin.
- The pressure at point c equals the initial pressure (p_0), which is 0.191 bar.
- The temperature at point c (T_c) is calculated to be approximately 328 Kelvin.
- The work done by gas (w_G) involves an integral of specific volume (v) over pressure change (dp) plus a change in enthalpy (Δh_e).
- The energy balance for a stationary process (likely referring to a nozzle) is given, involving mass flow rate (ṁ), enthalpy change (h_e - h_a), and a term involving the square of velocities (w_5^2 - w_6^2).
- The final velocity (w_G) is calculated using the square root of twice the enthalpy difference between points 5 and 6, and a term involving the square of velocity at point 5.
- The enthalpy difference (h_5 - h_6) is calculated using the specific heat at constant pressure (C_p) and the temperature difference (T_2 - T_1), with a specific heat value of 1.006.