Delta q subscript is, shr equals h subscript 6 minus h subscript 0 minus T subscript 0 times (s subscript 6 minus s subscript 0) plus k times e subscript 6 minus k times e subscript 0.

h subscript 6 minus h subscript 0 equals c subscript pL times (T subscript 6 minus T subscript 0) equals 85.43 kilojoules per kilogram.

s subscript 6 minus s subscript 0 equals c subscript pL times the natural logarithm of (T subscript 6 divided by T subscript 0) minus R times the natural logarithm of (P subscript 6 divided by P subscript 0).

R equals c subscript pL minus (c subscript pL divided by k).

This equals 0.3013 kilojoules per kilogram Kelvin.

Table:
The table has columns labeled P, T, and W. The rows are labeled with numbers and terms such as isentropic, isobaric, and adiabatic. Specific values are given for P, T, and W at different states, such as:
- At state 0, P equals 0.121, T equals 293.15, and W equals 200.
- At state 5, P equals 0.5, T equals 431.9, and W equals 200.
- At state 6, P equals 0.121, and W equals 328.07.

S dot subscript ex, str equals m dot subscript 6 times h subscript 6 minus m dot subscript 5 times (s subscript 5 minus s subscript 0) plus one-half times m dot subscript 6 times w subscript 6 squared minus one-half times m dot subscript 0 times w subscript 0 squared.

This equals 40.673 kilojoules per kilogram.