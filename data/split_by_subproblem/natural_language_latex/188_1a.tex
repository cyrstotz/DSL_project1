The section is titled "Required: Q dot out."

The first equation is:
Zero equals the mass flow rate times the difference between the enthalpy at the inlet and the enthalpy at the outlet, plus Q dot R, plus Q dot out.

The second equation is:
Negative Q dot out equals the mass flow rate times the difference between the enthalpy at the inlet and the enthalpy at the outlet, plus Q dot R. This is labeled as equation 1 minus Q dot out.

The third equation is:
Q dot out equals the mass flow rate times the difference between the enthalpy at the outlet and the enthalpy at the inlet, minus Q dot R.

The values given are:
The mass flow rate is 0.3 kilograms per second, the enthalpy at the inlet is 282.98 kilojoules per kilogram, the enthalpy at the outlet is 419.04 kilojoules per kilogram, and Q dot R is 100 kilojoules per second.

Substituting these values into the equation for Q dot out gives:
Q dot out equals 0.3 kilograms per second times (419.04 kilojoules per kilogram minus 282.98 kilojoules per kilogram) minus 100 kilojoules per second.

This results in:
Negative 62.16 kilojoules per second.

Finally, it states that since the mass flow rate is increasing, Q dot out is positive 62.16 kilojoules per second.