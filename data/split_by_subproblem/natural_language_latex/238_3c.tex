c) To solve this task, we need the energy balance equation:
The derivative of E with respect to t equals the sum over i of m dot i times the quantity h i plus c i squared over 2 plus g z i over 2, plus the sum over j of Q dot j, minus the sum over k of W dot k.

System:

Description of the diagram: The diagram shows a rectangular box with a horizontal line at the top representing the system boundary. Inside the box, there is a small circle at the top center with a vertical line extending downwards, representing a piston or similar mechanism. Below the piston, there are two small rectangles side by side, representing different phases or components within the system.

m u1 minus m u2 equals Q u2.

u1 equals u liquid at 1 bar plus x ice,1 times the quantity u solid at 1 bar minus u liquid at 1 bar.

u liquid at 1 bar equals negative 0.045 kilojoules per kilogram.

u solid at 1 bar equals 333.458 kilojoules per kilogram.

u1 equals negative 200.0928 kilojoules per kilogram.

u2 equals

m times the quantity u2 minus u1 equals Q dot (according to the energy balance; solved with gas system).

u2 gas minus u1 equals c v times the quantity T2 minus T1 equals c v times the quantity 0 degrees Celsius minus 500 degrees Celsius equals 0.632 kilojoules per kilogram Kelvin times negative 500 Kelvin equals negative 331.5 kilojoules per kilogram.

m dot gas times the quantity u2 minus u1 equals Q.

Q dot equals negative 1.134 kilojoules (this much heat is removed from the gas).