The total pressure \( P_{ag} \) is equal to the sum of the ambient pressure \( P_{amb} \), the pressure due to the mechanical component \( P_{mk} \), and the pressure due to the external weight \( P_{ew} \), where the ambient pressure \( P_{amb} \) is 1 bar.

The pressure \( P_{mk} \) is calculated as the force per unit area, with the force being the product of mass (32 kg) and acceleration due to gravity (9.81 m/s²), and the area being the area of a circle with radius 0.05 m. This calculation results in \( P_{mk} \) being 39965 Pascals.

Similarly, the pressure \( P_{ew} \) is calculated using the mass of 0.1 kg and the same acceleration and area as before, resulting in \( P_{ew} \) being 125 Pascals.

Thus, the total pressure \( P_{ag} \) is the sum of 100000 Pascals, 39965 Pascals, and 125 Pascals, which equals 140090 Pascals or 1.4 bar.

The mass of the gas \( m_g \) is calculated using the ideal gas law rearranged in terms of mass, where \( P_{ag} \), \( V_g \), \( R \), and \( T_{ag} \) are the pressure, volume, specific gas constant, and temperature of the gas respectively. The specific gas constant \( R \) is calculated by dividing the universal gas constant (8.3145 Joules per mole Kelvin) by the molar mass of the gas (50 grams per mole), resulting in 166.3 Joules per kilogram Kelvin.

Finally, substituting the values into the mass formula, the mass of the gas \( m_g \) is calculated to be 3.4 grams.