The first graph is a pressure-volume diagram with several curves. These curves start at a high pressure and low volume, then decrease in pressure as the volume increases, forming a series of downward-sloping curves.

The second graph is a pressure-temperature diagram. The y-axis is labeled with pressure and the x-axis is labeled with temperature. There are two curves: one labeled as critical temperature and another labeled as triple point temperature. The curve for the critical temperature is higher and more to the right than the curve for the triple point temperature. The x-axis has markings at negative 10 degrees Celsius, 0 degrees Celsius, 10 degrees Celsius, and 20 degrees Celsius. There is a point labeled X on the triple point temperature curve.

T1 equals negative 20 degrees Celsius, which is 253.15 Kelvin.