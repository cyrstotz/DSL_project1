Delta F equals the sum of m sub i times h sub i plus one half Q sub i, equals m sub 2 times d sub 2 minus m sub 1 times d sub 1, equals (m sub 1 plus Delta m sub 12) times d sub 2 minus m sub 1 times d sub 1.

m sub 2 times d sub 2 plus Delta m sub 12 times d sub 2 minus m sub 1 times d sub 1 equals Delta m sub 12 times h sub 12 plus Q sub R sub 12.

Delta m sub 12 equals m sub 2 times d sub 2 minus m sub 1 times d sub 1 minus Q sub E sub 12 divided by h sub 12.

u sub 1 equals 418.94 plus 0.005 times (2566.5 kilojoules per kilogram minus 418.94) equals 429.38 kilojoules per kilogram.

a sub 2 equals u of T equals 20 degrees Celsius equals 292.95 kilojoules per kilogram.

h sub 12 equals h sub p at 20 degrees Celsius equals 83.96 kilojoules per kilogram.

Delta m sub 12 equals m sub 1 times (5735 kilograms times 292.95 kilojoules per kilogram minus 5735 kilograms times 429.38 kilojoules per kilogram minus 35 times 10 to the power of 6 kilojoules) divided by (83.96 kilojoules per kilogram minus 292.95 kilojoules per kilogram).

This equals 33.46 kilograms.