1.1.45, step 2 to 3:
Q equals h sub 3 minus (h sub 2 minus h sub 1) plus dot W.
The ratio of dot m over dot m equals the ratio of dot V sub k over dot V sub k.
1.1.45, step 2 to 3:
Q equals dot m times (h sub 2 minus h sub 1) plus dot V sub k.
The ratio of dot m over dot m equals the ratio of dot V sub k over (h sub 3 minus h sub 2).

s sub 1 equals s sub 3, s sub 2 equals s sub 3.
T sub 1 equals 20 degrees Celsius which is 293.15 Kelvin.
s sub 3 equals 20 plus 4 plus 0.1 times 10.
h sub 2 equals 20 plus 4 times 10.
h sub 2,0 equals 20 plus c.