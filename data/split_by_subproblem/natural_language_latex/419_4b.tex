The rate of change of V, denoted as d dot V, is described as strong.

The rate of change of energy E with respect to time t is equal to the sum of kinetic energy, heat flow rate denoted as dot Q, and work rate denoted as dot W.

Zero equals the mass flow rate dot m times the difference in enthalpy h2 minus h3 plus the work rate dot W.

The difference in enthalpy h2 minus h3 is a function of temperature T2.

The temperature T2 is minus 6 degrees Celsius.

The value of mu3 is 813.

The entropy s2 equals s3, both having a value of 100 kilojoules per kilogram Kelvin.

The logarithm of 4 equals 244.9 kilojoules per kilogram.

The logarithm of 8 equals 292.54 kilojoules per kilogram.

The enthalpy h2 is calculated as 292.54 plus the fraction (8 plus 8) divided by -4.8 times the difference (244.9 minus 292.54), resulting in 243.72 kilojoules per kilogram.

The entropy s2 is calculated as 0.925 kilojoules per kilogram Kelvin plus the fraction (8 minus 6) divided by (8 minus 4) times the difference (0.925 minus 0.9239), resulting in 0.9226 kilojoules per kilogram Kelvin.

For h3 at 8 bar:

Using TAB A-12 interpolation,

The entropy s3 is 0.9226 kilojoules per kilogram Kelvin.

The saturation entropy ssat is 0.8066 kilojoules per kilogram Kelvin.

The entropy s40 is 0.9376 kilojoules per kilogram Kelvin.

The enthalpy h3 is calculated as 264.75 plus the fraction (0.9276 minus 0.8066) divided by (0.9376 minus 0.8066) times the difference (273.15 minus 264.75), resulting in 267.82 kilojoules per kilogram.

The work W is equal to the mass flow rate dot m times the difference in enthalpy h2 minus h1.

The mass flow rate dot m is calculated as the work W divided by the difference in enthalpy h2 minus h1, which equals 28 times 10 to the power of -3 kilojoules per second divided by the difference (268.82 minus 243.72), resulting in approximately 0.0015 kilograms per second.