c) First equation about the gas:
The rate of change of energy E with respect to time t is equal to the sum over i of the mass flow rate of i (denoted as dot m sub i) times the sum of the specific enthalpy of i (h sub i), the ratio of pressure of i (p sub i) to density of i (rho sub i), and half the square of the velocity of i (v sub i squared), plus the heat transfer rate (dot Q) minus the volume flow rate (dot V).

Second equation:
The change in internal energy, Delta U, is equal to the heat transferred from state 1 to state 2, denoted as Q sub 12.

Third equation:
The product of the mass of the gas (m sub gas) and the difference in specific internal energy between state 2 and state 1 (u sub 2 minus u sub 1) equals the heat transferred from state 1 to state 2, Q sub 12.

Fourth equation:
The heat transferred from state 1 to state 2, Q sub 12, equals the product of the mass of copper (m sub Cu), the specific heat capacity of copper (c sub u), and the temperature difference from state 2 to state 1 (T sub 2 minus T sub 1), assuming a perfect gas (perfekter Gas).

Fifth equation:
The heat transferred from state 1 to state 2, Q sub 12, is calculated as 3.6 times 10 to the power of 3 kilograms times 0.633 kilojoules per kilogram Kelvin times the difference between 273.15 Kelvin and 773.15 Kelvin.

Sixth equation:
This results in a heat transfer of negative 1139.4 kilojoules, which is also negative 1139.4 joules.

Seventh equation:
It implies that 1139.4 joules were transferred.