The rate of work done, W_k dot, equals 28 watts. The temperature T_2 is equal to T_1, which is the initial temperature T_i minus 6 degrees. The initial temperature T_i is minus 10 degrees Celsius. Therefore, T_2 equals minus 76 degrees Celsius, which is highlighted in a box. The equation zero equals m dot times the difference of h_e and h_a plus W_k dot. W_k dot equals m dot times the ratio of h_a over the difference of h_3 and h_2.

There is a table labeled "Tab A-10: State" with three columns labeled "2", "3", and "8 bar". The rows are labeled "p", "x", and "T". The values in the table are:
- State 2: pressure p equals 7.5748 bar, x equals 1, temperature T equals minus 76 degrees Celsius.
- State 3: pressure p equals 8 bar.

Additional Information provided is S_g equals 0.9298 kilojoules per kilogram Kelvin times 0.9298 kilojoules per kilogram Kelvin.

In the student solution:
- For State 3: pressure p equals 8 bar, entropy s equals 0.9298 kilojoules per kilogram Kelvin. This implies overheating.
- In reference A-72: at pressure p equals 8 bar, the enthalpy h_3 is calculated as 264.15 minus 273.66 times the ratio of (0.9298 minus 0.9066) over (0.9066 minus 0.9298) plus 264.75, which equals 277.37 kilojoules per kilogram, highlighted in a box.
- The enthalpy h_2 equals the enthalpy at saturation h_g at temperature T equals minus 76 degrees Celsius minus 237.74 kilojoules per kilogram.
- The ratio of 28 kilowatts over the difference of h_3 and h_2 equals 0.83 grams per second, which equals 3 kilograms per hour in refrigerant R_134a.

In a table:
- The columns are labeled z and z, with rows for pressure p, quality x, and enthalpy h. Both columns have pressure p equals 8 bar, quality x equals 0, and the enthalpy h in the first column equals 277.37 kilojoules per kilogram, and in the second column, it equals the enthalpy at saturation h_f at 8 bar, which is 93.42 kilojoules per kilogram.

From Tab A-10 at pressure p equals p_2 equals 1.5248 bar, the quality x_1 equals the ratio of h_1 minus h_f over h_g minus h_f, which equals 0.83 and 0.308, both highlighted in boxes.