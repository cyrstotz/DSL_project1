Graph description:
The graph is a pressure-volume (p-v) diagram. The x-axis is labeled as 'v' and the y-axis is labeled as 'p'.
There are several curves and points labeled as follows:
1. A curve starting from the origin (0,0) and moving upwards to the right, labeled as 'p0 equals p6'.
2. A curve starting from a point on the y-axis labeled 'p2 equals p3' and moving upwards to the right, labeled 'isochor'.
3. A curve starting from a point on the y-axis labeled 'p4 equals p5' and moving upwards to the right, labeled 'adiabatic'.
4. A horizontal line connecting the points on the y-axis labeled 'p2 equals p3' and 'p4 equals p5', labeled 'isotherm'.
5. Points labeled 1, 2, 3, 4, 5, and 6 are marked on the graph.

The expression 'omega0 squared minus omega a squared over 2 equals dot d exshr'.
The expression 'c p times (dot T0 minus T0) over T0 minus T0 times (c p dot s) times the natural log of (T0 over T0 e) minus R times the natural log of (p0 over p e) equals dot d exshr'.
The expression '1.006 times the natural log of ((293.15 K minus 328.074 K) over 2) equals 293.15 times the natural log of (1.006 times the natural log of (237.15 K over 328.074 K))'.
The expression '-2.933 times omega0 squared over dot v g plus (omega2 squared minus omega a squared over 2) equals 148.648 dot m times 8.04'.
The expression '(1 over dot d 0) minus 2.933 times omega0 squared over dot v g plus (200 dot c squared dot m squared over (3.5 plus 50)) times (omega2 squared over 2) equals 145.548 times (omega2 squared over dot v g)'.
The inequality '-1 less than or equal to Delta exsr equals 100 times (omega2 squared over dot v g)'.