The equation is pV equals m times the fraction R over M times T.

R equals the fraction R over M equals the fraction 8.314 kilojoules per kilomole Kelvin divided by 50 kilograms per kilomole, which equals 0.16628 kilojoules per kilogram Kelvin.

R equals 0.1663 kilojoules per kilogram Kelvin.

p subscript s,1 equals the fraction F over A plus p subscript amb.

A equals pi times r squared equals pi times (5 times 10 to the power of negative 2 meters) squared equals 7.853981 times 10 to the power of negative 3 square meters.

p subscript s,1 equals the fraction 32 kilograms times 9.81 meters per second squared divided by 7.853981634 times 10 to the power of negative 3 square meters plus 100,000 Newtons per square meter.

This equals 139,969.538 Newtons per square meter.

This equals 1.3967 bar.

The fraction pV over RT equals m subscript g equals the fraction (139,969.538 Pascals times 3.14 times 10 to the power of negative 3 cubic meters) divided by (0.16628 kilojoules per kilogram Kelvin times (500 plus 273.15 Kelvin)).

This equals 3.418687423 Newton meters per the fraction Newton meters per kilogram.

M subscript g equals 3.41879.