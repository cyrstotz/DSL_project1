v_2 equals v_1 divided by 3.

v_1 of p(T) equals v_2 of T because it is an incompressible fluid.

v_1 at 30 degrees Celsius equals.

p_1 times A equals p_0 times A plus m times g times y.

This implies that p_0,1 equals p_0 plus m times g divided by 4.

4 equals pi times D squared divided by 4 equals pi divided by 400 square meters.

Equals 1.4 bar.

pV equals mRT.

R equals R divided by M equals 8314 Joules per kilomole Kelvin divided by 50 kilograms per kilomole equals 166.28 Joules per kilogram Kelvin.

m_3 equals p_1 times V_1 divided by R times T_1 equals 3.62 times 10 to the power of minus 3 kilograms equals 3.62 grams.