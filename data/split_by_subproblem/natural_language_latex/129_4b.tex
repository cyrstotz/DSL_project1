Zero equals the mass flow rate of air times the difference between enthalpy at state two and enthalpy at state three, minus the input work rate. Given that the temperature at state two equals the temperature at state one plus six Kelvin, and the temperature at state one equals the temperature at state five plus ten Kelvin.

The temperature at state five, which has crossed out text, equals negative twenty-five degrees Celsius plus five, resulting in negative twenty degrees Celsius.

The enthalpy at state two, at negative twenty degrees Celsius and with a dryness fraction of one, equals 24.26 kilojoules per kilogram, with a note that says tau A minus ten.

The enthalpy at state three equals 33 minus 42 kilojoules per kilogram, with a note that says tau A minus eleven.

The rate of change of r times x equals the ratio of the input work rate to the difference in enthalpy between state two and state three.