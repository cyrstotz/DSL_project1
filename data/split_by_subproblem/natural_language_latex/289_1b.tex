The heat output, denoted as Q with a dot above and subscript "aus," equals 65 kilowatts.

The reciprocal of I subscript kS equals the fraction where the numerator is the difference between s subscript ein and s subscript aus, and the denominator is the difference between s subscript aus and s subscript ein.

The product of temperature T and differential ds equals the difference between differential U and the product of pressure p and differential volume dV, which is equal to differential enthalpy dH.

The differential enthalpy dH equals the product of dot U and p times differential volume dV.

The reciprocal of I subscript kS equals the fraction where the numerator is the integral from s subscript ein to s subscript aus of dH, and the denominator is the difference between s subscript aus and s subscript ein. This is equal to the fraction where the numerator is the difference between h subscript aus and h subscript ein, and the denominator is the difference between s subscript aus and s subscript ein, which equals the constant C times the fraction where the numerator is the difference between temperatures T subscript aus and T subscript ein, and the denominator is the difference between s subscript aus and s subscript ein.

The reciprocal of I subscript kS equals the fraction where the numerator is the difference between temperatures T subscript aus and T subscript ein, and the denominator is the product of mu and the natural logarithm of the ratio of T2 to T1. This is equal to the fraction where the numerator is the difference between temperatures T subscript aus and T subscript ein, and the denominator is the natural logarithm of the ratio of T subscript aus to T subscript ein, which equals 293.1226 Kelvin.