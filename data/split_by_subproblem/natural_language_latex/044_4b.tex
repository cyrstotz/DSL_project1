Energy-related equations:

Zero equals the mass flow rate times the difference between the enthalpy at exit and the enthalpy at entry, plus the heat transfer rate, where the process is adiabatic and involves kinetic work.

The mass flow rate equals the kinetic work divided by the difference between the enthalpy at state 2 and the enthalpy at state 1.

The temperature at point AB and the enthalpy at exit are mentioned.

The enthalpy at state 2, with a quality of 1, is at a temperature of negative 10 degrees Celsius, as read from a table diagram.

The temperature at point AB equals A plus zero minus N over 2. The enthalpy at state 2 equals the average of 247.34 kilojoules per kilogram and 249.45 kilojoules per kilogram, as read from a table diagram.

This results in 247.345 kilojoules per kilogram.

The enthalpy at state 3 would be at state 2 with the same entropy.

The entropy at state 2 equals the entropy at state 3, or it is adiabatic and reversible.

The entropy at state 2 equals the sum of 0.5233 and 0.5274, resulting in 0.5233 kilojoules per kilogram Kelvin.

The temperature at point AB equals A plus 1.2.

The entropy at state 3 equals 0.5233 kilojoules per kilogram Kelvin, extrapolated between the saturation temperature of 37.33 degrees Celsius and temperature at state 4 of 0.5373.

The enthalpy at state 3 equals 223.66 kilojoules per kilogram plus the difference between 0.5233 kilojoules per kilogram Kelvin and 0.5373 kilojoules per kilogram Kelvin.

This results in 265.524 kilojoules per kilogram.

The mass flow rate equals 9.787 kilojoules per second divided by the difference between 265.524 kilojoules per kilogram and 247.345 kilojoules per kilogram, resulting in 0.2066 kilograms per second.