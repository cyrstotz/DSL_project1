The input temperature T_KF,ein is 288.15 Kelvin and the output temperature T_KF,aus is also 288.15 Kelvin. The input pressure P_eine is equal to the output pressure P_aus, indicating an isobaric process.

In the energy balance section, the equation states that zero equals the mass flow rate dot m times the difference in enthalpy between the input and output (h_ein minus h_aus) plus the heat flow rate dot Q_F for a boiling liquid (water).

The difference between the heat flow rate dot Q_R and dot Q_F equals the heat flow rate dot Q_aw.

The mass flow rate dot m is 0.3 kilograms per second.

The input temperature T_ein is 70 degrees Celsius.

The output temperature T_aus is 100 degrees Celsius.

The heat flow rate dot Q_F is calculated as the mass flow rate dot m times the difference in enthalpy between the output and input (h_aus minus h_ein).

This results in 0.3 times (430.325 minus 304.649) kilojoules per kilogram.

Which equals 37.7 kilojoules per second.

The difference between the heat flow rate dot Q_R and dot Q_F equals the heat flow rate dot Q_aw.

The heat flow rate dot Q_aw is calculated as 100 kilowatts minus 37.7 kilowatts, resulting in 63.297 kilowatts.