The heat output, denoted as Q dot with subscript "aus," equals 65 kilowatts.

The mean temperature, denoted as T bar with subscript "MF," is calculated as the integral of temperature T with respect to s from s subscript "ein" to s subscript "aus," divided by the difference (s subscript "aus" minus s subscript "ein"). This is equal to q with subscript "kV" divided by (s subscript "aus" minus s subscript "ein").

From the energy balance, q with subscript "kV" equals h subscript "aus" minus h subscript "ein," which equals c times m dot times (T subscript "aus" minus T subscript "ein").

The difference (s subscript "aus" minus s subscript "ein") equals the integral from T subscript "ein" to T subscript "aus" of (c times m dot divided by T) dT, which simplifies to c times m dot times the natural logarithm of (T subscript "aus" divided by T subscript "ein").

Therefore, the mean temperature T bar with subscript "MF" equals (T subscript "aus" minus T subscript "ein") divided by the natural logarithm of (T subscript "aus" divided by T subscript "ein"), which equals 293.123 Kelvin.