The first equation is:
p subscript 1 delta equals p subscript ambient plus F over A plus g times (m subscript k plus m subscript ew) all over (D over 2 squared times pi).
This simplifies to:
p subscript 1 delta equals 100,000 plus (9.81 times (32 plus 0.4)) divided by (0.0025 times pi).
This further simplifies to:
p subscript 1 delta equals 100,000 plus 40,054 which equals 140,054 Pascals.

The second set of equations starts with:
pV equals nRT.
It is stated that n equals m over M and R bar equals 8.314 Joules per mole Kelvin.
The equation:
g times m subscript 3 over s equals (p subscript 1 delta times V subscript 1 delta) divided by (R bar) plus M times g.
This simplifies to:
g times m subscript 3 over s equals (140,054 times 0.00314 times 50 times 10 to the power of negative 3) divided by (8.314 times 273.15).
This results in:
g times m subscript 3 over s equals 0.003921 kilograms.