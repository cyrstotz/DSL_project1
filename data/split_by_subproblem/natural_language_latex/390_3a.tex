Item a):

- \( p_1 \) is derived from \( p_{atm} + F_G \frac{m}{A} + F_{OBV} \frac{V}{A} \)

The area \( A \) is calculated as:
\[ A = 8.95 \, \text{cm}^2 \times \pi = 28 \pi \, \text{cm}^2 \]

The pressure \( p_1 \) is calculated as:
\[ p_1 = 1 \, \text{bar} + 32 \, \text{kg} \times 9.81 \, \frac{\text{m}}{\text{s}^2} \times \frac{1}{28 \pi \times 10^{-4} \, \text{m}^2} + 0.12 \, \text{kg} \times 9.81 \, \frac{\text{m}}{\text{s}^2} \times \frac{1}{28 \pi \times 10^{-4} \, \text{m}^2} \]

This simplifies to:
\[ = 1 \times 10^5 \, \frac{\text{N}}{\text{m}^2} + 35966.3 \, \frac{\text{N}}{\text{m}^2} + 42.3 \, \frac{\text{N}}{\text{m}^2} \]

Resulting in:
\[ = 1.40 \, \text{bar} \]

- \( m_g \): Using the equation \( pV = mRT \), we find \( m_g \) as:
\[ m_g = \frac{p_3 V_3}{R T_{g_3}} \]

This is calculated as:
\[ -m_g = \frac{p_3 V_3}{R T_{g_3}} = \frac{p_3 V_3}{R T_{g_3}} \]

\[ = \frac{1.4 \times 10^5 \, \frac{\text{N}}{\text{m}^2} \times 3.74 \times 10^{-3} \, \text{m}^3}{8.314 \, \frac{\text{J}}{\text{mol} \cdot \text{K}} \times 50 \, \text{J} / \text{mol} \cdot \text{K}} \]

\[ = 0.003479 \, \text{kg} = 3.479 \, \text{g} \]