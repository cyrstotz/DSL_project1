The first graph is a Temperature-Entropy (T-S) diagram. The vertical axis is labeled T in Kelvin and the horizontal axis is labeled S in Joules per kilogram Kelvin. There are three isotherms drawn, each labeled with a different temperature. The lowest isotherm is labeled T_0. There are four points labeled 1, 2, 3, and 4. The process path starts at point 1, moves to point 2, then to point 3, and finally to point 4. The path from point 1 to point 2 is a vertical line, indicating an isothermal process. The path from point 2 to point 3 is a diagonal line, indicating an adiabatic process. The path from point 3 to point 4 is a horizontal line, indicating an isentropic process.

The second graph is another Temperature-Entropy (T-S) diagram. The vertical axis is labeled T in Kelvin and the horizontal axis is labeled S in Joules per kilogram Kelvin. There are five points labeled 1, 2, 3, 4, and 5. The process path starts at point 1, moves to point 2, then to point 3, then to point 4, and finally to point 5. The path from point 1 to point 2 is a vertical line, indicating an isothermal process. The path from point 2 to point 3 is a diagonal line, indicating an adiabatic process. The path from point 3 to point 4 is a horizontal line, indicating an isentropic process. The path from point 4 to point 5 is a vertical line, indicating an isothermal process. The path from point 5 to point 1 is a diagonal line, indicating an adiabatic process. There are arrows indicating the direction of the process path.