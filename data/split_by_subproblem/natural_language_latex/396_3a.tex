R equals the average R divided by M, which equals 0.166289 Joules per kilogram Kelvin.

Force Equilibrium:

The mass of ice equals the fraction of ice times the mass of the evaporator water, which equals 0.06 kilograms.

The sum of the mass of the container, the mass of the evaporator water, and the mass of the ice, all multiplied by the acceleration due to gravity, plus the ambient pressure times the area, equals the pressure at point 1 times the area.

The pressure at point 1 equals the sum of the mass of the container, the mass of the evaporator water, and the mass of the ice, all multiplied by the acceleration due to gravity, divided by the area, plus the ambient pressure, which equals 1.49155 bar.

The mass at point 1 equals the pressure at point 1 times the volume at point 1, divided by R times the temperature at point 1, which equals 3.42304 grams.

The area A equals pi times the diameter squared divided by 4, which equals 0.007853982 square meters.

Diagram Description:

There is a diagram showing the forces acting on a system. The diagram consists of a horizontal line with arrows pointing downwards and upwards. The downward arrows are labeled as follows from left to right:
- ambient pressure times area,
- mass of the container times gravity,
- sum of the mass of the evaporator water and the mass of the ice times gravity.

The upward arrow is labeled:
- pressure at point 1 times area.