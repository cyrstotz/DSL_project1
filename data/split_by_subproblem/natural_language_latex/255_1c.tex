Coolant: closed system, steady flow process in combination

The equation is zero equals the mass flow rate times the difference between the entropy at exit and the entropy at entry, plus the heat transfer rate out divided by temperature, plus the generated entropy.

The generated entropy equals the mass flow rate times the difference between the entropy at entry and the entropy at exit, minus the heat transfer rate out divided by temperature, which equals 212.77 Joules per second.

Between reactions and coolant, not only in the coolant.

Problem 1:

Initial temperature, T1, is 70 degrees Celsius.
Final temperature, T2, is 20 degrees Celsius.
Cooling capacity is 35 kilojoules.

Solution:

First Law of Thermodynamics over a closed tank. Semi-open system.

The change in internal energy from state 1 to state 2 equals the sum of heat transfer minus the sum of work (xp = 0.005).
The sum of heat transfer equals the heat of reaction (crossed out) (total mass 1) derived through the state equation.

Equation 7: mass at state 2 times specific internal energy at state 2 minus mass at state 1 times specific internal energy at state 1 equals the change in mass from state 1 to state 2 times specific enthalpy plus the heat of reaction (mass at state 2 equals mass at state 1 plus change in mass).
Internal energy at state 2 times A divided by negative 2 equals 478.94 plus xp times (1250.5 minus 478.94) minus 429.37 kilojoules per kilogram.
Internal energy at state 2 times A divided by negative 2 equals 292.95 kilojoules per kilogram (x equals 0, since only liquid now).
Specific enthalpy from state 1 to state 2 times A divided by negative 2 equals 83.96 kilojoules per kilogram.

Equation 7 (with change in mass): specific internal energy at state 2 minus mass at state 1 times specific internal energy at state 1 equals the change in mass from state 1 to state 2 times specific enthalpy minus the heat of reaction.
The change in mass times (specific internal energy at state 2 minus specific enthalpy) equals negative heat of reaction plus mass at state 1 times specific internal energy at state 1 minus mass at state 1 times specific internal energy at state 2.
The change in mass equals (mass at state 1 times heat transfer from state 1 to state 2 plus mass at state 1 times specific internal energy at state 1 minus mass at state 1 times specific internal energy at state 2) divided by (specific internal energy at state 2 minus specific enthalpy), which equals 3500.37 kilograms.