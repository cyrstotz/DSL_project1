The specific heat capacity, \( C_v \), is given as 0.633 Joules per gram per Kelvin, which is equivalent to 633 Joules per kilogram per Kelvin divided by 1000.

The heat transfer from state 1 to state 2, denoted as \( \dot{Q}_{1 \rightarrow 2} \), is calculated as the product of \( C_v \), the mass at state 1, \( m_{s,1} \), and the change in temperature, \( \Delta T \), which is the difference between temperature at state 2, \( T_2 \), and temperature at state 1, \( T_1 \).

The mass at state 1, \( m_{s,1} \), is determined by the equation \( \frac{p_1 V_1}{R T_1} \), and the relationship \( p_0 V_2 = m_{s,2} R T_2 \) holds for state 2.

The gas constants for both states are equal, \( R_1 = R_2 \).

From the equality of the gas constants, it follows that \( \frac{p_1 V_1}{R_1 T_1} = \frac{p_2 V_2}{R_2 T_2} \), which implies that \( p_1 = p_2 \).

The ratio of volume to temperature at state 1 is equal to the ratio of volume to temperature at state 2, leading to \( T_2 = \frac{V_2}{V_1} T_1 \cdot \frac{T_2}{80^\circ \text{C}} \).

Finally, substituting the values into the heat transfer equation, \( \dot{Q}_{1 \rightarrow 2} \) is calculated as 633 Joules per kilogram per Kelvin times 0.003449 kilograms times the temperature difference (420 Kelvin minus 350 Kelvin), which results in \( x \) Joules.