The rate of heat transfer, denoted as Q dot, equals the mass flow rate, denoted as m dot, times the expression in parentheses which includes the difference in enthalpy between the exit and the approach, denoted as h_e minus h_a, plus half the difference in the squares of the exit and approach velocities, denoted as w_e squared minus w_a squared over 2.

The rate of heat transfer, Q dot, equals zero.

The square of the approach velocity, w_a squared, equals the difference in enthalpy between the exit and the approach, h_e minus h_a, plus half the square of the exit velocity, w_e squared over 2.

The approach velocity, w_a, equals the square root of the difference in enthalpy between the exit and the approach, h_e minus h_a, plus half the square of the exit velocity, w_e squared over 2.

The approach velocity, w_a, equals the square root of the specific heat at constant pressure, c_p, times the temperature difference between T_5 and T_6, plus half the square of the exit velocity, w_e squared over 2.

The ratio of the temperatures T_0 over T_5 equals the ratio of pressures p_0 over p_5 raised to the power of (n-1) over n times T_5, which is also equal to (k-1) over k times T_5, and further equals 0.191 over 0.5 raised to the power of 0.4 over 1.4 times 431.9 Kelvin.

This results in a temperature of 328.07 Kelvin.

The approach velocity, w_a, equals the square root of 1.006 kilojoules per kilogram Kelvin times the temperature difference 431.9 minus 328.07, plus half of 220 meters squared per second squared, which equals 682.79 meters per second.