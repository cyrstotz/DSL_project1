a) p_{g,1} equals what?

Specific heat at constant volume, C_v, is 0.633 kilojoules per kilogram Kelvin.

Molar mass of the gas, M_g, is 50 kilograms per mole.

Temperature, T, is 500 degrees Celsius which is equal to 773 Kelvin.

Volume, V, is 3.14 liters which is equal to 3.14 times 10 to the power of minus 3 cubic meters.

The equation pV equals mRT is given as unknown.

1) Find p.

2) Use m equals pV divided by RT to find m.

p_{1,g} equals what?

Pressure, p, is equal to ambient pressure plus 32 kilograms times gravity divided by the area A_0.

This is equal to 1 bar plus the fraction of 32 kilograms times meters times 9.81 meters per second squared divided by A_0 in square meters.

The area A_0 is pi times r squared which equals pi times (40 divided by 2) squared equals pi times 20 squared equals 400 pi equals 2511 square centimeters.

2511 square centimeters equals 2511 times 0.01 squared square meters.

Pressure, p, equals 1 bar plus the fraction of 32 times 9.81 divided by 2511 times 10 to the power of minus 4.

This results in 1 bar plus 39971 Pascals which equals 1 bar plus 0.3997 bar equals 1.3997 bar.

m equals what?

Mass, m, equals 1.3997 bar times 3.14 times 10 to the power of minus 3 cubic meters divided by R times 773 Kelvin.

Gas constant, R, is 8.314 kilojoules per kilomole Kelvin which implies kilojoules per kilogram Kelvin equals 0.1663.

Mass, m, equals the fraction of 1.3997 times 10 to the power of 5 kilograms meters times 3.14 times 10 to the power of minus 3 cubic meters times kilograms times Kelvin divided by meters squared times seconds squared times 0.1663 kilojoules times 773 Kelvin.

Kilojoules equals 10 to the power of 3 kilograms meters squared per seconds squared which implies mass, m, equals 3.42 times 10 to the power of minus 3 times the fraction of kilograms meters cubic meters kilojoules Kelvin seconds squared divided by meters squared seconds squared kilograms meters squared Kelvin.

Mass, m, equals 3.42 times 10 to the power of minus 3 kilojoules equals 3.42 grams.