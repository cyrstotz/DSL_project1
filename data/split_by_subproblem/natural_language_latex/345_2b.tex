The heat transfer rate, Q dot, equals the mass flow rate, m dot, times the difference in enthalpy from state 5 to state 6 plus half the difference of the squares of the velocities at state 5 and state 6.

Assuming an isentropic process, where n equals k equals 1.4,

The temperature at state 6, T6, equals the temperature at state 5, T5, times the power of the ratio of pressure at state 6 to pressure at state 5 raised to the power of (n-1) divided by n, which equals 328.07 Kelvin.

The heat transfer rate, Q dot, equals the integral from state 5 to state 6 of the specific heat at constant pressure of the liquid, cpL, times the differential of temperature, dT, plus half the difference of the squares of the velocities at state 5 and state 6.

This equals cpL times the difference in temperature from T5 to T6 plus half the difference of the squares of the velocities at state 5 and state 6.

Half the square of the velocity at state 5 equals twice cpL times the difference in temperature from T5 to T6 plus twice the square of the velocity at state 6.

The velocity at state 6, w6, equals 311.85 meters per second.