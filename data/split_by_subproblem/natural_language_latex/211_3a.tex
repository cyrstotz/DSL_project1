A equals pi times the square of half the diameter D, which is calculated as pi times the square of 0.05 meters, resulting in 7.854 times 10 to the power of negative 3 square meters.

The force F_gmk equals 38 kilograms times 32 kilograms times 9.81 meters per second squared, which equals 3.13 Newtons.

The force F_gkw equals 0.1 kilograms times 9.81 meters per second squared, which equals 0.981 Newtons.

P equals P_amb plus the ratio of F_sk to A plus the ratio of F_gkw to A, resulting in 1.401 bar.

For the equation mg: density times volume equals m star times R times T, it implies that m_g equals the ratio of density times volume to R times T, which is calculated as 50 kilograms per mole times 1.401 times 10 to the power of 5 Pascals times 3.14 times 10 to the power of negative 3 cubic meters divided by R times 773.15 Kelvin, resulting in 3.42 kilograms.