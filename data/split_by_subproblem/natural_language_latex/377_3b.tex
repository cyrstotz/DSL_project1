To check if the values are correct, take the pressure of the gas \( p_{\text{ag}} \) as 1.5 bar and the mass of the gas \( m_{\text{g}} \) as 3.6 grams.

The pressure on the gas remains, \( p_{\text{ag}} = p_{\text{ag}} \).

In the second state, the water is incompressible, and the ratio of the initial water volume \( V_{1w} \) to the final water volume \( V_{2w} \) is equal to the ratio of the initial gas volume \( V_{1g} \) to the final gas volume \( V_{2g} \).

The pressure on the gas does not change, still being 1.5 bar, \( p_{2g} = p_{1g} = 1.5 \text{bar} \).

Using the ideal gas law \( PV = mRT \), the temperature of the gas in state 2, \( T_{2g} \), is calculated as \( \frac{p_{2} V_{2}}{mR} \).

The adiabatic exponent \( K \) is calculated as \( \frac{c_p}{c_v} = \frac{0.8984}{0.633} = 1.262 \).

There is a diagram with the following elements:
- A fraction with \( \left(\frac{p_2}{p_1}\right)^{\frac{K-1}{K}} \) on the left side.
- An arrow pointing to the right from the fraction.
- The fraction \( \frac{V_1}{V_2} \) on the right side.
- An arrow pointing down from the fraction.
- The fraction \( \left(\frac{V_1}{V_2}\right)^{K-1} \) on the bottom.
- An arrow pointing to the left from the fraction.
- The fraction \( \left(\frac{p_2}{p_1}\right)^{\frac{1}{K-1}} \) on the left side.
- An arrow pointing up from the fraction.
- The fraction \( \frac{V_1}{V_2} \) on the top.