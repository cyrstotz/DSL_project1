c) The product of m1 and v1 equals 510 Joules per kilogram, and the initial temperature T0 is 340 Kelvin.

Exergy of a flow

The exergy of the flow, denoted as e_ex, str, is equal to the mass flow rate dot m times the expression (h1 minus h0 minus T0 times (s1 minus s0) plus the ratio of ke over pe).

This implies that the exergy of the flow, e_ex, str, equals h1 minus h0 minus T0 times (s1 minus s0) plus ke.

The kinetic energy term ke is calculated as one half times the fraction (510 Joules per kilogram times 2) divided by (2 times 200 squared), which equals 110050 Joules per kilogram, or 110 kiloJoules per kilogram.

The change in enthalpy from h0 to h1 is the integral from T0 to T1 of cp(T) dT.

This implies that the change in enthalpy from h0 to h1 equals 1.006 kiloJoules per kilogram Kelvin times 340 Kelvin minus 1.006 kiloJoules per kilogram Kelvin times 243.15 Kelvin, which equals 97 kiloJoules per kilogram.

The change in entropy from s0 to s1 is cp times the integral from T0 to T1 of dT over T minus the gas constant R times the natural logarithm of the ratio of p1 over p0.

This implies that the change in entropy from s0 to s1 equals 1.006 kiloJoules per kilogram Kelvin times the natural logarithm of the ratio of 340 Kelvin over 243.15 Kelvin minus 8.314 kiloJoules per kilogram Kelvin times the undefined expression (1 over 0).

This implies that the change in entropy from s0 to s1 equals 0.3372 kiloJoules per kilogram Kelvin.

This implies that the exergy of the flow, e_ex, str, equals 97 kiloJoules per kilogram plus 243.15.

The exergy of the flow, e_ex, str, equals 97 kiloJoules per kilogram plus 243.15 kiloJoules per kilogram plus 0.3372 kiloJoules per kilogram Kelvin plus 110 kiloJoules per kilogram.

The exergy of the flow, e_ex, str, equals 125.06 kiloJoules per kilogram.