In the section titled "R134a," a table is presented with columns labeled P and T, representing pressure and temperature respectively. The entries in the table are as follows:
- Row 1: Temperature is 50.33 degrees Celsius, pressure is not specified.
- Row 2: Pressure is 8 bar, temperature is not specified.
- Row 3: Pressure is 8 bar, temperature is 31.33 degrees Celsius.
- Row 4: Neither pressure nor temperature is specified.

Several equations are listed:
- The temperature T1 minus 6 Kelvin equals T2.
- The value of x2 is 1.
- The value of x4 is 0.
- The terms "gas," "isotherm," and "isobar" are mentioned.
- The terms "liquid," "isentrop," and "isotherm" are mentioned in German.
- T1 equals 10 Kelvin plus Tges.
- This results in T1 being 283.15 Kelvin when adding 273.15 to 10 Kelvin.
- T2 is calculated as T1 minus 6 Kelvin, resulting in 277.15 Kelvin.

In the subsection titled "a2)," a graph description is provided:
- The graph is a pressure-volume (P-V) diagram.
- The x-axis is labeled with specific volume in cubic meters per kilogram, and the y-axis is labeled with pressure in bar.
- Four points are marked as 1, 2, 3, and 4 on the graph.
- Point 1 is at the bottom left, point 2 is directly to the right of point 1, point 3 is above point 2, and point 4 is to the left of point 3.
- The process from point 1 to 2 is labeled as "isotherm," from 2 to 3 as "isobar," from 3 to 4 as "isotherm," and from 4 to 1 as "isentrop."