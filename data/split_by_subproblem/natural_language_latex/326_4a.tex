The first graph is a pressure-temperature diagram, abbreviated as p-T diagram. The x-axis is labeled as T in degrees Celsius and the y-axis is labeled as p in bar. There is a point labeled "Trippel" and a shaded region indicating different phases.

The second graph is another p-T diagram with more detailed annotations. The x-axis is labeled as T in degrees Celsius and the y-axis is labeled as p in bar. The graph shows regions labeled "Gas", "Solid" (Fest in German), and "isobar". There is a point labeled "Trippel" and lines indicating "isotherm" and "isotherm 2".

The third graph is a p-T diagram with the x-axis labeled as T in degrees Celsius and the y-axis labeled as p in bar. It shows a point labeled "Trippel" and lines indicating "10K above sublimation point" (10K über sublimationspunkt in German) and "10K above sublimation point". There is also a region labeled "Boiling below triple point" (Sieder unter Trippelpunkt in German).