Description of the diagram:
The diagram is a horizontal cross-section of a wall. The wall is represented by a rectangular box with a hatched pattern inside. The top and bottom boundaries of the wall are marked by dashed orange lines. There is an arrow labeled with "Q dot out" pointing upwards from the middle of the wall, indicating heat flow. Below the wall, there is a label "H2O". The top of the wall is labeled "Coolant" and the right side is labeled "Wall".

Balance equation: Entropy balance around the wall:
Zero equals the quotient of "Q dot out" over "T H2O" minus the quotient of "Q dot out" over "T KF" plus "S dot generated".

"T H2O" equals "T R" equals 100 degrees Celsius equals 373.15 Kelvin.

"S dot generated" equals "Q dot out" times the sum of the reciprocal of "T H2O" plus the reciprocal of "T KF".

"T KF" equals 295 Kelvin.

"S dot generated" equals 65 kilowatts times the difference of the reciprocal of 295 Kelvin minus the reciprocal of 373.15 Kelvin equals 0.04194 kilojoules per Kelvin.

The table includes columns labeled Delta M12, Q, W, Z, and T. The rows include:
- The first row under the headers has "Q dot out" under Q and zero under Z.
- The second row has 35 megajoules under Q, one under Z, and 100 degrees Celsius under T.
- The third row has two under Z and 70 degrees Celsius under T.