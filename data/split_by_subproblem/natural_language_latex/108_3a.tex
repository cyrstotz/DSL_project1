a) The pressure at ground level is 2, what is the mass of gas?

The pressure of gas 1 is equal to the pressure at the outer medium 1 plus the force per area, which is the mass of the substance times gravity divided by the area.

The area is equal to pi times the radius squared, which is pi times the diameter squared divided by 4, equaling 0.0079 square meters.

The mass of the gas is the sum of mass 1 and mass E1 times gravity, which equals 32.76 kilograms times gravity, resulting in 321.801 Newtons. Therefore, the mass of the gas divided by the area is 0.4 bar.

The pressure at ground level is equal to the pressure of gas 1, which is 0.6 bar, equaling 2 kilograms.

The equation pV equals mRT implies that pressure equals the pressure of gas 1, volume equals 3.14 times 10 to the power of negative 3 cubic meters, and temperature equals 293.15 Kelvin.

The mass of the gas is the pressure times volume divided by R times T, which simplifies to R over R, resulting in 166.28 Joules divided by 8.314 kilograms Kelvin.

The mass of the gas is the pressure times volume divided by R times T, resulting in 3.42 grams.

For Tg2 equals 2 and Pg2 equals 2,

The specific heat at constant pressure is R times T times the specific heat at constant volume, equaling 0.799 Joules per kilogram Kelvin, and the pressure at 1g equals the pressure at 2g, which is 14 bar.

Energy balance:

The mass of the gas times internal energy 2 equals the mass of the gas times internal energy 1 plus mass E1 times internal energy E1 minus internal energy E1.

The mass of the gas times the difference in internal energy between 2 and 1 plus mass E1 times the difference in internal energy between 2E1 and 1E1 equals zero.

The difference in internal energy between 2 and 1 equals the specific heat at constant volume times the difference in temperature between 2 and 1.

The mass of the gas times internal energy 2 equals mass E1 times internal energy E1.