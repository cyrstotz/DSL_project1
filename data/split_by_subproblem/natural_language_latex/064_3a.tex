The temperature \( T_V \) is 500 degrees Celsius and the initial volume \( V_0 \) is 3.14 liters.

The equation \( P \cdot V = m \cdot R \cdot T \) represents the relationship between pressure, volume, mass, the gas constant, and temperature.

The gas constant \( R \) is calculated as \( R = \frac{Q}{M} \), where \( Q \) is 8.3145 Joules per Kelvin per mole and \( M \) is 30 kilograms per mole, resulting in \( R = 0.16628 \) Joules per kilogram per Kelvin.

The total pressure \( P_{ges} \) is the sum of the initial pressure \( P_0 \) and the additional pressure due to a mass of 32 kilograms over an area \( A \), with gravity \( 9.81 \) meters per second squared. This calculation results in \( P_{ges} = 1 \) bar plus \( 3.14 \) bar, totaling \( 4.14 \) bar.

Using the equation \( P \cdot V = m \cdot R \cdot T \), the mass \( m \) is derived as \( m = \frac{P \cdot V}{R \cdot T} \), which calculates to \( 3.48 \) kilograms when substituting the values \( P = 1.4 \times 10^5 \) Pascals, \( V = 3.14 \times 10^{-3} \) cubic meters, \( R = 0.16628 \) Joules per kilogram per Kelvin, and \( T = 773.15 \) Kelvin.