p_0 equals p_C.

Static fixed point in the diagram: zero equals m dot times (h_s minus h_6) plus Q dot minus W dot.

W dot equals m dot times (h_s minus h_6).

h_s at 0.5 bar equals 431.9 Kelvin.

Interpolate for x_1 equals 430 Kelvin, x_2 equals 440 Kelvin, in Table A22.

m dot total equals m dot 1 plus m dot 2, where m dot 1 equals 5.293 times m dot 2.

m dot total equals 6.293 times m dot 2.

The total exergy rate, E dot ex,tot, equals m dot ex times (h_0 minus h_1 minus T_0 times (s_0 minus s_1)) plus one half times (u_0 squared minus u_1 squared).

The total exergy rate, E dot ex,tot, equals m dot ex times (h_0 minus h_1 minus T_0 times (s_0 minus s_1)).

Total exergy, E ex,tot, equals m dot ex times (h_0 minus h_1 minus T_0 times (s_0 minus s_1) plus one half times (u_0 squared plus u_1 squared)) plus the sum of (1 minus T_0 over T) times Q dot minus the sum of W dot.

Delta c_x,1sK equals 100 kilojoules per kilogram.

Exergy rate, e dot ex,1Q, equals (1 minus T_0 over T_B) times Q dot equals (1 minus 293 Kelvin over 1283 Kelvin) times 1195 kilojoules per kilogram equals 269.72 kilojoules per kilogram.

O equals s_0 minus s_1 plus Q dot over T_B equals s dot ex,1 minus one over m dot ex times E dot ex,ver equals T_0 times s dot ex,2, where E dot ex,ver equals T_0 times s dot ex,2, and E dot ex,ver equals T_0 times s dot ex,2.

s dot ex,2 equals s_5 minus s_0 plus Q_B dot over T_B equals the integral from T_0 to T_6 of c_p^0 over T dT minus R times ln (P_6 over P_5), where R equals R over M_air and M_air equals 28.973 kilograms per mole.

e ex,ver equals T_0 over T_0 times (c_p^0 times ln (T_6 over T_0) minus R over M_air times ln (P_6 over P_0) minus Q_B dot over T_B).