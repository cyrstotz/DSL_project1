Section a) Qaus

Stationary flow process:

The equation is zero equals the mass flow rate of the coolant fluid times the difference between the enthalpy at the inlet and the outlet, plus the sum of heat flow rates minus the sum of work done by turbines.

Because the power at the outlet of the coolant fluid equals the power at the inlet of the coolant fluid equals zero, it implies that the sum of the work done by turbines equals zero.

The equation simplifies to zero equals the mass flow rate of the coolant fluid times the difference between the enthalpy at the inlet and the outlet, plus the heat flow rate at the outlet.

The heat flow rate at the outlet equals the mass flow rate of the coolant fluid times the difference between the enthalpy at the outlet and the inlet.

The difference between the enthalpy at the outlet and the inlet equals the integral from the inlet temperature to the outlet temperature of the specific heat capacity times the differential of temperature plus the velocity over the mass flow rate times the difference between the second and first pressure.

This equals zero because there is no pressure loss, and the pressure at the outlet of the coolant fluid equals the pressure at the inlet of the coolant fluid.

The difference between the enthalpy at the outlet and the inlet equals the specific heat capacity times the difference between the outlet temperature and the inlet temperature, which equals the specific heat capacity times 10 Kelvin.