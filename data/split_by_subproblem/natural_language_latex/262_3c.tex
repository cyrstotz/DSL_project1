Q subscript 12

x subscript E3,1 equals the fraction m subscript E3 over m subscript EW equals 0.6

Energy balance:

m dot equals 0

dE over dt equals the sum of m dot times epsilon plus the sum of Q minus the sum of U dot subscript i

Delta U subscript 12 equals the sum of Q minus the sum of U dot subscript i, which is negligible

Delta U subscript 12 equals Q subscript 12

mg times (u2 minus u1) equals Q subscript 12

The integral from T1 to T2 of c subscript v dt equals c subscript v times (T2 minus T1)

mg times c subscript v times (T2 minus T1) equals Q subscript 12

T2 equals 0 degrees Celsius equals 273.15 Kelvin

T1 equals 500 degrees Celsius equals 773.15 Kelvin

c subscript v equals 0.163 kJ per kg per Kelvin

Q subscript 12 equals the fraction 108.29 over 1083 kJ

X subscript Ice,2

X subscript Ice,1 equals 0.6

It is given above, Q subscript ab, we take 1500 Joules from the task

Energy:

Delta U subscript 21 equals m times f times theta plus Q subscript 12 minus W subscript n

Released heat: Q subscript 12 equals negative Q subscript 12

Delta U subscript 21 equals negative Q subscript 12

m subscript Ew times (U2 minus U1) equals negative Q subscript 12

U2 equals negative Q subscript 12 over m subscript Ew plus U1

Values from the given table, x equals 0.6, p equals 1.7 bar

U1 equals U subscript solid plus x times (U subscript liquid minus U subscript solid)

U1 equals negative 13,4102 kJ per kg

U2 equals Q subscript 12 over m subscript Ew plus U1 equals the fraction 1500 times 10^3 over m subscript Ew plus U1 equals negative 28,4102 kJ per kg

Interpolating with the given table

X subscript Ice,2 equals the fraction 333.858 minus (-0.095) over 1 minus 0

X subscript Ice,2 equals negative 333.858 minus (-0.095) equals negative 28,4102 minus (-0.095) plus 0.085