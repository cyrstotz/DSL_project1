d) The energy input \( E_{zu} \) is equal to the ratio of \( Q_{zu} \) over \( W_{t} \), where \( Q_{zu} \) is equal to \( Q_u \).

(Diagram: A vertical rectangle with labels \( Q_u \) entering from the left, \( Q_{zu} \) exiting from the right, and \( W_t \) exiting from the bottom.)

This implies that zero equals the mass flow rate \( \dot{m} \) times the difference in enthalpy \( h_2 - h_1 \) plus \( Q_u \).

\( Q_u \) equals the mass flow rate \( \dot{m} \) times the difference in enthalpy \( h_2 - h_1 \).

\( h_2 \) equals \( h_1 \) plus 35.66 percent of the difference between \( h_{q_2} \) and \( h_q \), which equals 93.42 watts per second.

\( h_q \) equals 23.1 times 62 watts per second, which equals 0.084 times 42 watts per second, resulting in 88.91 watts.

This implies that the kinetic energy \( E_k \) is equal to the ratio of \( Q_{zu} \) over \( W_t \), which is \( 91 \) watts divided by \( 28 \) watts, resulting in a boxed value of 3.25.