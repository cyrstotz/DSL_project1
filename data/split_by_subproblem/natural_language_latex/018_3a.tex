The pressure \( P_{G1} \) is equal to three-fourths \( mg \).

The area \( A \) is calculated as the square of half the diameter \( D \) times pi, which equals 0.007854 square meters.

The new pressure \( P_{\text{neu}} \) is the force \( F \) divided by the area \( A \), which is the mass \( m \) times gravity \( g \) divided by \( A \) plus the extra weight \( m_{\text{EW}} \) times gravity \( g \) divided by \( A \).

This results in \( \frac{32 \times 9.81}{0.007854} + \frac{0.1 \times 9.81}{0.007854} \), which simplifies to 40000 Pascals plus 35 Pascals.

Converting to bar, this is 0.4 bar.

The total pressure \( P_{G1} \) is the initial pressure \( P_0 \) plus the new pressure \( P_{\text{neu}} \), resulting in 1.4 bar.

The mass \( m_g \) is unknown and needs to be calculated.

Using the ideal gas law \( pV = mRT \), the mass \( m \) can be expressed as \( \frac{pV}{RT} \).

The gas constant \( R \) is calculated as the universal gas constant \( \mathcal{R} \) divided by the molar mass \( M \), which is \( \frac{8.314}{50} = 0.16628 \).

The pressure \( P \) is set equal to \( P_{G1} \).

The mass \( m_g \) is then \( \frac{P_{G1} V}{RT} \), which is \( \frac{1.4 \, \text{bar} \cdot 3.14 \, L}{0.16628 \, \frac{\text{m}^2}{\text{s}^2 \cdot \text{K}} \cdot (500 + 273.15) \, \text{K} \cdot 1000} \).

This results in 0.00342 kilograms, which is 3.42 grams.