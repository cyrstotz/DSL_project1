In the center of the content, there is a picture of size 100 by 100 units. In this picture:
- A horizontal vector is drawn from the point (10,10) to the right, extending 80 units.
- A vertical vector is drawn from the point (10,10) upwards, extending 80 units.
- A line is drawn from the point (10,10) to the point (70,70) at a 45-degree angle.
- Another line is drawn from the point (10,10) to the point (40,70) at a steeper angle.
- A horizontal line is drawn from the point (40,40) to the point (70,40).
- A vertical line is drawn from the point (70,40) to the point (70,70).
- Three filled circles (each with a radius of 3 units) are placed at points (40,40), (70,40), and (70,70).
- Labels "1", "2", "3", and "4" are placed near points (25,25), (55,25), (55,55), and (75,75) respectively.
- The bottom left corner has a label "T1 = -10 degrees Celsius".
- The label "T (K)" is placed near the right end of the horizontal vector.
- The label "p (bar)" is placed near the top end of the vertical vector.
- The words "flüssiges Gas" (liquid gas) are placed near the top center, above the point (40,75).
- The word "Kondensat" (condensate) is placed to the right of the point (80,40).