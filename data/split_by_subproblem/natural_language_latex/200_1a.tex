The equations and expressions provided are as follows:

1. The enthalpy at exit temperature, denoted as \( h_e \), at 70 degrees Celsius is 292.98 kilojoules per kilogram.
2. The enthalpy at ambient temperature, denoted as \( h_a \), at 100 degrees Celsius is 479.04 kilojoules per kilogram.
3. The equation \( 0 = \dot{m}(h_e - h_a) + \sum \dot{Q} \) simplifies to \( \dot{m}(h_e - h_a) + \dot{Q}_R - \dot{Q}_{\text{aus}} \).
4. The heat output \( \dot{Q}_{\text{aus}} \) is calculated as 0.3 kilograms per second times the difference in enthalpy (292.98 kilojoules per kilogram minus 479.04 kilojoules per kilogram) plus 100 kilowatts, resulting in 62.182 kilowatts.

5. The enthalpy \( h_1 \) is a mix of \( h_g \) and \( h_f \) weighted by \( x_D \) and results in 930.32 kilojoules per kilogram.
6. The enthalpy \( h_2 \) at 70 degrees Celsius is 292.49 kilojoules per kilogram.
7. The system described is a semi-open system.
8. The internal energy change from state 1 to state 2, \( m_2 u_2 - m_1 u_1 \), equals the mass flow rate change times the enthalpy difference \( h_e - h_2 \) plus heat \( Q \).
9. The internal energy \( u_2 \) at 20 degrees Celsius is 292.95 kilojoules per kilogram.
10. The internal energy \( u_1 \) is a mix of \( u_g \) and \( u_f \) at 100 degrees Celsius and 20 degrees Celsius respectively, weighted by \( x_D \), resulting in 929.378 kilojoules per kilogram.
11. The equation \( (m_1 + \Delta m) u_2 - m_1 u_1 = \Delta m h_{em} + \cancel{Q} \) simplifies to \( 0 = m_1 u_2 - m_1 u_1 + Q \) and further to \( \frac{h_{em} - u_2}{u_1 - u_2} \).
12. The enthalpy \( h_{em} \) at 20 degrees Celsius is 83.96 kilojoules per kilogram.
13. The mass \( m_1 \) is 5755 kilograms.
14. The heat \( Q \) is 35000 kilojoules.
15. The change in mass \( \Delta m \) is 3589.37 kilograms.