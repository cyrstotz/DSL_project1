Here is the translation of the LaTeX content into natural language:

There is a table with eight columns labeled as follows: blank, P, T, V, W dot, Q dot, S dot, and X. The rows of the table are filled as follows:
- In the first row, P equals p1 which is equal to p2, T equals 1.5748, and the other cells are blank.
- In the second row, P equals 1.5748, T equals negative 10 degrees Celsius, W dot equals 28 watts, Q dot equals 0, S dot equals 0, and X equals 1.
- In the third row, P equals 8 bar, and the other cells are blank.
- In the fourth row, P equals 8 bar, and W dot, Q dot, and S dot are all 0.

The equations and statements following the table are:
- p1 equals p2.
- T2 equals T1 minus 6 Kelvin. T1i is approximately negative 10 degrees Celsius as per figure 5 where pressure equals 1 mbar and matches c times 10 degrees Celsius.
- T2 equals negative 10 degrees Celsius.
- Q12 equals 0.
- Q23 equals 0.
- S23 is adiabatic and reversible, equals 0.
- p2 is found in Table A-10 at T equals negative 10 degrees Celsius and equals 1.5748 bar which is equal to p1.

Additional information includes:
- Temperature would decrease: Sublimation is an endothermic process, meaning that when water evaporates, it absorbs heat. Additionally, there is heat from exhaust and liquid.
- The energy rate, E dot n, equals the ratio of Q dot zu to W dot 1, which is also equal to the ratio of Q dot zu to the sum of Q dot ab and Q dot zu.
- The ratio of Q dot zu to W dot 1 equals 8 epsilon.
- 0 equals the product of m dot and the difference of h_e and h_a plus Q dot n.
- Q dot n equals the product of m dot and the difference of h_a and h_e.
- h_a is found in Table A-10 at T equals negative 76 degrees Celsius for pure gaseous state, h_g, and equals 237.14 watts per kilogram.
- W dot n equals 28 watts.