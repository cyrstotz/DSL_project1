The pressure \( p_{g2} \) is equal to the fraction of the product of mass \( m_g \), gas constant \( R \), and temperature \( T \) divided by volume \( V \).

The text states "Force equilibrium:"

The force due to gravity \( F_g \) is equal to the force due to pressure \( F_p \) which is also equal to \( F_{p2} \).

The equation given is the mass \( m_g \) times gravity \( g \) plus the ambient pressure \( p_{amb} \) times the square of half the diameter \( D \) divided by two, all multiplied by \( \pi \), equals the square of half the diameter \( D \) divided by two, squared, multiplied by \( \pi \) and \( p_1 \).

The pressure \( p_2 \) is equal to the fraction of the product of mass \( m_g \) and gravity \( g \) divided by the square of half the diameter \( D \) divided by two, squared, multiplied by \( \pi \), plus the ambient pressure \( p_{amb} \), which equals 1.40 bar, denoted as A.

The mass \( m_g \) is equal to the mass \( m_2 \) divided by the difference in density \( \lg - Dichte \).

The mass \( m_g \) is equal to twice the volume \( V \) divided by the product of the gas constant \( R \) and temperature \( T \).

The gas constant \( R \) is equal to the fraction of the constant \( k \) divided by the molar mass \( M \), which equals 166.28, and is calculated as 4157 divided by 25.

The mass \( m_g \) is 3.449 grams, denoted as B.

A is defined as \( p_2 \) in Pascals.

B is defined as \( m_g \) in kilograms.