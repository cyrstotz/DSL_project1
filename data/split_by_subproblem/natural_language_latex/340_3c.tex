The kinetic energy \( E_{\text{kin}} \) is equal to one half times the mass \( m \) times the square of the velocity \( v \) squared.

The rate of change of energy with respect to time \( \frac{dE}{dt} \) is equal to the mass \( m \) times the rate of change of temperature with respect to time \( \frac{dT}{dt} \) plus the product of \( \dot{u} \) and \( \dot{v} \).

The statement of the first law (Satz der 1).

The difference between internal energy states \( U_1 \) and \( U_2 \) is equal to the heat \( Q \) minus the work \( W \).

The work \( W \) is equal to the integral from volume \( V_1 \) to \( V_2 \) of the pressure \( p \) times the differential volume \( dV \), which simplifies to \( p \) times the difference \( V_2 - V_1 \).

The volume \( V_2 \) is equal to the mass \( m \) times the average gas constant \( \bar{R} \) times the temperature \( T_2 \) divided by the pressure \( p \).

This expression simplifies to \( 0.005 \) kilograms times \( 287 \) Joules per kilogram Kelvin times \( 273.15 \) Kelvin divided by \( 1.4 \) grams per cubic centimeter.

This calculation results in \( 7.507 \) cubic meters.

Which further simplifies to \( 0.0066 \) cubic meters.

And finally, it equals \( 3.04 \) times some unspecified unit.