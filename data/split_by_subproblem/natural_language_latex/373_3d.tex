The temperature \( T_{x_2} \) is 90 degrees Celsius.

First Law of Thermodynamics:
The change in internal energy \( \Delta U_{x_1 BW} \) is equal to heat \( Q \) minus work \( W \), which calculates to \( (7.36c \times 2 - 0.2852) \) kilojoules, resulting in \( 7.0748 \) kilojoules.

The change in internal energy \( \Delta U_{x_2} \) is equal to the mass of the working body \( m_{BW} \) times the difference in specific internal energy from state 1 to state 2 (\( u_2 - u_1 \)).

The specific internal energy at state 2 \( u_2 \) is calculated as \( \frac{1 - x_2}{v_f (0.0931 \, c)} + \frac{x_2}{v_f (0.201 \, c)} \).

The specific internal energy at state 1 \( u_1 \) is \( v_f (0^\circ C) (1 - x_1) + v_f (0^\circ C) x_1 \), where \( x_1 = 0.16 \).

The mass of the working body times the difference in specific internal energy from state 2 to state 1 is equal to \( x_2 \) times the difference in specific internal energy at temperatures \( T_x \) and \( T_1 \) plus the specific internal energy at \( T_1 \) minus \( u_1 \), which equals \( \Delta u_{x_2} \).

The value \( x_2 \) is calculated as \( \frac{\Delta u_{x_2} + u_f (T_1) - u_f (T_x)}{u_f (T_x) - u_f (T_1)} \).

The value \( x \) is calculated as \( \frac{7.0748 \, kJ}{0.1 \, kg} \) and further simplified using specific internal energy values, resulting in \( 0.575 \).