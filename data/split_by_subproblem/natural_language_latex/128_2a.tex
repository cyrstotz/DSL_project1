Diagram:
A horizontal membrane is shown with arrows pointing upwards labeled \( P_0 \).
Above the membrane, there are three blocks labeled \( m_{ew} \).
The membrane is labeled as "Membrane".

The equation \( m k g + p_0 A t + m_{ew} g = p_{G1} A \).

The equation \( \frac{g}{A} (m k + m_{ew}) + p_0 = p_{G1} \).

The equation \( \frac{3.8 \, \frac{m}{s^2}}{(0.5 \, m)^2 \pi} (32 \, kg + 0.1 \, kg) + 1 \, \text{bar} = 1.9 \, \text{bar} = p_{G1} \).

The value \( 0.05 m \).

Diagram:
A horizontal piston-cylinder arrangement is shown. The piston is labeled \( m_{G1} \).
The volume \( V_{G1} \) is shown to the right of the piston.

The equation \( p_{G1} V_{G1} = \frac{R}{M_{gas}} T_{G1} \cdot m_{G1} \).

The equation \( m_{G1} = \frac{p_{G1} V_{G1}}{T_{G1}} \cdot \frac{M_{gas}}{R} = \frac{1.9 \, \text{bar} \cdot 3.14 \cdot 10^{-3} \, m^3}{773 \, K} \cdot \frac{50 \, kg}{8.314 \, \frac{kJ}{kmol \cdot K}} \).

The result \( = 3.42 \, kg = m_{G1} \).