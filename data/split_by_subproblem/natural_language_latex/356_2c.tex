c) The change in specific exergy due to the stream, denoted as Delta e_x,str, is equal to e_x,str,6 minus e_x,str,0, which is to be determined.

The change in specific exergy at state 6, Delta e_x,str,6, is given by the formula:
Delta h_0G minus T_0 times (s_6 minus s_0) plus half the change in the square of the angular velocity from state 0 to state 6.

The change in enthalpy from state 0 to state G, Delta h_0G, is calculated by integrating c_p dT from T_0 to T_6, which simplifies to c_p times (T_6 minus T_0). Substituting the values, it equals 360 Joules per kilogram Kelvin times (273.15 Kelvin minus 288.15 Kelvin), resulting in -5400 Joules per kilogram.

The change in entropy from state 0 to state 6, Delta s_60, is calculated by integrating c_p over T from T_0 to T_6 minus R times the natural logarithm of (p_6 over p_0). This simplifies to c_p times (natural logarithm of T_6 minus natural logarithm of T_0) minus R times the natural logarithm of (p_6 over p_0). Substituting the values, it equals 360 Joules per kilogram Kelvin times (natural logarithm of 288.15 Kelvin minus natural logarithm of 273.15 Kelvin) minus 287 Joules per kilogram Kelvin times the natural logarithm of (1 over 1), resulting in 6.30135 Joules per kilogram Kelvin.

The change in half the square of the angular velocity from state 0 to state 6, Delta (omega squared_6 minus omega squared_0 over 2), equals (omega squared_6 minus omega squared_0 over 2), which is 102.648 Joules per kilogram.

Therefore, the change in specific exergy, Delta e_x,str, equals Delta h minus T_0 times Delta s plus Delta (omega squared over 2), resulting in 118.426 Joules per kilogram.