Change in energy, denoted as Delta E, equals heat added, Q, minus work done, W. The work done, W, is calculated as the pressure at state 2 times volume at state 2 minus the pressure at state 1 times volume at state 1. The volume, V, is similarly calculated as pressure at state 2 times volume at state 2 minus pressure at state 1 times volume at state 1. Using the ideal gas law, the volume at state 2 is given by the mass of the gas times the gas constant times the temperature of the gas at state 2. The volume at state 2 is 0.0011 cubic meters. The work done is also expressed as pressure times the difference between volume at state 2 and volume at state 1, resulting in work done as negative 254.3 Joules. The change in energy is also expressed as mass times the difference in internal energy from state 1 to state 2. For a perfect gas, the difference in internal energy between two states is the heat capacity at constant volume times the difference in temperature between the two states. The heat added is calculated as mass times the heat capacity at constant volume times the temperature difference plus the work done. The heat capacity at constant volume is 633 Joules per kilogram Kelvin, the temperature at state 2 is 0 degrees Celsius, and the ambient temperature is 500 degrees Celsius. The heat added is negative 1367 Joules.

There are no graphs, figures, or diagrams in the provided image.