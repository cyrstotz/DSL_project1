Part b:

- The temperature at equilibrium water, denoted as \( T_{EW} \), is \( 0^\circ C \).
- Meis is calculated as \( 0.6 \times 0.1 \times y \).
- The change in energy, denoted as \( \Delta E \), is equal to \( Q \) minus \( W_{vu} \).
- The expression \( \text{Meis}(u_2) + m_{EW}(u_2) + M \) is given without further context.
- The volume of equilibrium water at state 2, denoted as \( V_{2EW} \), is equal to \( V_{2} \) times the average of equilibrium water, denoted as \( \overline{EW} \).

Graphical Content Description:

1. A table with three columns and two rows:
   - The first row contains headers: an empty cell, "T", and "V".
   - The second row contains the values: "1", "600°", and "3.14L".

2. A rectangle labeled "1 bar" on the left side and "1.456 bar" on the right side.

3. A small square with no content inside.

4. A box labeled "P_{g,1}" on the left side and "T_{1}" on the right side. Below the box, there is an arrow pointing downwards with the text "Equilibrium" and "through its" and "somehow equal".

Problem 3, Subsection b):

- Refers to another system.
- The change in energy, denoted as \( \Delta E \), is equal to \( Q_{zu} \) minus \( W_v \).
- The equation \( c_v (T_2 - T_1) = m g c_p (T_2 - T_1) - W_v \) is given, followed by an integral of pressure over volume, denoted as \( \int p \, dV \).
- The specific heat at constant pressure, denoted as \( c_p \), is equal to the gas constant \( R \) plus the specific heat at constant volume \( c_v \).