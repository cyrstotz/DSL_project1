d) The mass generation rate is denoted as \( \dot{m}_{\text{gen}} \).

The exergy outflow rate is denoted as \( \dot{e}_{\text{xs, out}} \).

For the energy balance for a steady flow:

The equation is \( G = -\Delta e_{\text{xs}} + \left( 1 - \frac{T_0}{T_e} \right) Q_B - F_{\text{ex, out}} \).

Exergy is calculated as \( \text{Exergy} = -\Delta e_{\text{xs}} + \left( 1 - \frac{T_0}{T_e} \right) Q_B \).

This simplifies to \( = -63.5 \frac{\text{kJ}}{\text{kg}} + \left( 1 - \frac{243.15 \text{K}}{1239 \text{K}} \right) 1135 \frac{\text{kJ}}{\text{kg}} \).

Which results in \( = 506 \frac{\text{kJ}}{\text{kg}} \).

Task 2:

Graph Description: The graph is a plot with the x-axis labeled as entropy in \( \frac{kJ}{kg \cdot K} \) and the y-axis labeled as temperature in Kelvin (K). There are three curves labeled as 1, 2, and 3. Curve 1 starts at the origin and rises steeply, then curves to the right. Curve 2 starts slightly above the origin, follows a similar path to curve 1 but is slightly higher. Curve 3 starts even higher and follows a similar path to curves 1 and 2 but is the highest of the three. There are two horizontal lines intersecting the curves. The first horizontal line is labeled \(0.16\) and intersects all three curves at different points. The second horizontal line is labeled \(0.5 \, \text{kw}\) and intersects the curves at higher points. There is also a label \(0.191 \, \text{bar}\) near the top right of the graph.