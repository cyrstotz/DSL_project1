Section 4(a) is focused on the required P-T (Pressure-Temperature).

Under the subsection "Graphical Descriptions," there are descriptions of two graphs:

First Graph:
- This is a P-T diagram.
- The vertical axis represents Pressure (P) in millibars.
- The horizontal axis represents Temperature (T) in degrees Celsius.
- The vertical axis has a marking at 5 millibars.
- The horizontal axis has markings at -50 degrees Celsius, -20 degrees Celsius, 0 degrees Celsius, and 10 degrees Celsius.
- There are two curves on the graph:
  - The first curve, labeled \( F_{eis1} \), starts from the bottom left, rises steeply, and then curves to the right.
  - The second curve, labeled \( S_{a3} \), also starts from the bottom left but rises less steeply before curving to the right.
- The region above the second curve is labeled "Liquid" (Flüssig).
- The region between the two curves is labeled "Triple Point" (T_{tripel}).
- The region below the first curve is labeled "Water in Foods" (Wasser in Lebensmitteln).

Second Graph:
- This is another P-T diagram.
- The vertical axis represents Pressure (P) in millibars.
- The horizontal axis represents Temperature (T) in degrees Celsius.
- The vertical axis has a marking at 5 millibars.
- The horizontal axis has a marking at 10 degrees Celsius.
- There are two curves on the graph:
  - The first curve, labeled \( F_{eis1} \), starts from the bottom left, rises steeply, and then curves to the right.
  - The second curve, labeled \( S_{a3} \), also starts from the bottom left but rises less steeply before curving to the right.
- There are two points marked on the second curve:
  - Point 1 is labeled \( 1 \) and is located on the curve.
  - Point 2 is labeled \( 2 \) and is located on the curve, with a horizontal line extending to the right labeled "T = constant" (T = const).
- The region above the second curve is labeled "Liquid" (Flüssig).
- The region between the two curves is labeled "Triple Point" (T_{tripel}).