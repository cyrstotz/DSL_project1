The temperature of ice, \( T_{\text{Eis},1} \), is equal to the temperature of ice, \( T_{\text{Eis},2} \), because \( x_2 \) is greater than zero.

The change in internal energy from state 1 to state 2 for ice, \( \Delta u_{12,\text{Eis}} \), is equal to zero minus \( Q_{12} \).

The mass of ice, \( m_{\text{Eis}} \), times the difference in internal energy between state 2 and state 1 for ice, \( (u_{1,2,\text{Eis}} - u_{1,1,\text{Eis}}) \), equals negative \( Q_{12} \).

The internal energy of ice in the liquid state at state 1, \( u_{1,1,\text{Eis},\text{flüssig}} \), is negative 0.045 kilojoules per kilogram, according to Table 1.

The internal energy of ice in the solid state at state 1, \( u_{1,1,\text{Eis},\text{fest}} \), is negative 333.458 kilojoules per kilogram.

The internal energy of ice at state 2, \( u_{1,2,\text{Eis}} \), is equal to the internal energy of ice in the solid state at state 1, \( u_{1,1,\text{Eis},F} \), plus the product of one minus the fraction of ice at state 1, \( (1 - x_{\text{Eis},1}) \), and the difference between the internal energy of ice in the liquid state and the internal energy of ice in the solid state at state 1. This results in negative 200.093 kilojoules per kilogram.

The internal energy of ice at state 1, \( u_{2,1,\text{Eis}} \), is equal to the negative of \( Q_{12} \) divided by the mass of ice, \( m_{\text{Eis}} \), plus the internal energy of ice at state 1, \( u_{1,1,\text{Eis}} \). This results in negative 192.05 kilojoules per kilogram.

The fraction of ice at state 2, \( x_{1,2,\text{Eis}} \), is calculated as the ratio of the difference between the internal energy of ice at state 2 and the internal energy of ice in the solid state at state 2, to the difference between the internal energy of ice in the liquid state and the internal energy of ice in the solid state at state 2, plus one. This results in approximately 0.5759.