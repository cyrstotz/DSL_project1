The equations are as follows:

1. The product of pressure \( p_{s,1} \) and volume \( V_{s,1} \) equals the product of mass \( m_{s,1} \), gas constant \( R \), and temperature \( T_{s,1} \):
\[ p_{s,1} V_{s,1} = m_{s,1} \cdot R T_{s,1} \]

2. The gas constant \( R \) is defined as the universal gas constant \( \overline{R} \) divided by the molar mass \( M_s \):
\[ R = \frac{\overline{R}}{M_s} \]

3. The pressure \( p_{s,1} \) is calculated as the force divided by the area, which is given by:
\[ p_{s,1} = \frac{\text{force}}{\text{area}} = \frac{32 \, \text{kg} \cdot 5.6 \, \text{m/s}^2}{5 \, \text{cm}^2 \cdot \pi} + 1 \, \text{bar} \]
Converting units and calculating, the pressure becomes:
\[ p_{s,1} = 32 \, \text{kg} \cdot \frac{5.6 \, \text{m/s}^2}{5 \, \text{cm}^2 \cdot \pi} + 1 \cdot 10^5 = 35565.5 \, \text{Pa} + 10^5 \, \text{Pa} = 135565.5 \, \text{Pa} \]

4. The mass \( m_{s,1} \) is then calculated using the formula:
\[ \Rightarrow m_{s,1} = \frac{p_{s,1} V_{s,1}}{R T_{s,1}} = \frac{135565.5 \, \text{Pa} \cdot 3.141 \cdot 10^{-3}}{8.314 \, \frac{\text{J}}{\text{mol} \cdot \text{K}} \cdot 77.15 \, \text{K}} = 3.449 \cdot 10^{-3} \, \text{kg} = 0.003449 \, \text{kg} \]