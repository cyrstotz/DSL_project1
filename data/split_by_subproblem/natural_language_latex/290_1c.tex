The total mass \( m_{ges} \) and the external exergy \( ex_{str} \) are equal to the difference between the internal exergy at point 6 \( ex_{inter,6} \) and the internal exergy at point 0 \( ex_{inter,0} \).

This implies that the change in external exergy \( \Delta ex_{str} \) is equal to the enthalpy at point 6 minus the enthalpy at point 0 minus the product of the temperature at point 0 and the difference in entropy between point 6 and point 0, plus the kinetic energy at point 6 minus the kinetic energy at point 0.

The kinetic energy at point 6 \( k_{e6} \) is equal to one half times the square of the velocity at point 0 \( w_0^2 \).

The graph is a T-s diagram with the temperature \( T \) on the vertical axis and the entropy \( s \) on the horizontal axis. The axes are labeled as follows:
- The vertical axis is labeled \( T \) in Kelvin.
- The horizontal axis is labeled \( s \) in kilojoules per kilogram Kelvin.

The graph contains several curves and points:
- There is a black curve starting from the origin (0,0) and rising steeply, then leveling off and fluctuating before rising again.
- Points are marked along the curves with numbers 0 through 6.
- Points 0, 1, and 2 are connected by a black curve.
- Points 2, 3, and 4 are connected by a magenta curve.
- Points 4, 5, and 6 are connected by another magenta curve.
- Points 3 and 5 are connected by a vertical magenta line.
- Points 0, 6, and 4 are connected by a black curve.
- The points are labeled as follows:
  - Point 0 is at the origin.
  - Point 1 is on the initial black curve.
  - Point 2 is on the initial black curve.
  - Point 3 is on the magenta curve.
  - Point 4 is on the magenta curve.
  - Point 5 is on the magenta curve.
  - Point 6 is on the black curve.

There is a note on the right side of the graph that says "ignore the blue curve!"