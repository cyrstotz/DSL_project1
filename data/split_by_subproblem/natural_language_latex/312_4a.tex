In part (a), the pressure \( p_i \) is equal to \( p_3 \) which is 8 bar.

The value for \( p_4 \) is not provided.

See Figure 5.

Verbal Description of Graphs:

The first graph is a plot with the pressure \( p \) on the vertical axis and the temperature \( T \) on the horizontal axis. It displays a dome-shaped curve that starts from the origin, rises to a peak, and then symmetrically falls back down.

The second graph is also a plot with the pressure \( p \) on the vertical axis and the temperature \( T \) on the horizontal axis. This graph includes several lines and points:
- A line labeled "4" begins from the lower left and ascends towards the upper right.
- A line labeled "2" also starts from the lower left and ascends towards the upper right, running parallel to the line labeled "4".
- A line labeled "g" begins from the lower left and ascends towards the upper right, parallel to the other two lines.
- A point labeled "Tripel" is where all the lines converge.
- An arrow labeled "Mengegebiet" points upwards from the point labeled "Tripel".