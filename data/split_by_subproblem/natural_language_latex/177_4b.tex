b) The mass flow rate of R134a is denoted as dot m subscript R134a.

The temperatures are given as T1 equals 23 degrees Celsius and T2 equals negative 22 degrees Celsius.

1. Energy balance over the compressor from state 2 to state 3:

The equation is zero equals the mass flow rate of R134a times (h2 minus h3) plus dot W subscript k.

The enthalpy at state 2, h2, at negative 22 degrees Celsius is 237.08 kilojoules per kilogram.

For state 3, moving from s2 equals s3, where s2 is the entropy at negative 22 degrees Celsius, which is 0.9351 kilojoules per kilogram Kelvin.

The entropy at state 3, s3, is 0.9351 kilojoules per kilogram Kelvin.

The enthalpy at state 3, h3, as a function of pressure in bar and 0.9351 kilojoules per kilogram Kelvin, is calculated as 0.9351 kilojoules per kilogram Kelvin plus (0.9351 minus 0.9066) kilojoules per kilogram Kelvin times (264.18 minus 0.9066) divided by 2.3 equals 264.18 kilojoules per kilogram.

Reiterating the energy balance equation:

Zero equals the mass flow rate of R134a times (h2 minus h3) plus dot W subscript k.

Solving for the mass flow rate of R134a, it is equal to negative dot W subscript k divided by (h2 minus h3), which equals negative 0.028 kilowatts divided by (237.08 kilojoules per kilogram minus 264.18 kilojoules per kilogram) equals 3.349 kilograms per hour.