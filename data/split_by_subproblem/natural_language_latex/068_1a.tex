The rate of change of entropy, S dot, equals zero implies that zero equals the mass flow rate at the inlet times the enthalpy at the inlet minus the mass flow rate at the outlet times the enthalpy at the outlet plus the rate of heat transfer in minus the rate of heat transfer out.

This leads to the rate of heat transfer out being equal to the mass flow rate times the difference in enthalpy between the inlet and the outlet plus the rate of heat transfer in.

It is stated that the moisture is the same at both the inlet and outlet.

This implies that the moisture fraction at the inlet, x_e, equals zero and the moisture fraction at the outlet, x_f, equals one.

From the table, the enthalpy at the inlet, h_e, at a pressure of 700 kPa is 292.98 kilojoules per kilogram. The enthalpy at the outlet, h_aus, at a pressure of 1000 kPa is 434.04 kilojoules per kilogram.

The rate of heat transfer out is then calculated as the mass flow rate times the difference in enthalpy (292.98 minus 434.04 kilojoules per kilogram) plus 200 kilojoules per second.

Substituting the values, the rate of heat transfer out is 0.3 kilograms per second times the difference in enthalpy (292.98 minus 434.04 kilojoules per kilogram) plus 200 kilojoules per second.

Finally, the rate of heat transfer out is calculated to be 62.438 kilowatts.