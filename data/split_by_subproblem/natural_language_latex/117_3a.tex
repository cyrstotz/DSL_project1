3a) The specific heat capacity at constant volume, c_v, is 0.633 kilojoules per kilogram Kelvin. The molar mass of the gas, M_g, is 50 kilograms per kilomole.

The gas constant, R_g, is calculated as 8.314 kilojoules per kilomole Kelvin divided by 50 kilograms per kilomole, which simplifies to R_g over M_g, resulting in 0.16628 kilojoules per kilogram Kelvin.

The initial pressure of the gas, P_{g,1}, is calculated using the formula involving mass m, gravitational acceleration g, and diameter d, but this formula is crossed out.

Instead, P_{g,1} is calculated as the ambient pressure plus the pressure due to the mass of the gas over the area, which is given by the mass times gravitational acceleration divided by pi times the square of half the diameter. This calculation results in 1.3997 bar after adding 7 bar, but the addition is crossed out, and the final pressure is 1.3997 bar.

The mass of the gas, m_g, is calculated using the ideal gas law formula: pressure times volume divided by the gas constant times temperature. Substituting the values, the mass of the gas is found to be 0.00341876 kilograms, which is equivalent to 3.41876 grams.