The graph described is a pressure-temperature diagram, labeled with temperature in degrees Celsius on the x-axis and pressure in millibar on the y-axis. It features a curve that delineates the liquid and gas phases, starting from the origin and rising towards the right. A point on the curve is labeled as "Triple point". There are three marked points on the graph: point 1 is located in the liquid region, point 2 is on the boundary curve, and point 3 is in the gas region. These points are connected by arrows showing a process sequence from point 1 to point 2, and then from point 2 to point 3.

In the equations section:
- The enthalpy at point 3, denoted as \( h_3 \), is calculated for a pressure of 8 bar and entropy of 0.9298 kilojoules per kilogram Kelvin, referring to Table A12.
- The enthalpy for an entropy of 0.9066 kilojoules per kilogram Kelvin is given as 264.15 kilojoules per kilogram.
- The enthalpy at point 3, for the same entropy of 0.9298 kilojoules per kilogram Kelvin, is calculated but the result is not directly provided. Instead, it is computed using a formula involving the difference in enthalpy and entropy values between two states, resulting in an addition of 264.15 kilojoules per kilogram.
- The enthalpy for an entropy of 0.9399 kilojoules per kilogram Kelvin is 273.66 kilojoules per kilogram.

The mass flow rate of R1342 is calculated using the formula involving the power of -28 times 10 to the power of -3 kilowatts divided by the difference in enthalpy between states 2 and 3. The result is 0.000834 kilograms per second, which converts to 3.0024 kilograms per hour.