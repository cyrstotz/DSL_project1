- There is a graph with the vertical axis labeled P and the horizontal axis labeled T.
- A line starts from the origin and goes upwards to the right, labeled \( T_{\text{inel}} \).
- Another line starts from the origin and goes upwards to the right, but at a steeper angle.
- A horizontal line intersects the steeper line at point 1 and is labeled \( p_4 \).
- A vertical line goes down from point 1 to point 2 on the first line.
- A dashed line goes from point 2 downwards to the left, ending at point 3 on the horizontal axis, labeled \( T_z \).

The equations are:
- \( S_4 \) equals \( S_{1,4} \)
- \( S_4 \) equals \( S_f \) plus
- \( X \) equals the fraction where the numerator is \( S_4 - S_f \) and the denominator is \( S_9 - S_f \)