The problem asks for the value of the rate of entropy production, denoted as S dot subscript "er2".

The rate of change of entropy S with respect to time t at steady state is given by the equation:
m dot times Q dot subscript "aus" divided by T subscript "KF" plus S dot subscript "er2".

The expression for S dot subscript "er2" is given by:
m dot times (h subscript a minus h subscript e) minus Q dot subscript "aus" divided by T subscript "KF".

The enthalpy at 298.15 Kelvin, denoted as h subscript a, is equal to h subscript f of T subscript 2 plus v superscript i times (p minus p subscript "sat" of T).

The enthalpy at 288.15 Kelvin, denoted as h subscript e, is equal to h subscript f of T subscript a plus v subscript i times (p minus p subscript "sat" of T).

The change in enthalpy, Delta h, is equal to h subscript f of T subscript 2 minus h subscript f of T subscript a, which is equal to c subscript lp times (T subscript 2 minus T subscript a).

Figure Description:
There is a diagram showing a rectangular box with a smaller rectangle inside it. The smaller rectangle is labeled "298.15 K" on the top and "288.15 K" on the bottom. There is an arrow pointing to the right labeled "prisoner". To the right of the diagram, there is a label:
Q dot subscript "aus" equals 65 kilowatts,
T subscript "KF" equals 295 Kelvin.