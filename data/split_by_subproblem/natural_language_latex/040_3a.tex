P sub g,1 equals P sub omb plus P sub EW plus P sub MK.
P sub EW equals M sub EW divided by (pi times (D over 2) squared over 4), and P sub MK equals M sub KG divided by (pi times (D over 2) squared over 4).
P sub g,1 equals 1 bar plus 127.91 Pascals plus 39969.3 Pascals.
P sub g,1 equals 1.49 bar.

P sub g,1 times V sub g,1 equals m sub g times R times T sub g,1, where R equals R over M equals 8.314 Joules per mole Kelvin times 1 over 80 kilograms per kilomole.
m sub g equals (P sub g,1 times V sub g,1) divided by (R times T sub g,1) equals 0.005492 kilograms.
R equals 166.28 Newton meters per kilogram Kelvin.
m sub g equals 3.92 grams.