The equation for the rate of entropy generation, S dot subscript "erz", equals the mass flow rate, m dot, times the sum of entropy at the inlet, s subscript "ein", and entropy at the outlet, s subscript "aus", plus the rate of heat transfer, Q dot subscript "zu", divided by the temperature, T subscript "w".

The average temperature, T bar, equals the difference in enthalpy at the outlet, h subscript "aus", and enthalpy at the inlet, h subscript "ein", divided by the difference in entropy at the outlet, s subscript "aus", and entropy at the inlet, s subscript "ein".

From Table A-2:

The enthalpy at the outlet, h subscript "aus", at 100 degrees Celsius equals 419.04 kilojoules per kilogram.

The enthalpy at the inlet, h subscript "ein", at 70 degrees Celsius equals 232.95 kilojoules per kilogram.

The entropy at the inlet, s subscript "ein", at 70 degrees Celsius equals 0.9503 kilojoules per kilogram Kelvin.

The entropy at the outlet, s subscript "aus", at 100 degrees Celsius equals 1.3069 kilojoules per kilogram Kelvin.

For item c):
The change in entropy, Delta S subscript "12", equals the mass at state 2, m subscript "2", times the entropy at state 2, s subscript "2", minus the mass at state 3, m subscript "3", times the entropy at state 3, s subscript "3".
The mass at state 2, m subscript "2", equals the mass at state 1, m subscript "1", plus the change in mass at state 2, Delta m subscript "2".
The entropy at state 1, s subscript "1", at 100 degrees Celsius equals 7.306 kilojoules per kilogram Kelvin.
The entropy at state 2, s subscript "2", at 20 degrees Celsius equals 0.5545 kilojoules per kilogram Kelvin.