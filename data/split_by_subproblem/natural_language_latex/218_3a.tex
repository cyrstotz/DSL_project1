The force \( F_n \) is equal to the product of mass \( m_k \) and gravity \( g \), plus the product of the mass of the extra weight \( m_{ew} \) and gravity \( g \), plus the pressure \( p_0 \).

Substituting the values, we have negative 32 kilograms times 9.81 meters per second squared plus 0.1 kilograms times 9.81 meters per second squared.

This equals 314.8 kilograms times meters per second squared plus 100,000 Pascals.

The pressure \( p_g \) times the area \( A \) equals \( F_n \).

Solving for \( p_g \), we have \( p_g \) equals the sum of 314.8 and 100,000 Pascals divided by \( A \).

The area \( A \) is equal to the square of the radius \( r \) times pi, which is 0.05 meters squared times pi, resulting in 0.00785 square meters.

Thus, \( p_g \) equals the sum of 314.8 and 100,000 Pascals divided by 0.00785 square meters, which equals 1.4 bar.

The mass \( m_g \) is calculated using the formula \( m_g = \frac{pV}{RT} \).

Substituting the values, we have \( \frac{1.5 \times 10^5 \times 3.14 \times 10^{-3}}{4668 \times 773 \, \text{K}} \).

This results in 3.166 grams.

The pressure \( p_g \) is 1.5 bar.

The temperature \( T \) is 773 Kelvin.

The volume \( V \) is 0.3 times 3.14 times \( 10^{-3} \) cubic meters.

The gas constant \( R \) is calculated as \( R \) divided by the molar mass \( M \).

This results in \( \frac{8.314 \, \text{m}^3 \cdot \text{Pa}}{\text{K} \cdot \text{mol}} \) divided by 50 kilograms per mole, which equals 166.28 Joules per mole.