The rate of exergy change, denoted as E-dot-ex, is given by the formula:
E-dot-ex equals m-dot times the quantity h-zero minus h-six minus T-zero times the quantity s-zero minus s-six plus q-dot.

The difference in enthalpy between state zero and state six, h-zero minus h-six, is calculated using the specific heat capacity at constant pressure, c-p, and the temperature difference T-zero minus T-six. This is given by:
h-zero minus h-six equals c-p times the quantity T-zero minus T-six equals 7000 Joules per kilogram-Kelvin times the quantity 340.01 Kelvin minus 273.15 Kelvin plus 30 Kelvin, which equals 57.07 Joules per kilogram.

The term T-zero times the quantity s-zero minus s-six is calculated as:
T-zero times the quantity s-zero minus s-six minus T-zero times c-p times the natural logarithm of the ratio T-zero over T-six equals the quantity 273.15 Kelvin minus 30 Kelvin minus 1.006 Joules per kilogram-Kelvin times the natural logarithm of the ratio 340.01 over 273.15 Kelvin minus 30 Kelvin, which equals 170 Joules per kilogram.

The heat transfer rate per unit mass, q-dot, is calculated using the velocities at state two and state zero:
q-dot equals the quantity v-two squared minus v-zero squared over two equals 170 Joules per kilogram.

Finally, substituting these values back into the formula for E-dot-ex, we get:
E-dot-ex equals m-dot times the quantity 57 minus 52 plus 170, which equals 63 Joules per kilogram.