The rate of heat output is denoted by Q dot subscript "aus".

The derivative with respect to time of the sum of the product of mass and specific enthalpy of each component, plus xi times heat minus xi times work, equals zero.

The mass flow rate entering is equal to the mass flow rate exiting, and both are 0.3 kilograms per second.

The specific enthalpy entering is the specific enthalpy at 80 degrees Celsius, and the specific enthalpy exiting is the specific enthalpy at 40 degrees Celsius.

h1, h2 plus v times (r1 times g minus h times f) at 10 kilojoules equals 252.38 plus 0.005 times (2626.8 minus 292.88), which results in 304.64, and this is one third of the total.

h2 equals 448.04 plus 0.005 times (2676.1 minus 448.04), which results in 430.32, and this is also one third of the total.

The heat input Q subscript "zu" equals the mass flow rate times (h1 minus h2) plus the rate of heat input minus the rate of heat output.

The rate of heat output is equal to the mass flow rate times (h1 minus h2) plus the rate of heat input, which equals 0.3 times (304.64 minus 430.32) plus 100 kilowatts, resulting in 62.286 kilowatts, which is approximately 62.3 kilowatts.