The volume of the gas, denoted as \( V_{\text{Gas}} \), is calculated using the formula \( \pi \cdot r^2 \cdot h \), which equals \( 3.14 \cdot 1 \) and results in \( 3.14 \cdot 10^{-3} \) cubic meters. The area \( A \) is calculated as \( \frac{1}{4} \cdot \pi \cdot d^2 \) which equals \( 0.7854 \) square meters. The temperature \( T_{\mu} \) is converted from \(-500^\circ \text{C}\) to \( 773.15 \) Kelvin.

There is a diagram with two horizontal arrows: one pointing downwards labeled with \( 10^5 \text{Pa} / A \) and another pointing upwards labeled with \( 32 \, \text{kg} \cdot g \).

The pressure of the gas, \( P_{\text{Gas,1}} \), is calculated as \( 10^5 \, \text{Pa} \) plus \( 32 \, \text{kg} \cdot 9.81 \, \frac{\text{m}}{\text{s}^2} \cdot \frac{1}{0.7854 \, \text{m}^2} \). It is further simplified and recalculated, resulting in \( 10^5 + 10,3247.3222 \, \text{Pa} \).

The mass of the gas \( m_g \) is determined by the formula \( M_g \cdot \frac{P_1 \cdot V_1}{R \cdot T_1} \) where \( M_g \) is the molar mass, \( P_1 \) is the pressure, \( V_1 \) is the volume, \( R \) is the gas constant, and \( T_1 \) is the temperature. This results in \( 50 \cdot \frac{10^3 \, \text{m}^3 \cdot \text{Pa}}{\text{mol} \cdot \text{K}} \). The mass \( m_g \) is also expressed as \( 2 \left( \frac{4.8 \, \text{kg}}{\text{mol}} \right) \). The molar mass \( M_g \) is given as \( 50 \left( \frac{\text{kg}}{\text{mol}} \right) \) and then converted to \( 50 \cdot 10^{-3} \left( \frac{\text{kg}}{\text{mol}} \right) \).