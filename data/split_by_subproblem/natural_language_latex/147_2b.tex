Energy analysis of a stationary flow process:

The ratio of temperature T6 to T5 is equal to the ratio of pressure P6 to P5 raised to the power of (n-1)/n. Therefore, T6 equals T5 times the ratio of P6 to P5 raised to the power of (n-1)/n, which equals 293.15 Kelvin times the ratio of 0.4 bar to 0.5 bar raised to the power of 1.4 over 1.4, resulting in a temperature of 328.07 Kelvin.

Energy balance of a stationary flow process:

Zero equals the mass flow rate times the difference in enthalpy between state 5 and state 6 plus half the difference in the squares of the velocities at state 5 and state 6. This simplifies to zero equals the difference in enthalpy between state 5 and state 6 plus half the difference in the squares of the velocities at state 5 and state 6.

The difference in enthalpy between state 6 and a reference state s multiplied by 2 equals half the difference in the squares of the velocities at state 6 and state s. The square of the velocity at state 6 equals the square of the velocity at state s minus two times the difference in enthalpy between state 6 and state s, assuming a constant specific heat at constant pressure and ideal gas behavior.

The velocity at state 6 equals the square root of the square of the velocity at state s minus two times the specific heat at constant pressure times the ratio of mass flow rates (which cancels out) times the difference in temperature between state 6 and state s. This results in the square root of 220 square meters per second squared minus two times 1.006 kilojoules per kilogram Kelvin times the difference in temperature between 328.07 Kelvin and 293.15 Kelvin, resulting in a velocity of 50.25 meters per second.

The reference temperature T0 is calculated as 273.15 Kelvin minus 30 Kelvin, resulting in 243.15 Kelvin.