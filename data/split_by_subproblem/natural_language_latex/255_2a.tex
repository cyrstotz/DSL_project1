The first graph is a plot with the vertical axis labeled as T times s over k, and the horizontal axis is unlabeled. The graph contains a wavy line that starts from the origin, rises to a peak, falls, rises again to a higher peak, and then falls again.

The second graph is a plot with the vertical axis labeled as T times s over k and the horizontal axis labeled as s in units of kilojoules per kilogram Kelvin. The graph contains several points labeled from 0 to 6. The points are connected by lines as follows:
- A steep upward line from Point 0 to Point 1.
- A less steep upward line from Point 1 to Point 2.
- A horizontal line from Point 2 to Point 3.
- A downward line from Point 3 to Point 4.
- A wavy line from Point 4 to Point 5.
- A vertical downward line from Point 5 to Point 6.
The points are connected by smooth curves or straight lines, and the graph includes several horizontal lines indicating different levels of T times s over k. Point 5 is labeled with Ps equals Pu and Point 6 is labeled with Po.

The graph is a complex, hand-drawn diagram with multiple intersecting lines and annotations. The x-axis is labeled with "T" (temperature) and the y-axis is labeled with "p" (pressure). There are several regions and points marked on the graph:
- A point labeled "Triple" (triple point).
- Regions labeled "sublimation" and "condensation".
- Multiple phases are indicated, including "solid," "liquid," and "gas."
- The graph includes several curves and lines that intersect at various points, indicating phase transitions.
- There are arrows and numbers (1, 2, 3, etc.) indicating specific points or paths on the graph.

For the heat transfer system, the change in entropy from state 1 to state 2 is given by the equation:
Delta S12 equals delta m2 times s2 plus Q_R over T plus dot xi_m.