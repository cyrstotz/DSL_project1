The problem includes a table with three rows and seven columns. The entries in the table are as follows:
- First row: 0, 1, 2, 3, 4, 5, 6
- Second row: 0.5, 0.151, empty, empty, empty, empty, 0.151
- Third row: 431.3K, empty, empty, empty, empty, empty, 213.15
- Fourth row: 220, empty, empty, empty, empty, empty, empty

The equations provided are:
- The gas constant R is equal to c_p minus c_v, which equals 2.87943 kilojoules per kilogram Kelvin, which is also 2.87943 Joules per kilogram Kelvin.
- The specific heat at constant volume for liquid, c_{vL}, is equal to the specific heat at constant pressure for liquid, c_{pL}, divided by k, which equals 0.78057 kilojoules per kilogram Kelvin.
- The process from state 5 to state 6 is an adiabatic reversible and isentropic process, meaning the entropy at state 5 equals the entropy at state 6, s_5 equals s_6.
- The entropy at temperature T_5 is 2.66533 kilojoules per kilogram Kelvin.
- The temperature at state 6, T_6, is equal to the temperature at state 5, T_5, multiplied by the ratio of pressure at state 6 to pressure at state 5 raised to the power of R divided by c_{vL}, which results in 2.66533 kilojoules per kilogram Kelvin.
- The temperature at state 6, T_6, is 328.097 Kelvin.
- The change in entropy from state 4 to state 2, Δs_{42}, is equal to c_p times the natural logarithm of the ratio of temperature at state 4 to temperature at state 2.
- The first heat system equation from state 5 to state 6 is zero equals m_H times the difference in enthalpy from state 6 to state 5 plus half the work done from state 6 to state 5.
- This results in m_H times the difference in temperature from state 6 to state 5 plus the work done from state 6 to state 5 equals 236.755 meters per second.
- This is also equal to the square root of two times k times the sum of enthalpies at state 6 and state 5 equals the square of the work done from state 6 to state 5.