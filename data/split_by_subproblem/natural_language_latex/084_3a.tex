a) \( p_{g,1} \), \( m_g \) in the cylinder:

The specific heat at constant volume \( Cv \) is \( 0.623 \frac{J}{kgK} \).

The pressure \( p_{g,1} \) inside the cylinder is given by the sum of the ambient pressure \( p_{amb} \), the term \( \left( \frac{m_g}{A_{Zyl}} \right) \), and the term \( \left( \frac{m_{W,1}g}{A_{Zyl}} \right) \).

This results in \( \Delta h^3 Pq \left( \frac{32.9}{0.00785 \, m^2} \right) + \left( \frac{0.1 \cdot 9.81}{0.00785} \right) \).

Which equals \( 10^5 \, m + 39,983.8 \, Pa + 1.25 \, kPa \).

This simplifies to \( 1.4 \, bars \).

The area \( A_{Zyl} \) of the cylinder is calculated as \( \left( \frac{D}{2} \right)^2 \cdot \pi \), which equals \( (0.05 \, m)^2 \cdot \pi \).

The mass \( m_g \) of the gas is given by \( p_{A,g} \cdot V_{g,1} = m_g \cdot R_g \cdot T_{g,1} \).

The gas constant \( R_g \) is \( \frac{8.314}{50 \, \frac{kg}{mol}} = 166.28 \).

Thus, \( m_g \) is calculated as \( \frac{p_{A,g} \cdot V_{g,1}}{R_g \cdot T_{g,1}} = \frac{1.4 \, bar \cdot 3.14 \cdot 10^{-3} \, m^3}{166.28 \cdot 773 \, K} = 0.00342 \, kg \).

Which is \( 3.42 \, g \).

The volume \( 3.14 \, L \) is \( 3.14 \, dm^3 \), which equals \( 3.14 \cdot 10^{-3} \, m^3 \).

The temperature \( 500^\circ C \) is \( 773 \, K \).