The change in internal energy, denoted as Delta u, equals zero. The specific heat capacity of water, denoted as C_w, is 4.214 Joules per gram Kelvin. The temperature T_AB is defined as A minus 1 Theta.

The equation involves the mass of gas at state 1, denoted as m_g,1, multiplied by the specific heat at constant volume, C_v, and the difference between the temperature at state 1, T_1, and the equilibrium temperature, T_eq. Additionally, the mass of evaporated water, denoted as m_ew, multiplied by the specific heat capacity of water, C_w, and the difference between the initial temperature of evaporated water, T_ew,1, and the equilibrium temperature, T_eq.

This leads to the equilibrium temperature, T_eq, being calculated as the sum of the product of m_g,1, C_v, and T_1, and the product of m_ew, C_w, and T_ew,1, all divided by the sum of the product of m_g,1 and C_v, and the product of m_ew and C_w. The result is 275.71 Kelvin, which equals 2.56 degrees Celsius.

The pressure p_3,2 equals the pressure p_3,1, indicating that if we want to maintain isothermal conditions at another point, the change in internal energy, Delta u, remains zero, thus the pressure remains the same.