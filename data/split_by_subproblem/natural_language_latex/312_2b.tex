The equations describe a flow process:

1. Zero equals the mass flow rate times the difference in enthalpy between state 5 and state 6 plus half the difference in the squares of the velocities at state 5 and state 6, plus the heat flow rate, which is not equal to the power flow rate.

2. Zero equals the specific heat capacity of air times the temperature difference between state 5 and state 6 plus half the difference in the squares of the velocities at state 5 and state 6.

3. Zero equals the mass flow rate times the difference in entropy between state 5 and state 6 plus the heat flow rate divided by temperature plus the entropy generation, assuming the process is reversible (denoted by 0, reversible).

4. The difference in entropy between state 5 and state 6 equals the standard entropy at temperature T5 minus the standard entropy at temperature T6 minus the specific gas constant of air times the natural logarithm of the ratio of pressure at state 5 to pressure at state 6, which equals zero.

5. The specific gas constant of air equals the universal gas constant divided by the molar mass of air, calculated as 0.287 kilojoules per kilogram Kelvin.

Continuation:

1. The standard entropy at temperature T6 equals the standard entropy at temperature T5 minus the specific gas constant of air times the natural logarithm of the ratio of pressure at state 5 to pressure at state 6, which simplifies to the standard entropy at temperature T5 minus the specific gas constant of air times the natural logarithm of the ratio of pressure at state 5 to a reference pressure.

2. The standard entropy at temperature T5 is calculated using a linear interpolation between the standard entropies at 430 Kelvin and 440 Kelvin to find the value at 434.5 Kelvin, resulting in 2.0638 kilojoules per kilogram Kelvin.

3. The standard entropy at temperature T6 is given as 1.7336 kilojoules per kilogram Kelvin. The temperature T6 is calculated using a linear interpolation method, resulting in 328.6 Kelvin.

4. Zero equals the specific heat capacity of air times the temperature difference between state 5 and state 6 plus half the difference in the squares of the velocities at state 2 and state 6. The square of the velocity at state 6 is calculated using the square of the velocity at state 5 plus twice the specific heat capacity of air times the temperature difference between state 5 and state 6, resulting in a velocity at state 6 of 390.26 meters squared per second.