The total pressure \( p_{ges} \) is equal to the sum of the ambient pressure \( p_{amb} \), the pressure due to the weight \( p_{ew} \), and the pressure due to the force \( p_{u} \). The ambient pressure \( p_{amb} \) is given as 1 bar, which is equal to \( 10^5 \) Pascals.

The area of the membrane and the piston is calculated as:
\[ A = 7 \pi^2 = 2 \cdot \pi \cdot 0.05^2 \]
which results in \( A = 0.0157 \) square meters.

The pressure due to the weight \( p_{ew} \) is calculated using the formula:
\[ p_{ew} = \frac{F}{A} = \frac{0.1 \cdot 9.81}{0.0157} = 62.48 \) Pascals.

The pressure due to the force \( p_{u} \) is calculated as:
\[ p_{u} = \frac{F}{A} = \frac{32 \cdot 9.81}{0.0157} = 199964.90 \) Pascals.

The total pressure \( p_{ges} \) is then calculated by adding the ambient pressure, the pressure due to the weight, and the pressure due to the force:
\[ p_{ges} = 10^5 \, \text{Pa} + 62.48 \, \text{Pa} + 199964.90 \, \text{Pa} \]
\[ = 120052.38 \, \text{Pa} \]
which is approximately \( 1.2 \) bar.