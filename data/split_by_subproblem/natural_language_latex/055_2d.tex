Q_in minus dE/dt is greater than the sum of dot E_stor plus the sum of dot E_xa minus the sum of [dot W_n minus p_0 times dV/dt] minus dot E_ver. 

Zero equals dot m times [h_e minus h_a minus T_0 times (s_e minus s_a) plus ke plus pe] plus the sum of (1 minus T_0 over T) times dot Q_j minus the sum of dot W_in minus dot E_ver.

Exer equals exer.

exer equals h_0 minus h_6 minus T_0 times (s_0 minus s_6) plus ke plus pe minus dot Q_ex over dot m minus the sum of dot W_ex over dot m.

Find the sum of (1 minus T_0 over T) times dot Q_j over dot m.

This equals the sum of (1 minus T_0 over T) times q_j which equals (1 minus 243.15 K over 1289 K) times 1185 kJ/kg which equals 968.581 kJ/kg.

The conditions are: 0 is less than eta and eta is less than 1, eta_vs is less than 1, eta_vs equals reversible over actual which equals (h_0 minus h_ns) over (h_0 minus h_1), and the energy balance is 0 equals dot m times (h_0 minus h_1) plus dot Q minus dot W.

dot W over dot m equals W_tot equals h_0 minus h_4 equals C_p dot m times (T_0 minus T_1) equals 1.006 times (243.15 K minus T_1).

Find T_1.

In steps 1-4, zero equals dot m times (h_4 minus h_1), h_4 equals h_1, and since IG m.f. const. C_p implies T_4 equals T_1.

Find T_4.

p_4 equals p_3, so steps 3-4 are isentropic, s_3 equals s_4, therefore T_4 equals T_3 due to p_6 proportional to T which equals 1289 K.

W_tot equals 1.006 times (243.15 minus 1289) equals -1272.438 kJ/kg.

Exer equals h_0 minus h_6 minus T_0 times (s_0 minus s_6) plus 968.581 plus 1272.438 equals C_p times (T_0 minus T_6) minus T_0 times (s_0 minus s_6).