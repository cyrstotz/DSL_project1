The graph is a plot with the x-axis labeled as entropy in kilojoules per kilogram and the y-axis labeled as temperature in Kelvin. It features three curves:
1. The first curve, labeled "Isobare," starts at the origin, rises steeply, peaks, and then falls, resembling a bell curve.
2. The second curve is a straight line with a positive slope, labeled "Isotherm."
3. The third curve is another straight line with a positive slope, steeper than the second curve, also labeled "Isotherm."

The graph is a Temperature-Entropy (T-S) diagram. The vertical axis is labeled as Temperature (T) and the horizontal axis as Entropy (S). It includes three isobaric lines (constant pressure) drawn diagonally from the bottom left to the top right, labeled as "Isobar." There are three points labeled 1, 2, and 3:
- Point 1 is at the intersection of the lowest isobar and the vertical axis.
- Point 2 is on the highest isobar.
- Point 3 is on the middle isobar.
A vertical line connects points 1 and 3, and a horizontal line connects points 3 and 2. A curved line connects points 2 and 1. The region between the vertical and horizontal lines is labeled as "liquefaction," and the region between the horizontal and curved lines is labeled as "evaporation."

The text "Verflüssigen sind adiab. reversib. isotherm" translates to "Liquefaction processes are adiabatic reversible isothermal." The repeated word "isotherm" emphasizes the isothermal nature of the process.