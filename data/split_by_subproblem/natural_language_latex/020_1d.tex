d) The equation M subscript 2 h subscript 2 minus M subscript 1 h subscript 1 equals Delta M subscript 12 h subscript ein plus Q subscript Rm. This implies M subscript 2 h subscript 2 equals M subscript 1 plus Delta M subscript 12 equals Q subscript Rm equals M subscript 1 times (h subscript 2 minus h subscript 1) plus Delta M subscript 12 times (h subscript 2 minus h subscript 12,ein). This further implies Delta M subscript 12 equals the fraction of (Q subscript R,12 minus M subscript 1 times (h subscript 2 minus h subscript 1)) over (h subscript 12,ein minus h subscript ein) equals the fraction of (U subscript 2 minus U subscript ein) over 7 subscript TAB-A2 equals U subscript 2 equals U subscript f at 20 degrees Celsius equals 292.95 kilojoules per kilogram. U subscript f equals U subscript f at 100 degrees Celsius plus x subscript 0 times (U subscript g at 100 degrees Celsius minus U subscript f at 100 degrees Celsius) equals 419.34 kilojoules per kilogram plus 0.005 times (2506.5 minus 419.34) kilojoules per kilogram equals 429.37 kilojoules per kilogram. h subscript g equals 429.37 kilojoules per kilogram and h subscript ein equals h subscript f at 20 degrees Celsius 7 subscript TAB-A2 equals 83.96 kilojoules per kilogram. Delta U subscript 12 equals the fraction of (35,000 kilojoules per second minus 5.755 kilograms per second times (292.95 minus 429.37) kilojoules per kilogram) over 83.96 kilojoules per kilogram equals 392.4 kilograms per second.