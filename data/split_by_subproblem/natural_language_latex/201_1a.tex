a) GS: The heat output rate, denoted as Q-dot-out.

Implies that zero equals the mass flow rate in (m-dot-in) times the difference in enthalpy between the entrance (h_e) and the exit (h_a) plus the heat transfer rate (Q-dot-R), which equals the heat output rate (Q-dot-out).

The enthalpy at the entrance conditions (70 degrees Celsius, 1 bar) equals the enthalpy of vaporization at 70 degrees Celsius plus 292.98 kilojoules per kilogram, and the enthalpy of fluid at 70 degrees Celsius equals 292.98 kilojoules per kilogram.

The enthalpy at the exit conditions (100 degrees Celsius, 1 bar) equals the enthalpy of vaporization at 100 degrees Celsius plus 2257 kilojoules per kilogram, and the enthalpy of fluid at 100 degrees Celsius equals 419.04 kilojoules per kilogram.

Implies that the heat output rate (Q-dot-out) equals the mass flow rate in (m-dot-in) times the difference in enthalpy between the entrance (h_e) and the exit (h_a) plus the heat transfer rate (Q-dot-R), which equals 76.98 kilowatts and 62.78 kilowatts.

a) GS: The change in mass of the substance, denoted as Delta m_s.

Zero equals the mass flow rate of water (m-dot-w) times the difference in enthalpy between the entrance (h_e) and the new condition (h_n) plus the heat transfer rate (Q-dot-R), which leads to the heat output rate (q_aus).

The change in mass of the substance (Delta m_s) equals the mass flow rate of water (m-dot-w) times the change in time (Delta t).

The temperature changes from 100 degrees Celsius to 70 degrees Celsius.