a) p_G,1, m_g

The pressure p_G,1 is equal to p_0 plus the mechanical force F_mech divided by the area A, which simplifies to p_0 plus the sum of 32 kg times 9.81 meters per second squared plus 0.1 kg times 5.81 meters per second squared, all divided by A.

The pressure p_G,1 equals 1.1 bar, which implies that the change in pressure, Delta p, is 1.4 bar.

Using the ideal gas law pV equals mRT, the mass m_g is calculated as the pressure p_G,1 times the volume V_G,1 divided by the gas constant R times the temperature T_G,1. This results in 1.4 bar times 3.14 times 10 to the power of negative 3 cubic meters, divided by 8.314 Joules per mole per Kelvin times 773.15 Kelvin.

The mass m_g equals 0.00342 kg, which is 3.42 grams.