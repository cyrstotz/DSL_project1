The equations are as follows:

1. The product of pressure (p) and volume (V) equals the product of mass (m), the gas constant (R), and temperature (T), where the sum of the mass of water vapor and ice is 0.16 kilograms.
2. The sum of forces equals zero.
3. The equation for pressure at state 1 (p_s1) times area (A) minus the product of 0.1 and gravitational acceleration (g) minus the product of 32 and g minus the product of ambient pressure (p_amb) and area (A) equals zero.
4. The pressure at state 1 (p_s1) is calculated as the sum of the product of (0.1 plus 32) and gravitational acceleration (g) divided by area (A) plus the ambient pressure (p_amb).
5. The area (A) is calculated as the square of half of 10 millimeters, times pi, which equals 0.0078 square meters.
6. The pressure at state 1 (p_s1) is calculated as the sum of the product of (0.1 plus 32) and 9.81 (value for g) divided by 0.0078, plus 110,000 (expressed in scientific notation as 1.10^5), resulting in 1.4 bar.
7. The gas constant (R) is calculated as the universal gas constant (8314) divided by the molar mass (M), which is 50, resulting in 166.28.
8. The mass of gas at state 1 (m_g,1) is calculated using the ideal gas law, where it is the product of pressure at state 1 (p_s1), volume of gas at state 1 (V_g,1) divided by the product of the gas constant (R) and temperature of gas at state 1 (T_g,1), resulting in 0.0034 kilograms or 3.4 grams.

Energy balance equations:
1. The difference between heat transfer (Q) and work done (W) equals the change in internal energy (ΔU) plus the change in kinetic energy (ΔKE) plus the change in potential energy (ΔPE).
2. The heat transfer from state 1 to state 2 (Q_12) equals the work done from state 1 to state 2 (W_12) plus the change in internal energy from state 1 to state 2 (ΔU_12).
3. The change in internal energy from state 1 to state 2 (ΔU_12) is calculated as the difference in internal energy between state 2 and state 1, which is the product of the mass of gas (m_g) and the difference in specific internal energy (u_2 - u_1), which is further calculated as the product of specific heat at constant volume (c_v) and the temperature difference between state 2 and state 1 (T_g,2 - T_g,1).
4. The result of the calculation is -1.14 kilojoules.