The change in energy rate for the system, denoted as Delta dot E subscript x, sis, equals the change in energy rate for the surroundings, denoted as Delta dot E subscript x, siro.

The change in energy rate for the sir, denoted as Delta dot E subscript x, sir, equals the mass flow rate of the gas, denoted as dot m subscript gs, times the expression in parentheses: h subscript 6 minus h subscript 0 minus T subscript 0 times (s subscript 6 minus s subscript 0) plus omega subscript 6 squared divided by 2, minus the mass flow rate of the gas times the expression in parentheses: h subscript 0 minus h subscript 0 minus T subscript 0 times (s subscript 5 minus s subscript 0) plus omega subscript 5 squared divided by 2.

This simplifies to the mass flow rate of the gas times the expression in parentheses: h subscript 6 minus h subscript 0 minus T subscript 0 times (s subscript 6 minus s subscript 0) plus omega subscript 6 squared divided by 2 minus omega subscript 5 squared divided by 2. The term "Stoffmeckell" is mentioned, which might be a reference or a label.

The difference h subscript 6 minus h subscript 0 equals c subscript p times (T subscript 6 minus T subscript 0), and the difference s subscript 6 minus s subscript 0 equals c subscript p times the natural logarithm of (T subscript 6 divided by T subscript 0). The natural logarithm of 1 equals 0.

The change in specific energy rate for the sir, denoted as Delta dot e subscript x, sir, equals the energy rate for the sir divided by the mass flow rate of the gas, which simplifies to c subscript p times (T subscript 6 minus T subscript 0) minus T subscript 0 times the expression in parentheses: c subscript p times the natural logarithm of (T subscript 6 divided by T subscript 0) minus omega subscript 6 squared minus omega subscript 5 squared divided by 2.

This further simplifies to 1.006 kilojoules per kilogram Kelvin times (T subscript 6 minus T subscript 0) minus 1.006 times 10 to the power of 3 Joules per kilogram times T subscript 0 times the natural logarithm of (T subscript 6 divided by T subscript 0) minus omega subscript 6 squared minus omega subscript 5 squared divided by 2.

This results in 37,551,579 Joules per kilogram minus 37,551 kilojoules per kilogram equals 37,551 kilojoules per kilogram, which equals the change in specific energy rate for the sir, denoted as Delta dot e subscript x, sir.