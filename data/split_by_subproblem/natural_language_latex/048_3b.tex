The pressure \( p_{g,2} \) equals the ambient pressure \( p_{\text{amb}} \), which is 1 bar, because there is thermodynamic equilibrium. The gas does not expand further.

Graph: A horizontal line representing \( pV = \text{const} \) with arrows pointing to the right indicating expansion.

The ratio of the final temperature \( T_{g,2} \) to the initial temperature \( T_{g,1} \) is given by \( \left( \frac{p_{g,2}}{p_{g,1}} \right)^{\frac{n-1}{n}} \).

\( K \) is defined as the ratio of \( C_p \) to \( C_v \).

\( C_p \) is calculated as \( R + C_v \) which equals \( 0.17 \frac{\text{J}}{\text{g K}} + 0.633 \frac{\text{J}}{\text{g K}} = 0.80 \frac{\text{J}}{\text{g K}} \). Therefore, \( k \) is \( \frac{0.80 \frac{\text{J}}{\text{g K}}}{0.633 \frac{\text{J}}{\text{g K}}} = 1.26 \).

The final temperature \( T_{g,2} \) is calculated as \( T_{g,1} \left( \frac{p_{g,2}}{p_{g,1}} \right)^{\frac{n-1}{n}} = 446.13^\circ \text{C} \).