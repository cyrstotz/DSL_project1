The text describes two items, each containing a description of a Pressure-Temperature (P-T) diagram:

(i) The first graph is a Pressure-Temperature diagram where the vertical axis is labeled P with the unit bar, and the horizontal axis is labeled T with the unit U. The graph features a bell-shaped curve labeled NS, starting from the bottom left, rising to a peak, and then descending to the bottom right. Four points are marked on the graph: point 1 at the left base of the curve, point 2 at the peak, point 3 at the right base, and point 4 slightly above point 1 on the left side. The segment between points 1 and 2 is labeled "adiabat", the segment between points 2 and 3 is labeled "isobar", and the segment between points 3 and 4 is labeled "adiabat reversible".

(ii) The second graph is also a Pressure-Temperature diagram with the vertical axis labeled P in bars and the horizontal axis labeled T in U. It features a similar bell-shaped curve labeled NS. There is a point marked at the peak of the curve, labeled as point 2. The segment above point 2 is labeled "isotherm", and the point itself is labeled "Triple point".