The change in entropy, denoted as Delta S, is calculated as m2 times s2 minus m1 times s1.

For s1, it is given by the entropy at 70 degrees Celsius, denoted as s_c (70 degrees C), plus x times the difference between the entropy of gas at 700 degrees Celsius, s_g (700 degrees C), and s_c at 70 degrees C. This calculation results in 7.306 kilojoules per kilogram Kelvin plus 0.005 times the difference between 7.375 kilojoules per kilogram Kelvin and 7.306 kilojoules per kilogram Kelvin, which equals 7.337 kilojoules per kilogram Kelvin.

For s2, it is calculated similarly as the entropy at 70 degrees Celsius, s_c (70 degrees C), plus x times the difference between the entropy of gas at 70 degrees Celsius, s_g (70 degrees C), and s_c at 70 degrees C. This results in 0.956 kilojoules per kilogram Kelvin plus 0.005 times the difference between 2.785 kilojoules per kilogram Kelvin and 0.956 kilojoules per kilogram Kelvin, which equals 0.968 kilojoules per kilogram Kelvin.

Finally, Delta S is calculated using the masses 5755 kg and 3600 kg for m2 and m1 respectively, and the previously calculated entropies. The result is (5755 kg minus 3600 kg) times 0.968 kilojoules per kilogram Kelvin minus 5255 kg times 7.337 kilojoules per kilogram Kelvin, which equals 5563.4 kilojoules per Kelvin.