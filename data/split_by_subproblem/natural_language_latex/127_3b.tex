b)

The pressure \( P_{1,g} \) equals \( P_{2,g} \) since equation ref 1 does not change.

The temperature \( T_2 \) equals \( T_1 \) times the ratio of \( P_2 \) over \( P_1 \) raised to the power of \( \frac{n-1}{n} \), where \( n \) equals the ratio of \( c_p \) over \( c_v \), which equals \( \frac{R + c_v}{c_v} \), simplifying to \( \frac{R}{c_v} + 1 \).

This simplifies further to \( \frac{8.314}{50} + 0.633 = 1.2667 \).

Then, \( T_2 \) equals 500 times \( \left( \frac{4}{1} \right) \) raised to the power of \( \frac{0.2667}{1.2667} \), which results in 7.

Finally, \( T_2 \) equals \( T_1 \) since the pressure does not change, keeping \( T \) constant.