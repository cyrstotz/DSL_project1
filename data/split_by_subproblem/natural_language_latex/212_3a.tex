Given \( p_c, n, m_c \),

The equation \( p \cdot V = m \cdot R \cdot T \) leads to \( m_c = \frac{p \cdot V}{R \cdot T} \).

The gas constant \( R \) is given by \( R = \frac{R}{M} = 166.28 \, \frac{J}{kg \cdot K} \) which equals \( 0.050 \, \frac{kg}{mol} \).

The volume is \( 0.00314L \).

The temperature \( T \) is \( 738.15K \).

The mass \( m \) is \( 2.687g \).

The equation \( p_6 \cdot A = p_0 \cdot A + m_c \cdot g \) leads to \( p_G = 1.100 \, bar \).

The area \( A \) is calculated as \( (0.1m)^2 \cdot \pi \).