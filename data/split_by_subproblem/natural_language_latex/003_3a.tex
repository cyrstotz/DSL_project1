The pressure \( p_{01} \) is calculated as the sum of the pressure due to mass \( m_a \) times gravity \( g \) divided by area \( A \), the pressure due to mass \( m_{2W} \) times gravity \( g \) divided by area \( A \), plus the ambient pressure \( p_{\text{amb}} \). This is given by the equation \( p_{01} = \frac{32 \cdot g \cdot (1 \, \text{m}^2)}{(0.05 \, \text{m})^2 \cdot \pi} + \frac{0.1 \cdot g \cdot (1 \, \text{m}^2)}{(0.1 \, \text{m})^2 \cdot \pi} + 1.6 \, \text{bar} \). The result is \( p_{01} = 140099.4406 \, \text{Pa} \) or \( 1.4 \, \text{bar} \).

The mass \( m_1 \) is calculated using the ideal gas law rearranged to \( m_1 = \frac{p_1 \cdot V_1 \cdot M}{R \cdot T_1} \), where \( p_1 \) is the pressure, \( V_1 \) is the volume, \( M \) is the molar mass, \( R \) is the gas constant, and \( T_1 \) is the temperature. Substituting the values, \( m_1 = \frac{140000 \cdot 0.003 \, \text{m}^3 \cdot 73.3 \, \text{g/mol}}{(8.314 \, \text{J/(mol K)}) \cdot 50 \, \text{K}} \), the mass \( m_1 \) is calculated to be \( 0.00341943 \, \text{kg} \).