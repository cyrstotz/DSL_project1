The heat output is denoted by Q-dot with subscript "aus".

In the Ureactor equation, the mass of gas, m_g, multiplied by Q-dot_r equals 5755 kg times (1 minus x) times Q-dot_f times x times Q-dot_g.

Q-dot_f is given as 418.94 kilojoules per kilogram, and Q-dot_g is 2506.5 kilojoules per kilogram.

Q-dot_r is 30 times 10 to the power of 6 kilojoules per second.

At 70 degrees Celsius, Q-dot_r equals the mass of gas times Q-dot_f at 70 degrees Celsius, which leads to 292.88 kilojoules per kilogram times 5755 kg equals 1.65 times 10 to the power of 6 kilojoules.

This results in minus 28.31 times 10 to the power of 6 kilojoules to be added in the form of water.

For the problem d):

The change in internal energy of water, Delta U_w, is the internal energy of water at 70 degrees Celsius minus the internal energy at 20 degrees Celsius, which equals minus 8.3 kilojoules per kilogram plus 252.85 kilojoules per kilogram.

The change in mass times Delta U_w plus 35 times 10 to the power of 3 kilograms equals 28.31 times 10 to the power of 6 kilojoules.

The change in mass, Delta m, is calculated as (28.31 times 10 to the power of 6 kilojoules minus 35 times 10 to the power of 3 kilojoules) divided by Delta U_w.

Delta U_w is 208 kilojoules per kilogram.

Delta m is 135 times 0.284 times 10 to the power of 3 kilograms.