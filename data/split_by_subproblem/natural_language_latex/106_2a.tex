The y-axis is labeled as T in Kelvin and the x-axis is labeled as S in kilojoules per kilogram Kelvin.

The graph is a T-S diagram with the following characteristics:
- There are six points labeled from 0 to 6.
- Point 0 is at the origin.
- Point 1 is directly above point 0.
- Point 2 is to the right of point 1.
- Point 3 is above point 2.
- Point 4 is below point 3 and to the right of point 2.
- Point 5 is to the right of point 4.
- Point 6 is below point 5 and to the right of point 0.
- The path from point 0 to point 1 is vertical.
- The path from point 1 to point 2 is horizontal.
- The path from point 2 to point 3 is curved upwards.
- The path from point 3 to point 4 is curved downwards.
- The path from point 4 to point 5 is horizontal.
- The path from point 5 to point 6 is vertical.
- The path from point 6 to point 0 is diagonal.
- There are labels along the paths indicating processes with mass flow rates denoted by \( \dot{m}_R \), \( \dot{m}_M \), and \( \dot{m}_{\text{super}} \).
- The label "isobars" is written near the top of the graph.
- The label "0.19 bar" is written near the top right of the graph.

The table includes columns for pressure (P), temperature (T), specific humidity (\(\omega\)), and other unspecified columns. Specific entries include:
- At point 0: Pressure is 0.19 bar, temperature equals 243.15 Kelvin or -30 degrees Celsius, and speed is 200 meters per second.
- At point 5: Pressure is 0.5 bar, temperature is 431.9 Kelvin, and speed is 220 meters per second.
- At point 6: Pressure is 0.19 bar.

The following equations are also provided:
- \(Q_{0A} = 0\)
- \(\dot{m}_R = 5.235\)
- \(q_B = 1.955 \frac{kJ}{kg}\)
- \(\Delta S = 0\)

In the second task:
- The equation \(0 = \dot{m} (s_0 - s_6) + \frac{\dot{Q}_B}{T_B} + \dot{S}_{erzeugt}\) is labeled as the first line.
- \(\dot{S}_{erzeugt} = s_6 - s_0 - \frac{q_6}{T_B} = c_p \ln \left( \frac{T_6}{T_0} \right) - \frac{q_6}{T_B} = -625.73 \frac{kJ}{kg \cdot K}\)
- \(e_{x,verl} = s_{erzeugt} \cdot T_0 = -152.144 \frac{kJ}{kg}\)