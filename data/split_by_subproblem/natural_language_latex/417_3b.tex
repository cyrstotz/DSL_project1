Section 3.b

Required: T_g2 and p_g2.

The pressure p_g2 is equal to p_1 plus the fraction of the ice mass m_eis times gravity g divided by area A, plus again the fraction of the ice mass m_eis times gravity g divided by area A, which equals p_g1. This results in 1.4 bar, because the mass of the ice doubles after submersion (density difference negligible), so the pressure remains the same.

The temperature T_g2 is equal to T_g1, which is 0 degrees Celsius or 273.15 Kelvin, because there is still ice in the water. The water does not get warmer as long as the ice has not melted.