a) For a perfect gas, the pressure \( p_{g1} \) and the mass \( m_g \) are given by the following equations:

The pressure \( p_{g1} \) is calculated as:
\[
p_{g1} = \frac{m_g R T_{g1}}{V_{g1}}
\]

Further, \( p_{g1} \) can also be expressed as:
\[
p_{g1} = \frac{m_g g}{A} + \frac{m_{ew} g}{A} + p_{amb}
\]
where \( A \) is the area, calculated as:
\[
A = \pi \left(\frac{D}{2}\right)^2 = 7.854 \cdot 10^{-3} \, \text{square meters}
\]

The value of \( p_{g1} \) is given as:
\[
p_{g1} = 40.0948 \, \text{bar} \quad \frac{N}{m^2} + 1 \, \text{bar}
\]
and simplifies to:
\[
= 9.0000 \, \text{bar} \quad 1.401
\]

The mass \( m_g \) is calculated using the formula:
\[
m_g = \frac{p_{g1} V_{g1}}{R T_{g1}}
\]
where \( R \) is the specific gas constant, given by:
\[
R = \frac{R'}{M_g} = 166.28 \, \frac{\text{Joules}}{\text{kg Kelvin}}
\]

Finally, the mass \( m_g \) is:
\[
= 2.4542 \cdot 10^{-2} \, \text{kg}
\]