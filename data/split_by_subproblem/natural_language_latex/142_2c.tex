e subscript x, actual equals h subscript 1 minus h subscript 2 minus T subscript 0 times (s subscript 1 minus s subscript 2) plus the fraction (omega subscript 2 squared minus omega subscript 1 squared) over 2. This equals h subscript 1 minus h subscript 2 minus T subscript 0 times the fraction (q subscript 2 over T subscript 2 minus q subscript 1 over T subscript 1) plus the fraction (omega subscript 2 squared minus omega subscript 1 squared) over 2.

T subscript 0 equals 273.15 Kelvin.

q subscript 1 minus q subscript 0 equals the integral from T subscript 0 to T subscript 1 of c subscript p dT, which equals c subscript p times (T subscript 1 minus T subscript 0).

q subscript 2 minus q subscript 0 equals the integral from T subscript 0 to T subscript 2 of c subscript p dT, which equals c subscript p times (T subscript 2 minus T subscript 0).

q subscript 2 minus q subscript 0 equals the integral from T subscript 0 to T subscript 2 of the fraction c subscript p over T dT minus R ln (the fraction p subscript 2 over p subscript 0), which equals 0 degrees T subscript 2 minus 0 degrees T subscript 1 minus 0.

e subscript x, actual equals 133.6 kilojoules per kilogram.

omega subscript 2 equals 50 times 2.25 meters per second.

omega subscript 0 equals 200 meters per second.

T subscript 0 equals 273.15 Kelvin.

s subscript 2 at 325.07 Kelvin equals s subscript 2 at T subscript 2 plus the integral from T subscript 0 to T subscript 2 of the fraction c subscript p over T dT, which equals s subscript 2 at T subscript 0 plus the fraction c subscript p over T dT, which equals 2.703 kilojoules per kilogram Kelvin.

s subscript 0 equals 273.15 Kelvin.

s subscript 0 at 325.07 Kelvin equals s subscript 0 at T subscript 2 plus the fraction c subscript p over T dT, which equals 2.703 kilojoules per kilogram Kelvin, which equals 2.654 kilojoules per kilogram Kelvin.