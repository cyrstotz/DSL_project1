Stationary flow process:

The equation is zero equals mass flow rate times the difference in enthalpy at point a minus enthalpy at point a plus the change in potential energy divided by mass flow rate, plus heat flow rate minus work flow rate.

From point 2 to point 3: zero equals mass flow rate times the difference in enthalpy between point 2 and point 3, plus heat flow rate from 2 to 3 minus work flow rate on component k.

The mass flow rate times the difference in enthalpy between point 2 and point 3 equals the work flow rate on component k.

The mass flow rate equals the work flow rate on component k divided by the difference in enthalpy between point 2 and point 3, leading to pressure at point 3 equals 8 bar.

At point 2, x equals 1 indicating saturated steam.

Isobaric condition: pressure at point 3 equals pressure u, and pressure at point 2 equals pressure u.

At point u, x equals 0 leading to enthalpy at u from Table A-M for 8 bar.

The enthalpy at u equals the enthalpy at f for 8 bar, which is 93.42 kilojoules per kilogram.

From point 2 to point 3, the process is reversible adiabatic leading to a change in entropy equals zero.

Entropy balance from point 2 to point 3:

Zero equals mass flow rate times the difference in entropy between point 2 and point 3 plus heat flow rate divided by temperature plus entropy flow rate z.

The change in entropy equals zero.

Entropy at point 03 equals entropy at point 3.

For p-pH: Temperature a equals 10 Kelvin above the sublimation point.

For b-pi: 5 millibar below the triple point leading to pressure at point 2 equals 1 millibar.

Phase and Sublimation Point: Line Lei Triple, see diagram Figure 5: Temperature i equals negative 10 degrees Celsius.

The condensation temperature equals Temperature i minus 6 Kelvin, resulting in negative 16 degrees Celsius.

The enthalpy at l2 equals the enthalpy at g for negative 16 degrees Celsius from Table A-10, enthalpy g equals enthalpy 2 equals 237.74 kilojoules per kilogram, leading to entropy 2 equals entropy 3 equals 0.529 kilojoules per kilogram Kelvin.

Entropy 2 equals entropy 3 equals entropy 3 at 8 bar from Table A-12, entropy at 31.33 degrees Celsius equals 0.9066 kilojoules per kilogram Kelvin, enthalpy at 31.33 degrees Celsius equals 294.15 kilojoules per kilogram, entropy at 40 degrees Celsius equals 0.9374 kilojoules per kilogram Kelvin, enthalpy at 40 degrees Celsius equals 273.66 kilojoules per kilogram.

Interpolating enthalpy 3:

Enthalpy 3 equals the difference in enthalpy between 40 degrees Celsius and 31.33 degrees Celsius divided by the difference in entropy between 40 degrees Celsius and 31.33 degrees Celsius times the difference in entropy 3 minus entropy at 31.33 degrees Celsius plus enthalpy at 31.33 degrees Celsius, resulting in enthalpy 3 equals 271.313 kilojoules per kilogram.

The mass flow rate equals the work flow rate divided by the difference in enthalpy between point 2 and point 3, resulting in mass flow rate equals negative 0.02 kilojoules per kilogram, mass flow rate equals 8.34 times 10 to the power of negative 4 kilograms per second.

For task 4, item b:

Heat flow rate K equals 0.12 kilojoules, work flow rate t equals work flow rate K equals 0.028 kilowatts, efficiency K equals heat flow rate K divided by the absolute value of work flow rate K equals 4.2857.