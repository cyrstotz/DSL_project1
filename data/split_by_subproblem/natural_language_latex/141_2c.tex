c) Mass flow rate generated.

The change in excess energy, denoted as Delta e subscript xs, equals the excess energy input minus the excess energy output.

The change in excess energy input, Delta e subscript xs, in, is given by the formula:
h minus h subscript 0 minus T subscript 0 times (s minus s subscript 0) plus kinetic energy plus potential energy.

The excess energy input, e subscript xs, in, is calculated as:
Delta h minus T subscript 0 times Delta s plus kinetic energy.

This is equal to:
36.91 kilojoules per kilogram minus 243.15 kilojoules per kilogram times 0.335 plus 48.05 kilojoules per kilogram minus 63.5 kilojoules per kilogram.

Delta h is calculated using the formula:
specific heat at constant pressure for an ideal gas times (T subscript e minus T subscript 0), which equals:
1.006 kilojoules per kilogram Kelvin times (340 Kelvin minus 243.15 Kelvin),
resulting in 36.91 kilojoules per kilogram.

Delta s is calculated as:
specific heat at constant pressure for an ideal gas times the natural logarithm of (T subscript e over T subscript 0) minus the gas constant times the natural logarithm of (p subscript e over p subscript 0),
which simplifies to:
1.006 kilojoules per kilogram Kelvin times the natural logarithm of (340 Kelvin over 243 Kelvin) minus 0.287 times the natural logarithm of 1, resulting in 0.335.

The specific heat at constant volume for an ideal gas, c subscript v superscript ig, is given by:
specific heat at constant pressure for an ideal gas divided by k, which equals:
1.006 divided by 1.4, resulting in 0.719.

The gas constant, R, is 0.287.

The kinetic energy, k.e, is calculated as:
one half times (velocity at e squared minus velocity at 0 squared), which equals:
one half times (510 meters per second minus 200 meters per second squared),
resulting in 48,050 Joules per kilogram, which is 48.05 kilojoules per kilogram.