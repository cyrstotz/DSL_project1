Required: \( p_{g1} \), \( m_g \)

The temperature \( T_{g1} \) is given as 500 degrees Celsius, which is equal to 773.15 Kelvin.

The volume \( V_{g1} \) is given as 3.14 liters, which is equal to 0.00314 cubic meters.

The gas constant \( R_g \) is calculated as the ratio of the universal gas constant \( \overline{R} \) over the molar mass \( M \), which equals 0.1663 kilojoules per kilogram Kelvin.

The relationship between pressure, volume, mass, gas constant, and temperature for gas 1 is given by the equation:
\[ p_{g1} V_{g1} = m_g R_g T_{g1} \]

The diagram includes:
- A vertical arrow pointing downwards labeled with the force due to gravity on the mass \( m_k \) and the additional mass \( m_{EW} \), represented as \( (m_k + m_{EW})g \).
- A horizontal arrow pointing to the right labeled with the ambient pressure times area, \( p_{amb} A \).
- A rectangle representing a piston, with the label \( p_{amb} A \) on the left side.

The equation balancing the forces is:
\[ p_{amb} A + (m_k + m_{EW})g = p_{1} A \]

Solving for \( p_1 \) gives:
\[ p_{amb} + \frac{g}{A} (m_k + m_{EW}) = p_{1} = 1.4 \, bar = p_{g1} \]

Finally, the mass \( m_g \) of the gas can be calculated using the equation:
\[ \frac{p_{g1} V_{g1}}{R_g T_{1}} = m_g \]