- Graph 1: A graph with the y-axis labeled P in bars and the x-axis labeled T in Kelvin. The graph displays a dome-shaped curve with a peak labeled "1500 kJ". There are two horizontal dashed lines intersecting the curve, one at the top and one at the bottom. The bottom intersection is labeled "2".
- Graph 2: A similar graph with the y-axis labeled P in bars and the x-axis labeled T in Kelvin. This graph also shows a dome-shaped curve with a peak. There is a horizontal dashed line intersecting the curve at the bottom, labeled "2". Additionally, there is a shaded region to the right of the curve.

The table contains the following entries:
- Row 1: P, T, v
- Row 2: 1, p_u, (empty cell)
- Row 3: 2, p_Boon, -22 degrees Celsius
- Row 4: 3, 8 bar, (empty cell)
- Row 5: 4, 8 bar, (empty cell)

Equations and values:
- a_44 equals 0.
- p_u equals 5 millibars plus p_ipp.
- T_i minus 10 Kelvin equals T_sublimation point.
- x_2 equals 1.
- x_3 equals 1.
- x_4 equals 0.
- T_i minus T_3 equals 6 Kelvin.

- Graph 3: A graph with the y-axis labeled P in bars and the x-axis labeled T in Kelvin. The graph displays a dome-shaped curve with a peak. There is a horizontal dashed line intersecting the curve at the bottom, labeled "2". Additionally, there is a shaded region to the right of the curve.
- Graph 4: A graph with the y-axis labeled P in bars and the x-axis labeled T in Kelvin. The graph shows a dome-shaped curve with a peak. There are two horizontal lines intersecting the curve, one at the top and one at the bottom. The bottom intersection is labeled "2". The top intersection is labeled "isotropic". There is a point labeled "4" on the top line and a point labeled "3" on the bottom line.