b)

The derivative of E_p with respect to t equals the sum of dot m_i times the quantity h_i plus the fraction where the numerator is k_e_i plus p_e_i and the denominator is zero, plus the sum of dot Q minus the sum of dot W.

Zero equals dot m_Rohm times the quantity h_2 minus h_3 plus 28W.

h_2 equals the average of h_3 at T equals T_i minus 6K.

h_3 raised to the power of zero equals s_2 equals s_3 implies that it is not necessary to interpolate for h_3.

s_3 equals s_ink (up equals down).