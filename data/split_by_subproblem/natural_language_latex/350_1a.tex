The section discusses the energy balance around a reactor, specifically focusing on the heat output, denoted as Q-dot-out. It starts by mentioning "instead of FP" (possibly a reference to a fixed point or a specific condition).

The energy balance equation around the reactor is given by:
Zero equals the mass flow rate of water (m-dot-H2O) times the difference in enthalpy between the inlet and outlet (h-in minus h-out), plus the heat added to the reactor (Q-dot-R), minus the heat output (Q-dot-out).

From this, the heat output Q-dot-out is derived as:
Q-dot-out equals the mass flow rate of water (m-dot-H2O) times the difference in enthalpy between the inlet and outlet (h-in minus h-out), plus 100 kilowatts.

Substituting the values, it calculates:
Q-dot-out equals 0.3 times (292.58 minus 419.01) plus 100, which equals 62.182 kilowatts.

Additionally, the enthalpy at the inlet, when boiling, at the low-pressure limit, is given as:
h-in-boiling equals 292.98 kilojoules per kilogram, taken from Table A-2 at 70 degrees Celsius.

The enthalpy at the outlet, when boiling, at the low-pressure limit, is:
h-out-boiling equals 419.01 kilojoules per kilogram, also from Table A-2 at 100 degrees Celsius.