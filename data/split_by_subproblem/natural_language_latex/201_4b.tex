- GSI: mass flow rate denoted by m-dot subscript 1234.

The equation zero equals m-dot subscript 1234 times the difference of h3 and h2 plus Q-dot subscript ab.

The equation zero equals m-dot subscript 1234 times the difference of h2 and h3 minus W-dot subscript K implies m-dot subscript 1234 equals W-dot subscript K divided by the difference of h2 and h3.

It implies h2 equals h subscript g, and s2 equals s3.

h3 at 8 bar, temperature T3.

Graph Description:

The graph is a pressure-volume (p-V) diagram with the following details:

- The x-axis is labeled as T in Kelvin.
- The y-axis is labeled as p in bar.
- There are four points labeled 1, 2, 3, and 4.
- The process from point 1 to point 2 is a horizontal line indicating an isobaric process.
- The process from point 2 to point 3 is a curved line indicating an adiabatic reversible process leading to an isentropic process.
- The process from point 3 to point 4 is another horizontal line indicating an isobaric process.
- The process from point 4 to point 1 is a curved line indicating an isentropic process.