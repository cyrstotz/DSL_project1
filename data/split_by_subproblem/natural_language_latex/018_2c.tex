c) The total mass flow rate, denoted as m-dot-ges, equals what? The change in specific energy at steady conditions, denoted as Delta e-x-stc, equals what?

The equation is:
m-dot-ges times the quantity (h-zero minus h-e plus the fraction (w-e squared minus w-zero squared) over 2) plus the sum of Q-dot plus W-dot-t equals zero. Here, W-t equals W-dot-t divided by m-dot, which is equal to the integral from 0 to 6 of delta phi plus Delta e plus Delta pe.

This equals negative n times the integral of p dv plus Delta e.

This further simplifies to negative n times the quantity (R times (T-six minus T-zero) divided by (1 minus n) plus one-half times m-dot times (w-zero squared minus w-six squared)).

Finally, m-dot-ges times the quantity (h-six minus h-zero minus T-zero times (s-six minus s-zero) plus Delta phi e plus Delta pe) equals Delta e-x-stc.