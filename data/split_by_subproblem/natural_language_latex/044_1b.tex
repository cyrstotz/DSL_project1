The heat input rate, denoted as Q-dot subscript zu, equals 65 kilowatts.

The temperature of the cooling fluid, denoted as T subscript KF, is calculated as the ratio of S subscript e to S subscript a multiplied by the temperature T subscript eS, which equals four-thirds times the difference between h subscript ea and h subscript ep divided by the difference between S subscript a and S subscript e.

The rate of change of energy, denoted as dE over dt, equals the mass flow rate, denoted as m-dot, times the difference between the enthalpy of the cooling fluid, h subscript KF, and the enthalpy of the fluid, h subscript Fa, plus the heat input rate Q-dot subscript zu, plus beta.

The heat input rate, Q-dot subscript zu, equals the difference between the enthalpy of the fluid, h subscript Fa, and the enthalpy of the evaporator, h subscript ep, multiplied by the mass flow rate of the cooling fluid, m-dot subscript KF.

The terms "isobaric," "isochoric," and "reactions" are mentioned.

The difference between the enthalpy of the fluid, h subscript Fa, and the enthalpy of the cooling fluid, h subscript KF, equals c-star times the difference in temperatures T subscript 2 minus T subscript 1, plus v-star times the difference in pressures p subscript 2 minus p subscript 1.

The difference between S subscript a and S subscript e equals c-star times the natural logarithm of the ratio of T subscript 2 to T subscript 1.

The temperature of the cooling fluid, T subscript KF, calculated as the ratio of the difference between h subscript ea and h subscript ep to the difference between S subscript a and S subscript e, equals 253.21 Kelvin.