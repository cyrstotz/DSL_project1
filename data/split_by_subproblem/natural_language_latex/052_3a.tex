The pressure \( p_{a,1} \) equals the ambient pressure \( p_{\text{amb}} \) plus the term \( \frac{m_{\text{new}} \cdot g}{A} \) added twice. This leads to the area \( A \) being equal to \( \pi \) times the square of \( \frac{5}{2} \), which calculates to \( 0.00785 \) square meters.

It follows that \( 1 \) bar plus \( \frac{m_{\text{new}} \cdot g}{A} \) equals \( 1.401 \) bars.

The equation \( pV = mRT \) leads to \( R \) being equal to \( \frac{R}{\mu} \), which calculates to \( 166.28 \) Joules per kilogram Kelvin.

It follows that the natural logarithm of the fraction \( \frac{p_{\text{ex}} V_{G1}}{R \cdot T_{\text{bar}}} \) equals \( 3.422 \) grams.