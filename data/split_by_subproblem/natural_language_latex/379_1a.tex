The pressure \( p_{s1, A} \) is equal to the ambient pressure \( p_{amb} \) plus \( p_{m1} \) plus \( p_{m, ew} \) minus \( p_{amb} \) plus the force \( F_G \) divided by the area \( A \). This simplifies to 1 bar plus the gravitational force \( 9.81 \) times the sum of \( 0.15 \) kilograms and \( 32 \) kilograms, all divided by \( \frac{\pi}{4} \) times the square of half the diameter \( d \). This results in 1 bar plus \( 40034.2 \) Pascals, which equals \( 1.400342 \) bar, and this value is assigned back to \( p_{s1, A} \).

For the second equation, the product of \( p_{s1} \) and \( V_{s1} \) equals the mass \( m_{s1} \) times the gas constant \( R \) times the temperature \( T_{s1} \). The gas constant \( R \) is given as \( 8314 \) Joules per kilomole Kelvin, which is equivalent to \( 166.28 \) Joules per kilogram Kelvin. Using the ideal gas law \( pV = RT \), the mass \( m_{s1} \) is calculated as \( 166.28 \) Joules per kilogram Kelvin times \( 773.15 \) Kelvin, divided by \( 1.400342 \) times \( 10^5 \) Pascals times \( 3.14 \) times \( 10^{-2} \) cubic meters. This results in \( 0.003426 \) kilograms, which is equivalent to \( 3.42 \) grams.