The derivative of E dot with respect to time equals the sum of m dot times h plus the sum of Q dot minus the sum of W dot.

Zero equals m dot times the difference of h subscript ein minus h subscript aus plus Q dot subscript R minus Q dot subscript a,w.

Q dot subscript a,w equals Q subscript a,w times m dot subscript a times the difference of h subscript ein minus h subscript aus.

h subscript ein equals h at 70 degrees Celsius, boiling liquid, which is equal to h at 30 degrees Celsius, 3.1482 bar, and equals 232.88 kilojoules per kilogram.

h subscript aus equals h at 100 degrees Celsius, boiling liquid, which implies h at 100 degrees Celsius, 1.014 bar, and equals 419.04 kilojoules per kilogram.

This implies that Q dot subscript a,w equals 100 plus 0.3 times the difference of 232.88 minus 419.04.

This is approximately 62.2 kilowatts.

h subscript n2 equals h1 at 100 degrees Celsius, x equals 0.005, which is h1 at 100 degrees Celsius plus x times the difference of h subscript g2 at 100 degrees Celsius minus h1 at 100 degrees Celsius, and is approximately 430.3 kilojoules per kilogram (A-2).

h2 equals h at 30 degrees Celsius, x equals 0, which is h subscript f at 30 degrees Celsius, and equals 292.88 kilojoules per kilogram (A-2).

h subscript en12 equals h at 20 degrees Celsius, x equals 0, which is h subscript f at 20 degrees Celsius, and equals 83.96 kilojoules per kilogram (A-2).

This implies that Delta h subscript 12 equals 535 divided by 83.96 times the difference of 292.88 minus 430.3 divided by 1 minus 83.96.

This is approximately 109.5 kilojoules.