Where? T2?
Since it is adiabatic, isentropic:
The ratio of pressure p2 over p1 raised to the power of (n-1) over n equals the ratio of T2 over T1. Therefore, T2 equals 434.9 times the ratio of 0.15 times 10 to the power of 5 over 0.5 times 10 to the power of 5, all raised to the power of (1.4 minus 1) divided by 1.4, which equals 328.075 Kelvin.
Where T2 equals T6 equals 328.075 Kelvin.
We assume that w squared over 2 equals the work done equals the integral of v dp equals R times T times the natural logarithm of the ratio of p2 over p1 equals 8.314 over 28.97 times 10 to the power of -3 times 434.9 Kelvin times the ratio of 0.15 bar over 0.5 bar.
This equals R over M times T times the natural logarithm of the ratio of p2 over p1 equals negative 115.280 kilojoules.
Thus, the work done.
The square root of negative 115.280 times negative 1 times 2 equals w equals 488.423 meters per second.