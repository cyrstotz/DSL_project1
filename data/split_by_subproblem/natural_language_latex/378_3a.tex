In CGS units, for magnesium, it implies that the mass pressure \( p_m \) is equal to the mass of air \( m_a \) times the specific gas constant for air \( R_a \) times the temperature \( T_1 \) divided by the volume of air \( V_a \).

The gas constant \( R \) is equal to the energy \( E \) divided by the product of magnesium mass \( Mg \) and equals \( 166.28 \) joules per kilogram per Kelvin.

The mass of air \( m_a \) is equal to the pressure of air \( p_a \) times the volume of air \( V_a \) divided by the product of the specific gas constant for air \( R_a \) and the temperature of air \( T_a \).

The area \( A_2 \) is equal to the temperature \( T_1 \) times the square of the radius \( r \), which is also equal to pi \( \Pi \) times the square of the radius \( r \) divided by 9, and this is equal to pi \( \Pi \) times the radius \( r \) divided by 400.

The pressure \( p \) is equal to the mass of Earth \( m_E \) times the volume \( V \) divided by the area \( A_2 \), plus the product of 32 kilograms and the acceleration due to gravity \( g \) divided by the area \( A_2 \), plus the mass pressure \( p_m \), and this equals \( 160 \) kilopascals minus the atmospheric pressure of \( 1.6 \) bar.

The mass of air \( m_a \) is \( 3.4217 \) grams.