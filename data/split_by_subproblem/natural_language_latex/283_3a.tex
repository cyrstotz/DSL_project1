**Force Equilibrium:**

The equation \( p_{g,1} \times \left(\frac{D}{2}\right)^2 \times \pi = p_{amb} \times \left(\frac{D}{2}\right)^2 \times \pi + m_k \times g \)

implies that \( p_{g,1} = p_{amb} + \frac{m_k \times g}{\left(\frac{D}{2}\right)^2 \times \pi} = 100000 \, \text{Pascals} + \frac{32 \, \text{kilograms} \times 9.81 \, \text{newtons per kilogram}}{(0.05 \, \text{meters})^2 \times \pi} \)

equals \( \boxed{1.3997 \, \text{bar} = p_{g,1}} \).

**According to the Ideal Gas Law:**

The mass of the gas \( m_g \) is given by \( m_g = \frac{p_{g,1} \times V_{g,1}}{R_g \times T_{g,1}}, \) where \( R_g = \frac{R}{M_g} \) thus \( R_g = \frac{8.314 \, \text{kilojoules per kilomole per Kelvin}}{50 \, \text{kilograms per kilomole}} = 166.3 \, \text{joules per kilogram per Kelvin} \).

This leads to \( m_g = \frac{139970 \, \text{Pascals} \times 3.14 \times 10^{-3} \, \text{cubic meters}}{166.3 \, \text{joules per kilogram per Kelvin} \times 773.15 \, \text{Kelvin}} = \boxed{3.42 \, \text{grams} = m_g} \).