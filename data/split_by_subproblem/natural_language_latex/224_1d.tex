The change in internal energy, Delta U, is equal to the mass flow rate, m dot, times the difference between the enthalpy at the inlet, h subscript ein superscript w, and the enthalpy at the outlet, h subscript aus superscript w, plus the heat input rate, Q dot subscript zu, minus the heat output rate, Q dot subscript aus, which equals zero.

The enthalpy at the inlet, h subscript ein superscript w, at 20 degrees Celsius is 83.96 kilojoules per kilogram, labeled as A2.

The change in internal energy, Delta U, is equal to the total mass flow rate, m dot subscript ges, times the difference between the internal energy at state 1, u1, times the sum of the total mass, ges, and the mass change from state 1 to state 2, Delta m12, and the internal energy at state 2, u2.

The internal energy at the saturated liquid state, uf, at 40 degrees Celsius plus the difference between the internal energy at the critical point, uc, and the internal energy at the saturated liquid state, uf, equals 429.3778 kilojoules per kilogram.

The internal energy at state 2, u2, at 70 degrees Celsius is boxed and equals 292.5 kilojoules per kilogram.

In the solution section:

The total mass, m subscript ges, times the difference between the internal energy at state 1, u1, and state 2, u2, minus the mass change from state 1 to state 2, Delta m12, times the internal energy at state 2, u2, equals the mass change from state 1 to state 2, Delta m12, times the enthalpy at the critical state, h subscript e,n double prime.

The mass change from state 1 to state 2, Delta m12, is equal to the total mass, m subscript ges, times the difference between the internal energy at state 1, u1, and state 2, u2, divided by the sum of the enthalpy at the critical state, h subscript e,n double prime, and the internal energy at state 2, u2, equals 6.408 T.