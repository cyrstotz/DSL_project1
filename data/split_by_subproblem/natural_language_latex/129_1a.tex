The rate of heat output, denoted as Q dot out.

It implies that the rate of heat output, Q dot out, equals the mass flow rate in, m dot in, times the difference in enthalpy between the output and input, h out minus h ein, which equals the rate of heat input, Q dot in, minus the rate of heat output, Q dot out.

The enthalpy at the output, h out, is approximately equal to the enthalpy at the fluid state at 100 degrees Celsius, which is 419.04 kilojoules per kilogram.

The enthalpy at the input, h ein, is approximately equal to the enthalpy at the fluid state at 70 degrees Celsius, which is 293.15 kilojoules per kilogram, as taken from TA-2.

It implies that the rate of heat output, Q dot out, equals the rate of heat input, Q dot in, plus the mass flow rate at the input, m dot ein, times the difference in enthalpy between the input and output, h ein minus h out, which equals 100 kilojoules per second plus 0.3 kilograms per second times the difference 293.15 minus 419.04 kilojoules per kilogram.

This results in 62.1 kilojoules per second.

For part a), the change in entropy from state 1 to state 2, Delta S 1,2, equals the change in mass from state 1 to state 2, Delta m 1,2, times the entropy of water at 20 degrees Celsius, S w, minus the cold heat transfer, Q kalt 1,2, divided by the cold fluid temperature, T KF.

This equals the entropy of water at 20 degrees Celsius, S w, which is 0.256 kilojoules per kilogram Kelvin, as taken from T A minus 2.

The change in entropy from state 1 to state 2, Delta S 1,2, equals 3127 kilograms times 0.256 kilojoules per kilogram Kelvin minus 5000 kilojoules divided by 293.15 Kelvin, resulting in an underlined value of 808.1 kilojoules per Kelvin.