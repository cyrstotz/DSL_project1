b)
The equation m times g equals F.

The pressure p equals the mass m times g divided by 0.05 square meters, which equals 40.099 bar.

The pressure p at 1.5 equals the atmospheric pressure plus the weight pressure, which equals 1.40094 bar.

1.40 bar.

An empty box.

T subscript g,1 and an average of T subscript g,2.

3c).
A diagram is described where two vertical lines are drawn with a horizontal line connecting them at the middle. An upward and a downward arrow are placed at the center of the horizontal line, indicating a quantity Q dot.

The equation zero equals m dot subscript w times (h1 minus h2) plus Q dot minus W dot.

W dot equals R times (T2 minus T1) plus m dot subscript a times (v2 minus v1).

W dot increases.

The equation equals R times (T2 minus T1) divided by (1 minus n).

m dot subscript g times (h1 minus h2) minus Q dot.

Mu equals zero.

Underlined, Q dot equals m dot subscript g times c subscript p times (T1 minus T2) equals 7.083 kilojoules.