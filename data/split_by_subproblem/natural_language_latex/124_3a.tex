The pressure \( p_{g_1} \) is equal to the sum of the product of mass \( m \) times gravity \( g \) divided by area \( A \), and atmospheric pressure \( p_{atm} \) divided by area \( A \), which equals \( \pi \) times the diameter \( D \) squared, which is calculated as \( \frac{8}{4} \) times \( (0.7m)^2 \).

The pressure \( p_{g_1} \) is also equal to the sum of the product of the total mass \( m_w + m_k \) times gravity \( g \) divided by \( \pi \) times the radius squared \( (0.1m)^2 \), plus the atmospheric pressure \( p_{atm} \), which equals the sum of \( (0.7kg + 32kg) \) times \( 9.87 \frac{m}{s^2} \) divided by \( \pi (0.7m)^2 \) plus \( 10^5 \) Pascals.

This results in \( 7.7 \times 10^5 \) Pascals, which is equivalent to \( 7.7 \) bar.

The product of pressure \( p \) and volume \( V \) equals the product of mass \( m \), gas constant \( R \), and temperature \( T \), which is also equal to the product of mass \( m \), the ratio of gas constant \( R \) over molar mass \( M \), and temperature \( T \), where \( R \) is given as \( 8.374 \frac{kg \cdot m^2}{s^2 \cdot K \cdot mol} \).

The mass of gas \( m_g \) is calculated as the product of pressure \( p_{g_1} \), volume \( V_{g_1} \), and molar mass \( M_g \) divided by the product of gas constant \( R \) and temperature \( T \), which equals \( \frac{7.7 \times 10^5 \) Pascals times \( 3.79 \times 10^{-3} m^3 \) times \( 50 \frac{g}{mol} \) divided by \( 8.379 \frac{kg \cdot m^2}{s^2 \cdot K \cdot mol} \) times \( 773.75 K \).

This results in \( 28.40 g \).