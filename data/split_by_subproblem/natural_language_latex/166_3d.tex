1. The first equation states that the change in internal energy, denoted as Delta U, equals the heat added, Q, plus the product of mass flow rate, dot m, and half the square of velocity, v squared over 2.

2. The heat, Q, is expressed as the change in internal energy, Delta U, which is equal to the product of mass, m, and specific internal energy at state 1, u1, minus the product of mass, m, and specific internal energy at state a1, ua1.

3. The specific internal energy at state 1, u1, is defined as the specific internal energy at F1, uF1, plus the product of x and the difference between the specific internal energy at F2, uF2, and uF1. This value is given as negative 200.1168 kilojoules per kilogram.

4. The equation L implies that the product of mass, m, and specific internal energy at state 2, u2, equals the product of mass, m, and specific internal energy at state 1, u1, plus the heat, Q.

5. This results in 1.200 kilojoules per kilogram plus 0.1 times negative 200.1168 kilojoules per kilogram.

6. Simplifying the above gives 1.200 kilojoules per kilogram minus 28.58 kilojoules per kilogram.

7. The specific internal energy at state 2, u2, is calculated by subtracting the product of negative 140.1168 kilograms meters per second from the product of negative 28.518 kilograms meters per second, all divided by 0.115 kilograms.

8. The fraction of ice at state 2, xEis,2, is determined by dividing the difference between u2 and the specific internal energy of ice at 0.005 degrees Celsius, uEis,0.005 degrees Celsius, by the difference between the specific internal energy of ice at 49.005 degrees Celsius, uEis,49.005 degrees Celsius, and uEis,0.005 degrees Celsius. This calculation is repeated four times, all yielding the same formula.

9. The final result for the fraction of ice at state 2 is 0.1570 or 57 percent.