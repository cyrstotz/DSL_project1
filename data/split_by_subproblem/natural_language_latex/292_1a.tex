a) The heat flow out, denoted as Q-dot-out, in kilowatts is to be determined.

The equation provided is:
Zero equals the mass flow rate, m-dot, times the quantity of enthalpy at the exit minus enthalpy at the entrance plus half the difference of the square of velocity at the exit and the square of velocity at the entrance, plus phi times the difference in height between the exit and the entrance, plus heat flow Q, minus the product of mass flow rate, m-dot, and specific volume at the exit. This equation is under isobaric conditions.

Another equation given is:
Zero equals the mass flow rate, m-dot, times the difference in enthalpy between the exit and the entrance, plus the heat flow rate in, Q-dot-R, minus the heat flow rate out, Q-dot-out.

Solving for Q-dot-out, it is equal to the mass flow rate, m-dot, times the difference in enthalpy between the exit and the entrance, plus the heat flow rate in, Q-dot-R, which equals 0.3 kilograms per second times the difference between 292.82 kilojoules per kilogram and 419.04 kilojoules per kilogram, plus 100 kilowatts.

This results in approximately 62.182 kilowatts.

Table A 2 provides:
The enthalpy at 70 degrees Celsius is 292.82 kilojoules per kilogram, denoted as h-e.
The enthalpy at 100 degrees Celsius is 419.04 kilojoules per kilogram, denoted as h-a.