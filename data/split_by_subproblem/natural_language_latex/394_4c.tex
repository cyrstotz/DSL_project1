The mass flow rate from point 2 to 1 to 4 is zero, first F.P.

The time derivative of mass flow rate times the difference in enthalpy between point 4 and point 1 plus the canceled heat transfer at point 4, which is zero due to being adiabatic, minus the canceled work at point 4, which is zero due to being isentropic.

The enthalpy at point 4 equals the enthalpy at point 4.

The enthalpy at point 1 equals the fluid enthalpy.

The ratio of the difference between the enthalpy at point 1 and the fluid enthalpy to the difference between the gas enthalpy and the fluid enthalpy equals x1 at pressure p1.

The enthalpy at point 4 for a pressure of 8 bar equals 2641.15 kilojoules per kilogram, which is a total absolute dryness plus x1.

The pressure at point 4 equals the pressure at point 3.

The enthalpy at point 4 equals the fluid enthalpy plus x1 times the difference between the gas enthalpy and the fluid enthalpy at pressure p1.