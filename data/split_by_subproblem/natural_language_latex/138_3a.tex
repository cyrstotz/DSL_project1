The content includes an image labeled "piston gas" and a set of equations and notes related to the gas in the piston and a system involving ice.

For the gas in the piston:
- The initial temperature of the gas, \( T_{g,1} \), is 500 degrees Celsius, which is equivalent to 773.15 Kelvin.
- The initial volume of the gas, \( V_{g,1} \), is 3.94 liters, which is equivalent to 0.00394 cubic meters.
- The mass of the gas, \( M \), is 50 grams.
- The specific heat at constant volume, \( C_v \), is 0.833 kilojoules per kilogram per Kelvin.

Note: The gas in the piston is isolated, leading to the equation:
\[ p_{g,1} \cdot V_{g,1} = m_g \cdot R \cdot T_{g,1} \]

For the ice system:
- The mass of the ice, \( m_{eis} \), is 0.4 kilograms.
- The initial temperature of the ice, \( T_{eis,1} \), is 0 degrees Celsius, which implies that \( T_{eis,1} \) is equal to the equilibrium temperature \( T_{eq,1} \).
- The initial fraction of ice, \( x_{eis,1} \), is 0.6, which implies that \( x_{eis,1} \) is equal to the equilibrium fraction \( x_{eq,1} \).

Through the ice-water mixture:
- The equilibrium pressure, \( p_{eq,1} \), is equal to the atmospheric pressure plus the ratio of the mass of the gas times the acceleration due to gravity, divided by the area \( A \). This results in an equilibrium pressure \( p_{eq,1} \) of 1.93 bar.

From this, it follows that:
- The pressure of the gas, \( p_{g,1} \), is equal to the equilibrium pressure \( p_{eq,1} \), which is 1.93 bar.
- The temperature of the gas, \( T_{g,1} \), can be calculated using the equation \( T_{g,1} = \frac{p_{g,1} \cdot V_{g,1}}{R \cdot m_g} \). Substituting the values gives a result of 3.42 grams, which seems to be an error in units or calculation as it should be in Kelvin or degrees Celsius.