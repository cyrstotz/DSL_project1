The pressure \( P_{g1} \) is represented by a diagram where two forces \( F_{g1} \) and \( F_{g2} \) are indicated with vectors pointing towards each other, and a rectangle is drawn around these forces. The expression for \( P_{g1} \) is given as the sum of ambient pressure \( P_{amb} \), the ratio of mass \( m \) times gravity \( g \) over area \( A_K \), and the ratio of extra weight mass \( m_{EW} \) times gravity \( g \) over area \( A_m \). Both areas \( A_K \) and \( A_m \) are equal and calculated as \( \frac{D^2}{4} \cdot \pi \) where \( D \) is 0.1 meters, resulting in an area of 0.00785 square meters.

The pressure \( P_{g1} \) is then calculated to be 1.1401 bar using the given values for mass, gravity, and area, and converting the units from Pascals to bar.

The ideal gas law is used next, expressed as \( pV = mRT \). From this, the mass \( m \) is calculated by rearranging the formula to \( m = \frac{pV_1}{RT_1} \) and substituting the values for pressure, volume, and the sum of the initial temperature \( T_1 \) and an additional 500 K. The result is approximately 3.4 grams.

The gas constant \( R \) specific to the substance is calculated by dividing the universal gas constant by the molar mass \( M_{CH} \), resulting in 166.3 Joules per kilogram Kelvin.

The pressure remains the same as in state 1 because the system expands upwards against lower pressure and the mass remains the same, indicated as \( P_2 = 1.1401 \) bar.

Since the system is in thermodynamic equilibrium, the temperature \( T_2 \) is equal to the equilibrium temperature \( T_E \) at pressure \( P_2 \), which is 0 degrees Celsius, and is the same as \( T_{ZG1} \) and \( T_{ZEW} \). The value of \( T_E \) at \( P_2 \) is referenced from Table 1.