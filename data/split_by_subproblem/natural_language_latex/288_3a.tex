The first equation is:
P subscript g,1 times the fraction of D squared times pi over 4 equals m subscript k plus P subscript AMB times the fraction of D squared times pi over 4 plus m subscript eis.
Solving for P subscript g,1, we get:
P subscript g,1 equals the fraction of m subscript k plus m subscript eis over the fraction of D squared times pi over 4 plus P subscript AMB equals the fraction of 32 plus 0.1 over 0.12 times pi divided by 4 plus 100000 equals 104087.09 Pascals.

The second set of equations is:
P subscript g,1 times V subscript g,1 equals m subscript g,1 times R times T subscript g,1.
Solving for R, we have:
R equals the fraction of R bar over M subscript g equals the fraction of 8.314 over 50 times 10 to the power of negative 3 equals 166.28 Joules per kilogram per Kelvin.
Solving for m subscript g,1, we get:
m subscript g,1 equals the fraction of P subscript g,1 times V subscript g,1 over R times T subscript g,1 equals the fraction of 104087.09 times 3.74 times 10 to the power of negative 3 over 166.28 times 773.15 equals 2.54228 times 10 to the power of negative 3 kilograms.