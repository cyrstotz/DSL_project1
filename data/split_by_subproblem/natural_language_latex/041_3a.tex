The temperature \( T_{g,1} \) is equal to 500 degrees Celsius.
The volume \( V_{g,1} \) is equal to 3.14 liters.
The product of pressure \( p_{g1} \) and volume \( V_{g1} \) equals the product of the gas constant \( R \) and temperature \( T_{g1} \).
The pressure \( p_{g2} \) is equal to the fraction of \( R T_{g1} V_{g1} \) divided by \( V_{g,1} \).

The ratio of volume \( V_{g,1} \) to mass \( m_{g} \) is equal to the volume \( V_{g,1} \) divided by 0.0000628 cubic meters per kilogram.

The volume \( V_{g,1} \) is equal to the volume \( V_{g,1} \) divided by the mass \( m_{g,1} \).

The mass \( m_{g,1} \) is unknown.

There is an image of a gas container.

The force \( F \) is equal to mass \( m \) times gravity \( g \), calculated as 32.1 times 9.81.

The area \( A \) is equal to pi times the square of 5 centimeters.

The pressure \( p \) is equal to the force \( F \) divided by the area \( A \), calculated as \( 32.1 \times 9.81 \) divided by \( \pi \times (5 \, \text{cm})^2 \), resulting in 60091.1 Pascals, which equals 0.6 bar.

The mass \( m_{g} \) is equal to the pressure \( p_{g} \) times volume \( V_{1} \) divided by the product of the gas constant \( R \) and temperature \( T_{1} \), resulting in 0.9769 kilograms, approximately 0.98 kilograms.

The ratio \( \frac{R}{M} \) is given without further context.