The graph is described as a Pressure-Temperature (P-T) diagram. The x-axis is labeled Temperature (T) and the y-axis is labeled Pressure (P). There is a curve starting from the origin and moving upwards to the right, labeled "solid". Another curve starts from the same origin and moves upwards to the right, labeled "liquid". These two curves meet at a point labeled "triple point". Below the "solid" curve, the region is labeled "gas". There is also a horizontal line extending from the "triple point" to the right, labeled "isotherm".

The equations are as follows:
- \( x_1 \)
- \( p_1 \) equals \( p_2 \)
- \( p_2 \) equals 1.212152 bar, which is equal to \( p_1 \)
- \( p_1 \) equals negative 22 bar
- \( p_2 \) equals negative 7.272 bar
- The process is isenthalpic throttling
- The change in entropy \( s_2 - s_1 \) equals 0.345 kilojoules per kilogram Kelvin
- \( \alpha \) equals the fraction \(\frac{s_1 - s_f}{s_g - s_f}\) which is 0.303
- \( s_f \) equals 0.0807
- \( s_g \) equals 0.9357