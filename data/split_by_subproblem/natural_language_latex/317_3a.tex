The pressure \( p_{3,1} \) is equal to the sum of the ambient pressure \( p_{\text{amb}} \) and the doser pressure \( p_{\text{doser}} \). The doser pressure \( p_{\text{doser}} \) is calculated as the force \( F \) divided by the area \( A \), which is further expressed as \( \frac{m_{\text{d}} \cdot g + m_{\text{env}} \cdot 32 \frac{\text{kg}}{\text{m}^2} \cdot 0.76 \frac{\text{kg}}{\text{m}^2} \cdot 389 \frac{\text{m}}{\text{s}^2}}{\pi \left( \frac{D}{2} \right)^2} \). Substituting the values, it simplifies to \( \frac{32 \frac{\text{kg}}{\text{m}^2} \cdot 0.76 \frac{\text{kg}}{\text{m}^2} \cdot 389 \frac{\text{m}}{\text{s}^2}}{\pi \left( \frac{10 \text{cm}}{2} \right)^2} \) which equals \( 0.900 \text{ bar} \).

Therefore, \( p_{3,1} \) equals \( 1 \text{ bar} + 0.9 \text{ bar} = 1.9 \text{ bar} \).

The product of \( p_{3,1} \) and \( V_1 \) equals the product of the mass of gas \( m_{\text{g}} \), the gas constant \( R \), and the temperature \( T_1 \). The gas constant \( R \) is calculated as \( \frac{\bar{R}}{M} = \frac{8.314 \frac{\text{J}}{\text{mol K}}}{50 \frac{\text{kg}}{\text{kmol}}} \) which equals \( 166.3 \frac{\text{J}}{\text{kg K}} \).

The mass of the gas \( m_{\text{g}} \) is then calculated as \( \frac{p_{3,1} \cdot V_1}{R \cdot T_1} = \frac{1.9 \text{ bar} \cdot 0.0037 \text{ m}^3}{166.3 \frac{\text{J}}{\text{kg K}} \cdot 773.75 \text{ K}} \) which equals \( 0.00392 \text{ kg} \) or \( 3.92 \text{ g} \).