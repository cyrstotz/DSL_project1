a) The pressure \( p_{g1,1} \) is equal to the ambient pressure \( p_{amb} \) plus the ratio of the product of mass \( m \), acceleration due to gravity \( g \), and area \( A_u \), simplified to \( p_{amb} + \frac{m \cdot g}{A_u} \).

The area \( A_u \) is calculated as pi times the square of half the diameter \( d \), which is \( \pi \left(\frac{d}{2}\right)^2 \). Substituting \( d = 0.05 \) meters, \( A_u \) equals \( 0.0078 \) square meters.

The pressure \( p \) is defined as the force \( F \) divided by the area \( A \), which implies that the force \( F \) is the product of pressure \( p \) and area \( A \).

Substituting the values into the equation for \( p_{g1,1} \), the pressure \( p_{g1,1} \) is the ambient pressure \( p_{amb} \) plus the ratio of the product of mass \( 32 \) kg, acceleration due to gravity \( 9.81 \) meters per second squared, and area \( 0.0078 \) square meters, resulting in \( 1.4 \) bar.

The pressure \( p_{g1,1} \) is approximately \( 1.3886 \) bar, which is approximately \( 1.4 \) bar.

The equation \( pV = mRT \) implies that the ratio of pressure \( p_{g1,1} \) times volume \( V_1 \) over the product of gas constant \( R_g \) and temperature \( T_1 \) equals the ratio of pressure \( p_{g1,2} \) times volume \( V_1 \) over the product of gas constant \( R_g \) and temperature \( T_2 \).

The mass of gas \( m_g \) is calculated as \( 3 \) moles, given the ratio \( \frac{3 \, \text{mol}}{1 \, \text{mol}} \).

The mass of gas \( m_g \) is \( 3.42 \) grams.