1. HS around engines:

Zero equals m dot times (h subscript e minus h subscript a plus omega subscript e squared minus omega subscript a squared all over 2) plus Q dot minus W dot.

Zero equals h subscript o minus h subscript s plus (omega subscript out squared minus omega subscript zero squared all over 2).

Omega subscript e equals 2 times (h subscript o minus h subscript s) plus omega subscript out squared; h subscript o minus h subscript o equals the integral from T subscript zero to T subscript o of c subscript p of T dT.

T subscript o equals T subscript s times (p subscript o over p subscript s) raised to the power of (n minus 1 over n) equals 431.3 Kelvin times (0.191 times 10 to the power of 5 Pascals over 0.5 times 10 to the power of 5 Pascals) raised to the power of (0.4 over 1.4).

T subscript s equals 328.075 Kelvin.

Therefore, h subscript o minus h subscript o equals c subscript p times (T subscript o minus T subscript o) equals 1.006 kilojoules per kilogram Kelvin times (328.075 Kelvin minus 243 Kelvin).

h subscript o minus h subscript o equals 85.41355 kilojoules per kilogram.

Delta h subscript 02 equals negative 85.435 kilograms per second times (6.3 cubic meters per kilogram times 2 plus (200 square meters per second squared)) equals crossed out.

Equals negative 130871 meters per second must be positive, as it is in the same direction as w subscript e.

Delta h subscript 02 equals 130871 square meters per second squared.

2c

w subscript zero equals 510 meters per second, T subscript zero equals 340 Kelvin, R equals c subscript p minus c subscript v, c subscript p equals c subscript v times kappa, kappa equals c subscript p over c subscript v.

c subscript v equals c subscript p over kappa.

c subscript v equals 1.006 kilojoules per kilogram Kelvin over 1.4.

c subscript v equals 0.71857 kilojoules per kilogram Kelvin.

Zero equals e subscript 1sf plus equals [h subscript c minus h subscript zero minus T subscript zero times (s subscript c minus s subscript zero)].

s subscript 1sf plus minus s subscript 1sf minus equals the integral from T subscript zero to T subscript c of c subscript p of T over T dT minus R ln (p subscript c over p subscript zero).

Equals c subscript p of T times (T subscript c minus T subscript zero) minus R ln (p subscript c over p subscript zero) equals 1.006 kilojoules per kilogram Kelvin times (310 Kelvin minus 243.15 Kelvin).

Minus (1.006 minus 0.71857) ln (0.891 times 10 to the power of 5 Pascals over 0.194 times 10 to the power of 5 Pascals).

Equals 1.006 kilojoules per kilogram Kelvin times (310 Kelvin minus 243.15 Kelvin) minus 0.28743 ln (0.891 times 10 to the power of 5 Pascals over 0.194 times 10 to the power of 5 Pascals).

s subscript 1sf plus minus s subscript 1sf minus equals c subscript p ln (T subscript c over T subscript zero) equals 1.006 kilojoules per kilogram Kelvin times 1.0006 times 10 to the power of negative 3 kilograms Kelvin per kilojoule ln (310 Kelvin over 243.15 Kelvin).

Equals 337.27888 Joules per kilogram Kelvin.

Delta e subscript 1sf plus equals 85.4355 kilograms per second minus 243.15 Kelvin times 337.27888 Joules per kilogram Kelvin.

Delta e subscript 1sf plus equals 3426.188858 Joules per kilogram.