The graph description is as follows: The graph is plotted on grid paper with the x-axis labeled as 's' with units in kilojoules per kilogram Kelvin (kJ/kgK) and the y-axis labeled as 'T' with units in Kelvin (K). The graph depicts a cycle with six points labeled 0, 1, 2, 3, 4, and 6. The cycle starts at point 0, moves to point 1, then to point 2, 3, 4, and finally back to point 6. The segments between these points are labeled as follows:
- From point 0 to 1: Isentropic
- From point 1 to 2: Isentropic
- From point 2 to 3: Isobaric, reversible, adiabatic
- From point 3 to 4: Isobaric, irreversible, with a change in entropy not equal to zero, at 0.5 bar
- From point 4 to 6: Isentropic
- From point 6 to 0: Isobaric

The equation provided is:
The rate of exergy destruction, denoted as dot e subscript exstr, equals the expression in parentheses: 55.925 kilojoules per kilogram minus 293.15 Kelvin times 0.303 kilojoules per kilogram Kelvin plus half of 857.294 square meters per square second minus half of 700 square meters per square second. Simplifying this, it equals 55.925 kilojoules per kilogram minus 73.7272 kilojoules per kilogram plus 128.648 kilojoules per kilogram minus 70 kilojoules per kilogram, which results in 120.846 kilojoules per kilogram.