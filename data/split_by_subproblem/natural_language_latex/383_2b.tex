Energy balance around the thrust nozzle:

The equation zero equals m dot times the quantity h5 minus h6 plus the quantity w5 squared minus w6 squared all over two.

h5 minus h6 equals dots.

This implies w6 equals the square root of two times the quantity h5 minus h6 plus w5 squared.

This equals the square root of two times 104.5 kilojoules per kilogram plus 220 meters per second squared, which equals 220.57 meters per second.

Student Solution:

The ratio T6 over T5 equals the ratio P6 over P5 raised to the power of (n minus 1) over n.

Cp equals 1.006 kilojoules per kilogram Kelvin.

n equals k equals 1.4.

This implies T6 equals T5 times the ratio P6 over P5 raised to the power of (n minus 1) over n equals 437.9 Kelvin times the ratio 0.191 over 0.15 raised to the power of (1 minus 1.4) over 1.4 equals 328 Kelvin.

This implies h5 minus h6 equals Cp times the quantity T5 minus T6 equals 104.95 kilojoules per kilogram.