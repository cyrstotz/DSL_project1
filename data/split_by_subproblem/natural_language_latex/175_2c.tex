The rate of change of specific energy, denoted as e dot, is given by the equation:

e dot equals E dot over m dot equals the quantity h naught minus h six minus T naught times the quantity s naught minus s six plus one-half times the quantity omega naught squared minus omega six squared.

The term h naught minus h six is equal to negative of h c minus h naught, which equals the exponential of T naught minus T six, resulting in 1.1006 Joules per kilogram times the quantity 293.15 Kelvin minus 328.07 Kelvin.

This results in negative 85.359 Joules per kilogram, since P six equals P naught.

The difference s six minus s naught is calculated as the integral from T one to T two of c p of T over T dT minus R times the natural logarithm of P six over P naught.

This equals the natural logarithm of T naught over T naught times c p, which is 0.3013 Joules per kilogram Kelvin.

Therefore, s naught minus s six equals negative of s six minus s naught, which is negative 0.3013 Joules per kilogram Kelvin.

Substituting these values, e dot equals negative 85.359 Joules per kilogram times 243 m dot times the quantity negative 0.3013 Joules per kilogram Kelvin plus one-half times the quantity (200 meters per second squared minus 510 meters per second squared).

This results in 2.167 times 10 to the power of 7 Joules per kilogram.