Let's set up the energy balance:

The rate of change of energy with respect to time is equal to the sum of the mass flow rate times the sum of enthalpy and the ratio of kinetic energy to potential energy, plus the sum of heat transfer rate minus the sum of work rate.

(Note: The term "o, da adiab" and "a shovel cell does no work" are crossed out.)

The ratio of temperature T6 to T5 is equal to the ratio of pressure Pc to Ps raised to the power of (n-1) over n. Therefore, T6 equals the ratio of Pc to Ps raised to the power of (n-1) over n times T5, which equals 0.7586 times 431.9 Kelvin, resulting in T6 equals 328.07 Kelvin.

Zero equals the mass flow rate times the bracket of enthalpy at state 5 minus enthalpy at state 6 plus half the square of velocity at state 5 minus half the square of velocity at state 6.

Enthalpy at state 5 minus enthalpy at state 6.

Enthalpy at state 5 minus enthalpy at state 6 equals the specific heat at constant pressure times the difference in temperature between state 5 and state 6, which equals 1.006 kilojoules per kilogram Kelvin times (431.9 Kelvin minus 328.07 Kelvin), resulting in 104.44825 kilojoules per kilogram.

Half the square of velocity at state 5 equals enthalpy at state 5 minus enthalpy at state 6 plus half the square of velocity at state 6.

The square of velocity at state 6 equals twice the sum of enthalpy at state 5 minus enthalpy at state 6 plus half the square of velocity at state 6.

Velocity at state 6 equals the square root of twice the sum of enthalpy at state 5 minus enthalpy at state 6 plus half the square of velocity at state 6, which equals the square root of twice 104.448, resulting in 20.47 meters per second.