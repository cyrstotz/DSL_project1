The given task is to find \( p_{g1} \) and \( m_g \).

The universal gas constant \( R \) is calculated as:
\[ R = \frac{\bar{R}}{M} = \frac{166.28 \text{ joules per kelvin per kilogram}}{1} \]

The pressure due to ambient conditions plus the pressure due to the mass of the gas \( m_g \) over area \( A \) is calculated as:
\[ p_{\text{amb}} + \frac{m_g \cdot g}{A} = 32 \times 9.81 \text{ meters per second squared} = 40,094.9 \text{ pascals} \]

The area of a circle with radius 0.05 meters is:
\[ (0.05 \text{ meters})^2 \cdot \pi \]

The gas pressure \( p_{g1} \) is then:
\[ p_{g1} = p_{\text{amb}} + 40,094.9 \text{ pascals} = 140,094.9 \text{ pascals} = 1.4 \text{ bar} \]

The mass of the gas \( m_g \) is calculated using the ideal gas law:
\[ m_g = \frac{p \cdot V}{R \cdot T} = \frac{1.4 \text{ bar} \cdot 0.0034 \text{ cubic meters}}{166.28 \text{ joules per kelvin per kilogram} \cdot 773.15 \text{ kelvin}} = 3.1 \text{ grams} \]