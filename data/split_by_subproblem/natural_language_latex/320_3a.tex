The equation p subscript A times vector V subscript A equals m subscript k times R times vector T subscript 1.

The following forces are acting downwards:
- m subscript EW times g,
- m subscript k times g,
- p subscript 0 times A,
- p subscript A times A.

The equation p equals F divided by A.

The area A equals pi times r squared, which equals pi times (0.1 divided by 2 meters) squared, resulting in 0.007854 square meters.

The equation m subscript EW times g plus m subscript k times g plus p subscript 0 times A equals p subscript 1 times A.

The pressure p subscript 1 equals (0.4 kg times 9.81 meters per second squared plus 32 kg times 9.81 meters per second squared plus 10 to the power of 5 Pascals times 0.007854 square meters) divided by 0.007854 square meters, resulting in 1.4 bar.

The work W subscript g1 equals p subscript 0 times V subscript A3 divided by R times T subscript 1, which equals (1.4 times 10 to the power of 5 Pascals times 0.00314 cubic meters) divided by (166.25 Joules per kilogram Kelvin times (500 plus 273.15 Kelvin)).

The gas constant R equals the universal gas constant R divided by M, which equals (8.314 Joules per mole Kelvin divided by 50 kilograms per kilomole), resulting in 166.28 Joules per kilogram Kelvin.

The mass m equals 32 divided by 9.81, resulting in 3.449 kilograms.