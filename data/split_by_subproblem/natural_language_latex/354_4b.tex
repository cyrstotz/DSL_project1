b)

- A small diagram is drawn, but it is not clear what it represents.

Temperature in the evaporator: T_i equals G_k

Titus Diagram: minus 10 degrees Celsius, T_f equals minus 16 degrees Celsius

From x_2 equals n, one finds the entropy and enthalpy from Table A-10

Entropy at minus 10 degrees Celsius equals 0.5298 kilojoules per kilogram Kelvin and enthalpy at minus 10 degrees Celsius equals 237.74 kilojoules per kilogram Kelvin

Since the compressor is adiabatically reversible, it holds that

S_2 equals S_3

From Table A12, using the entropy, the enthalpy can now be determined. It is also known from A10 that

s_2 equals s_3 equals 6.528 kilojoules per kilogram Kelvin

h_3 equals the enthalpy at 8 bar and 90 degrees Celsius minus the enthalpy at 8 bar and T_satt, divided by s_3 minus the entropy at 8 bar and T_satt, plus the enthalpy at 8 bar and T_satt

equals 277.37 kilojoules per kilogram Kelvin

Therefore, for the balance equation and the compressor, we obtain adiabatically

0 equals m dot times (h_2 minus h_3) plus Q dot minus W dot

m dot equals W dot divided by (h_2 minus h_3) equals 8.34 times 10 to the power of minus 4 kilograms per second