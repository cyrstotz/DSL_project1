A graph is drawn with pressure \( P \) on the y-axis and temperature \( T \) on the x-axis. The y-axis is labeled with \( P \) and the unit \([ \text{bar} ]\). The x-axis is labeled with \( T \) and the unit \([ \text{K} ]\). There are three curves on the graph:
- The first curve, labeled "Isothermal Expansion", starts from the origin and curves upwards steeply.
- The second curve, labeled "Isobaric Cooling", starts from a point on the y-axis and curves upwards more gently.
- The third curve, labeled "Triple Point", is a horizontal line intersecting the second curve.

Points 1, 2, and 3 are marked on the graph:
- Point 1 is on the first curve.
- Point 2 is on the second curve.
- Point 3 is on the third curve.

The region between the first and second curves is labeled "Solid".
The region between the second and third curves is labeled "Liquid".
The region to the right of the third curve is labeled "Gas".
The temperature difference \(\Delta T = 10K\) is marked on the x-axis.