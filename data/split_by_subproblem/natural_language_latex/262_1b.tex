The average temperature \( \overline{T}_{KF} \) is given by the ratio of the integral of temperature \( T \) with respect to entropy \( S \) from \( S_a \) to \( S_e \) over the difference \( S_a - S_e \).

The differential enthalpy \( dH \) is expressed as \( T dS + V dp \) under isobaric conditions.

The change in enthalpy \( dH \) is equal to \( h_a - h_e \).

Thus, the average temperature \( \overline{T}_{KF} \) can be rewritten as \( \frac{h_a - h_e}{S_a - S_e} \).

In German terms, \( \overline{T}_{KF} \) is also expressed as \( \frac{h_{aus} - h_{ein}}{S_{aus} - S_{ein}} \) where 'aus' means 'out' and 'ein' means 'in'.

For an ideal fluid:
The entropy \( S_{if} \) is a function of temperature \( S_p(T) \).

The difference in entropy from 'in' to 'out' is \( S_p(T_{aus}) - S_p(T_{ein}) \).

The difference in enthalpy from 'in' to 'out' is given by the integral of the specific heat capacity of the ideal fluid \( c_{if}(T) \) over temperature from \( T_{ein} \) to \( T_{aus} \) plus the product of volume \( v \) and the difference in pressure \( if(p_{aus} - p_{ein}) \).

This difference simplifies to \( c_{if}(T) \cdot (T_{aus} - T_{ein}) \).

The integral from \( T_{ein} \) to \( T_{aus} \) of \( \frac{c_{if}(T)}{T} \) equals \( c_{if}(T) \left[ \ln \left( \frac{T_{aus}}{T_{ein}} \right) \right] \).

Therefore, \( \overline{T}_{KF} \) simplifies to \( \frac{c_{if} \cdot T_{aus} - T_{ein}}{c_{if} \ln \left( \frac{T_{aus}}{T_{ein}} \right)} \) which equals 293.12 Kelvin.

The temperatures \( T_{aus} \) and \( T_{ein} \) are 298.15 Kelvin and 288.15 Kelvin respectively.

For the second part:
The rate of change of entropy \( \frac{dS}{dt} \) equals \( \dot{E} \frac{Q}{T} + \dot{S}_{er} \).

The difference in entropy rate from 'in' to 'out' is \( \frac{\dot{Q}_{aus}}{T} + \dot{S}_{er} \).

The entropy production rate \( \dot{S}_{er} \) is the difference in entropy rate from 'in' to 'out' plus \( \frac{\dot{Q}_{aus}}{T} \).

Under steady state conditions, \( \dot{S}_{er} \) equals the difference in entropy rate from 'in' to 'out'.

The equation simplifies to \( 0 = \dot{m} \left( s_{ein} - s_{aus} \right) + \dot{E} \frac{Q}{T} + \dot{S}_{er} \).

The entropy production rate \( \dot{S}_{er} \) is given by \( \dot{m} \left( s_{aus} - s_{ein} \right) - \left( \frac{\dot{Q}_{aus}}{T_R} + \frac{Q_{aus}}{T_{KF}} \right) \).

The mass flow rate \( \dot{m} \) is 0.3.

From Table A-2:
The entropy at 70 degrees Celsius is 0.3579 kJ/kgK.
The entropy at 100 degrees Celsius is 1.3669 kJ/kgK.

The entropy production rate \( \dot{S}_{er} \) is calculated as 0.1056 kJ/Ks minus the ratio \( \frac{100}{(100+273.15)} \) plus \( \frac{Q_{aus}}{T_{KF}} \).

Finally, \( \dot{S}_{er} \) equals 49.75 Watts per Kelvin.