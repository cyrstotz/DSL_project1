The average temperature for the subscript "u,f" is equal to the heat transfer rate Q subscript "u" divided by the product of the mass flow rate and the difference between the entropies s subscript "A" and s subscript "e".

The heat transfer rate for the subscript "u,s" is equal to the heat transfer rate Q dot divided by the product of the mass flow rate and the difference between the entropies s subscript "A" and s subscript "e".

The heat transfer rate Q dot equals Q dot divided by h, which equals the change in enthalpy divided by the negative change in entropy, and this equals the average temperature T bar.

The average temperature T bar equals the product of area A and mass flow rate dot m divided by h, which equals 376.136 kilojoules per second, and is approximated to 380 kilojoules per second.

The average temperature T bar equals the product of heat transfer rate dot Q and the difference between temperatures T2 and T1 divided by dot Q, plus dot Q times the natural logarithm of the ratio of T2 to T1, which equals negative 253.12 kilojoules per second.

For Problem 3.6, the solution includes:
- The average temperature T bar for the subscript "UF" is 255 Kelvin.
- The entropy S for the subscript "EF" equals negative of the ratio of heat transfer rate dot Q to the average temperature T bar for the subscript "Tank" plus the ratio of dot Q to the average temperature T bar for the subscript "UF", which equals 0.08827 kilojoules per second.
- The average temperature T bar for the subscript "Tank" equals three-fourths of GSG times four divided by the change in entropy Delta S equals GSG times 358 Kelvin.

Additional calculations include:
- The mass m2 times enthalpy h2 plus mass m1 times enthalpy h1 equals sm times enthalpy hf at 20 degrees Celsius plus Q minus Q.
- The sum of mass plus sm times enthalpy h2 minus mass m1 times enthalpy h1 equals sm times enthalpy hf at 20 degrees Celsius.
- The sm equals the difference of mass m1 times enthalpy h1 minus mass m1 times enthalpy h2 divided by the product of enthalpy h2 times enthalpy h12 minus enthalpy hf, which equals 3,668.088 kilograms.

The enthalpies are:
- h1 equals 418.94 Joules.
- h2 is approximately 282.38 or 272.58.
- hf equals 81.36.