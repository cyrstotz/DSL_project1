The rate of heat output, denoted as Q dot subscript "aus", equals the mass flow rate, m dot, times the difference between the enthalpy at exit, h subscript e, and the enthalpy at entrance, h subscript a, plus the rate of heat addition, Q dot subscript R.

The enthalpy at exit, h subscript e, is equal to the enthalpy at temperature T subscript e, plus the enthalpy of vaporization at a temperature of 700 degrees Celsius, minus the enthalpy at exit, h subscript e.

The enthalpy difference between exit and ambient, h subscript ea, is equal to the enthalpy at temperature T subscript e, plus the enthalpy of vaporization at a temperature of 700 degrees Celsius, which equals the enthalpy at ambient, h subscript a.

From Table A2:

The enthalpy at exit, h subscript e, is 2333.845 kilojoules per kilogram.

The enthalpy at ambient, h subscript a, is 2252 kilojoules per kilogram.

Therefore, the rate of heat output, Q dot subscript "aus", equals 0.3 times the difference between 2252 and 2333.8, plus 700 kilowatts, which results in 70.36 kilowatts.