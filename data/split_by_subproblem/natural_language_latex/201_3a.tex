a) GSI: p_g1, m_g

The expression for p_g1 is given by the formula:
p_g1 equals the fraction of m_k times g over the quantity D over 2 squared times pi, plus p_amb, plus the fraction of m_ev times g over the quantity D over 2 squared times pi, which equals 7.4 bar.

The values are provided as:
m_k times g equals 373.22 kilograms times meters per second squared, and m_ev times g equals 0.997 kilograms times meters per second squared.

The area, given by the quantity D over 2 squared times pi, is 0.00785 square meters.

The mass m_g is calculated using the formula:
m_g equals the fraction of p_g1 times V_g1 over R_g times T_g1, which equals 0.0034 kilograms or 3.4 grams.

The gas constant R_g is calculated as:
R_g equals the fraction of R over M_g, which equals 166.28 Joules per kilogram Kelvin.

The molar mass M_g is given as:
M_g equals 50 grams per mole.