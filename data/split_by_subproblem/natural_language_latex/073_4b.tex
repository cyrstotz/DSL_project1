The table contains four rows, each with three columns labeled as blank, T(K), and P[x]. Here are the details for each row:

1. The temperature is 31.33 Kelvin. The process is described as adiabatic, which equals an isothermal and isentropic process. It is noted that h4 equals h1.
2. The temperature is given in two units: 22 degrees Celsius and 295.15 Kelvin. The process is isobaric, and x2 is calculated using the formula (h2 - hf) / (hg - hf). The heat transfer rate, Q dot from 1 to 2, is also mentioned.
3. The temperature is 31.33 Kelvin and the pressure is 8 bar. The work done by the compressor, W dot K, is 28 Watts. The process is isotropic, implying s2 equals s3.
4. The temperature is 31.33 Kelvin and the pressure is 8 bar. x4 equals 0.

Additional equations provided:
- h1 equals h4 equals hf at 8 bar, approximately 93.42 kilojoules per kilogram.
- s2 equals sg at 8 bar, which is 0.9066 kilojoules per kilogram Kelvin.
- For the process from 2 to 3, which is isentropic, s2 equals s3 equals 0.9066 kilojoules per kilogram Kelvin.
- This implies that T3 equals T_sat.

Under the section "1. HS am Verdichter" (1st Law of Thermodynamics at the Compressor):
- The rate of change of energy, dE/dt, equals the mass flow rate of R134a, m dot R134a, times (h2 - h3) plus Q dot minus W dot K.
- W dot K equals m dot R134a times (h2 - h3).
- m dot R134a equals W dot K divided by (h2 - h3).