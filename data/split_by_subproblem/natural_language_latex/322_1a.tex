The heat input denoted as Q-dot-subscript-zu with energy balance around the fluid (stationary) is described by the equation:
Zero equals the sum of m-dot times the quantity h-subscript-i plus v-subscript-i squared over two plus g z-subscript-i, plus the sum of Q-dot-subscript-j minus the sum of W-dot-subscript-j.

This implies that zero equals m-dot times the quantity h-subscript-ein minus h-subscript-aus minus Q-dot-subscript-zu plus Q-dot-subscript-N.

From this, it follows that Q-dot-subscript-zu equals m-dot times the quantity h-subscript-ein minus h-subscript-aus plus Q-dot-subscript-N.

Here, h-subscript-ein is the enthalpy at 90 degrees Celsius for boiling liquid, and h-subscript-aus is the enthalpy at 100 degrees Celsius for boiling liquid.