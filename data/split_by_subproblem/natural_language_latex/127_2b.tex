Energy balance of the entire turbine

The mass flow rate exiting the stream is equal to the mass flow rate times the expression in brackets: enthalpy at point 0 minus enthalpy at point 6 minus the product of temperature at point 0 and the difference in entropy between points 6 and 0 plus half the square of the exit velocity.

This equals the total mass flow rate times the expression in brackets: enthalpy at point 5 minus enthalpy at point 6 plus half the difference between the square of the velocity at point 5 and the square of the velocity at point 6.

The total mass flow rate times the expression in brackets: enthalpy at point 5 minus enthalpy at point 6 plus half the difference between the square of the velocity at point 5 and the square of the velocity at point 6 equals zero.

The work per unit mass flow rate is two times the expression in brackets: the specific heat at constant pressure times the difference in temperature between points 5 and 6 plus half the square of the shaft speed.

This equals two times the expression in brackets: 1.006 times the difference between 843.9 Kelvin and 828.07 Kelvin plus the square of 220.

This results in 323 meters per second.

The temperature at point 6 is equal to the temperature at point 5 times the power of the ratio of pressure at point 0 to pressure at point 5 raised to the power of (nu minus 1) divided by nu.

This equals 843.9 times 0.737 equals 828.07 Kelvin.