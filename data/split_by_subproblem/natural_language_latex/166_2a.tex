The graph description is of a Pressure-Volume (P-V) diagram where the x-axis is labeled as V (Volume) and the y-axis is labeled as p (Pressure). The graph contains six points labeled from 0 to 6. The path of the curve starts at point 0, progresses to point 1, then to point 2, and continues to point 3. From point 3, it moves to point 4, then to point 5, and finally to point 6. Two isobars are marked as P0 and P5. The segment of the curve from point 0 to point 1 is a horizontal line, indicating a constant pressure process. The segment from point 1 to point 2 is a vertical line, indicating a constant volume process. The segment from point 2 to point 3 is a diagonal line, indicating an adiabatic process. The segment from point 3 to point 4 is again a horizontal line, indicating a constant pressure process. The segment from point 4 to point 5 is a vertical line, indicating a constant volume process. The segment from point 5 to point 6 is a diagonal line, indicating another adiabatic process.

The equation is zero equals the mass flow rate times the difference in enthalpy minus the product of the reference temperature and the difference in entropy, plus the changes in kinetic and potential energy, plus the sum of one minus the ratio of the reference temperature to the temperature at state i times the heat transfer at state i.

The exergy destruction is equal to the difference in enthalpy minus the product of the reference temperature and the difference in entropy, plus the changes in kinetic and potential energy, plus the sum of one minus the ratio of the reference temperature to the temperature at state i times the heat transfer at state i, minus the sum of work interactions.

The sum of work interactions at the exit minus the inverse exergy.

In the combustion chamber, heat is added.

The mass flow rate in the combustion chamber is one divided by 6.293 times the total mass flow rate.

L equals q_B, which is the heat added per unit mass, calculated as the total heat added divided by the mass flow rate in the combustion chamber, which simplifies to 111.75 kilojoules per kilogram.

The heat added per unit mass based on the total mass flow rate is 111.75 kilojoules per kilogram divided by 6.293.

The temperature is 1289 Kelvin.

The exergy destruction is equal to the change in exergy due to the stream plus the sum of one minus the ratio of the reference temperature to the temperature at state i times twice 43.15 kilojoules per kilogram divided by 1289 Kelvin times 11.95 kilojoules per kilogram divided by 6.293.

This results in 254.073 kilojoules per kilogram.