The equation is:
O equals the mass flow rate at the inlet times the difference in enthalpy between the inlet and the outlet plus the heat transfer rate minus the heat loss rate minus the power output.

The mass flow rate at the inlet equals the mass flow rate at the outlet, which is denoted as m dot.

In the case of a steady-state process:
The heat loss rate equals the mass flow rate times the difference in enthalpy between the inlet and the outlet plus the heat transfer rate.

The heat loss rate is calculated as 0.3 kilograms per second times the difference between 293.88 kilojoules per kilogram and 930.12 kilojoules per kilogram plus 100 kilowatts, which equals 58.86 kilowatts.

The enthalpy at the inlet, at 70 degrees Celsius, is 292.88 kilojoules per kilogram.

X_D is the ratio of the difference between the outlet enthalpy and the saturated liquid enthalpy to the difference between the saturated vapor enthalpy and the saturated liquid enthalpy, evaluated at 100 degrees Celsius.

Thus, the outlet enthalpy equals the saturated liquid enthalpy plus X_D times the difference between the saturated vapor enthalpy and the saturated liquid enthalpy, which calculates to 930.12 kilojoules per kilogram.

The entropy at the inlet, at 70 degrees Celsius, is 0.9549 kilojoules per kilogram Kelvin.

The rate of entropy change during evaporation is calculated as 58.86 kilowatts divided by 293.15 Kelvin plus 0.3 kilograms per second times the difference between 1.3379 kilojoules per kilogram Kelvin and 0.9549 kilojoules per kilogram Kelvin, which equals 100 kilowatts divided by 371.15 Kelvin.

This results in 0.096 kilowatts per Kelvin.