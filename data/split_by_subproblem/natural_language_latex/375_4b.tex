Zero equals the mass flow rate times the difference between enthalpy at state 2 and enthalpy at state 3, plus the heat transfer rate, plus the work rate.

The pressure at state 3 is 8 bar.
The temperature at state 2 is 6 Kelvin below the temperature at state i.
The temperature at state i is 273 Kelvin plus 10 Kelvin, which equals 283 Kelvin.
Therefore, the temperature at state 2 is 283 Kelvin minus 6 Kelvin, which equals 277 Kelvin (5 degrees Celsius).

The enthalpy at state 2 is interpolated using the formula: y equals (x minus x1) divided by (x2 minus x1) times (y2 minus y1) plus y1.

The enthalpy at 40 degrees Celsius is 249.53 kilojoules per kilogram.
The enthalpy at 80 degrees Celsius is 251.80 kilojoules per kilogram.

The enthalpy at 50 degrees Celsius is calculated as one-fourth times (the temperature at state 2 minus 40 degrees Celsius) divided by (80 degrees Celsius minus 40 degrees Celsius) times (the enthalpy at 80 degrees Celsius minus the enthalpy at 40 degrees Celsius) plus the enthalpy at 40 degrees Celsius.

The enthalpy at state 2 is 250.087 kilojoules per kilogram.

The enthalpy at state 3 (at 5 bar) is considered.
The entropy at state 2 equals the entropy at state 3.
The entropy at state 2 is interpolated.