(b) The temperature and the pressure remain constant:

The temperature at the second equilibrium state equals the temperature at the first equilibrium state, which is 0 degrees Celsius.

The pressure of the gas at the second state equals the pressure of the gas at the first state, which is equal to the ambient pressure plus the quotient of the mass of the piston times the acceleration due to gravity, divided by the area of the piston (which is pi times the square of half the diameter). This equals 133970 Pascals or 1.34 bar.

This is because we are in a two-phase region, where heat addition only leads to an increase in temperature and pressure once all the ice has melted. This is not the case when the ice fraction at the second state is greater than zero.

The temperature is 0 degrees Celsius, as the ice does not completely melt (the ice fraction at the second state is greater than zero) and therefore the temperature of the ice remains constant, both the temperature of the gas and the ice at the second state are 0 degrees.

To ensure no more heat is transferred, the temperatures of the gas and the ice at the second state must both be 0 degrees. The pressure also remains constant, as the atmospheric pressure and the weight of the piston as well as the water-ice mixture remain constant.

3.(c) Gas mixture energy balance:

The change in energy equals the mass of the gas at the second state times the internal energy of the gas at the second state minus the mass of the gas at the first state times the internal energy of the gas at the first state, which equals the negative of the heat transferred and the work done.

If the temperature of the equilibrium water at the second state equals the temperature of the gas at the second state, then no more heat is transferred.

Therefore, the temperature of the gas at the second state equals the temperature of the equilibrium water at the second state, which is 0 degrees Celsius or 273.15 Kelvin.

Additionally, the pressure remains constant: the pressure of the gas at the second state equals the pressure of the gas at the first state, which is 1.4 bar.

The mass also remains the same: the mass of the gas at the second state equals the mass of the gas at the first state, which is 0.00342 kilograms.

The volume of the gas at the second state equals the mass of the gas at the second state times the specific gas constant times the temperature of the gas at the second state, divided by the pressure of the gas at the second state, which is 0.001109 cubic meters.

The work done equals the pressure of the gas at the first state times the difference in volume between the second and the first state, which equals -284.48 Joules.

The heat transferred equals the mass of the gas times the difference in internal energy of the gas between the first and the second state minus the work done, which equals the mass of the gas times the specific heat at constant volume times the difference in temperature between the first and the second state minus the work done, which equals 1367.5 Joules.