The expressions are as follows:

- The variables mu_gn and mu_g are questioned as to what their values are.
- The equation pV equals mRT.
- The sum of 0.1 kg and 32 kg.

A description is provided:
- A figure is described as a rectangular container with a horizontal line inside it, indicating a liquid level.

For the calculation:
- The mass of gas, m_g, is calculated using the formula m_g equals pV divided by RT.
- The gas constant, R, is calculated as R equals the universal gas constant, R-bar, divided by mu, which equals 8.314 Joules per mole Kelvin divided by 50 grams per mole.
- This results in R equals 0.16628 Joules per gram Kelvin, which is also expressed as 166.28 Joules per kilogram Kelvin.
- The mass is calculated to be 0.003922 kg.
- The final calculated mass of the gas is underlined as 3.422 grams.