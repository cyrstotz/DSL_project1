The change in external energy, denoted as delta e subscript ext, equals the mass flow rate of the gas, denoted as dot m subscript gas, times the expression in parentheses. This expression includes the difference in enthalpy between state 6 and state 5, minus the product of the ambient temperature T subscript 0 and the difference in entropy between state 6 and state 5, plus half the difference of the squares of velocity W at state 6 and state 5.

The total mass, denoted as m subscript ges, equals an unspecified expression represented by dots.

The equation number 4 is mentioned without further context.

The time t subscript k, XUV, equals the ratio of T over S times s subscript e, z, which further equals T times the ratio of the total mass m subscript ges over S times the specific heat at constant pressure c subscript p times the natural logarithm of the ratio of the temperature T subscript 0 over T subscript 1.

Theta equals the negative product of the total mass m subscript ges and the ratio of s minus S over S, plus s subscript e, z.

The specific heat at constant pressure c subscript p times the natural logarithm of the ratio of the temperature T subscript 0 over T subscript 6.