Part (d)

The equation for \( e_k \) is given by \( e_k = \frac{\dot{Q}_{zu}}{\dot{Q}_{ab} - \dot{Q}_{zu}} \), which is also equal to \( \frac{Q_{zu}}{W_E} \) and \( \frac{Q_K}{W_k} \).

The heat transfer \( Q_k \) is calculated as \( Q_k = m(h_2 - h_1) \).

The heat transfer \( Q_c \) is calculated as \( Q_c = m \cdot \frac{G(40)}{60^2 \text{ seconds}} \cdot (251.8 - 95.42) \) and the result is \( 0.76 \text{ kJ} \).

The graph is a plot on grid paper with the horizontal axis labeled as \( T \) and the vertical axis labeled as \( P \). It contains a curve that starts from the bottom left, rises to a peak labeled \( T_{\text{krit}} \), and then descends towards the bottom right. There are four points marked on the graph:
- Point 1 is located on the left side of the curve, below the peak.
- Point 2 is on the right side of the curve, at the same height as point 1.
- Point 3 is above point 2, on a smaller curve that branches off from the main curve.
- Point 4 is on the left side of the curve, at the same height as point 3.

A horizontal line passes through points 1 and 2. Another line branches off from point 2 and goes upwards, passing through point 3. The label "150 bar" is written near the line that passes through point 3.

The equation \( E \) is calculated as \( E = \frac{176 \, \text{kWh}}{28 \, \text{W}} \) and the result is \( 6.2857 \).