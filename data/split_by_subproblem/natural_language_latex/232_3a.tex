The pressure \( p_{3,1} \) is equal to the initial pressure \( p_0 \) plus the ratio of the mass of ice \( m_{eis} \) over area \( A \) plus the product of the mass of ice \( m_{eis} \), acceleration due to gravity \( g \), divided by area \( A \).

This is calculated as:
\( 1 \times 10^5 \) Pascals plus \( \frac{32.9 \) grams\( }{0.00785} \) plus \( \frac{0.41 \times 9.81}{0.00785} \).

This results in \( 1.401 \times 10^5 \) Pascals, which equals \( 1.407 \) bar.

The area \( A \) is given by \( \pi r^2 \), which is \( \pi \times (0.05 \) meters\()^2 \).

This results in \( 0.00785 \) square meters.

The mass of ice \( m_{eis} \) is calculated using the formula \( \frac{pV}{RT} \).

Where \( R \) is \( 8.314 \) Joules per mole per Kelvin and \( T \) is \( 273.15 \) Kelvin.

The mass of ice \( m_{eis} \) is then calculated as \( \frac{p_3 V_3}{RT} \).

This is \( \frac{1.401 \times 10^5 \) Pascals \( \times 3.14 \times 10^{-3} \) cubic meters\( }{166.28 \) Joules per Kelvin\).

This results in \( 0.00342 \) kilograms, which equals \( 3.42 \) grams.