Graph 1: p(T) Diagram
The graph shows a plot of pressure p (in bar) on the vertical axis versus temperature T (in Kelvin) on the horizontal axis. The curve starts at the origin, rises to a peak, and then falls back down, forming a bell shape. The peak of the curve is labeled as "critical point". To the left of the peak, the region is labeled "wet steam". Several lines are drawn from the bottom left to the top right, intersecting the curve.

Graph 2: T(p) Diagram
The graph shows a plot of temperature T (in Kelvin) on the horizontal axis versus pressure p (in bar) on the vertical axis. The curve starts at the origin and rises steeply, then levels off slightly, and continues to rise. The curve is labeled "Liquid" on the lower part and "Gas" on the upper part. A horizontal line is drawn from the curve to the right, labeled "triple point". Another line is drawn vertically from the horizontal line, labeled "Sublimation". The point where the horizontal line meets the curve is labeled "T equals 401 Kelvin".

The efficiency epsilon is equal to the ratio of the absolute value of the heat transfer rate from the system Q_ab to the heat transfer rate from state 2 to state 1 Q_2 to 1, which is equal to the ratio of the absolute value of the heat transfer rate at point 1c Q_1c to the absolute value of the work done at point 1 W_1, which is also equal to the ratio of the absolute value of the heat transfer rate at point 1c Q_1c to the difference between the heat transfer rate from the system Q_ab and the heat transfer rate from state 2 to state 1 Q_2 to 1.
The heat transfer rate at point 1c Q_1c is equal to the mass flow rate m dot times the difference in enthalpy between state 2 and state 3 h_2 minus h_3.
The heat transfer rate from the system Q_ab is equal to the mass flow rate m dot times the difference in enthalpy between state 4 and state 3 h_4 minus h_3.