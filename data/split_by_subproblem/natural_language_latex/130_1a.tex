The heat output rate, denoted as Q dot subscript "aus":
Q dot subscript "aus" equals Q dot subscript "r" minus Q dot subscript "w".
Q dot subscript "w" approaches zero, which equals m dot times (h subscript "e" minus h subscript "a") plus Q dot subscript "w".
This implies that Q dot subscript "w" equals m dot times (h subscript "a" minus h subscript "e").
The enthalpy at point e, h subscript "e", is calculated as 292.98 plus 0.005 times (2626.5 minus 292.98), resulting in 304.65 kilojoules per kilogram.
The enthalpy at point a, h subscript "a", is calculated as 419.01 plus 0.005 times (2676.1 minus 419.01), resulting in 430.33 kilojoules per kilogram.
At point T subscript "AB" A-2, Q dot subscript "w" equals G subscript "B" times (430.33 minus 304.65), which equals 37.7 kilojoules.
The final value L is calculated as 100 kilograms minus 37.7 kilograms, resulting in 62.30 kilojoules, which equals Q dot subscript "aus".