1. T0, p0 leads to adiabatic, isentropic compression with w12 less than lambda.
2. p1 is greater than p0, and mass flow rate mk is added in the mixer.
3. Mass flow rate mk undergoes high-pressure compression, adiabatic, leading to p2, T2.
4. From point 2 to 3: Combustion chamber, isobaric, heat addition.
5. From point 3 to 4: Adiabatic, irreversible, passed through a turbine with ne less than lambda.
6. After the mixing chamber: T3, w15, p5.
7. Irreversible, adiabatic, passed through a turbine leading to p0, remains at T0.

Graph Description:
The graph is a plot with the x-axis labeled as entropy S in kJ per kg per K and the y-axis labeled as temperature T in Kelvin. The origin is marked as T0.

- A curve starts from point 2 at the bottom left, moving upwards and to the right, labeled as HO with a note "is added".
- The curve continues to point 3, which is marked with a vertical line and labeled as 3.
- From point 3, the curve moves downwards to point 4, which is labeled as 4.
- There is a small segment labeled as 4s between points 4 and 5.
- The curve then moves to point 6, which is labeled as 6.
- Finally, the curve returns to the origin, labeled as O.

Several lines indicating different pressures:
- A line labeled p2 is drawn above the curve.
- Another line labeled pA is drawn even higher.
- A line labeled p0 = p6 is drawn below the curve, intersecting the x-axis at the origin.

Additional notes on the graph include:
- "Irreversible exit" near point 4.
- "Reversible adiabatic entry" near point 6.
- "p6 = p0 - T6 = T0" near the origin.

5) Omega0, T0 at the entrance, pc equals p0.

Isothermal, isobaric, and reversible.

Since it is adiabatic, it becomes isentropic with k equals kappa equals 1.4.

T0 over Ts equals (ps over p0) raised to the power of (k-1) over k, implies T0 equals Ts times (p0 over ps) raised to the power of (k-1) over k.

With mass flow rate m dot equals density rho times area A times omega s, implies omega equals m dot over (rho times A), implies omega0 equals 510 meters per second, Ts equals 340 Kelvin.

Delta Ex, str between 6 and 0.

Delta Ex, str equals (Delta dot Ex, str over m dot) equals (he - ha - h0 (ea - e0) + Delta ke).

1d. Gas model: he - ha equals h0 - ha equals Cp (T0 - Ts).

1d. Gas model: se - sa equals s0 - sa equals (T0 over Ts) times Cp minus R ln (p0 over ps).

R equals Cp minus Cv implies Cp over Cv equals kappa implies Cv equals Cp over kappa.

Implies R equals Cp minus (Cp over kappa) equals 1.006 kJ per kg per K minus 1.006 over 1.4 equals 0.287 equals Cp times j.

Implies Delta Ex, str equals (Cp j (T0 - Ts) - T0 (ln (T0 over Ts) times Cp j - R ln (p0 over ps)) + ((omega2 squared over 2) - (omega0 squared over 2)).

Qex, vel equals Ex, str times O equals sum dot Ex, str plus sum dot Ex, Q minus sum dot EW minus dot Ex, vel equals (dot Ex, vel over m dot) equals (sum dot Ex, str over m dot).

O equals sum dot Ex, str plus dot Ex, Q minus dot Wt minus dot Ex, vel, Od, diss remains constant and its influence is negligible.

Implies dot Qex, vel equals (Ex, vel over m dot ges) equals Delta dot Qex, str plus (1 minus (T0 over TB)) times dot QB.

Implies dot Qex, vel minus Delta dot Qex, str plus (1 minus (T0 over TB)) times dot QB equals 100 kJ per kg (1 minus (243.15 Kelvin over 293 Kelvin)) minus 10 kJ per kg approximately equals 106.72 kJ per kg.