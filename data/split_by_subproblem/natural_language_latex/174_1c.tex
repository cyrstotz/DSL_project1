The rate of entropy generation is denoted as S dot subscript "erz".

The average temperature T bar subscript "R" is calculated as the average of 70 degrees Celsius and 100 degrees Celsius, which equals 85 degrees Celsius.

The equation for entropy generation rate is given by the mass flow rate times the difference in entropy at states e2 and a, plus the heat transfer rate Q dot subscript "j" divided by the temperature T subscript "j", equals the rate of entropy generation S dot subscript "erz".

The entropy at state e2, s subscript e2, at 70 degrees Celsius is 7.7553 kilojoules per Kelvin.

The entropy at state a, s subscript a, at 100 degrees Celsius is 7.3540 kilojoules per Kelvin.

The rate of entropy generation, S dot subscript "erz", is equal to the mass flow rate times the difference between the entropy at state a and state e.

The rate of change of entropy with respect to time, dS/dt, is the sum of the heat transfer rate Q dot divided by the temperature T plus the rate of entropy generation S dot subscript "erz".

The rate of entropy generation, S dot subscript "erz", is equal to the heat transfer rate Q dot divided by the temperature T, which equals 0.221 kilowatts per Kelvin, and this is equal to the rate of entropy generation S dot subscript "erz".

The heat transfer rate Q dot is equal to the negative of the heat transfer rate Q dot subscript "aus", which is negative 6.5 kilowatts.