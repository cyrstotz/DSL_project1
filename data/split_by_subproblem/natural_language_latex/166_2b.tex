Reversible adiabatic equals isentropic.

First law of thermodynamics:

Zero equals the mass flow rate times the quantity of enthalpy at exit plus half the exit velocity squared minus enthalpy at entrance minus half the entrance velocity squared plus the sum of heat transfer rates equals zero.

This implies that enthalpy at exit plus half the exit velocity squared equals enthalpy at entrance plus half the entrance velocity squared.

For an isentropic process where n equals 1.4 equals kappa,

The ratio of temperature T2 over T1 equals the ratio of pressure P2 over P1 raised to the power of (kappa minus one) divided by kappa.

This implies that T2 equals T1 times the ratio of P2 over P1 raised to the power of (kappa minus one) divided by kappa.

T6 equals T5 times the ratio of P0 over P5 raised to the power of (kappa minus one) divided by kappa, which equals 328.07 Kelvin.

The difference in enthalpy between exit and entrance equals the work at entrance, which implies that the difference in enthalpy between exit and entrance plus half the exit velocity squared equals half the entrance velocity squared.

The work at entrance equals the square root of two times the difference in enthalpy between exit and entrance plus the square of exit velocity.

The difference in enthalpy between exit and entrance equals the difference in enthalpy between state 5 and state 6, which equals the specific heat at constant pressure times the difference in temperature between T4 and T2, which equals 1.005 kilojoules per kilogram Kelvin times (431.9 Kelvin minus 328.07 Kelvin).

The work at entrance equals: (the equation is incomplete here).