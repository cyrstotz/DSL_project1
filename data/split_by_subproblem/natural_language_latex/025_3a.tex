P sub g1,1 plus V sub g1,1 equals m sub g1 times R times T sub g1.

R equals 8.314 kilojoules per 80 kilograms, which implies 0.166 kilojoules per kilogram Kelvin.

P sub g1,1 equals question mark.

Wait for equilibrium.

P sub g1,1 equals (m sub ew plus m sub k) times g plus P sub amb.

A sub Zylinder equals (0.5 times 10 to the power of negative 2 meters divided by 2) squared times pi.

This equals 7.854 times 10 to the power of negative 3 square meters.

This equals (0.1 plus 32 kilograms times 9.81 meters per second squared) divided by 7.854 times 10 to the power of negative 3 square meters plus 1 times 10 to the power of 5 Newtons per square meter.

This equals 4.09 times 10 to the power of 5 Pascals equals 4.09 bar plus 1 bar.

This equals 5.09 bar.

This equals 0.4 bar plus 1 bar equals 1.4 bar.

m sub g equals (P sub g1,1 minus U sub g) divided by (R times T sub g1) equals (1.4 times 10 to the power of 2 kiloPascals minus 3.14 times 10 to the power of negative 3) divided by (0.166 kilojoules per kilogram Kelvin times 373.15 Kelvin).

This equals 3.425 grams.