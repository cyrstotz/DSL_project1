The mass flow rate of R134a is denoted as m-dot subscript R134a, followed by the question "1HS". The equation O equals m-dot times the difference of h2 minus h3, plus Q-dot minus W-dot subscript u. The ratio of W-dot subscript u over the difference of h2 minus h3 equals the mass flow rate of R134a. The mechanical power input W-dot subscript u is given as 28 watts. The initial temperature Ti is minus 10 degrees Celsius, which is equivalent to 263.15 Kelvin. The enthalpy h3 equals hf, which is 264.15 kilojoules per kilogram, as found in Table A-11. The mechanical power input W-dot subscript u equals the mass flow rate of R134a times the specific heat capacity Cp times the difference of T2 minus T3. The temperature T3 is 31.33 degrees Celsius, while T2 is not specified. It is known that the entropy s2 equals s3.