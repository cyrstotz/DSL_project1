Energy balance of semiconductors:

Delta E equals m subscript 2 times u subscript 2 minus m subscript 1 times u subscript 1 equals the sum over Delta m subscript i times the quantity h subscript i plus v subscript i squared over 2 plus g times z subscript i end quantity plus the sum over Q subscript i minus the sum over W subscript i minus the sum over mu subscript i times n subscript i.

Equals (m subscript 1 plus Delta m) times u subscript 2 minus m subscript 1 times u subscript 1 minus Delta m times h subscript e plus Q subscript 12.

m subscript 1 times u subscript 1 plus Delta m times u subscript 2.

Minus Delta m times canceled h subscript e.

Delta m equals the fraction with numerator Q subscript 12 plus m subscript 1 times u subscript 1 minus m subscript 1 times u subscript 2 and denominator u subscript 2 minus h subscript e, with the note "mit".

Equals 2.7 times 10 to the power of 6 kilograms.

Equals 3924.3 kilograms.

Q subscript 12 equals 35 times 10 to the power of 6 Joules equals 35 times 10 to the power of 6 Joules.

m subscript 1 equals 5755 kilograms.

u subscript 1 equals u subscript f at 100 degrees Celsius plus x times u subscript g at 100 degrees Celsius.

h subscript e equals h subscript e at 20 degrees Celsius.

u subscript 2 equals u subscript f at 70 degrees Celsius.

Confirming:

u subscript 1 equals 418.94 plus 0.005 times (2506.5 minus 418.94).

u subscript 2 equals 292.75.

h subscript e equals 83.96.