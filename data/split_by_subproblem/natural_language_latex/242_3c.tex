The change in internal energy from state 1 to state 2, denoted as Delta U_g1,2, is equal to the heat added, Q, minus the work done, W.

The work done from state 1 to state 2, W_g1,2, is calculated as the integral of pressure times the change in volume, which simplifies to pressure times the difference between volume at state 2 and volume at state 1.

The volume at state 1, V_1, is 3.14 liters.

The volume at state 2, V_2, is given by the formula mRT divided by p, which equals 2.477 times 10 to the power of negative 3 cubic meters.

Substituting the values, the work done, W_g1,2, is found to be 95.47.

Reiterating, the change in internal energy from state 1 to state 2 is the heat added minus the work done.

Adding the change in internal energy and the work gives the heat added from state 1 to state 2, Q_g1,2, which is 1017.65 Joules per kilogram.

The equation involving mass, m, specific heat at constant volume, c_v, the temperature difference from state 2 to state 1, plus the work, equals the heat added, 1017.65 Joules per kilogram, denoted as Q.

Finally, the heat added, Q, is negative 1017.65.