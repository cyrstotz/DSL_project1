1. Left-hand side (Reactor): The change in energy, Delta E, equals the mass flow rate at the outlet, m-dot subscript u2, times the specific enthalpy at the outlet, h subscript u2, minus the mass flow rate at the inlet, m-dot subscript u1, times the specific enthalpy at the inlet, h subscript u1, which simplifies to the mass flow rate at the outlet, m-dot subscript u2, times the difference in specific enthalpies at the outlet and inlet, h subscript u2 minus h subscript u1, plus the heat transfer out, Q-dot subscript aus.

This implies that negative m-dot subscript u2 equals the change in specific enthalpy at the outlet, Delta h subscript u2, divided by the difference in specific enthalpies minus the heat transfer out, h subscript u2 minus h subscript u1 minus Q-dot subscript aus, which simplifies to the mass flow rate at the outlet times the difference in specific enthalpies divided by the heat transfer out, m-dot subscript u2 times (h subscript u2 minus h subscript u1) over Q-dot subscript aus.

This further implies that negative m-dot subscript u2 equals the difference in specific enthalpies, h subscript u2 minus h subscript u1, divided by itself, h subscript u2 minus h subscript u1.

The specific enthalpy at the outlet, h subscript u2, at 20 degrees Celsius is 252.98 kilojoules per kilogram, as found in Table A2.

The specific internal energy at the inlet, u subscript 1, equals the quality, x, times the difference in specific internal energies at saturation, u subscript fg, at 100 degrees Celsius, plus one minus the quality, times the specific internal energy at the fluid state, u subscript f, at 100 degrees Celsius.

The specific internal energy at the outlet, u subscript 2, at 20 degrees Celsius is 252.98 kilojoules per kilogram, as found in Table A2.

The change in mass flow rate at the outlet, Delta m subscript u2, is 253.6 kilojoules per kilogram.