The content includes several mathematical expressions and equations related to thermodynamics and fluid dynamics:

1. The transitions from state 4 to 5 and from state 1 to 5 are both isobaric (constant pressure processes).

2. Given values are pressures \( p_1, p_2 \), and temperature \( T_5 \). The pressure \( p_1 \) is defined as \( p_0 \) plus \( 0.75 \) times \( B_r \).

3. Using the formula for temperature ratio across a process, where \( \frac{T_6}{T_5} \) equals \( \left( \frac{p_6}{p_5} \right)^{\frac{n-1}{n}} \) with \( n \) equal to \( k \) which is \( 1.4 \). This results in the calculation of \( T_6 \) as \( 437.9 \) Kelvin times \( \left( \frac{0.93 \times 10^5}{0.5 \times 10^5} \right)^{\frac{0.4}{1.4}} \), yielding \( 328.07 \) Kelvin. The temperature \( T_0 \) is set to \( 328.07 \) Kelvin.

4. Using the energy equation, \( \frac{dE}{dt} = \dot{m} \left( h_{e1} - h_{e2} \right) + \dot{Q} - \dot{W} \), several conditions are set to zero involving mass flow rate \( \dot{m} \), specific enthalpies \( h_1, h_{e1}, h_5, h_6 \), and velocities \( v, w_6 \). The equation simplifies to \( 0 = \dot{m} \left( cp(T_5 - T_6) + \frac{v^2}{2} - \frac{w_6^2}{2} \right) \).

5. The mass flow rate \( \dot{m} \) is calculated as \( \frac{\dot{Q}_{13}}{u_6} = 5.203 \) and \( q_b \) as \( \frac{\dot{Q}_{13}}{u_6} = 1.085 \).

6. The term "Aerot" is mentioned but not further elaborated or defined in the provided text.