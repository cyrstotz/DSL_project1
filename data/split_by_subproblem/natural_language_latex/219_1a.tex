Zero equals the mass flow rate at the inlet times the difference between the enthalpy at the inlet and the enthalpy at the outlet, plus the heat transfer rate from the reactor, plus the heat transfer rate at the outlet. The enthalpy at the inlet is the enthalpy of fluid at 70 degrees Celsius with a quality of zero, which equals 252.80 kilojoules per kilogram. The enthalpy at the outlet is the enthalpy of fluid at 100 degrees Celsius, which equals 419.04 kilojoules per kilogram. The heat transfer rate at the outlet equals the mass flow rate at the inlet times the difference between the enthalpy at the outlet and the enthalpy at the inlet, minus the heat transfer rate from the reactor, which equals negative 62.182 kilowatts. This negative value indicates that heat is being removed from the reactor.

The first law of thermodynamics for a steady flow process is represented as the change in enthalpy plus the reversible heat transfer equals zero.

In a table format, the temperature at state 1 is 100 degrees Celsius, at state 2 is 70 degrees Celsius, and the change is 20 degrees Celsius. The mass at state 1 is 5755 kilograms, and the quality at state 1 is 0.005, at state 2 is zero.

The heat transfer and work are both zero.

For the first law of thermodynamics:
The negative of the mass at state 2 times the internal energy at state 2 minus the mass change times the internal energy change equals the mass change times the enthalpy at the inlet.

The internal energy at state 1 is calculated as the internal energy of fluid plus the quality times the difference between the internal energy of steam and the internal energy of fluid, which equals 428.38 kilojoules per kilogram. The internal energy at state 2 is the internal energy of fluid at 70 degrees Celsius, which equals 252.35 kilojoules per kilogram. The enthalpy at the inlet is the enthalpy of fluid at 20 degrees Celsius, which equals 83.36 kilojoules per kilogram.

The mass at state 2 equals the mass at state 1 plus the mass change.

The mass change times the enthalpy at the inlet minus the sum of the mass at state 1 and the mass change times the internal energy at state 2 equals the mass at state 1 times the internal energy at state 1.

This leads to the negative of the mass change times the difference between the enthalpy at the inlet and the internal energy at state 2 equals the mass at state 1 times the internal energy at state 2 minus the mass at state 1 times the internal energy at state 1.

Finally, solving for the mass change gives 1178 kilograms.