T sub 2 equals negative 10 degrees Celsius which is equal to 263.15 Kelvin.

(from graph)

h sub 2 equals h sub g of negative 10 degrees which is calculated as (h of negative 8 minus h of negative 12) divided by (negative 8 plus 12) times (negative 10 plus 12) plus h of negative 12 minus table A-10.

This equals 241.345 kilojoules per kilogram.

Isentropic

s sub 2 equals s sub 3.

This implies s sub 2 equals s sub g of negative 10 degrees which is calculated as (s of negative 8 minus s of negative 12) divided by (negative 8 plus 12) times (negative 10 plus 12) plus s of negative 12 minus table A-10.

This equals 0.9253 kilojoules per kilogram Kelvin.

This implies h sub 3 at 8 bar, s sub 2 is calculated as (h of 40 minus h of 31.33) divided by (s of 40 minus s of 31.33) times (s sub 2 minus s of 31.33) plus h of 31.33.

This equals 269.92 kilojoules per kilogram.

This implies mass flow rate equals the work rate from 2 to 3 divided by (h sub 2 minus h sub 3) equals (3.798 times 10 to the power of 4 kilojoules per second) divided by (241.345 minus 269.92 kilojoules per kilogram) equals 3.52 kilograms per second.

Throttle Adiabatic: h sub 4 prime equals h sub 4.

p sub 2 equals p sub 1 minus 10 degrees equals (p of negative 12 plus p of negative 8) divided by 2 equals approximately 2.0122 bar, approximately 2 bar.

x sub A equals (h sub 4 minus h sub f of p sub 2) divided by (h sub f of p sub 2 minus h sub f of p sub 2) equals 0.277.