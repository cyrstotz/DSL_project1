Diagram of a rectangular box:
- A rectangle with horizontal sides labeled 0 and 6.
- The vertical sides are not labeled.
- There are arrows indicating directions along the sides of the rectangle.

The rate of change of energy with respect to time is equal to the sum over i of the mass flow rate of i times the sum of the enthalpy of i at time t, half the velocity of i at time t squared, plus g times the height of i at time t, plus the sum over i of the heat flow rate of i at time t, minus the sum over i of the work flow rate of i at time t.

Zero equals the mass flow rate at 6 times the sum of the enthalpy at 0 plus a constant k0 minus the enthalpy at 6 minus a constant k6, which implies that the enthalpy at 6 equals the enthalpy at 0 minus k0 plus k6.

The enthalpy at 0 equals half the square of the velocity, which is half the square of 1200 meters per second, resulting in 20,000 square meters per second squared.

Half the square of the velocity at 6 equals the enthalpy at 0 minus k0 plus k6.

The difference between the enthalpy at 0 and the enthalpy at 6 equals the integral from temperature at 0 to temperature at 6 of the specific heat capacity times the differential of temperature, which equals 1.066 kilojoules per kilogram times the difference between -243.15 Kelvin and 328.07 Kelvin, resulting in +85.92 kilojoules per kilogram.

The mass flow rate at 6 equals the mass flow rate at 0, with the exiting velocity in a nozzle being isentropic.

The ratio of the temperature at 6 to the temperature at 0 equals the ratio of the pressure at 6 to the pressure at 0 raised to the power of (kappa minus 1) divided by kappa, which implies that the temperature at 6 equals 437.9 Kelvin times the ratio of 0.196 bar to 0.56 bar raised to the power of (7 minus 1) divided by 7, resulting in 328.07 Kelvin.

The velocity at b equals the square root of 2 times the sum of the enthalpy at 0 minus the enthalpy at 6 plus an additional enthalpy term.

This equals the square root of 2 times the sum of t plus 85,429.5 joules per kilogram plus 20,000 square meters per second squared.

This results in a velocity of 49.19 meters per second.

The units of square meters per second squared are equivalent to Newton meters per kilogram, which are also equivalent to joules per kilogram.