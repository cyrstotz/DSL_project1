The pressure \( p_{g,1} \) is equal to the ambient pressure \( p_{amb} \) plus the product of the sum of the masses \( m_k \) and \( m_{ew} \), the acceleration due to gravity \( g \), and the reciprocal of the area \( A \). This is expressed as:
\[ p_{g,1} = p_{amb} + (m_k + m_{ew}) \cdot g \cdot \frac{1}{A} \]
Substituting the values, it becomes:
\[ p_{g,1} = 1.10^5 \, \text{Pa} + \frac{(m_k + m_{ew}) \cdot g}{A} \]
The area \( A \) is given by:
\[ A = \pi \cdot d^2 \left[ \frac{\text{m}}{4} \right] = 0.052 \pi \approx 0.0079 \, \text{m}^2 \]
Substituting the values for \( m_k \), \( m_{ew} \), \( g \), and \( A \) into the equation for \( p_{g,1} \), we get:
\[ p_{g,1} = 1.10^5 \, \text{Pa} + \frac{(32 \, \text{kg} + 0.1 \, \text{kg}) \cdot 9.81 \, \frac{\text{m}}{\text{s}^2}}{0.0079 \, \text{m}^2} \]
\[ p_{g,1} \approx 140094 \, \text{Pa} \]
\[ \approx 1.4 \, \text{bar} \]

For the mass \( m_{g,1} \), using the ideal gas equation \( pV = mRT \), we rearrange to find \( m_{g,1} \):
\[ m_{g,1} = \frac{RT_{g,1}}{p_{g,1} V_{g,1}} \]
The specific gas constant \( R \) is calculated as:
\[ R = \frac{R}{M_{g,1}} = \frac{8.314 \, \frac{\text{kJ}}{\text{kmol K}}}{50 \, \frac{\text{kg}}{\text{kmol}}} \approx 0.166 \, \frac{\text{kJ}}{\text{kg K}} \]
Substituting the values into the equation for \( m_{g,1} \), we get:
\[ m_{g,1} = \frac{1.4 \cdot 10^5 \, \text{Pa} \cdot 3.14 \cdot 10^{-3} \, \text{m}^3}{0.166 \, \frac{\text{kJ}}{\text{kg K}} \cdot (50 + 273.15) \, \text{K}} \]
\[ \approx 2.92 \, \text{g} \]