Graph 1:

The graph is a plot with the vertical axis labeled T and the horizontal axis labeled S. The plot starts at the origin and rises steeply, then curves back down, forming a loop. The loop is labeled with points 1, 2, and 3, with arrows indicating the direction of movement through these points. The graph then continues to rise again after the loop.

Graph 2:

The second graph is a plot with the vertical axis labeled T with units in Kelvin [K] and the horizontal axis labeled S with units in [K/log K]. The plot starts at the origin and rises steeply to point 1, then curves back down to point 2, rises again to point 3, and then falls to point 4. The graph then rises slightly to point 5 and falls again to point 6. The points are labeled sequentially from 1 to 6, with arrows indicating the direction of movement through these points. There is a note below the graph stating "1 - higher temperature than 4".

Energy Equation: Steady Flow Process, First Law of Thermodynamics

The equation is zero equals the mass flow rate times the difference in enthalpy from state 2 to state 1, plus half the difference between the square of the speed at state 2 and the square of the speed at state 6, plus the sum of heat transfers minus the sum of work inputs.

It is noted that the sum of heat transfers is zero, as the process is adiabatic.

The sum of work inputs minus the work outputs is equal to the mass flow rate times the integral from state 1 to state 2 of specific volume times differential pressure, minus the integral from state 1 to state 2 of mass flow rate times efficiency times the integral of pressure times differential specific volume.

This results in negative mass flow rate times the gas constant times efficiency times the difference in temperature from state 2 to state 1, divided by 1 minus efficiency.

Efficiency is given as k equals 1.4.

The ratio of the speed at state s to the speed at input equals the ratio of the speed at output to the speed between states 1 and 2 minus 1, which implies thermal efficiency.

The equation is enclosed in a box, indicating it is a final or important result.