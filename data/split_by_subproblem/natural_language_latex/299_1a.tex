a) What is the value of Q dot out? The equation is zero equals m dot in times (h in minus h out) plus m dot core times (h core minus h out) plus Q dot R minus Q dot out.

This leads to the equation Q dot out equals m dot in times (h in minus h out) plus Q dot R.

For A-2: The enthalpy at the inlet at 70 degrees Celsius is 292.98 times 10 to the power of 3.

For A-2: The enthalpy at the outlet at 100 degrees Celsius is 419.04 times 10 to the power of 3, which leads to the equation Q dot out equals 0.3 kilograms per second times (h in minus h out) plus 100 kilowatts.

The result is Q dot out equals 67.832 kilowatts.