T sub G,2 equals the fraction with numerator Q sub EW plus k times Q sub G,1 and denominator m times c sub m.

T sub G,2 equals the fraction with numerator Q sub EW,1 plus k times Q sub G,1 and denominator m times c sub m.

T sub G,2 equals the fraction with numerator Q sub EW,2 plus k times Q sub G,2 and denominator m times c sub m.

This equation is repeated several times with the same variables and constants.

Q sub 1,2 equals m sub g times C sub p times the quantity T sub g,2 minus T sub g,1.

C sub p equals k times C sub v equals 0.80 kilojoules per kilogram per Kelvin.

Q sub 1,2 equals 3.6 grams times 0.8 kilojoules per kilogram per Kelvin times the quantity 0.003 degrees Celsius minus 500 degrees Celsius equals negative 1.44 kilojoules.