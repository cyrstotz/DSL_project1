A equals the square of half of D times pi, which equals 5 square centimeters, or 7.85 times 10 to the negative fifth square meters.

The sum of the masses of K and W divided by A plus the ambient pressure equals p sub g,1, which equals the sum of 0.1 kilograms and 32 kilograms times 9.81 meters per second squared, divided by 7.85 times 10 to the negative fifth square meters, plus 1 bar, equals 9.40 bar.

p times V equals m times R times T.

The mass of g equals p sub g,1 times V sub g,1 divided by R times T sub g,1, which equals 1.406 bar times 3 cubic meters divided by 8.314 Joules per mole Kelvin times 773.15 Kelvin, equals 3.42 grams.

R equals the universal gas constant R divided by the molar mass of g, which equals 8.314 Joules per mole Kelvin divided by 50 grams per mole, equals 0.17 Joules per gram Kelvin.