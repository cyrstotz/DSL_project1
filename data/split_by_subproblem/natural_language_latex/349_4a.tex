The graph is a phase diagram with pressure \( p \) on the y-axis and temperature \( T \) on the x-axis. The y-axis is labeled \( p \) in bars and the x-axis is labeled \( T \) in Kelvin. There are three regions labeled:
- "Solid" in the lower left region.
- "Liquid" in the upper region.
- "Gaseous" in the lower right region.

The boundary lines between these regions are drawn as follows:
- A curve separating the "Solid" and "Liquid" regions.
- A curve separating the "Liquid" and "Gaseous" regions.
- A curve separating the "Solid" and "Gaseous" regions.

The point where all three curves meet is labeled "Triple point".

There are two arrows labeled "i" and "ii" indicating transitions:
- Arrow "i" starts in the "Solid" region, moves vertically upwards, and then horizontally to the right into the "Liquid" region.
- Arrow "ii" starts in the "Solid" region, moves horizontally to the right, and then vertically upwards into the "Gaseous" region.

The equation \( \mathcal{E}_K = \frac{|\dot{Q}_{ab}|}{|\dot{W}_e|} = \frac{-\dot{Q}_{ab}}{-\dot{W}_K} \) is also provided.