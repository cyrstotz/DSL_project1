The derivative with respect to time t is equal to the mass flow rate dot m times the sum of the enthalpy at inlet h sub i, half of the speed of light squared, plus xi times the heat transfer rate Q sub i divided by the mass flow rate dot m, minus the sum of the work rate W sub i divided by the mass flow rate dot m.

The rate of heat transfer out dot Q sub aus is equal to the mass flow rate dot m times the difference between the enthalpy at inlet h sub ein and the enthalpy at outlet h sub aus, plus the rate of heat transfer out dot Q sub aus, plus the rate of heat transfer R dot Q sub R.

The rate of heat transfer out dot Q sub aus is equal to the mass flow rate dot m times the difference between the enthalpy at inlet h sub ein and the enthalpy at outlet h sub aus, plus the rate of heat transfer R dot Q sub R.

The enthalpy at inlet h sub ein is equal to the enthalpy of fluid h sub f at 70 degrees Celsius plus x times the difference between the enthalpy of gas h sub g at 70 degrees Celsius and the enthalpy of fluid h sub f at 70 degrees Celsius, from table A-2.

This equals 301.65 kilojoules per kilogram.

The enthalpy at outlet h sub aus is equal to the enthalpy of fluid h sub f at 100 degrees Celsius plus x times the difference between the enthalpy of gas h sub g at 100 degrees Celsius and the enthalpy of fluid h sub f at 100 degrees Celsius, from table A-2.

This equals 430.33 kilojoules per kilogram.

The rate of heat transfer out dot Q sub aus is 62.296 kilowatts.