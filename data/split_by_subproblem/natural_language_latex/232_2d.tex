Zero equals the change in exergy of sir plus the exergy of Q minus the work input minus the exergy of velocity. The exergy of Q is calculated as one minus the ratio of T0 over T, multiplied by Q, which equals one minus the ratio of 243.75 over 300, multiplied by 179.5, resulting in 36.958 kilojoules per kilogram.

The exergy of velocity is calculated as the square of velocity over two plus the integral from s0 to s2 of T0 ds, which simplifies to the square of velocity over two plus T0 times the difference between s2 and s1 minus Q over T.

The difference between s2 and s1 is calculated as the natural logarithm of the ratio of T0 over T0 minus R times the ratio of P0 over P0, which simplifies to the natural logarithm of the ratio of 300 Kelvin over 243.75 Kelvin minus the ratio of 1.393 kilojoules per kilogram Kelvin over 1.239 kilojoules per kilogram Kelvin, resulting in 1.262 kilojoules per kilogram.

Finally, the exergy of velocity is calculated as 243.15 plus 1.262, resulting in 306.855 kilojoules per kilogram.