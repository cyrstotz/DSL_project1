The average temperature for the cooling fluid, denoted as T-bar subscript KF, is questioned.

The average temperature, T-bar, is defined as the integral of temperature T with respect to s from start point s_a to end point s_e, divided by the difference between s_e and s_a.

The integral from T_1 to T_2 of c over T with respect to T equals c times the natural logarithm of the ratio of T_a over T_e.

The product of T and ds equals q.

In the energy balance equation, zero equals the mass of the cooling fluid, m subscript KF, times the difference between the enthalpy at the end point and the start point, plus the heat output Q subscript out.

q equals the heat output Q subscript out divided by the mass of the cooling fluid, which is also equal to the mass of the cooling fluid times the difference between the enthalpy at the end point and the start point divided by c times the difference between T_a and T_e, and this is denoted as psi, which equals c times theta times the difference between T_a and T_e.

The average temperature for the cooling fluid, T-bar subscript KF, is calculated as q times the difference between T_a and T_e divided by q times the natural logarithm of the ratio of T_a over T_e, which equals 293.12 Kelvin.