The ice fraction x subscript Eis comma 2 is greater than zero, T subscript 3 comma 2 minus question mark, p subscript 3 comma 2 minus question mark.

R and m are constants, V is smaller.

p subscript 3 comma 2 because p subscript 3 comma 2 is not dependent on volume but only on p and m.

Therefore, T subscript 1 equals the fraction of pV over mR, T is smaller.

Polytropic temperature ratio.

T subscript 2 equals T subscript 1 times the fraction of p subscript 2 over p subscript 1 raised to the power of (n minus 1) over n.

C subscript p equals R plus C subscript v equals 0.1663 kilojoules per kilogram Kelvin plus 0.633 kilojoules per kilogram Kelvin equals 0.7993 kilojoules per kilogram Kelvin.

Continuing section 3b:

kappa equals the fraction of c subscript p over c subscript v equals the fraction of 0.7993 over 0.633 equals 1.263.

V becomes smaller, p becomes larger, and since the gas also loses heat, T is smaller.

T subscript 2 equals 773 Kelvin times the fraction of p subscript 2 over 1.5.

The fraction 0.26 over 1.26.