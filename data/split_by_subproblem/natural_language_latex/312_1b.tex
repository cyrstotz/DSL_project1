(b)

The average temperature T-bar for the cooling fluid is given by the integral of temperature T with respect to s from s-out to s-in, divided by the difference between s-out and s-in.

The ratio of the difference in enthalpy (h-out minus h-in) to the difference in entropy (s-out minus s-in) for an ideal fluid.

The differential change in enthalpy dH equals T times ds plus V times dP, which under the conditions of being at point O and isobaric, leads to:

This equals the specific heat at constant pressure for the cooling fluid, c_p,KF, times the difference in temperature (T-out minus T-in) plus the volume V times the difference in pressure (P-out minus P-in).

This further simplifies to the specific heat at constant pressure for the cooling fluid, c_p,KF, times the natural logarithm of the ratio of T-out to T-in.

The ratio of the difference in temperatures (T-out minus T-in) to the natural logarithm of the ratio of T-out to T-in equals the difference in temperatures 298.15 Kelvin minus 288.15 Kelvin divided by the natural logarithm of the ratio of 298.15 Kelvin to 288.15 Kelvin, which equals 293.12 Kelvin.