The enthalpy at state s, denoted as h_s, is equal to the specific heat capacity at constant pressure, c_p, multiplied by the temperature at state s, T_s, which equals 431.49 kilojoules per kilogram.

The energy balance for the nozzle, which is adiabatic, shows that the heat transfer Q equals zero and the system is steady.

The equation for the system is zero equals the mass flow rate, denoted as dot m with subscript ges, times the quantity of enthalpy at state 5 minus enthalpy at state 6 plus one-half the square of velocity at state 5 minus one-half the square of velocity at state 6.

From this, it follows that the enthalpy at state 6, h_6, equals the enthalpy at state 5, h_5, plus one-half times the difference between the square of velocity at state 5 and the square of velocity at state 6, according to the ideal gas law.

For the condition T_0, the process from s to s is isentropic, implying that the polytropic exponent n equals the specific heat ratio k.

The polytropic temperature ratio is given by the temperature at state 6, T_6, equals the temperature at state s, T_s, times the ratio of pressure at state 0 to pressure at state s raised to the power of (k-1) divided by k, which results in 293.9 Kelvin, where the enthalpy at state 6, h_6, equals the specific heat capacity at constant pressure, c_p, times the temperature at state 6, T_6.

It follows that the velocity at state 6, w_6, equals the square root of two times the difference between the enthalpy at state s and the enthalpy at state 6 plus the square of the velocity at state 5, which equals 571 meters per second.

The text "Rückseite!" translates to "Back side!" indicating there might be additional content on the back side of the page.