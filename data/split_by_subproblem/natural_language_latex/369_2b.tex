T sub S equals 437.9 Kelvin.
w sub s equals 220 meters per second.
p sub S equals 0.5 bar.
m sub s equals ingeo.

h sub s equals c sub p times T sub S, assuming an ideal gas.
h sub s equals 7.006 kilojoules per kilogram Kelvin times 437.9 Kelvin equals 43.69 kilojoules per Kelvin.

For an adiabatic process: T sub 0 equals T sub S times the ratio of p sub 0 over p sub S raised to the power of R over c sub p equals 437.9 Kelvin times the ratio of 0.1 over 0.5 bar raised to the power of 0.19 kilojoules per kilogram Kelvin over 0.4 kilojoules per kilogram Kelvin equals 325.07 Kelvin.

h sub 0 equals c sub p times T sub 0 equals 330 kilojoules per kilogram, leading to the overall system.

For Q dot equals m dot times (h sub 0 minus h sub 6 plus the difference of w sub 2 squared minus w sub 1 squared over 2) plus Q.

h sub 0 equals c sub p times 263.15 Kelvin equals 2.4 kilojoules per kilogram.

Additional Notes:
- The student mentions "Thrust nozzle adiabatic: Stationary" and "reversible thrust nozzle: S dot generated equals zero."
- There are notes about "adiabatic" and "reversible (adiabatic)" processes.
- The student also writes "V. dp c sub v (p sub 0 minus p sub 3)" and "c sub v p sub 0 minus p sub 3".