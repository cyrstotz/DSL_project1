Q equals the sum of E subscript ex, ist, 0-6, plus E subscript ex, Q, minus E subscript ex, vel.

E subscript ex, vel equals the sum of E subscript ex, ist, 0-6 and E subscript ex, Q.

E subscript ex, Q equals the product of one minus the fraction T subscript 0 over T, and the absolute value of dot Q.

Dot Q equals 11.85 kilojoules per kilogram.

The sum of dot m subscript in and dot m subscript ex equals 6.733 times dot m.

Dot m subscript in equals 5.203.

Dot m subscript in equals 5.203 times dot m.

Dot m subscript in equals dot m times the fraction of m subscript in over m subscript ex.

Dot m subscript in equals dot m times the fraction of m subscript in over m subscript ex equals dot m times the fraction 1.

The total force vector F subscript ges equals the derivative with respect to time of m times vector v, which equals m times vector a plus dot m times vector v, labeled as equation (1).

Vector a subscript exp equals vector a subscript exp, stör-f plus the fraction of dot m over m times vector v subscript exp.

This equals negative 11.6 times the unit 6 per second squared plus the fraction 110.5 kilograms per second over 6 kilograms per second equals 73.3 kilograms per second, circled.

This leads to vector a subscript exp, neu.