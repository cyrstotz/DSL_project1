The derivative of E dot with respect to time is equal to the mass flow rate at the exit, m dot subscript e, times the difference in enthalpy between the exit and the ambient, h subscript e minus h subscript a, plus the heat transfer rate Q dot subscript R, plus the heat transfer rate Q dot subscript aus, minus the product of X dot and zero.

The enthalpy at the exit, h subscript e, is equal to x subscript D times the enthalpy at the fluid state at 70 degrees Celsius, h subscript f at 70 degrees, plus one minus x subscript D times the enthalpy at the fluid state at 70 degrees Celsius, h subscript f at 70 degrees, which equals 304.65 kilojoules per kilogram, as referenced from Table A2.

The ambient enthalpy, h subscript a, is equal to x subscript D times the enthalpy at the fluid state at 200 degrees Celsius, h subscript f at 200 degrees, plus one minus x subscript D times the enthalpy at the fluid state at 200 degrees Celsius, h subscript f at 200 degrees, which equals 403,430.33 kilojoules per kilogram, and is equal to the enthalpy at the fluid state at 200 degrees Celsius, h subscript f at 200 degrees.

The product of the mass flow rate at the exit, m dot subscript e, and the difference in enthalpy between the exit and the ambient, h subscript e minus h subscript a, plus the heat transfer rate Q dot subscript R, is equal to the heat transfer rate Q dot subscript aus, which equals positive 62.182 kilojoules per second.