a) The heat flow out, denoted as Q-dot-out, using Energy Balance I is described by the equation:

Zero equals m-dot times (h1 minus h2) plus Q-dot-2 equals m-dot times (h1 at 20 degrees Celsius minus h2 at 100 degrees Celsius) minus Q-dot-out.

The difference in enthalpy (h1 minus h2) for water is given by h1 at T equals 343.14 Kelvin minus h2 at T equals 373.14 Kelvin.

Under the subsection "Steam Table x1 equals x2":

The enthalpy h1 is equal to h_f plus x1 times (h_fg minus h_f) and has temperature T1.

The value of h1 is 304.649.

The enthalpy h2 is equal to h_f plus x2 times (h_fg2 minus h_f2).

The value of h2 is 430.82.

This implies "See page 3".

The temperature at h1 is 292.98, and the enthalpy at v2 is 419.04.

The enthalpy of vaporization at h_fg1 is 262.68 and at h_fg2 is 267.6, both in kilojoules per kilogram.

The mass flow rate m-dot leads to x1 minus x2.

In the student solution section:

a) Description of two graphs:
- Left Graph: A graph with pressure P on the vertical axis and volume V on the horizontal axis features two curves: one descending from the top left to the right and another ascending from the bottom left to the right, forming a closed loop with a clockwise direction. Points 1 and 2 are marked on this loop.
- Right Graph: A graph with pressure P on the vertical axis and enthalpy h on the horizontal axis displays three curves: one descending from the top left to the right, another ascending from the bottom left to the right, and a third curve ascending from the middle left to the right. Points 1, 2, and 3 are marked, with directional arrows from 1 to 2 and from 2 to 3. The label "Compressor" is near the top curve.

The efficiency eta_c is given by the ratio of Q-dot-in to W-dot-r, which equals the absolute value of Q-dot-l divided by the sum of the absolute values of Q-dot-l and W-dot-r.