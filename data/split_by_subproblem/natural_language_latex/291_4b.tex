(b) see

MiR134a adiabatic

From state 2 to state 3 is isentropic, the work done \( W_k \) is 28 watts, the pressure at state 3 \( P_3 \) is 8 bar, the quality at state 3 \( x_3 \) is 1, and the pressure at state 2 \( P_2 \) is equal to the pressure at state 1 \( P_1 \).

The equation is zero equals the mass flow rate times the difference in enthalpy from state 2 to state 3 plus \( \frac{c}{k} \) plus \( \frac{pe}{k} \) plus the heat transfer rate minus the work done, under adiabatic conditions.

The work done \( W_k \) equals the mass flow rate times the difference in enthalpy from state 2 to state 3, leading to the mass flow rate of R134a being equal to the work done divided by the difference in enthalpy from state 2 to state 3.

The entropy at state 2 \( S_2 \) equals the entropy at state 3 \( S_3 \).

The enthalpy at state 2 \( h_2 \) at quality \( x_2 = 1 \) and temperature \( T_2 = 9^\circ C \) is calculated as follows: \( h_2 = \frac{254.03 - 251.80}{12 - 8} (9 - 8) + 251.80 = 252.358 \frac{kJ}{kg} \) with a delta of 10.

The entropy at state 3 \( S_3 \) equals the entropy at state 2 \( S_2 \) and is calculated as \( S_3 = S_2 = 0.9 \frac{232 - 99.150}{12 - 8} (9 - 8) + 99.150 = 99.1455 \frac{kJ}{kgK} \).

At 8 bar, 5.53 with a delta of 12.

The enthalpy at state 3 \( h_3 \) is calculated as \( h_3 = 273.66 - 266.15 (0.9 (1455.8 - 9906) + 266.15 \).

The values \( 0.9374 - 0.9065 \).

The enthalpy at state 3 \( h_3 \) is \( 266.605 \frac{kJ}{kg} \).

The mass flow rate of R134a is calculated as \( \dot{m} \text{R134a} = \frac{-1 W_k}{h_2 - h_3} = \frac{4.965 \frac{kg}{s}}{1.965 \frac{g}{s}} \).