b) The process from state 5 to state 6 is an isentropic state change. The specific enthalpy \( h_e \) is equal to the square of the velocity \( w \) divided by two, all over the mass flow rate \( \dot{m} \), which equals \( gAw \). The specific heat ratio \( k \) is 1.4. The temperature ratio \( \frac{T_6}{T_5} \) is equal to the pressure ratio \( \frac{p_5}{p_6} \) raised to the power of \( \frac{k-1}{k} \). The temperature at state 6 \( T_6 \) is equal to the temperature at state 5 \( T_5 \) times the pressure ratio \( \frac{p_5}{p_6} \) raised to the power of \( \frac{k-1}{k} \), which equals 568.58 Kelvin. The pressure at state 6 \( p_6 \) is equal to the base pressure \( p_0 \), which is 0.191 bar. The temperature at state 5 \( T_5 \) is 431.9 Kelvin. The pressure at state 5 \( p_5 \) is 0.5 bar.

Energy balance around the container:
The equation is zero equals the total mass flow rate \( \dot{m}_{\text{ges}} \) times the sum of the specific enthalpy difference \( h_S - h_6 \) and the kinetic energy difference \( \frac{w_S^2 - w_6^2}{2} \), plus zero. Another form of the equation is zero equals the total mass flow rate \( \dot{m}_{\text{ges}} \) times the sum of the specific heat at constant pressure \( c_p \) times the temperature difference \( T_5 - T_6 \) and the kinetic energy difference \( \frac{w_S^2 - w_6^2}{2} \). Rearranging gives the kinetic energy difference \( \frac{w_6^2 - w_S^2}{2} \) times the total mass flow rate \( \dot{m}_{\text{ges}} \) equals the total mass flow rate \( \dot{m}_{\text{ges}} \) times the product of the specific heat at constant pressure \( c_p \) and the temperature difference \( T_5 - T_6 \). Solving for \( w_6^2 \) gives it as twice the product of the specific heat at constant pressure \( c_p \) times the temperature difference \( T_5 - T_6 \) minus \( w_S^2 \). Finally, \( w_6 \) is the square root of twice the difference between the product of the specific heat at constant pressure \( c_p \) times the temperature difference \( T_5 - T_6 \) and \( w_S^2 \).