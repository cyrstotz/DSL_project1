A graph is drawn with the x-axis labeled as S in units of u per kg and the y-axis labeled as T in Kelvin. Several curves are drawn, each representing different pressures. Points are labeled as 1, 2, 3, 4, 5, and 6. Point 1 is at the intersection of the lowest pressure curve and the y-axis. Point 2 is on the next higher pressure curve, connected to point 1 by a line labeled "reversible". Point 3 is on the same pressure curve as point 2, connected by an isobaric line. Point 4 is on a higher pressure curve, connected to point 3 by a line labeled "reversible". Point 5 is on the same pressure curve as point 4, connected by an isobaric line. Point 6 is on the lowest pressure curve, connected to point 5 by a line labeled "reversible".

In the table:
- The first column is labeled Z, the second column is labeled D, and the third column is labeled T.
- The entries in the table are as follows:
  - Row 1: Z = 1, D = 0.191 bar, T = -30°C
  - Row 2: Z = 2, D = P2, T is not specified
  - Row 3: Z = 3, D = P2, T is not specified
  - Row 4: Z = 4, D and T are not specified
  - Row 5: Z = 5, D = 0.8 bar, T = 481.8 K
  - Row 6: Z = 6, D = 0.191 bar, T is not specified