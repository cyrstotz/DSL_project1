a) 

i) The description of the graph is as follows: The graph is a Pressure-Temperature (P-T) diagram. The x-axis is labeled "T" for temperature and the y-axis is labeled "P" for pressure. There is a curve that starts from the bottom left and rises to the top right, representing the phase boundary between gas and liquid. The region to the left of the curve is labeled "Flüssig" which means liquid, and the region to the right is labeled "Gas". There are four points labeled 1, 2, 3, and 4, with arrows indicating transitions between these points. Below the graph, the text "Alle Zustände sind im Dreiphasengebiet" is written, which translates to "All states are in the three-phase region."

ii) The description of the second graph is as follows: The graph is another Pressure-Temperature (P-T) diagram. The x-axis is labeled "T" for temperature and the y-axis is labeled "P" for pressure. There is a curve that starts from the bottom left and rises to the top right, representing the phase boundary between gas and liquid. The region to the left of the curve is labeled "Flüssig" which means liquid, and the region to the right is labeled "Gas". There are two points labeled 1 and 2, with an arrow indicating a transition between these points. Below the graph, the text "ΔT=10K" is written, indicating a temperature difference of 10K.