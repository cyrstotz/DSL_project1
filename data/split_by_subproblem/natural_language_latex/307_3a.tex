The temperature \( T_{g,1} \) is 500 degrees Celsius.
The volume \( V_{g,1} \) is 3.14 liters, which is equivalent to 0.00314 cubic meters.

Graphical Description:
There is a rectangular container with a piston inside. The piston is represented by a horizontal line with a shaded area above it. There are three arrows pointing downwards on top of the piston, labeled from left to right as:
- \( V_{pamb} \)
- \( \frac{m_{k} \cdot g}{A} \)
- \( \frac{m_{EW} \cdot g}{A} \)

There is one arrow pointing upwards from below the piston, labeled \( P_{g,1} \).

The pressure \( P_{g,1} \) is calculated as the sum of the ambient pressure \( P_{amb} \), the pressure due to mass \( m_{k} \) and gravity \( g \) over area \( A \), and the pressure due to mass \( m_{EW} \) and gravity \( g \) over area \( A \). The area \( A \) is 0.007854 square meters. The calculation results in a pressure of 1.4 bar.

The relationship between pressure \( P_{g,1} \), volume \( V_{g,1} \), mass \( m_{g} \), gas constant \( R \), and temperature \( T_{g,1} \) is given by the equation \( P_{g,1} V_{g,1} = m_{g} R T_{g,1} \). The gas constant \( R \) is calculated as \( \frac{\bar{R}}{M_{g}} = 0.16628 \, \frac{kJ}{kg \cdot K} \). The mass \( m_{g} \) is then calculated to be 3.42 grams using the given formula.