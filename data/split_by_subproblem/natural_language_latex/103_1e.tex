Section d) Subtask

1. The following equations are given:
   - \( m_1 \) equals 5755 kilograms.
   - \( X_1 \) equals 0.005.
   - \( T_1 \) equals 100 degrees Celsius.
   - \( T_{\text{ein}} \) equals 20 degrees Celsius.
   - \( m_2 \) equals \( m_1 \) plus \( \Delta m \).
   - \( \Delta m \).
   - \( T_2 \) equals 70 degrees Celsius.

   The equation \( m_2 U_2 - m_1 U_1 = \dot{m} h_{\text{ein}} + Q_{\text{aus}} 12 \) is provided.

   At 100 degrees Celsius:
   - \( U_1 \) is calculated as \( U_f (100^\circ \text{C}) + X_b \left( h_g (100^\circ \text{C}) - U_f (100^\circ \text{C}) \right) \), which simplifies to 0.025 times (296.95 minus 292.95), resulting in 418.99 kilojoules per kilogram plus 0.005 times (2506.5 kilojoules per kilogram minus 418.99 kilojoules per kilogram), giving a final value of 429.38 kilojoules per kilogram.

   At 70 degrees Celsius:
   - \( U_2 \) is calculated as 202.95 kilojoules per kilogram plus 0.005 times (2469 kilojoules per kilogram minus 292.95 kilojoules per kilogram), resulting in 303.83 kilojoules per kilogram.

   At 20 degrees Celsius:
   - \( h_{\text{ein}} \) is calculated as \( h_f (20^\circ \text{C}) + X_0 (h_g (20^\circ \text{C}) - h_f (20^\circ \text{C})) \), which simplifies to 79.77 kilojoules per kilogram plus 0.005 times (2536.7 kilojoules per kilogram minus 741.7 kilojoules per kilogram), giving a final value of 87.08 kilojoules per kilogram.

   The equation \( m_1 U_2 - m_1 U_1 + \Delta m U_2 - \Delta m h_2 = Q_{\text{aus}} \) is provided. Solving for \( \Delta m \), it is calculated as \( \frac{Q_{\text{aus}} + (m_1 (U_1 - U_2))}{U_2 - h_e} \), resulting in 3495 kilograms.

2. The equation \( \Delta S_{12} = m_2 S_2 - m_1 S_1 = \Delta m S_e + \frac{QR}{T} + S_{er} \) is provided.