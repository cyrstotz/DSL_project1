Problem d

1. Main equation over ice melting, according to the solution on pages 5/6.

The product of the mass of the melting ice and the change in internal energy from state 1 to state 2 equals the heat transferred in the process, denoted by Q12, under constant volume.

The change in internal energy from state 1 to state 2 is equal to the heat transferred divided by the mass of the melting ice.

The change in internal energy from state 1 to state 2 is the difference between the internal energy at state 2 and state 1.

The pressure at state 2 for the melting ice is the pressure at state 1 for the melting ice minus 7.6 bar.

The internal energy at state 2 under a pressure of 7.6 bar is equal to the internal energy of the solid phase plus x1 times the difference between the internal energy of fusion and the internal energy of the solid phase.

The internal energy at state 2 is equal to the sum of the heat transferred and the product of the internal energy at state 1 and the mass of the melting ice, all divided by the mass of the melting ice. This simplifies to the heat transferred divided by the mass of the melting ice plus the internal energy at state 1.

The internal energy at state 1 at 0 degrees Celsius is given in a table. The pressure on the melting ice, p12, is 7.394 bar and 3.746 bar.

The internal energy at state 2 is the internal energy of the solid phase plus x1 times the difference between the internal energy of fusion and the internal energy of the solid phase. This results in a subtraction of two energy values in kilojoules per kilogram.

x1 is 1 minus the fraction of ice, which is 0.0.

x2 is the fraction of the difference between the internal energy at state 2 and the internal energy of the solid phase over the difference between the internal energy of fusion and the internal energy of the solid phase, which equals 0.429.

The fraction of ice at state 2 is 1 minus 0.429, which equals 0.570.