At negative 22 degrees Celsius, the pressure \( p_2 \) equals 1.2192 bar, which is equal to \( p_1 \), since it is isobaric.

The enthalpy \( h_2 \) at negative 22 degrees Celsius is given by \( h_g \) at negative 22 degrees Celsius minus \( h_n \), which equals 234.08 kilojoules per kilogram.

The first graph is a Pressure versus Temperature diagram. The horizontal axis is labeled Temperature and the vertical axis is labeled Pressure. The graph consists of a closed loop with four points labeled 1, 2, 3, and 4. The loop is a quadrilateral with the following characteristics:
- Point 1 is at the bottom left.
- Point 2 is at the bottom right.
- Point 3 is at the top right.
- Point 4 is at the top left.
The path follows the sequence from point 1 to point 2 to point 3 to point 4 and back to point 1. There is a horizontal line at the top of the graph labeled 8 bar.

The second graph is also a Pressure versus Temperature diagram but with a different scale. The horizontal axis is labeled Temperature in Kelvin and the vertical axis is labeled Pressure in bar. The graph consists of a closed loop with four points labeled 1, 2, 3, and 4. The loop is a quadrilateral with the following characteristics:
- Point 1 is at the bottom left.
- Point 2 is at the bottom right.
- Point 3 is at the top right.
- Point 4 is at the top left.
The path follows the sequence from point 1 to point 2 to point 3 to point 4 and back to point 1. The graph also includes a curved line that starts at the bottom left, rises to a peak near point 4, and then descends to the bottom right.