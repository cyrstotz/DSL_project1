The equation p sub 5 times v sub 5 equals 2 times T sub 5 implies v sub 5 equals the fraction of 2 times T sub 5 over p sub 5, which equals 3.476 cubic meters per kilogram.

The military fraction of T sub 4 over T sub 5 equals the fraction of p sub 4 over p sub 5 raised to the power of (n minus 1) over n implies T sub 4 equals T sub 5 times the fraction of p sub 4 over p sub 5 raised to the power of (n minus 1) over n, which equals 328.07 Kelvin.

The military p sub 6 equals p sub 0.

First law of thermodynamics: U dot equals m dot times the difference of h sub 5 minus h sub 4 implies the square of v sub 5 minus the square of v sub 4 plus kinetic energy sub 6 minus kinetic energy sub 5 equals zero.

c sub v times the difference of T sub 4 minus T sub 5 plus one half times the difference of the square of w sub 2 minus the square of w sub 5 equals zero.

The square of w sub 2 equals the square of w sub 5 minus 2 times c sub v times the difference of T sub 4 minus T sub 5, which equals 50.244 square meters per second squared.