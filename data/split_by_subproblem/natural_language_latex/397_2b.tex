From the energy balance:
Zero equals the mass flow rate times the quantity of exit enthalpy minus inlet enthalpy plus half the difference of the squares of exit and inlet velocities, plus the sum of heat transfer rates minus the sum of work rates equals zero.
The heat transfer rate is mentioned as 'irren co.'.
The work transfer rate is less than or equal to zero.
q equals the difference of inlet enthalpy and exit enthalpy divided by the mass flow rate.
The exit velocity is greater than 200 meters per second.
Half the difference of the squares of exit and inlet velocities equals the difference of inlet and exit enthalpies minus q divided by the mass flow rate, which equals the specific heat at constant pressure times the difference of inlet and exit temperatures minus q divided by the mass flow rate.
From state 5 to state 6 is isentropic, hence the ratio of temperatures at state 6 and state 5 equals the ratio of pressures at state 6 and state 5 raised to the power of (n-1)/n, which equals the ratio of the base pressure to pressure at state 5 raised to the power of 0.4 over 1.4, which equals the ratio of 0.191 to 0.5 raised to the power of 0.4 over 1.4, resulting in 0.76.
The temperature at state 6 equals 0.76 times the temperature at state 5, which equals 0.76 times 423 Kelvin, resulting in 318.7 Kelvin.

Half the difference of the squares of velocities at state 5 and state 6 equals the specific heat at constant pressure times the difference of temperatures at state 6 and state 5, which equals 1.006 kilojoules per kilogram Kelvin times the difference of 328.07 Kelvin and 431.91 Kelvin, resulting in -104.45 kilojoules per kilogram.
The velocity at state 6 equals the square root of twice the product of -104.45 kilojoules per kilogram plus the square of velocity at state 5, resulting in 510 meters per second.
The temperature at state 6 is 340 Kelvin, and the velocity at state 6 is 510 meters per second.
The change in exergy due to isentropic processes equals the enthalpy at state 6 minus the base enthalpy minus the base temperature times the difference of entropies at state 6 and base, plus half the difference of the squares of velocities at state 6 and base, which equals the specific heat at constant pressure times the difference of temperatures at state 6 and base minus the base temperature times the specific heat at constant pressure times the natural logarithm of the ratio of temperatures at state 6 and base minus the gas constant times the natural logarithm of the ratio of pressures at state 6 and base plus half the difference of the squares of velocities at state 6 and base.
Phi equals the ratio of the difference of specific heats at constant pressure and constant volume to the specific heat at constant volume, which equals the specific heat at constant pressure divided by the temperature, resulting in 0.287.
The base temperature equals 273.15 Kelvin minus 30, resulting in 243.15 Kelvin.
The change in exergy due to isentropic processes equals 1.006 times the difference of 340 Kelvin and 243.15 Kelvin plus 243.15 times 1.006 times the natural logarithm of the ratio of 340 Kelvin to 243.15 Kelvin minus 0.287 times the natural logarithm of 1 plus half the difference of the squares of 200 meters per second and 510 meters per second, resulting in -117.8 kilojoules per kilogram.
The change in exergy due to reversible processes equals 100 kilojoules per kilogram.
The exergy due to reversible processes equals the base temperature times the reversible entropy equals the base temperature times the specific heat times the natural logarithm of the difference of entropies at state 1 and base minus the heat transfer rate divided by the temperature.
The mass flow rate equals the exergy due to reversible processes divided by the base temperature times the difference of entropies at state 1 and base, which equals the exergy due to reversible processes divided by the base temperature times the specific heat times the natural logarithm of the ratio of temperatures at state 6 and base, resulting in 1.27 kilograms per second.