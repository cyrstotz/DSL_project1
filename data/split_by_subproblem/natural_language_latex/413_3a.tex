Given: \( p_0 \), \( m \), \( g \)

The equation \( pV = mRT \)

Where:
\( T = 500^\circ C \), \( V = 3.14 \times 10^{-3} \) cubic meters

\( R = \frac{33.14}{50} \) Joules per kilogram Kelvin

\( p = \) (not specified)

The pressure \( p_n \) is given by \( p_0 + \frac{mg}{A} + \frac{0.1 mg}{A} \) (in the case of high ice water)

The area \( A \) is \( \pi \times (6.05 \times 10^{-2})^2 \) square meters

This implies \( p_n = 1.15 \) bar

Thus, the mass \( m \) is calculated as \( \frac{pV}{RT} = 0.003 \) kilograms, which equals \( \boxed{0.3 \) grams}