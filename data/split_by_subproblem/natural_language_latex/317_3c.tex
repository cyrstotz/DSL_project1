1. From the first law of thermodynamics, we have the change in work, denoted as Delta W, equals Q_{1,2} minus W_{1,2}.

2. Delta W is equal to the mass of gas, denoted as m_{g}, times the specific heat at constant volume, c_v, times the change in temperature, Delta T.

3. W_{1,2} equals W_{1,2} new (since it is frictionless),
   which equals the integral from V_1 to V_2 of p dV,
   which equals p_{3,1} times (V_2 minus V_1),
   which equals 1.9 bar times (V_2 minus 3.29 liters),
   which equals negative 289.2 Joules.

4. V_2 is calculated as the mass of gas, m_{g}, times the gas constant, R, times the temperature T_2, all divided by the pressure p_{3,2}, which equals 0.001109 cubic meters.

5. Delta W is again calculated as the mass of gas, m_{g}, times the specific heat at constant volume, c_v, times the change in temperature, Delta T,
   which equals 3.92 grams times 0.673 Joules per gram Kelvin times 0,
   which equals 0.