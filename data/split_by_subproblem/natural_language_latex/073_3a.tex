a) The pressure \( P_{GA} \) is equal to the ambient pressure \( P_{amb} \) plus the term \( \frac{m_K \cdot g}{A} \) plus the term \( \frac{m_{EW} \cdot g}{A} \) plus the pressure \( P_{EW} \).

The pressure \( P_{EW} \) at the temperature \( T_{EW} \) equals zero degrees Celsius is 1.1 bar.

The area \( A \) is calculated as \( \pi \) times the square of half the diameter \( d \), which is \( \pi \left( \frac{0.1 \, \text{m}}{2} \right)^2 \) resulting in \( 0.00785 \, \text{m}^2 \).

Substituting the values, the pressure \( P_{GA} \) is calculated as 1 bar plus \( \frac{32 \, \text{kg} \cdot 9.18 \, \text{m/s}^2}{0.00785 \, \text{m}^2} \) plus \( \frac{0.1 \, \text{kg} \cdot 9.81 \, \text{m/s}^2}{0.00785 \, \text{m}^2} \) plus 1.1 bar, which equals 2.8 bar.

1G The equation \( P_{GA} V_{GA} = m_{GA} R T_{GA} \) relates the pressure, volume, mass, gas constant, and temperature of the gas.

The gas constant \( R \) is calculated as \( \frac{\bar{R}}{M_g} \), where \( \bar{R} \) is 8.314 J/(mol K) and \( M_g \) is 0.05 kg/mol, resulting in \( R = 166.28 \, \frac{\text{J}}{\text{kg K}} \).

The mass \( m_{GA} \) of the gas is calculated using the formula \( m_{GA} = \frac{P_{GA} V_{GA}}{R T_{GA}} \).

Substituting the values, \( m_{GA} \) is \( \frac{2.8 \cdot 10^5 \, \text{Pa} \cdot 3.14 \cdot 10^{-3} \, \text{m}^3}{166.28 \, \frac{\text{J}}{\text{kg K}} \cdot 273.15 \, \text{K}} \), which equals 0.006833 kg or 6.84 grams.