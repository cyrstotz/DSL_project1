c) The total mass flow rate, denoted as m-dot subscript ges.

The total mass flow rate times the fraction of specific heat at constant pressure times the temperature difference between T subscript z and T subscript 1, all over the difference of half the square of W subscript 6 and half the square of W subscript 1, equals 243.15 Kelvin.

Zero equals negative change in kinetic energy due to stream, equals m-dot times the difference in enthalpy between e and a minus T subscript 0 times the difference in entropy between e and a plus ke.

The difference in entropy between e and a equals specific heat at constant pressure times the natural logarithm of the ratio of T subscript a to the power of e over T subscript a minus the gas constant times the natural logarithm of the ratio of p subscript e over p subscript a, equals negative 0.301 kilojoules per kilogram Kelvin.

The total mass flow rate tends to zero equals m-dot times the bracket of enthalpy at g minus enthalpy at 6 plus half the difference of the square of W subscript 5 and the square of W subscript 6.

The difference in enthalpy between g and 6 equals specific heat at constant pressure times the difference in temperature between T subscript 5 and T subscript a.

Zero equals m-dot times the difference in enthalpy between 0 and 5 plus q subscript 8 times the total mass flow rate times the fraction of K over m plus the total mass flow rate times half the difference of the square of W subscript 0 and the square of W subscript 5.

Specific heat at constant pressure times the difference in temperature between T subscript 0 and T subscript 5 equals negative 181.8825 minus 4200.

Referring to the total mass flow rate.

The total mass flow rate equals m-dot subscript M equals m-dot subscript K.

The total mass flow rate equals 5.293 times the ratio of m-dot subscript K over m subscript K.

The total mass flow rate equals 6.293 times m-dot subscript K.

The total mass flow rate equals 6.293.