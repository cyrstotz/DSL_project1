The rate of change of \( e_{x, \nu} \) is equal to the rate of change of \( e_{x, s12} \) plus the product of \( \frac{T_0}{T_3} \) and \( \dot{q} \).

The rate of change of \( s_{ex} \) is equal to \( s_0 \) minus \( s_1 \) plus the fraction of \( \dot{q} \) over \( T_3 \), which equals \( c_p \) times the natural logarithm of the fraction \( \frac{T_0}{T_0} \) minus \( R \) times the natural logarithm of the fraction \( \frac{p_0}{p_0} \) plus the fraction of \( \dot{q} \) over \( T_3 \). This simplifies to \( 1.006 \) kilojoules per kilogram Kelvin times the natural logarithm of the fraction \( \frac{243.15 K}{328.07 K} \) plus the fraction \( \frac{1.195 \) kilojoules per kilogram}{128 seconds Kelvin} \), which equals \( 0.6257 \) kilojoules per kilogram Kelvin.

The rate of change of \( E_{x, \nu} \) is equal to the product of \( \dot{s}_{ex} \) and \( T_0 \).

The rate of change of \( e_{x, \nu} \) is equal to the product of \( \dot{s}_{ex} \) and \( T_0 \), which equals \( 152.14 \) kilojoules per kilogram.