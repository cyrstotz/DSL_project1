Part 1:

The ambient pressure \( P_{\text{amb}} \) equals the atmospheric pressure \( P_{\text{atm}} \), which is 100,000 Pascals.

The cabin pressure \( P_{\text{kab}} \) is calculated as the force \( F_s \) divided by the area \( A \), which is further calculated as mass \( m \) times gravity \( g \) divided by \( \pi \) times the square of half the diameter \( D \). This results in 39,969.154 Pascals, which equals 0.3997 bar.

The evaporation pressure \( P_{\text{es}} \) is calculated as the evaporation force \( F_{\text{ev}} \) divided by the area \( A \), which is the mass of evaporation \( m_{\text{ev}} \) times gravity \( g \) divided by \( \pi \) times the square of half the diameter \( D \). This results in 0.00078 Pascals.

The total pressure \( P_{g1} \) is the sum of the ambient pressure \( P_{\text{amb}} \), cabin pressure \( P_{\text{kab}} \), and evaporation pressure \( P_{\text{ev}} \), totaling 139,969.5408 Pascals, which equals 1.399695 bar.

The molecular weight of gas \( mg \) equals the gas constant \( R_g \), which is calculated as \( R \) divided by the molar mass \( M_s \), resulting in 206.628 Joules per kilogram Kelvin.

The temperature \( T_{g1} \) is 500 degrees Celsius, which is equivalent to 773.15 Kelvin.

The volume \( V_{g1} \) is 3.74 liters, which is 0.00374 cubic meters.

Thus, the mass \( mg \) is calculated as the pressure \( P_{g1} \) times the volume \( V_{g1} \) divided by the gas constant \( R_g \) times the temperature \( T_{g1} \), resulting in 0.08412 kilograms, which is 84.12 grams.

Temperature \( T_{37} \) is 0.003 degrees Celsius, which is 273.153 Kelvin.

Graphical Content: There is a small, roughly circular scribble below the equation. The scribble consists of several overlapping loops, drawn in a hurried manner.