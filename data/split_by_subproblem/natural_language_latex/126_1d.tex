d) First main equation.

The change in energy, Delta E, is equal to the change in mass between state 1 and 2 times h, plus the heat transfer rate, Q dot.

m2 times u2 minus m1 times u1 equals the change in mass between state 1 and 2 times h plus the heat transfer rate, Q dot.

The change in mass between state 1 and 2 is equal to one over h times the quantity m2 times u2 minus m1 times u1 minus Q dot.

h equals the enthalpy at 20 degrees Celsius, which is crossed out 88 and replaced by 83.96 kilojoules per kilogram, with a note A2 and illegible text.

The heat transfer rate, Q dot, is negative 35 megajoules.

Calculating m2 times u2 minus m1 times u1:

The temperature T1 is 100 degrees Celsius and T2 is 70 degrees Celsius.

m1 equals 5755 kilograms.

m2 equals m1 plus Delta m.

u2 equals the internal energy at T2, which is 232.15 kilojoules per kilogram, noted as A2.

u1 equals the internal energy at T1, which is 418.94 kilojoules per kilogram, noted as A2.

Therefore, the change in mass between state 1 and 2 is equal to one over h times m2 times u2 plus one over h times Delta m times u2 minus m1 times u1 times one over h minus Q dot times one over h.

Delta m times the quantity one minus u2 over h equals one over h times the quantity m2 times u2 minus m1 times u1 minus Q dot over h.

Delta m equals one over 83.96 times the quantity 5755 times 232.15 minus 5755 times 418.94 minus negative 35 times 10 to the power of 3 divided by 83.96, which equals crossed out 3888 and replaced by 3965.9 kilograms.