The mass flow rate of R134a is denoted by m-dot subscript R134a.

The initial temperature, T subscript i, is 10 degrees Celsius.
The temperature T subscript Lambda is 4 degrees Celsius.

There is a diagram included, which is a circular representation with three points labeled 1, 2, and 3. Point 1 is at the top of the circle, point 2 is on the left side, and point 3 is on the right side. An arrow pointing from point 1 to point 2 is labeled with W-dot subscript K. Another arrow points from point 2 to point 3.

The term "Stat. Fließg." is mentioned, which translates to "Stationary flow."

The equation provided is zero equals m-dot times (h subscript 2 minus h subscript 3) plus W-dot subscript K.

It is stated that the process is adiabatically reversible, which is indicated by s subscript 2 equals s subscript 3.

Reference to Table A-11 is made.

The enthalpy h subscript 4 equals h subscript ePhi at 8 bar, which is 93.42 kilojoules per kilogram, and this is equal to h subscript 1.

The pressure p subscript 2 equals p subscript 1.
The pressures p subscript 3 and p subscript 4 are both 8 bar.
The enthalpy h subscript 3 equals h subscript 4, indicating a throttling process.

The mass flow rate m-dot is calculated as W-dot subscript K divided by (h subscript 2 minus h subscript 3).