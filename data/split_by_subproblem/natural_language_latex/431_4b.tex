- The second graph is a plot with the y-axis labeled as \( p(T) \) and the x-axis labeled as \( T(K) \). The graph displays a straight line that begins at the origin and ascends to the right. This line is labeled as "gaseous". There are two points marked on the line:
    - Point 1 is located at the bottom left.
    - Point 2 is located at the top right.
  - There is an arrow pointing from point 1 to point 2 labeled "Step ii".

- The third graph is a plot with the y-axis labeled as \( p(T) \) and the x-axis labeled as \( T(K) \). The graph depicts a curve that starts at the bottom left, ascends to a peak, and then descends. The regions under the curve are labeled as follows:
    - The region to the left of the peak is labeled "1".
    - The region under the peak is labeled "2".
    - The region to the right of the peak is labeled "3".
  - There is an arrow pointing from the left side of the peak to the right side labeled "Step i".

- The equation \( 0 = \dot{m}_2 (h_2 - h_3) + \dot{Q} - \dot{w} \).
- The equation \( h_2 = \frac{s_2 - s_3}{s_2 - s_4} h_4 (27.7.15 K) = \frac{2.95.55 \frac{kJ}{kg}}{0.916 \frac{kJ}{kg \cdot K}} \).
- The temperature \( T_i \) is \( 10^\circ C \) which equals \( 283.15 K \).
- The pressure \( p_{\text{1bar}} \) is constant.
- The pressure \( p_1 = 1 \text{mbar} = p_2 \).
- The temperature \( T_2 = 283.15 K - 6 K = 277.15 K \).
- The equation \( h_1 = \frac{s_2 - s_3}{s_2 - s_4} (h_3 - h_2) + h_3 \).