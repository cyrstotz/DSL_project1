Exergy of a flow:

The exergy of the flow is given by the equation:
dot E subscript x, st equals dot m times the quantity h minus h subscript 0 minus T subscript 0 times the quantity s minus s subscript 0 plus one-half ke plus one-half pe.

State 0:
The exergy at state 0 is given by:
dot E subscript x,0 equals dot m times g times the quantity h subscript 0 minus h subscript 0 minus T subscript 0 times the quantity s subscript 0 minus s subscript 0 plus ke,
which simplifies to dot m times g times ke equals dot m times g times e subscript 0.
The exergy e subscript x,0 equals one-half times w subscript 0 squared equals 20000 meters squared per second squared.

State 6:
The exergy at state 6 is given by:
dot E subscript x,6 equals dot m times the quantity h subscript 6 minus h subscript 0 minus T subscript 0 times the quantity s subscript 6 minus s subscript 0 plus one-half times w subscript 6 squared.
The change in enthalpy h subscript 6 minus h subscript 0 equals c subscript p times the quantity T subscript 6 minus T subscript 0 equals 97,431 kilojoules per kilogram.
The change in entropy s subscript 6 minus s subscript 0 equals c subscript p times the natural logarithm of the ratio T subscript 6 over T subscript 0 minus R times the natural logarithm of the ratio p subscript 6 over p subscript 0.
The specific gas constant R subscript L equals R over M equals R over 28.97 kilojoules per kilomole Kelvin equals 0.287 kilojoules per kilogram Kelvin.
The exergy e subscript x,6 equals 97,431 kilojoules per kilogram minus R times the ratio p subscript 6 over p subscript 0.
The change in entropy s subscript 6 minus s subscript 0 equals the integral from T subscript 0 to T subscript 6 of c subscript p over T dT equals c subscript p times the quantity natural logarithm of T subscript 6 minus natural logarithm of T subscript 0.
The change in entropy s subscript 6 minus s subscript 0 equals 0.337 kilojoules per kilogram Kelvin.
The exergy e subscript x,6 equals 97,431 kilojoules per kilogram minus 243.15 Kelvin times 0.337 kilojoules per kilogram Kelvin plus one-half times 510 meters squared per second squared,
which equals 130065.5 kilojoules per kilogram.

Problem 2:

The change in exergy Delta e subscript x equals e subscript x6 minus e subscript x0 equals 110065.5 kilojoules per kilogram.