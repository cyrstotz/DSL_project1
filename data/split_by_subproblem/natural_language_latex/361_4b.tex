The initial temperature, denoted as T_i, is negative 70 degrees Celsius.
The temperature of the evaporator, denoted as Tverdampfer, is negative 16 degrees Celsius.

Let z equal 3.
The equation zero equals the mass flow rate (denoted as m dot) times the difference between the enthalpy at the exit (h_e) and the enthalpy at the entrance (h_a) minus the power of the compressor (denoted as W_k dot).

The enthalpy at the exit, h_e, at negative 16 degrees Celsius is calculated as follows:
h_e equals the enthalpy of gas at negative 16 degrees Celsius, which is 237.97 minus 236.0, resulting in 237.74 kilojoules per kilogram.

The enthalpy at the entrance, h_a, for 8 bar and quality x equals 0.92, is calculated using the formula:
h_a equals (275.96 minus 264.75) divided by (0.9284 minus 0.8066) plus 264.75, resulting in 274.57 kilojoules per kilogram.

The mass flow rate, denoted as m dot, is calculated as:
m dot equals the power of the compressor divided by the difference between h_e and h_a, which is 8.34 divided by (237.74 minus 274.57), resulting in 8.34 times 10 to the power of negative 4 kilograms per second.