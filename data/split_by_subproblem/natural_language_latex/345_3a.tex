The pressure \( p_{g1} \) equals the mass \( m_g \) divided by the area \( A \) times the initial pressure \( p_0 \). The area \( A \) is given by the square of three halves of \( r \) times pi, which equals \( 7.85 \times 10^{-3} \) square meters.

The mass \( m \) is equal to \( n \) times \( k \) plus the mass of the Earth's weight \( m_{EW} \), which totals \( 37.1 \) kilograms.

The pressure \( p_{g1} \) is \( 140.1 \) kilopascals, which is equivalent to \( 1.40 \) bar.

The constant \( R \) is calculated as the fourth power of \( s_{31} \) divided by \( M_g \), resulting in \( 0.16625 \) kilojoules per kilogram Kelvin.

The product of \( p_{g1} \) and \( V_{g1} \) equals the product of \( m_g \), \( R \), and \( T_{g1} \). From this, it follows that \( m_g \) equals \( 3.9 \times 10^{-3} \) kilograms.