The mass flow rate for input, denoted as m-dot-subscript-ein, equals the mass flow rate for output, denoted as m-dot-subscript-aus, and both are equal to 0.3 kilograms per second.

The process is a steady flow process.

The sum of heat transfer rates, denoted as Sigma Q-dot, equals Q-dot-subscript-e (heat transfer rate entering) minus Q-dot-subscript-aus (heat transfer rate exiting).

The heat transfer rate exiting, denoted as Q-dot-subscript-aus, equals the mass flow rate, m-dot, times the difference in enthalpy from input to output, h-subscript-ein minus h-subscript-aus, plus the heat transfer rate entering, Q-dot-subscript-e.

(Note: Q-dot-subscript-aus is defined as negative!)

The enthalpy at input, denoted as h-subscript-ein, equals the enthalpy at 70 degrees Celsius, h-subscript-g, which is 2668 kilojoules per kilogram.

The enthalpy at output, denoted as h-subscript-aus, equals the enthalpy at 400 degrees Celsius, h-subscript-g, which is 2764.1 kilojoules per kilogram (refer to Table A-2).

The heat transfer rate exiting, Q-dot-subscript-aus, is 85.24 kilowatts.