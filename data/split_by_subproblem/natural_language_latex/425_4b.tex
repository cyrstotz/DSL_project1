b)

The initial temperature \( T_i \) is minus 10 degrees Celsius.

Energy balance at the compressor:

The equation is zero equals the mass flow rate \( \dot{m} \) times the difference between \( h_2 \) and \( h_3 \) minus the compressor work \( W_k \).

The mass flow rate \( \dot{m} \) is equal to the compressor work \( W_k \) divided by the difference between \( h_2 \) and \( h_3 \).

The enthalpy \( h_2 \) at temperature \( T \) equal to \( T_i \) minus 6 Kelvin is 237.7 kilojoules per kilogram, as per reference (A-10).

The enthalpy \( h_3 \) is unknown.

The entropy \( s_3 \) equals \( s_2 \) which is equal to \( s_9 \) at temperature \( T \) equal to \( T_i \) minus 6 Kelvin, and it is 0.928 kilojoules per kilogram Kelvin, as per reference (A-10).

The enthalpy \( h_3 \) is calculated as \( h \) at entropy \( s = s_3 \) and pressure \( p = 8 \) bar, which is equal to \( h \) at saturation temperature \( T_{\text{sat}} \) and 8 bar minus \( h \) at 40 degrees Celsius and 8 bar minus \( h \) at saturation temperature \( T_{\text{sat}} \) and 8 bar minus \( s \) at 40 degrees Celsius and 8 bar minus \( s \) at saturation temperature \( T_{\text{sat}} \) and 8 bar times \( 3 - 3 \) at saturation temperature \( T_{\text{sat}} \) and 8 bar.

This results in 277.31 kilojoules per kilogram.

The mass flow rate \( \dot{m} \) is calculated as 0.28 times 10 to the power of 7 kilowatts divided by the difference between \( h_2 \) and \( h_3 \), resulting in 0.00834 kilograms per second.