0.5 equals m subscript 2 times s subscript 2 minus m subscript 1 times s subscript 1.

0.5 equals m subscript 2 times s subscript 2 minus m subscript 1 times s subscript 1 equals 717.3 kilojoules per kilogram Kelvin.

m subscript 2 equals m subscript ges plus m dot equals 330 kilograms plus 5755 kilograms equals 9085.0 kilograms.

m subscript 1 equals 5755 kilograms.

s subscript 2 equals s subscript f at 70 degrees equals 0.5958.

s subscript 1 equals s subscript f at 100 degrees equals 1.3069.

Graphical Description:
The diagram in the top right corner of the page is a rectangular box divided into two horizontal sections. The top section is labeled with T and an arrow pointing upwards labeled Q. The bottom section is labeled with Q and 1000.