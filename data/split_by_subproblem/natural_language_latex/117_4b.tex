The mass flow rate denoted by m-dot subscript R and the work rate denoted by W-dot subscript k equals 28 Watts.

There is a static freezing point at the capacitor with k subscript e equals p subscript e equals zero and Q-dot is adiabatic.

The equation zero equals m-dot subscript k times the expression (canceled h subscript 2 minus h subscript 3 plus the fraction (w subscript 2 squared minus w subscript 3 squared) over 2 plus g times the fraction (z subscript 2 minus z subscript 3) over 2) plus Q-dot plus W-dot subscript k.

This implies that W-dot subscript k equals m-dot subscript k times (h subscript 2 minus h subscript 3).

This further implies that m-dot subscript k equals W-dot subscript k divided by (h subscript 2 minus h subscript 3).

The entropy s subscript 2 equals s subscript 3, indicating an isentropic process.

There is unreadable text.

The quality x subscript A equals zero, the enthalpy h subscript A equals h subscript f at 8 bar equals 93.42 kilojoules per kilogram, as referenced from Table A-7.

Temperature T subscript i is questioned. T subscript i equals T subscript sublimation plus 70 Kelvin equals 70 Kelvin.

This implies that i subscript 2 equals T subscript i minus 6 Kelvin equals 4 Kelvin.

The enthalpy h subscript 2 equals h subscript g at 4 Kelvin equals h subscript gt.

Temperature T subscript i is questioned again. T subscript i equals T subscript sublimation plus 70 Kelvin equals 0 degrees Celsius plus 70 Kelvin equals 70 degrees Celsius.

This implies that T subscript 2 equals 70 degrees Celsius minus 30 degrees Celsius equals 40 degrees Celsius.

The enthalpy h subscript 2 equals h subscript g at 40 degrees Celsius equals 194.19 kilojoules per kilogram, as referenced from Table A-10.

The entropy s subscript 3 equals s subscript 2, thus s subscript 2 equals s subscript g at 40 degrees Celsius equals 0.9769 kilojoules per kilogram Kelvin.

The quality x subscript 3 equals the fraction (s subscript 3 minus s subscript f at 8 bar) divided by (s subscript g at 8 bar minus s subscript f at 8 bar) equals the fraction (0.9769 minus 0.3459) divided by (0.9066 minus 0.3459).