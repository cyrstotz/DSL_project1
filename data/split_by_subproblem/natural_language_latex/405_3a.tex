- \( p_{g1} \) and \( m_g \) are sought.
- \( T_{g1} \) equals 773.15 Kelvin and \( V_{g1} \) equals 0.00314 cubic meters.
- The equation \( pV = mRT \) is given.
- \( m_2 \) equals 32 kilograms and \( m_{EW} \) equals 0.1 kilograms.
- \( R \) is calculated as \( \frac{1}{M} \times \bar{R} \) which equals \( \frac{166.28 \, \text{Joules}}{\text{kg} \cdot \text{Kelvin}} \).
- \( p_{g1} \) is calculated as \( \frac{m_{g1} \times R \times T_{g1}}{V_{g1}} \).
- \( p_{g1} \) is recalculated including \( m_{EW} \) as \( \frac{(m_{g1} + m_{EW}) \times R \times T_{g1}}{V_{g1}} \).
- \( p_{g1} \) equals 1000 times 4.732 Pascals.
- \( m_g \) is calculated as \( \frac{R \times T_{g1}}{p_{g1} \times V} \).
- \( p_{g1} \) is calculated as \( p_{amb} + \frac{3}{A} \times (m_{K} + m_{EW}) \).
- \( p_{g1} \) equals 1400 times 4.8573 Pascals.