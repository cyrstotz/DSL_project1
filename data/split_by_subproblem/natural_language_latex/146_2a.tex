The section describes a sequence of transitions between points labeled from 0 to 8. The transitions include adiabatic processes (where no heat is exchanged) and isobaric processes (where the pressure remains constant). Specifically, the sequence is as follows:

- Starting from point 0, moving to point 1 through an adiabatic process where the ratio R is less than 1.
- Moving from point 1 to point 2, then to point 3 through an isobaric process.
- Continuing from point 3 to point 4, then moving to point 5 through an adiabatic process where the ratio R is less than 1.
- Moving from point 5 to point 6, then to point 7 through an isobaric process.
- Finally, moving from point 7 to point 8 through another isobaric process.

The section titled "Kernstrom" describes a graph which is a Temperature-Entropy (T-s) diagram. The x-axis represents entropy with units Joules per kilogram Kelvin, and the y-axis represents temperature in Kelvin. The graph includes several points and lines:

- Point 1 is located at the bottom left.
- Point 2 is directly above point 1 and is connected by a vertical line labeled "adiabat".
- Point 3 is to the right of point 2 and is connected by a horizontal line labeled "isobar".
- Point 4 is below point 3 and is connected by another vertical line labeled "adiabat".
- Point 5 is to the left of point 4 and is connected by a horizontal line labeled "isobar".
- Point 6 is below point 5 and is connected by a vertical line labeled "adiabat".
- Point 7 is to the right of point 6 and is connected by a horizontal line labeled "isobar".
- Point 8 is below point 7 and is connected by a vertical line labeled "adiabat".
- The diagram also includes several isothermal and isobaric lines parallel to the x-axis and y-axis respectively.
- The region enclosed by points 1, 2, 3, and 4 is shaded.
- The pressures at points 1 and 3 are indicated as 0.5 bar, and at points 6 and 0 as 0.15 bar.