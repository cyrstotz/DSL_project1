A graph is drawn with the x-axis labeled as S in units of t1 per kg and the y-axis labeled as T in units of E. The graph contains a curve with five marked points labeled from 0 to 5. The curve starts at point 0, moves vertically up to point 1, then follows a curved path to point 2, continues to point 3, then to point 4, and finally to point 5. The segment from point 0 to point 1 is labeled as "isentropic". The segment from point 1 to point 2 is labeled as "isobaric". The segment from point 2 to point 3 is labeled as "isobaric p2 equals p3". The segment from point 3 to point 4 is labeled as "isobaric p3 equals p2". The segment from point 4 to point 5 is labeled as "isobaric p0 equals p0". There is a small inset graph on the right side with axes labeled T and S. It shows a simplified version of the main graph with points labeled 0, 1, and 2.

State Table:

The table has columns labeled State, p in bar, T in Kelvin, m dot in kg per second, s, h, and w in meters per second. The rows are filled as follows:
- State 1: s1 equals s0.
- State 2: p2 equals p3.
- State 3: p3 equals p2.
- State 4: no values provided.
- State 5: p equals 0.5 bar, T equals 431.9 Kelvin, w equals 220 meters per second.
- State 6: no values provided.
- State 0: p equals 0.101 bar, T equals 243.15 Kelvin, s0 equals s1, w equals 200 meters per second.