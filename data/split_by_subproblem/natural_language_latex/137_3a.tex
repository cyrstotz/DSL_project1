The pressure \( p_{gz} \) equals \( p_{w} \), which equals \( p_{1} \), which equals \( p_{2} \), and is calculated as \( \frac{m_{k} \cdot g}{A} + p_{L} \). Substituting the values, it becomes \( \frac{0.38397 \, \text{kg} \cdot 9.81 \, \text{m/s}^2}{0.00485 \, \text{m}^2} + 1 \, \text{bar} \), which simplifies to \( 1.3837 \, \text{bar} \).

The equation \( pV = mRT_{1} \) is given.

The mass of gas \( m_{g} \) is calculated using the formula \( \frac{p_{1}V}{RT_{1}} \) (substitute the values). Substituting the values, \( m_{g} \) becomes \( \frac{1.3837 \, \text{bar} \cdot 0.006349 \, \text{m}^3}{0.08314 \, \text{bar} \cdot \text{m}^3/\text{K} \cdot 273.15 \, \text{K}} \), which equals \( 3.92 \, \text{g} \).