a)

P subscript g1, M subscript g

A equals pi times r squared equals 0.031415 square meters

P equals F times g

P subscript g1 equals A times ((0.1 kilograms plus 32 kilograms) times g) plus 1 bar

equals 0.031415 square meters times (32 kilograms) times 9.81 meters per second squared plus 100,000 Pascals

equals 100003.893 Pascals

M subscript g equals 50 kilograms per kilomole

V equals 3.14 Liters equals 0.00314 cubic meters

T equals 500 degrees Celsius equals 773.15 Kelvin

P subscript g1 times V subscript g1 equals M times (R over M) times T

P subscript g1 times V subscript g1 times M subscript g equals M times (R over M) times T

M equals (100003.893 Pascals times 0.00314 cubic meters times 50 kilograms per kilomole) divided by (8.314 Joules per kilomole Kelvin times 773.15 Kelvin) equals 2.44 grams