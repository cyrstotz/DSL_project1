The change in exergy, denoted as delta e_x1, str, is given by the formula:
h_0 minus h_6 minus T_0 times (s_6 minus s_0) plus delta ke. This can be rewritten as:
(h_0 minus h_6 minus T_c times (s_6^0 minus s_0)) plus the difference of w_2 squared minus w_6 squared all over 2.

Given from problem 2:
h_0 minus h_6 equals c_p times (T_0 minus T_6), which equals 1.006 kilojoules per kilogram-Kelvin times (243.15 Kelvin minus 340 Kelvin).

The temperature T_c is calculated as -30 plus 273.15, which equals 243.15 Kelvin.

Approximately, h_0 minus h_6 equals -97.43 kilojoules per kilogram.

s_0 minus s_6 equals zero, since the process is adiabatic and reversible.

Therefore, delta e_x1, str equals (h_0 minus h_6 plus the difference of w_2 squared minus w_6 squared all over 2).
This equals (-97.43 kilojoules per kilogram plus the difference of 200 meters squared per second squared minus 5 meters squared per second squared all over 2).
This simplifies to (-97.43 kilojoules per kilogram plus the difference of 200 squared meters squared per second squared minus 5 times 10 squared meters squared per second squared all over 2).

Approximately, this results in +207.48 kilojoules per kilogram.