The equation m times the difference between U subscript 2a and U subscript 1a plus Q subscript 12 equals zero.

A table with two columns labeled 'Gas' and 'EW' is present, but no entries are shown in the table.

The equation m subscript g times c subscript v times the difference between T subscript a2 and T subscript a1 is not equal to Q subscript 12.

c subscript v equals 0.633 kilojoules per kilogram Kelvin.

The equation m subscript g times c subscript v times delta T equals Q subscript 12.

0.0277 kilograms times 0.633 times the inverse of kilogram Kelvin times negative 499.997 Kelvin equals Q subscript 12.

Negative 7.21 kilojoules equals Q subscript 12.

This implies 7.21 kilojoules.

Heat from gas to EW.

The symbol d crossed out, followed by AGW.

This implies double crossed out d, U subscript zg minus U subscript zew equals zero.

M subscript g times U subscript zg minus M subscript EW times U subscript zew equals zero.

The label 'EW'.

The symbol A followed by dot Q subscript zu times the natural logarithm of the ratio of U subscript 2 minus U subscript 1 to U subscript 1 equals Q subscript zu.