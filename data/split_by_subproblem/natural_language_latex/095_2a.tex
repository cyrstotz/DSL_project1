The section includes several mathematical and descriptive elements related to thermodynamics:

1. The symbol \( T \, [K] \) represents temperature in Kelvin.
2. The value \( T_0 = 243.15 \) specifies an initial temperature of 243.15 Kelvin.
3. The symbol \( s \, \left[ \frac{kJ}{kg \cdot K} \right] \) represents entropy with units kilojoules per kilogram Kelvin.

The description provided is for a Temperature-Entropy (T-s) diagram:
- The y-axis is labeled as temperature in Kelvin.
- The x-axis is labeled as entropy in kilojoules per kilogram Kelvin.
- The diagram includes six points labeled from 0 to 6, connected by lines representing different thermodynamic processes:
  - From Point 0 to Point 1: A vertical line labeled as "adiabat" and "isentrop".
  - From Point 1 to Point 2: Another vertical line labeled as "adiabat" and "isentrop".
  - From Point 2 to Point 3: A horizontal line labeled "isobar".
  - From Point 3 to Point 4: A curved line labeled "adiabat irreversibel" (irreversible adiabat).
  - From Point 4 to Point 5: A horizontal line labeled "isobar".
  - From Point 5 to Point 6: A vertical line labeled "adiabat, reversibel" (reversible adiabat) and "isentrop".
- The points have specific annotations:
  - Point 0: Temperature is 243.15 Kelvin.
  - Point 2: Temperature is 1753 Kelvin.
  - Point 5: Pressure is 6.5 bar.
  - Point 6: Pressure is 0.19 bar.

Additionally, a table is provided with columns labeled P (pressure), T (temperature), and Notes:
- Row for Point 0: Pressure is 0.6591, Temperature is 243.15 Kelvin.
- Row for Point 1: Labeled as "isotherm", Temperature is 1753 Kelvin, noted as "adiabat, reversibel".
- Row for Point 2: Noted as "isentrop".
- Row for Point 3: Noted as "isobar".
- Row for Point 4: Noted as "adiabat, irreversibel".
- Row for Point 5: Pressure is 6.5 bar, Temperature is 3 Kelvin above some base (denoted as (B) 3K), noted as "isobar".
- Row for Point 6: Pressure is 0.19 bar, noted as "adiabat, reversibel".