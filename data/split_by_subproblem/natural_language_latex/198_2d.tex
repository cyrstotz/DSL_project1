The variable \( e_{\text{verl}} \) is crossed out and equals itself.

In another section, \( e_{\text{verl}} \) is equal to \( T_0 \) multiplied by \( s_{\text{verl}} \).
\( s_{\text{verl}} \) is equal to \( \dot{s}_{\text{verl}} \).
\( \dot{s}_{\text{verl}} \) is equal to \( \dot{m} \) times the difference \( (s_0 - s_6) \) plus \( \frac{\dot{Q}}{T_0} \).
\( s_{\text{verl}} \) is equal to \( \dot{m} \) times the difference \( (s_0 - s_6) \).
\( s_{\text{verl}} \) is equal to \( c_p \) times the natural logarithm of the fraction \( \frac{T_6}{T_0} \).
This expression simplifies to \( 1.006 \frac{\text{kJ}}{\text{kgK}} \) times the natural logarithm of the fraction \( \frac{328.85 \text{K}}{243.15 \text{K}} \).
This results in \( 0.3014 \frac{\text{kJ}}{\text{kgK}} \).
Finally, \( e_{\text{verl}} \) is equal to \( 243.15 \text{K} \) multiplied by \( 0.3014 \frac{\text{kJ}}{\text{kgK}} \), which equals \( 73.285 \frac{\text{kJ}}{\text{kg}} \).