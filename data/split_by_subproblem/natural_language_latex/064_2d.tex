The equation is O equals the fraction with numerator E subscript ex, ist plus ex subscript q minus W minus p subscript 0 times the derivative of V with respect to t, and denominator T subscript 0, minus ex subscript u subscript 0.

The equation for ex subscript q is 1 minus the fraction T subscript 0 over T subscript B, and q subscript B equals 969.58 kilojoules per kilogram.

The value of ex subscript ext is 969.58 plus 100, which equals 1069.58.

The graph is a plot with the vertical axis labeled a prime in units of T subscript k and the horizontal axis labeled S in units of L F per kilogram per Kelvin. The plot contains several curves and points:
- There is a curve starting from the origin (0,0) and moving upwards and to the right, labeled with points 1, 2, 3, 4, and 5.
- Point 1 is located near the origin.
- Point 2 is slightly above and to the right of point 1.
- Point 3 is further up and to the right, with a label "End" pointing to it.
- Point 4 is slightly below point 3, with a label "End" pointing to it.
- Point 5 is below point 4, with a label "End" pointing to it.
- There is another curve starting from the origin and moving upwards and to the right, labeled with points 6 and 7.
- Point 6 is located near the origin.
- Point 7 is slightly above and to the right of point 6.