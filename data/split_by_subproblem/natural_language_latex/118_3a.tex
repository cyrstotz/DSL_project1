The pressure of the gas is equal to the ambient pressure plus the pressure of the column.

This implies that the pressure of the gas in state 1 is equal to 1 bar plus the product of g, 3Z divided by pi times the square of half the diameter D, all multiplied by the unit conversion factors for kilograms per cubic meter and meters per second squared. This results in 32 kilograms times 9.81 meters per second squared, which equals 39.24 kilogram meters per second squared, equivalent to 39.24 Newtons.

Therefore, the pressure of the gas in state 1 is 1 plus 0.4 bar, which equals 1.4 bar.

The mass of the gas is calculated by the formula mass of gas equals pressure times volume divided by R times T, where R is replaced by the universal gas constant divided by the molar mass times 10 to the power of negative 3.

This results in 1.4 times 10 to the power of 5 times 3.14 times 10 to the power of negative 3, divided by the universal gas constant over the molar mass times 10 to the power of negative 3 times 273.15, which equals 3.42 grams.