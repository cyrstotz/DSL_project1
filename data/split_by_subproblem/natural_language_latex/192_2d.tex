Description of the diagram:

The diagram is a rectangular box labeled "system boundary." Inside the box, there are five shapes representing components of a system, drawn from left to right. The first shape is a small rectangle, followed by a larger rectangle, then a trapezoid, another rectangle, and finally a triangle pointing to the right.

On the left side of the box, there is an arrow pointing into the box labeled with mass flow rate in, denoted as m-dot subscript i. On the right side of the box, there is an arrow pointing out of the box labeled with mass flow rate out, also denoted as m-dot subscript i.

Above the box, there are two equations:
The rate of heat transfer, Q-dot, equals zero.
The rate of work done, W-dot, equals zero, with a note that heat is used for compression.

Below the box, there is a label "stationary."

The rate of change of energy with respect to time, dE_x/dt, equals the sum of the rate of energy in minus the energy lost.
This simplifies to the rate of specific energy in at the initial state minus the specific energy at state 6 minus the energy lost.

To the right of the equations, there is a note "not correct" and an arrow pointing to Q-dot not equal to zero.

For the equations under "Aufgabe 2d)":
The equation O equals specific energy at state 0 minus specific energy at state 6.
Entropy equation: O equals the mass flow rate times the difference in entropy between exit and entrance plus the rate of heat transfer over temperature plus the rate of entropy generation.
This simplifies to the difference in entropy between exit and entrance plus the rate of heat transfer over temperature plus the rate of entropy generation.
The rate of entropy generation equals the full specific energy over the reference temperature.
This simplifies further to the difference in entropy between exit and entrance plus the rate of heat transfer over temperature plus the full specific energy over the reference temperature.
The equation involving constants and logarithms of temperatures plus the rate of heat transfer over temperature plus the full specific energy over the reference temperature equals zero.
The rate of heat transfer per unit mass equals 1.455 kilojoules per kilogram.
The temperature T equals 1285 Kelvin.
The full specific energy is calculated using the negative of constants times the logarithm of the ratio of reference temperature to temperature at state 6 plus the rate of heat transfer over temperature, all multiplied by the reference temperature.
The full specific energy equals negative 152.1 kilojoules per kilogram.