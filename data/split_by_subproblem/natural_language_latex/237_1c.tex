The rate of entropy generation, denoted as S dot subscript erz, equals the mass flow rate, m dot, times the fraction of the difference in enthalpy at the outlet, h subscript aus, and the inlet, h subscript ein, over s, plus the heat transfer rate, Q dot, over the temperature T.

The entropy at the outlet, s subscript aus, is calculated as follows: the difference between 7.393 and 1.306 kilojoules per kilogram Kelvin, multiplied by 2257.0, plus 1.306 kilojoules per kilogram Kelvin, all divided by the difference between 2576.1 and 499.04 kilojoules per kilogram Kelvin, equals 7.3547 kilojoules per kilogram Kelvin.

The entropy at the inlet, s subscript ein, is calculated as follows: 0.856 times the difference between 7.393 and 0.856 kilojoules per kilogram Kelvin, divided by the difference between 26.26 and 252.38 kilojoules per kilogram Kelvin, plus 0.856 kilojoules per kilogram Kelvin, equals 7.7852 kilojoules per kilogram Kelvin.

The rate of entropy generation, S dot subscript erz, is 0.70268 kilojoules per kilogram.

26.6 kilojoules per kilogram.

For Problem 1:
The change in internal energy, Delta U, equals the heat input rate, Q dot subscript in12, plus the product of the mass flow rate from state 1 to 2, m dot subscript 12, and the internal energy U.

The mass flow rate from state 1 to 2, m dot subscript 12, equals the heat input rate, Q dot subscript in12, divided by U subscript 12, which is the difference in internal energy from state 2 to state 1, calculated as m dot times the difference between U subscript 2 and U subscript 1.

The internal energy at state 1, U subscript 1, equals the saturated liquid internal energy, U subscript f, minus omega times the difference between the saturated vapor internal energy, U subscript g, and U subscript f. The value is given as 100 raised to the power of 8 equals 40R.15 kilojoules per kilogram.

U subscript 12 equals the saturated liquid internal energy, U subscript f, plus x times the difference between the saturated vapor internal energy, U subscript g, and U subscript f, equals 2333.33 kilojoules per kilogram.

The quality at state 1 to 2, x subscript 12, is 0.03.

The change in entropy from state 1 to 2, Delta S subscript 12, involves the mass flow rate, m dot, but the equation is incomplete.