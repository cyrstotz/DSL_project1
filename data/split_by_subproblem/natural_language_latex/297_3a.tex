Problem statement:

The pressure \( p_{\text{M,gr,1}} \) is equal to the ambient pressure \( p_{\text{amb}} \) plus the force per unit area due to gravity \( \frac{F_{\text{gr,k}}}{A} \) plus the force per unit area due to mass \( \frac{F_{\text{m,EW}}}{A} \).

The gas constant \( R \) divided by the molar mass \( M_r \) equals 8.166 joules per kilogram Kelvin, and the area \( A \) is equal to pi times the square of half the diameter \( d \), which equals 7.854 times 10 to the negative third square meters.

The force per unit area due to gravity \( \frac{F_{\text{gr,k}}}{A} \) is calculated as 32 kilograms times 9.81 meters per second squared divided by 7.854 times 10 to the negative third square meters, resulting in 39,969.98 Pascals.

The force per unit area due to mass \( \frac{F_{\text{m,EW}}}{A} \) is calculated as the gravitational constant \( G \) times 4 kilograms times 9.81 meters per second squared divided by \( A \), resulting in 12,461.50 Pascals, leading to a pressure \( p_{\text{gr,1}} \) of 140,098.38 Pascals.

The mass \( m_{g} \) is calculated as the pressure \( p_{v,1} \) times the volume \( V \) divided by \( R \) times the temperature \( T \), resulting in 3.483 times 10 to the negative third kilograms, which is 3.483 grams.

The temperature \( T \) is 773.15 Kelvin.