Given:
- The temperature \( T_{gz} \) is 500 degrees Celsius.
- The volume \( V_{gz} \) is 3.14 times 10 to the power of negative 3 cubic meters.

The pressure \( p_{gz} \) is calculated using the formula:
\[ p_{gz} = \frac{m \cdot R \cdot T}{V} \]

Where the gas constant \( R \) is given by:
\[ R = \frac{\bar{R}}{M} = \frac{8.314 \, \text{Joules per mole Kelvin}}{50 \, \text{kilograms per kilomole}} = 0.16628 \, \text{kilogram meters squared per second squared Kelvin kilogram} \]

The pressure \( P \) is calculated using the formula:
\[ P = \frac{F}{A} = \frac{32 \, \text{kilograms} \cdot 9.81 \, \text{meters per second squared} + 0.1 \, \text{kilograms} \cdot 9.81 \, \text{meters per second squared}}{25 \, \text{meters} \cdot (0.05)^2 \pi} + 100 \, \text{Pascals} + 1 \, \text{bar} + 10^5 \, \text{Pascals} \]

The pressure \( p \) is:
\[ p = 8 \cdot 10^5 \, \text{bar} = 1,000,094.94006 \, \text{Pascals} \]

The mass \( m \) is calculated using the formula:
\[ m = \frac{PV}{RT} = 5.273 \, \text{kilograms} \quad (\text{Probably not}) \]

Revised calculation for mass \( m \) is:
\[ m = \frac{1,000,094.9 \, \text{Pascals} \cdot 3.14 \cdot 10^{-3} \, \text{cubic meters}}{0.16628 \, \text{kilogram meters squared per second squared Kelvin kilogram} \cdot (500 \, \text{Kelvin} + 273.15 \, \text{Kelvin})} = 3.927 \, \text{kilograms} \]