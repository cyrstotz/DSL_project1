The graph is a Temperature-Entropy (T-s) diagram. The x-axis is labeled with entropy in units of kilojoules per kilogram Kelvin, and the y-axis is labeled with temperature in Kelvin. There are six points labeled from 1 through 6. The graph displays several isobars, which are constant pressure lines, labeled as \( p_0 \), \( p_1 \), \( p_2 \), \( p_3 \), and \( p_4 \). These isobars are curved lines that generally slope upwards to the right. The process path is indicated by arrows connecting the points in the sequence: from 1 to 2, from 2 to 3, from 3 to 4, from 4 to 5, and from 5 to 6. The path from point 1 to point 2 and from point 3 to point 4 is vertical, indicating an isentropic process, which means constant entropy. The path from point 2 to point 3 and from point 4 to point 5 is horizontal, indicating an isothermal process, which means constant temperature. The path from point 5 to point 6 is a downward sloping line, indicating a polytropic process. Additionally, there are dashed lines indicating the saturation lines.