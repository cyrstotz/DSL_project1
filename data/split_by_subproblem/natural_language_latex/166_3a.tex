a) The cylinder has an area A equals pi times the square of the diameter d divided by 2, which equals 7.854 times 10 to the power of negative 3 square meters.

The initial pressure p1,0 equals the mass m times the gas constant R times the temperature T divided by the volume V, and we do not know m and p.

The equation pV equals mRT.

That is, the initial pressure p1,0 times the area A equals the mass of water vapor m_e,w times the gravitational acceleration g plus the mass of air m_k times g plus the atmospheric pressure p_atm times the area A.

The initial pressure p1,0 equals the mass of water vapor m_e,w times g divided by the area A plus the mass of air m_k times g divided by the area A plus the atmospheric pressure p_atm.

The initial pressure p1,0 equals.

The mass of gas m_g equals the initial pressure p1,0 times the volume of gas V_g-1 divided by the gas constant R minus the temperature at state g-1 T_g-1 divided by the temperature at state g-1 T_g-1.

The gas constant R equals the universal gas constant R bar divided by the molar mass of the gas M_g, which equals 0.163 times the universal gas constant R bar per kilogram Kelvin.