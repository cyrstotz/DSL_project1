Delta S equals m subscript 2 times s subscript 2 minus m subscript 1 times s subscript 1.

This equals the quantity m subscript 1 plus Delta m subscript 12, all multiplied by s subscript 2, minus m subscript 1 times s subscript 1.

Delta S subscript 12 equals 1387.22 kilojoules per Kelvin.

Table A2:

s subscript 2 equals s subscript f at 70 degrees Celsius equals 0.9549 kilojoules per kilogram Kelvin.

s subscript 1 equals s subscript f at 100 degrees Celsius plus x subscript 0 times the quantity s subscript g at 100 degrees Celsius minus s subscript f at 100 degrees Celsius.

s subscript f at 100 degrees Celsius equals 1.306 kilojoules per kilogram Kelvin and s subscript g at 100 degrees Celsius equals 7.3549 kilojoules per kilogram Kelvin.

s subscript 1 equals 1.33714 kilojoules per kilogram Kelvin.