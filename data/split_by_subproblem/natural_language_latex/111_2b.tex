Energy balance around aircraft engine:

Stationary implies that zero equals the mass flow rate (geometric) times the quantity of enthalpy at point 0 minus enthalpy at point 6 plus half the difference of the square of internal velocity minus the square of velocity at point 6, plus heat added minus volume flow rate in.

The nozzle is adiabatic reversible, implying it is isentropic.

Polytropic relationship:
The ratio of temperature at point 6 to temperature at point 5 equals the ratio of enthalpy at point 6 to enthalpy at point 5, raised to the power of (k-1) divided by k.

This implies that temperature at point 6 equals 431.9 times (0.191 times the reciprocal of 0.5), which equals 328.75 Kelvin.

This implies from equation (I) that zero equals the mass flow rate (geometric) times the quantity of enthalpy at point 5 minus enthalpy at point 6 plus half the difference of the square of velocity at point 5 minus the square of velocity at point 6.

Energy balance around the nozzle:

Stationary implies that zero equals the mass flow rate (geometric) times the quantity of enthalpy at point 5 minus enthalpy at point 6 plus half the difference of the square of velocity at point 5 minus the square of velocity at point 6, plus the ratio of heat flow rate to mass flow rate (geometric) minus the sum of the ratio of work rate to mass flow rate (geometric).

This implies that zero equals the specific heat at constant pressure times the difference in temperature between point 5 and point 6 plus half the difference of the square of velocity at point 5 minus the square of velocity at point 6.

This implies that the square of velocity at point 6 equals twice the specific heat at constant pressure times the difference in temperature between point 5 and point 6 plus the square of velocity at point 5.

Velocity at point 6 equals the square root of twice times 1.006 times the difference between 431.9 Kelvin and 328.75 Kelvin times 1000 plus the square of 22.

This equals 457.29 meters per second.

This equals 507.24 meters per second.