The pressure \( P_{S1} \) and the product of mass \( m \) and gravitational acceleration \( g \) are unknown.

The measurable diameter \( D \) is 10 centimeters, which equals 0.1 meters.

The area \( A \) is calculated using the formula for the area of a circle, \( A = \left( \frac{d}{2} \right)^2 \pi \), resulting in 0.00785 square meters.

The diagram or formula includes atmospheric pressure \( \text{Patm} \), mass per length \( \frac{m}{l} \) \( \text{mew} \), and gas pressure \( \text{Pgas} \).

The balance equation is given by \( \text{Patm} \cdot A + m_k \cdot g + m_{ew} \cdot g = \text{Pgas} \cdot A \).

The gas pressure \( \text{Pgas} \) is calculated as \( \text{Patm} + \frac{(m_k + m_{ew}) \cdot g}{A} \), resulting in 1 Bar plus a pressure calculated from 32.1 kilograms times 9.81 meters per second squared divided by 0.00785 square meters, all in Pascals.

This results in 1 Bar plus 40,059.44076 Pascals.

The adjusted pressure \( \text{Paas} \) is 1 Bar plus 0.401 Bar, totaling 1.401 Bars.

The mass \( \text{maas} \) is calculated using the ideal gas law formula \( \frac{P \cdot V}{R \cdot T} \).

The volume \( V \) is 3.4 liters, which is 0.0034 cubic meters.

The temperature \( T \) is 500 degrees Celsius, which is 773.15 Kelvin.

The gas pressure \( P_{\text{Gas}} \) is 1.401 Bars, which equals 140,059 Pascals.

The gas constant \( R \) is calculated as \( \frac{8.314 \text{ kJ}}{\text{kmol} \cdot \text{K}} \) divided by \( 50 \text{ kg}/\text{kmol} \), resulting in 0.16628 kJ per kg per Kelvin.

The mass \( \text{maas} \) is then recalculated using the new values, resulting in 0.00392 kilograms or 3.922 grams.

The mass \( m \) is 0.1 kilograms and \( x_{0,5} \) is 0.6.

The system of ice water and gas is in equilibrium.

The internal energy per mass unit for gas \( u_1 \) is calculated as \( \frac{u_2}{m} \), resulting in 483.40335 kJ per kg.

The internal energy per mass unit for gas \( u_2 \) is 472.3 kJ per kg.

The internal energy for ice water \( u_1 \) is not specified.

The pressure \( p \) is unknown and calculated using the formula \( \frac{m \cdot g}{A} + p_{\text{atm}} = p \), resulting in 1.4 bars.

The internal energy \( u_2 \) is calculated using the formula \( u_{\text{fest}} + x \cdot (u_{\text{flüssig}} - u_{\text{fest}}) \), with \( x = 0.4 \), indicating 0.4 times 0.6 ice.

The internal energy \( u_2 \) is calculated as -333.458 kJ per kg plus 0.4 times -0.06 kJ per kg, resulting in -333.458 kJ per kg.

The internal energy \( u_2 \) is recalculated as -200.0328 kJ per kg.

The change in energy \( \Delta E \) is calculated as \( Q_{1} - W_{\text{in}} \), resulting in \( (u_2 - u_1) \cdot m \).

The internal energy \( u_2 \) is recalculated using the heat transfer \( Q_{12} \), resulting in -186.618 kJ per kg.

The fraction of water \( x \) is calculated using the formula \( \frac{u_2 - u_{\text{fest}}}{u_{\text{flüssig}} - u_{\text{fest}}} \), resulting in 0.441.

The fraction of ice \( x_{\text{Eis}} \) is 1 minus \( x \), resulting in 0.559.