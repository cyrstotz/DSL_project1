The specific heat capacity at constant pressure for a gas, denoted as \( C_{p, \text{gas}} \), is given by the formula \( \frac{R}{M} + c_v \) which equals \( 7.99 \times 28 \) joules per kilogram per Kelvin.

The change in enthalpy due to temperature, denoted as \( m_f (h_{f2} - h_{f1}) \), is equal to \( m_f (h_{m2} - h_{m1}) \).

The equation \( m_s c_p (T_{m2} - T_2) \) is equal to \( m_s c_p (T_{m2} - T_2) \).

Since in state 2, water is still at \( 0^\circ C \) and state 2 is the gaseous state, the gas is also at \( 0^\circ C \).

The heat transfer from state 1 to state 2, denoted as \( Q_{12} \), is given by \( m_g (h_{2g} - h_{2f}) \).

This can be further expressed as \( m_g C_{p,g} (T_2 - T_m) \).

The numerical value of this heat transfer is \( 1366.77 \) joules.

The temperature difference \( T_2 - T_m \) is \( 500 \) Kelvin.