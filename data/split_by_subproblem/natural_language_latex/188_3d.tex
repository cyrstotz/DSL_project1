The values are given as follows:
- \( p_1 \) is 1.4 bar,
- \( m_{\text{ev}} \) (mass of evaporated substance) is 0.1 kg,
- \( m_{\text{eis}} \) (mass of ice) is 0.06 kg,
- \( m_{\text{los}} \) (mass of dissolved substance) is 0.04 kg.

For \( u_1 \):
- \( u_1 \) is calculated as the sum of \( u_{\text{fest}} \) (energy per unit mass of solid) times \( m_{\text{fest}} \) (mass of solid) plus \( u_{\text{flüss}} \) (energy per unit mass of fluid) times \( m_{\text{los}} \) (mass of dissolved substance).
- Substituting the values, \( u_1 \) equals -333.638 kJ/kg times 0.06 kg plus -0.045 kJ/kg times 0.04 kg, which results in -20.01 kJ.

For \( u_2 \):
- \( u_2 \) is calculated as \( u_1 \) plus \( \dot{Q}_{12} \) (heat added from state 1 to 2).
- Substituting the values, \( u_2 \) equals -20.01 kJ plus 1.082 kJ, resulting in -18.927 kJ.

The value of \( u_{25} \) is -18.927 kJ, which is also expressed as -189.273 kJ/kg.

For \( x \):
- \( x \) is calculated using the formula involving \( u_{25} \), \( u_{\text{eis}} \) (energy per unit mass of ice), \( R \) (a constant), \( u_{\text{flüssig}} \) (energy per unit mass of liquid), and \( u_{\text{fest}} \).
- The formula simplifies to a fraction involving these terms, resulting in a value of \( 4.32 \times 10^{-1} \).