p1 equals 1 bar plus the fraction of 32 kilograms over meters squared times 0.05 meters over 0.05 meters squared, which equals 1 bar plus atmospheric pressure, resulting in a total of 1.399695 bar.

M equals the fraction of p times V over R times T1, which equals the fraction of 1.399695 bar over 0.0009473 kilograms.

R equals the fraction of pi over M subscript H2 squared, which equals 0.16683 liters squared per kilogram.

Delta E subscript EV equals Delta U subscript EW equals dot Q subscript 12.

U subscript 2EW minus U subscript 1EW is not equal to Q subscript 12, which equals the fraction of 1500 over Mev, resulting in 75 kilojoules per kilogram.

U1 equals the sum of w1 times U subscript rest at 0 degrees plus 0.4 times U subscript raising at 0 degrees, allowed, which equals negative 200.0928 kilojoules per kilogram.

U2 equals the sum of x2 times U subscript rest at 0.009 degrees plus w times (1 minus x2) times U subscript finishing at 0.009 degrees, allowed.

U subscript rest at 0 degrees equals negative 333.438 kilojoules per kilogram.

U subscript finishing at 0 degrees equals negative 0.045 kilojoules per kilogram.

U subscript rest at 0.009 degrees equals negative 333.442 kilojoules per kilogram.

U subscript finishing at 0.009 degrees equals negative 0.033 kilojoules per kilogram.

Negative x2 times 333.442 kilojoules per kilogram times (7 minus x2) times 0.033 kilojoules per kilogram equals the sum of 75 kilojoules per kilogram minus 200.0928 kilojoules per kilogram, resulting in negative 195.0928 kilojoules per kilogram.

x2 equals the fraction of 185.0928 kilojoules per kilogram minus 0.033 kilojoules per kilogram over 333.442 kilojoules per kilogram minus 0.033 kilojoules per kilogram, resulting in approximately 0.555.