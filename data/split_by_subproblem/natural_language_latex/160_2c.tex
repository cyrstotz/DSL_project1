The mass flow rate times the extraction work is equal to the mass flow rate times the quantity of enthalpy minus reference enthalpy minus the product of reference temperature and the difference in entropy plus kinetic energy term plus a canceled term of pressure energy to the power of zero.

The extraction work at state s is equal to the enthalpy at state s minus reference enthalpy minus the product of reference temperature and the difference in entropy at state s plus half the square of velocity at state s. The extraction work at reference state is equal to reference enthalpy minus itself minus the product of reference temperature and the difference in entropy at reference state plus half the square of velocity at reference state.

The change in extraction work is equal to the enthalpy at state s minus reference enthalpy minus the product of reference temperature and the difference in entropy at state s plus half the square of velocity at state s minus half the square of velocity at reference state.

The difference in enthalpy between state s and reference state is equal to the specific heat at constant pressure times the difference in temperature between state s and reference temperature. The difference in entropy between state s and reference state is equal to the specific heat at constant pressure times the natural logarithm of the ratio of temperature at state s to reference temperature minus the gas constant times the natural logarithm of the ratio of pressure at state s to reference pressure.

The change in extraction work is equal to the specific heat at constant pressure times the difference in temperature between state s and reference temperature minus the product of reference temperature and specific heat at constant pressure times the natural logarithm of the ratio of temperature at state s to reference temperature plus half the square of velocity at state s minus half the square of velocity at reference state.

This results in 1.006 kilojoules per kilogram Kelvin times the difference between 340 Kelvin and the sum of 30 and 273.15 Kelvin minus the sum of 30 and 273.15 Kelvin times 1.006 kilojoules per kilogram Kelvin times the natural logarithm of the ratio of 340 Kelvin to the sum of 30 and 273.15 Kelvin plus half the square of 340 minus half the square of 200. This equals 110.65 kilojoules per kilogram, approximately equal to 110.1 megajoules per kilogram.