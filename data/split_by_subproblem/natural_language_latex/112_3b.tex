The temperature \( T_{g1,2} \) increases as \( p_{g1,2} \) increases. The pressure \( p_{g1,2} \) equals \( p_{g1,1} \), indicating that the pressure exerted from outside on the piston (by external weight and piston) does not change. Consequently, the temperature \( T_{g1,2} \) is calculated as \( \frac{p_{g1,1} \cdot V_{3,1}}{m_g \cdot R} \), which equals \( \frac{5 \cdot 10^5 \cdot 2}{2 \cdot 8314} \), resulting in \( 1200^\circ C \). The temperature \( T_{g1,2} \) reaches equilibrium, meaning the temperature becomes uniform, and eventually, \( T_{g1,2} \) equals \( T_{EW,2} \).