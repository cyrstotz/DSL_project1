The variables are \( P_{g1} \), \( V_{g1} \), and \( m_g \). The equation \( P_{g1} V_{g1} \) equals \( R_{mg} T_{g1} \), which is further expanded. The gas constant \( R_g \) is calculated as \( R_g = \frac{R}{M_g} \), where \( R \) is \( 8.314 \frac{J}{\text{mol K}} \) and \( M_g \) is \( 50 \frac{g}{\text{mol}} \), resulting in \( R_g = 0.166 \frac{J}{g K} \) or \( 166 \frac{J}{kg K} \). The sum of \( P_{EW} \) and \( P_{G1} \) is given by \( \frac{m_g}{\pi D^2_g} + \text{Pamb} \).