Graph 1: A graph with the y-axis labeled p and the x-axis labeled T. There are two intersecting curves. The first curve starts from the top left and curves downwards to the bottom right. The second curve starts from the bottom left and curves upwards to the top right. The intersection point of the two curves is labeled with points 1, 2, 3, and 4, forming a cycle. The point 1 is at the bottom left of the intersection, point 2 is at the bottom right, point 3 is at the top right, and point 4 is at the top left. The cycle is labeled as "Cycle". There is an additional line labeled T_F that intersects the cycle from the bottom left to the top right.

Graph 2: A graph with the y-axis labeled p and the x-axis labeled T. There are three regions labeled "solid", "liquid", and "gaseous". The "solid" region is at the top left, the "liquid" region is at the bottom left, and the "gaseous" region is at the bottom right. There is a curve separating the "solid" and "liquid" regions, and another curve separating the "liquid" and "gaseous" regions. The intersection of these curves is labeled as point 1. There is a vertical line from point 1 to point 2, labeled "sublime". There is a horizontal line from point 2 to point 3, labeled "via 1". There is a vertical line from point 3 to point 4, labeled "via 2". The area enclosed by these points is shaded and labeled as "i".

Graph 3: A small horizontal line labeled "gas" on the left and "liquid" on the right. Points 2 and 1 are marked on this line.

Graph 4: A vertical line with points 4 and 1 marked on it.

Graph 5: A curve with points 2 and 3 marked on it. The curve is labeled "itself".

The equation is epsilon_u equals the absolute value of i_u divided by the absolute value of i_a0 minus the absolute value of i_z0, which equals the absolute value of i_u divided by the absolute value of i_a0 minus the absolute value of i_u.