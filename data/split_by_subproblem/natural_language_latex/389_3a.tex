Item a) The pressure \( p_{a,1} \) is equal to \( p_0 \) plus the fraction of \( m k g \) over \( A \) plus the fraction of \( m_{EW} g \) over \( A \).

The area \( A \) is equal to pi times the square of half the diameter \( D \), which equals 0.007854 square meters.

The initial pressure \( p_0 \) is \( 10^5 \) Pascals plus the fraction of 32 kilograms times 9.81 meters per second squared over 0.007854 square meters plus the fraction of 0.1 kilograms times 9.81 meters per second squared over 0.007854 square meters.

The pressure \( p_{a,1} \) equals 140.1 kiloPascals.

The product of pressure \( p \) and volume \( V \) equals the gas constant \( R \) times the temperature \( T \).

The gas constant \( R \) is the universal gas constant \( \overline{R} \) divided by the molar mass \( M \), which equals 8.314 Joules per mole Kelvin divided by 50 grams per mole, resulting in 0.16628 Joules per gram Kelvin.

The volume \( V_{a,1} \) is the product of \( R \) and the temperature \( T_{a,1} \) divided by \( p_{a,1} \), which equals 0.16628 times 773.15 divided by 140.1, resulting in 0.9176 cubic meters per kilogram.

The mass \( m \) is the volume \( V_{a,1} \) divided by \( V_{a,1} \), which equals 0.003316 cubic meters divided by 0.9176 cubic meters per kilogram, resulting in approximately 0.003972 kilograms or 3.972 grams, denoted as \( m_{a,1} \).