a) Q subscript aus

1. First Law of Thermodynamics:
The rate of heat transfer Q dot subscript E equals the sum over i of mass flow rate m dot subscript i times enthalpy h subscript i at the inlet minus the sum over j of mass flow rate m dot subscript j times enthalpy h subscript j at the outlet plus the sum over i of heat transfer rate Q dot subscript i at state i minus the sum over j of work rate W dot subscript j at state j.
Q dot subscript E equals mass flow rate at the inlet m dot subscript ein times the sum of enthalpy at the inlet h subscript ein and enthalpy at the outlet h subscript aus plus the rate of heat transfer Q dot subscript R plus the rate of heat transfer Q dot subscript aus.

State Properties:
The difference between enthalpy at the inlet h subscript ein and enthalpy at the outlet h subscript aus.

State 1:
Inlet: 70 degrees Celsius.
From Table A-2.
And q subscript 2 equals q subscript 1 plus the difference between q subscript 2 and q subscript 1.

State 2:
Inlet: 100 degrees Celsius.
From Table A-2.
And q subscript 2 equals q subscript 1 plus the difference between q subscript 2 and q subscript 1.

Enthalpy at the inlet h subscript ein equals 297.35 plus 0.005 times 0.025 times the difference between 297.35 and 253.15 kilojoules per kilogram equals 304.643 kilojoules per kilogram.

Enthalpy at the outlet h subscript aus equals 473.04 plus 0.005 times 0.10 times 5 times 297.6 times the difference between 473.04 and 253.15 kilojoules per kilogram equals 430.325 kilojoules per kilogram.

The rate of heat transfer Q dot subscript aus equals 0.3 kilograms per second times the difference between 304.643 kilojoules per kilogram and 430.325 kilojoules per kilogram plus 100 kilowatts per kilogram minus the rate of heat transfer Q dot subscript aus.

Therefore, the rate of heat transfer Q dot subscript aus equals 62.29 kilojoules.