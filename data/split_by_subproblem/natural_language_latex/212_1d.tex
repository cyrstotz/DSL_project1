The equations are as follows:

1. The mass fraction of ice at state 2, denoted as \( x_{Eis,2 \, \text{ges}} \), is given by the ratio of the remaining mass \( m_{Rest} \) to the sum of the remaining mass \( m_{Rest} \) and the mass of the liquid \( m_{Flüssig} \).

2. The change in internal energy from state 1 to state 2, denoted as \( u_2 - u_1 \), is equal to the specific heat at constant volume \( c_V \) times the ratio of the temperature difference \( T_2 - T_1 \) to the initial temperature \( T_1 \). This results in a change in internal energy \( \Delta u_2 \) of negative 200.091 kilojoules per kilogram.

3. The equation \( x \cdot u_{Eis,0^\circ} + (1 - x) \cdot u_{Fl,0^\circ} = -200.093 \, \text{kJ/kg} \) represents a balance of internal energies at 0 degrees Celsius, where \( x \) is the fraction of ice and \( 1-x \) is the fraction of liquid.

4. The internal energy at state 2, \( u_2 \), is expressed as a weighted average of the internal energies of ice at 0.03 degrees Celsius \( u_{Eis,0.03^\circ} \) and liquid at 0.03 degrees Celsius \( u_{Fl,0.03^\circ} \), weighted by \( x \) and \( 1-x \) respectively.

5. From the calculations, the value of \( x \) is determined to be 0.600.