The pressure \( p_T \) is 0.5 bar, the temperature \( T_5 \) is 473.1 Kelvin, and the velocity \( w_5 \) is 720 meters per second.

For \( T_6 \) with a polytropic ratio, \( n \) equals \( \kappa \) which is 1.4.

The formula for \( T_6 \) is given by:
\[ T_6 = T_5 \left( \frac{p_6}{p_5} \right)^{\frac{n-1}{n}} \]

Where \( p_6 \) is 0.181 bar.

This results in \( T_6 \) being 328.1 Kelvin.

Underlined result: \( T_6 = 328.1 \, \text{K} \)

This leads to the first law of thermodynamics for a stationary process:
\[ 0 = \dot{m} \left( h_e - h_a + \frac{1}{2} (w_e^2 - w_a^2) \right) + \cancel{\dot{Q}} - \cancel{\dot{W}} \]
This is under adiabatic conditions where velocity \( v \) equals specific heat at constant volume \( c \).

The equation simplifies to:
\[ 2(h_6 - h_5) = w_5^2 - w_6^2 \]

Rearranging gives:
\[ 2(h_6 - h_5) - w_5^2 = -w_6^2 \]

Taking the square root of both sides:
\[ \sqrt{w_5^2 + 2(h_5 - h_6)} = w_6 \]

The enthalpy difference \( h_5 - h_6 \) is calculated using:
\[ h_5 - h_6 = c_p (T_5 - T_6) \]

Where the specific heat at constant pressure \( c_p \) is 1.006.

Underlined result: \( w_6 = 507.2 \, \text{m/s} \)