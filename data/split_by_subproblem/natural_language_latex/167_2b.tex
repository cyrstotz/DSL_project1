The rate of heat transfer, Q dot, equals the mass flow rate, m dot, times the expression in parentheses which includes the difference in enthalpy between the exit and the approach, h sub e minus h sub a, plus half the difference in the squares of the exit and approach velocities, w sub e squared minus w sub a squared. This is approximately equal to Q dot minus W dot squared.

This implies that the exit temperature, T sub e, equals the initial temperature, T sub 1, times the ratio of pressure p sub 2 to p sub 1 raised to the power of (gamma minus one) divided by gamma.

The pressure p sub 1 is equal to the gas constant R times the temperature T divided by the specific volume v.