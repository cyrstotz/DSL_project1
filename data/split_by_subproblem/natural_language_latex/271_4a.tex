Here is a detailed verbal description of the graph:

The graph is drawn on grid paper. The x-axis is labeled 'T [K]' (temperature in Kelvin) and the y-axis is labeled 'p-bar' (average pressure). The graph displays a curve that begins at the origin and curves upwards to the right.

Several annotations and arrows are present on the graph:
- A point on the curve is marked and labeled as 'i'.
- From point 'i', a horizontal arrow extends to the right and is labeled 'T_tripol' (triple point temperature).
- From point 'i', a vertical arrow extends downwards and is labeled 'ii'.
- Another point on the curve, to the right of point 'i', is labeled 'ii'.
- From point 'ii', a vertical arrow extends downwards.
- The region to the left of the curve is labeled 'Fest' (solid).
- The region to the right of the curve is labeled 'Flüssig' (liquid).
- The region below the curve is labeled 'Gas' (gas).