b) Energy balance around the compressor, static flow process with mass flow rate (m dot)

The equation is zero equals m dot times the quantity of h subscript e minus h subscript a plus the quantity of w subscript e squared minus w subscript a squared all over 2 plus g times the quantity of z subscript e minus z subscript a, all this plus Q minus W.

The equation is zero equals m dot times the quantity of h subscript 2 minus h subscript 3 plus Q minus W, which leads to m dot equals W divided by the quantity of h subscript 2 minus h subscript 3.

Item 4:
Temperature T subscript 2 in the evaporator equals 257.15 Kelvin, read from Figure 5 (see page 4).

Pressure p subscript 2 from Table A-10 at 257.15 Kelvin equals negative 16 degrees Celsius.

p subscript 2 equals 1.5748 bar.

Pressure p subscript 2 from Table A-12 at negative 16 degrees Celsius indicates that the vapor is saturated. Interpolation is done between temperatures T subscript sat equals negative 18.8 degrees Celsius and T subscript sat equals negative 12.73 degrees Celsius.

Using the interpolation formula:
y equals y subscript 1 plus the fraction of y subscript 2 minus y subscript 1 over x subscript 2 minus x subscript 1 times the quantity of x minus x subscript 1.

p subscript 2 equals 1.4 bar plus the fraction of 1.8 bar minus 1.4 bar over negative 12.73 degrees Celsius plus 18.8 degrees Celsius times the quantity of negative 16 degrees Celsius plus 18.8 degrees Celsius equals 1.58 bar.

h subscript 2 equals h subscript g from Table A-11 interpolated for 1.58 bar.

h subscript 2 equals 237.77 kilojoules per kilogram.