The problem consists of several equations:

1. Delta S subscript 12.
2. m subscript 2 times S subscript 2 minus m subscript 1 times S subscript 1 equals.
3. Delta S subscript 12 equals 15475 kilojoules per Kelvin.

Under the subsection "TAB A-Z":

4. S subscript 1 equals S subscript f plus x times (S subscript g minus S subscript f).
5. S subscript f at 100 degrees Celsius equals 1.3069.
6. S subscript g at 100 degrees Celsius equals 7.3595.
7. x equals 0.005.
8. S subscript 1 equals 1.33717.
9. S subscript 2 equals S subscript f at 700 degrees Celsius, which is referred to as "TAB A-Z", equals 0.9549.