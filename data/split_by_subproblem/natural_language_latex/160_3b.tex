b) The value of \( x_{eis,2} \) is greater than 0, and \( x_{eis,1} \) equals the ratio of \( m_{eis} \) to \( m_{ew} \), which is 0.6.

It follows that \( m_{eis} \) equals 0.1 times \( m_{ew} \), which is 0.06 kilograms.

**Gas and EW Thermodynamic Equilibrium**

Because the density of ice and water are the same, the mass (and volume) of ice water does not change. From the equilibrium constant, it is seen that \( p_{g2} \) must equal \( p_{g1} \).

The value of \( p_{g1,2} \) is 1.4 bar.

b) continued.

A diagram showing two horizontal sections. The top section is labeled "EV" and the bottom section is labeled "Gas". There is an arrow labeled "\(\dot{Q}\)" pointing upwards from the "Gas" section to the "EV" section.

The mass \( m_{g1} \) equals \( m_{g2} \), which is 3.422 grams.

Because the EV can then be treated like water in wet steam and \( x_2 \) is greater than 0, the temperature \( T_{EV,2} \) equals \( T_{EV,1} \), which is 0 degrees Celsius.

For thermodynamic equilibrium, the temperature \( T_{g,2} \) must equal \( T_{EV,2} \), which is 0 degrees Celsius.