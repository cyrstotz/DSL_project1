The first equation is:

p_a1 equals the fraction with numerator m_w times g plus m_e times g plus p_amb times A, and denominator A.

This simplifies to:

p_a1 equals the fraction (m_e plus m_w) over A times g plus p_amb.

Substituting values, it becomes:

p_a1 equals the fraction 0.7 kilograms plus 32 grams over pi times (0.7 meters squared) times 9.81 meters per second squared plus 1.05 Newtons per square meter.

This results in:

p_a1 equals 1.40 bar.

The next equation is:

pV equals mRT implies m equals the fraction pV over RT times M_w.

Substituting values into this equation, we get:

m equals the fraction 7.4 times 10 to the power of 2 Pascals times 3.74 times 10 to the power of -3 cubic meters over 8.31 Joules per Kelvin per mole times 373.15 Kelvin times 50 grams per mole.

This results in:

m equals 3.66 grams.