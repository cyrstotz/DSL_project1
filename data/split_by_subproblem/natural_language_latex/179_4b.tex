In the table provided, the columns are labeled from 1 to 4, and the rows are labeled T for temperature, p for pressure, and h for enthalpy. The values in the table are as follows:
- In the second column, the temperature is -6 degrees Celsius and the pressure is 8 bar.
- In the third column, the temperature is 37.33 degrees Celsius and the pressure is 8 bar.
- The other entries in the table are empty.

The temperature T1 is given as -6 degrees Celsius.

The term "adiabat reversibel" translates to "reversible adiabatic," which is equivalent to "isentropic."

The entropy s2 equals s3 and is calculated as 0.9273 plus the result of (-6 plus 6 divided by 8.4) times 0.9235, which equals 0.9273 minus 0.9226 kilojoules per kilogram Kelvin.

The equation involving mass flow rate (m dot) times the difference in enthalpy (h2 minus h3) minus the power output (W dot subscript x) equals zero.

The mass flow rate (m dot) is calculated as the power output (W dot subscript x) divided by the difference in enthalpy (h2 minus h3), which equals 0.00170 kilojoules per second, or 3.973 kilograms per second.

The enthalpy h2 is calculated as 264.7 plus the result of (-6 plus 6 divided by 8.4) times (274.54 minus 264.54), which equals 273.72 kilojoules per kilogram.

The enthalpy h3 is calculated as 0.0066 plus the result of (0.932 minus 0.0066 divided by 273.66 minus 264.75) times (273.66 minus 264.75), which simplifies to 0.9334 minus 0.0066, resulting in 65.050 kilojoules per kilogram.