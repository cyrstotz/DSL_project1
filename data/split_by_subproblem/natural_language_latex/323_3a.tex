Subsection 3a) Equilibrium \( p_{g_1} \)

Diagram: A horizontal rectangle with arrows pointing downwards labeled \( P_0 \), \( \frac{m_{E_1}g}{A} \), \( \frac{m_{A_1}g}{A} \) and an arrow pointing upwards labeled \( qg_{g_1} \).

The pressure \( p_{g_1} \) is equal to \( P_0 \) plus \( \frac{m_{E_1}g}{A} \) plus \( \frac{m_{A_1}g}{A} \).

This equals \( 10^5 \) Pascals plus \( 0.14 \) kilograms times \( 9.81 \) meters per second squared times \( \frac{1}{A} \) plus \( 32 \) kilograms times \( 9.81 \) meters per second squared times \( \frac{1}{A} \).

This results in \( 140144.78 \) Pascals (Continue calculating with other values).

The diameter \( \sigma \) is \( 0.05 \) meters.

The area \( A \) is \( \pi \) times \( \sigma^2 \), which is \( 0.05 \) meters times \( \pi \).

The area \( A \) is \( 0.00785 \) square meters.

Using the ideal gas law \( pV = mRT \):

The gas constant \( R \) is \( \frac{R^*}{M} \) which equals \( 8.314 \) Joules per mole Kelvin times \( 10^{-3} \) cubic meters per mole Kelvin times \( \frac{50 \) kilograms}{kilomole} \).

\( R \) equals \( 166.28 \) Joules per kilogram Kelvin.

The equation \( p_{g_1} V_{g_1} = m_{g_2} R T_{g_1} \) implies:

\( m_{g_2} \) equals \( \frac{140144.78 \) Pascals times \( 3.14 \times 10^{-3} \) cubic meters}{166.28 \) Joules per kilogram Kelvin times \( (273.15 + 500) \) Kelvin}.

This results in \( 3.422 \) grams.