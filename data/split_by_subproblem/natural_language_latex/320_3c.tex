The change in energy, denoted as Delta E, is equal to the heat transfer rate from state 1 to state 2, denoted as Q-dot with subscript 12, minus the work rate from state 1 to state 2, denoted as W-dot with subscript 12.

The mass at state 2 times the internal energy at state 2 minus the mass at state 1 times the internal energy at state 1 is equal to the heat transfer rate from state 1 to state 2 minus the work done from state 1 to state 2, denoted as W subscript A2.

The work done from state 1 to state 2, W subscript A2, is equal to the integral of pressure times the change in volume, which equals R times the difference in temperature between state 2 and A2 divided by 1 minus n. Substituting values, it equals 166.25 Joules per kilogram Kelvin times the difference between 273.15 Kelvin and 773.15 Kelvin divided by 1 minus 1.263, resulting in 316.12 kilojoules per kilogram.

The value of n is calculated as the ratio of specific heat at constant pressure, c_p, to specific heat at constant volume, c_v, which is 0.79528 divided by 0.633, approximately equal to 1.263.

The heat transfer rate from state 1 to state 2, Q subscript 12, is equal to the change in energy plus the work done from state 1 to state 2, which equals the mass at state g times the difference in temperature between state 2 and g times the specific heat at constant volume times the volume. Substituting values, it equals 0.00349 times the difference between 273.15 Kelvin and 773.15 Kelvin times 0.633 kilojoules per kilogram Kelvin plus 316.12 kilojoules per kilogram, resulting in -315.12 kilojoules per kilogram.