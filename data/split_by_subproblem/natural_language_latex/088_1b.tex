The average temperature T-bar subscript KF equals the integral from s-subscript-in to s-subscript-out of T ds, divided by s-subscript-in minus s-subscript-out.

For an isobaric process, the average temperature T-bar subscript KF equals h-subscript-out minus h-subscript-in, divided by s-subscript-out minus s-subscript-in.

For an ideal fluid, the average temperature T-bar subscript KF equals the integral from T1 to T2 of c-subscript-p dT plus s times T2 minus T1, divided by c-subscript-p dT.

This simplifies to c-subscript-p times T2 minus T1, divided by c-subscript-p times the natural logarithm of T2 over T1.

Further simplifying, it becomes T2 minus T1, divided by the natural logarithm of T2 over T1.

Finally, the average temperature T-bar subscript KF equals T2 minus T1, divided by the natural logarithm of T2 over T1, which equals 293.12 Kelvin.