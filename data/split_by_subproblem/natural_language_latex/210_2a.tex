Graph 1: The first graph is a plot with the vertical axis labeled T in Kelvin and the horizontal axis labeled S in kilojoules per kilogram Kelvin. The graph contains several curves:
- A blue curve that starts from the origin, rises to a peak, and then falls back down symmetrically.
- Several intersecting lines of different slopes crossing through the blue curve.

Graph 2: The second graph is a plot with the vertical axis labeled T in Kelvin and the horizontal axis labeled S in kilojoules per kilogram Kelvin. The graph contains several curves and points:
- A series of curves that form a zigzag pattern, labeled with numbers 1 through 6 at each turning point.
- A line labeled P0 that intersects the zigzag pattern.
- Two arrows indicating different processes:
  - An arrow labeled s equals constant pointing horizontally to the right.
  - An arrow labeled s not equal to constant pointing upwards and to the right.

Below the second graph, there is a handwritten note:
"State 4 is not necessarily on the ps isobar."