The equations provided are as follows:

1. The input temperature \( T_{\text{ein}} \) is 70 degrees Celsius.
2. The output temperature \( T_{\text{aus}} \) is 100 degrees Celsius.
3. The routine temperature \( T_{\text{rout}} \) is 100 degrees Celsius.

4. The initial temperature \( T_{\text{kein}} \) is 298.15 Kelvin.
5. The final temperature \( T_{\text{kaus}} \) is 286.15 Kelvin.

6. The total mass \( m_{\text{ges}} \) is 5755 micrograms.
7. The fraction \( x_D \) is 0.005, which is the ratio of \( m_D \) to \( m_{\text{ges}} \).
8. The heat rate \( \dot{Q}_R \) is 10 kilowatts.

9. The unknown heat rate \( \dot{Q}_{\text{aus}} \) is to be determined.

10. A statistical expression is given as \( \left[ \text{stat.} \frac{2x}{m} \right] \).

11. The balance equation is given as:
   \[
   0 = \dot{m}_{\text{ein}} (h_e - h_a) + \dot{m}_{\text{aus}} (h_e - h_a) + \dot{Q}_R + \dot{Q}_{\text{aus}} - \cancel{Z\dot{W}}
   \]
   where \( Z\dot{W} \) is cancelled out.

12. The expression for \( \dot{Q}_{\text{aus}} \) is derived as:
    \[
    \dot{Q}_{\text{aus}} = 2\dot{m}_{\text{ein}} (h_e - h_a) + \dot{Q}_R
    \]

13. Substituting values, \( \dot{Q}_{\text{aus}} \) is calculated as:
    \[
    \dot{Q}_{\text{aus}} = 2 \cdot 0.3 \frac{kg}{s} (292.98 - 419.04) \frac{kJ}{kg} + 10\,\text{kW}
    \]
    which simplifies to:
    \[
    = 26.36\,\text{kW}
    \]