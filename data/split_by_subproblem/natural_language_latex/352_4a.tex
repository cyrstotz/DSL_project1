Description of the graph:

The graph is a pressure (p) versus temperature (T) diagram. The p-axis is vertical and the T-axis is horizontal.

- There is a curve starting from the origin and rising upwards, representing the phase boundary between different states of matter.
- The curve has a point labeled "Triple" where three phases coexist.
- To the left of the curve, the region is labeled "Solid".
- To the right of the curve, the region is labeled "Liquid".
- Below the curve, the region is labeled "Gas".
- There are three points marked on the graph:
  - Point 1 is on the solid side of the curve.
  - Point 2 is on the liquid side of the curve, vertically above point 1.
  - Point 3 is on the gas side of the curve, horizontally to the right of point 2.
- A horizontal arrow labeled "isobar" points from point 2 to point 3.
- A vertical arrow labeled "isochor" points from point 1 to point 2.

List of items:
1. State at the beginning
2. State after (i), that is after isochoric cooling
3. State after (ii), that is after isobaric pressure reduction