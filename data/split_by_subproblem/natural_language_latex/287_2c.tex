Item c) Delta e subscript x, s equals dot m times the quantity h subscript c minus h subscript zero minus T subscript zero times the quantity s subscript c minus s subscript zero plus k subscript e, g minus k subscript e, zero.

In the aligned equations:
1. s subscript c minus s subscript zero equals s superscript o of T subscript g minus s superscript o of T subscript zero minus R times the natural logarithm of the ratio p over p superscript o.
2. s superscript o of T subscript g minus s superscript o of 32 degrees Fahrenheit equals 1.73943 minus 1.78243 times the quantity 32 times 67 degrees Fahrenheit minus 325 over 325.
3. s superscript o of T subscript zero minus s superscript o of 213.15 degrees Kelvin equals crossed out 1.5187 minus 1.67824 times the quantity 213.15 minus 260 over 260 minus 240 plus 1.67824 times the quantity 1711 over 1711.
4. h subscript zero equals h of 213.15 degrees Kelvin equals the fraction 256.65 minus 240.02 over 240 minus 240 times the quantity 243.15 minus 260 plus 240.02 equals 293.21.
5. Delta e subscript x, s equals dot m times the quantity 328.46 minus 243.21 minus 213.15 times the quantity 1.7794 minus 1.4811 minus one half times the quantity 568.26 squared minus 260 squared.