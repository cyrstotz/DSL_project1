Graph Description:
The graph is a plot with the vertical axis labeled P in bars and the horizontal axis labeled T in Kelvin. The graph shows a curve that starts at the origin, rises to a peak labeled "Triple Point", and then descends. There are two points marked on the curve: point (1) is on the ascending part of the curve, and point (2) is on the descending part. An arrow points from point (1) to point (2).

Title: Energy
Subtitle: Energy balance at the compressor

- Step 2: s2 equals s3
- Step 3: s2 equals s3, 8 bars

Graph Description:
The graph is a simple line graph with the vertical axis labeled s and the horizontal axis labeled PT. The line starts at the origin and rises linearly. There is a point marked on the line with x2 equals 1.

The table includes:
- Row 1: P equals 1.2132
- Row 2: T equals -22 degrees Celsius
- Row 3: P equals 8 bars
- Row 4: P equals 8 bars, T equals 51.33, S equals 0.3485, h equals 83.42, x is not specified

- x2 equals 1
- x4 equals 0

Process Descriptions:
- isobar
- isochor
- isotherm
- isobar

Continuation of section a:
The equation is zero equals the mass flow rate of the refrigerant times (h2 minus h3) plus the dot work.

Table A 11:
hf at 8.06 w, x equals 0 is 83.42 kilojoules per kilogram equals h4
This implies T4 equals 31.33 degrees Celsius

Section C:
x2 equals 1, T2 equals -22 degrees Celsius
This refers to Table A 10
P2 equals 1.2492
hg2 equals 234.08 kilojoules per kilogram
sg2 equals 0.8354 kilojoules per kilogram K
This implies isotherm, thus P2 equals P4 equals 1.2492
x4 equals (s minus sf) divided by (sg minus sf)

Problem 1:
Epsilon u equals the absolute value of dot Q2u divided by (the absolute value of dot Q2u minus the absolute value of dot Q1u) equals the absolute value of dot Q2u divided by (T dot W1 plus 1) equals the absolute value of dot Q1u divided by the absolute value of dot Wu1.