d)

The equation m dot subscript 2 times u subscript 2 minus m subscript 1 times u subscript 1 equals the sum over i of m dot subscript i times h subscript i if and only if Q subscript aus,72.

Delta m times h subscript ein equals m subscript 2 times u subscript 2 minus m subscript 1 times u subscript 1 plus Q subscript aus,72.

Delta m times (h subscript ein minus u subscript 2) equals m subscript 1 times (u subscript 2 minus u subscript 1) plus Q subscript aus,72.

Delta m equals 1 over (h subscript ein minus u subscript 2) times (m subscript 1 times (u subscript 2 minus u subscript 1)) plus Q subscript aus,72.

Delta m equals dots equals 3307.3 kilograms.

2)

Delta S equals m subscript 2 times s subscript 2 minus m subscript 1 times s subscript 1.

Equals (5755 kilograms plus 3302 kilograms) times 0.2594 kilojoules per kilogram Kelvin.

Minus 5755 kilograms times 7.306 kilojoules per kilogram Kelvin equals 7768 kilojoules per kilogram.

s subscript 1 equals 7000 kilojoules per kilogram equals 7.306 kilojoules per kilogram Kelvin.

s subscript 2 equals 7000 kilojoules per kilogram equals 0.2594 kilojoules per kilogram Kelvin.

m subscript 2 equals m subscript 1 plus Delta m.