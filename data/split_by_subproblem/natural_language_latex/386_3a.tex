The pressure \( p_{g,1} \) equals the ambient pressure \( p_{amb} \) plus the fraction of \( \frac{m k g}{A} \) plus the fraction of \( \frac{m_{ew} g}{A} \).

The area \( A \) is equal to \( 2 \pi r^2 \) which equals \( 0.03142 \) square meters.

This results in a value of \( 1.1 \) bar.

The product of pressure \( p_1 \) and volume \( V_1 \) equals the product of mass \( m \), gas constant \( R \), and temperature \( T_1 \).

The mass \( m \) is calculated as \( \frac{p_1 V_1}{R T_1} \), which equals \( \frac{1 \) bar times \( 3.1 \) liters divided by \( 8.314 \) joules per mole Kelvin times \( 500 \) degrees Celsius.

This calculation results in a mass of \( 2.687 \) grams.