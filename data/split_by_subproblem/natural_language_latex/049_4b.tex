The rate of work done \( \dot{W}_k \) is equal to the mass flow rate \( \dot{m} \) times the difference in enthalpy between \( h_3 \) and \( h_4 \), which implies that the mass flow rate \( \dot{m} \) is equal to the rate of work done \( \dot{W}_k \) divided by the difference in enthalpy between \( h_3 \) and \( h_4 \).

We now consider \( h_3 \) and \( h_4 \) from the tables A-10, A-11, A-12.

\( h_2 \) can be interpolated from \( h_f \) at \( T_1 = 6 \) degrees Celsius, where \( T_1 \) is found to be \(-20^\circ C\) using the pressure-temperature diagram (Table A-10).

The enthalpy \( h_{2g} \) at \( T = -20^\circ C \) is 231.62 and almost entropy \( s_2 \) at \( T = -20^\circ C \) is 0.8390.

Since the process from \( 2 \rightarrow 3 \) is adiabatic reversible, the change in entropy \( \Delta s \) is zero, implying \( s_2 = s_3 \).

From Table A-12, we find \( h_3 \) at \( s_3 \) to be approximately 279.14, calculated as \( 273.66 + \left( \frac{(298.33 - 273.66)}{(891.11 - 0.8390)} \right) (0.8390 - 0.5930) \).

Thus, the mass flow rate \( \dot{m} \) is equal to \( \frac{\dot{W}_k}{h_3 - h_4} = \frac{2.6}{47.55} = 0.055 \) kilograms per second, which is approximately 2.37 kilograms per hour.

\( T_1 \) was found using the binary and \(-20^\circ C\), implying \( T_2 = -20 - 6 = -26^\circ C \).