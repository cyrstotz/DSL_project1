a) The content includes several mathematical expressions and calculations:

1. A table with two columns. The first column header is \( F_S \) and the second column header is \( T_{P_1} \). Below the headers, the first column contains the number 1 and the second column contains '1 bar'.

2. The equation \( p_1 = p_0 + \frac{M \dot{V}_S}{A} + \frac{\text{MEW} \cdot S}{A} \).

3. The area \( A \) is calculated as \( A = \eta \frac{D^2}{4} = 0.07854 \, m^2 \).

4. The pressure \( p_1 \) is initially calculated and then crossed out as \( 7.03337 \, \text{bar} \), and recalculated as \( 1.0401 \, \text{bar} \) which is repeated for emphasis.

5. The mass \( M_S \) is calculated using the formula \( M_S = \frac{p_S V_S}{R T_S} \).

6. The result of the mass calculation is \( 0.025401 \, kg \) and also expressed as \( 2.5401 \, S \) which is underlined.

7. The gas constant \( R \) is calculated as \( R = \frac{\bar{R}}{M_S} = \frac{8.314 \, \frac{J}{mol \cdot K}}{50.16 \, \frac{g}{mol}} = 166.28 \, \frac{J}{kg \cdot K} \).