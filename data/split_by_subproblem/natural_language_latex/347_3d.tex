Energy balance on the closed system:
The change in internal energy, Delta oE, equals the mass of gas, m_g, times the difference in internal energy from state 1 to state 2, which equals the heat added, Q_12, minus the work done, W_12.
The work done, W_12, equals the integral of pressure p with respect to volume V, which equals the atmospheric pressure, p_a, times the difference in volume from V_1 to V_2, given that the volume of gas at state 2, V_g2, equals the mass of gas, m_g, divided by the pressure at state 2, p_02.
The volume of gas at state 2, V_g2, is 0.000111 cubic meters, which leads to V_g2 equals 1.401 times 10 to the power of 5 Pascals divided by the difference between 0.000111 and 0.00349 cubic meters, resulting in -294.5 Joules.
The difference in internal energy from state 1 to state 2, u_2 minus u_1, equals the specific heat at constant volume, c_v, times the difference in temperature from T_g to T_g2.

First, we need to calculate u_m.
By analogy, for steam: x_m is the same as x_gas, all values are from Table 1 to 2.1 bar.
L_q equals u_f plus x times (u_ice minus u_f,gas) equals (-0.045 plus 0.6 times (-333.859 plus 0.045) kilojoules per kilogram).
u_m equals -200.0828 kilojoules per kilogram.
Energy conservation to the ice: the mass of ice-water, m_ew, times the difference in internal energy from state 1 to state 2, equals the absolute value of Q_12 minus Q_2,EW.
L_q times u_2 equals u_1 plus the absolute value of Q_12 divided by the mass of ice-water, m_ew, equals -200.0828 kilojoules per kilogram plus 1.3675 kilojoules per kilogram divided by 0.1 kilogram equals -186.4 kilojoules per kilogram equals u_2.
Solving for x_2: x_2 equals (u_2 minus u_f) divided by (u_ice minus u_f).
Values in 1.8 bar table, since we are still in steam and p_1 equals p_2, T_1 equals T_2.
x_2 equals (-186.4 plus 0.045) divided by (-333.859 plus 0.045) equals 0.555 equals x_2.