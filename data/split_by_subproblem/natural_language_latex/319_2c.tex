Delta e subscript ex, Stk equals the quantity h subscript 6 minus h subscript 0 minus T subscript 0 times the quantity s subscript 6 minus s subscript 0 plus ke subscript 6 minus ke subscript 0.

This equals c subscript p times the quantity T subscript 6 minus T subscript 0 minus T subscript 0 times the quantity c subscript p times the natural logarithm of the fraction T subscript 6 over T subscript 0 minus R times the natural logarithm of the fraction p subscript 6 raised to the power 0 over p subscript 0 plus one half times w subscript 6 squared minus one half times w subscript 0 squared.

This equals 85.432425 kilojoules per kilogram minus 73.275 kilojoules per kilogram plus one half times w subscript 6 squared minus one half times w subscript 0 squared plus 40.55.55.7.45.

This equals 16.254 kilojoules per kilogram.

The derivative with respect to t of the quantity dot m times e subscript x equals the sum of dot m times e subscript x, ein plus the sum of dot m times e subscript x, aus minus the sum of the quantity dot m times the fraction p raised to the power 0 over rho raised to the power 0 times the derivative of rho raised to the power 0 with respect to t minus dot E subscript x, verl.

Zero equals Delta dot E subscript x, sonst minus dot E subscript x, verl.

Dot E subscript x, verl equals Delta dot E subscript x, sonst.

Dot E subscript x, verl equals Delta e subscript x, sonst.

Dot m equals 16.254 kilojoules.