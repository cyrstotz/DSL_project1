T1 equals negative 10 degrees Celsius.

The internal upper temperature T2 is equal to negative x times 6k minus 10 degrees Celsius.

h1 equals 75 kilojoules per kilogram times A times 10 minus 10 degrees Celsius, leading to hfg equals 217.7 kilojoules per kilogram.

For h2, s1 equals s2 during adiabatic fog formation, leading to s2 equals s1 plus sfg equals 0.92 kilojoules per kilogram Kelvin equals sf.

Interpolate h2 with temperature T or A, leading to s2 equals 0.92 kilojoules per kilogram Kelvin.

s2 equals 0.92 equals 0.68 plus x times (1.92 minus 0.68), resulting in x equals 0.6.

h2 equals 0.6 times 217.7 plus 75.17 equals 207.7 kilojoules per kilogram.

The mass flow rate m dot equals the power Wk dot divided by h2 minus h1, equals negative 20 kilojoules per second divided by 207.7 minus 217.7, resulting in 0.6 kilograms per second.