The mass flow rate of R134a is 4 kilograms per hour, and the temperature T2 is minus 22 degrees Celsius.

The initial temperature Ti is calculated as T2 plus 6 Kelvin, which equals minus 16 degrees Celsius.

The steady flow process from point 4 to point 1 is described by the equation:
Zero equals the mass flow rate of R134a times the difference in enthalpy from point 4 to point n plus the change in kinetic energy from point 4 to point 1 divided by 2 plus the gravitational potential energy difference from point 4 to point 1 plus the heat transfer minus the work done.

From this, it follows that the enthalpy at point 4 equals the enthalpy at point 1.

The enthalpy at point 4, which is the saturated liquid enthalpy at 8 bar, is 93.42 kilojoules per kilogram and is equal to the enthalpy at point n.

The steady flow process from point 1 to point 2 is described by the equation:
Zero equals the mass flow rate times the difference in enthalpy from point 1 to point 2 plus the change in kinetic energy from point 1 to point 2 divided by 2 plus the gravitational potential energy difference from point 1 to point 2 plus the heat transfer minus the work done.

From this, it follows that the heat transfer is equal to the mass flow rate times the difference in enthalpy from point 2 to point 1.

The enthalpy at point 2, which is the saturated vapor enthalpy at minus 22 degrees Celsius, is 234.08 kilojoules per kilogram.

The heat transfer is calculated as the mass flow rate of 4 kilograms per hour converted to seconds, times the difference in enthalpy from point 2 to point 1, resulting in 0.75629 kilowatts or 756.29 watts.

The quality at point n, xn, is the ratio of the difference in enthalpy at point 1 and the saturated liquid enthalpy at point n to the difference between the saturated vapor and saturated liquid enthalpy at point n.

Graphs and Figures:
- Figure 1: A graph with temperature in Kelvin on the x-axis and pressure in bar on the y-axis, featuring a closed loop with points labeled 1, 2, 3, and 4. The path from point 1 to point 2 is marked as "sotrop". The paths from point 2 to 3 and from point 3 to 4 are curved lines, and the path from point 4 to 1 is a horizontal line.
- Figure 2: A graph with temperature in degrees Celsius on the x-axis and pressure in millibar on the y-axis, showing a closed loop with points labeled 1, 2, 3, and 4. The paths from point 1 to 2 and from point 4 to 1 are straight lines, with the latter having a downward arrow indicating a decrease. The paths from point 2 to 3 and from point 3 to 4 are curved lines.