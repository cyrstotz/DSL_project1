The pressure \( p \) is equal to the force \( F \) divided by the area \( A \), which is calculated as the area of the piston divided by the force through \( m = 32 \) kg measured \( 0.1 \) kg.

The area \( A \) is equal to pi times 20 cm times 0.1 m.

The area \( A \) is also calculated as the square of half of 0.1 m plus 0.00785.

The force \( F \) is equal to mass \( m \) times acceleration \( a \), which is the sum of the measured mass and the eternal mass, resulting in 324.9 N.

The gauge pressure \( p_g \) is the ambient pressure \( p_{\text{amb}} \) plus the force divided by the area, which equals 740 times 7744 Pa.

The product of pressure \( p \) and volume \( V \) equals mass \( m \) times the gas constant \( R \) times temperature \( T \).

The mass \( m \) is calculated as the product of pressure \( p \) and volume \( V \) divided by the product of the gas constant \( R \) and temperature \( T \), resulting in 3.422 kg.

The gas constant \( R \) is the pressure \( p \) divided by the mass \( m \), equaling 0.16628.

The volume \( V \) is 3.14 L, which is also 3.14 dm³ or 0.00314 m³.