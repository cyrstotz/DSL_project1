a) Required: \(P_{g,1}\), \(m_g\)

The universal gas constant \(R\) is calculated as follows:
\[ R = \frac{\bar{R}}{M} = \frac{8.314 \, \text{kilojoules per kilomole Kelvin}}{50 \, \text{kilograms per kilomole}} = 0.1663 \, \text{kilojoules per kilogram Kelvin} \]

The specific heat capacity at constant volume \(C_v\) is:
\[ C_v = 0.633 \, \text{kilojoules per kilogram Kelvin} \]

The specific heat capacity at constant pressure \(C_p\) is:
\[ C_p = R + C_v = 0.7993 \, \text{kilojoules per kilogram Kelvin} \]

The pressure \(P_{g,1}\) is described as the pressure from above:
\[ \Rightarrow P_{g,1} = 1 \, \text{bar} + g \cdot \frac{(m_u + m_{EW})}{A} \quad \text{where} \quad A = (10 \, \text{centimeters})^2 \cdot \pi \]

Solving for \(P_{g,1}\) gives:
\[ \Leftrightarrow P_{g,1} = 1 \, \text{bar} + 9.81 \, \text{meters per second squared} \cdot \frac{(32 \, \text{kilograms} + 0.1 \, \text{kilograms})}{\frac{\pi}{100} \, \text{square meters}} = 1 \, \text{bar} + 0.1 \, \text{bar} = 1.1 \, \text{bar} \]

Since the gas is perfect, the following equation holds:
\[ pV = mRT \]

Therefore, the mass \(m_g\) of the gas is calculated as:
\[ m_g = \frac{P_{g,1} \cdot V_{g,1}}{R \cdot T_{g,1}} = \frac{1.1 \, \text{bar} \cdot 3.14 \, \text{liters}}{0.1663 \, \text{kilojoules per kilogram Kelvin} \cdot 500^\circ \text{Celsius}} = 2.688 \, \text{grams} \]