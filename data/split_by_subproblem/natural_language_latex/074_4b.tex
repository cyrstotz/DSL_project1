Subsection b) Cooling Circuit

A table is presented with the following columns and rows:
- The first column header is "State", the second column header is "p", and the third column header is empty.
- The first row contains "1" under State and "p1" under p.
- The second row contains "2" under State and "p1 minus 16 degrees" under p.
- The third row contains "3" under State and "8 bar" under p.
- The fourth row contains "4" under State and "8 bar" under p.

Additional notes:
- The second row is circled.
- There is an arrow pointing from the second row to the right, labeled as "h2".
- Another arrow points from the second row to the right, labeled as "x2 equals 1".
- An arrow points from the fourth row to the right, labeled as "x4 equals 0, h4".

The temperature Ti is given as minus 20 degrees Celsius (from the diagram).

In the equations section:
- The enthalpy at state 1 (8 bar, x1 equals 0) is initially marked as 264.45 kilojoules per kilogram but is corrected to 93.42 kilojoules per kilogram, denoted as h1.
- The enthalpy at state 2 (minus 16 degrees Celsius, x2 equals 1) is initially marked as 2401.15 kilojoules per kilogram but is corrected to 93.92 kilojoules per kilogram.

The temperature T1 is calculated as minus 20 degrees Celsius minus 0 degrees Celsius, resulting in minus 10 degrees Celsius.

Subsection Stationary Flow Process:
- The equation is given as zero equals m dot times (he minus ha) plus Q dot minus W dot.
- Another equation is given as zero equals m times (ha minus h2).