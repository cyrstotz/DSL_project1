e_k equals the ratio of Q dot k to the absolute difference between the absolute values of Q dot 20 and Q dot k20.

Theta equals m dot 2 times the difference between h4 and h3 minus Q dot 20, which implies that Q dot 20 equals 0.1892 kilowatts. h4 equals 0.342 kilojoules per kilogram. h3 equals h_g at 8 bar, which is 260.15 kilojoules per kilogram. Q dot k equals m dot times the difference between h2 and h1.

The heat transfer Q equals m_f times the difference between h_f2 and h_f1.

Adiabatically, h1 equals h_f at 8 bar, which is 934.2 kilojoules per kilogram, and this is marked with a circled AM.

h_f1 equals x_f times the difference between h_g and h_f.

p1 equals p2, and at minus 22 degrees Celsius, h_f equals 233.08 kilojoules per kilogram, and this is marked with a circled AM.

1.2192 kilograms per kilogram, marked with a circled 410.

h_f equals the ratio of 25.77 minus 9.32 to 1.4 minus 1.2, which simplifies to the ratio of 1.2192 minus 1.2 plus 21.32 to 1.4 minus 1.2, resulting in 21.2472 kilojoules per kilogram.

h_g equals the ratio of 236.04 minus 233.86 to 1.4 minus 1.2, which simplifies to the ratio of 1.2192 minus 1.2 plus 233.86 to 1.4 minus 1.2, resulting in 233.069 kilojoules per kilogram.

x_f equals 0.358.