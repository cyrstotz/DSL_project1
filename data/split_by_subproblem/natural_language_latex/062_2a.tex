The change in enthalpy, \( h_f - h_0 \), is equal to the integral from initial temperature \( T_0 \) to final temperature \( T_f \) of \( c_p \, dT \), which simplifies to \( c_p (T_f - T_0) \). Substituting the values, \( c_p = 1.006 \, \frac{\text{kJ}}{\text{kg K}} \) and \( T_f = 431.9 \, \text{K} \), \( T_0 = 288.07 \, \text{K} \), the result is \( 107.5 \, \frac{\text{kJ}}{\text{kg}} \).

The graph is a plot with the x-axis labeled as entropy per unit mass per Kelvin, and the y-axis labeled as \( T_{XY} \). It features three curves, each representing different values of pressure \( P \). The curves are labeled as follows:
- The first curve is labeled \( P_0 \).
- The second curve is labeled \( P_1 = P_0 \).
- The third curve is labeled \( P_2 = P_1 \).

Arrows point upwards along the curves, indicating the direction of increasing values. The points on the curves are marked with numbers 2, 4, and 6, respectively.