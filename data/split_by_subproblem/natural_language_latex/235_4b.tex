b)

T2 equals Ti minus 6 Kelvin, which equals 257.15 Kelvin, which is equivalent to negative 16 degrees Celsius.

TAB - 10:
h2 equals the heat of vaporization at T2, which is 238.74 kilojoules per kilogram.
s2 equals the entropy at T2, which is 0.9298 kilojoules per kilogram Kelvin, and this is equal to s3.
h3 equals the enthalpy at 8 bar and entropy s3, which is calculated as the enthalpy at 8 bar and 0.9066 kilojoules per kilogram Kelvin plus the ratio of the difference in entropy s3 minus 0.9066 kilojoules per kilogram Kelvin over the difference 0.9374 minus 0.9066 kilojoules per kilogram Kelvin, multiplied by the heat of vaporization at 8 bar and 0.9374 kilojoules per kilogram Kelvin.
The enthalpy at 8 bar and 0.9066 kilojoules per kilogram Kelvin equals 271.31 kilojoules per kilogram.

TAB - A 12:
The heat of vaporization at 8 bar and 0.9066 kilojoules per kilogram Kelvin equals 284.15 kilojoules per kilogram.
The heat of vaporization at 8 bar and 0.9374 kilojoules per kilogram Kelvin equals 273.66 kilojoules per kilogram.

Energy Balance Compressor

The rate of change of energy with respect to time equals the sum of the mass flow rate times the sum of enthalpy, half the velocity squared, and the product of gravitational acceleration and height, plus the sum of heat transfer rates minus the sum of work rates.

Adiabatic

Zero equals the mass flow rate times the difference in enthalpy from h2 to h3 minus the work rate from state 2 to state 3.

The work rate from state 2 to state 3 equals the negative of the compressor work rate.

The mass flow rate equals the negative of the compressor work rate divided by the difference in enthalpy from h2 to h3, which equals 2.89 kilograms per hour.