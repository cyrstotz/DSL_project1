The first graph described is a T-s diagram, where T (temperature) is on the vertical axis labeled in Kelvin (K) and s (entropy) is on the horizontal axis labeled in kilojoules per kilogram Kelvin [kJ/kg·K]. This graph includes several isobars and isotherms. The isobars are curved lines, while the isotherms are horizontal lines. Points labeled 0, 1, 2, 3, 4, and 5 are marked on the graph, with arrows showing the direction of the process. The process sequence starts at point 0, then moves sequentially through points 1, 2, 3, 4, and ends at point 5. Each isobar is labeled with its respective pressure, and each isotherm is labeled with its respective temperature.

The second graph is also a T-s diagram with T on the vertical axis labeled in Kelvin (K) and s on the horizontal axis labeled in kilojoules per kilogram Kelvin [kJ/kg·K]. This graph similarly includes several isobars and isotherms. Points 0, 1, 2, 3, 4, and 5 are marked on this graph as well, with arrows indicating the direction of the process. The process starts at point 0 and progresses through points 1, 2, 3, 4, and concludes at point 5. Each isobar is labeled with its respective pressure, and each isotherm is labeled with its respective temperature. Additionally, this graph includes a label indicating an isochore process.