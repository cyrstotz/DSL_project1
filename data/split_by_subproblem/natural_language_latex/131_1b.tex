From point 5 to point 6, the energy equation is used. The rate of change of energy with respect to time is equal to the change in energy, which is given by the mass flow rate times the sum of enthalpy plus half the velocity squared plus the sum of heat transfer per unit mass flow rate minus the sum of work per unit mass flow rate. The equation p times v equals m times R times T is stated. The initial temperature T0 is given by the initial pressure p0 times initial volume v0 divided by the mass m times the specific heat capacity c times the gas constant R. The mass of the component M_c is equal to the mass of the substance m_s. The equation for the component conditions divided by the gas constant and the component temperature equals the substance conditions divided by the gas constant and the substance temperature, which implies that the component temperature T_c is equal to the component pressure p_c times component volume V_c divided by the substance pressure p_s times substance volume V_s times the substance temperature T_s.