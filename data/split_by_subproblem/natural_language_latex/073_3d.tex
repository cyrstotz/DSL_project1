d) First equation around ice:
The rate of change of energy with respect to time is equal to the sum over i of the mass flow rate of i times the sum of the enthalpy of i, the ratio of pressure of i to density of i, and half the velocity of i squared, plus the rate of heat transfer minus the rate of volume flow.

Change in internal energy is equal to the heat transfer from state 1 to state 2.

The product of the mass of melted water and the difference in specific internal energy between state 2 and state 1 equals the heat transfer from state 1 to state 2.

The difference in specific internal energy between state 2 and state 1 equals the ratio of the heat transfer from state 1 to state 2 to the mass of melted water, leading to the specific internal energy at state 2 being the sum of the specific internal energy at state 1 and the ratio of the heat transfer from state 1 to state 2 to the mass of melted water.

The specific internal energy at state 1 equals the sum of the residual specific internal energy and the product of the ice fraction at state 1 and the difference between the melting specific internal energy and the residual specific internal energy.

This equals negative 333.485 plus 0.6 times negative 0.045 times 333.485 kilojoules per kilogram.

This results in negative 133.421 kilojoules per kilogram.

The specific internal energy at state 2 equals negative 133.421 kilojoules per kilogram plus the ratio of 1.5 kilojoules to 0.1 kilograms, resulting in negative 118.421 kilojoules per kilogram.

The temperature of the ice at state 2 equals the temperature of the ice at state 1 plus 0.003 degrees Celsius.

The ice fraction at state 2 equals the ratio of the difference between the specific internal energy at state 2 and the residual specific internal energy to the difference between the melting specific internal energy and the residual specific internal energy, which calculates to 0.64.