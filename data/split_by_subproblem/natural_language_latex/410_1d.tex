The change in energy, denoted as Delta E, is equal to the input smash plus heat input minus heat output minus the square of U2.

The equation m2 times v2 minus m1 times v1 equals the input smash, with m2 being equal to m1 plus Delta m.

Expanding this, (m1 plus Delta m) times v2 minus m1 times v1 equals the input smash.

Further simplifying, m2 times v2 plus Delta m times v2 minus m1 times v1 equals the input smash.

The input smash is then defined as m2 times v2 minus m1 times v1, which simplifies to m1 times (v2 minus v1).

Delta m is calculated as m1 times (v2 minus v1) divided by (v_ein minus v2).

Delta m is calculated to be 5755 kilograms times (v2 minus v1) divided by (v_ein minus v2), resulting in 3469.9 kilograms.

Delta m12 is stated as 3469.9 kilograms.

In the table:
- U2 at 70 degrees Celsius is 292.95 kilojoules per kilogram.
- U1 at 100 degrees Celsius is 418.94 kilojoules per kilogram.
- h_ein at 20 degrees Celsius is 83.96 kilojoules per kilogram.
- m1 is 5755 kilograms.