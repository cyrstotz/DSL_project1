Zero equals the mass flow rate times the difference in enthalpy between the exit and the entrance plus half the difference in the squares of the velocity at the exit and the entrance, plus the heat transfer rate minus the work rate.

This implies that the heat transfer rate equals the mass flow rate times the specific heat capacity at constant pressure times the temperature difference between the exit and the entrance plus half the square of the velocity at the exit.

The difference in enthalpy between the exit and the entrance equals the integral from the entrance temperature to the exit temperature of the specific heat capacity with respect to temperature, which simplifies to the specific heat capacity times the temperature difference between the exit and the entrance.

The term "adiabat" is mentioned.

The ratio of the stagnation temperature to the static temperature equals the ratio of the exit pressure to the entrance pressure raised to the power of (kappa minus one divided by kappa). This implies that the static temperature equals 184.9 times 0.191 raised to the power of 0.4 over 1.4, resulting in 323.075 Kelvin.

The square of the velocity at the exit equals two times the sum of the specific heat capacity times the temperature difference between two points plus half the square of the velocity at the exit, which equals 257.28 Kelvin per kilogram.

The velocity at the exit equals 507.24 meters per second.