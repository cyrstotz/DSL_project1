b)

A table with columns labeled phi, p, T, h, and s, containing the following entries:
- Row 1: phi equals 1, p equals 1.3748, T equals -16 degrees Celsius, h and s are blank.
- Row 2: x equals 1, p equals 2, p1, T equals -16 degrees Celsius, h and s are blank.
- Row 3: phi equals 3, p equals 8, T, h, and s are blank.
- Row 4: x equals 0, p equals 4, T equals 8 degrees Celsius, h and s are blank.

The temperature T1,2 is calculated as Ti minus 6K, resulting in -16 degrees Celsius.

The enthalpy h2 is equal to hg at -16 degrees Celsius, referring to table A10.

The enthalpy h2 is 237.74 kilojoules per kilogram.

The entropy s2 equals s3, which is sg at -16 degrees Celsius, and s2 is 0.9288 kilojoules per kilogram Kelvin.

The pressure p3 is 8 bar, and from table A-M, it is observed that we are in the vapor region, referring to table A-12.

The enthalpy h3 is calculated using the formula: h_sat plus the fraction (h at 40 degrees Celsius minus h_sat) divided by (s at 40 degrees Celsius minus s_sat) times (s3 minus s_sat).

The values are:
- h at 40 degrees Celsius is 273.66.
- h_sat is 269.45.
- s_sat is 0.9066.
- s at 40 degrees Celsius is 0.9374.

The calculated h3 is 271.3 kilojoules per kilogram.

The enthalpies h2 and h3 are 237.74 and 271.3 kilojoules per kilogram, respectively.

From state 2 to state 3, applying the first law of thermodynamics:
- The equation 0 equals m dot times (h2 minus h3) plus Q dot minus W dot.
- The mass flow rate m dot equals W dot divided by (h2 minus h3).

The process is adiabatic, and the work W dot is -28 kilowatts per kilogram.

The mass flow rate m dot is 0.884 grams per second.

The final mass flow rate m dot is 0.884 grams per second.