The rate of change of kinetic energy, denoted as dot E subscript kf, is equal to the integral from a to e of T ds divided by the difference s_a minus s_e. This is exemplified by the formula (T_aus minus T_ein) divided by the natural logarithm of (T_aus divided by T_ein), which equals 293.12 Kelvin.

The difference s_a minus s_e equals c_l times the natural logarithm of (T_aus divided by T_ein), assuming an ideal fluid and isobaric conditions.

The integral from a to e of T ds, denoted as q_zu, equals h_aus minus h_ein, according to the first law of thermodynamics.

This is further simplified to c_l times (T_aus minus T_ein), assuming an ideal fluid and isobaric conditions.

The temperature T_aus is given as 298.15 Kelvin.

The temperature T_ein is given as 288.15 Kelvin.