Gas: \(P_6, O_6\)

The specific work \(w_s\) is 220 Joules per kilogram.

The critical pressure \(P_c\) is 0.5 bar.

The temperature \(T_5\) is 431.9 Kelvin.

For an adiabatic process:
The ratio of temperatures \(T_2\) over \(T_1\) equals the ratio of pressures \(P_2\) over \(P_1\) raised to the power of \((n-1)/n\). Therefore, the ratio \(T_c\) over \(T_5\) equals the ratio \(P_c\) over \(P_5\) raised to the power of \((n-1)/n\).

This implies that \(T_6\) equals 431.9 times the ratio of \(0.191 \times 10^5\) over \(0.5 \times 10^5\) raised to the power of \((1.4-1)/1.4\), which calculates to 328.07 Kelvin, approximately 328.1 Kelvin.

The work \(w_0\) equals \(h_c\) minus half of \(w^2\).

For a stationary flow process:
Zero equals the mass flow rate times the bracket of \(h_c\) minus \(h_1\) plus half the difference of \(w_c^2\) minus \(w_1^2\), which implies \(h_c\) minus \(h_1\) plus half the difference of \(w_c^2\) minus \(w_1^2\) equals zero.

\(h_c\) plus the product of specific heat at constant pressure \(c_p\) and the difference \(T_c\) minus \(T_1\) equals \(c_p\) times the difference of 328.1 minus 431.9, which calculates to -104.1 times 2728 Joules per kilogram Kelvin.

Zero equals half the difference of \(w_c^2\) minus \(w_1^2\) minus 104.1 times 2728.

This implies that twice the product of 104.1 and 2728 equals \(w_1^2\) minus \(w_c^2\).

Finally, \(w_1^2\) equals 760.444 minus 400.55 meters per second.