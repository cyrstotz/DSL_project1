The rate of heat transfer, denoted as Q dot, is equal to the mass flow rate of the cooling fluid, denoted as m dot subscript KF, multiplied by the difference in entropy at the exit and the entrance, denoted as s subscript e minus s subscript a, plus the rate of heat loss, denoted as Q dot subscript aus, divided by the average temperature, denoted as T bar, plus the generated entropy rate, denoted as s dot subscript erz.

This implies that Q dot equals 6.33 kilowatts per Kelvin times the natural logarithm of the ratio of the inlet temperature, T subscript ein, to the outlet temperature, T subscript aus, plus 0.33 kilowatts plus 293.12 Kelvin plus the generated entropy rate, s dot subscript erz.

Substituting the values, Q dot equals 6.33 kilowatts per Kelvin times the natural logarithm of the ratio of 288.15 Kelvin to 278.15 Kelvin, plus 6.33 kilowatts plus 293.12 Kelvin plus the generated entropy rate, s dot subscript erz.

This leads to the conclusion that the generated entropy rate, s dot subscript erz, equals zero because there is no pressure drop, indicated by the phrase "da kein Druckabfall" which translates to "since there is no pressure drop."