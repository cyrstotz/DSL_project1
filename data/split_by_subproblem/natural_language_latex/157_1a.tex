The mass flow rate of water, denoted as m-dot subscript w, equals 0.3 kilograms per second.

The total mass, denoted as m subscript ges, equals 5755 kilograms.

The equation zero equals m-dot subscript w times (h subscript ein minus h subscript aus) plus Q-dot subscript e plus Q-dot subscript aus minus zero (where the zero term is cancelled).

The heat flow rate out, denoted as Q-dot subscript aus, equals negative Q-dot subscript e plus m-dot subscript w times (h subscript aus minus h subscript ein).

The enthalpy at the inlet, denoted as h subscript ein, equals h subscript f at 70 degrees Celsius plus 0.005 times (h subscript fg at 70 degrees Celsius minus h subscript f at 70 degrees Celsius), which equals 304.65 kilojoules per kilogram.

The enthalpy at the outlet, denoted as h subscript aus, equals h subscript f at 100 degrees Celsius plus 0.005 times h subscript fg at 100 degrees Celsius, which equals 430.825 kilojoules per kilogram.

Therefore, the heat flow rate out, denoted as Q-dot subscript aus, equals negative 62.2972 kilojoules per second, which is approximately negative 62.3 kilowatts.