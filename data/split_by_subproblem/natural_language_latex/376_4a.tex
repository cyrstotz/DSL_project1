The mass flow rate of the refrigerant times the difference in enthalpy between state 2 and state 3 equals the compressor work, represented by the equation:
m-dot subscript R times (h subscript 2 minus h subscript 3) equals W-dot subscript K.

x subscript 2 equals 1.

The enthalpy at state 2, h subscript 2, equals the saturation enthalpy, h subscript g, at 10 degrees Celsius.

Interpolation is performed between 8 degrees Celsius and 12 degrees Celsius.

The calculation is as follows: (259.03 minus 257.8) divided by (12 minus 8) times (10 degrees Celsius minus 8 degrees Celsius) plus 257.8 equals h subscript 2 equals 258.396 kilojoules per kilogram.

h subscript 3 corresponds to a pressure of 8 bar.

The temperature T subscript i is less than 4 degrees Celsius.

h subscript 3 is not specified further.

The mass flow rate of the refrigerant for refrigerant 137a equals the compressor work divided by the difference in enthalpy between state 2 and state 3, represented by the equation:
m-dot subscript R, 137a equals W-dot subscript K divided by (h subscript 2 minus h subscript 3).

Graph Description:
There is a graph with two axes. The horizontal axis is labeled T and the vertical axis is labeled h. Both axes have arrows pointing in the positive direction. The graph does not contain any plotted data or additional markings.