First, the pressure in the gas container must be calculated:
Initial pressure + Pressure from the weight + Ambient pressure

The pressure due to the load (P_L) is calculated as:
P_L equals F over A, which equals (32 kg times 9.81 meters per second squared) divided by (pi times (5 times 10 to the power of negative 2 meters) squared) equals 39.96 Newtons per square meter.

The initial pressure (P_EW) is calculated as:
P_EW equals F over A, which equals (0.74 kg times 9.81 meters per second squared) divided by (pi times (5 times 10 to the power of negative 2 meters) squared) equals 724.304 Newtons per square meter.

The ambient pressure (P_amb) is:
P_amb equals 10 to the power of 5 Newtons per square meter.

The total pressure at state 1 (P_g,1) is:
P_g,1 equals P_L plus P_EW plus P_amb equals 740059.44 Newtons per square meter.

The gas constant (R) is calculated as:
R equals the universal gas constant (R) over M, which equals (8.314 Joules per mole Kelvin) divided by (50 grams per mole) equals 0.16628 Joules per gram Kelvin.

The pressure at state 1 (P_g,1) is:
P_g,1 equals 1.4 bar.

The mass of the gas (m_g) is calculated as:
m_g equals (P_g,1 times V_g,1) divided by (R times T_g,1), which equals (1.4 times 10 to the power of 5 Newtons per square meter times 3.14 times 10 to the power of negative 3 cubic meters) divided by (0.16628 Joules per gram Kelvin times 773.15 Kelvin) equals 3.419 grams.

The ice-water is because the water molecules are in the solid-liquid region or an isotherm until all ice (solid) has melted. Thus, the initial temperature of the ice-water (T_EW,1) is still 0 degrees Celsius. Therefore, the gas temperature (T_g,2) must also be 0 degrees Celsius, maintaining thermodynamic equilibrium.

T_g,2 equals 0 degrees Celsius.

The masses acting on the gas are also the same. Thus, p_g,2 equals 1.46 bar.

p_g,2 equals (m_g,1 divided by V_g,1) times (T_g,2 over T_g,1) times (V_g,2 over V_g,2).