The specific gas constant, R_g, is calculated as R divided by M_g, which equals 266.28 Joules per kilogram Kelvin.

The area, A, is calculated using the formula for the area of a circle, pi times D squared divided by 4, which equals 0.0078 square meters.

The term "Kräfte-GGW" translates to "Force Equilibrium".

The force equilibrium equation is given by the pressure at state 1, p_{g,1}, times the area, A, equals the mass of the body, m_k, times the acceleration due to gravity, g, plus the ambient pressure, p_{amb}, times the area, A, plus the mass of the additional weight, m_{ew}, times the acceleration due to gravity, g.

Solving for p_{g,1}, the pressure at state 1, it is calculated as (m_k times g plus p_{amb} times A plus m_{ew} times g) divided by A, which equals 1.4 bar.

The mass of the gas, m_g, is calculated using the ideal gas law rearranged for mass, which is (p_{g,1} times the volume at state 1, V_{g,1}) divided by (R_g times the temperature at state 1, T_{g,1}), resulting in 3.42 grams.