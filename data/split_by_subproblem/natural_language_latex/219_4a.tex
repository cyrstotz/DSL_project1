The first graph is a Pressure-Temperature (P-T) diagram with the following details:
- The x-axis is labeled T.
- The y-axis is labeled P.
- There are four points labeled 1, 2, 3, and 4 along the x-axis.
- The y-axis has two points labeled 8 bar and 0 bar.
- There is a curve starting from the bottom left, rising to a peak, and then descending towards the right. This curve is labeled "subcooled liquid" in red.
- Another curve starts from the bottom left, intersects the first curve at a point labeled "triple point" in red, and then continues upwards to the right. This curve is labeled "superheated steam" in red.
- The area between these two curves is labeled "wet steam region" in red.

The second graph is a Pressure-Temperature (P-T) diagram with the following details:
- The x-axis is labeled T.
- The y-axis is labeled P.
- There are three regions labeled "solid," "liquid," and "gas."
- The "solid" region is at the bottom left, the "liquid" region is at the top, and the "gas" region is at the bottom right.
- The boundary between the "solid" and "liquid" regions is a straight line rising from the bottom left to the top right.
- The boundary between the "liquid" and "gas" regions is a horizontal line starting from the "triple point" and extending to the right.
- The boundary between the "solid" and "gas" regions is a straight line rising from the bottom left to the top right, intersecting the "triple point."