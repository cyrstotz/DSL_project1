The equations provided are as follows:

1. The pressure \( p_{3.1} \) is equal to the external pressure \( p_e \) plus the ratio of the product of mass \( m_1 \) and gravity \( g \) to the area \( A \), minus the ratio of the product of the equivalent mass \( m_{eq} \) and gravity \( g \) to the area \( A \).

2. The area \( A \) is defined as \( \pi \) times the square of half the diameter \( D \), which simplifies to \( \pi \) times the square of \( 0.05 \) meters, resulting in an area of \( 0.0079 \) square meters.

3. Substituting the values into the equation for \( p_{3.1} \), it is calculated as \( 100 \) kilopascals plus the ratio of \( 32 \) kilograms times \( 9.81 \) meters per second squared to \( 0.0079 \) square meters, and the ratio of \( 0.1 \) kilograms times \( 9.81 \) meters per second squared to \( 0.0079 \) square meters, resulting in a pressure of \( 1.4 \) bar.

4. The ideal gas law is represented as \( pV = mRT \).

5. The mass \( m_2 \) is calculated using the modified ideal gas law equation, where \( p_{3.1} \) times the volume \( V_{3.1} \) equals the mass \( m_2 \) times the ratio of the gas constant \( R \) to the molar mass \( M_g \) times the temperature \( T_{3.1} \). Solving for \( m_2 \) gives \( 3.422 \) grams or \( 0.00342 \) kilograms.

6. The gas constant \( R \) is defined as the universal gas constant \( \bar{R} \) divided by the molar mass \( M_g \), which equals \( 0.166 \) kilojoules per kilogram Kelvin.

7. The volume \( V_{3.1} \) is given as \( 3.14 \) liters, which is equivalent to \( 0.00314 \) cubic meters.