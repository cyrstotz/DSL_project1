c) Increase in flow energy:

The difference in flow energy between states x and r is given by the equation:
e subscript x, stro minus e subscript r, stro equals v subscript r times (s subscript r minus s subscript 6) plus Delta e subscript ke. This implies that e subscript x, stro minus e subscript r, stro equals the expression in square brackets: h subscript 6 minus h subscript 0 minus T subscript 0 times (s subscript 6 minus s subscript 0) plus w subscript 6 squared over 2 minus w subscript 0 squared over 2.

The difference in entropy between states 6 and 0 is given by the integral from T subscript 0 to T of c subscript p over T dT equals R times the natural logarithm of (p subscript 6 over p subscript 0). This implies that s subscript 6 minus s subscript 0 equals the integral from T subscript 0 to T subscript 0 of c subscript p over T dT equals c subscript p times the natural logarithm of (T subscript 0 over T subscript 0) equals 0.30 kilojoules per kilogram.

The difference in entropy between states 0 and 6 is given by c subscript p times the natural logarithm of (T subscript 0 over T subscript 6) equals negative 0.30 kilojoules per kilogram.

The difference in kinetic energy terms is given by w subscript 6 squared over 2 minus w subscript 0 squared over 2 equals w subscript 6 squared over 2 minus w subscript 0 squared over 2 equals 10.8 times 10 to the power of 3 kilojoules per kilogram.

The difference in enthalpy between states 0 and 6 is given by c subscript p times (T subscript 0 minus T subscript 6) equals negative 95.43 kilojoules per kilogram. This implies that h subscript 6 minus h subscript 0 equals 95.43 kilojoules per kilogram.

Total: The total difference in flow energy between states x and r is given by e subscript x, stro minus e subscript r, stro equals 95.43 kilojoules per kilogram minus 243.15 times 0.30 minus 10.8 times 653 equals negative 96.41 kilojoules per kilogram.