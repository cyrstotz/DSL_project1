The rate of heat transfer, Q dot, equals the mass flow rate, m dot, times the difference in enthalpy between h subscript a and h subscript e.

The difference in enthalpy, h subscript a minus h subscript e, equals the specific heat capacity at constant pressure, C subscript p, times the difference in temperature between the exhaust temperature, T subscript exaus, and the entry temperature, T subscript e, ein.

For problem 9, part a:

The product of the total mass, m subscript ges, the specific heat capacity, c subscript p, and the difference in temperatures, T subscript 1 minus T, equals the product of the mass flow rate between state 1 and 2, m dot subscript 12, the specific heat capacity, c subscript p, and the ratio of the temperature difference, T subscript 1 minus T, to the difference in temperatures, T subscript 2 minus T subscript 2.

The temperature T subscript 1 is 1000 degrees Celsius.

The temperature T is 70 degrees Celsius.

The temperature T subscript 2, also denoted as T subscript ein, is 200 degrees Celsius.

The mass flow rate between state 1 and 2, m dot subscript 12, equals the ratio of the product of the total mass, m subscript ges, and the temperature difference, T subscript 1 minus T, to the difference in temperatures, T subscript 2 minus T subscript 2, which results in 34.53 kilograms.