The rate of extended energy, denoted as E-dot with subscript x,verl, equals the average temperature T-bar with subscript KF, multiplied by the rate of ore supply S-dot with subscript erz. This implies that the rate of ore supply S-dot with subscript erz equals the rate of extended energy E-dot with subscript x,verl divided by the average temperature T-bar with subscript KF.

The rate of extended energy E-dot with subscript x,verl equals the mass flow rate m-dot multiplied by the expression in brackets: the enthalpy at the inlet h with subscript e,in minus the enthalpy at the outlet h with subscript e,us minus the average temperature T-bar with subscript KF multiplied by the difference in entropy at the inlet and outlet (s with subscript e,in minus s with subscript e,us), plus the extended energy due to heat E-dot with subscript x,Q.