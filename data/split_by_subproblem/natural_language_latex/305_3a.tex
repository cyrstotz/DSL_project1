The pressure at g1, denoted as p_{g1}, is equal to the ambient pressure, p_{amb}, plus the pressure due to the piston, p_{Kolben}. The pressure due to the piston, p_{Kolben}, is calculated as 32 kilograms times 9.81 meters per second squared, divided by the square of 0.05 meters times pi, which equals 35.9635 kiloPascals. Therefore, p_{g1} equals 100 kiloPascals plus 35.9635 kiloPascals, which results in 135.97 kiloPascals.

The mass at g1, denoted as m_{g1}, is calculated as the volume at g1, V_{g1}, times the pressure at g1, p_{g1}, divided by the gas constant, R, times the temperature at g1, T_{g1}. This is calculated as 0.00314 cubic meters times 135.97 kiloPascals, divided by 8.314 kilojoules per kilomole Kelvin times 50 kilograms per kilomole times 273.15 Kelvin, which results in 3.41 grams.