The graph description states that the graph is a plot with the x-axis labeled as 's' and the y-axis labeled as 'T'. It includes several curves with the origin marked as '0'. There are six points labeled from '0' to '6' along the curves, connected by lines, and annotations such as '0.5 bar' and '0.1971 m'. Points '2', '3', '4', and '5' are connected by arrows indicating transitions between these points. The labels 'm_K' and 'm_n' are also present along the curves.

The equation provided is:
Delta e_x_str equals e_x_str_6 minus e_x_str_0 equals h_6 minus h_0 minus T_0 times (s_6 minus s_0) plus (w_6 squared minus w_0 squared) divided by 2.

This simplifies to:
c_p times [T_0 minus T_6 minus T_0 times the natural logarithm of (T_6 divided by T_0)] plus (w_6 squared minus w_0 squared) divided by 2.

Further simplification gives:
c_p times [340 Kelvin minus 243.15 Kelvin minus 243.15 Kelvin times the natural logarithm of (340 divided by 243.15)] plus (340 squared minus 200 squared) divided by 2.

The result is:
25.5 kilojoules per kilogram.

The graphical content description includes a diagram of a nozzle depicted as a converging-diverging shape with an inlet on the left side labeled as 'r' and an outlet on the right side labeled as 'b'. The flow direction is indicated by an arrow pointing from left to right.

Next to the nozzle diagram, several equations and expressions are provided:
Zero equals dot S_irr not equal to dot m times (s_6 minus s_0) plus dot Q divided by T_0, ad.

s_6 minus s_0 equals c_p times the natural logarithm of (T_6 divided by T_0) minus R times the natural logarithm of (p_6 divided by p_0) equals zero.

T_6 divided by T_0 equals (p_6 divided by p_0) raised to the power of (kappa minus 1 divided by kappa).

T_6 equals T_0 times (p_6 divided by p_0) raised to the power of (kappa minus 1 divided by kappa) times 328.1 Kelvin.

This implies w_6 equals 20.