Phi dot equals Phi dot subscript f plus x times the quantity Phi dot subscript g minus Phi dot subscript f.

x equals the fraction with numerator q minus Phi dot subscript f and denominator Phi dot subscript g minus Phi dot subscript f.

Figure Description:
There is a diagram labeled "Exercise 1b" with the following details:
- A rectangular box with three points labeled "1", "2", and "3".
- Point "1" is on the left side of the box, point "2" is on the top side, and point "3" is on the right side.
- Inside the box, there is a circular shape with an arrow pointing from the center to the top right, labeled "adiabatic".
- The box is labeled "Sas".