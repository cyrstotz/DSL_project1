The equation for the heat output, denoted as Q dot subscript "aus", is asked to be found.

The first law of thermodynamics is represented as:
Zero equals m dot times (hein minus haus) plus Q dot subscript "ars" plus Q dot subscript "R" not equal to W dot.

The heat output, Q dot subscript "aus", is then given by:
m dot times (haus minus hein) minus Q dot subscript "R".

Referring to Table A2 for boiling liquid where x equals zero.

The enthalpy at haus (100 degrees Celsius) is 419.04 kilojoules per kilogram.

The enthalpy at hein (70 degrees Celsius) is 292.98 kilojoules per kilogram.

Substituting these values into the equation for Q dot subscript "aus", we get:
0.3 times (419.04 minus 292.98) minus 100 kilowatts.

Finally, the heat output, Q dot subscript "aus", is calculated to be negative 63.652 kilowatts.