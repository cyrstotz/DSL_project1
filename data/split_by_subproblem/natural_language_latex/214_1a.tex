All values from Table A-2:
- The enthalpy at the inlet, denoted as \( h_{\text{ein}} \), for a temperature of 70 degrees Celsius is 252.98 kilojoules per kilogram.
- The enthalpy at the outlet, denoted as \( h_{\text{aus}} \), for a temperature of 100 degrees Celsius is 419.09 kilojoules per kilogram.
- The enthalpy \( h_R \) at 100 degrees Celsius plus \( x_0 \) times the difference between \( h_g(100^\circ C) \) and \( h_f(100^\circ C) \) equals 430.33 kilojoules per kilogram.

Energy balance:
In an adiabatic environment, the change in internal energy \( \frac{dU}{dt} \) equals zero.
- The equation is zero equals the mass flow rate at the inlet times the difference between the enthalpy at the inlet and \( h_R \) plus the mass flow rate at the outlet times the difference between \( h_R \) and the enthalpy at the outlet plus the heat flow rate \( \dot{Q}_R \) plus the heat flow rate \( \dot{Q}_{\text{aus}} \).
- This simplifies to the mass flow rate at the inlet times the difference between the enthalpy at the inlet and the enthalpy at the outlet plus \( \dot{Q}_R \) plus \( \dot{Q}_{\text{aus}} \).
- Numerically, this is -37.82 kilowatts plus 100 kilowatts plus \( \dot{Q}_{\text{aus}} \).
- Solving for \( \dot{Q}_{\text{aus}} \), it equals -62.182 kilowatts.