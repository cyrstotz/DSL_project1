The graph is a Temperature-Entropy (T-S) diagram. The x-axis is labeled as S in units of kilojoules per kilogram Kelvin. The y-axis is labeled as T in Kelvin. There are three pressure lines labeled \( p_0 \), \( p_1 \), and \( p_c \) with \( p_c \) being much greater than \( p_0 \). The graph contains five points labeled 1, 2, 3, 4, and 5. The process from point 1 to point 2 is a vertical line labeled "isentrop". The process from point 2 to point 3 is a diagonal line labeled "n less than 1". The process from point 3 to point 4 is a horizontal line labeled "isentrop". The process from point 4 to point 5 is a diagonal line labeled "n less than 1". The process from point 5 to point 1 is a vertical line labeled "isentrop".

The conditions at various points are given as follows:
- \( z_0 \): 0.1 bar, -30 degrees Celsius
- \( z_1 \): \( p_1 \) (adiabatic, n less than 1)
- \( z_2 \): isentropic, \( p_2 \)
- \( z_3 \): isochoric
- \( p^4 \) equals \( p_4 \) equals \( p_5 \)