- T subscript a12 equals 70 degrees Celsius
- Delta m subscript 12
- T subscript a1 equals 20 degrees Celsius
- x equals 1
- m subscript 1 equals 5755 kilograms
- m subscript 2 equals f of m subscript 1, m subscript rein
- Q equals m subscript 2 times u subscript 2 minus m subscript 1 times u subscript 1 or m subscript rein times h
- Q equals m subscript rein times u subscript 2 plus m subscript 1 times (u subscript 2 minus u subscript 1) minus m subscript rein times h
- m subscript rein equals (Q minus m subscript 1 times (u subscript 2 minus u subscript 1)) divided by (u subscript 2 minus h)
- TA2, u subscript 1 equals 100 degrees Celsius, x equals 0.005
- u subscript 1 equals 429.39 kilojoules per kilogram
- u subscript 2 equals T of 70 equals 246.6 kilojoules per kilogram
- Q equals (m subscript 1 plus m subscript rein) times u
- Q equals m subscript 1 times u subscript 1 plus 2538.4 kilojoules per kilogram
- TA20, h subscript rein equals 2538.1 kilojoules per kilogram
- Q equals m subscript rein times h
- m subscript rein equals 682.357