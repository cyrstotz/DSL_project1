T sub 6 equals the fraction P sub 6 over P sub 5 raised to the power of n minus 1 over n, times T sub 5.

This equals the fraction 0.1 bar over 0.5 bar raised to the power of 0.4 over 1.4, times 431.8 Kelvin, which equals 328.07 Kelvin.

Q equals m dot times the quantity h sub 5 minus h sub 6 plus the fraction w sub 5 squared minus w sub 6 squared over 2.

This equals m dot times the quantity h sub 5 minus h sub 6 plus one half m dot w sub 5 squared minus one half m dot w sub 6 squared.

One half m w sub 6 squared equals m dot times the quantity h sub 5 minus h sub 6 plus one half m w sub 5 squared times 1 times the fraction 2 over m.

w sub 6 squared equals 2 times the quantity h sub 5 minus h sub 6 plus w sub 5 squared times 1.

w sub 6 equals the square root of 2 times the quantity h sub 5 minus h sub 6 plus w sub 5 squared.

This equals the square root of negative 2 times c sub p times the quantity T sub 6 minus T sub 5 plus w sub 5 squared.

This equals the square root of negative 2 times 1.006 kilojoules per kilogram Kelvin times the quantity 328.07 Kelvin minus 431.9 Kelvin plus 220 meters per second squared.

This equals 507.25 meters per second.