The pressure and mass of the gas in the cylinder are considered. The pressure \( P_G \) is given by the equation:
\[ P_G = P_0 + P_{\text{innen}} + P_{\text{ew}} \]
\[ = 1 \, \text{bar} + \frac{M \cdot g}{A} + \frac{M \cdot g}{A} \]

The area \( A \) is calculated as:
\[ A = \pi r^2 = \pi \cdot 5 \, \text{cm}^2 \]
\[ = \pi \cdot (0.05 \, \text{m})^2 = 0.00785 \, \text{m}^2 \]

The internal pressure \( P_{\text{innen}} \) is calculated as:
\[ P_{\text{innen}} = \frac{32 \, \text{kg} \cdot 9.81 \, \frac{\text{m}}{\text{s}^2}}{0.00785 \, \text{m}^2} = \frac{0.01 \, \text{kg} \cdot 9.81 \, \frac{\text{m}}{\text{s}^2}}{0.00785 \, \text{m}^2} \]
\[ = 10^5 \, \text{Pa} + 39.36 \cdot 10^2 \, \text{Pa} + 124.1 \, \text{Pa} \]
\[ = 1.4 \cdot 10^5 \, \text{Pa} \approx 1.4 \, \text{bar} \]

Using the ideal gas law for mass:
\[ m_G = \frac{P_G V_1}{R T_1} \]
\[ D V = n R T \quad (V = 3.4 \, \text{L} = 0.0034 \, \text{m}^3) \]

The gas constant \( R_3 \) is calculated as:
\[ R_3 = \frac{R}{M} = \frac{8.314 \, \frac{\text{m}^3 \text{bar}}{\text{kmol} \cdot \text{K}}}{50 \, \frac{\text{kmol}}{\text{kmol}}} = 0.16628 \, \frac{\text{m}^3 \text{bar}}{\text{kg} \cdot \text{K}} \]

The mass \( m_G \) is then calculated as:
\[ m_G = \frac{1.4 \cdot 10^5 \, \text{Pa} \cdot 0.0034 \, \text{m}^3}{0.16628 \, \frac{\text{m}^3 \text{bar}}{\text{kg} \cdot \text{K}} \cdot 273.15 \, \text{K}} \]
\[ = 0.10342 \, \text{kg} \]

Lastly, the time is noted as:
\[ 3.42 \, \text{s} \]