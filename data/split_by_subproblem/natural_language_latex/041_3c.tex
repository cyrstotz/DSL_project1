The change in internal energy per unit time is equal to the sum of the mass flow rate times the average enthalpy for each i, plus the sum of heat transfer rates for each j, minus the sum of work rates for each n.

The change in internal energy is equal to the heat transfer from state 1 to state 2 minus the work done from state 1 to state 2.

The difference between the internal energy at state 2 and state 1 is equal to the heat transfer from state 1 to state 2 minus the work done from state 1 to state 2, assuming the mass remains constant.

The difference between the specific internal energies at state 2 and state 1 is equal to the specific heat transfer from state 1 to state 2 minus the specific work done from state 1 to state 2.

The heat transfer from state 1 to state 2 is equal to the difference in internal energy between state 2 and state 1 plus the work done from state 1 to state 2, only from gas (volume work).

This is equal to the mass of the wet part times the difference in specific internal energies between state 2 and state 1 for the wet part, plus the mass of gas times the difference in specific internal energies between state 2 and state 1 for the gas, plus the integral from state 1 to state 2 of pressure times the change in volume (volume work).

The integral from state 1 to state 2 of the change in volume times pressure is equal to the pressure times the difference in volume between state 2 and state 1, at 9.4 bar.

Isentropic Equation:

The ratio of temperature at state 2 to state 1 is equal to the ratio of volume at state 1 to volume at state 2 raised to the power of n minus 1.

n equals k, which is equal to the sum of specific gas constant and specific heat at constant volume divided by the specific heat at constant volume, and is approximately 1.2627.

The ratio of volume at state 1 to volume at state 2 is equal to the ratio of temperature at state 2 to temperature at state 1 raised to the power of 1 over n minus 1.

The volume at state 2 is equal to the volume at state 1 times the ratio of temperature at state 2 to temperature at state 1 raised to the power of 1 over n minus 1, which is equal to the volume at state 1 divided by the mass of gas times the reciprocal of the ratio of 273.15 Kelvin to 773.15 Kelvin raised to the power of 1 over n minus 1, approximately 0.1682.

The volume at state 2 is equal to the mass of gas times the volume at state 2.

The integral from state 1 to state 2 of the change in volume times the saturation pressure is approximately 0.006467.