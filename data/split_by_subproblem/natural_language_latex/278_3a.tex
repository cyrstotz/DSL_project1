The equations and expressions are as follows:

1. The terms \( m_g \) and \( p_{g1}^a \) are given.
2. The forces compared to \( p_{g1} \) are expressed as \( p_{g1} \frac{D^2}{4 \pi} = g (m_{ew} + m_e) + p_{umb} \frac{D^2}{4 \pi} \).
3. Simplifying the expression for \( p_{g1} \), we get \( p_{g1} = \frac{4}{D^2 / \pi} g (m_{ew} + m_e) + p_{umb} \).
4. Substituting values, \( p_{g1} \) becomes \( \frac{4}{(D^2 / \pi)} 9.81 \, \text{m/s}^2 (0.14 \, \text{kg} + 32 \, \text{kg}) + 10^5 \, \text{Pa} \).
5. The result is \( p_{g1} = 1.40 \, \text{bar} \).

Next set of equations:
1. The relationship \( m_g R T_1 = p_{g1} V_{g1} \) is given.
2. Solving for \( m_g \), we find \( m_g = \frac{p_{g1} V_{g1}}{R T_1} = \frac{1.4 \cdot 10^5 \, \text{Pa} \cdot 3.14 \cdot 10^{-2} \, \text{m}^3}{166.28 \, \text{J/(kg K)} \cdot 273.15 \, \text{K}} \approx 3.45 \, \text{kg} \).

Lastly:
1. The gas constant \( R \) is calculated as \( R = \frac{R}{M_g} = \frac{8.314 \, \text{J/mol K}}{50.10 \, \text{g/mol}} = 166.28 \, \text{J/kg K} \).