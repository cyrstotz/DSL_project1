The atmospheric pressure, denoted as \( p_{\text{atm}} \), is equal to 1 kilogram.

The temperature in grams, denoted as \( T_{\text{gm}} \), is 500 degrees Celsius, which is equivalent to 773.15 Kelvin.

The volume in meters, denoted as \( V_{\text{m}} \), is 3.14 liters, which is equivalent to 0.00314 cubic meters.

The mass in meters, denoted as \( m_{\text{m}} \), is 0.1 kilograms.

The revised pressure, denoted as \( p_{\text{rev}} \), is calculated as the atmospheric pressure plus the ratio of mass to area.

The revised pressure is calculated as 100 kilopascals plus the ratio of 32 kilograms to an area of 5.21 square centimeters.

The area \( A \) is calculated using the formula for the area of a circle, \( \pi r^2 \), where \( r \) is 5 centimeters, resulting in an area of 78.54 square centimeters.

The revised pressure is then recalculated as 100 kilopascals plus the pressure resulting from the ratio of 32 kilograms to the recalculated area of 78.54 square centimeters, resulting in a revised pressure of 135.55 kilopascals.

The revised pressure is confirmed to be 135.55 kilopascals.

The pressure in sine, denoted as \( p_{\text{sin}} \), is 140 kilopascals.

The mass of gas, denoted as \( m_{\text{g}} \), is calculated using the ideal gas law formula \( \frac{pV}{RT} \), resulting in a mass of 0.001232 kilograms.

The gas constant \( R \) is calculated by dividing 8.314 Joules per mole Kelvin by 18.02 kilograms per mole, resulting in a gas constant of 0.4614 kilojoules per kilogram Kelvin.