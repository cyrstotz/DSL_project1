The equations provided are as follows:

1. \( p_{g1} \) equals \( m_{g1} \) times \( g \).
2. \( C_v \) equals 0.653 Joules per gram Kelvin.
3. \( M_g \) equals 50 kilograms per kilomole.
4. \( T_{g1} \) equals 500 degrees Celsius, which is 773.15 Kelvin.
5. \( V_{g1} \) equals 3.14 liters, which is 0.00314 cubic meters.
6. \( p \) equals \( \frac{mRT}{V} \).
7. \( m \) is not completed.

Next set of equations:
1. \( A \) equals \( \left( \frac{1}{2} \right)^2 \pi \) times \( (0.05 \text{ meters})^2 \) times \( x \) equals 7.854 times \( 10^{-3} \) square meters.
2. \( p \) equals \( \frac{F}{A} \).
3. \( p \) equals \( p_{\text{amb}} \) plus \( \left( 32 \text{ kilograms} \times 9.81 \frac{\text{ meters}}{\text{ seconds}^2} \right) \) times \( \frac{1}{7.854 \times 10^{-3} \text{ square meters}} \).
4. This results in \( 100000 \text{ Pascals} \) plus \( 3946.4 \text{ Pascals} \) equals \( 1.3947 \times 10^5 \text{ Pascals} \).
5. This is defined as \( p_1 \).

Final set of equations:
1. \( M \) equals \( \frac{pV}{RT} \).
2. \( R \) equals \( \frac{\bar{R}}{M} \) equals 0.1663.
3. \( M \) equals \( \frac{1.3947 \times 10^5 \text{ Pascals} \times 0.00314 \text{ cubic meters}}{0.1663 \times 773.15 \text{ Kelvin}} \) equals \( 3.418 \times 10^{-3} \text{ kilograms} \).
4. This is equivalent to 3.418 grams.