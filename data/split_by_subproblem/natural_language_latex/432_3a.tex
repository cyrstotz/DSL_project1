Required: \( p_{3v} \), \( m_g \)

The temperature \( T_{3v,1} \) is 500 degrees Celsius and the volume \( V_{3v,1} \) is given.

The gas constant \( R_g \) is calculated as \( R \) divided by \( M_g \), which equals \( \frac{8.314 \frac{J}{mol \cdot K}}{36 \frac{kg}{kmol}} \) resulting in \( 0.461 \frac{J}{kg \cdot K} \).

The difference between \( C_p \) and \( C_v \) equals \( R \), leading to \( C_p \) being \( \frac{0.461 \frac{J}{kg \cdot K}}{0.653 \frac{J}{kg \cdot K}} \), which simplifies to \( C_p = 825.2 \frac{J}{kg \cdot K} \).

The equation \( pV = mRT \) is given.

The sum of \( p_{3v} \) and \( p_{ev} \) equals \( p_{amb} \) plus \( \frac{m_g}{V_{3v,2}} \).

The evaporator pressure \( p_{ev} \) is 1.4 bar (from Table 1).