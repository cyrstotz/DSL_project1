Ideal Gas Law:

The product of pressure (p) and volume (V) equals the product of mass (m), gas constant (R), and temperature (T):
\[ pV = mRT \]

The mass (m) can be isolated from the equation as:
\[ m = \frac{pV}{RT} \]

For a specific gas (G):
\[ m_{G} = \frac{p_{G1,1} V_{G1,1}}{R T_{G1,1}} = 3.478 \text{ grams} \]

Where the gas constant (R) is defined as:
\[ R = \frac{R}{M} = \frac{166.28}{46} \text{ kilograms per Kelvin} \]

For the second part:
The pressure \( p_{G1,2} \) is equal to \( p_{G1,1} \), because the pressure still pushes the piston upwards, or is in equilibrium.