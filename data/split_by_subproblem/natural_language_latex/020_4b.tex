The work done by the compressor, denoted as W_k, is equal to the mass flow rate of the refrigerant, denoted as m-dot_R, multiplied by the difference in enthalpy between state 2 and state 3. This can be rearranged to find the mass flow rate of the refrigerant as W_k divided by the difference in enthalpy between state 2 and state 3.

The entropy at state 2 is equal to the entropy at state 3, which is equal to the saturation entropy at 5 degrees, denoted as s_g(5). The enthalpy at state 2 is equal to the saturation enthalpy at temperature T_x, denoted as h_g(T_x). The enthalpy at state 3 is determined by interpolation from temperature T_x and entropy s_3.

The expression ends with m-dot_R, indicating the mass flow rate of the refrigerant.