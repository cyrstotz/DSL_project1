e subscript k, ist,0 equals e subscript k, ist,0 plus u subscript c, ist squared over 2 equals u subscript c, ist squared over 2.

h subscript c minus h subscript 0 equals c subscript p times (T subscript 6 minus T subscript 0) equals 1.006 kilojoules per kilogram Kelvin times (382.07 Kelvin minus 243.15 Kelvin) equals 85.43 kilojoules per kilogram.

s subscript 6 minus s subscript 0 equals c subscript p times the natural logarithm of (T subscript 6 over T subscript 0) minus R times the natural logarithm of (p subscript 6 over p subscript 0) equals 1.006 kilojoules per kilogram Kelvin times the natural logarithm of (382.07 over 243.15) minus 0.30 kilojoules per kilogram Kelvin.

Delta e subscript k, ist equals u subscript c, ist squared over 2.

Delta e subscript k, ist equals 85.43 kilojoules per kilogram minus 243.15 times 15 Kelvin times (0.3 kilojoules per kilogram Kelvin) plus (498.80 meters per second squared minus 200 meters per second squared) over 2.

equals 116.63 kilojoules per kilogram.