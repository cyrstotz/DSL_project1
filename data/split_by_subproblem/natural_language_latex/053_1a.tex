Q equals the mass flow rate (m dot) times the difference in enthalpy (h1 minus h2) plus the heat rate added (Q dot R) minus the heat rate lost (Q dot aus) minus the electrical work rate (W dot e).

h1 and h2 refer to the values in Table A2 for saturated liquid.

The enthalpy at 70 degrees Celsius, hf(70 degrees C), is 292.58 kilojoules per kilogram, which equals h1.

The enthalpy at 100 degrees Celsius, hf(100 degrees C), is 419.09 kilojoules per kilogram, which equals h2.

The electrical work rate, W dot e, is the integral from 1 to 2 of the volumetric flow rate (V dot) times the change in pressure dp, which approximates to V dot times the change in pressure (Delta p) times the mass flow rate (m dot).

The volumetric flow rate, V dot, is the average of the specific volume at 100 degrees Celsius and 70 degrees Celsius, which is 1.03375 cubic meters per kilogram.

The electrical work rate, W dot e, is V dot times 0.03375 times the pressure difference (p2 minus p1) times the mass flow rate (m dot), which equals 0.2153 cubic meters bar per second times 100 kilojoules per cubic meter bar, resulting in 21.513 kilowatts.

The heat rate lost, Q dot aus, is the mass flow rate (m dot) times the difference in enthalpy (h1 minus h2) plus the heat rate added (Q dot R) minus the electrical work rate (W dot e), which equals 83.68 kilowatts.