The mole fraction of water, \(X_{H_2O}\), is zero and the mole fraction of mercury, \(X_{Hg}\), is one.

The value of \(X_1\) is 0.2 and the molecular equivalent weight (MEW) is 0.1 kilograms.

The average quantity, \(\bar{Q}\), is 0.633 kilograms per cubic meter.

Omar equals 0.633, 3.42 times 10 to the power of negative 3, and 0.002464 kilograms per Kelvin.

This equals 2.64 Joules per Kelvin.

The product of density and gravitational acceleration times height, \(p g = \rho g h\), is considered under conditions of constant pressure (Isobar) and involves viscosity and the ratio of volume to mass, \(\frac{V}{m}\).

The rate of change of energy with respect to time, \(\frac{dE}{dt}\), equals zero plus \(Q\) and is represented by the vector \(\vec{r}^o\).

\(dL - Q\) represents a difference or subtraction involving \(Q\).

The equation \(m c (T_2 - T_1) = Q\) relates the heat transferred to the mass, specific heat, and temperature change.

\(Q\) equals the mass of the gas times the specific heat at constant volume times the temperature difference, \(Q = m_{gas} c_v (T_2 - T_1)\).

The ratio of \(Q\) to the product of the mass of the gas and its specific heat at constant volume equals the difference between the final temperature and the mean temperature, \(\frac{Q}{m_{gas} c_v} = T_2 - T_m\).

The final temperature \(T_2\) equals the initial temperature \(T_1\) plus the ratio of \(Q\) to the product of the mass of the gas and its specific heat at constant volume, \(T_2 = T_1 + \frac{Q}{m_{gas} c_v}\).

The final temperature \(T_2\) is calculated as 775.15 minus the result of the division of 0.002464 by the product of 3.42 times 10 to the power of negative 3 and 0.633.