Energy balance over compressor from state 2 to state 3:

Zero equals mass flow rate times the difference between the outlet enthalpy and the inlet enthalpy minus the work done by the turbine.

This implies that the work done by the turbine equals the mass flow rate times the difference between the outlet enthalpy and the inlet enthalpy.

The temperature difference T2 minus T1 equals the outlet temperature plus 10 Kelvin.

The temperature difference T2 minus T1 equals 6 Kelvin.

T2 equals T1 plus 10 degrees Celsius, as taken from the table.

T2 equals T1 plus 6 Kelvin.

T1 equals 10 degrees Celsius, as taken from the table.

T2 equals 16 degrees Celsius, where x equals 1, indicating saturated steam.

TAB equals A minus 10, and h2 equals the outlet enthalpy which is 237.74 kilojoules per kilogram.

For subsection b.2):

T2 equals negative 16 degrees Celsius.

h2 equals 237.74 kilojoules per kilogram.

p3 equals 8 bar.

This implies a centrifugal process.

Entropy s2 equals s3.

Entropy s2 equals 0.8328 kilojoules per kilogram Kelvin, which is equal to s3.

Referring to TAB A-12 at 8 bar.

The saturation entropy S_sat equals 0.92066 kilojoules per kilogram Kelvin at 31.33 degrees Celsius.

Entropy at 20 degrees Celsius equals 0.8374 kilojoules per kilogram Kelvin.

This implies interpolation for T3:

Thus, T3 equals... (the value is not provided in the text).