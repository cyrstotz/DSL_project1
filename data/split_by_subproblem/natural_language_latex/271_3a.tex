In the center, there is a diagram depicting a force system. The diagram includes a downward vector labeled F at the top, and various vectors pointing upwards from the bottom labeled with rho_1 and rho_2 at the extreme ends, and other vectors in between.

The force F is calculated as the sum of the masses of K and E multiplied by the gravitational acceleration g, which equals (32 kg plus 0.1 kg) times 9.81 Newtons per kilogram, resulting in 314.9 Newtons.

The atmospheric force F_atm is calculated as the product of rho_0 and area A, which equals 100,000 Newtons per square meter times pi times the square of 0.05 meters, resulting in 7.854 Newtons.

The force equilibrium equation states that F plus F_atm equals rho_1 times A.

From this, rho_1 is derived as (F plus F_atm) divided by A, which equals (314.9 Newtons plus 7.854 Newtons) divided by pi times the square of 0.05 meters, resulting in 14.0 bar.

For an ideal gas, the equation p times V equals m times R times T is given. From this, m is derived as (p_1 times V_1) divided by (R times T_1).

The gas constant R is calculated as R divided by mu, which equals 8.314 Joules per mole Kelvin divided by 50 grams per mole, resulting in 0.166 Joules per gram Kelvin.

In the solution section, the mass m_m is calculated using the formula involving energy, specific heat capacity, and temperature change, which equals (4.18 times 10^3 Joules) divided by (0.466 Joules per gram Kelvin times 3.73 times 15 Kelvin), resulting in 3.43 grams or 0.00343 kilograms.