The equation is given as:

e subscript x, str, c minus e subscript x, str, 0 equals h subscript 6 minus h subscript 0 minus T subscript 0 times (S subscript 6 minus S subscript 0) plus w subscript 6 squared over 2 minus w subscript 0 squared over 2.

It follows that w subscript 0 equals 200 meters per second, w subscript 6 equals 510 meters per second, and h subscript 6 minus h subscript 0 equals c subscript p times (T subscript 6 minus T subscript 0) equals 1.006 kilojoules per kilogram Kelvin times (340 Kelvin minus 243.15 Kelvin) equals 97.43 kilojoules per kilogram.

S subscript 6 minus S subscript 0 equals c subscript p times the natural logarithm of (T subscript 6 over T subscript 0) minus R times the natural logarithm of (P subscript 6 over P subscript 0) equals 1.006 kilojoules per kilogram Kelvin times the natural logarithm of (340 Kelvin over 243.15 Kelvin) equals 0.337 kilojoules per kilogram Kelvin.

e subscript x, str, c minus e subscript x, str, 0 equals 97.43 minus 243.15 Kelvin times (0.337 kilojoules per kilogram Kelvin) plus 510 meters per second squared over 2 minus 200 meters per second squared over 2 equals:

Delta e subscript x, str equals 125.47 kilojoules per kilogram.