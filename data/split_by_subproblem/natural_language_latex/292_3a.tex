a) The pressure \( p_{3,1} \) is given as 1.6 bar; the mass \( m_g \) in kilograms is unknown.

The pressure \( p_{3,1} \) is equal to the atmospheric pressure plus the ratio of the mass \( m_g \) times the gravitational acceleration \( g \) over the area \( A \), which calculates to \( 1.0 \) bar plus \( \frac{32 \text{ kg} \times 9.81 \text{ m/s}^2}{0.05 \text{ m}^2} \) equals \( 1.39 \text{ bar} \).

Using the ideal gas law rearranged to solve for mass \( m_g \), we have \( m_g = \frac{pV}{RT} \), which calculates to \( \frac{1.39 \times 10^5 \text{ Pa} \times 3.14 \times 10^{-3} \text{ m}^3}{106.28 \text{ J/mol K} \times 773.15 \text{ K}} \) equals \( 3.39 \text{ g} \).

The gas constant \( R \) is calculated as \( \frac{8.314 \text{ J/mol K}}{50 \text{ g/mol}} \) equals \( 106.28 \text{ J/kg K} \).

The temperature \( T_1 \) is \( 773.15 \text{ K} \).