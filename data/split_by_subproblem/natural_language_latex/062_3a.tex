The content includes several mathematical expressions and descriptions:

1. \( p_{g,1}, \quad m_g \) - This represents two variables, \( p_{g,1} \) and \( m_g \).

2. \( A = \Pi \left( \frac{D}{2} \right)^2 \) - The area \( A \) is defined as pi times the square of half the diameter \( D \).

3. A vertical array of terms:
   - \( \downarrow p_{\text{amb}} \cdot A \) indicates a downward force due to ambient pressure times area.
   - \( \downarrow p_{g,1} \) indicates a downward force due to pressure \( p_{g,1} \).
   - \( \uparrow p_{\text{EV1}} \cdot A \) indicates an upward force due to pressure \( p_{\text{EV1}} \) times area.

4. \( p_{g,1} = p_{\text{EV1}} + \frac{m_g \cdot g}{A} + p_{\text{amb}} \) - This equation relates \( p_{g,1} \) to \( p_{\text{EV1}} \), the mass \( m_g \), gravitational acceleration \( g \), area \( A \), and ambient pressure \( p_{\text{amb}} \).

5. \( p_{g,1} = p_{\text{EV1}} + \frac{m_g \cdot g}{A} + p_{\text{amb}} = \frac{32 \, \text{kg} \cdot 9.81 \, \frac{\text{m}}{\text{s}^2}}{\Pi \left( \frac{D}{2} \right)^2} + 1.10^5 \, \frac{\text{N}}{\text{m}^2} = \boxed{1.40 \, \text{bar}} \) - This is a detailed calculation of \( p_{g,1} \) resulting in 1.40 bar.

6. \( pV = mRT, \quad T_{g,1} = 773.15 \, \text{K} \) - The ideal gas law, where \( p \) is pressure, \( V \) is volume, \( m \) is mass, \( R \) is the gas constant, and \( T \) is temperature.

7. \( m_g = \frac{p_{g,1} V_{g,1}}{R T_{g,1}} = \frac{1.40 \cdot 10^5 \, \frac{\text{N}}{\text{m}^2} \cdot 3.14 \cdot 10^{-3} \, \text{m}^3}{0.166 \cdot 10^3 \, \frac{\text{J}}{\text{kg} \cdot \text{K}} \cdot 773.15 \, \text{K}} = 3.43 \, \text{kg} \) - This is a calculation of the mass \( m_g \) using the ideal gas law.

8. \( \left|Q_1\right| = 4500 J \) - The absolute value of \( Q_1 \) is 4500 Joules.

9. \( V_{1, \text{EW}} = V_{2, \text{EW}} \rightarrow v_{1, \text{EW}} = v_{2, \text{EW}} \) - The volumes \( V_{1, \text{EW}} \) and \( V_{2, \text{EW}} \) are equal, implying that the velocities \( v_{1, \text{EW}} \) and \( v_{2, \text{EW}} \) are also equal.

10. Graph Description: A graph with a horizontal axis and a wavy line oscillating above and below the axis, starting at the left, rising to a peak, falling to a trough, and continuing in a similar pattern without labeled axes.

11. \( \text{TAB AB} V_{\text{EW}} = 0.6 \cdot \frac{v_f(0.6) + (1-0.6) \cdot v_f(0.6)}{125.98 \, \frac{m^3}{kg}} \) - A calculation involving \( V_{\text{EW}} \).

12. \( \chi_2 = \frac{v_2 - v_f}{v_g - v_f} = \frac{u_2 - u_f}{u_g - u_f} \) - Expressions for \( \chi_2 \) in terms of \( v \) and \( u \) variables.

13. \( T_{\text{EW}} = \otimes \Delta U = \frac{\dot{Q}_{12} - W_{12}}{\dot{V}_{12}} \) - An expression relating temperature change \( \Delta U \) to heat flow \( \dot{Q}_{12} \), work \( W_{12} \), and volume change rate \( \dot{V}_{12} \).

14. \( \Delta U = m(u_2 - u_1) \) - The change in internal energy \( \Delta U \) is the product of mass \( m \) and the difference in specific internal energies \( u_2 - u_1 \).

15. \( u_2 - u_1 = c^t(T_2 - T_1) \) - The difference in specific internal energies is the product of a constant \( c^t \) and the temperature difference \( T_2 - T_1 \).