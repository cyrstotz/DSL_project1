The text starts with a possibly incomplete or misspelled German phrase "müs beregn ex,verel" which might mean "must calculate ex,verel". Then, it proceeds with a series of equations:

1. E subscript x,verel equals T subscript 0 times S subscript ex.
2. E subscript x,verel equals T subscript 0 times the quantity [dot m times (s subscript a-se) minus Q dot subscript j over T subscript B].
3. This is simplified to T subscript 0 times the quantity [dot m times (s subscript B-se) minus Q subscript B over T subscript B].
4. Further simplification leads to T subscript 0 times the quantity [dot m times cp times the natural logarithm of (T subscript 6 over T subscript 0) minus Q subscript B over T subscript B].
5. E subscript x,verel is then expressed as T subscript 0 times the quantity [cp times the natural logarithm of (T subscript 6 over T subscript 0) minus Q subscript B over T subscript B].
6. Finally, the value of E subscript x,verel is calculated to be negative 152.15 Watts.