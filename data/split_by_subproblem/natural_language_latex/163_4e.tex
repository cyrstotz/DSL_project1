Assumption:

The mass flow rate m1 dot equals the mass flow rate m2 dot.

The temperature T_K equals the initial temperature T_i, under the assumption that x is greater than 0.

Throttle Isenthalpic:

Refer to table TAB - A11.

At 18 bar, the pressure p4 and the enthalpy h4 equal the fluid enthalpy h_f at 18 bar, which is 83.142 kilojoules per kilogram.

The enthalpy h1 equals h4, which is 83.142 kilojoules per kilogram.

T_2 from Solution:

The temperature T_2 is negative 72 degrees Celsius.

Refer to table TAB A-10.

The enthalpy h2 equals the vapor enthalpy h_g at negative 72 degrees Celsius, which is 734.08 kilojoules per kilogram.

The enthalpy h3 at 8 bar, where the entropy S3 equals S2.

The entropy S2 equals S3, which equals the entropy S_g4 at negative 72 degrees Celsius, which is 0.8351 kilojoules per kilogram Kelvin.

Linear Interpolation using table TAB A-12:

For 8 bar superheated steam,

The enthalpy h3 at 0.9351 kilojoules per kilogram Kelvin equals the linear interpolation between the enthalpies at 0.8374 and 0.86066 kilojoules per kilogram Kelvin, calculated and added to the enthalpy at 0.86066 kilojoules per kilogram Kelvin.

The enthalpy h3 equals 727.95 kilojoules per kilogram.

Graphical Representation:

There is a diagram with two axes. The horizontal axis is labeled with temperature in degrees Celsius, and the vertical axis is labeled with pressure in bar. There are four marked points connected by lines forming a rectangle. The points are labeled with the numbers 1, 2, 3, and 4. The line from point 1 to point 2 is marked as "isentropic," and the line from point 3 to point 4 is marked as "isothermal."

The compressor work W_K dot equals negative 28 watts.

The mass flow rate of R134a equals the difference in enthalpy h2 minus h3 divided by (734.08 minus 27.92) kilojoules per kilogram, which equals 0.1720 grams per second.

If the mass flow rate m dot equals 1 kilogram per second, then m dot equals 758 grams per second.

The temperature T1 equals T2, which is negative 22 degrees Celsius.

The enthalpy h1 equals the vapor enthalpy h_g1 at 8 bar, which is 93.42 kilojoules per kilogram.

The quality x1, where x1 is greater than 0 indicating wet steam.

The quality x1 equals the ratio of h1 minus the fluid enthalpy h_f1 to the difference in vapor and fluid enthalpies at 8 bar.

The quality x1 equals 0.3277, implying 32.77 percent.

The efficiency epsilon_K equals the ratio of heat input Q_dot to the compressor work W_K dot.

The heat input Q_dot equals the mass flow rate of R134a times the difference in enthalpy h2 minus h1.

The heat input Q_K equals 0.172 kilograms per second times 10^-3 times the difference in enthalpy 734.08 minus 93.42 kilojoules per kilogram.

This equals 100.10 watts.

The efficiency epsilon_K equals the ratio of 100.13 watts to 28 watts, which is 3.618.

The temperature was cooled down to temperature Tz.

Thermodynamic equilibrium is established.

A diagram is drawn with the following details:

- The diagram is a square cycle with four points labeled 1, 2, 3, and 4 in a clockwise direction.
- The x-axis is labeled as absolute pressure p_abs.
- The y-axis is labeled as temperature T_KL.
- The top side of the square is labeled as "Isobar".
- The right side of the square is labeled as "Isochore".
- The bottom side of the square is labeled as "Isobar".
- The left side of the square is labeled as "Isochore".