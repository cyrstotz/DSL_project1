The mass flow rate of substance \( s \) equals the mass flow rate of substance \( b \), which is a stationary reversible process. This implies that the entropy of \( s \) equals the entropy of \( b \), leading to a polytropic compression where \( n \) equals \( k \) equals 1.4.

The ratio of the initial temperature \( T_0 \) to the temperature of \( s \), \( T_s \), equals the ratio of the initial pressure \( p_0 \) to the pressure of \( s \), \( p_s \), raised to the power of \( \frac{n-1}{n} \). The temperature of \( b \), \( T_b \), equals \( T_s \) times the ratio of the pressure of \( b \), \( p_b \), to \( p_s \), raised to the power of \( \frac{n-1}{n} \), which equals 328.1 Kelvin.

For \( W_b \) and \( h_s \), the entropy \( s_b \) is considered. The equation is zero equals the mass flow rate times the change in enthalpy plus half the change in the square of the velocities plus the heat transfer, which is adiabatically zero.

The change in enthalpy from \( h_b \) to \( h_s \) equals the integral from \( T_b \) to \( T_s \) of the specific heat at constant pressure, \( c_p \), with respect to temperature, which simplifies to \( c_p \) times the difference \( T_s - T_b \).

The equation zero equals the change in enthalpy plus half the change in the square of the velocities, which simplifies to \( c_p \) times the difference \( T_s - T_b \) plus half the change in the square of the velocities.

The equation for \( c_p \) times the difference \( T_b - T_s \) times 2 equals the change in the square of the velocities, leading to the velocity of \( b \), \( v_b \), being 390.93 meters per second.

The energy \( e_{s + str} \) equals the change in enthalpy minus the product of the change in pressure and the change in entropy. The difference between \( e_{s + str_2} \) and \( e_{s + str_2} \) equals the change in enthalpy minus the product of the change in pressure and the change in entropy.

The change in enthalpy from \( h_0 \) to \( h_b \) equals the integral from \( T_0 \) to \( T_b \) of the specific heat at constant pressure, \( c_p \), with respect to temperature, which simplifies to the specific heat at constant volume, \( c_v \), times the difference \( T_b - T_0 \), equaling 41.08 kilojoules per kilogram.

The ratio \( n \) equals \( k \) equals the ratio of \( c_p \) to \( c_v \), where \( c_v \) equals \( c_p \) divided by \( k \), equaling 0.419 kilojoules per kilogram Kelvin, and the pressure \( p_1 \) equals \( p_0 \) with the natural logarithm of 1 equals 0.

The change in entropy from \( s_0 \) to \( s_b \) equals the integral from \( T_0 \) to \( T_b \) of \( \frac{c_p}{T} \) with respect to temperature minus the gas constant \( R \) times the natural logarithm of the ratio of \( p_b \) to \( p_0 \), which simplifies to \( c_p \) times the natural logarithm of the ratio \( T_b \) to \( T_0 \) minus \( R \) times the natural logarithm of the ratio \( p_b \) to \( p_0 \).

The change in specific volume from \( v_0 \) to \( v_b \) equals the integral from \( T_0 \) to \( T_b \) of \( \frac{c_p}{T} \) with respect to temperature, which simplifies to \( c_v \) times the difference \( T_b - T_0 \), equaling 41.08 kilojoules per kilogram.

The specific volume \( v_b \) minus the ratio of the gas constant \( R \) times the temperature \( T_b \) to the pressure \( p_b \) equals 7.93 square meters per kilogram.