Reversible adiabatic isentropic equation:

The pressure \( p_6 \) equals \( p_5 \).

The ratio of temperatures \( T_6 \) to \( T_5 \) is equal to the ratio of pressures \( p_6 \) to \( p_5 \) raised to the power of \( \frac{n-1}{n} \).

Thus, \( T_6 \) equals \( T_5 \) times the ratio of \( p_6 \) to \( p_5 \) raised to the power of \( \frac{n-1}{n} \).

Calculating it, \( T_6 \) equals \( 431.3 \) Kelvin times \( \left( \frac{0.834}{0.5} \right)^{\frac{0.4}{1.4}} \) which results in \( 328.07 \) Kelvin, which is \( T_6 \).

The value of \( n \) or \( k \) is \( 1.4 \).

Stationary balance:

The equation is zero equals the mass flow rate \( \dot{m} \) times the expression \( \left[ h_e - h_a + \frac{w_e^2}{2} - \frac{w_a^2}{2} \right] \) times \( \dot{A} \) squared.

The difference in enthalpy \( h_5 - h_6 \) equals the specific heat at constant pressure \( c_p \) times the difference in temperatures \( T_5 - T_6 \), which calculates to \( 1.006 \frac{kJ}{kg \cdot K} \times (431.3 K - 328.07 K) = 104.45 \frac{kJ}{kg} \).

The velocity \( v_a \) is calculated as the square root of \( 104.45 \frac{kJ}{kg} \times 2 \times \left( \frac{220 \frac{m}{s}}{2} \right)^2 \), resulting in \( 390.96 \frac{m}{s} \).