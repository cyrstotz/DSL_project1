The given text contains several equations and expressions:

1. The first equation is:
   Zero equals m subscript gs times the quantity h subscript 5 minus h subscript 6 plus one half times the quantity omega subscript 5 squared minus omega subscript 6 squared, plus Q dot subscript SG superscript 0 plus W dot subscript S6.

2. The second equation is:
   Zero equals m subscript gs times the quantity h subscript 4 minus h subscript 6 plus one half times the quantity omega subscript 4 squared minus omega subscript 6 squared, plus Q dot subscript SG superscript 0 minus Q dot subscript 20 superscript 0.

3. The third equation is:
   h subscript 4 minus h subscript 6 equals C subscript p times the quantity T subscript 4 minus T subscript 6.

4. The fourth equation is:
   The ratio of T subscript 1 over T subscript 5 equals the quantity p subscript c over p subscript 5 raised to the power of kappa minus 1 over kappa, which implies T subscript 6 equals 433.51 Kelvin times the quantity 0.15 bar over 0.5 bar raised to the power of 0.4 over 1.4, resulting in T subscript 6 equals 325.07 Kelvin.

5. The fifth equation is:
   C subscript p times the quantity T subscript 0 minus T subscript 6 plus one half times the quantity omega subscript 2 squared minus omega subscript 6 squared equals zero.

6. The sixth equation is:
   Two times C subscript p times the quantity T subscript 0 minus T subscript 6 minus omega subscript 0 squared equals negative omega subscript 6 squared.

7. The seventh equation is:
   Omega subscript 0 equals the square root of omega subscript 6 squared plus two times C subscript p times the quantity T subscript 0 minus T subscript 6.

8. The eighth equation is:
   Omega subscript 0 equals the square root of 200 kilojoules per kilogram Kelvin times two times 1.006 kilojoules per kilogram Kelvin times the quantity 243.15 Kelvin minus 325.07 Kelvin, resulting in omega subscript 0 equals 453.19 meters per second.

For the second part:

1. The first equation is:
   Zero equals m dot subscript gs times the quantity Delta e subscript x, str minus E dot subscript x, ver.

2. The second equation is:
   Zero equals the quantity Delta e subscript x, str minus E dot subscript x, ver.

3. The third equation is:
   e subscript x, ver equals 122.21 kilojoules per kilogram.