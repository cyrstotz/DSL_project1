a) In a centered table, there is a diagram represented inside a minipage. The diagram shows a flow process involving a reactor. The mass flow rate entering is denoted as \( \dot{m}_{\text{ein}} \), which flows downwards and then rightwards into a reactor. The reactor has a heat transfer rate of \( \dot{Q}_R = 100 \text{kW} \). The flow continues downwards from the reactor as the mass flow rate exiting, denoted as \( \dot{m}_{\text{aus}} \).

The equation given is:
Zero equals the mass flow rate times the difference in enthalpy at the inlet and the outlet, plus the heat transfer rate into the reactor minus the heat transfer rate out, expressed as:
\[ 0 = \dot{m} (h_{\text{ein}} - h_{\text{aus}}) + \dot{Q}_R - \dot{Q}_{\text{aus}} \]

The enthalpy at the inlet, \( h_{\text{ein}} \), is given as 292.39 kilojoules per kilogram, with a reference to Table A-2 at 70 degrees Celsius.

The enthalpy at the outlet, \( h_{\text{aus}} \), is equal to \( h_f \) at 100 degrees Celsius, which is 419.04 kilojoules per kilogram, also referenced from Table A-2.

The heat transfer rate out, \( \dot{Q}_{\text{aus}} \), is calculated as the mass flow rate times the difference in enthalpy at the inlet and the outlet, plus the heat transfer rate into the reactor, which equals 62.182 kilowatts.