a) E Reactor

Steady-state flow process:

The equation is zero equals the mass flow rate times the quantity of enthalpy at the inlet minus enthalpy at the outlet minus one half of the velocity squared at the inlet plus one half of the velocity squared at the outlet, plus the heat transfer rate into the reactor plus the heat transfer rate out of the reactor.

The sum of the mass flow rates equals zero.

The mass flow rate at the inlet equals the mass flow rate at the outlet.

Enthalpy values, $h_{\text{ein}}$ and $h_{\text{aus}}$, are taken from Table A 2:

Enthalpy at the inlet, $h_{\text{ein}}$, at 70 degrees Celsius is 292.98 kilojoules per kilogram.

Enthalpy at the outlet, $h_{\text{aus}}$, at 100 degrees Celsius is 419.01 kilojoules per kilogram.

The heat transfer rate out of the reactor, $\dot{Q}_{\text{aus}}$, is calculated as the mass flow rate times the difference in enthalpy at the inlet and outlet plus the heat transfer rate into the reactor.

This equals 0.3 kilograms per second times the difference between 292.98 kilojoules per kilogram and 419.01 kilojoules per kilogram plus 100 kilowatts.

This results in 61.7 kilowatts.

Student Solution

a) Energy balance of the open system:
The equation is the mass flow rate at state 2 times the enthalpy at state 2 minus the mass flow rate at state 1 times the enthalpy at state 1 plus the sum of heat transfers into the system minus the sum of work transfers out of the system equals the time derivative of the sum of kinetic energy and potential energy, which is zero.

The equation is negative heat transfer rate times the sum of mass flow rate times enthalpy plus kinetic energy plus potential energy plus the sum of heat transfers into the system minus the sum of work transfers out of the system.

The mass flow rate at state 2 equals the mass flow rate at state 1 plus an additional mass flow rate.

The equation is the sum of the mass flow rate and the mass flow rate at state 1 times the enthalpy at state 2 minus the mass flow rate at state 1 times the enthalpy at state 1 equals zero, implying that the additional mass flow rate is zero.

The enthalpy at state 2 equals the enthalpy at state 1 plus the specific heat capacity times the temperature difference between state 2 and state 1 divided by 100.

The enthalpy at state 2 equals the enthalpy at state 1 plus the square root of the specific heat capacity times the enthalpy at state 2 equals the enthalpy at state 1 at 700 degrees Celsius because the heat transfer rate at state 2 is zero.

The enthalpy at state 1 is 118.9 kilojoules per kilogram plus 0.005 times 2000.5 kilojoules per kilogram equals 130.17 kilojoules per kilogram.

The enthalpy at state 2 is 282.95 kilojoules per kilogram (from Table A-2).

The total heat transfer rate, $\dot{Q} \sum$, equals the mass flow rate at state 1 times the difference in enthalpy between state 2 and state 1.

The ratio of the mass flow rate at state 1 to the difference in enthalpy at the inlet minus the enthalpy at state 2 equals 5735.6 plus the difference between 130.17 kilojoules per kilogram and 707.05 kilojoules per kilogram equals 2.92 kilojoules per kilogram minus 282.95 kilojoules per kilogram.