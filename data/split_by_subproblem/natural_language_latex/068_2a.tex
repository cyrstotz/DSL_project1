a) FBD:

The equation for \( p_1 \) is given by \( p_1 = p_{\text{amb}} + p_K + p_{\text{ew}} \), which equals \( p_{\text{amb}} + \frac{g_1 (m_K + m_{\text{ew}})}{\frac{\pi D^2}{4}} \).

Description of the diagram:
The diagram consists of a rectangular container divided into three sections. The top section is labeled "p_{\text{amb}}" with arrows pointing downwards. The middle section is labeled "EW". The bottom section is labeled "p_K, g_1, p_{\text{amb}}, p_{\text{ew}}" with arrows pointing upwards.

The equation for \( p_C \) is given by \( p_C = \frac{m_K g}{\frac{\pi D^2}{4}} + p_{\text{ew}} \), which equals \( \frac{m_{\text{ew}} g}{\frac{\pi D^2}{4}} \).

It follows that \( p_1 = 10^5 \, \text{N/m}^2 + \frac{9.82 \, \text{m/s}^2 \cdot 1.83 + 0.21 \, \text{kg} \cdot g_1}{\pi \cdot 0.1 \, \text{m}^2} = 1,401.20 \, \text{N/m}^2 \).

This is greater than \( 3,401.20 \, \text{bar} \).

It implies that \( g_1 \) is less than or equal to the ideal \( g_2 \), stare perfect!

It follows that \( p_{g12} \frac{V_{g12}}{T_{g12}} = R_g \frac{T_{g12}}{V_{g12}} \) which implies \( V_{g12} = \frac{V_{g12}}{m_g} = \frac{R_g T_{g12}}{p_{g12}} \) from the magnitude model.

It implies that \( m_g = \frac{V_{g12} \cdot p_{g12}}{R_g \cdot T_{g12}} = \frac{V_{g12} \cdot p_{g12} \cdot R_g}{R - T_{g12}} = \frac{3.14 \cdot 10^{-3} \, \text{m}^3 \cdot 1.5 \cdot 10^5 \, \text{N/m}^2}{8.314 \, \frac{\text{J}}{\text{mol} \cdot \text{K}} \cdot (500 + 273.15) \, \text{K}} \).

It implies that \( m_g = 3.664 \, \text{g} \).