b) Given: T_g2, the ratio p1 over p2.

The density difference between ice and water is negligibly small.

Therefore, the ratio p1 over p2, since the pressure equilibrium in state 2 does not change compared to state 1.

n equals the ratio of c_p over c_V.

c1) Energy balance, closed system in equilibrium:

Q equals W_u plus (U_2 minus U_1) times mg.

W_u equals R times (T_2 minus T_1) over (1 minus n) equals 316.458 Joules per kilogram.

n equals the ratio of c_p over c_v, which equals the ratio of c_p over (R plus c_v), which equals (166.28 plus 633) over 633 equals 1.2626.

Q equals 316.458 Joules per kilogram plus c_v times (T_2 minus T_1).

d1) Required: X_0,2.

X_0 equals 0.1, m equals 0.1 kilograms, U_new equals U_old.

Closed system energy balance:

Q equals m times u_2 minus m times u_1.

u_1 equals X times u_f,1 plus X times (u_g,1 minus u_f,1).

u_f is ice, u_f is liquid.

u_f,1 equals 0 degrees Celsius, u_f,1 equals -0.045 kilojoules per kilogram, T_tab.

u_g,1 equals -333.458 kilojoules per kilogram.

Therefore, u_1 equals -200.0928 kilojoules per kilogram.

3d, continuation:

h_2 equals the ratio of Q over m plus c times Delta t, which results in (1500 Joules over 0.1 kilograms) plus (-200.0528 times 10^3 Joules per kilogram).

equals -185.0928 kilojoules per kilogram, T_3 equals 0.003 degrees Celsius.

Therefore, x_2 equals the ratio of (h_2 minus h_liquid,2) over (h_gas,1 minus h_liquid,2) at 0.003 degrees Celsius equals 0.555.