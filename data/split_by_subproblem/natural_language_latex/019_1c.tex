The entropy generation rate, denoted as S dot subscript "erz", is given by the equation:

S dot subscript "erz" equals m dot times the difference (s subscript e2 minus s subscript e1) plus Q dot subscript R divided by T.

This simplifies to:

m dot times zero plus Q dot subscript R divided by 100 degrees Celsius.

Which results in:

0.268 kilowatts per Kelvin.

Since the calculation was made for water, we need to adjust the signs:

S dot subscript "erz" equals 0.268 kilowatts per Kelvin.

For the section titled "HAUFENWEISES SYSTEM", the equations are as follows:

The natural logarithm of u2 minus the natural logarithm of u1 equals Delta m subscript "ein" times the natural logarithm of "hein" plus Q subscript "zu" minus Q subscript "ab".

Delta times (the natural logarithm of t times Delta m subscript "ein") times u2 minus the natural logarithm of u1 equals Delta m subscript "ein" times the natural logarithm of "hein" plus zero.

Delta m subscript "ein" times (u2 minus "hein") plus the natural logarithm of u2 minus the natural logarithm of u1 equals zero.

Delta m subscript "ein" equals the natural logarithm of (u2 minus u1) divided by (u2 minus "hein").

Further calculations involve specific values and states:

5755 kilograms times (U at 100 degrees Celsius, x equals 0.005 minus U1) minus U at 70 degrees Celsius, x equals 0 minus h times (Q at 20 degrees Celsius, x equals 0).

5755 kilograms times (123.38 minus 282.85) kilojoules per kilogram minus 282.85 kilojoules per kilogram minus 83.86 kilojoules per kilogram.

U at 100 degrees Celsius, x equals 0.005 equals (418.96 plus 0.005) times (2506.5 minus 18.34) equals 428.38 kilojoules per kilogram.

Delta m subscript "ein" equals 3756.8 kilograms.