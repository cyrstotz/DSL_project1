The equation is Q equals m dot times the difference of h subscript 2 and h subscript 3, plus Q dot minus W.

The graph is a plot with the vertical axis labeled p with an upward arrow and the horizontal axis labeled V with a rightward arrow. The vertical axis has a subscript "out" and the horizontal axis has a subscript "m^3/kg". There are four points labeled 1, 2, 3, and 4. The points are connected by curves forming a closed loop. The curve from point 1 to point 2 rises steeply, then the curve from point 2 to point 3 descends and curves back up to point 4, and finally, the curve from point 4 to point 1 completes the loop.

The kinetic energy E subscript K is equal to Q subscript K divided by W subscript K.

The equation O equals m dot times the difference of h subscript 1 and h subscript 2, plus Q minus W.

The equation Q equals m dot times the difference of h subscript 2 and h subscript 1.

The entropy S subscript 4 equals S subscript 1.

The entropy S subscript 4 equals S subscript f at 800 degrees Celsius, which is 0.3450 kilojoules per kilogram Kelvin, equals S subscript 1 according to table A-17.

The pressure P subscript 2 equals P subscript 1.

The pressure P subscript 2 equals 1.0296 bar, which is approximately 1.02 bar according to table A10 at 42 degrees Celsius.

The pressure P subscript 1 equals P subscript 2 bar, which implies S subscript 1 equals S subscript f plus x subscript 1 times the difference of S subscript g and S subscript f.

The value x subscript 1 equals the difference of S subscript 1 and S subscript f divided by the difference of S subscript g and S subscript f, which approximately equals 30 percent or 0.3.

The kinetic energy E subscript K equals Q subscript K divided by W subscript K.

The equation O equals m dot times the difference of h subscript 1 and h subscript 2, plus Q.

The equation Q equals m dot times the difference of h subscript 2 and h subscript 1.

The enthalpy h subscript 2 interpolated.

The enthalpy h subscript 2 at 42 degrees Celsius according to table A10 is 234.08 kilojoules per kilogram.

The enthalpy h subscript 1 equals h subscript f plus x subscript 1 times the difference of h subscript g and h subscript f, which equals 271.32 plus 0.3 times 271.32, resulting in 81.082 kilojoules per kilogram according to table A11.

The equation Q equals m dot times the difference of 234.08 and 81.082, which equals 59 kilojoules per W.

The kinetic energy E subscript K equals Q subscript K divided by W subscript K, which equals 21.29.