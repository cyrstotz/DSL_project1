p subscript S1, m subscript S1, p subscript V1 equals R T subscript 1.

p subscript 1 V subscript 1 equals m R T subscript 1 implies p subscript 1 equals m R T subscript 1 over V subscript 1 equals R T subscript 1 over V subscript 1.

p subscript ag equals p subscript amb plus (M subscript g times g plus m subscript EW times g) over A.

p subscript ag equals p subscript amb plus p subscript m plus p subscript EW equals 1 bar plus 10 to the power of 5 plus (M subscript g times g) over 0.1 meters squared plus (0.1 kg times g) over 0.1.

p subscript ag equals 10,3149,33 Pascals equals 1.034147 bar.

m subscript 1 equals (103.147 kPa times 3.14 times 10 to the power of -3 cubic meters) over (0.16628 kJ/kgK times (500 K plus 273.15 K)) equals 0.002578 kg equals 7.578 g equals m subscript 1.