T6 equals 328.1 Kelvin. T0 equals 243.15 Kelvin. W6 equals 507.2 meters squared per second squared. u0 equals 200 meters per second. cp equals 1.006 kilojoules per kilogram Kelvin.

The change in specific energy from state 0 to state 6 is equal to h6 minus h0 minus T0 times (s6 minus s0) plus half of W6 squared. This is equal to h6 minus h0 minus T0 times (s6 minus s0) plus half of W6 squared minus half of u0 squared. This is equal to cp times (T6 minus T0) minus T0 times (cp times the natural logarithm of T6 over T0 minus R times the natural logarithm of p6 over p0) plus half of (W6 squared minus u0 squared). This is equal to cp times (T6 minus T0) minus T0 times (cp times the natural logarithm of T6 over T0 plus half of (W6 squared minus u0 squared)). This equals 12.165 kilojoules per kilogram plus 102625.8 meters squared per second squared. This equals 12.165 kilojoules per kilogram plus 108.63 kilojoules per kilogram. This equals 120.8 kilojoules per kilogram.

Graphical Description:

The graph is a Temperature-Entropy (T-S) diagram with the following details:

- The x-axis is labeled as entropy in kilojoules per kilogram Kelvin.
- The y-axis is labeled as temperature in Kelvin.
- The graph starts at a temperature of 243.15 Kelvin on the y-axis.
- There are several curves and lines labeled as follows:
  - A curve labeled "ISOBAR" starting from point 1 to point 2.
  - A curve labeled "ISOBAR" starting from point 3 to point 4.
  - A curve labeled "ISOBAR" starting from point 5 to point 6.
  - A curve labeled "ISENTROP" starting from point 1 to point 3.
  - A curve labeled "ISENTROP" starting from point 4 to point 6.
  - A curve labeled "ISENTROP" starting from point 2 to point 5.
- Points are marked with numbers 1 through 6.
- There are arrows indicating the direction of the processes between the points.
- The graph is titled "KEZNBROT".