Energy balance standard flow process around water:

The equation is zero equals the mass flow rate times the difference between the enthalpy at the inlet and the enthalpy at the outlet, plus the heat input minus the heat output.

This implies that the heat output equals the mass flow rate times the difference between the enthalpy at the outlet and the enthalpy at the inlet, multiplied by a constant plus the heat input.

The enthalpy at the inlet at 700 degrees Celsius and quality x equals zero is given in Table A-2 as 292.98 kilojoules per kilogram.

The enthalpy at the outlet at 400 degrees Celsius and quality x equals zero is given in the same table as 149.04 kilojoules per kilogram.

This implies that the heat input equals 0.3 times the sum of 149.04 and 292.98 plus the heat input, which equals 62.18 kilowatts.

Energy balance in a semi-open system:

The equation is the mass at state 2 times the internal energy at state 2 minus the mass at state 1 times the internal energy at state 1 equals the change in mass at the inlet times the enthalpy plus the canceled heat minus the canceled work, which equals zero since the heat input equals the heat output.

The mass at state 1 is 5755 kilograms.

For state 1, the temperature is 100 degrees Celsius and the quality is 0.005.

This implies that the enthalpy from Table A-2 plus the quality times the difference between 2506.65 and 445.94 equals 429.39 kilojoules per kilogram.

For state 2, the temperature is 70 degrees Celsius and the quality is zero.

This implies that the enthalpy from Table A-2 equals 232.95 kilojoules per kilogram.

The change in mass equals the mass at state 2 minus the mass at state 1.

For the reference state, the temperature is 20 degrees Celsius and the quality is zero.

This implies that the enthalpy from Table A-2 equals 83.86 kilojoules per kilogram.

This implies that the mass at state 2 times the internal energy at state 2 minus the mass at state 1 times the internal energy at state 1 equals the change in mass times the enthalpy at the reference state.

The mass at state 2 times the internal energy at state 2 minus the enthalpy equals the mass at state 1 times the internal energy at state 1 minus the mass at state 1 times the enthalpy at the reference state.

This implies that the mass at state 2 equals the mass at state 1 times the internal energy at state 1 minus the mass at state 1 times the enthalpy at the reference state divided by the internal energy at state 2 minus the enthalpy at the reference state, which equals 11,520.1 kilograms.

This implies that the change in mass at state 2 equals the mass at state 2 minus the mass at state 1, which equals 5765.1 kilograms.