The magnitude of Q12 is 1.5 kJ.

The goal is to find X_Eis,2.

The temperature at equilibrium state 2, T_EW,2, is equal to the temperature at equilibrium state 1, T_EW,1.

The heat transfer Q12 is equal to the mass at state 2 times the internal energy at state 2 minus the mass at state 1 times the internal energy at state 1.

The mass at equilibrium state 2, m_EW,2, is equal to the mass at equilibrium state 1, m_EW,1, and the mass of ice at state 2, m_Eis,2, is equal to the mass at state 2 minus the mass at state 1.

The internal energy at state 1, u1, is equal to the internal energy at the freezing point, uf, plus X_Eis,1 times the difference between the internal energy of ice, u_Eis, and uf. This is in a solid state.

At 0 degrees Celsius and pressure p_Eis,1 equal to the ambient pressure p_amb:

The ambient pressure plus the mass times gravity divided by the square of half the diameter D times pi equals 1.9 bar.

The internal energy at the freezing point, uf, is -0.045 kJ/kg, and the internal energy of ice, u_Eis, is -333.458 kJ/kg.

Therefore, the internal energy at state 1, u1, is -200.00 kJ/kg.

Therefore, the heat transfer Q12 is equal to the mass at equilibrium state 1 times the difference between the internal energy at state 2 and state 1.

Therefore, Q12 divided by the mass at equilibrium state 1 equals the difference between the internal energy at state 2 and state 1, which leads to Q12 divided by the mass at equilibrium state 1 plus the internal energy at state 1 equals the internal energy at state 2.

The internal energy at state 2, u2, is 1.5 divided by 2 minus 785.09 kJ/kg.

Therefore, the internal energy at state 2, u2, is equal to the internal energy at the freezing point, uf, plus X_Eis,2 times the difference between the internal energy of ice, u_Eis, and uf, under the same temperature and pressure.

Therefore, the ratio of u2 minus uf to u_Eis minus uf equals X_Eis,2, which leads to X_Eis,2 being equal to 0.555 divided by 1.