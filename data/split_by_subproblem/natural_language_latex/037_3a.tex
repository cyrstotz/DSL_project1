- There is a horizontal bar labeled "Piston" with an upward arrow labeled \( P_{EW} \) pointing towards it.
- Below this, there is another horizontal bar labeled "Membrane" with a downward arrow labeled \( P_{gas,1} \) pointing towards it.
- Both bars are connected by a curly brace on the right side, indicating that \( 1 \, \text{bar} = P_{EW} = P_{gas,1} = 1.01325 \, \text{bar} \).

The equation \( P_{gas,1} V_{g1} = m_{g1} R T_{g1} \) is given.

The gas constant \( R \) is calculated as \( R = \frac{\overline{R}}{M_g} = \frac{0.76628 \, \text{Joules per gram Kelvin}}{0.76628 \, \text{kilograms per kilogram Kelvin}} = 0.76628 \, \text{kilojoules per kilogram Kelvin} \).

The mass \( m_{g1} \) is calculated using the formula \( m_{g1} = \frac{P_{g1} V_{g1}}{R T_{g1}} \).

The volume \( V_{g1} \) is given as \( 3.14 \, \text{liters} = 3.14 \, \text{cubic decimeters} = 3.14 \times 10^{-3} \, \text{cubic meters} \).

The temperature \( T_{g1} \) is given as \( 500^\circ \text{Celsius} = 773.15 \, \text{Kelvin} \).

The mass \( m_{g1} \) is calculated as \( m_{g1} = \frac{1.01325 \, \text{bar} \times 3.14 \times 10^{-3} \, \text{cubic meters}}{0.76628 \, \text{kilojoules per kilogram Kelvin} \times 773.15 \, \text{Kelvin}} = 2.47485 \, \text{grams} \).

The mass \( m_{g1} \) is reiterated as \( 2.47485 \, \text{grams} \) using the conversion factor \( \left( \frac{\text{Newton meter}}{\text{kilogram Kelvin}} \cdot \frac{\text{kilogram}}{\text{Newton meter}} \right) = 2.47485 \, \text{grams} \).