The equation is m subscript 2 times u subscript 2 minus m subscript 1 times u subscript 1 equals delta m times h plus Q minus V dot subscript 0.

m subscript 2 equals 5.755 kilograms, delta m.

m subscript 2 equals 5.755 kilograms, m.

u subscript 1 equals u subscript f at 20 degrees Celsius plus x times (u subscript g at 20 degrees Celsius minus u subscript f at 20 degrees Celsius) equals 429.38 kilojoules per kilogram plus 0.005 times (2506.5 kilojoules per kilogram minus 429.38 kilojoules per kilogram).

u subscript 2 equals u subscript f at 70 degrees Celsius plus x times (u subscript g at 70 degrees Celsius minus u subscript f at 70 degrees Celsius) equals 302.83 kilojoules per kilogram plus 0.005 times (2594.6 kilojoules per kilogram minus 302.83 kilojoules per kilogram).

h equals h subscript f at 20 degrees Celsius equals 83.96 kilojoules per kilogram.

Insert (substitute):

(5.755 kilograms minus delta m) times 302.83 kilojoules per kilogram minus 5.755 kilograms times 429.38 kilojoules per kilogram equals delta m times 83.96 kilojoules per kilogram plus 35 times 10 to the power of 3 kilojoules.

Solve for delta m:

delta m equals 1.77252 times 10 to the power of 3 kilograms.

Equals 1773 kilograms.