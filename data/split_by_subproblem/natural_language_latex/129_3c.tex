c) The heat transfer, denoted as Q_12, is equal to the specific heat at constant pressure, C_p, multiplied by the mass of substance S, m_S, and the difference in temperature from T_1 to T_2. The specific heat at constant pressure, C_p, is given by the formula pi divided by the molar mass of substance S, M_S, plus the specific heat at constant volume, C_v.

This can be expanded to the expression (pi divided by M_S plus C_v) multiplied by (T_2 minus T_1). Substituting the values, this becomes (8.31 times the volume of S7, V_S7, times the temperature of S7, T_S7, divided by 5 times the volume of k, V_k, times the mass of substance S, m_S, plus 0.63 times the volume of S7, V_S7, per Kelvin) multiplied by 3.42 grams times (50 Joules per Kelvin).

The result of this calculation is 1366.9 Joules.