The temperature \( T_g \) is 500 degrees Celsius.
The volume \( V_s \) is 3.14 liters.
The product of pressure \( P_s \) and volume \( V_s \) equals the mass of gas \( M_g \) times the ratio of the universal gas constant \( \bar{R} \) over the molar mass \( \bar{M} \) times the temperature \( T_g \).

Subsection: Force Equilibrium

There is a sketch of a piston in a cylinder. The piston is represented by two horizontal lines that limit the cylinder. The piston is in the middle of the cylinder and is influenced by two arrows pointing downwards and one arrow pointing upwards.
- The upper arrow points downwards and is labeled with \( P_{\text{amb}} \cdot A \).
- The lower arrow points upwards and is labeled with \( P_g \cdot A \).
- Another arrow points downwards and is labeled with \( M_k \cdot g \).
- Next to the piston, it states "in equilibrium".

The force equation is given by:
\[ P_g \cdot A = P_{\text{amb}} \cdot A + M_k \cdot g \]
The area \( A \) is calculated as:
\[ A = \left( \frac{D}{2} \right)^2 \pi = 0.00785 \, \text{m}^2 \]
The pressure \( P_g \) is:
\[ P_g = P_{\text{amb}} + \frac{M_k \cdot g}{A} \]
The acceleration due to gravity \( g \) is:
\[ g = 9.81 \, \frac{\text{m}}{\text{s}^2} \]
The pressure \( P_g \) is:
\[ P_g = 139370 \, \text{Pa} \]
\[ = 1.4 \, \text{bar} \]

The mass of the gas \( M_g \) is calculated as:
\[ M_g = \frac{P_g \cdot V_s}{\frac{\bar{R}}{\bar{M}} \cdot T_g} \]
\[ = \frac{1.4 \cdot 10^5 \cdot 3.14 \cdot 10^{-3}}{8.314 \, \frac{\text{J}}{\text{mol} \cdot \text{K}} \cdot 773.15 \, \text{K}} \]
\[ = 3.42 \cdot 10^{-3} \, \text{kg} \]

The temperature \( T_g \) is 773.15 Kelvin.
The volume \( V_s \) is 3.14 times 10 to the power of negative 3 cubic meters.