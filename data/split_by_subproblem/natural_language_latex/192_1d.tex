For Z1:
- The mass m1 is 37.75 kilograms.
- The temperature T1 is 100 degrees Celsius.
- The value x1 is 0.005.

For Z2:
- The mass m2 is m1 plus delta m.
- The temperature T2 is 70 degrees Celsius.
- The value x2 is 0.005.

Deriving the first equation:
- The equation m2 times u2 minus m1 times u1 plus canceled delta kinetic energy plus canceled delta potential energy equals delta m times h_hm plus the sum of canceled Q dot minus canceled W dot, where Qein equals Qaus and V is constant.
- Delta m equals the fraction of (m2 times u2 minus m1 times u1) over h_hm.

For u1 at 100 degrees Celsius and x equals 0.005:
- u1 equals u_f plus 0.005 times (u_g minus u_f).
- u1 equals 428.38 kilojoules per kilogram.
- u_f equals 418.94.
- u_g equals 2506.5 (A2).

For u2:
- u2 equals u_f at 70 degrees plus 0.005 times (u_g at 70 minus u_f at 70).
- u2 equals 303.8 kilojoules per kilogram.
- u_f at 70 equals 292.95.
- u_g at 70 equals 2468.6 (A2).

For h_hm at 70 degrees Celsius and x equals 0.005:
- h_hm equals h_f at 70 plus x times (h_g at 70 minus h_f at 70).
- h_hm equals 96.23 kilojoules per kilogram.
- h_f at 70 equals 83.96.
- h_g at 70 equals 2538.1 (A2).

Graphical Description:
- The diagram is a simple square with arrows indicating the flow of mass. The top and bottom of the square are labeled with "35M" each. An arrow labeled delta m dot points into the left side of the square. The right side of the square is not labeled. The diagram represents a control volume with mass flow rates.

For A1d:
- Delta m12 equals the fraction of ((m1 plus delta m12) times u2 minus m1 times u1) over h_am.
- Delta m times h_am equals m1 times u2 plus delta m12 times u2 minus m1 times u1.
- Delta m times (h_am minus u2) equals m1 times u2 minus m1 times u1.
- Delta m12 equals the fraction of m1 times (u2 minus u1) over (h_am minus u2).
- Delta m12 equals 3452 kilograms.