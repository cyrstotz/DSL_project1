The total pressure \( p_{3T} \) is given by the sum of the momentum per unit area of two masses plus an initial pressure \( p_0 \). It is expressed as:

\[ p_{3T} = \frac{m_{1}v_{1}}{A} + \frac{m_{2}v_{2}}{A} + p_{0} \]

Substituting the values, the equation becomes:

\[ p_{3T} = \frac{0.1 \, \text{kg} \cdot 3 \, \text{m/s}}{A} + \frac{32 \, \text{kg} \cdot 0.1 \, \text{m/s}^2}{A} + 10^5 \, \text{Pa} \]

The area \( A \) is calculated as:

\[ A = \left(\frac{3}{2}\right)^2 \cdot \pi = (0.5)^2 \cdot \pi \]

Thus, substituting the area back into the equation for \( p_{3T} \), we get:

\[ p_{3T} = \frac{0.1 \, \text{kg} \cdot 3 \, \text{m/s}}{A} + \frac{32 \, \text{kg} \cdot 0.1 \, \text{m/s}^2}{A} + 10^5 \, \text{Pa} \]

Finally, the total pressure \( p_{3T} \) is calculated to be:

\[ = 140.0374 \, \text{Pa} = 1.4 \, \text{bar} \]