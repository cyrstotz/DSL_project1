The graph description is as follows: The graph is a plot with the vertical axis labeled T(u) and the horizontal axis labeled s(u/usu). The graph consists of several points and lines connecting them. The points are labeled as follows:
- Point 0 at the origin.
- Point 1 above and to the right of point 0.
- Point 2 directly to the right of point 1.
- Point 3 above and to the right of point 2.
- Point 4 below and to the right of point 3.
- Point 5 below and to the left of point 4.
- Point 6 below and to the right of point 5.

The lines connecting the points are as follows:
- A solid line from point 0 to point 1.
- A solid line from point 1 to point 2.
- A solid line from point 2 to point 3.
- A solid line from point 3 to point 4.
- A solid line from point 4 to point 5.
- A solid line from point 5 to point 6.
- A dashed line from point 0 to point 6.

The graph also includes several horizontal dashed lines labeled P1, P2 equals P3, P4, and P5, P1.

The equation R equals Cp minus Cp divided by 1.4 equals 288 Joules per kilogram Kelvin.

The equation T0 equals T5 times the ratio of P0 over P5 raised to the power of (k plus 1) divided by k equals 288.07 Kelvin.

The equation w06 equals R times the ratio of (T0 minus T0) over (n minus 1) equals 61.2 kiloJoules.