- A graph is drawn with the x-axis labeled as T in degrees Celsius and the y-axis labeled as p in bar.
- There is a curve starting from the origin and bending upwards to the right.
- Three points are marked on the curve: point 1 at the lower left, point 2 at the upper right, and point 3 at the middle of the curve.
- A horizontal line is drawn from point 2 to the right, and a vertical line is drawn from point 3 upwards to intersect the horizontal line at point 2.
- The region above the curve is labeled "liquid" and the region below the curve is labeled "gas".

- A circular diagram is drawn with three points labeled 1, 2, and 3.
- Point 1 is at the left, point 2 is at the top, and point 3 is at the right.
- An arrow labeled W_k points upwards from the bottom of the circle.
- Another arrow labeled m points to the right from point 1 to point 3.

- The equation zero equals m dot times (h_2 minus h_3) minus W_k.
- W_k divided by (h_2 minus h_3) equals m dot.
- h_2 minus h_3 times (T_1 minus t_0).
- T_1 equals negative 20 degrees Celsius which is 253.15 Kelvin.
- T_2 equals 247.15 Kelvin or negative 26 degrees Celsius.
- h_2 equals h_3 at negative 26 degrees Celsius equals 231.62 kilojoules per kilogram.
- h_3, s: isotropic: s_3 equals s_2 equals s_0 at negative 26 degrees Celsius equals 0.8930 kilojoules per kilogram Kelvin.
- phi h_3 integration.
- h_3 equals h at 40 degrees Celsius, 85 meters plus the fraction of h at 50 degrees Celsius, 80 meters minus h at 40 degrees Celsius, 80 meters divided by (0.8930 kilojoules per kilogram Kelvin minus 0.8930 kilojoules per kilogram Kelvin) times (0.8930 kilojoules per kilogram Kelvin minus 0.8930 kilojoules per kilogram Kelvin).
- Equals 274.17 kilojoules per kilogram.
- m dot equals negative 28 kilojoules per second divided by (h_2 minus h_3) equals 0.658 times 10 to the power of negative 4 kilojoules per second equals 2.4 kilojoules per second.