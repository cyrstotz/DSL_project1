The content includes:

- A table with four columns labeled 1, 2, 3, and 4. The rows are labeled p, x, T, and h. The table contains the following values:
  - p: 810,000, 810,000
  - x: 1, 0
  - T: T_i minus 6 equals negative 22 degrees Celsius, T_i minus 6
  - h: 93.42, 234.08, 93.42

- A set of equations:
  - Zero equals m dot times (h_1 minus h_2) plus Q dot in.
  - Q dot in equals m dot times (h_2 minus h_1).
  - Zero equals m dot times (h_2 minus h_3) minus W dot k.
  - m dot equals W dot k divided by (h_2 minus h_3) equals negative 28 kilowatts divided by (234.08 minus 93.42) kilojoules per kilogram equals 0.083 kilograms per second.

- A box with the following text:
  - p_3 equals p_4
  - h_1 equals h_4
  - T_4 equals T_4

- A note indicating an adiabatic process:
  - s_2 equals s_3 equals s_2 minus s_2 at x equals 1 equals 0.3351 kilojoules per kilogram Kelvin

- A reference to a table:
  - h_2 at negative 22 degrees Celsius and x equals 1 equals 234.08 kilojoules per kilogram
  - s_2 at 8 bar equals 0.3351

- An interpolation formula:
  - h_3 equals h at 40 plus (h at 50 minus h at 40) divided by (s at 50 minus s at 40) times s_3
  - h_3 equals 293.66 plus (284.33 minus 293.66) divided by (0.3354 minus 0.3214) times s_3