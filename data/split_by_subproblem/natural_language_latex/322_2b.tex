w_0 and T_6 at the exit. This leads to the energy equation for a steady flow process:
0 equals h times the quantity h_5 minus h_6 minus the fraction (w_2 squared minus w_1 squared) over 2, plus the sum of Q dot j over m dot, minus the sum of W dot k over m dot.
This implies 2 times (h_6 minus h_5) plus w_5 squared minus w_6 squared.
w_6 equals the square root of 2 times (h_5 minus h_6) plus w_5 squared.
h_6 minus h_5 equals the integral from T_5 to T_6 of c_p dT, which equals c_p times (T_6 minus T_5).
This implies T_5 minus T_6 is adiabatic - reversible, leading to isentropic.
This leads to the polytropic temperature ratio with n equals k equals 1.4.
T_6 over T_5 equals (p_6 over p_5) raised to the power of (n-1) over n, which leads to T_6 equals T_5 times (p_6 over p_5) raised to the power of (n-1) over n.
This results in T_6 equals 431.9 Kelvin times (0.193 bar over 0.1 bar) raised to the power of 0.4 over 1.4, which equals 326.07 Kelvin.
This implies h_6 minus h_5 equals c_p times (T_6 minus T_5), which equals 1.006 kilojoules per kilogram Kelvin times (326.07 Kelvin minus 431.9 Kelvin), which equals -106.45 kilojoules per kilogram.
This leads to w_6 equals the square root of 2 times (h_5 minus h_6) plus w_5 squared, which equals the square root of 2 times 106.45 kilojoules per kilogram, which equals 243.54 meters per second.