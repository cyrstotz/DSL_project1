- \( s_2 \) equals \( s_3 \).
- \( T_i \) equals negative 10 degrees Celsius.
- \( T_{\text{ver}} \) equals negative 16 degrees Celsius, which is equal to \( T_2 \).
- \( h_2 \) equals 237.74 kilojoules per kilogram.
- \( s_2 \) equals \( s_3 \) equals 0.8289 kilojoules per kilogram Kelvin.
- \( h_3 \) is calculated as follows: the difference between 273.66 kilojoules per kilogram and 264.75 kilojoules per kilogram, divided by the difference between 0.8374 kilojoules per kilogram Kelvin and 0.8066 kilojoules per kilogram Kelvin, multiplied by the difference between 0.8289 kilojoules per kilogram Kelvin and 0.8066 kilojoules per kilogram Kelvin, plus 264.75 kilojoules per kilogram. This results in \( h_3 \) equaling 271.31 kilojoules per kilogram.
- The equation \( 0 \) equals \( \dot{m}_{R12} \) times the difference between \( h_2 \) and \( h_3 \) plus \( \dot{W}_H \) implies \( \dot{m}_{R12} \) equals the negative of \( \dot{W}_H \) divided by the difference between \( h_2 \) and \( h_3 \), which equals 3.00 kilograms per hour.

The temperature would decrease because the heat flow removes energy from the system (at constant volume).