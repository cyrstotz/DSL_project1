Underline reference 1. The pressure P_g equals the ambient pressure P_amb plus the term (4 times the mass m_u times the gravitational acceleration g) divided by (D squared times pi) plus the term (4 times the mass m_ew times g) divided by (D squared times pi), where the area A equals (D squared divided by 4 pi).

This simplifies to:
1 bar plus (4 times g divided by (D squared times pi)) times (m_u plus m_ew).

Substituting the values, it becomes:
1 bar plus (4 times 9.81 divided by (0.7 meters squared times pi)) times (32 kilograms plus 0.7 kilograms).

Thus, P_g equals 1.4 bar.

The mass m_g is calculated as follows:
m_g equals (P_g times V_1,1 times M) divided by (R times T_1,1), which equals (1.4 bar times 3.74 liters times 50 grams per mole) divided by (8.314 joules per mole Kelvin times 77.75 Kelvin).

This results in m_g equals 3.43 grams.