The description is of a graph with the vertical axis labeled as T in Kelvin and the horizontal axis labeled as S in kilojoules per kilogram Kelvin. The graph begins at the origin and initially rises steeply, then becomes a horizontal line at point 2. From point 2, it ascends again to point 3, reaching a peak at point 4. After point 4, the graph goes downward to point 5 and then sharply ascends again. The points are labeled as follows:
- Point 2 is located on the horizontal line.
- Point 3 is located at the peak of the curve.
- Point 4 is located on the descending part of the curve.
- Point 5 is located at the sharp rise after the descent.
Additionally, there is a horizontal line connecting points 2 and 4, and a vertical line connecting point 2 to the horizontal axis.