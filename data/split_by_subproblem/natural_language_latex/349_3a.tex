The equations and expressions from the student solution are as follows:

1. The pressure \( p_{1,g} \) is equal to the product of the mass \( m_g \), the gas constant \( R \), and the temperature \( T_{g,1} \), all divided by the volume \( V_{g,1} \):
   \[
   p_{1,g} = \frac{m_g \cdot R \cdot T_{g,1}}{V_{g,1}}
   \]

2. The mass \( m_g \) is calculated by dividing the product of the pressure \( p_{1,g} \) and the volume \( V_{g,1} \) by the product of the gas constant \( R \) and the temperature \( T_{g,1} \):
   \[
   m_g = \frac{p_{1,g} \cdot V_{g,1}}{R \cdot T_{g,1}}
   \]

3. The gas constant \( R \) is the universal gas constant \( \overline{R} \) divided by the molar mass \( M \), where \( \overline{R} \) is 8.314 Joules per mole Kelvin:
   \[
   R = \frac{\overline{R}}{M} \quad \overline{R} = 8.314 \frac{J}{mol \cdot K}
   \]

4. The gas constant \( R \) is given as 166.28 Joules per kilogram Kelvin:
   \[
   R = 166.28 \frac{J}{kg \cdot K}
   \]

5. The pressure \( p_{1,g} \) is calculated using the formula involving the diameter \( D \), the masses \( m_k \) and \( m_{eV} \), and the ambient pressure \( p_{amb} \):
   \[
   p_{1,g} = \frac{\pi}{4} D^2 \cdot (m_k + m_{eV}) + p_{amb}
   \]

6. The pressure \( p_{1,g} \) is stated to be 1.00 bar:
   \[
   p_{1,g} = 1.00 \, \text{bar}
   \]

7. The mass \( m_g \) is given as 5.84 grams:
   \[
   m_g = 5.84 \, g
   \]

8. The temperature \( T_{g,2} \) is equal to the temperature \( T_{g,1} \) because for ideal gases, temperature and pressure are coupled.

9. The variable \( x \) is calculated as the difference between \( u \) and \( u_f \) divided by the difference between \( u_g \) and \( u_f \):
   \[
   x = \frac{u - u_f}{u_g - u_f}
   \]

10. Since the membrane is heat overlaid, the temperatures \( T_{g,1} \) and \( T_{g,2} \) are both 0.003 degrees Celsius:
    \[
    T_{g,1} = T_{g,2} = 0.003^\circ C
    \]

11. The change in energy \( \Delta E \) is equal to the negative of the heat \( Q_{12} \):
    \[
    \Delta E = -Q_{12}
    \]

12. The energy per unit mass \( m_{ev} \) is given by \( u \) which equals negative 15 times \( a^2 \) divided by four times \( g \):
    \[
    \frac{\Delta E}{m_{ev}} = u = -15 \frac{a^2}{4g}
    \]

13. The variable \( x \) is recalculated using specific values for \( T_{g}^b \), resulting in 0.0449:
    \[
    x = \frac{T_{g}^b 75 - (-0.077)}{-233.442 + 0.022} = 0.0449
    \]