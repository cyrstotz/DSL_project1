The pressure \( p_{g1} \) is equal to the mass \( m_g \) divided by the area \( A \).

Since the mixture is incompressible and negligible at pressure, the pressure \( p_{g1} \) is equal to the mass \( m_g \) divided by the area \( A \) plus the atmospheric pressure \( p_{atm} \).

The area \( A \) is the area of the cylinder.

The area \( A \) is calculated as \( \pi \) times the square of half the diameter \( d \), which is \( \pi \left( \frac{d}{2} \right)^2 \). Substituting \( d = 0.1 \) meters, the area \( A \) becomes \( \frac{\pi}{400} \) square meters.

The pressure \( p_{g1} \) is then calculated as \( \frac{32 \, kg}{\frac{\pi}{400} \, m^2} \) times \( 9.81 \, \frac{m}{s^2} \) plus \( 100000 \, Pa \), resulting in \( 131396.54 \, Pa \).

The product of pressure \( p_{g1} \) and volume \( V_1 \) equals the gas constant \( R \) times the mass \( m_g \) times the temperature \( T \).

The mass \( m_g \) is calculated as \( \frac{p_1 V_1}{R \cdot T} \).

The gas constant \( R \) is \( \frac{8.314 \, \frac{J}{mol \cdot K}}{\frac{50 \, kg}{kmol}} \), which simplifies to \( \frac{R}{M} = 166.28 \, \frac{J}{kg \cdot K} \).

Finally, the mass \( m_g \) is calculated as \( \frac{131396.54 \cdot 0.00314 \, m^3}{166.28 \cdot 273.15 \, K} \), resulting in \( 0.0034 \, kg \) or \( 3.4 \, g \).