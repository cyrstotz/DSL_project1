The heat transfer rate is given by the equation:
Q dot equals m dot times the quantity h subscript e minus h subscript a plus the quantity w subscript e squared minus w subscript a squared all over 2.

The ratio of pressure P subscript 6 over P subscript 5 raised to the power of k minus 1 over k equals the ratio of volume V subscript 5 over V subscript 6, which also equals the ratio of temperature T subscript 6 over T subscript 5.

The temperature T subscript 6 equals T subscript 5 times the ratio of pressure P subscript 6 over P subscript 5 raised to the power of k minus 1 over k, which equals 328.07 Kelvin.

The product of pressure P subscript 5 and volume V subscript 5 equals the gas constant R times the temperature T subscript 5.

The gas constant R equals the universal gas constant bar R over the molar mass M, which equals 0.2867.

The volume V subscript 5 equals the gas constant R times the temperature T subscript 5 over the pressure P subscript 5.

The volume V subscript 5 equals 7.48 cubic meters per kilogram.

The volume V subscript 6 equals the volume V subscript 5 times the square root of the ratio of temperature T subscript 6 over T subscript 5.

The volume V subscript 6 equals 4.8377 cubic meters per kilogram.

The mass is constant.

The mass flow rate in equals density rho subscript 1 times area A subscript 1 times velocity w subscript 1 equals density rho subscript 2 times area A subscript 2 times velocity w subscript 2, leading to the assumption that area A subscript 3 equals area A subscript 2, with no other idea.

The temperature T subscript 1 equals 431.9 Kelvin and the volume V subscript 5 equals 2.42 cubic meters per kilogram.

The temperature T subscript 6 equals 322.07 Kelvin and the volume V subscript 6 equals 4.93717 cubic meters per kilogram.

The entropy s subscript 2 equals the entropy s subscript 6.

The velocity w subscript 6 equals the ratio of pressure p subscript 5 over density rho subscript 6 times the velocity w subscript 5.

The velocity w subscript 6 equals the ratio of volume V subscript 6 over V subscript 5 times the velocity w subscript 5.

The velocity w subscript 6 equals 437.49 meters per second.