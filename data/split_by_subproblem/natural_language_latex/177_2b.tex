Theta equals the total mass flow rate times the quantity of h5 minus h6 plus the difference of u5 squared and u6 squared all over 2, minus the mass flow rate times r with a dot subscript plus 5c. r with a dot subscript plus 5c equals the mass flow rate times the negative integral from 1 to 3 of p over rho d rho, which equals negative mass flow rate times p2 minus p1 times v. p times v equals R times T, v5 equals R times T5 over P5, and R equals c sub p superscript ig minus c sub v superscript ig. k equals c sub p superscript ig over c sub v superscript ig, and p times c sub v superscript ig over k equals the quantity k minus 1 over k times c sub p superscript ig equals 0.287 kilojoules per kilogram Kelvin. v5 equals 0.287 kilojoules per kilogram Kelvin times 439.9 Kelvin over 50000 Pascals equals 0.00247 cubic meters per kilogram. r with a dot subscript plus 5c equals.

The ratio of T6 over T5 equals the ratio of P6 over P5 raised to the power of n minus 1 over n. T6 equals T5 times the ratio of P6 over P5 raised to the power of n minus 1 over n equals 437.5 Kelvin times the ratio of 191000 Pascals over 50000 Pascals raised to the power of 0.4 over 1.4 equals 32 Kelvin.