The fraction of ice, denoted as \( X_{\text{Eis}} \), is equal to the mass of ice \( m_{\text{Eis}} \) divided by \( m_{\text{ew}} \).
The mass \( m_{\text{ew}} \) is 0.14 kilograms.
The temperature \( T_{\text{ew}} \) is 0 degrees Celsius.
The external pressure \( p_{\text{aussen}} \) is 1.14 bar.
The pressure of ice \( p_{\text{Eis}} \) is equal to the ambient pressure \( p_{\text{amb}} \) plus the force \( F \) divided by the area \( A \).
The force \( F \) is equal to mass \( m \) times the acceleration due to gravity \( g \).
The area \( A \) is calculated as \( r^2 \cdot \pi \), which equals \( (0.05 \, \text{m})^2 \cdot \pi \) resulting in \( 7.854 \cdot 10^{-3} \, \text{m}^2 \).
The pressure \( p_{\text{ew}} \) is calculated as 1000 kilograms times \( \frac{F}{A} \), resulting in \( 3.9 \cdot 10^5 \, \text{Pascal} \), which equals 139283.44 Pascal, approximately 4.3 bar, and is equal to the external pressure \( p_{\text{aussen}} \).
The enthalpy \( h_2 \) is calculated as \( h_f \) plus \( X \) times the difference between \( h_{\text{solid}} \) and \( h_f \).
The volume \( V_2 \) is calculated as \( V_f \) plus \( X \) times the difference between \( V_{\text{solid}} \) and \( V_f \).