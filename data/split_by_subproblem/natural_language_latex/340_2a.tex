The graph description is as follows: The graph is a plot with the vertical axis labeled T (temperature) and the horizontal axis labeled s (entropy). It consists of a closed loop with four distinct points labeled 1, 2, 3, and 4. The path from point 1 to point 2 is labeled "isobar", the path from point 2 to point 3 is labeled "isotherm", the path from point 3 to point 4 is labeled "isobar", and the path from point 4 to point 1 is labeled "isotherm". Additionally, there is a point labeled 6 connected to point 4 by a vertical line. The path from point 4 to point 6 is labeled "expander", and the path from point 6 to point 1 is labeled "isotherm". The graph also includes arrows indicating the direction of the process along the paths.

The equation is given as:
The change in enthalpy (H dot subscript 2 minus H dot subscript 1) equals the product of the ambient temperature (T subscript 0) and the change in entropy (S dot subscript 2 minus S dot subscript 1), plus half the product of the mass flow rate (m dot) and the square of velocity at point 2 (v subscript 2 squared) minus half the product of the mass flow rate (m dot) and the square of velocity at point 1 (v subscript 1 squared), plus the product of heat transfer rate (Q dot) and the difference in temperature between ambient and source (T subscript 0 minus T subscript s), minus the integral of the product of heat transfer rate (Q dot) and the ratio of ambient temperature to source temperature (T subscript 0 over T subscript s), minus the product of gas constant (R) and the difference in the product of mass flow rate, pressure, and inverse of density at point 2 and point 1 ((m dot p subscript 2 over rho subscript 2) minus (m dot p subscript 1 over rho subscript 1)), equals the product of mass flow rate (m dot), specific heat at constant pressure (c subscript p), and the difference in temperature between point 2 and point 1 (T subscript 2 minus T subscript 1).

The following calculations are performed:
- The result equals 7.005 times the difference between 1.5 times the natural logarithm of the ratio (3.2 times 10 to the power of 5 over 2.9 times 10 to the power of 5) and 2.9 times 10 to the power of 5 times the natural logarithm of the ratio (3.2 times 10 to the power of 5 over 2.9 times 10 to the power of 5), plus half of 3.2 times 10 to the power of 5 times the product of 2.0 times specific heat at constant pressure over 2.
- The result equals 96,200 plus half of 96,200.
- The result equals the product of specific heat at constant pressure (c subscript p), mass flow rate (m dot) over ambient temperature (T subscript 0) times the change in enthalpy (H dot subscript 2 minus H dot subscript 1), plus half the square of velocity at point 2 (v subscript 2 squared) minus half the square of velocity at point 1 (v subscript 1 squared), plus the product of heat transfer rate (Q dot) and the difference in temperature between ambient and source (T subscript 0 minus T subscript s), minus the integral of the product of heat transfer rate (Q dot) and the ratio of ambient temperature to source temperature (T subscript 0 over T subscript s), minus the product of gas constant (R) and the difference in the product of mass flow rate, pressure, and inverse of density at point 2 and point 1 ((m dot p subscript 2 over rho subscript 2) minus (m dot p subscript 1 over rho subscript 1)).