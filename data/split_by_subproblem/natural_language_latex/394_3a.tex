State 1 is denoted by \( P_{a,1}, m_g \).
The molar mass \( M_g \) is 50 micrograms per kilomole.
The force due to the mass of the object \( m_u \) times gravity \( g \) is directed downwards.
The force due to the mass of water \( m_w \) times gravity \( g \) is also directed downwards.
The force due to the initial pressure \( P_0 \) times the area \( A \) is directed upwards.
The force due to the pressure at state 1 \( P_{a,1} \) times the area \( A \) is directed upwards.
The pressure \( P_2 \) equals the ambient pressure \( P_{amb} \).

At equilibrium, the force due to the pressure at state 1 \( P_{a,1} \) times the area \( A \) equals the sum of the force due to the initial pressure \( P_0 \) times the area \( A \), the force due to the equivalent mass \( m_{ew} \) times gravity \( g \), and the force due to the mass of the object \( m_u \) times gravity \( g \).

The pressure at state 1 \( P_{a,1} \) is calculated as the ambient pressure \( P_{amb} \) plus gravity \( g \) times the sum of the equivalent mass \( m_{ew} \) and the mass of the object \( m_u \) divided by the area \( A \), which equals 1.4 bar.

The area \( A \) is calculated as pi times the radius squared, which is pi times half the diameter squared, resulting in \( 7.85 \times 10^{-3} \) square meters.

The product of the pressure at state 1 \( P_{a,1} \) and the volume at state 1 \( V_{a,1} \) equals the mass of gas \( m_g \) times the gas constant \( R \) times the temperature at state 1 \( T_{a,1} \).

The mass of gas \( m_g \) is calculated as the gas constant \( R \) times the temperature at state 1 \( T_{a,1} \) divided by the product of the pressure at state 1 \( P_{a,1} \) and the volume at state 1 \( V_{a,1} \), resulting in 0.282 grams.

The gas constant \( R \) is calculated as the universal gas constant \( \bar{R} \) divided by the molar mass \( M_g \), resulting in 0.1883 Joules per kilogram Kelvin.