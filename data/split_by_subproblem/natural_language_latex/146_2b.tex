Section b)

What is omega_6 and T_6?

Given values:
- T_5 equals 431.9 Kelvin,
- P_3 equals 0.5 bar,
- omega_5 equals 220 meters per second,
- P_6 equals P_0 equals 0.15 bar.

The value of n and kappa is 1.4.

The ratio of T_2 over T_1 equals the ratio of P_2 over P_1 raised to the power of (n-1) over n. This implies that the ratio of T_6 over T_5 equals the ratio of P_6 over P_5 raised to the power of (n-1) over n.

From this, T_6 equals T_5 times the ratio of P_6 over P_5 raised to the power of (n-1) over n, which calculates to 328.07 Kelvin.

The equation 0 equals mass flow rate times [h_6 minus h_5 plus omega_6 squared over 2 minus omega_5 squared over 2] leads to 0 equals h_5 minus h_6 plus omega_6 squared over 2 minus omega_5 squared over 2.

From this, h_6 minus h_5 equals omega_5 squared over 2 minus omega_6 squared over 2, leading to omega_6 equals the square root of omega_5 squared plus 2 times (h_5 minus h_6).

h_5 equals h_4 plus (431.9 Kelvin minus 430 Kelvin), which simplifies to h_5 equals (440.6 minus 431.9) kilojoules per kilogram plus 431.9.