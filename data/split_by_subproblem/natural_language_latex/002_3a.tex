The pressure \( p_{G1} \) is equal to the energy \( E \) divided by the area \( A \).

The area \( A \) is calculated as the square of half the diameter \( D \) times pi.

The force \( F \) is equal to the mass of the Earth weight \( m_{EW} \) times the acceleration due to gravity \( g \) plus the mass of the object \( m_{K} \) times the acceleration due to gravity \( g \).

The pressure \( p_{G1} \) is equal to the sum of the masses \( m_{EW} \) and \( m_{K} \) times the acceleration due to gravity \( g \), divided by the area calculated as the square of half the diameter \( D \) times pi, and this equals 1.4 bar.

The product of the pressure \( p_{G1} \) and the volume \( V_{G1} \) is equal to the mass \( m \) times the gas constant \( R \) times the temperature \( T_{G1} \).

The mass \( m \) is calculated as the product of the pressure \( p_{G1} \) and the volume \( V_{G1} \) divided by the ratio of the gas constant \( R \) to the molar mass \( M_G \) times the temperature \( T_{G1} \), and this equals 3.42 grams.