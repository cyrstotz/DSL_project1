Through \( j_{jw} \) \( P_{gz} = P_{gauss} \)

The equation for \( P_{gauss} \) is given by:
\[ P_{gauss} = P_{amb} + (mL + mEW) \cdot \frac{A \cdot g}{A} \]

Where \( A \) is defined as:
\[ A = \pi \cdot r^2 \quad \pi \cdot (5 \, \text{cm})^2 = \pi \cdot 0.05^2 \, \text{m}^2 \]

This leads to the calculation of \( P_{gauss} \):
\[ \Rightarrow P_{gauss} = 10^5 \, \text{Pa} + \frac{32 \, \text{kg} \cdot 0.1 \, \text{g}}{\pi \cdot 5^2 \, \text{cm}^2} \cdot 9.81 \, \text{m/s}^2 = 1.046 \, \text{bar} \]

The mass \( M_g \) is calculated as:
\[ M_g = \frac{P \cdot V_A}{R \cdot T_A} = \frac{1.04 \cdot 10^5 \, \text{Pa} \cdot 3.14 \cdot 10^{-3} \, \text{m}^3}{8.314 \, \frac{\text{J}}{\text{mol} \cdot \text{K}} \cdot 50 \, \text{K}} = 0.00254 \, \text{kg} \]

Finally, the conversion:
\[ \frac{8.314 \, \frac{\text{J}}{\text{mol} \cdot \text{K}}}{50 \, \text{K} \cdot 10^{-3} \, \text{m}^3} = 2.54 \, \text{g} = 1 \, \text{Mg} \]