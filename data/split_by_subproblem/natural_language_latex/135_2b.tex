The subsection describes the following:

1. The nozzle is isentropic.
2. The ratio of temperatures \( T_0 \) over \( T_5 \) is equal to the ratio of pressures \( p_0 \) over \( p_5 \) raised to the power of \( \frac{n-1}{n} \). This implies that \( T_6 \) equals \( T_5 \) times the ratio of \( p_6 \) over \( p_5 \) raised to the power of \( \frac{n-1}{n} \).
3. Substituting the values, \( T_6 \) equals 431.9 Kelvin times the ratio of 0.79 bar over 0.5 bar raised to the power of \( \frac{0.4}{1.4} \), which results in \( T_6 \) being 328.05 Kelvin.

For the static FP:

1. The equation is zero equals the mass flow rate \( \dot{m} \) times the difference in enthalpies \( h_5 - h_6 \) plus half the difference in the squares of velocities \( w_5^2 - w_6^2 \) plus the heat added per unit mass flow rate \( \frac{Q}{\dot{m}} \).
2. Zero equals half the square of velocity \( w_6 \).
3. \( w_6 \) equals the square root of twice the difference in enthalpies \( h_5 - h_6 \) plus the square of \( w_5 \).
4. Substituting the values, \( w_6 \) equals the square root of twice the product of specific heat at constant pressure \( c_p \) and the difference in temperatures \( T_5 - T_6 \) plus 70 plus the square of 220 meters per second.
5. This results in \( w_6 \) being 507.24 meters per second.