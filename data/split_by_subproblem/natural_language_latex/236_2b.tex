The given section includes several mathematical expressions and calculations:

1. The speed \( w_s \) is 220 meters per second.
2. The temperature \( T_s \) is 431.9 Kelvin.
3. The pressure \( P_s \) is 0.5 bar, which is equivalent to 50 kilopascals.
4. The mass flow rate \( \dot{m} \) is calculated as the product of area \( A \), density \( \rho \), and speed \( w \).

Under the subsection titled "Energy balance around the blade base: Stationary flow process with \( \dot{m} \)":
- The equation simplifies to zero equals \( \dot{m} \) times the expression \( h_e - h_a + \frac{(\omega_e^2 - \omega_a^2)}{2} \) plus terms that cancel out.
- Another equation simplifies to zero equals \( \dot{m} \) times the expression \( h_s - h_0 + \frac{w_s^2 - w_0^2}{2} \).
- It is derived that \( h_0 - h_s \) equals \( \frac{w_s^2 - w_0^2}{2} \).
- Expanding and rearranging terms, \( w_0^2 \) is expressed in terms of \( h_s \), \( h_0 \), and \( w_s^2 \), and further simplified using the integral of \( c_p(T) \) from \( T_0 \) to \( T_s \).

Under the subsection titled "Calculation of \( T_0 \) using polytropic temperature relation":
- The ratio \( \frac{T_0}{T_s} \) is given by \( \left( \frac{P_0}{P_s} \right)^{\frac{n-1}{n}} \), leading to the calculation of \( T_0 \) as 328.07 Kelvin using the values of \( P_0 \), \( P_s \), and \( n \).

Under the subsection titled "Calculation":
- \( w_0^2 \) is calculated using the specific heat capacity \( c_p \), the temperature difference \( T_s - T_0 \), and \( w_s^2 \).
- \( w_0 \) is then calculated as the square root of the expression for \( w_0^2 \).
- The final values for \( w_0^2 \) and \( w_0 \) are 48008.906 square meters per second squared and 220.47 meters per second, respectively.