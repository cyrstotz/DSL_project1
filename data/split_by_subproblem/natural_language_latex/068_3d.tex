The heat transfer rate from the system, denoted as Q dot subscript ab, equals the mass flow rate of R-134a multiplied by the difference in enthalpy between state 2k and state 1, as referenced from section 4. b.

The mass flow rate at the exit, denoted as m dot subscript e, equals the mass flow rate times the ratio of specific volume at state 2 to state 1, which simplifies to 1.3333 grams per second.

The heat transfer rate into the system, denoted as Q dot subscript zu, equals the compressor heat transfer rate, Q dot subscript k, plus the heat transfer rate from the system, Q dot subscript ab.

The efficiency of the compressor, denoted as epsilon subscript k, is calculated as the absolute value of the heat transfer rate into the system divided by the absolute value of the difference between the heat transfer rate from the system and the heat transfer rate into the system.

This efficiency is further calculated as the sum of the compressor work rate and the product of the mass flow rate and the enthalpy difference between state 2k and state 1, divided by the difference between the product of the mass flow rate and the enthalpy difference from state 2k to state 1 minus the compressor work rate plus the product of the mass flow rate at the exit and the enthalpy difference from state 2 to state 1.

This results in a value of 1.228.

The rate of energy change for the fluid particle, denoted as E dot subscript fp, approaches zero and is approximately equal to the mass flow rate times the difference between the specific enthalpy at the exit and the square of the specific enthalpy at state 1, multiplied by the dimensionless temperature difference times the work done during volume change from state 4 to 2.

The work done during volume change from state 4 to 2, denoted as W subscript v12, is calculated as the integral from state 4 to 2 of pressure with respect to volume, which equals the mass flow rate at the exit times the gas constant divided by the ratio of the gas constant to specific enthalpy times the temperature difference from state 1 to state 2, resulting in 1 minus h.

Refer to Box 11 for additional details.

Box 1 contains the specific heat capacity at constant pressure for R-134a, denoted as Cp.