Section 4, Task e:

S7 equals the fraction with numerator m1 times S1 at 400 degrees Celsius plus m2 times S2 at 200 degrees Celsius, and denominator m1 plus m2.

Table A-2, Table A-2.

S7 equals m1 times S1 at 100 degrees Celsius plus m2 times S2 at 200 degrees Celsius divided by the sum of m1 and m2.

This equals the fraction with numerator m1 times the sum of Sf at 100 degrees Celsius plus x times the difference between Sg at 100 degrees Celsius and Sf at 100 degrees Celsius plus m2 times Sf at 200 degrees Celsius, and denominator m1 plus m2.

This equals m1 times 1.337 kilojoules per kilogram Kelvin plus m2 times 0.2966 kilojoules per kilogram Kelvin.

Sf at 7, p equals Sf at T equals 0.2966 kilojoules per kilogram Kelvin.

Sn equals 8.715 megajoules per Kelvin.

S2 equals Sf at 200 degrees Celsius equals 0.9599 kilojoules per kilogram Kelvin.

S2 equals the fraction with numerator m2 times S2 and denominator m2 equals 8.753 megajoules per Kelvin.

Delta S12 equals S2 minus Sn equals 74 kilojoules per Kelvin.