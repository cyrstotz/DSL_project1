a) T(s)

- The graph is a plot with an x-axis labeled s and a y-axis labeled T.
- There are six points labeled 0, 1, 2, 3, 4, 5, and 6.
- The points are connected by lines forming a path: from 0 to 1, from 1 to 2, from 2 to 3, from 3 to 4, from 4 to 5, and from 5 to 6.
- There are three isobars labeled p0, p3, and p5 running parallel to the x-axis.
- The path from 0 to 1 is vertical, from 1 to 2 is horizontal, from 2 to 3 is vertical, from 3 to 4 is horizontal, from 4 to 5 is vertical, and from 5 to 6 is horizontal.
- The points 1, 3, and 5 are on the isobar p3, and points 0, 2, 4, and 6 are on the isobars p0 and p5.

a)

The total energy E_total is equal to the sum of E_cycle and E_cycle.
The rate of energy input dot E_in times Q_in is equal to the product of 1 minus the ratio of T0 over T times dot Q.

Input: first law

The rate of mass input dot m_in is equal to the rate of mass output dot m_out minus dot m_in equals dot m_out minus 5.235 times time.
The rate of mass input dot m_in times the sum of 1 plus 5.235 equals the rate of mass output dot m_out.
The rate of mass input dot m_in is equal to the rate of mass output dot m_out divided by 6.235.