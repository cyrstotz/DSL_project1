The partial derivative of E with respect to t equals the sum over i of the mass flow rate of i times the sum of the enthalpy of i at time t, half the square of the velocity of i, plus the product of the gravitational constant and the height of i at time t, plus the sum over j of the heat flow rate of j at time t, minus the sum over n of the work rate into the system at time t.

Zero equals the total mass times the quantity of the enthalpy at point 5 minus the enthalpy at point 6 plus half the square of the velocity at point 5 minus half the square of the velocity at point 6, minus the integral from 1 to 2 of the density times the differential volume.

The difference in enthalpy between point 5 and point 6 equals the integral from the temperature at point 6 to the temperature at point 5 of the specific heat capacity as a function of temperature with respect to temperature, which equals the specific heat capacity times the difference between the temperature at point 5 and the temperature at point 6.

The total mass is denoted as \( m_{ges} \).

Graph Description:
The graph features an x-axis labeled as s and a y-axis labeled as T. It includes a curve starting from the bottom left, peaking, and then descending, labeled NS. There are six points labeled from 1 to 6. Point 1 is at the bottom left, and point 2 is directly above point 1, connected by a curve labeled "isentrop". Point 3 is to the right of point 2, connected by a curve labeled "reversible adiabatic". Point 4 is above point 3, connected by a curve labeled "isotherm". Point 5 is to the right of point 4, connected by a curve labeled "p_0 equals p_6". Point 6 is below point 5, connected by a curve labeled "isotherm". The graph also shows arrows indicating the direction of the processes between the points.