The equation is given as:
P naught equals P naught one plus N sub g over A plus m times S dot equals 7.105 times P naught plus the quantity 32 times 0.12 times the fraction 3.87 over 3.7.

The next equation is:
The fraction 10.1 times 1 squared over pi equals 7.16.

The final equation is:
m sub u equals the fraction P naught times V naught over R times T naught equals the fraction P naught times V naught over R times T naught equals the fraction 7.5 times 10 to the power of 7 Pascals times 0.003714 cubic meters over 8.314 Joules per mole Kelvin times 500 Kelvin equals 0.00366 kilograms.