- 2 to 3 is isotropic.
- 0, stable fixed point, adiabatic.
- The derivative with respect to time equals the mass flow rate of component i times the enthalpy h_i plus the sum of heat flow rates equals the sum of work rates.
- 0 equals the mass flow rate of R134a times the difference in enthalpy between state 2 and state 3 plus the work rate on the compressor.
- The mass flow rate of R134a equals the work rate on the compressor divided by the difference in enthalpy between state 2 and state 3.
- The enthalpy at state 2 equals the saturated vapor enthalpy at temperature T_2, where T_2 is unknown.
- The enthalpy at state 3 equals the saturated liquid enthalpy at 8 bar plus x times the difference between the saturated vapor and saturated liquid enthalpy at 8 bar, where x is unknown.
- Isotropic implies that the entropy at state 2 equals the entropy at state 3, and the entropy at state 2 equals the saturated vapor entropy at temperature T_2.

In the student solution:
- X_3 equals the ratio of the difference between entropy at state 3 and the saturated liquid entropy to the difference between the saturated vapor and saturated liquid entropy at 8 bar.
- The entropy at state 3 equals the entropy at state 2, and the entropy at state 2 equals the saturated vapor entropy at temperature T_2.
- Temperature T_2 is unknown.
- The text states "need T_2, but how do I do that?"