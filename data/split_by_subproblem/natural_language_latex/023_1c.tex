The rate of change of entropy generation, denoted as S dot subscript ex, is questioned as unknown.

A diagram is presented, consisting of two horizontal lines at different levels connected by a vertical line. The upper horizontal line represents mass flow in and out, labeled m subscript ein and m subscript aus respectively. The lower horizontal line represents temperature in and out, labeled T subscript R,ein and T subscript R,aus respectively. A heat flow, q subscript ab, is indicated at the connection of the vertical line between the two horizontal lines. Arrows indicate the direction of flow towards and away from the center at both levels.

The temperature at the inlet, T subscript R,ein, is given as 343.15 Kelvin, and the temperature at the outlet, T subscript R,aus, is given as 373.15 Kelvin.

The difference between the outgoing and incoming entropy rates, S dot subscript out minus S dot subscript in, equals the generated entropy rate, S dot subscript gen.

The generated entropy rate, S dot subscript gen, is expressed as the difference between the outgoing and incoming entropy rates plus the heat flow rate, Q dot, divided by the temperature, T subscript R.

The generated entropy rate, S dot subscript gen, is further detailed as the difference between the heat flow rate out divided by the outlet temperature minus the heat flow rate in divided by the inlet temperature, which equals the excess entropy rate, S dot subscript ex.

The temperature, T subscript W, is given as 100 degrees Celsius.

The excess entropy rate, S dot subscript ex, is calculated to be 0.496 kilojoules per kilogram per Kelvin.

A table is presented with entropy values at 70 degrees Celsius and 100 degrees Celsius. At 70 degrees Celsius, the entropy in the fluid phase, s subscript f, is 0.954 kilojoules per kilogram per Kelvin, and in the gas phase, s subscript g, is 7.7553 kilojoules per kilogram per Kelvin. At 100 degrees Celsius, the entropy in the fluid phase, s subscript f, is 1.306 kilojoules per kilogram per Kelvin, and in the gas phase, s subscript g, is 7.3599 kilojoules per kilogram per Kelvin.

The heat flow at 40 degrees Celsius, Q subscript 40 degrees C, is mentioned without further details.

The entropy, s, is expressed as the entropy in the fluid phase, s subscript f, plus a fraction x times the difference between the entropy in the gas phase, s subscript g, and the entropy in the fluid phase, s subscript f. The calculated value is 1.33749 kilojoules per kilogram per Kelvin.