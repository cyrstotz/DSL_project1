The rate of change of energy with respect to time is zero.

The mass flow rate times the sum of the enthalpy at the inlet and half the square of the velocity at the inlet minus the mass flow rate times the sum of the enthalpy at the outlet and half the square of the velocity at the outlet plus the heat transfer rate equals zero.

Zero equals the mass flow rate times the difference in enthalpy between the inlet and the outlet plus half the difference between the square of the velocity at the inlet and the square of the velocity at the outlet plus the heat transfer rate.

Zero equals half the difference between the square of the velocity at the inlet and the square of the velocity at the outlet plus the heat transfer rate plus the difference in enthalpy between the inlet and the outlet.

The square of the velocity at the outlet equals the square of the velocity at the inlet plus two times the heat transfer rate, which implies that the velocity at the outlet equals the square root of the sum of the square of the velocity at the inlet and two times the heat transfer rate.

The square of the velocity at the outlet equals the square of the velocity at the inlet plus two times the heat transfer rate plus two times the specific heat at constant pressure times the difference in temperature between T6 and T1.

For T6, the entropy at point 6 equals the entropy at point 6.

The difference in entropy between point 5 and point 6 equals the specific heat at constant pressure times the natural logarithm of the ratio of temperature at point 5 to temperature at point 6 minus the gas constant times the natural logarithm of the ratio of pressure at point 5 to pressure at point 6, which equals zero.

The natural logarithm of the ratio of temperature at point 5 to temperature at point 6 equals the ratio of the gas constant to the specific heat times the natural logarithm of the ratio of pressure at point 5 to pressure at point 6.

The ratio of temperature at point 5 to temperature at point 6 equals the ratio of pressure at point 5 to pressure at point 6 raised to the power of the ratio of the gas constant to the specific heat, which implies that the ratio of temperature at point 6 to temperature at point 5 equals the ratio of pressure at point 6 to pressure at point 5 raised to the power of the ratio of the gas constant to the specific heat.

Temperature at point 6 equals temperature at point 5 times the ratio of pressure at point 6 to pressure at point 5 raised to the power of the ratio of the gas constant to the specific heat.

Temperature at point 6 equals 431.8 Kelvin times the ratio of 0.434 bar to 0.85 bar raised to the power of the difference of 1.4 minus 1 divided by 1.4, which equals 328.07 Kelvin.

The velocity at the outlet equals the square root of the sum of the square of the velocity at the inlet, two times the heat transfer rate, and two times the specific heat at constant pressure times the difference in temperature between T6 and T1, which equals 206.36 meters per second.