The change in exergy due to internal irreversibilities, denoted as delta e subscript x comma istr, is equal to e subscript x comma istr comma e minus e subscript x comma istr comma zero.

The negative change in exergy due to internal irreversibilities is equal to h subscript e minus h subscript a minus T subscript zero times (s subscript e minus s subscript a) plus delta e.

This expression simplifies to h subscript zero minus h subscript six minus T subscript zero times (s subscript zero minus s subscript six) plus (h subscript e subscript one minus h subscript e subscript six).

The difference h subscript zero minus h subscript six is equal to C subscript p times (T subscript zero minus T subscript six).

The difference s subscript zero minus s subscript six is equal to negative (C subscript p times the natural logarithm of (T subscript six over T subscript zero) minus R times the natural logarithm of (P subscript six over P subscript zero)), under the condition of constant pressure (isobar).

This further simplifies to negative C subscript p times the natural logarithm of (T subscript six over T subscript zero).

The difference h subscript e subscript one minus h subscript e subscript two is equal to (W subscript u comma eff squared divided by 2) minus (W subscript zero squared divided by 2).

The temperature T subscript six is equal to negative 30 degrees Celsius, which is 243.15 Kelvin.

The exergy e subscript x comma istr is equal to C subscript p times (T subscript zero minus T subscript six) minus T subscript zero times (negative C subscript p times the natural logarithm of (T subscript six over T subscript zero) plus (W subscript one squared divided by 2) minus (W subscript zero squared divided by 2)).

This expression simplifies to 1.006 kg per kgK times (243.15 Kelvin minus 328.0254 Kelvin) plus 243.15 Kelvin times (1.006 kg per kgK times the natural logarithm of (321.025 over 243.15) plus (2005 squared divided by 2) minus (507.243 squared divided by 2)).

Continuing with the task, the exergy e subscript str is equal to h subscript zero minus h subscript zero minus T subscript zero times (s subscript zero minus s subscript zero) plus (v subscript zero squared divided by 2) minus (v subscript one squared divided by 2).

This simplifies to c subscript p times (T subscript six minus T subscript zero) minus T subscript zero times (s subscript zero times the natural logarithm of (T subscript six over T subscript zero)) plus (v subscript zero squared divided by 2) minus (v subscript one squared divided by 2).

The final value is 120.806 kilojoules per kilogram.