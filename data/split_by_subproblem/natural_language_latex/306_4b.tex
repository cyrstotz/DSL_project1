The work done, denoted as W_k, is equal to the mass flow rate, denoted as dot m, times the difference in enthalpy between h_2 and h_3.

The initial temperature, T_i, is negative 20 degrees Celsius.

This implies that the temperature T_2 is negative 26 degrees Celsius, because T_2 is equal to T_1 minus 6.

The enthalpy at negative 26 degrees Celsius and x equals 1, denoted as h_2, is 231.82 kilojoules per kilogram, as taken from table A-10.

For h_3, the entropy s_2 is equal to s_3, which is 0.9390 kilojoules per kilogram Kelvin.

This implies that h_3 at 8 bar and s_3 is calculated as (289.39 minus 273.66) divided by (0.9390 minus 0.9379) times (0.9390 minus 0.9379) plus 273.66, which equals 274.29 kilojoules per kilogram.

This implies that the mass flow rate, dot m, is equal to (231.82 minus 292.15) divided by negative 28 times 10 to the power of negative 3, which equals 0.658 kilograms per second.