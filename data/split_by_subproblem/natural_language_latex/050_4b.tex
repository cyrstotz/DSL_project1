- The process from state 2 to state 3 is adiabatic and reversible.
- There is a circular diagram with arrows pointing into and out of it, labeled as state 3.
- The energy balance equation is given by:
  The derivative of E with respect to t equals zero equals the mass flow rate times the difference in enthalpy from state 2 to state 3 plus the heat input minus the work done.
- The enthalpies are given as:
  h subscript 2 at x equals 1,
  h subscript 3 at 8 bar.
- The work done is given as:
  W subscript K equals negative 28 Watts.

Graph 1:
The graph is a plot with pressure P on the vertical axis and temperature T on the horizontal axis. The curve starts from the origin and rises non-linearly, representing the gas phase. There is a label "compressed" pointing to a point on the curve.

Graph 2:
The graph is a plot with pressure P on the vertical axis and temperature T on the horizontal axis. There are two lines intersecting. The first line, labeled 1, starts from a higher pressure and decreases linearly. The second line, labeled 2, starts from a lower pressure and increases linearly. The region between the lines is shaded and labeled "blow".

Graph 3:
The graph is a plot with pressure P on the vertical axis and temperature T on the horizontal axis. There are four points labeled 1, 2, 3, and 4 forming a cycle. The process between points 1 and 2 is labeled "isobar", between points 2 and 3 is labeled "reversible adiabatic", between points 3 and 4 is labeled "isobar", and between points 4 and 1 is labeled "entropy changes".