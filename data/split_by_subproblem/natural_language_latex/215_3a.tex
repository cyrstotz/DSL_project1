The equations and expressions are as follows:

1. p times V equals R times T.
2. p times V equals m times average R times T.
3. m equals the fraction of p times V over average R times T.

The value of R is given as the fraction of average R over M equals 0.16628 Joules per Kelvin per gram.

Average R is expressed in units of Joules times mole over mole times Kelvin times gram.

The pressure p is derived from the force F from above as P equals the fraction of F over A.

The area calculation is pi times r squared equals pi times 0.05 squared equals 0.0078 square meters.

The force F is equal to m times g.

The sum of 32 plus 0.1 equals 314.9 Newtons.

Therefore, P equals the fraction of F over A equals 40 kiloPascals.

The total pressure p_ges is the sum of p_amb plus p equals 1.4 Bar.