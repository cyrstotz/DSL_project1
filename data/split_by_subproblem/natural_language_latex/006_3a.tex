The variables are \( p_{a,1} \) and \( m_g \).

The pressure \( p \) is given as \( 10^5 \) Pascals.

The pressure \( p_{g,1} \) is calculated as the ambient pressure \( p_{\text{amb}} \) plus the ratio of \( 32 \) kilograms times \( 9.81 \) meters per second squared divided by the square of \( 5 \times 10^{-3} \) meters, plus the ratio of \( 0.1 \) kilograms times \( 9.81 \) meters per second squared divided by \( 5 \times 10^{-3} \) meters times pi. This results in \( 4109444.059 \) Pascals.

The ideal gas law is represented as \( PV = mRT \).

The gas constant \( R \) is \( 8.314 \) Joules per mole Kelvin.

The equation \( \frac{50 \, \text{J}}{\text{mol} \cdot \text{K}} = \frac{4109444.059 \, \text{Pa} \cdot 3.14 \cdot 10^{-3} \, \text{m}^3}{0.166 \, \frac{\text{J}}{\text{mol} \cdot \text{K}} \cdot 7.73 \cdot 15} \) is given, which simplifies to \( 0.1 \) kilograms.