The graph is described as a Temperature-Entropy (T-S) diagram. The x-axis is labeled S with units Joules per kilogram Kelvin. The y-axis is labeled T with units Kelvin. The graph features a closed loop with six points labeled from 1 to 6. These points are connected by lines, some straight and some curved, following the sequence from point 1 to 2 to 3 to 4 to 5 to 6 and back to 1. The area between points 3 and 4 is shaded. The graph includes several annotations:
- The pressure at point 3 is equal to the pressure at point 2, denoted as P3 equals P2.
- The pressure at points 1, 4, and 5 are equal, denoted as P1 equals P4 equals P5.
- The pressure at point 6 is denoted as P6.
- The heat transfer rates are denoted as Q dot from 5 to 6, Q dot from 5 to 1, and again Q dot from 5 to 6.
- The term "isobar" is noted near the line connecting points 1 and 2.
- The term "isentrop" is noted near the line connecting points 5 and 6.