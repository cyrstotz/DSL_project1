b) The temperature \( T_6 \) is equal to the temperature \( T_5 \) multiplied by the ratio of pressure \( p_6 \) over \( p_5 \) raised to the power of \( \frac{\kappa - 1}{\kappa} \), which results in \( 328.67 \) Kelvin.

From the first law of thermodynamics applied between states 5 and 6:
The net heat transfer is zero, which equals \( \dot{Q} \) times the difference in enthalpy \( h_5 - h_6 \) plus half the difference in the squares of the velocities \( \omega_5^2 \) and \( \omega_6^2 \). This implies that \( \omega_6^2 \) equals \( \omega_5^2 \) plus two times the specific heat capacity of air \( c_{p, \text{Luft}} \) times the difference in temperatures \( T_5 - T_6 \), resulting in \( 25730 \) meters squared per second squared. The velocity \( \omega_6 \) is then \( 507.24 \) meters per second.

The process is isentropic, meaning the exponent \( n \) equals \( k \).