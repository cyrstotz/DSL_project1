a) p [bar]

Description:
- Graph 1:
  - The graph is a pressure-temperature (p-T) diagram.
  - The x-axis is labeled T [K].
  - The y-axis is labeled p [bar].
  - There is a curve that starts from the origin, rises to a peak, and then falls back down, forming a dome shape.
  - The left side of the dome is labeled "subcooled liquid".
  - The right side of the dome is labeled "superheated vapor".
  - The region under the dome is labeled "wet steam".
  - Two vertical lines are drawn from the x-axis to the curve, labeled 2 and 1 respectively.

- Graph 2:
  - The graph is another pressure-temperature (p-T) diagram.
  - The x-axis is labeled T [°C].
  - The y-axis is labeled p [bar].
  - There is a curve that starts from the origin, rises to a peak, and then falls back down, forming a dome shape.
  - The left side of the dome is labeled T_ci.
  - The right side of the dome is labeled T_ci.
  - The peak of the dome is labeled 1.
  - A point on the left side of the dome is labeled 2.
  - The y-axis has a label p_T and a mark indicating p_T = 6 bar.

a) Energy flow around the internal refrigerant in the heat exchanger:

- Description of the diagram:
  - The diagram is a rectangular representation of a heat exchanger.
  - The rectangle is divided into two horizontal sections.
  - The upper section is shaded with diagonal lines, and the lower section is outlined with a dashed orange line.
  - Inside the lower section, there is a downward arrow labeled Q̇_K and a horizontal arrow labeled R134.
  - Equation: SFP: 0 = ṁ_R [h_1 - h_2] + Q̇_K

- Repeated energy flow around water:
  - The second diagram is similar to the first one, with a rectangular representation of a heat exchanger.
  - The upper section is shaded with diagonal lines, and the lower section is outlined with a dashed orange line.
  - Inside the lower section, there is a downward arrow labeled Q̇_K and a horizontal arrow labeled R134.
  - Additionally, there is a small arrow pointing to the right at the top right corner of the rectangle.
  - Equations:
    - Q = m (s_2 - s_a) = Q̇_K / T_i - Q̇_K / T_ii
    - Q̇_K = m (s_2 - s_a) * (-T_ii + T_i)
    - Q = ṁ h_1 - ṁ
    - ṁ_R = -Q̇_K / (h_1 - h_2)

- Table:
  - The table has 6 columns and 6 rows.
  - The first row contains the headers: W, Q, Z, p, T.
  - The second row has the value 0 under column Z.
  - The third row has the value 7 under column Z.
  - The fourth row has the value 2 under column Z.
  - The fifth row has the value 28W under column Q, 3 under column Z, and 86w under column p.
  - The sixth row has the value 4 under column Z and pa under column p.
  - The seventh row has the value 7 under column Z.