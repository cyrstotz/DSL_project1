a) The area A is given by the formula for the area of a circle, pi times the square of the radius, which is half the diameter D. Substituting D/2 for the radius, we have A equals pi times (D/2) squared, which equals 25 pi square centimeters, or 25 pi times 10 to the power of negative 4 square meters.

The pressure at the top, denoted as p_top, is calculated using the formula: (mass of the kettle plus the mass of the extra weight) times gravity, divided by the area A, plus the ambient pressure. Substituting the values, (32 kilograms plus 0.1 kilograms) times 9.81 Newtons per kilogram, divided by 25 pi times 10 to the power of negative 4 square meters, plus 1 times 10 to the power of 5 Pascals, approximately equals 140 kiloPascals, which is 1.4 bar.

Mechanical Equilibrium:
The pressure p_g1 is approximately equal to p_top, which is about 1.4 bar.

The gas constant R_g is calculated by dividing the universal gas constant R by the molar mass of the gas M_g. Substituting the values, 8.314 Joules per mole Kelvin divided by 50 times 10 to the power of negative 3 kilograms per mole, equals 166.28 Joules per kilogram Kelvin.

The mass of the gas M_g is calculated using the formula: pressure of the gas times the volume of the gas divided by the product of the gas constant and the temperature of the gas. Substituting the values, 1.4 times 10 to the power of 5 Pascals times 3.14 times 10 to the power of negative 3 cubic meters, divided by 166.28 Joules per kilogram Kelvin times 273.15 Kelvin, approximately equals 0.0034 kilograms, which is 3.4 grams.