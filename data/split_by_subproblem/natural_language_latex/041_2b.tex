The process from state 5 to state 6 is adiabatic and reversible.

The rate of change of energy with respect to time is equal to the total mass flow rate times the sum of the enthalpy difference between state 5 and state 6 and half the difference of the squares of velocities at state 5 and state 6, plus the sum of heat transfer rates minus the sum of work rates.

The equation simplifies to zero equals the total mass flow rate times the sum of the enthalpy difference between state 5 and state 6 and half the difference of the squares of velocities at state 5 and state 6, minus the work rate.

The system is considered as an ideal gas.

The enthalpy difference between state 5 and state 6 is the integral of the specific heat at constant pressure from the reference temperature to the temperature at state 5, which simplifies to the specific heat at constant pressure times the temperature difference between state 5 and state 6.

The equation for entropy balance in a steady flow process states that the rate of change of entropy with respect to time is zero, which implies that the entropy at state 5 equals the entropy at state 6, indicating an isentropic process.

The rate of change of energy with respect to time is the sum of the energy flow rates of incoming and outgoing streams minus the sum of work rates and the sum of energy loss rates.

The equation simplifies to zero equals the total mass flow rate times the sum of the enthalpy difference between state 6 and state 5, minus the product of the reference temperature and the entropy difference between state 6 and state 5 plus a constant, minus the work rate.