The variable \( p_{s,1} \) is mentioned.

**Force equilibrium on the membrane:**

The forces \( F_g \), \( F_{EW} \), and \( F_{yp} \) are considered.

An upward force is indicated by \( p_{s,1} \).

The area \( A \) is calculated as \( \left(\frac{d}{2}\right)^2 \pi \). Substituting \( d = 0.04 \, \text{m} \), the area \( A \) is calculated to be \( 6.7559 \cdot 10^{-4} \, \text{m}^2 \).

The product of \( p_{s,1} \) and \( A \) is equal to the sum of \( m_g \cdot g \), \( c_{1,kg} \cdot g \), and \( 10^5 \, \text{Pa} \cdot A \).

The pressure \( p_{s,1} \) is calculated to be \( 41.05 \, \text{bar} \).

The mass \( m_g \) is calculated using the formula \( m_g = \frac{p_{s,1} \cdot V_{g,1}}{R \cdot T_{g,1}} \), resulting in \( m_g = 0.1 \, \text{kg} \).

The gas constant \( R \) is recalculated as \( R = \frac{R}{M} = \frac{8.314 \, \text{J/mol} \cdot \text{K}}{50} = 0.166 \, \text{J/g} \cdot \text{K} \).

The change in Gibbs free energy \( \Delta G \) is \( 31.65 \, \text{kJ} \).

The term "1. HS." is mentioned but not elaborated on.