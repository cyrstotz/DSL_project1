b) The process from 5 to 6 is isentropic.

The temperature at state 6 is given by the formula:
T6 equals T5 times the ratio of P6 over P5 raised to the power of (n minus 1) divided by n.

This results in:
T6 equals 328.07 Kelvin.

Energy Balance:
The equation is zero equals the total mass flow rate times (enthalpy at state 5 minus enthalpy at state 6) plus mass flow rate at state 1 times (the difference of the squares of velocity at state 5 and state 6 divided by 2) plus the heat transfer rate minus the work rate.

It simplifies to:
Two times (enthalpy at state 6 minus enthalpy at state 5) equals the square of velocity at state 5 minus the square of velocity at state 6, which leads to the square of velocity at state 6 minus the square of velocity at state 5 equals two times (enthalpy at state 6 minus enthalpy at state 5).

The equations are:
The square of angular velocity at state G minus the square of angular velocity at state 0 equals two times (enthalpy at state G minus enthalpy at state 0).
Zero equals mass times gravity times (enthalpy at state 5 minus enthalpy at state G) plus mass flow rate times gravity times (the difference of the squares of angular velocity at state 5 and state G divided by 2) plus the canceled heat transfer minus the canceled work.
The square of angular velocity at state G equals two times (enthalpy at state 5 minus enthalpy at state G) plus the square of angular velocity at state 5.
This simplifies to:
The square of angular velocity at state G equals two times the specific heat at constant pressure times (temperature at state 5 minus temperature at state 6) plus the square of angular velocity at state 5.

Diagram: A diagram with three arrows labeled as follows:
- The first arrow is labeled Q.
- The second arrow is labeled W.
- The third arrow is labeled Q minus W.

Additionally, there is a note:
- Isobaric condition: The ratio of temperature to volume is constant.