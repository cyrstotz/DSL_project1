For the subtask labeled b), which is about a 2-3 compressor operating adiabatically (no heat transfer, denoted as Q_k dot equals zero):

The first law of thermodynamics is expressed as:
The rate of change of energy, dE/dt, equals the mass flow rate, m dot, times the sum of enthalpy h, kinetic energy term k squared over 2, and the pressure-volume work term p over rho, plus the heat transfer rate, Q dot, minus the sum of work rates, W_u dot.

The equation simplifies to:
Zero equals the mass flow rate, m dot, times the difference in enthalpy from state 2 to state 3, minus the work rate of the compressor, W_k dot. From this, the mass flow rate, m dot, is derived as the work rate of the compressor, W_k dot, divided by the difference in enthalpy from state 2 to state 3.

The mass flow rate, m dot, is given, with m_R dot equal to 4 kilograms per hour, and the temperature at state 2, T_2, is minus 22 degrees Celsius.