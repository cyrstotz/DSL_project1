- The pressure \( p_{G1} \) equals 8 times \( m_0 \).
- The area \( A \) is calculated as the square of half the diameter \( D \) times pi, which equals \( 7.85 \times 10^{-3} \) square meters.
- There is a diagram described as a horizontal bar with labels:
  - On the left side of the bar: \( p_{G1} \)
  - Above the bar, pointing downwards are \( m_{KG} \times g \) and \( p_{amb} \)
  - Below the bar, pointing upwards is \( m_{EW} \times g \)
  - On the right side of the bar: \( p_{amb} \)
- The equation \( p_{G1} \times A \) equals \( m_{KG} \times g \) plus \( m_{EW} \times g \) plus \( p_{amb} \times A \).
- The pressure \( p_{G1} \) is calculated as \( g \) times the sum of \( m_{KG} \) and \( m_{EW} \) divided by \( A \), plus \( p_{amb} \).
- Substituting values, \( p_{G1} \) equals \( g \) times the sum of 32 kg and 0.1 kg divided by \( 7.85 \times 10^{-3} \) square meters, plus 1 bar, resulting in \( p_{G1} = 1.401 \) bar.
- The mass of gas \( m_g \) is calculated using the ideal gas law rearranged to \( m_g = \frac{p_A \times V_A \times M_g}{R \times T_A} \), where \( T_A = 773.15 \) Kelvin.
- Substituting values, \( m_g \) equals \( 1.4 \) bar times \( 3.14 \times 10^{-3} \) cubic meters times 50 kg/kmol divided by \( 8314 \) Joules/(kmol Kelvin) times 773.15 Kelvin, resulting in \( m_g = 0.0084 \) kg or \( 3.42 \) grams.
- The pressure \( p_{G2} \) equals \( 1.401 \) bar.
- The temperature \( T_{G2} \) and \( T_{EW2} \) are both \( 0^\circ \) Celsius.
- The ice fraction \( X_{Eis,2} \) is greater than 0, indicating that the temperature of the EW mixture has not yet increased.