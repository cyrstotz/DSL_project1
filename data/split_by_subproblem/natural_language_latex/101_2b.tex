The equations provided are as follows:

1. \( pV = \frac{M}{m} \cdot \overline{R} \cdot T \)
   - The product of pressure (p) and volume (V) equals the ratio of molar mass (M) to mass (m) multiplied by the average gas constant (\(\overline{R}\)) and temperature (T).

2. \( \frac{pV}{m} = \frac{M \cdot \overline{R} \cdot T}{p \cdot V} \)
   - The ratio of the product of pressure and volume to mass equals the product of molar mass, average gas constant, and temperature divided by the product of pressure and volume.

3. \( pV = \frac{M \cdot \overline{R} \cdot T}{p \cdot V} \cdot \overline{R} \cdot T \)
   - The product of pressure and volume equals the expression \(\frac{M \cdot \overline{R} \cdot T}{p \cdot V}\) multiplied by the average gas constant and temperature.

4. \( p^2 = \frac{M \cdot \overline{R} \cdot T^2 \cdot \overline{R}}{V^2} \)
   - The square of pressure equals the product of molar mass, the square of the average gas constant, the square of temperature, all divided by the square of volume.
   - This leads to a pressure \( p = 1,14712,03 \) pascal, which is approximately \( 1.15 \) bar, denoted as \( p_{0,1} \).

For \( m_g \):
1. \( pV = mRT \)
   - The product of pressure and volume equals the product of mass (m), the gas constant (R), and temperature (T).

2. \( m = \frac{pV}{RT} = 2.81 \cdot 10^{-3} \) kilograms, which is equivalent to \( 2.81 \) grams, denoted as \( mg \).