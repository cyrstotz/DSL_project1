First equation:
1 HS: The change in internal energy for gas, Delta U_g, equals Q minus W, where Q equals Delta U_g plus W.

Second equation:
Work done, W, equals the integral from V1 to V2 of P1,3 times dV, which equals P1,3 times (V2 minus V1) times P1,3.

Third equation:
The change in internal energy for gas, Delta U_g, equals m_g times (U2 minus U1), which equals m_g times (T2 minus T1) times c_v.

Fourth equation:
For V2: V2 equals (m_g times R times T2) divided by P1,3, equals (3.9787 times 10 to the power of minus 3 kg times 8.314 Joules per mol Kelvin times 273.15 Kelvin) divided by (1.3187 times 10 to the power of 5 Newtons per square meter) times (50 kg per kmol).

Fifth equation:
Equals 1.708 times 10 to the power of minus 3 cubic meters.

Sixth equation:
Work done, W, equals (crossed out term) times 1.3187 times 10 to the power of 5 Newtons per square meter, equals negative 0.2812 kilojoules, equals negative 0.2827 Joules.

Seventh equation:
The change in internal energy for gas, Delta U_g, equals 3.9787 times 10 to the power of minus 3 kg times (273.15 Kelvin minus 773.15 Kelvin) times 0.653 kilojoules per kg Kelvin, equals negative 1.0920 kilojoules.

Eighth equation:
Heat Q equals Delta U_g plus W, equals negative 1.3662 kilojoules, which is the heat released by the gas.

Ninth equation:
Q12 equals positive 1.3662 Joules.