a) What are \( p_{3,11} \) and \( m_g \)?

\( p_{3,11} \) equals the height from the top, since it is the center of gravity, divided by \( A \).

This implies that \( F \) equals \( p_{3,11} \times A \), which equals \( m \times g \times \frac{p_{amb}}{p_{Erd}} \).

The mass \( m_{Ew, S} \) equals \( 100 \times 3 \times \ldots \) and results in \( 314.501 \, \text{N} \).

The area \( A \) equals \( \pi \times \left(\frac{d}{2}\right)^2 \) which equals approximately \( 3 \ldots \times 10^{-4} \).

This implies that \( \frac{F_r}{A} + p_{amb} \) equals \( 1.4 \, \text{bar} \).

=> Ideal Gas Law

\( pV = mRT \) implies \( m = \frac{pV}{RT} \) which equals \( 3.418241 \, \text{g} \).

\( R \) equals \( \frac{R}{M_S} \) which equals \( 0.16628242 \).