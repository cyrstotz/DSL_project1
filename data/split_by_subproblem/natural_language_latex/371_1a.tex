a) Sought: Q dot out.

1. First Law of Thermodynamics for a Steady Flow Process.

The equation m dot times (h in minus h out) plus Q dot 12 equals Q dot out.

Q dot out equals Q dot 12 plus m dot times (h in minus h out).

h in equals h at T in equals 20 degrees Celsius, which equals h fg at 20 degrees Celsius, equals 2333.8 kilojoules per kilogram (from Table A-2).

h out equals h at T out equals 100 degrees Celsius, which equals h fg at 100 degrees Celsius, equals 2257.0 kilojoules per kilogram (from Table A-2).

Therefore, Q dot out equals 100 kilowatts plus 0.3 kilograms per second times (2333.8 kilojoules per kilogram minus 2257.0 kilojoules per kilogram).

This results in 123.24 kilowatts.

Burning with T KF equals 295 Kelvin, Q dot out equals 65 kilowatts.

Steady Flow Dynamics:

0 equals m dot times (s e minus s i) plus the sum of Q dot plus the sum of S dot ex.

s e equals s i, from Table A2:

s e equals 0.954 kilojoules per kilogram Kelvin, s i equals 1.306 kilojoules per kilogram Kelvin.

s e minus s i equals negative 0.352 kilojoules per kilogram Kelvin.

S dot ex equals Q dot over T plus Q dot out over T minus m dot times (s e minus s i).

This equals negative 165 kilojoules over 295 Kelvin minus 0.3 kilograms per second times negative 0.352 kilojoules per kilogram Kelvin over T, equals 664.32 kilowatts per Kelvin.