The equations provided are as follows:

1. The product of density (rho) and velocity (v) equals the gas constant (R) divided by the molar mass (M) times the temperature (T). The molar mass of air (M_L) is given as 28.52 kilograms per kilomole, as per reference TAD A1.

2. The rate of heat transfer (Q dot) equals the mass flow rate (m dot) times the difference in enthalpy between state 5 and state 6 plus half the difference of the squares of the velocities at state 5 and state 6, plus the adiabatic heat transfer rate (Q dot adiabatic) minus the rate of work done by losses (L dot d).

3. The rate of heat transfer (Q dot) equals the difference in enthalpy between state 5 and state 6 plus half the difference of the squares of the velocities at state 5 and state 6 minus the gas constant (R) divided by the molar mass of air (M_L) times the difference in temperatures between state 6 and state 5 divided by (1 minus n).

4. The temperature at state 6 (T_6) equals the temperature at state 5 (T_5) times the power of the ratio of the reference pressure (p_0) to the pressure at state 5 (p_5) raised to the exponent of the gas constant (R) divided by the specific heat at constant pressure (c_p). The temperature at state 6 is given as 368.07 Kelvin.

5. The rate of work at state 6 (W dot 6) equals the square root of the specific heat at constant pressure (c_p) times the difference in temperatures between state 5 and state 6 plus half the square of the velocity at state 5 minus two times the gas constant (R) divided by the molar mass of air (M_L) times the difference in temperatures between state 6 and state 5 divided by (1 minus n). The equation is not completed.