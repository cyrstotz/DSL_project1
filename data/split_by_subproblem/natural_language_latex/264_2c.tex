e subscript x, stc equals e subscript x, stc, zero equals e subscript x, stc, six equals the quantity h subscript zero one minus h subscript zero minus T subscript zero times the quantity s subscript zero one minus s subscript zero plus p subscript zero times the quantity v subscript zero one minus v subscript zero minus the quantity h subscript six minus h subscript zero minus T subscript zero times the quantity s subscript six minus s subscript zero plus p subscript zero times the quantity v subscript six minus v subscript zero equals h subscript zero one plus T subscript zero times the quantity s subscript zero minus s subscript zero one plus s subscript six minus s subscript zero minus h subscript six. e subscript x, stc equals h subscript zero one minus h subscript six plus T subscript zero times the quantity s subscript six minus s subscript zero one. Negative 30 degrees Celsius equals 273.15 minus 30 Kelvin equals 243.15 Kelvin.

Table A-22:
h subscript zero one equals 250.65 minus 240.02 times the fraction 24 times 3.75 minus 240 over 250 minus 240 plus 240.42 equals 244.7 kilojoules per kilogram. h subscript zero, we know.

R equals R bar over M subscript Luft. S subscript six minus S subscript zero equals S subscript six degree (6) minus S subscript six degree (0) minus R times the natural logarithm of the fraction P subscript two over P subscript one.

Interpolation:
S subscript six degree equals 1.54191 minus 1.47624 times the fraction 243.75 minus 240 over 250 minus 240 plus 1.47624 equals 1.4911 (result). c equals 1.49113.