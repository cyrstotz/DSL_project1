For part b):

The equation TdS equals delta Q.

The average temperature T times the difference in entropy (s2 minus s1) equals q12.

q12 equals h2 minus h1.

The average temperature T equals the difference in enthalpy (h2 minus h1) divided by the difference in entropy (s2 minus s1).

h2 minus h1 equals cif times the difference in temperature (T2 minus T1).

s2 minus s1 equals cif times the natural logarithm of the ratio of T2 to T1.

Therefore, the average temperature T equals the difference in temperature (T2 minus T1) divided by the natural logarithm of the ratio of T2 to T1, which equals 233.1216 Kelvin.

For part c):

A diagram is presented with vectors and lines indicating a system with temperatures and heat flows. The heat flow out is labeled as Qaus at different points, and temperatures are marked at 100 degrees Celsius and T_KF.

The wall is described as an adiabatic system boundary.

Entropy balance is given by the equation:
The ratio of Qaus over T_Reaktor minus the ratio of Qaus over T_KF plus S_erz equals zero.

S_erz equals Qaus times the difference between the reciprocal of T_KF and the reciprocal of T_Reaktor.

This results in 4.5946 times 10 to the power of negative 3 kilowatts per Kelvin, which equals 95.496 watts per Kelvin.