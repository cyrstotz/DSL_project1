The equations and their translations are as follows:

1. \( p_{g1} = \frac{F}{A} + p_{amb} \)
   - The pressure \( p_{g1} \) equals the force \( F \) divided by the area \( A \) plus the ambient pressure \( p_{amb} \).

2. \( A = \pi \left( \frac{d}{2} \right)^2 = 0.00785 \, \text{m}^2 \)
   - The area \( A \) equals pi times the square of half the diameter \( d \), which equals 0.00785 square meters.

3. \( F = F_{G,1} + F_{G,EW} = (m_t + m_{EW}) \cdot g = (32 \, \text{kg} + 0.7 \, \text{kg}) \cdot 9.81 = 319.901 \)
   - The force \( F \) equals the sum of \( F_{G,1} \) and \( F_{G,EW} \), which equals the total mass \( m_t \) plus the extra weight mass \( m_{EW} \) times the acceleration due to gravity \( g \), resulting in 319.901 Newtons.

4. \( p_{g1} = \frac{319.901}{0.00785} + 1.013 \cdot 10^5 = 140148 \, \frac{\text{N}}{\text{m}^2} \)
   - The pressure \( p_{g1} \) equals 319.901 divided by 0.00785 plus 101300, resulting in 140148 Pascals.

5. \( p \cdot V = n \cdot R \cdot T \Rightarrow n = \frac{p \cdot V}{R \cdot T} = 3.42 \, \text{g} \)
   - The product of pressure \( p \) and volume \( V \) equals the product of the amount of substance \( n \), the gas constant \( R \), and temperature \( T \). Solving for \( n \) gives 3.42 grams.

6. \( T = 500^\circ \text{C} = 773.15 \, \text{K} \)
   - The temperature \( T \) is 500 degrees Celsius, which equals 773.15 Kelvin.

7. \( p = 146148 \, \text{Pa} \)
   - The pressure \( p \) is 146148 Pascals.

8. \( V = 0.00314 \, \text{m}^3 \)
   - The volume \( V \) is 0.00314 cubic meters.

9. \( n = \frac{m}{M} \)
   - The amount of substance \( n \) equals the mass \( m \) divided by the molar mass \( M \).

10. \( R = \frac{R}{M} = \frac{8.314}{50} = 0.16628 \, \frac{\text{J}}{\text{mol} \cdot \text{K}} = 0.16628 \, \frac{\text{J}}{\text{g} \cdot \text{K}} \)
    - The specific gas constant \( R \) equals the universal gas constant divided by the molar mass \( M \), resulting in 0.16628 Joules per gram Kelvin.

11. \( T_{2g} = T_{2w} \quad \text{otherwise heat will still flow} \)
    - The temperature \( T_{2g} \) equals the temperature \( T_{2w} \), otherwise heat will continue to flow.

12. \( p_{S2} = p_{g1} \quad \text{Since the mass of } \Sigma W \text{ remains the same} \)
    - The pressure \( p_{S2} \) equals the pressure \( p_{g1} \) since the mass of the total system \( \Sigma W \) remains constant.

13. \( T_{ew} = 0^\circ C \quad \text{since otherwise no ice is left at equilibrium} \)
    - The temperature \( T_{ew} \) is 0 degrees Celsius since otherwise no ice would be left at equilibrium.

14. \( \text{thus} \quad T_{gr} = 0^\circ C \)
    - Thus, the temperature \( T_{gr} \) is 0 degrees Celsius.