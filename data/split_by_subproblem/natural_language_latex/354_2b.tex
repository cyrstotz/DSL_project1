w_s equals 220 meters per second.
p_3 equals 6.5 bar.

The mass flow rate, denoted as m dot, equals p_0 times v_0 divided by R times T_0, which is equal to p_0 times v_6 divided by R times T_6.

Note: The student has written some additional notes and equations around this part, but they are not clearly legible.

Since the thrust nozzle is adiabatically reversible, it holds that T_3 equals T_0 times the ratio of p_3 over p_0 raised to the power of (n minus 1) divided by n, which equals 328.07 Kelvin.

Note: The student has written "Since the nozzle is adiabatic, it holds that Q equals 0".

0 equals m dot times (h_e minus h_a) plus m dot times (w_2 squared minus w_1 squared) divided by 2.
w_2 squared divided by 2 equals c_p times (T_5 minus T_6) plus w_5 squared divided by 2.

w_0 equals the square root of 2 times c_p times (T_5 minus T_6 divided by 6) plus w_5 squared, which equals 506.02 meters per second.