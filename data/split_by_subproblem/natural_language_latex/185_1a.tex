The heat flow out, denoted as Q dot out, is equal to the mass flow rate, m dot, times the difference in enthalpy between h1 and h2, plus the heat flow rate Q dot R. This is mathematically represented as:

Q dot out equals m dot times (h1 minus h2) plus Q dot R.

Substituting the values, we get:

Q dot out equals 0.3 times (-125.67 kilojoules per kilogram) plus 100 kilojoules per second, which results in Q dot out equals 62.29 kilojoules per second.

For the entropy changes, the difference between s2 and s4 is given by the entropy at temperature T2 minus the entropy at temperature T3 minus the energy E1P divided by the temperature TP. Mathematically, it is:

(s2 minus s4) equals s superscript i of T2 minus s superscript i of T3 minus E1P over TP.

The difference between s2 and s3 is the entropy at temperature T4 minus the entropy at temperature T3, represented as:

(s2 minus s3) equals s superscript i of T4 minus s superscript i of T3.

The reciprocal of the average temperature, 1 over T bar, is equal to the constant c times the natural logarithm of the ratio of T2 to T3, expressed as:

1 over T bar equals c times ln (T2 over T3).

The entropy production rate, S dot EZ, is calculated as the heat flow rate Q dot ab divided by the temperature T ab minus the difference between T2 and the cooling temperature T KL divided by T KL. Substituting the values, we get:

S dot EZ equals 62.29 kilowatts times (373.15 minus 293.12) divided by (373.15 times 293.12), which results in S dot EZ equals 48.94 watts per Kelvin.