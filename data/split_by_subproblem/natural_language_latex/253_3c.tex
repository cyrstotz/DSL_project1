1. First Law of Thermodynamics at the Gas
- The change in internal energy from state 1 to state 2 is given by the equation:
  Delta U from 1 to 2 equals Q from 1 to 2 minus W from 1 to 2.
- Expressing this in terms of mass, specific heat at constant volume, and temperature change, we have:
  mass of gas times specific heat at constant volume times (temperature at state 2 of gas minus temperature at state 1 of gas) must equal Q from 1 to 2 minus W from 1 to 2.
- The work done from state 1 to state 2 is represented as:
  W from 1 to 2 equals the integral from 1 to 2 of pressure times differential volume.
- Simplifying this, it becomes:
  Gas constant times (temperature at state 2 of gas minus temperature at state 1 of gas).
- Numerically, this work is:
  negative 83.14 kilojoules.
- Therefore, the heat transferred from state 1 to state 2 is calculated as:
  mass of gas times specific heat at constant volume times (temperature at state 2 of gas minus temperature at state 1 of gas) minus 83.14 kilojoules.
- This results in:
  negative 82 kilojoules.