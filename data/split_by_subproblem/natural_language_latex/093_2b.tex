Energy Balance:
- The rate of heat transfer minus the rate of work done equals the mass flow rate times the difference in enthalpy between state 5 and state 6 minus half the difference in the squares of velocities at state 5 and state 6 plus the rate of power times the integral of the mass flow rate over the rate of heat transfer plus the rate of heat transfer.
- The difference in enthalpy between state 5 and state 6 equals the integral from temperature T1 to T2 of the specific heat at constant pressure times the differential of temperature minus the specific heat at constant pressure times the difference between temperature T2 and T1 divided by the difference between temperature T2 and T1.
- The ratio of velocity at state 2 to velocity at state 1 equals the gas constant times the ratio of the difference between temperature T2 and T1 over temperature T1 divided by the difference between temperature T2 and T1.
- The gas constant R equals the specific heat at constant pressure minus the specific heat at constant volume.
- The specific heat at constant pressure equals the specific heat at constant volume plus the gas constant.
- The gas constant R equals the specific heat at constant pressure times one minus the reciprocal of the heat capacity ratio, which equals 0.287 kilojoules per kilogram Kelvin.
- For an isentropic compression:
- The temperature at state 6 equals the temperature at state 5 times the ratio of pressure at state 6 to pressure at state 5 raised to the power of the heat capacity ratio minus one over the heat capacity ratio, which equals 526.634.

The equation hs times the ratio of initial velocity to the speed of light plus half the square of the initial velocity equals the efficiency times the difference between the initial temperature and the temperature at state s, which equals half the square of the initial velocity.