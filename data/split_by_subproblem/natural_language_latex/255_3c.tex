m1 equals m2 and p1 equals p2 because the same "weight" from the pressure piece.

This implies an isothermal polytropic change.

T2 equals T1 times the ratio of V1 over V2 raised to the power of n minus 1.

p2 times v2 equals R times T2, which is equivalent to v2 equals R times T2 divided by p2.

T2 equals T1 times the ratio of V1 over R times T2 divided by p2 raised to the power of n minus 1, which equals T1 times R times T2 divided by V1 times p2.

In Problem c:

T912 equals 0.003 degrees Celsius.
p2 equals 7.900 bar.
m2 equals 3.427 times 10 to the power of minus 3 kilograms.

First law of thermodynamics over gas: according to Lösens page 5/6m.

Change in internal energy U12 equals m times Q12 minus work W0, where kinetic energy and potential energy are negligible.
T2 equals 500 degrees Celsius.

Change in internal energy U12d equals C_V times (T2 minus T1) equals negative 376.49 kilojoules, assuming an ideal gas.

U12 equals m2 times change in specific internal energy u12 equals 1.0824 kilojoules equals underline Q12.