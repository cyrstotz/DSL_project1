T sub v equals 4 Kelvin over T sub sbl, which implies T sub v equals negative 18 degrees Celsius.
E equals 3.1, between 2 and 3 (stationary, adiabatic).
The mass flow rate from state 2 to state 3 in sbl, denoted as m dot sub 23sbl, equals the work rate sub v divided by the difference in enthalpy from state 2 to state 3, approximately equals 0.834 kilograms per second.
The enthalpy at state 2, denoted as h sub 2, equals the enthalpy at the fluid state at negative 16 degrees Celsius, approximately equals 237.74 kilojoules per kilogram.
The enthalpy at state 3, denoted as h sub 3, equals the enthalpy at the fluid state at 8 bar and entropy equals s sub 3, approximately equals 237.66 minus 24.75 times negative 0.9374 minus 0.9066, which simplifies to (0.9298 minus 0.9066) times 24.75 plus 237.37, approximately equals 277.37 kilojoules per kilogram.
The entropy at state 2 equals the entropy at state 3, denoted as s sub 2 equals s sub 3, equals the entropy at the fluid state at negative 16 degrees Celsius, approximately equals 0.9298 kilojoules per kilogram Kelvin.