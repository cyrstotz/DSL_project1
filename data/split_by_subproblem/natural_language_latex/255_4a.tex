The description is of a graph which is a phase diagram. The y-axis is labeled as "P (Druck)" which translates to "P (Pressure)" and the x-axis is labeled as "T (°C)" representing temperature. The graph delineates three distinct regions labeled "fest", "flüssig", and "gas" corresponding to the solid, liquid, and gas phases respectively.

The boundary between the solid and liquid phases is depicted by a curve that originates from the origin and curves upwards. The boundary between the liquid and gas phases is represented by a horizontal line that extends from the right end of the solid-liquid boundary. The boundary between the solid and gas phases is shown as a curve that starts from the origin and curves downwards to the right.

A point labeled "Tripel" indicates where all three phases meet. Additionally, there are two points marked on the graph: one in the solid region and one in the liquid region. These points are connected by a vertical line indicating a phase transition from solid to liquid.

The graph also features an arrow pointing from the liquid region to the gas region, labeled "kritischer Punkt", which translates to "critical point".