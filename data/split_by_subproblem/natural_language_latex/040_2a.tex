A graph is drawn with the x-axis labeled as s in units of N/kgK and the y-axis labeled as T in units of Kelvin. The graph contains three curves labeled 1, 2, and 3, each representing different processes. Points are marked on the curves as follows:
- Curve 1 has points 1, 2, 3, 4, 5, 6.
- Curve 2 has points 1, 2, 3, 4, 5, 6.
- Curve 3 has points 1, 2, 3, 4, 5, 6.
The points are connected with arrows indicating the direction of the process. The points are labeled with p0, p1, p2, and ps.

A table is presented with columns labeled Zst, T, s, and P. The rows of the table are filled as follows:
- Row 1: Zst is 1, T is negative 30 degrees Celsius, s is s1, P is p0.
- Row 2: Zst is 2, T is not specified, s is s3 greater than s1, P is p2 greater than p0.
- Row 3: Zst is 3, T is not specified, s is s2 greater than s2, P is p1 equals p2.
- Row 4: Zst is 4, T is not specified, s is s3 greater than s3, P is p4 less than p3.
- Row 5: Zst is 5, T is 98.9 Kelvin, s is s9 greater than s3, P is 0.Shar equals p4 times p5.
- Row 6: Zst is 6, T is not specified, s is s6 equals s5, P is p0.