It is given that the product of pressure (P) and volume (V) equals the product of mass (m), the gas constant (R), and temperature (T), and the mass of the gas is denoted as \( m_g \).

The area \( A \) is calculated as the square of 0.05 meters times pi, which equals \( 7.854 \times 10^{-3} \) square meters.

The force \( FF \) is calculated as 1 bar plus the force due to 32 kg under gravity divided by the area, plus the force due to 9.1 kg under gravity divided by the same area.

The pressure \( p \) is given as 100000 Pascals, which is equivalent to 1 bar.

Using the ideal gas law rearranged, the mass \( m \) is calculated as the product of pressure and volume divided by the product of the gas constant and temperature.

The specific gas constant \( Q \) is calculated as the gas constant \( R \) divided by the molar mass \( M \), which is \( 8.314 \) kilojoules per kilomole Kelvin divided by 50 inverse moles, resulting in \( 0.1663 \) inverse grams Kelvin.

Finally, the mass of the gas \( m_g \) is calculated using the pressure, the volume, the specific gas constant, and the temperature, resulting in \( 3.42 \) grams.