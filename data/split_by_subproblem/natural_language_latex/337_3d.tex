d) The variable x subscript "Eis,2".

The temperature T subscript "Eis" equals 0.003 degrees Celsius.

The pressure p subscript "Eis" equals the ambient pressure p subscript "amb" plus the product of m subscript "e" and U.

The variable x subscript "Eis,1" equals the ratio of m subscript "Eis" to m subscript "w", which is 0.6, leading to m subscript "Eis" equals 0.6 times 0.1 kilograms.

At 0 degrees Celsius equals T subscript "ew".

The difference u subscript 2 minus u subscript 1 equals c subscript "if" times the difference T subscript 2 minus T subscript 1.

This leads to u subscript 2 equals c subscript "if" times the difference T subscript 2 minus T subscript 1 plus u subscript 1.

This leads to the instruction to interpolate.

u subscript 2 equals 3345.

At a pressure of 1 bar and temperature of 0.003.

Interpolate for x.

x equals the ratio of u subscript 2 minus u subscript "f" to u subscript "g" minus u subscript "f".