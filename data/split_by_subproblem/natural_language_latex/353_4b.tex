The mass flow rate of R134A is 4 kilograms per hour, and the temperature T2 is minus 22 degrees Celsius.

The pressure p2 at minus 22 degrees Celsius is 1.2192 BAR.

The enthalpy h2 remains the same, implying that the temperature T1 equals T2.

The enthalpy h4 equals h1, indicating that the throttle is isenthalpic, and the pressure p3 equals p4.

The enthalpy h4 at 8 BAR is 93.42 kilojoules per kilogram.

The quality x1 is calculated as the ratio of the difference between h1 and the enthalpy of fluid at minus 22 degrees Celsius to the enthalpy of vaporization at minus 22 degrees Celsius, approximately equaling 0.337.

For problem b:
- The temperature T1 is minus 20 degrees Celsius.
- The heat transfer rate dot Q changes the temperature from T1 to T2, which is minus 16 degrees Celsius.
- The entropy s2 is 0.9298 kilojoules per kilogram Kelvin, and s2 equals s3.
- The enthalpy h2 at minus 16 degrees Celsius is 237.74 kilojoules per kilogram.
- The enthalpy h3 is calculated based on the conditions at 8 BAR and temperatures of 40 degrees Celsius and 31.33 degrees Celsius, adjusted by the entropy difference, resulting in approximately 271.31 kilojoules per kilogram.

In the energy balance for the compressor:
- The mass flow rate of R134A is approximately 3 kilograms per hour, calculated based on the negative product of 28 and the power dot W divided by the difference between h2 and h3.