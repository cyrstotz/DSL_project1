The equations are as follows:

1. The product of pressure \( p_{G1} \) and area \( A \) equals the product of the total mass \( m_{ges} \) and gravitational acceleration \( g \).
2. The product of pressure \( p \) and volume \( V \) equals the product of mass \( m \), gas constant \( R \), and temperature \( T \).
3. The mass \( m_{g1} \) is to be determined.
4. The pressure \( p_{g1} \) is to be determined.
5. The gas constant \( R \) is given as 0.05 meters.
6. The area \( A \) is equal to \( \pi \) times the square of the radius \( r \), which equals 0.00785 square meters.
7. The product of pressure \( p_{G1} \) and area \( A \) equals the sum of the product of the total mass \( m_{ges} \) and gravitational acceleration \( g \), the product of the mass of the load \( m_{L} \) and gravitational acceleration \( g \), and the product of area \( A \) and ambient pressure \( p_{amb} \).
8. The pressure \( p_{G1} \) is calculated to be 1.401 bar.
9. The product of pressure \( p_{G1} \) and volume \( V_{G1} \) equals the product of mass \( m_{g1} \), gas constant \( R \), and temperature \( T_{G1} \).
10. The mass \( m_{g1} \) is calculated as the quotient of the product of pressure \( p_{G1} \) and volume \( V_{G1} \) divided by the product of gas constant \( R \) and temperature \( T_{G1} \), which equals 3.422 grams.
11. The modified gas constant \( R_2 \) is calculated as the quotient of gas constant \( R \) divided by the molar mass \( \mu \), which equals 0.166 Joules per kilogram Kelvin.