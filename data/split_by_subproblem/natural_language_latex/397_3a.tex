The equation is P times V equals m times R times T.

R equals the universal gas constant R bar divided by the molar mass M sub g, which equals 8.314 Joules per mole Kelvin divided by 50 kilograms per kilomole, resulting in 0.16628 kiloJoules per Kelvin.

The pressure P sub g equals 1 bar plus the force F divided by the area A, which equals 1 bar plus 32 kilograms times 9.81 meters per second squared divided by the area calculated as pi times the square of half of 10 meters, resulting in a total of 1.45 bar.

The pressure P sub g+1 equals 1.5 bar.

The mass m sub g equals R times T divided by P times V, which equals 0.16628 kiloJoules per Kelvin times 500 degrees Celsius divided by 1.45 bar times 3.14, resulting in 2.92 grams.