Energy Balance:

The mass flow rate at the entrance, denoted as m-dot subscript 2, is at a temperature of 20 degrees Celsius.

The temperature in the dryer is 70 degrees Celsius.

The heat rate, Q subscript R, is 35 kilowatts.

The rate of change of energy, dE/dt, equals m-dot subscript 2 times h subscript m2 plus Q subscript R, which equals the energy at 100 degrees Celsius minus the energy at 70 degrees Celsius, which is equal to mass times (u subscript 100 minus u subscript 70).

The energy at 100 degrees Celsius, E subscript 100 degrees Celsius, equals U-dot subscript 100 plus PE plus U-dot equals mass times u subscript 100.

The energy at 70 degrees Celsius, E subscript 70 degrees Celsius, equals U-dot subscript 70 equals mass times u subscript 70.

Implying that m-dot subscript 2 times h subscript m2 plus Q-dot subscript R equals 57.55 times (Q-dot subscript R divided by T subscript R) equals 57.55 times ((1.337 minus 0.889) divided by 0.889) equals 2604.1 times 10 to the power of 3.

The ratio of S-dot at 100 degrees Celsius to S-dot at 70 degrees Celsius equals 1.3069 plus 0.005 times (3.345 minus 1.3069) equals 1.337414 (from Table A2).

The ratio of S-dot at 70 degrees Celsius to S-dot at 20 degrees Celsius equals 0.5545 plus 0.005 times (7.7553 minus 0.5545) equals 0.5808.

Implying that m-dot subscript 2 equals (Q subscript R minus 200 times 10 to the power of 3) divided by (8.3 times 10 to the power of 3) equals m-dot subscript 2 equals 4680 kilograms.