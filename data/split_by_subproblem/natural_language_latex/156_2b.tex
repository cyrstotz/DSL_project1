The variables are w_e1 and T_e1. The equation is zero equals m dot times the quantity h_5 minus h_6 plus half of w_5 squared minus half of w_6 squared plus W dot from 5 to 6. With W dot from 5 to 6 equals m dot times the integral from 5 to 6 of v dp plus delta pe, and this is isobaric. This implies that h_5 minus h_6 plus half of the difference between w_5 squared and w_6 squared equals zero. This further implies that w_5 squared equals two times the quantity h_6 minus h_5 plus w_6 squared. This is equal to the square root of two times c_p times the difference between T_e6 and T_e5 plus w_6 squared. The temperature T_e6 equals T_5 times the ratio of P_6 to P_5 raised to the power of (n-1) over n, which equals 328.07 Kelvin.