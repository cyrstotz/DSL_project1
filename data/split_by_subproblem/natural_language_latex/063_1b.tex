T equals negative the integral from s1 to s2 of T ds divided by s1 minus s2, which equals the difference of h_a and h_e divided by the difference of s_a and s_e, which further equals L times the difference of T_a and T_e divided by L times the natural logarithm of T_a over T_e.

The integral from e to e of T ds equals q.

1HS: q equals h_a minus h_e.

This equals 298.15 Kelvin minus 288.15 Kelvin divided by the natural logarithm of 298.15 Kelvin over 288.15 Kelvin.

This results in 293.12 Kelvin.

Graphical Description:

The diagram is a rectangular box with the following details:
- The top left corner is labeled as 70 degrees.
- The top right corner is labeled as 100 degrees.
- The bottom left corner is labeled as 298.15.
- The bottom right corner is labeled as 298.15.
- There is a horizontal line in the middle of the rectangle labeled as 100 degrees.
- An arrow pointing to the right in the middle of the rectangle is labeled as Q_zu.
- An arrow pointing downwards from the middle of the rectangle is labeled as min.