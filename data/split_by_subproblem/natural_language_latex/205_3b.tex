b)

- Forces Equilibrium:
  - The sum of F_z, F_1, and F_3 equals F_u.
  - The sum of F_z, F_1, and F_3 equals p_0 times A.
  - p_0 equals the sum of F_z, F_1, and F_3 divided by A.
  - p_0 equals 140,050 Pascals.
  - p_0,1 equals 1.4 bar.

- Ideal Gas Equation: p times V equals m times R times T.
  - m_g equals p_1 times V_1 divided by R times T_1, leading to V_1 equals 3.14 liters, which is also 3.14 cubic decimeters or 0.00314 cubic meters.
  - T_1 equals 773.15 Kelvin.

- R equals R_bar divided by M_g times Joules per Kelvin per mole, which equals 8.314 Joules per mole per Kelvin times 0.05 kilograms per mole, resulting in 166.28 Joules per kilogram per Kelvin.

- m_g equals 0.0034217 kilograms.

b)

Since the EK mixture is incompressible, the forces equilibrium remains the same.

- The equation for F_u is given by the sum of p_1 times A_1, p_2 times A_2, and p_3 times A_3 divided by A.
- p_1,2 equals the sum of F_1, F_2, and F_3 divided by A, which equals p_tot, and is 1.1 bar.

However, the volume and temperature change.

- The product of p and V equals mRT, leading to T equals T_GG.
- T_um equals 0 degrees Celsius.
- p_1 equals 1.1 bar as per Table 1.

In the first law of thermodynamics:

- m times (u_2 minus u_1) equals Q_12 minus W_dot_12^0.
- u_2 minus u_1 equals Q_12 divided by m, leading to u_2 minus u_1 equals c_v times (T_2 minus T_1).
- c_v times (T_2 minus T_1) equals Q_12 divided by m.
- T_2 minus T_1 equals Q_12 divided by m times c_v.
- T_2 equals T_1 plus Q_12 divided by m times c_v.