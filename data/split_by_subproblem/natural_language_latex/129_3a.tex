The equation is the fraction of m subscript k times S over A plus the fraction of m subscript E V times S over A plus P subscript o equals P subscript S7.

A equals the fraction of dE times D squared over 4 times pi.

This implies that P subscript S7 equals P subscript emo plus the fraction of 3A over A times the sum of m subscript k and m subscript E V equals 760 times d.

P subscript S7 equals 1.401 bar.

m subscript S equals the fraction of rho times V over R times T, which equals the fraction of P subscript S7 times V subscript S7 over R times T subscript S7, which equals the fraction of 1.401 times 3.14 times 10 to the power of negative 3 times 50 grams per mole over 9.81 grams per cubic centimeter times V subscript k times the sum of 50 plus 273.15 Kelvin.

This implies that m subscript S equals 3.42 grams.