The total pressure \( p_{\text{ges}} \) is equal to the mass \( m \) times gravity \( g \) divided by the area \( A \). Since balance must be maintained, the total pressure \( p_{\text{ges}} \) is equal to the pressure from EW. The total pressure \( p_{\text{ges}} \) is also the sum of the atmospheric pressure \( p_{\text{amb}} \) and the pressure due to the piston, which is \( \frac{mg}{A} \). This results in \( p_{\text{ges}} \) being \( 32 \, \text{kg} \times 9.81 \, \text{m/s}^2 \) divided by \( \pi \times (0.05 \, \text{m})^2 \) plus \( p_{\text{amb}} \), which equals \( 100000 \, \text{Pa} \) plus \( \frac{32 \, \text{kg} \times 9.81 \, \text{m/s}^2}{\pi \times (0.05 \, \text{m})^2} \, \text{Pa} \), resulting in \( 1.4 \, \text{MPa} \).

The mass of the gas \( m_{\text{gas}} \) times the velocity of the gas \( v_{\text{gas}} \) leads to the equation \( \frac{pV}{R/M_{\text{g}} \cdot T} \), which calculates to \( \frac{3.14 \, \text{kg}}{50} \times (500 + 273.15) \), resulting in \( 0.003429 \, \text{kg} \times 8.14 \) and \( 0.003429 \, \text{kg} \times 3.42 \, \text{J} \).

The inequality \( x \leq 2 \) is given.

The equation \( m_2 u_2 - m_1 u_1 = \cancel{Q_{12}} + \varnothing \) is set, where \( m_2 \) is defined as \( m_w + \frac{t}{x} - x \epsilon_w \) and \( m_1 \) is \( 0.1 \, \text{kg} \).

The internal energy \( u_2 \) at temperature \( T = 0.0^\circ C \) is \( u_{\text{flüssig}} - x \cdot u_{\text{flüssig}} + x \left( u_{\text{fest}} - u_{\text{flüssig}} \right) \), and \( u_1 \) at the same temperature is \( u_{\text{flüssig}} + 0.6 \left( u_{\text{fest}} - u_{\text{flüssig}} \right) \).

This results in \( 0.1 - 0.045 + 0.6 \left( -333.458 + 0.045 \right) \) which equals \(-20.01\).

\( m_2 \) is \( 0.1 \, \text{kg} + \left( 1 - x_2 \right) \cdot 0.1 - x \epsilon_w = 0.1 \left( 1 - x_2 \right) \) and \( u_2 \) is \(-0.045 + \varnothing x_2 \left( -333.4 \right) \).

The equation \( m_2 u_2 = Q_{12} + m_1 u_2 \) leads to solving for \( x \).