The first graph is a T-s diagram where the y-axis is labeled as T and the x-axis is labeled as s in units of kilojoules per kilogram. The graph features a closed loop with six points labeled 0, 1, 2, 3, 5, and 6. The paths between these points are labeled as follows: from point 0 to point 1, from point 1 to point 2, from point 2 to point 3, from point 6 to point 0, all are labeled as "isobar". The path from point 3 to point 5 is labeled as "isentrop", and the path from point 5 to point 6 is also labeled as "isobar". Additionally, the graph includes lines labeled \( p_2 = p_3 \), \( p_5 \), and \( p_0 = p_6 \).

The second graph is another T-s diagram with the y-axis labeled as T and the x-axis labeled as s in units of kilojoules per kilogram. This graph also contains a closed loop with six points labeled 0, 1, 2, 3, 4.5, and 6. The paths between these points are labeled as follows: from point 0 to point 1, from point 1 to point 2, from point 2 to point 3, from point 6 to point 0, all are labeled as "isobar". The path from point 3 to point 4.5 is labeled as "isentrop", and the path from point 4.5 to point 6 is labeled as "isobar". The graph also includes lines labeled \( p_2 = p_3 \), \( p_4 = p_5 \) (which is higher than \( p_0 = p_6 \)), and \( p_0 = p_6 \).

The third graph is a simplified version of the second graph, containing the same points labeled 0, 1, 2, 3, 4.5, and 6. The paths between the points are the same as in the second graph.