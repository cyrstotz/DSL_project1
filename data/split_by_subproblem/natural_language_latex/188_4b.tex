The average temperature T-bar is negative 89 degrees Celsius.

The temperature T1 is negative 85 degrees Celsius.

The pressures p1 and p2 are equal and calculated as negative 35 degrees Celsius plus 32 degrees Celsius, resulting in 0.7704 bar minus 0.6832 bar, then adding 0.6832 bar.

The calculation negative 32 degrees Celsius plus 32 degrees Celsius results in zero.

This equals 0.6835 bar.

The power dot W equals the mass flow rate dot m times the difference in enthalpy between h2 and h3. The mass flow rate dot m is calculated as the power dot W divided by the difference in enthalpy between h2 and h3.

Interpolation is done via the ratio of p1 to p2.

The entropy s2 is 0.88925 kilojoules per kilogram Kelvin.

The change in entropy Delta s is interpolated at 0.8 bar.

The enthalpy h2 is calculated using the formula (s2 minus 0.9821) divided by (0.9814 minus 0.9821), then multiplying the result by (282.1 minus 278.26) and adding 278.26.

The result of dividing one by zero is 280.348 per kilogram.

The mass flow rate dot m is calculated as negative 20 Joules per second divided by the difference between 226.03 kilojoules per kilogram and 280.948 kilojoules per kilogram, resulting in 0.514 per theta over S.