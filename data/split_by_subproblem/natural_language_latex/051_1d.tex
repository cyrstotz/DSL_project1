The section contains a series of equations and values related to a thermodynamics problem:

1. For the first set of values:
   - The temperature \( T_{R1} \) is 100 degrees Celsius.
   - The fraction \( x_D \) is 0.005.
   - The mass \( m_1 \) is 5755 kilograms.

2. For the second set of values:
   - The temperature \( T_{R2} \) is 70 degrees Celsius.
   - The fraction \( x_2 \) is 0.
   - The mass \( m_2 \) is equal to \( m_1 \) plus \( \Delta m_{12} \).

The system is described as a semi-open system.

The energy change \( \Delta E \) is given by the equation:
\[ \Delta E = m_2 u_2 - m_1 u_1 = \Delta m_{12} (h_{ein}) + Q \]
Further simplifying and rearranging gives:
\[ (m_1 + \Delta m_{12}) u_2 - m_1 u_1 = \Delta m_{12} (h_e) + Q_{R12} \]
\[ m_1 u_2 - Q_{R12} - m_1 u_1 = \Delta m_{12} (h_e) - \Delta m_{12} u_2 \]
\[ m_1 u_2 - Q_{R12} - m_1 u_1 = \Delta m_{12} (h_e - u_2) \]

For the input conditions \( h_{ein} \):
- The temperature \( T_{ein} \) is 20 degrees Celsius.
- The fraction \( x \) is 0.
- The pressure \( p_{ein} \) is 0.02339.
- The enthalpy \( h_{ein} \) equals \( h_f \) which is 83.96 kilojoules per kilogram, taken from table A-2.

For \( U_2 \):
- The fraction \( x_1 \) is 0.005.
- \( U_2 \) at 70 degrees Celsius is 292.95, also from table A-2.

For \( U_A \):
\[ U_A = x_1 u_g (100^\circ C) + (1 - x_1) u_f (100^\circ C) \]
\[ = x_1 \left( 2506.5 \frac{\text{kJ}}{\text{kg}} \right) + (1 - x_1) \left( 419.89 \frac{\text{kJ}}{\text{kg}} \right) \]
\[ = 429.37 \frac{\text{kJ}}{\text{kg}} \]

Finally, the mass change \( \Delta m_{12} \) is calculated as:
\[ \Delta m_{12} = \frac{m_1 u_2 + Q_{R12} - m_1 u_1}{h_e - u_2} \]
\[ = 3589.37 \text{ kg} \]