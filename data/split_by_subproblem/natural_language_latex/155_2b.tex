b) At the nozzle:
Zero equals the mass flow rate times the quantity of enthalpy at exit minus enthalpy at entry plus half the difference of the square of velocity at exit and the square of velocity at entry.
Zero equals enthalpy at state s minus enthalpy at state 6 plus half the difference of the square of velocity at state s and the square of velocity at state 6.
This implies that enthalpy at state s minus enthalpy at state 6 equals specific heat capacity at constant pressure times the difference in temperature between state s and state 6.

Graph Description:
The graph is a Temperature-Entropy (T-s) diagram. The x-axis is labeled entropy in kilojoules per kilogram Kelvin, and the y-axis is labeled temperature in Kelvin. The y-axis has the values 1284.6, 1231.0, and 243.15 marked.

There are several curves and lines drawn on the graph:
- A red line labeled "isobaric" runs vertically.
- The process path is marked with points labeled 0, 1, 2, 3, 4, 5, and 6.
- The path from 0 to 1 is a steep curve upwards.
- The path from 1 to 2 is a vertical line upwards.
- The path from 2 to 3 is a horizontal line to the right.
- The path from 3 to 4 is a steep curve downwards.
- The path from 4 to 5 is a horizontal line to the right.
- The path from 5 to 6 is a vertical line downwards.

The graph is annotated with the words "On the other side" on the right side.

Entropy at state s minus entropy at state 6 times the reciprocal of zero equals specific heat capacity at constant pressure times the integral from temperature at state 6 to temperature at state s of the reciprocal of temperature with respect to temperature minus the gas constant times the natural logarithm of the ratio of pressure at state s to pressure at state 6.
Zero equals 1.006 kilojoules per kilogram Kelvin times zero minus the gas constant times the natural logarithm of the ratio of pressure at state s to pressure at state 6.
The gas constant equals specific heat capacity at constant pressure minus specific heat capacity at constant pressure divided by n, which equals 0.2874 kilojoules per kilogram Kelvin.
Pressure at state s equals 0.5 bar and pressure at state 6 equals 0.191 bar.
0.276575 kilojoules per kilogram Kelvin equals 1.006 kilojoules per kilogram Kelvin times the natural logarithm of the ratio of 431.9 Kelvin to temperature at state 6.
0.274925 equals the natural logarithm of the ratio of 431.9 Kelvin to temperature at state 6.
1.316433 equals the ratio of 431.9 Kelvin to temperature at state 6.
This implies temperature at state 6 equals 328.08 Kelvin.
This implies enthalpy at state s minus enthalpy at state 6 equals specific heat capacity at constant pressure times the difference in temperature between state s and state 6, which equals 104.44 kilojoules per kilogram.
Negative 104.44 kilojoules per kilogram equals half the square of velocity at state s minus half the square of velocity at state 6.
The square of velocity at state 6 equals twice the sum of half the square of velocity at state s and 104.44 kilojoules per kilogram, which equals 48608.88.
This implies velocity at state 6 equals 220.47 meters per second.