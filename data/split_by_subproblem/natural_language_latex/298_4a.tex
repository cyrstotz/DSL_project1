The graph is a Pressure-Temperature (P-T) diagram. The y-axis is labeled as p (pressure) and the x-axis is labeled as T (temperature in degrees Celsius). The graph shows a curve starting from the origin (0,0) and rising upwards. There is a horizontal line labeled "melting curve" intersecting the y-axis. The curve intersects this horizontal line at a point labeled "triple point". To the left of the curve, the region is labeled "solid". To the right of the curve, the region is labeled "gas". The curve continues upwards and to the right, labeled "boiling curve". The intersection of the curve with the horizontal line is marked with a vertical line labeled "triple". The region between the horizontal line and the curve is labeled "liquid". The point where the curve ends on the right is labeled T_c. There are arrows indicating the direction of phase transitions: solid to liquid (left to right) and liquid to gas (upwards).