Piston and mass on the gas in the cylinder

**State:**
The initial exhaust gas temperature \( T_{EG,1} \) is 500 degrees Celsius, and the initial gas volume \( V_{g,1} \) is 3.74 liters, which is equivalent to 0.00374 cubic meters.
The mass of the exhaust gas \( m_{EG} \) is 0.5 kilograms, the initial environment temperature \( T_{EU,1} \) is 0 degrees Celsius, and the product of \( x \) and the force \( F_{m,1} \) equals the product of the mass of the environment \( m_{EU} \), gravitational acceleration \( g \), and 0.5.

**Graph Description:**
There is a graph with two axes. The horizontal axis is labeled with volume flow rate \( V \) in cubic meters per second, and the vertical axis is labeled with pressure \( P \) in Newtons per square meter. There are three lines drawn on the graph:
1. The first line starts at \( P_{g,1} \) and \( V_{g,1} \) and is labeled with the total force \( F_{ges} \) calculated as \( 10^5 \) Newtons per square meter plus the product of \( (0.5 \, kg \times 9.81 \, \frac{m}{s^2} + 32 \, kg \times 9.81 \, \frac{m}{s^2}) \) divided by \( 2.5 \times 10^{-3} \, m^2 \).
2. The second line is labeled with the total force \( F_{ges} \) as \( 4.9 \times 10^5 \) Newtons per square meter.
3. The third line is labeled with the total force \( F_{ges} \) as \( 1.1 \times 10^6 \) Newtons per square meter.

The volume \( V \) in cubic meters ranges from \( 3.74 \times 10^{-3} \) to an unspecified upper limit.

**Graph Description:**
There is a diagram showing a piston with a force \( F \) acting on it. The force is calculated as \( F = A \times (m_{EU} + m_{c}) \times 9.81 \, \frac{m}{s^2} \). The area \( A \) is calculated as \( A = \pi \times \left(\frac{5 \times 10^{-2} \, m}{2}\right)^2 = 7.853 \times 10^{-3} \, m^2 \).

**By the Law:**
The force \( F \) equals the total force \( F_{ges} \) in equilibrium.

The total force \( F_{ges} \) is the product of the initial gas pressure \( P_{g,1} \) and the area \( A \).

Thus, the initial gas pressure \( P_{g,1} \) is calculated as \( \frac{F_{ges}}{A} \), which equals \( \frac{10^5 \, \frac{N}{m^2} + (32 \, kg \times 9.81 \, \frac{m}{s^2})}{7.853 \times 10^{-3} \, m^2} \).

This results in a pressure of \( 1.4003 \times 10^5 \, \frac{N}{m^2} \) or \( 1.4003 \) bar.