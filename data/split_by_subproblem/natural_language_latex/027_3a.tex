Pressure consists of the weight of fluid plus weight plus outside pressure:

The pressure at g1 is equal to the mass times gravity divided by area plus the mass of the extra weight times gravity divided by area plus the ambient pressure:

p subscript g1 equals (m times g divided by A) plus (m subscript EW times g divided by A) plus p subscript amb.

Given:
- The diameter is 10 centimeters, which equals 0.1 meters.
- Therefore, the area A equals 0.0314 square meters.

Calculating the pressure:
- Equals (32.9 times 9.81 divided by 0.0314) plus (0.1 times 9.81 divided by 0.0314) plus 1 bar.
- Equals 1.1002 bar.

Mass of gas using the ideal gas law:
- The pressure times volume equals mass times the gas constant times temperature.
- Therefore, mass equals (pressure times volume) divided by (gas constant times temperature) equals (1.1002 times 10 to the power of 5 Pascals times 0.00314 cubic meters) divided by (8.314 times 10 to the power of 3 Joules per mole Kelvin times 773.15 Kelvin) equals 6.0027 kilograms, which approximately equals 2.69 grams.