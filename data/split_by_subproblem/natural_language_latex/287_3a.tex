- 32 kilograms, 0.1 times 9.81 kilograms, 10.5 kilograms, labeled as pomi. A.
- \( p_{g1}, A \)
- \( A \) equals 0.05 square meters times pi.
- \( p_{g1} \) equals the sum of 32 times 9.81, 0.1 times 9.81, and 10.5 times 9.81, all divided by 0.05 times pi.
- This equals 14 bar.
- The equation \( pV = mRT \).
- \( m_g \) equals \( p_{g1} \) times \( V_{g1} \) divided by \( RT \).
- \( R \) equals \( R \) divided by \( M_g \), which equals 8.314 Joules per mole Kelvin divided by 50 kilograms per kilomole.
- \( m_g \) equals 14 times 10 to the power of 5 times 3.14 times 10 to the power of -3, divided by \( \frac{83.14}{50} \) times 77.15, which equals 0.0639 kilograms or 3.19 grams.

- \( T_{g2} \) is greater than or equal to 0 degrees Celsius.
- \( T_{g2} \) equals 0 degrees Celsius, because the equilibrium (EW) is still in the two-phase region, the temperature remains the same, and since the whole system is in equilibrium, \( T_{g2} \) also equals 0 degrees Celsius.
- \( p_{g1} \) equals \( p_{g2} \) equals 14 bar.
- Since the equilibrium still has the same mass and the external temperature and the weight of the piston remain the same, the pressure in equilibrium also remains the same.