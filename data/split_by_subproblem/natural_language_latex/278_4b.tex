Given:
- \( p_2 \) equals \( p_2 \)
- \( p_3 \) equals \( p_4 \) equals 3 bar
- \( p_2 \) equals \( p_1 \) minus \( p \)
- \( h_4 \) equals \( h_2 \) equals \( h_f \) at 80 bar equals 93.42 kilojoules per kilogram
- \( s_3 \) equals \( s_2 \)

The heat transfer \( Q \) is equal to the mass flow rate of refrigerant \( \dot{m}_{\text{R}} \) times the difference between \( h_2 \) and \( h_3 \) minus the work \( \dot{W}_{\text{K}} \). The mass flow rate of refrigerant \( \dot{m}_{\text{R}} \) is equal to the work \( \dot{W}_{\text{K}} \) divided by the difference between \( h_2 \) and \( h_3 \), which calculates to \( \frac{-28 \text{kW}}{(237.74 - 253.31) \text{kJ/kg}} \) equals 1.80 kilograms per second.

\( h_2 \) is equal to \( h_g \) at 257.15 Kelvin, which is also \( h_g \) at -16 degrees Celsius, and equals 237.74 kilojoules per kilogram (reference A10).

\( T_2 \) equals \( T_2 \) which is \( T_i \) minus \( \Delta t \) equals \( T_{\text{isoll}} \) plus (0 minus 0) Kelvin equals (273.15 Kelvin minus 20 Kelvin) plus 46 Kelvin equals 257.15 Kelvin, which is -16 degrees Celsius.

\( h_3 \) at \( s_3 \) equals \( s_2 \) is 93.42 kilojoules per kilogram plus the fraction \( \frac{264.15}{264.15} \).
\( s_2 \) equals \( s_{g1} \) at -16 degrees Celsius, which is 0.9298 kilojoules per kilogram Kelvin, equals \( s_3 \) (reference A10).

\( h_3 \) at 80 bar and \( s_2 \) equals 264.15 kilojoules per kilogram, which calculates to \( \frac{273.66 - 264.15}{0.9298 - 0.8066} \) equals 253.74 minus 0.8066 equals 253.32 kilojoules per kilogram.