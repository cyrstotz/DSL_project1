The pressure \( p_{g1} \) is equal to the ambient pressure \( p_{amb} \) plus the product of the sum of the mass of the object \( m_k \) and the mass of the extra weight \( m_{EW} \), the acceleration due to gravity \( g \), and the reciprocal of the area \( A \). The area \( A \) is calculated as the square of the diameter \( D \) divided by four, times pi, which equals 0.00785 square meters. Substituting the values, \( p_{g1} \) is calculated as 1 bar plus the product of 32 kilograms plus 0.76 kilograms, \( g \), and the reciprocal of 0.00785 square meters, resulting in a total pressure of 7.4 bar.

The mass of the gas \( m_g \) is calculated using the formula \( m_g = \frac{p_{g1} V_1}{R T_0} m \), where \( V_1 \) is 3.794 times 10 to the power of negative 3 cubic meters, and \( R \) is the universal gas constant \( \overline{R} \) divided by 50 grams per mole, resulting in 0.7663 kilojoules per kilogram Kelvin. Substituting these values, the mass of the gas \( m_g \) is found to be 3.919 grams.