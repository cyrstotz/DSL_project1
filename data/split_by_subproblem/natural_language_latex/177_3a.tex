a) \( p_{s1} \), \( m_s \)

The equation \( pV = nRT \).

The constant \( R \) is calculated as \( R = \frac{\bar{R}}{M} = \frac{8.314 \, \frac{J}{mol \cdot K}}{0.05 \, \frac{kg}{mol}} \) which approximately equals \( 166.28 \, \frac{J}{kg \cdot K} \).

Force equilibrium:

Diagram: A box labeled "Gas \( p_{s1} \)" with arrows pointing down labeled \( F_{ew} \) and \( m_s g \) and an arrow pointing up labeled \( F_{ew} \).

The equation \( F_{ew} + p_{amb} A + m_s g = p_{s1} A \).

Substituting the values, \( 0.1 \, kg \cdot 9.81 \, \frac{m}{s^2} + 100000 \, Pa \cdot \pi (0.05 \, m)^2 + 9.81 \, \frac{m}{s^2} \cdot 0.32 \, kg = p_{s1} \cdot \pi (0.05 \, m)^2 \).

Solving for \( p_{s1} \), we get \( p_{s1} = 1.401 \, bar \).

Ideal Gas Law:

The equation \( pV = mRT \).

Calculating the mass of the gas \( m_g \), \( m_g = \frac{p_{s1} V_1}{R T_1} = \frac{140094 \, Pa \cdot 0.00314 \, m^3}{166.28 \, \frac{J}{kg \cdot K} \cdot 273.15 \, K} \).

The mass of the gas \( m_g \) is \( 3.42 \, g \).