The gas has absorbed heat \( Q \) to melt a part of the ice. Therefore, \( T_{g,2} = T_{g,1} \) implies \( p_{g,2} < p_{g,1} \). The volume and mass of the gas remain constant. In the equilibrium state, \( T_{eq,1} = T_{eq,2} \).

The rate of change of energy is given by the sum of the rates of change of potential energy (PE), kinetic energy (KE), and internal energy (U), which equals the rate of heat transfer minus the rate of work done, leading to the change in internal energy \( \Delta U = Q \).

The potential and kinetic energy of the gas is zero. The internal energy depends only on temperature, so the temperature change of the gas is the same as that of the system. The energy balance of the entire system is given by:

The rate of change of energy is the sum of the rates of change of potential energy, kinetic energy, and internal energy, which equals the rate of heat transfer minus the rate of work done, leading to the change in internal energy \( \Delta U = Q - W \).

The change in internal energy of the ice water is equal to the heat plus the work done on the system:

\[
\Delta U = Q + p_{eq,1} \cdot (V_{eq,2} - V_{eq,1})
\]

Note: \( V_{eq,2} = V_{eq,1} \) and the ice water is incompressible. Therefore, the heat transferred from the gas to the ice water is:

\[
\Delta U = Q
\]

System Ice:
The internal energy of ice at \(0^\circ C\).

Question 4:
The heat transfer \( Q_{12} \) is given by the product of mass \( m_{12} \), specific heat at constant volume \( C_v \), and the temperature difference \( T_2 - T_1 \), which approximately equals \(-1092.42 \, J\).

The units are given by:
\[
\left[ \frac{kg \cdot \frac{KJ}{kg \cdot K} \cdot K}{J} \right]
\]

With \( T_2 = 0.0003^\circ C \).