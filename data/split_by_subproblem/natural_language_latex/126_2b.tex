1. High School Nozzle:
   - Stationary
   - The equation is zero equals the total mass flow rate times the quantity of h5 minus h6 plus half the difference of W5 squared and W6 squared, minus Wt, assuming isentropic conditions.
   - h5 minus h6 plus half the difference of W5 squared and W6 squared equals Wt.
   - Wt equals the integral from 5 to 6 of v dp, under isentropic conditions.
   - The ratio of T6 over T5 equals the ratio of v6 over v5 raised to the power of (n-1) over n, under isentropic conditions, equals 0.93 times 9 times the quantity of 0.14 times 1.4 minus 1 over 0.5, which equals 323.67 Kelvin.
   - h5 minus h6 equals cp times the ratio of T5 over T6, equals 1.006 times the difference of 431.9 and 328.07, which equals 104.95 kilojoules per kilogram.
   - The change in h plus half W5 squared minus half W6 squared equals the integral from 5 to 6 of v dp plus the change in kinetic energy.
   - The change in h plus half W5 squared minus half W6 squared equals the integral from 5 to 6 of v dp minus half W5 squared plus half W6 squared.
   - The change in h plus W5 squared minus W6 squared equals the integral from 5 to 6 of v dp.
   - The change in h plus W6 squared plus c over 5 times the integral from 5 to 6 of v dp equals W6 squared.

W squared equals 220 meters per second squared, repeated for emphasis.

The integral from 5 to 6 of v dp equals negative n times the integral from 5 to 6 of v dV, which equals negative n times R times the difference of T6 and T5 over 1 minus n.

R subscript L equals R over M subscript L, equals 0.287 kilojoules per kilogram.

Negative 1 times u times 0.287 times 10 to the power of 3 times the difference of 328.07 and 293.15 over 1.14 equals negative 104.2 kilojoules per kilogram.

This implies substitution.