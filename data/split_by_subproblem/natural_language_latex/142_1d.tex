At a temperature of 70 degrees Celsius, the change in length is 30 Kelvin.
The heat released, denoted as Q_R, is equal to the heat output, Q_aus, which is 35 megajoules.
The internal energy at 400 degrees Celsius for the liquid phase, u_f, is 148.94 kilojoules per kilogram.
The internal energy at 400 degrees Celsius for the gas phase, u_g, is 2956.5 kilojoules per kilogram.
C_12 is calculated as the internal energy of the liquid phase plus 0.005 times the difference between the internal energy of the gas phase and the liquid phase, resulting in 4205.378 kilojoules per kilogram.
The internal energy at 70 degrees Celsius, u_2, is 293.25 kilojoules per kilogram.
The internal energy u_3 is 93.95 kilojoules per kilogram.

From this, the equation is derived:
Negative m_gas times u_1 plus u_2 times m_gas plus u_g times the change in mass from state 1 to 2 equals the heat output.

Further simplifying, the equation becomes:
The sum of m_gas and the change in mass from state 1 to 2 times u_3 equals m_gas times u_1 minus the change in mass times u_3 equals the heat output.

Rearranging gives:
The heat output minus m_gas times u_2 plus m_gas times u_1 equals the change in mass times the difference between u_2 and u_3.

Solving for the change in mass, denoted as Delta m, gives:
Delta m equals the quotient of the heat output minus m_gas times u_2 plus m_gas times u_1 divided by the difference between u_2 and u_3, resulting in 3589.106 kilograms per kilogram.