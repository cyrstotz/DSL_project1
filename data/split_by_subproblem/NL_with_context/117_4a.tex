Two diagrams are drawn:  

1. The first diagram is labeled "p(T)" and "p(Tief)" (pressure vs temperature). It shows a curve with points labeled 1, 2, and 4. Point 1 is connected to point 2 with a vertical line labeled "sotrop." Point 4 is connected to point 1 with a horizontal line.  

2. The second diagram is labeled "p(Mass)" (pressure vs mass). It shows a curve with points labeled 1, 2, 3, and 4. Point 1 is connected to point 2 with a curved line. Point 2 is connected to point 3 with a straight line, and point 3 is connected to point 4 with another curved line.  

Both diagrams have axes labeled "T(K)" (temperature in Kelvin).  

---

The diagram is a pressure-temperature (p-T) graph.  

- The vertical axis is labeled as "p [mbar]" (pressure).  
- The horizontal axis is labeled as "T [K]" (temperature).  
- There is a curve representing the phase regions.  

Key points and processes are marked:  
- Point 1 connects to Point 2 via a line labeled "isentrop."  
- Point 2 connects to Point 3 via a line labeled "isobar."  
- Point 3 connects to Point 4 via another "isobar" line.  

The curve ends with an arrow labeled "ND" pointing to the right along the temperature axis.