p sub g equals p sub amb plus F sub G divided by A  

F sub G equals 32 kilograms times 9.81 meters per second squared equals 313.92 Newtons  

A equals pi times (D divided by 2) squared equals 0.007853982 square meters  

p sub g equals 1 bar plus F sub G divided by A equals 10 to the power of 5 Pascals plus 39963.59 Pascals equals 1.4 bar  

To determine the mass of the gas, we can apply the ideal gas law:  

m sub g equals (p sub g times V sub g,1) divided by (R times T sub g,1)  

R equals R divided by M sub gas equals 0.16625 cubic meters times Pascals divided by kilograms times Kelvin  

m sub g equals (1.4 bar times 3.14 times 10 to the power of negative 3 cubic meters) divided by (0.16625 cubic meters times Pascals divided by kilograms times Kelvin times 773.15 Kelvin) equals 0.00342 kilograms  

m sub g equals 3.42 grams