Three diagrams are drawn, each representing thermodynamic processes in a T-s diagram:  

1. The first diagram shows curves labeled with isobars (constant pressure lines) and points marked as 1, 5, and 6. The isobar at 0.5 bar is labeled, and the isobar at \( p_0 \) is also labeled. The axes are labeled as \( T \) (temperature in Kelvin) and \( s \) (specific entropy in \( \text{kJ/kg·K} \)).  

2. The second diagram shows similar curves with points labeled 1, 5, and 6. The isobar \( p_0 = 0.191 \, \text{bar} \) is labeled, and the process is described as "isentrope" (adiabatic and reversible). The axes are labeled as \( T \) (temperature in Kelvin) and \( s \) (specific entropy in \( \text{kJ/kg·K} \)).  

3. The third diagram shows curves with points labeled 1, 5, and 6. The isobars are labeled \( p = p_2 = p_3 \) and \( p_0 = 0.191 \, \text{bar} \). The axes are labeled as \( T \) (temperature in Kelvin) and \( s \) (specific entropy in \( \text{kJ/kg·K} \)).  

---

The diagram is a temperature-entropy (T-S) graph labeled with the following details:  

- The y-axis is labeled as "T [°C]" (temperature).  
- The x-axis is labeled as "S [kJ/kg·K]" (entropy).  

The graph includes several points and processes:  
1. Point "0" at the bottom left, labeled "p0" (pressure at state 0).  
2. Point "1" labeled "p5, pA" and "Kompression" (compression).  
3. Point "2" labeled "p1, p3" and "isentrop adiabatic" (isentropic adiabatic).  
4. Point "3" labeled "Isobar" (isobaric).  
5. A process labeled "Isobar Mischung" (isobaric mixing) between points "4/5".  
6. Point "6" labeled "adiabat, reversibel" (adiabatic, reversible).  

The graph shows curves connecting these points, representing thermodynamic processes.  

Additional labels:  
- "Isotherm" (isothermal) is written along the left side of the graph.  
- "Irreversibel adiabatic" (irreversible adiabatic) is written near the top of the graph.