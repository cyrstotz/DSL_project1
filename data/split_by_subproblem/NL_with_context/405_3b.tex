p subscript g1 equals 1.5 times 10 to the power of 5 Pascal.  
m subscript g equals 3.6 kilograms (from the task statement).  

T subscript EW equals T subscript g, therefore equilibrium.  
Delta U subscript EW equals Q subscript 12.  
Delta U subscript g.  

U subscript 2EW minus U subscript 1EW equals u subscript g2 minus u subscript g1.  
u subscript 2EW plus 2000 times 32.8 divided by 3 equals c subscript v times open parenthesis T subscript 1 minus T subscript 2 close parenthesis.  

p subscript g2 equals p subscript g1, therefore T subscript g2 equals T subscript g1 times open parenthesis p subscript g2 divided by p subscript g1 close parenthesis to the power of open parenthesis 1 divided by n close parenthesis.  

T subscript g2 equals p subscript g2 times U subscript g2 divided by m subscript g times R.  

u subscript EW equals x times u subscript g2 plus open parenthesis 1 minus x close parenthesis times u subscript g1.  
u subscript 2EW equals 2000 times 32.8 divided by 3.  

u subscript g2 minus u subscript g1 equals integral from T subscript 1 to T subscript 2 of c subscript v times dT equals c subscript v times open parenthesis T subscript 1 minus T subscript 2 close parenthesis.  

n equals c subscript p divided by c subscript v equals c subscript p minus R divided by c subscript p minus R equals c subscript v plus R divided by c subscript v.  

n equals 1.66269.