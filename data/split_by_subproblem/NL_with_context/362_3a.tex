\( p_{g,1} = p_{amb} + \frac{F}{A} \)  
\( F = 32 \, \text{kg} \cdot 9.81 \, \text{m/s}^2 = 314 \, \text{N} \)  
\( A = \pi \cdot r^2 = \pi \cdot (0.05 \, \text{m})^2 \)  
\( \frac{F}{A} = \frac{314}{0.00785} = 40 \, \text{kPa} \)  
\( p_{g,1} = p_{atm} + \frac{F}{A} = 1.14 \, \text{bar} \)  

\( pV = mRT \)  
\( R = \frac{R_u}{M} = \frac{8.314}{0.05} = 166.28 \)  
\( m_g = \frac{pV}{RT_1} = \frac{1.14 \cdot 10^5 \cdot 0.00314}{166.28 \cdot 773} = 3.42 \, \text{g} \)

Energy balance, with kinetic energy equals zero and potential energy equals zero:  
Change in internal energy equals heat minus work.  
Work equals zero because the process is constant volume.  

Therefore, heat equals change in internal energy, which equals mass times integral of specific heat capacity times change in temperature.  

Heat equals 0.003142 kilograms times (500 minus 0) times 0.632 kilojoules per kilogram Kelvin, which equals 1082.14 joules.

x equals u subscript 2 minus u subscript fest divided by u subscript flüssig minus u subscript fest equals 0.963