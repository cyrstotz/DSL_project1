A graph is drawn with the y-axis labeled as "T [K]" and the x-axis labeled as "S [kJ/kg·K]". The graph represents a T-S diagram with six points labeled as 0, 1, 2, 3, 4, 5, and 6.  
- Point 0 is at the bottom left, connected to point 1 by a vertical line.  
- Point 1 is connected to point 2 by a diagonal line.  
- Point 2 is connected to point 3 by a curved line labeled "isentropic".  
- Point 3 is connected to point 4 by a vertical line labeled "isobaric".  
- Point 4 is connected to point 5 by a diagonal line labeled "m_M".  
- Point 5 is connected to point 6 by a curved line labeled "isentropic".  
The pressure at point 0 is labeled as "0.191 bar".

O equals m dot multiplied by (s zero minus s six) plus Q dot divided by T B plus S dot erz.  

s erz equals s six minus s zero minus q B divided by T B equals c p multiplied by the natural logarithm of (T six divided by T zero) minus q B divided by T B equals negative six hundred twenty-five point seven three kilojoules per kilogram.  

Rewrite positive sign!  

ex verl equals s erz multiplied by T zero equals negative one hundred fifty-two point one four four kilojoules per kilogram.