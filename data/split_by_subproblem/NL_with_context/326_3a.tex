\( p_{g,1} = \frac{m_K \cdot g}{A} \)  
\( A = 25 \, \text{cm}^2 \cdot \pi = 78.5 \, \text{cm}^2 \)  
\( m_{\text{ges}} = 32 \, \text{kg} + 0.1 \, \text{kg} \, (m_{\text{EW}}) = 32.1 \, \text{kg} \)  
\( p_{g,1} = 1 \, \text{bar} + \frac{32.1 \, \text{kg} \cdot 9.81 \, \text{m/s}^2}{78.5 \, \text{cm}^2} = 10^5 \, \text{Pa} + 40.1 \cdot 10^3 \, \text{Pa} = 1.4 \, \text{bar} \)  

\( U_g = c_V \cdot m_g \cdot T_g \)  
\( c_V = 0.633 \, \text{kJ/kg·K} \)  
\( R = 8.314 \, \text{kJ/kmol·K} \)  
\( M_g = 50 \, \text{kg/kmol} \)  
\( R_g = \frac{8.314 \, \text{kJ/kmol·K}}{50 \, \text{kg/kmol}} = 0.16628 \, \text{kJ/kg·K} \)  
\( m_g = \frac{p_g \cdot V_g}{R_g \cdot T_g} = \frac{1.4 \, \text{bar} \cdot 3.14 \, \text{L}}{0.16628 \, \text{kJ/kg·K} \cdot 773.15 \, \text{K}} = 3.42 \, \text{g} \)  
\( U_g = 0.633 \, \text{kJ/kg·K} \cdot 773.15 \, \text{K} \cdot 3.42 \, \text{g} = 1.67 \, \text{kJ} \)  

\( U_{g,2} = U_g \)  
\( p_g = p_{\text{EW}} = 1.4 \, \text{bar} \, \text{im thermodyn GGW} \)  
\( T_{\text{EW}} = 0.000^\circ \text{C} \)  
\( Aus der Tabelle ist ersichtlich, dass wir bei \( x = 0.003^\circ \) \)  
\( noch einen festen Anteil im Gemisch. \)  

\( U_{\text{EW}} = (1 - x) \cdot U_{\text{FL}} + x \cdot U_{\text{Fest}} \)