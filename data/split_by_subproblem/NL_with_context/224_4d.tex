e subscript k, v, u equals T subscript 5 multiplied by S subscript erz equals T subscript 5 multiplied by m subscript erz multiplied by c subscript p multiplied by ln (T subscript 6 divided by T subscript 5).  

Theta equals m subscript erz multiplied by (s subscript 5 minus s subscript 6) plus S subscript erz.  

s subscript 5 minus s subscript 6 equals c subscript p multiplied by ln (T subscript 5 divided by T subscript 6).

e sub K equals Q dot sub K divided by (Q dot sub W minus Q dot sub K).  
Theta equals m dot sub R times (h4 minus h3) minus Q dot sub 30.  
Arrow indicates Q dot sub 30 equals 0.1897 kilowatts.  
h4 equals 0.342 kilojoules per kilogram.  
h3 equals h sub g (8 bar) equals 260.15 kilojoules per kilogram.  
Q dot sub K equals m dot times (h2 minus h1).  

A circle with "A11" is drawn at the bottom.