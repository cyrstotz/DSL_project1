Ideal Gas Law  

\( pV = mRT \rightarrow m = \frac{pV_1}{RT_1} \)  

\( m = \frac{1.1401 \, \text{bar} \cdot 10^5 \, \text{Pa/bar} \cdot 3.14 \, \text{L} \cdot 10^{-3} \, \text{m}^3/\text{L}}{166.3 \, \text{J/kg·K} \cdot (273.15 + 500 \, \text{K})} \)  

\( R = \frac{R_T}{M_g} = \frac{8.314 \, \text{J/mol·K}}{50 \, \text{kg/kmol}} = 166.3 \, \text{J/kg·K} \)  

\( m = 0.00362 \, \text{kg} \approx 3.62 \, \text{g} \)  

---

U subscript 1 EW equals U subscript ice open parenthesis EW close parenthesis plus X subscript 1 times U subscript ice open parenthesis EW close parenthesis minus U subscript water open parenthesis EW close parenthesis.  

Open parenthesis EW close parenthesis equals P subscript amb plus AK equals 1.14 bar.  

U subscript 1 EW equals negative 0.045 times 10 superscript 3 joules per kilogram plus 0.60 times open parenthesis negative 333.458 joules per kilogram plus 0.033 times 10 superscript 3 joules per kilogram close parenthesis.  

Equals negative 200.08 joules per kilogram.