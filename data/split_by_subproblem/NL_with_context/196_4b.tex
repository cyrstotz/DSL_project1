The following calculations and notes are written:  
- \( \dot{m}_{R134a} = ? \)  
- "Isobare Verdampfung R134a" (Isobaric evaporation of R134a).  
- \( W_k = 28 \, \text{W} \).  
- \( T_{\text{Verdampfer}} = T_i - 6 \, \text{K} = 257.15 \, \text{K} \).  
- \( T_i = -10^\circ \text{C} = 263.15 \, \text{K} \).  
- "Energiebilanz über Verdichter" (Energy balance over compressor):  
  \( 0 = \dot{m} (h_e - h_a) + \dot{Q} + W_{e,n} \).  
  - \( \dot{Q} = 0 \) (adiabatic).  
  - \( \dot{m} = \frac{W_k}{h_e - h_a} = \frac{-0.085 \, \text{kW}}{h_e - h_a} \).  

Further notes and crossed-out calculations:  
- \( h_e = ? \), \( T = -16^\circ \text{C} \) from R134a.  
- \( h_2 = h_e \), \( x_2 = 1 \).  
- "Tabelle A-10" (Table A-10).  
- \( T@-16^\circ \text{C}, h_g = 237.94 \, \text{kJ/kg} \), \( h_f = 3 \, \text{kJ/kg} \).  
- \( h_2 = h_e \).  
- \( h_a = ? \).

Table A-11  
p equals 8 bar  
hf equals 93.92 kilojoules per kilogram  
hg equals 269.15 kilojoules per kilogram  

adiabatic reversible  
Delta S equals 0  
s1 equals s3  
s2 equals ?  

Table A-10  
s at negative 10 degrees Celsius equals sg equals 0.9238 kilojoules per kilogram Kelvin  

Table A-12  
p equals 8 bar  
s at Tsat equals 0.5066  
s at 60 degrees Celsius equals 0.9379  

h3 equals h at 60 degrees Celsius minus h at saturation  
divided by s at 60 degrees Celsius minus s at saturation  
multiplied by (s minus s at saturation) plus h at saturation  

h3 equals  
273.66 kilojoules per kilogram minus 269.15 kilojoules per kilogram  
divided by 0.9379 kilojoules per kilogram Kelvin minus 0.5066 kilojoules per kilogram Kelvin  
multiplied by (0.9238 kilojoules per kilogram Kelvin minus 0.5066 kilojoules per kilogram Kelvin)  
plus 269.15 kilojoules per kilogram  

equals 271.313 kilojoules per kilogram  

m dot equals negative 0.028 kilowatts  
divided by 237.74 kilojoules per kilogram minus 271.313 kilojoules per kilogram  

equals 0.8333 kilograms per second  

---

Energy balance over compressor:  
Q equals m dot multiplied by (he minus ha) equals W dot in  

m dot equals W dot in divided by (he minus ha)  

he equals h2  
ha equals h3  

h2 from Table A-10  
T at 16 degrees Celsius: hg equals 237.74 kilojoules per kilogram