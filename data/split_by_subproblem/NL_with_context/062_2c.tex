h subscript t minus h subscript 0 equals the integral from T subscript 0 to T subscript 5 of c subscript p dT equals c subscript p times (T subscript 5 minus T subscript 0) equals 1.006 kilojoules per kilogram Kelvin times (431.9 Kelvin minus 243.15 Kelvin) equals 190.45 kilojoules per kilogram.  

w subscript 0 equals the square root of 2 times (h subscript t minus h subscript 0) plus w subscript 0 squared equals the square root of 2 times 104.45 times 10 to the power of 3 joules per kilogram plus (200 meters per second squared) equals 498.80 meters per second.  

e subscript x subscript str comma 6 minus e subscript x subscript str comma 0 equals h subscript 6 minus h subscript 0 minus T subscript 0 times (s subscript 6 minus s subscript 0) plus k subscript e subscript 6 minus k subscript e subscript 0.  

h subscript 6 minus h subscript 0 equals c subscript p times (T subscript 6 minus T subscript 0) equals 1.006 kilojoules per kilogram Kelvin times (382.01 Kelvin minus 243.15 Kelvin) equals 85.43 kilojoules per kilogram.  

s subscript 6 minus s subscript 0 equals c subscript p times ln(T subscript 6 divided by T subscript 0) minus R times ln(p subscript 6 divided by p subscript 0) equals 1.006 kilojoules per kilogram Kelvin times ln(382.01 divided by 243.15) minus 0.30 kilojoules per kilogram Kelvin.  

Delta k subscript e subscript 0 equals one-half times w subscript out squared.  

Delta e subscript x subscript str equals 85.43 kilojoules per kilogram minus 243.15 Kelvin times (0.3 kilojoules per kilogram Kelvin) plus (498.80 meters per second squared minus 200 meters per second squared) divided by 2.  

Equals 116.63 kilojoules per kilogram.