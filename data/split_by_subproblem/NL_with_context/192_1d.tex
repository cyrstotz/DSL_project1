State 1:  
\( m_1 = 5755 \, \text{kg} \)  
\( T_1 = 100^\circ \text{C} \)  
\( x_1 = 0.005 \)  

State 2:  
\( m_2 = m_1 + \Delta m \)  
\( T_2 = 70^\circ \text{C} \)  
\( x_2 = 0.005 \)  

First Law of Thermodynamics:  
\( m_2 u_2 - m_1 u_1 + \Delta KE + \Delta PE = \Delta m h_{hm} + \sum Q - W \)  
\( Q_{in} = Q_{out} \), \( V = \text{constant} \)  

Simplified:  
\( \Delta m = \frac{m_2 u_2 - m_1 u_1}{h_{hm}} \)  

\( u_1 \): From A-2  
\( T = 100^\circ \text{C}, x = 0.005 \)  
\( u_1 = u_f + 0.005 \cdot (u_g - u_f) \)  
\( u_f = 418.94 \, \frac{\text{kJ}}{\text{kg}} \)  
\( u_g = 2506.5 \, \frac{\text{kJ}}{\text{kg}} \)  
\( u_1 = 428.38 \, \frac{\text{kJ}}{\text{kg}} \)  

\( u_2 \):  
\( u_2 = u_f(70) + 0.005 \cdot (u_g(70) - u_f(70)) \)  
\( u_f(70) = 292.95 \, \frac{\text{kJ}}{\text{kg}} \)  
\( u_g(70) = 2968.6 \, \frac{\text{kJ}}{\text{kg}} \)  
\( u_2 = 303.8 \, \frac{\text{kJ}}{\text{kg}} \)  

\( h_{hm} \):  
\( T = 20^\circ \text{C}, x = 0.005 \)  
\( h_{hm} = h_f(20) + x \cdot (h_g(20) - h_f(20)) \)  
\( h_f(20) = 83.96 \, \frac{\text{kJ}}{\text{kg}} \)  
\( h_g(20) = 2538.1 \, \frac{\text{kJ}}{\text{kg}} \)  
\( h_{hm} = 96.23 \, \frac{\text{kJ}}{\text{kg}} \)

Delta m twelve equals (m one plus Delta m twelve) u two minus m one u one divided by h two.  

Delta m h two equals m one u two plus Delta m twelve u two minus m one u one.  

Delta m (h two minus u two) equals m one u two minus m one u one.  

Delta m twelve equals m one (u two minus u one) divided by h two minus u two.  

Delta m twelve equals 3452 kilograms.  

---