Two diagrams are drawn:

1. **Top Diagram**:  
   - The axes are labeled as follows:  
     - Vertical axis: "T [K]"  
     - Horizontal axis: "s [kJ/kg·K]"  
   - The diagram shows a process with numbered points labeled as 1, 2, 3, and 4.  
   - Arrows indicate the direction of the process between these points.  
   - There are intersecting lines across the diagram, and a smaller shape is drawn in the upper right corner with points labeled 1, 2, and 4.

2. **Bottom Diagram**:  
   - The axes are labeled as follows:  
     - Vertical axis: "T [K]"  
     - Horizontal axis: "s [kJ/kg·K]"  
   - The diagram shows a closed cycle with numbered points labeled as 1, 2, 3, 4, and 5.  
   - Arrows indicate the direction of the process between these points.  
   - Two segments of the cycle are labeled "isobar."

\( E_{x,verl} \) equals \( \dot{m} \) times \[ \( h_e - h_a - T_0 (s_e - s_a) + \delta h_e \) \] plus the summation over \( i \) of \[ \( (1 - \frac{T_0}{T_i}) \dot{Q}_j \) \]. This equals \( e_{x,verl} \dot{m} \).  

Zero equals \( \dot{m} \) times \[ \( h_e - h_a + \delta h_e \) \] plus the summation over \( j \) of \( \dot{Q}_j \).  

Therefore:  

\( E_{x,verl} \) equals \( \dot{m} \) times \[ \( -T_0 (s_e - s_a) \) \] plus the summation over \( i \) of \[ \( (1 - \frac{T_0}{T_i}) \dot{Q}_j \) \] minus the summation over \( j \) of \( \dot{Q}_j \).  

This equals \( e_{x,verl} \dot{m} \).  

\( e_{x,verl} \) equals \[ \( -T_0 (s_e - s_a) \) \] plus the summation over \( i \) of \[ \( (1 - \frac{T_0}{T_i}) \dot{Q}_j \) divided by \( \dot{m} \) \] minus the summation over \( j \) of \( \dot{Q}_j \).