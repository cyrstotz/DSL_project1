First, we need to calculate \( u_f \).  

Analogy to vapor: \( x_{EW} \) is the same as \( x_{gas} \), all values from Table 1 up to 2 bar.  

\( u_f = u_f + x_{EW} \cdot (u_{vapor} - u_{fluid}) = (-0.045 + 0.6 \cdot (-333.859 + 0.045)) \, \text{kJ/kg} \).  

\( u_f = -200.0928 \, \text{kJ/kg} \).  

Contribution of positive heat flux:  
\( Q_{12} \) = No heat transfer in EW.  

Energy balance in the gas:  
\( m_{EW} \cdot (u_2 - u_1) = Q_{12} - Q_{12,EW} \).  

\( u_2 = u_1 + \frac{Q_{12}}{m_{EW}} \).  

\( u_2 = -200.0928 \, \text{kJ/kg} + \frac{1.3675 \, \text{kJ}}{0.1 \, \text{kg}} = -186.4 + 13.675 \, \text{kJ/kg} = -186.4 + 178 \, \text{kJ/kg} = u_2 \).  

After \( x_2 \) is analyzed:  
\( x_2 = \frac{u_2 - u_f}{u_{vapor} - u_f} \).  

Values from Table 1, since we are still in vapor and \( p = p_2 \), \( T_1 = T_2 \).  

\( x_2 = \frac{-186.478 + 0.045}{-333.859 + 0.045} = 0.555 = x_2 \).