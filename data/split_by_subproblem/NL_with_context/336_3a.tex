A table is drawn with columns labeled \( T \), \( P \), \( V \), and \( x_{ice} \).  

For the gas:  
- Row 1: \( T = 500^\circ \text{C} \), \( P = 740 \, \text{mbar} \), \( V = \text{empty} \), \( x_{ice} = \text{empty} \).  
- Row 2: \( T = T_{end} \), \( P = \text{empty} \), \( V = \text{empty} \), \( x_{ice} = \text{empty} \).  

For the ice-water mixture (EW):  
- Row 1: \( T = 0^\circ \text{C} \), \( P = \text{empty} \), \( V = v_{1} \), \( x_{ice} = 0.6 \).  
- Row 2: \( T = T_{end} \), \( P = \text{empty} \), \( V = v_{1} \), \( x_{ice} = \text{empty} \).  

Below the table:  
\( p = \frac{F}{A} \), where \( A \) is the area of the piston.  

\( A \):  
\( D = 10 \, \text{cm} = 0.1 \, \text{m} \).  
\( A = \left( \frac{0.1 \, \text{m}}{2} \right)^2 \cdot \pi = 0.00785 \, \text{m}^2 \).  

\( F = m \cdot a = (m_{K} + m_{EW}) \cdot g = 374.9 \, \text{N} \).  

\( p_{g} = p_{amb} + \frac{F}{A} = 740 \, \text{mbar} = 740 \cdot 10^2 \, \text{Pa} \).  

Mass:  
\( p \cdot V = m \cdot R \cdot T \).  
\( R = \frac{R_u}{M} = 0.16628 \).  
\( m = \frac{p \cdot V}{R \cdot T} = 3.422 \, \text{kg} \).  

Volume:  
\( V = 3.14 \, \text{L} = 3.14 \, \text{dm}^3 = 0.00314 \, \text{m}^3 \).