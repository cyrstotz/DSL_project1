A graph is drawn with the y-axis labeled as "p (mbar)" and the x-axis labeled as "T (°C)". The graph shows a curve labeled "gas" and includes points labeled "Triple point", "1", "2", and "3". The curve transitions from "fest" (solid) to "flüssig" (liquid). The pressure values indicated are 0.5 mbar and 1 mbar.

h3 equals h (8 bar, s equals 0.9298 kilojoules per kilogram Kelvin) → TAB A12  

h (5 equals 0.9066 kilojoules per kilogram Kelvin) equals 264.15 kilojoules per kilogram  

h3 (s equals 0.9298 kilojoules per kilogram Kelvin) equals ? equals (273.66 kilojoules per kilogram minus 264.15 kilojoules per kilogram) divided by (0.9298 kilojoules per kilogram Kelvin minus 0.9066 kilojoules per kilogram Kelvin) times (0.9298 kilojoules per kilogram Kelvin minus 0.9066 kilojoules per kilogram Kelvin) plus 264.15 kilojoules per kilogram  

h (s equals 0.9399 kilojoules per kilogram Kelvin) equals 273.66 kilojoules per kilogram  

ṁ_R134a equals negative 28 times 10 to the power of negative 3 kilowatts divided by (h2 minus h3) equals 0.00083 kilograms per second equals 3.0024 kilograms per hour