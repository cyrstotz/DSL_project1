Given: \( p_{g,1} \)  

\( V_{g,1} = 3.14 \, \text{L} = 3.14 \cdot 10^{-3} \, \text{m}^3 \)  
\( T_{g,1} = 500^\circ \text{C} = 773.15 \, \text{K} \)  

\( \frac{p V}{m} = R T \)  

\( p_{g,1} \):  
\[ F_{\text{Kolben}} = m_K \cdot g + p_{\text{amb}} \cdot A = p_g \]  

\[ A = \pi \cdot r^2 = \pi \cdot \left(\frac{d}{2}\right)^2 = \pi \cdot \left(\frac{10 \cdot 10^{-2}}{2}\right)^2 \, \text{m}^2 \]  
\[ = \pi \cdot \frac{1}{400} \, \text{m}^2 \]  

\[ (m_K + m_{\text{EW}}) \cdot g \cdot A + p_{\text{amb}} = p_{g,1} \]  
\[ = 32.1 \, \text{kg} \cdot 9.81 \, \text{m/s}^2 \cdot \frac{\pi}{400} \, \text{m}^2 + 1.10^5 \, \text{Pa} \]  
\[ = 100002.42 \, \text{Pa} \]  

\[ m = \frac{p V}{R T} \]  

\[ R = \frac{R}{M} \]  
\[ M = 50 \, \text{kg/kmol} \]  
\[ R = \frac{100.28}{50} \]  

\[ \approx 1.13 \, \text{m} \]

m equals p times V divided by R times T.  
m equals 0.00244 kilograms equals 2.14 grams.  
m equals 0.1402 kilograms.  

R equals 162.28 joules per kilogram per kelvin.  
T equals 77 degrees Celsius equals 350.15 kelvin.  
p equals 1 bar equals 10 to the power of 5 pascals.  
V equals 3.14 times 10 to the power of negative 3 cubic meters.