F equals open parenthesis m subscript K plus m subscript EW close parenthesis multiplied by g equals open parenthesis 32 kilograms plus 0.1 kilograms close parenthesis multiplied by 9.81 newtons per kilogram equals 314.9 newtons.  

F subscript atm equals p subscript amb multiplied by A equals 10 to the power of 5 pascals multiplied by open parenthesis pi multiplied by open parenthesis 0.05 meters close parenthesis squared close parenthesis equals 785.4 newtons.  

Force equilibrium:  
F plus F subscript atm equals F subscript n.  

Therefore, p subscript 1 equals F subscript n divided by A equals open parenthesis 314.9 newtons plus 785.4 newtons close parenthesis divided by open parenthesis pi multiplied by open parenthesis 0.05 meters close parenthesis squared close parenthesis equals 1.4 times 10 to the power of 5 pascals.  

Ideal gas:  
p multiplied by V equals m subscript g multiplied by R multiplied by T.  

Therefore, m subscript g equals p subscript g multiplied by V subscript g divided by R multiplied by T subscript g.  

R equals R subscript u divided by M equals 8.314 joules per mole-kelvin divided by 50 kilograms per kilomole equals 0.166 joules per kilogram-kelvin.

m sub g equals (V sub g multiplied by p sub g) divided by (R sub g multiplied by T sub g).  
equals (3.14 multiplied by 5.14) divided by (0.166 multiplied by 500).  
equals 3.43 grams.  
equals 3.43 multiplied by 10 to the power of minus 3 kilograms.  

V sub g equals 3.14 liters.  
equals 3.14 multiplied by 10 to the power of minus 3 cubic meters.  

T sub g equals 500 degrees Celsius.  
equals 773.15 Kelvin.