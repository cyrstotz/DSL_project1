A graph is drawn with the y-axis labeled as "T [K]" and the x-axis labeled as "S [kJ/kg·K]". The graph depicts a thermodynamic process with six states labeled as 0, 1, 2, 3, 4, 5, and 6. The following annotations are made:  
- Between states 2 and 3: "isobar".  
- Between states 4 and 5: "isobar".  

Descriptions of states:  
- Zustand 0:  
  \( T_0 = -30^\circ \text{C} \)  

- Zustand 1:  
  "adiabat"  
  \( S_1 > S_0, T_1 > T_0 \)  

- Zustand 2:  
  "adiabat, reversibel"  
  \( S_2 = S_1, T_2 > T_1 \)  

- Zustand 3:  
  "isobar"  
  \( T_3 > T_2, S_3 >> S_2 \)  

- Zustand 4:  
  \( T_4 < T_3, S_4 > S_3, \text{irreversibel} \)  

- Zustand 5:  
  \( S_5 < S_4, T_5 \neq T_4 \)  

- Zustand 6:  
  \( S_6 = S_5, T_6 < T_5 \)

Momentum is constant.  

\( \dot{m}_{ein} = \rho_5 A_5 w_5 = \rho_6 A_6 w_6 \) → \( A_5 = A_6 \) → Assumption (no idea).  

\( T_5 = 431.9 \, \text{K} \)  
\( V_5 = 2.42 \, \text{m}^3/\text{kg} \)  
\( T_6 = 322.07 \, \text{K} \)  
\( V_6 = 4.93717 \, \text{m}^3/\text{kg} \)  

\( w_6 = \frac{p_5}{\rho_6} w_5 \)  
\( w_6 = \frac{V_6}{V_5} w_5 \)  
\( w_6 = 437.49 \, \text{m/s} \)  

\( s_6 = s_5 \)