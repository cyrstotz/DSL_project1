The following equations and calculations are written:  

1HS (First Law of Thermodynamics):  
hs minus h6 equals m multiplied by (w6 squared minus w5 squared) divided by 2 plus g multiplied by (ze minus z5) plus (Q minus W).  

Simplifications:  
m multiplied by w6 squared divided by 2 equals hs.  
m multiplied by w6 squared divided by 2 equals m multiplied by h6.  

From Table A-22:  
hs equals 600 multiplied by h6.  
w6 squared multiplied by hs divided by h6 equals w6.  

Interpolation:  
hs equals 431.63 minus 421.26 divided by (431.9 minus 430) plus 421.26 equals 421.43566.  
hs equals 421.44 kJ/kg.  

Additional notes:  
- "What is h6?"  
- "Consider T6."  

Equation for T6:  
T6 equals Tc multiplied by (Pc divided by Ps) raised to the power of (n minus 1 divided by n).  

Constants:  
n equals kappa equals Cp divided by Cv equals 1.006 kJ/kgK divided by Cp minus R.  
Given: n equals 1.4.

\( T_6 = 431.9 \, \text{K} \)  
\( \frac{191,100 \, \text{Pa}}{50,000 \, \text{Pa}} \)  
\( = 328.67469 \, \text{K} \)  
\( = 328.1 \, \text{K} \)  

Interpolation:  
\( h_6 = \frac{336.34 - 325.31}{328.1 - 325} \times (328.1 - 325) + 325.31 \)  
\( = 328.4031 \, \text{kJ/kg} \)  
\( = 328.40 \, \text{kJ/kg} \)  

\( w_0 = w_s = \sqrt{\frac{h_s}{k_hc}} \)  
\( = \sqrt{\frac{220 \, \text{m/s}^2}{328.4031 / 421.48366}} \)  
\( = 194.2051 \, \text{m/s} \)  
\( w_6 = 194.2 \, \text{m/s} \)