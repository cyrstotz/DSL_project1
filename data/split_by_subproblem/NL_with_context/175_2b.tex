0 equals m-dot times [h_0 minus h_6 plus (w_e squared minus w_a squared) divided by 2] divided by rho_0 plus q-dot_j minus w-dot_tn  

- Adiabatic to the outside implies q-dot_j equals 0  

w-dot_tn equals m times [h_5 minus h_6 plus (w_e squared minus w_a squared) divided by 2]  

[w_tn divided by m plus (h_6 minus h_5)] times 2 equals w_e squared minus w_a squared  

w_a squared equals w_e squared minus w-dot_tn divided by m-dot plus [h_5 minus h_6]  

v_5 equals m times R times T_5 divided by p_5  
equals (0.1286 joules per gram times 431.13 kelvin times R) divided by (0.5 times 10^5 pascal)  
equals (8.314 joules per mole divided by 28.97 grams per mole)  
equals 0.012867 joules per gram·kelvin  
equals 0.10024789 cubic meters per gram  

T_6 divided by T_5 equals (p_6 divided by p_5) raised to the power of [(kappa minus 1) divided by kappa]  
implies T_6 equals T_5 times [(p_6 divided by p_5) raised to the power of (4 divided by 14)]  
equals 328.107 kelvin  

h_6 minus h_5 equals c_p times [T_6 minus T_5]  

h_5 minus h_x equals c_p times [T_5 minus T_6]  
equals (1.006 kilojoules per kilogram·kelvin) times (431.13 kelvin minus 328.107 kelvin)  
equals 104.14 kilojoules per kilogram

w six squared equals w five squared minus W dot en divided by m dot plus h five minus h six.  

W dot en equals R times (T six minus T five) divided by one minus kappa equals 0.12367 joules per gram times (328.07 kelvin minus 431.13 kelvin) divided by 0.4.  

Equals negative 741.42 joules per gram.  

w six equals the square root of (200 meters per second squared plus 741.42 joules per gram plus 104.45 joules per kilogram).  

---