A graph is drawn with the y-axis labeled as 'T (K)' and the x-axis labeled as 'S (kJ/kg·K)'. The graph depicts a thermodynamic process with several labeled states (1, 2, 3, 4, 5, 6).  

- State 1: \( T_0, p_0 \)  
- State 2: \( p_1 > p_0 \)  
- State 3: Irreversible adiabatic compression.  
- State 4: Isobaric heat addition.  
- State 5: Mixing chamber, \( T_5, w_5, p_5 \).  
- State 6: Reversible adiabatic nozzle, \( p_0 \), \( w_6 > w_0 \).  

Key notes on the graph:  
- Between states 1 and 2: "adiabatic, irreversible \( w_L < L \) compression."  
- Between states 2 and 3: "High-pressure adiabatic process to \( p_2, T_2 \)."  
- Between states 3 and 4: "Isobaric heat addition."  
- Between states 4 and 5: "Mixing chamber."  
- Between states 5 and 6: "Reversible adiabatic nozzle."  

The graph includes isobars labeled \( p_0, p_1, p_2 \), and arrows indicating the direction of the process.  

Below the graph, the following steps are written:  
1. \( T_0, p_0 \): Adiabatic, irreversible \( w_L < L \) compression.  
2. \( p_1 > p_0 \): Mass flow rate \( \dot{m}_K \) is divided into mantle and core flows. High-pressure adiabatic process to \( p_2, T_2 \).  
3. \( 2 \rightarrow 3 \): Mixing chamber, mass flow rate divided.  
4. \( 3 \rightarrow 4 \): Adiabatic, irreversible turbine.  
5. \( 5 \): Mixing chamber, \( T_5, w_5, p_5 \).  
6. \( 6 \): Reversible adiabatic nozzle, \( p_0 \), \( w_6 > w_0 \).

w_6, T_6 at the outlet  
p_6 = p_0  

w_5 = 220 meters per second  

Stationary, adiabatic, and reversible.  

p_6 adiabatic and reversible, thermophysical. With kappa = 1.4:  

T_0 divided by T_6 equals (p_6 divided by p_0) raised to the power of kappa minus 1 divided by kappa.  
This implies T_6 equals T_5 multiplied by (p_0 divided by p_5) raised to the power of kappa minus 1 divided by kappa.  
This equals T_5 multiplied by (p_0 divided by p_5) raised to the power of kappa minus 1 divided by kappa.  

With m dot equals rho multiplied by A multiplied by w, rho equals m dot divided by A multiplied by w.  
This implies w_6 equals 510 meters per second, T_6 equals 390 Kelvin.

Exergy loss equals exergy balance.  

Zero equals the sum of exergy streams plus exergy heat minus work minus exergy loss.  

Exergy loss divided by mass flow rate equals the sum of exergy streams divided by mass flow rate.  

Zero equals the sum of exergy streams plus exergy heat minus work minus exergy loss.  

Exergy loss equals exergy heat divided by thermodynamic mean temperature minus the sum of exergy streams plus the term one minus T zero divided by T B multiplied by Q B.  

Exergy loss minus the sum of exergy streams plus the term one minus T zero divided by T B multiplied by Q B equals 100 kilojoules per kilogram multiplied by the term one minus 243.15 kelvin divided by 1289 kelvin minus 1.45 kilojoules per kilogram.  

Approximately equals 106.72 kilojoules per kilogram.