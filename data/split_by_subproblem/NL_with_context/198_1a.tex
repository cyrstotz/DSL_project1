1) \( \dot{Q}_{aus} \):  
\( Q = \dot{m}_{in} (h_{ein} - h_{aus}) + \dot{Q}_{in} + \dot{Q}_R \)  

\( \dot{Q}_{aus} = \dot{m}_{in} (h_{aus} - h_{ein}) \)  

\( h_{ein} = h_f (70^\circ C) = 292.98 \, \text{kJ/kg} \, \text{(TAB A2)} \)  
\( h_{aus} = h_f (100^\circ C) = 419.04 \, \text{kJ/kg} \, \text{(TAB A2)} \)  

\( \dot{Q}_{aus} = 0.3 \, \text{kg/s} \, (419.04 \, \text{kJ/kg} - 292.98 \, \text{kJ/kg}) - 100 \, \text{kW} \)  
\( = -62.182 \, \text{kW} \)  

=> Vorzeichen nach Vorzeichenkonvention wie auf Z u. Zusammenfassung  

2)  
\( \bar{T}_{KF} = \int_{e}^{a} T \cdot \dot{S} / (S_a - S_e) \)  

\( S_a - S_e = -C \)  

\( \bar{T}_{KF} = \frac{288.15 \, \text{K} + 298.15 \, \text{K}}{2} \)  
\( = 293.15 \, \text{K} \)