Two diagrams are drawn:  

1. **First Diagram (p-T Diagram)**:  
   - The x-axis is labeled as "T" (temperature).  
   - The y-axis is labeled as "p" (pressure).  
   - A curve is drawn with the following regions labeled:  
     - "unterkühlte Flüssigkeit" (subcooled liquid) on the left.  
     - "Tripelpunkt" (triple point) at the intersection of the curves.  
     - "NQ-Gebiet (Nassdampf)" (wet steam region) in the middle.  
     - "überhitzter Dampf" (superheated steam) on the right.  

2. **Second Diagram (Phase Diagram)**:  
   - The x-axis is labeled as "T" (temperature).  
   - The y-axis is labeled as "p" (pressure).  
   - Three regions are labeled:  
     - "Fest" (solid) on the left.  
     - "Flüssig" (liquid) in the middle.  
     - "Gas" (gas) on the right.  
   - The "Tripelpunkt" (triple point) is marked at the intersection of the phase boundaries.