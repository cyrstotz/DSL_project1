Two diagrams are drawn to represent phase regions in a pressure-temperature (p-T) diagram.  

**Diagram 1:**  
- The y-axis is labeled as 'p (mbar)' and ranges from 0.01 to 1, 10, and 100.  
- The x-axis is labeled as 'T (°C)' and ranges from -30 to -20, 0, and beyond.  
- The diagram includes phase regions labeled as 'fest' (solid), 'flüssig' (liquid), and 'gas' (gas).  
- A curve is drawn to indicate the transition between phases, with a marked 'Tripel' (triple point).  
- There is a note: '(ii) T2 -> T(i)'.  

**Diagram 2:**  
- The y-axis is labeled as 'p (mbar)' and ranges from 0.01 to 1, 10, and 100.  
- The x-axis is labeled as 'T (°C)' and ranges from -30 to -20, 0, and beyond.  
- The phase regions are labeled similarly: 'fest' (solid), 'flüssig' (liquid), and 'gas' (gas).  
- A curve is drawn to indicate the transition between phases, with a marked 'Tripel' (triple point).  

Additional notes:  
- 'Nassdampf gas' (wet steam gas) is written near the upper part of the diagram.  
- The phase regions are shaded and labeled accordingly.