A diagram is drawn showing two chambers labeled "Keller" (lower chamber) and "Membran" (membrane separating the chambers). The pressure in the upper chamber is labeled \( P_{EW} \), and the pressure in the lower chamber is labeled \( P_{gas,1} \).  

The text states:  
"1 bar = \( P_{EW} = P_{gas,1} = 1.01325 \, \text{bar} \)"  

An equation is written:  
\( P_{gas,1} V_{g,1} = m_{g,1} R T_{g,1} \)  

The gas constant \( R \) is calculated:  
\( R = \frac{\bar{R}}{M_g} = \frac{0.76628 \, \text{kJ}/\text{kg·K}}{50 \, \text{kg/kmol}} = 0.76628 \, \text{kJ}/\text{kmol·K} \)  

The mass of the gas \( m_{g,1} \) is calculated:  
\( m_{g,1} = \frac{P_{0,1} V_{g,1}}{R T_{g,1}} \)  

Values are substituted:  
\( V_{g,1} = 3.14 \, \text{L} = 3.14 \, \text{dm}^3 = 3.14 \cdot 10^{-3} \, \text{m}^3 \)  
\( T_{g,1} = 500^\circ \text{C} = 773.15 \, \text{K} \)  

The result is:  
\( m_{g,1} = 2.4745 \, \text{kg} \)  
\( m_{g,1} = 2.4748 \, \text{g} \)