A T-s diagram is drawn with labeled isobars and states. The diagram includes the following details:  
- State 1: \( s_1 = s_2 \), \( p = p_1 \)  
- State 2: \( p_2 = p_3 \)  
- State 3: \( p = p_3 \)  
- State 4: \( p = 0.5 \, \text{bar} \)  
- State 5: \( p = p_5 \), \( s_5 = s_6 \)  
- State 6: \( p = 0.5 \, \text{bar} \)  

Additional notes:  
- \( p_0 = 0.191 \, \text{bar} \)  
- \( T_0 = 243.15 \, \text{K} \)  
- \( w_{\text{Luft}} = 200 \, \text{m/s} \)  
- \( \eta_{V,s} < 1 \)  

A table is drawn with columns labeled \( p \), \( V \), \( T \), \( c_p \), \( Q \), and \( s \). The rows correspond to states 1 through 6, but only partial information is filled in:  
- State 1: \( p_0 = 0.191 \, \text{bar} \), \( T_0 = 243.15 \, \text{K} \)  
- State 2: \( p_2 = p_3 \)  
- State 5: \( p = 0.5 \, \text{bar} \), \( p_5 = p_6 \)