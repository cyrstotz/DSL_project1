Gas:  
\( c_V = 0.633 \, \text{kJ/kg·K} \)  
\( M_g = 50 \, \text{kg/kmol} \)  

\( p_{g,1} \): \( pV = RT \)  

\( R = \frac{R}{M_g} = \frac{8.314 \, \text{kJ/kmol·K}}{50 \, \text{kg/kmol}} = 0.16628 \, \text{kJ/kg·K} \)  

\( V_{g,1} = 3.14 \, \text{L} \)  
\( T_{g,1} = 500^\circ \text{C} \)  
\( m_{EW} = 0.1 \, \text{kg} \)  

**Diagram labeled "KGW":**  
A schematic showing forces acting on the piston and the area \( A \).  

\( p_{g,1} \cdot A = p_{amb} \cdot A + m_K \cdot g + m_{EW} \cdot g \)  

\( p_{g,1} = p_{amb} + \frac{m_K \cdot g}{A} + \frac{m_{EW} \cdot g}{A} \)  

\( p_{g,1} = 4.10^5 \, \text{N/m}^2 + \frac{32 \cdot 9.81 \, \text{N}}{0.00785398 \, \text{m}^2} + \frac{0.1 \cdot 9.81 \, \text{N}}{0.00785398 \, \text{m}^2} = 140094.4406 \, \text{N/m}^2 \approx 1.4 \, \text{bar} \)  

\( A = \frac{D^2 \cdot \pi}{4} = \frac{10^{-2} \, \text{m}^2 \cdot \pi}{4} = 0.00785398 \, \text{m}^2 \)  

\( pV = mRT \)  

\( m_{g,1} = \frac{p_1 \cdot V_{g,1}}{R \cdot T_1} = \frac{1.4 \cdot 10^5 \, \text{Pa} \cdot 3.14 \cdot 10^{-3} \, \text{m}^3}{0.16628 \, \text{kJ/kg·K} \cdot [500 + 273.15] \, \text{K}} = 0.00634 \, \text{kg} \approx 3.422 \, \text{g} \)  

---