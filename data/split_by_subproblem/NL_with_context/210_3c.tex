Q equals m times Cv times delta T (for the first law on the gas side).  

Diagram:  
A box labeled "Gas" with an arrow pointing to it labeled "Q" and another arrow labeled "Membrane".  

Q12 equals mg times Cv times (T2 minus T1).  
p1 equals p2.  
Cp equals R plus Cv.  

First law on the piston:  
m dot times (U2 minus U1) equals Q12 minus W12.  

W12: Since isochoric, h equals 0 plus perfect gas:  
R times (T2 minus T1) equals 8.314 kilojoules per kilogram Kelvin divided by 50 Kelvin times (0.0026 kilograms times 500 degrees Celsius).  
Equals negative 83.14 kilojoules.  

Therefore:  
Q12 equals mg times Cp times T.  
Q12 equals mg times Cv times (T2 minus T1) plus W12.  

Equals 0.0026 kilograms times 50 divided by 0.0026 kilograms divided by 0.633 kilojoules per kilogram Kelvin times (449.987 kilojoules per kilogram).  

Q12 equals mg times Cv times (T2 minus T1) plus W12.  
Equals 0.0026 kilograms times (0.633 times 449.987 kilojoules per kilogram) minus 83.14 kilojoules.  

Q12 equals.  

(Note: The last equation is incomplete and ends abruptly.)

First law of thermodynamics for a closed cylinder:  
U equals zero.  

Q one two equals m times c subscript v times delta T  
equals m subscript g times c subscript v times (T two minus T one)  
equals 0.0036 kilograms times 0.63 kilojoules per kilogram times (-449.387)  
equals negative 1.1394 kilojoules equals negative 1139.4 joules (underlined).