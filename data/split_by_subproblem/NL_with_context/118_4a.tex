A table is presented with the following columns: T, p (bar), V, U, and S. The rows are labeled z1, z2, z3, and z4.  
- z1: T = 6°C, p = 1 bar  
- z2: T = -6°C, p = 1 bar, U = 243.72, S = 0.9226  
- z3: T = 8°C, p = 8 bar  
- z4: T = 8°C, p = 8 bar, S = 0.9226  

Below the table:  
Ti = 0°C → T(z2) = -6°C  

A graph is drawn with the y-axis labeled T (K) and the x-axis labeled S (kJ/kg·K).  
The graph includes a dome-shaped curve representing phase regions.  
- The left side of the curve is labeled "Fest" (solid).  
- The right side of the curve is labeled "Dampf-Gasförmig" (vapor-gaseous).  
- Inside the curve, points z1, z2, z3, and z4 are marked, with arrows indicating transitions.  
- The region inside the curve is labeled "mass dampf gebiet" (mass vapor region).