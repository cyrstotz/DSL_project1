Two diagrams are drawn:

1. **First Diagram**:  
   - The axes are labeled with 'P' (pressure) on the vertical axis and 'T' (temperature) on the horizontal axis.  
   - A closed loop is drawn with four points labeled as '1', '2', '3', and '4'.  
   - The loop appears to represent a thermodynamic cycle.  

2. **Second Diagram**:  
   - The axes are labeled with 'P' (pressure) on the vertical axis and 'T' (temperature) on the horizontal axis.  
   - A curve is drawn, dividing the diagram into regions.  
   - The regions are labeled as 'fest' (solid), 'flüssig' (liquid), and 'gas/förmig' (gas/vapor).  
   - Two vertical lines are drawn within the diagram, labeled as 'i' and 'ii'.  
   - These lines appear to represent transitions between phases or states.