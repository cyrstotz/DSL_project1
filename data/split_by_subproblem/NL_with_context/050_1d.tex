d)  

T_{KF} = 70°C  
\( \Delta m_{12} \)  
T_{in,12} = 20°C  

State transition: 1 → 2  

Equation:  
m_2 u_2 - m_1 u_1 - \( \Delta m \) h_{sp} = Q  

Expanded:  
(m + \( \Delta m \)) u_2 - m_1 u_1 - \( \Delta m \) h_{sp} = Q  

m_2 = m_1 + \( \Delta m \)

m subscript 2 u subscript 2 plus m subscript e (u subscript 2 minus h subscript e) minus m subscript 1 u subscript 1 equals Q  

m subscript e equals (Q plus m subscript 1 (u subscript 1 minus u subscript 2)) divided by (u subscript 2 minus h subscript e)  

Q equals negative 35 megajoules  
m subscript 1 equals 5755 kilograms  

T subscript 1 equals 100 degrees Celsius  
T subscript 2 equals 70 degrees Celsius  
T subscript e equals 20 degrees Celsius  

Table 32:  
4.27, 38 kilojoules per kilogram plus 4.129, 44 plus 0.005 (205.045)  

u subscript 1 (70 degrees Celsius, saturated fluid)  
u subscript 2 (100 degrees Celsius, saturated fluid)  
u subscript e (20 degrees Celsius, saturated fluid)  
h subscript e (20 degrees Celsius, saturated fluid)  

x equals 0.005  

---