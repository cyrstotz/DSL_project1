\( T_i = -10^\circ \text{C} \)  

| \( \phi \) | \( p \) | \( T \) | \( h \) | \( s \) |  
|-----------|---------|---------|---------|---------|  
| 1         | 1.3748  | -16     |         |         |  
| \( x = 1 \) | 2       | p1      | -16     |         |         |  
| 3         | 8       |         |         | \( s_2 \) |  
| \( x = 0 \) | 4       | 8       |         |         |         |  

\( T_{h,2} = T_i - 6 \, \text{K} = -16^\circ \text{C} \)  

\( h_2 = h_g(-16^\circ \text{C}) \rightarrow A10 \)  
\( h_2 = 237.74 \, \text{kJ/kg} \)  

\( s_2 = s_3 = s_g(-16^\circ \text{C}) \)  
\( s_2 = 0.9288 \, \text{kJ/kgK} \)  

\( p_3 = 8 \, \text{bar} \rightarrow \) (illegible text)  

via A-11 sehe ich, wir sind im Dampf-Gebiet  
\( \rightarrow A-12 \)  

\( h_3 = h_{\text{sat}} + \frac{(h(40) - h_{\text{sat}})}{s(40) - s_{\text{sat}}} \cdot (s_3 - s_{\text{sat}}) \)  

\( h(40) = 273.66 \)  
\( h_{\text{sat}} = 269.45 \)  
\( s_{\text{sat}} = 0.9066 \)  
\( s(40) = 0.9374 \)  

\( h_3 = 271.3 \, \text{kJ/kg} \)