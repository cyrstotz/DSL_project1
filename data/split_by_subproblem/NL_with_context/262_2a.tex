A diagram is drawn with labeled axes. The vertical axis is labeled 'P' (pressure) and the horizontal axis is labeled 'T' (temperature). The diagram represents a p-T curve with various labeled points and regions:  
- Point 1 is marked near the bottom left.  
- Point 2 is marked slightly higher and to the right, connected by a line labeled 'isobar'.  
- Point 3 is marked further to the right and higher, labeled 'gas'.  
- Point 4 is marked at the peak of the curve, labeled 'Tcrit'.  
- The curve is labeled 'Non-Dampf' (non-vapor).  
- Below the curve, regions are labeled 'flüssig' (liquid) and 'fest' (solid).

(a) T-S Diagram  

The diagram shows a temperature (T) versus entropy (S) plot.  

- The vertical axis is labeled as T (K).  
- The horizontal axis is labeled as ΔS (kJ/kg·K).  
- The reference temperature \( T_0 = 293.15 \, \text{K} \) is noted at the top.  

Key points and processes are labeled:  
- Point 0: Reversible.  
- Point 1: Reversible, adiabatic.  
- Point 2: Reversible, isobaric.  
- Point 3: Isobaric.  
- Point 4: Adiabatic, irreversible.  
- Point 5: Isobaric.  
- Point 6: Reversible.  

The entropy values \( S_0, S_2, S_4 \) are marked along the horizontal axis.  

The diagram includes curved and straight lines connecting the points to represent different thermodynamic processes:  
- Isobaric processes are shown as horizontal lines.  
- Adiabatic processes are shown as sloped lines.  
- Reversible processes are explicitly labeled.  

No additional text or calculations are present.

\( w_6 = ? \)  
\( T_6 = ? \)  

\( n = k = 1.4 \)  

Ideal gas:  
\( \left( \frac{T_6}{T_5} \right) = \left( \frac{p_6}{p_5} \right)^{\frac{n-1}{n}} \)  

\( T_6 = T_5 \cdot \left( \frac{p_6}{p_5} \right)^{\frac{n-1}{n}} \)  

\( T_6 = ? \)  
\( p_6 = p_0 = 0.191 \, \text{bar} \)  
\( p_5 = 0.5 \, \text{bar} \)  
\( T_5 = 431.9 \, \text{K} \)  

\( T_6 = T_5 \cdot \left( \frac{p_6}{p_5} \right)^{\frac{n-1}{n}} = 328.07 \, \text{K} = T_6 \)  

---

\( w_6 = ? \)  

5 → 6 adiabatic, reversible!  
Energy balance:  
\( b_Q = 0 \)  
Stationary:  

\( O = \dot{m} \left[ h_5 - h_6 + \frac{w_5^2 - w_6^2}{2} + g(z_5 - z_6) \right] \)  

\( O = \dot{m} \left[ h_5 - h_6 + \frac{w_5^2 - w_6^2}{2} \right] + \dot{m} \left( \frac{n \cdot R(T_6 - T_5)}{n-1} \right) \)  

\( h_5 - h_6 + \frac{w_5^2 - w_6^2}{2} = \frac{n \cdot R(T_6 - T_5)}{n-1} \)  

\( \dot{m} \left( \frac{n \cdot R(T_6 - T_5)}{n-1} \right) = W_t \)  

\( W_t = \dot{m} \cdot \int_{p_1}^{p_2} v \, dp \)  

\( n \neq 1 \)  

\( \dot{m} \cdot \left( \frac{n \cdot R(T_6 - T_5)}{n-1} \right) = W_t \)  

\( c_{Luft} \)