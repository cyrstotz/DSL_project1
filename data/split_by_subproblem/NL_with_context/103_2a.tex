A table is presented with the following columns:  

- \( p \) (bar)  
- \( T \) (K)  
- \( w \) (meters per second)  
- \( \dot{m} \) (mass flow rate)  

The rows are labeled numerically from 1 to 6, with an additional row labeled 0. The values in the table are:  

1: Empty row.  
2: Empty row, labeled \( \dot{m}_K \).  
3: \( p_2 \), labeled \( \dot{m}_K \).  
4: Empty row, labeled \( \dot{m}_K \).  
5: \( p = 0.5 \), \( T = 431.9 \), \( w = 220 \), labeled \( \dot{m}_{ges} \).  
6: \( p = 0.191 \), empty \( T \), empty \( w \).  
0: \( p = 0.191 \), \( T = 243.15 \), \( w = 200 \), labeled \( \dot{m}_{ges} \).  

The word "isotrop" is written next to rows 2 through 4.  

Below the table, calculations are written:  

\( c_p = 1.006 \, \text{kJ/kg·K} \)  
\( k = \frac{c_p}{c_v} = 1.4 \)  
\( c_v = \frac{1.006 \, \text{kJ/kg·K}}{1.4} = 0.719 \, \text{kJ/kg·K} \)  

\( R = c_p - c_v = 0.287 \, \text{kJ/kg·K} \)  

The value \( R = 0.287 \, \text{kJ/kg·K} \) is underlined.

6 \( w_6^2 \)  

5 → 6 isentrop  
\( s_5 = s_6 \)  

\( w_{rev} = - \int_1^2 v dp + \Delta ke \)  
\( \Delta ke = \frac{w_6^2}{2} - \frac{w_5^2}{2} \)  

polytrop \( n \neq 1 \)  

\( \int_1^2 v dp = \frac{1}{n} \int_1^2 p dv \)  
\( - \int_1^2 v dp = \frac{n}{1} \int_1^2 p dv \)  

\( w_{rev} = \frac{w_{rev}}{\dot{m}} \)  

\( h_5 - h_6 + \frac{w_5^2 - w_6^2}{2} - w_{rev} = 0 \)  

\( w_{rev} = \int_1^2 p dv = \frac{R (T_6 - T_5)}{1 - n} = \)  

\( T_6 = T_5 \left( \frac{p_6}{p_5} \right)^{\frac{1.4 - 1}{1.4}} = 431.9 \, \text{K} - \left( \frac{0.191 \, \text{bar}}{0.5} \right)^{\frac{0.4}{1.4}} \)  

\( = 328.1 \, \text{K} \)  

\( \frac{0.287 \, \text{kJ/kg}}{1 - 1.4} (328.1 \, \text{K} - 431.9 \, \text{K}) = -79.5 \, \text{kW} \)  

\( w_{rev} \dot{m} = w_{rev} \)  

\( \dot{m} (h_5 - h_6 + \frac{w_5^2 - w_6^2}{2}) - w_{rev} = 0 \)  

\( \dot{m} c (T_5 - T_6) + \frac{w_5^2 - w_6^2}{2} - w_{rev} = 0 \)  

\( w_{rev} = 0 \)

A graph is drawn with the vertical axis labeled as "T" (temperature) and the horizontal axis labeled as "S" (entropy). The graph depicts a thermodynamic process with four labeled points: 1, 2, 3, and 4.  

- Point 1 is at the lower left.  
- Point 2 is to the right of point 1, forming a curve.  
- Point 3 is above point 2, connected by another curve.  
- Point 4 is vertically above point 3.  

The segment between points 3 and 4 is labeled "isentrop" (indicating an isentropic process).  

There are additional shaded regions and curved lines between points 1 and 3, suggesting transitions or processes.