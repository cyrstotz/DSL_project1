A diagram is drawn with labeled axes:  
- The horizontal axis is labeled as "s [kJ/kg·K]".  
- The vertical axis is labeled as "T [K]".  
Several curves are drawn, representing different processes:  
- A curve labeled "isentrop" connects points 1, 2, 3, and 4.  
- Another curve labeled "isentrop" connects points 5 and 6.  
- Points are labeled as follows:  
  - Point 0 is at the bottom left.  
  - Point 1 is connected to point 2.  
  - Point 2 is connected to point 3.  
  - Point 3 is connected to point 4.  
  - Point 4 is connected to point 5.  
  - Point 5 is connected to point 6.  
- Additional labels include "p0", "p5", "p6", and "p0 + p5".  

A table is present with columns labeled "T [K]" and "p [bar]":  
- Row 1: T = blank, p = blank.  
- Row 2: T = blank, p = p2.  
- Row 3: T = blank, p = p2.  
- Row 4: T = blank, p = blank.  
- Row 5: T = 431.9, p = 0.5.  
- Row 6: T = blank, p = 0.191.  

---

A diagram labeled "T-s diagram" is drawn. The x-axis is labeled as "s [kJ/kg·K]" and the y-axis is labeled as "T [K]".  
The diagram shows several curves and arrows:  
- A curve labeled "p0".  
- Another curve labeled "p5".  
- A third curve labeled "p6 = p5".  
- Arrows are drawn between points labeled "2", "4", and "6", indicating transitions between states.