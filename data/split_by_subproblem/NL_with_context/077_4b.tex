The mass flow rate of R134a (\( \dot{m}_{R134a} \)) is questioned.  

State transition (2 to 3): Isentropic, because it is reversible and adiabatic.  

Equation:  
\( \dot{m}_{R134a} \cdot (h_e - h_a) + \sum \dot{Q}_j - \sum \dot{W}_j = 0 \)  
Annotation: "Der Prozess ist stationär" (The process is stationary).  

Energy balance for the compressor (isentropic):  
\( \dot{m}(0) = 0 \)  

Energy balance during compression:  
\( \dot{m} \cdot (h_2 - h_3) = 28 \, \text{W} \)

Mass flow rate times (h2 minus h3) equals 28 watts. Therefore, mass flow rate equals 28 watts divided by (h2 minus h3).

x2 equals 1, which implies fully vapor (saturated).

What is Ti?

5 millibar below the triple point and 10 Kelvin above sublimation.  
Therefore, pressure equals 1 millibar and Ti equals minus 10 degrees Celsius, which is 263 Kelvin.  
For the evaporator: 257 Kelvin.

h2 equals h2g.  
h3 equals h3f (fully compressed) at 8 bar.  
h3f equals h3 equals 93.42 (from Table A.11).

The entropy from 2 to 3 remains constant, therefore s2 equals s3.  
s3 equals s3f equals 0.3459.

At x2 equals 1:  
s3f equals s2g equals s2 equals 0.3459, therefore pressure2 equals ?