A diagram is drawn showing a box labeled "EW" with an arrow pointing to it labeled "Q12."

The equation is written:
dE/dt equals the summation of Q minus the summation of W.

A note is written:
"0, due to small volume change."

The equation is written:
U2 minus U1 equals Q12.

Another equation is written:
U1 equals Uf plus x times (Ug minus Uf).

A note is written:
"Pressure in EW at state 1: equilibrium pressure assumed, p equals 1.1 bar."

From Table 1:
U1 equals negative 0.045 kilojoules per gram plus 0.6 times (negative 33.3 kilojoules per gram minus 0.045 kilojoules per gram).

The result is:
U1 equals negative 200.09 kilojoules per kilogram.

Another equation is written:
U1 equals U in EW equals negative 200.09 kilojoules per kilogram.

The equation is written:
U2 equals Q12 plus U1 equals 1366.4 kilojoules minus 200.1 kilojoules equals negative 186.6336 kilojoules.

The equation for x is written:
x equals (U2 minus Uf) divided by (Ug minus Uf).

Substituting values:
x equals (negative 186.6336 kilojoules plus 0.045 kilojoules per gram) divided by (negative 33.3 kilojoules per gram minus 0.045 kilojoules per gram).

The result is:
x equals 0.5587.

The equation is written:
m2,EW equals x2,EW times mEW.

Substituting values:
m2,EW equals 0.5587 times 0.1 kilograms equals 55.87 grams.

The result is underlined.

Energy balance for system boundary:  

\( \frac{dE}{dt} = \sum_j \dot{Q}_j - \sum_n \dot{W}_n \)  

\( U_2 - U_1 = Q_{12} - W_{12} \)  

Since we have no friction losses, \( W_{12} = \text{reversible} \).  

\( W_{12} = \int p \, dv \)  

\( m_g = \frac{V_{g,1}}{v_{g,1}} = \frac{3.14 \cdot 10^{-3} \, \text{m}^3}{3.94 \cdot 10^{-2} \, \text{m}^3/\text{kg}} = 0.07968397 \, \text{kg} \)