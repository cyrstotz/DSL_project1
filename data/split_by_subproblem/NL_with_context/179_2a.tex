A graph is drawn with the y-axis labeled as "T K" (temperature in Kelvin) and the x-axis labeled as "S kJ/kg" (entropy in kilojoules per kilogram). The graph depicts a thermodynamic process with labeled points:  
- Point 1 connects to Point 2 with an upward curve labeled "isobar."  
- Point 2 connects to Point 3 with a curve labeled "isenthalic."  
- Point 3 connects to Point 5 with a downward curve labeled "isobar."  
- Point 5 connects to Point 6 with a vertical line labeled "isobar."  
- Point 6 connects back to Point 1 with a curve.

Zero equals m multiplied by (h zero minus h six plus w zero squared minus w six squared divided by two).  

w six squared divided by two equals h zero minus h six plus w zero squared divided by two.  

w six equals the square root of (two multiplied by (h zero minus h six) plus w zero squared).  

w six equals 193 meters per second.  

w zero equals the square root of (two multiplied by c p multiplied by (T zero minus T six) plus w zero squared).  

T zero equals negative 30 degrees Celsius.  

T six equals 323.074 Kelvin.  

c p equals 1.006 kilojoules per kilogram Kelvin.  

---