A graph is drawn with a curved line representing phase regions. The x-axis is labeled as "T [K]" (temperature), and the y-axis is labeled as "p [bar]" (pressure). The graph includes the following annotations:  
- "fest" (solid) is labeled below the curve on the left side.  
- "flüssig" (liquid) is labeled below the curve on the right side.  
- "Tripelpunkt" (triple point) is marked at a specific point on the curve.  
- "dampf" (vapor) is labeled above the curve on the right side.  
- An arrow points upward from the triple point, labeled "2. St." (second step).  
- Another arrow points horizontally to the right, labeled "1. St." (first step).

A graph is drawn on a grid with the vertical axis labeled as "T [K]" and the horizontal axis labeled as "s [kJ/kg·K]". The graph represents a thermodynamic process with six states labeled as 0, 1, 2, 3, 4, 5, and 6.  

- State 0 is marked at the bottom left with "T₀ = 243.15".  
- State 1 is connected to State 2 by a vertical line labeled "isentrop".  
- State 2 is connected to State 3 by a horizontal line labeled "isobar".  
- State 3 is connected to State 4 by a diagonal line labeled "adiabat irreversible".  
- State 4 is connected to State 5 by a horizontal line labeled "isobar".  
- State 5 is connected to State 6 by a vertical line labeled "isentrop".  

Below the graph, there is a table with columns labeled "P", "T", and "Notes".  

- Row 0: "0.191 bar", "243.15 K", "η < 1".  
- Row 1: "isentrop", "Tₛ = 178.9", "adiabat reversible → isentrop".  
- Row 2: "isobar", "", "".  
- Row 3: "adiabat irreversible", "", "".  
- Row 4: "isobar", "T₅ = 431.9 K", "".  
- Row 5: "0.5 bar", "", "adiabat, reversible → isentrop".  

Additional notes are written next to the table:  
- "adiabat reversible → isentrop"  
- "adiabat irreversible"  

No further content is visible.