a) Existence requirement!  
Energy balance:  
0 = m dot times (h in minus h out) plus Q dot minus W dot  

h in = h of 70 degrees Celsius saturated liquid  
h out = h of 100 degrees Celsius saturated liquid  

h in = 292.88 kilojoules per kilogram  
h out = 419.17 kilojoules per kilogram  

Q dot = 100 kilowatts  

m dot times (419.17 minus 292.88) = 100 kilowatts  
m dot = 0.8 kilograms per second

A table is presented with columns labeled as follows:  

- Zustand (State): 0, 1, 2, 3, 4, 5, 6  
- T (Kelvin): 243.15, 243.15, 243.15, 243.15, 431.9, 431.9, 431.9  
- P (bar): 0.191, 0.5, 0.5, 12, 12, 0.5, 0.5  
- w (m/s): 200, 200, 200, 0, 0, 220, 220  
- h (kJ/kg): 0, 0, 0, 0, 0, 0, 0  

Notes are written above the table:  
- "s1 = s2"  
- "s2 = s3"  
- "p2 = p3"  
- "p3 = p4"  
- "s5 = s6"  
- "p5 = p6"  

Additional calculations are written below the table:  
- \( w_6 = \sqrt{2 \cdot (\Delta h)} = \sqrt{2 \cdot (8.97)} = 0.297 \, \text{kg} \)  
- "h = c_p \cdot T"  

No further content is visible.

A graph is drawn with the x-axis labeled as "s [kJ/kg·K]" and the y-axis labeled as "T [K]".  

The graph depicts a T-s diagram with several curves and points labeled:  
- Point "0" is marked at the bottom left.  
- Point "3" is marked higher up along the curve.  
- Point "5" is marked further to the right.  
- Point "6" is marked slightly above point "5".  

The curves are labeled as follows:  
- "Isobar (constant p)" is written near the curve connecting points "3" and "5".  
- "Isobar (constant p)" is also written near the curve connecting points "5" and "6".  
- "Isobar (constant p)" is written near the curve connecting points "0" and "3".  

Additional notes:  
- "Mixing" is written near the region between points "5" and "6".  
- "Expansion" is written near the curve between points "3" and "5".  
- "Compression" is written near the curve between points "0" and "3".