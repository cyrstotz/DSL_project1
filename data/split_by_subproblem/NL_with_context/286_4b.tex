An equation is written:  
0 equals m dot times (h four minus h one) plus epsilon Q minus epsilon W.  

Below, another equation is written:  
m dot equals W dot divided by (h three minus h two).

\( T_i = -10^\circ \text{C} \)  
\( T_2 = T_i \)  
\( h_{12} \) interpolated from Table A-10:  
\( h_{12} = \frac{h_g(-8^\circ \text{C}) - h_g(-12^\circ \text{C})}{-8^\circ \text{C} - (-12^\circ \text{C})} \cdot (-8^\circ \text{C} - T_i) + h_g(-12^\circ \text{C}) \)  
\( = 264.78 \, \text{kJ/kg} - 264.31 \, \text{kJ/kg} = 243.3 \, \text{kJ/kg} \)  

\( h_g \):  
\( \ln \left( \frac{p_3}{p_2} \right) = \frac{n}{T_2} \)  
\( p_{R134a}(p_3) \): \( h_{12} = 264.15 \)  
From Table A-12  

\( \dot{m} = \frac{28}{5} \cdot \frac{264.15 - 243.3}{1.4 \cdot 40} \cdot \frac{\text{kJ}}{\text{s}} \cdot \frac{\text{kg}}{\text{kJ}} \cdot \frac{\text{s}}{\text{h}} = 1.368 \cdot 10^{-3} \, \text{kg/s} = 4.932 \, \text{kg/h} \)