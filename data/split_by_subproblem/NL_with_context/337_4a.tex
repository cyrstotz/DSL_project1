The first diagram shows a pressure-temperature (P-T) graph with labeled axes: pressure in bar (P [bar]) and temperature in Kelvin (T [K]). A curve is drawn, representing phase regions. The curve includes a labeled point "1" at the lower pressure and a point "2" at a higher pressure. The graph illustrates the sublimation and phase transition points.  

The second diagram also shows a P-T graph with similar axes. A curve is drawn, and the phase regions are labeled. Point "2" is marked on the curve, and arrows indicate transitions.  

A table is included with columns for pressure (P) and temperature (T). The rows are labeled 1, 2, 3, and 4:  
- Row 1: P = pu, T = blank  
- Row 2: P = 8 bar, T = -22°C  
- Row 3: P = 8 bar, T = blank  
- Row 4: P = 8 bar, T = blank  

Additional notes:  
- "a_41 = 0"  
- "pu = Smbar + Pip"  
- "Ti - 10K = Tsublimationspunkt"  
- "x2 = 1"  
- "x3 = 1"  
- "x4 = 0"  
- "Ti - T3 = 6K"  
- "S3 = S2"  

Another P-T graph is drawn with labeled axes (P [bar] and T [K]). A curve is shown, and phase regions are indicated.  

The final P-T graph includes labeled points "2," "3," and "4." The curve is annotated with "isotrop" and "isobar."  

---