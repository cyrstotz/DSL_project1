A diagram is drawn showing a cylinder with two chambers. The upper chamber is labeled "EW" and contains a piston, while the lower chamber is labeled "Gas." An arrow labeled "Q" points upward from the gas chamber to the EW chamber.  

The equation:  
dE/dt equals the sum of m dot times e plus the sum of Q dot minus the sum of W dot.  

The following calculations are written:  

Delta U equals Q  
Delta u equals q  

q equals Q12 divided by m equals 1500 joules divided by 0.1 kilograms equals 15000 joules per kilogram equals 75 kilojoules per kilogram.  

u2 minus u1 equals 75 kilojoules per kilogram  
u2 equals 75 kilojoules per kilogram plus u1  

u1:  
p equals 1.1 bar  
T1 equals 0 degrees Celsius  
x1 equals 0.6  

u1 equals x1 times uFest plus (1 minus x1) times uFlüssig  
equals 0.6 times negative 333.458 plus 0.4 times negative 3.045  
equals negative 200.0928 kilojoules per kilogram.  

u2 equals 75 kilojoules per kilogram plus negative 200.0928 kilojoules per kilogram  
equals negative 185.0928 kilojoules per kilogram.  

xEis,2 equals  
x times uFest plus (1 minus xEis) times uFlüssig equals negative 185.0928.  

x times (uFest minus uFlüssig) minus 185 minus uFlüssig.  

xEis equals negative 185.0928 minus uFlüssig divided by uFest minus uFlüssig  
equals negative 185.0928 plus 0.045 divided by negative 333.458 plus 0.045  
equals 0.555.  

Final result:  
xEis,2 equals 0.555.