\( \dot{m} c_p (T_5 - T_6) + \dot{m} \left( \frac{w_5^2 - w_6^2}{2} \right) + \dot{Q} - \dot{W}_+ = 0 \)  

\( W_t = -\frac{R (T_6 - T_5)}{n - 1} - \text{Ak} \)  

\( \dot{W}_{\text{rot}} = vdp + KE \)  

Luft ⇒ Ideales Gas ⇒ \( -\frac{R (T_6 - T_5)}{n - 1} \)  

\( \dot{m} c_p (T_5 - T_6) + \dot{m} (w_5^2 - w_6^2) - \dot{m} \frac{R (T_6 - T_5)}{n - 1} \)  

\( \sqrt{c_p (T_5 - T_6) + w_5^2 - \frac{R (T_6 - T_5)}{n - 1}} = w_6 = 350 \, \text{m/s} \)  

\( R = \frac{c_p}{1.4} = 0.7986 \, \text{kJ/kg·K} \)  

\( R = c_p - c_v = 289 \, \text{J/kg·K} \)

The diagram is a T-s (temperature-entropy) graph with the following elements:  

- The vertical axis is labeled as "T [K]" (temperature in Kelvin).  
- The horizontal axis is labeled as "s [kJ/kg·K]" (entropy in kilojoules per kilogram per Kelvin).  

The graph includes several curves and points:  
- Two isobars are labeled as "0.5 bar" and "0.191 bar."  
- Saturated regions are marked with the word "saturated" near the curves.  
- Points are labeled as "0," "1," "2," "3," "4," and "5."  
- Arrows indicate the direction of the process between the points.  
- A thick line connects the points, illustrating the process path.  
- The labels "K" and "M" appear near specific sections of the graph.  

No additional text or equations are present.

A table is presented with columns labeled \( p \) and \( T \).  

Row 0:  
\( p \): 0.191 bar  
\( T \): -30 degrees Celsius, 293.15 Kelvin (\( T_0 \))  

Row 1:  
\( p \): 0.5 bar  
\( T \): (empty)  

Row 2:  
\( p \): (crossed out)  
\( T \): (empty)  

Row 3:  
\( p \): (crossed out)  
\( T \): (empty)  

Row 4:  
\( p \): 0.5 bar  
\( T \): (empty)  

Row 5:  
\( p \): 0.5 bar  
\( T \): 431.9 Kelvin  

Row 6:  
\( p \): 0.191 bar  
\( T \): (empty)  

Below the table, the text reads:  
\( T_s, p_5, w_5 \) given.