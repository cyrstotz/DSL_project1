Two diagrams are drawn, both labeled with axes and points.  

**First Diagram:**  
- The vertical axis is labeled as `T (K)` (temperature in Kelvin).  
- The horizontal axis is labeled as `s [2/u]` (entropy).  
- Points are marked as `0`, `1`, and `2`.  
- The curve starts at point `0` and moves upward to point `1`, then vertically to point `2`.  
- A note is written near point `0`: `p0 = 0.191 bar`.  

**Second Diagram:**  
- The vertical axis is labeled as `T (K)` (temperature in Kelvin).  
- The horizontal axis is labeled as `s [2/u]` (entropy).  
- Points are marked as `0`, `2`, `3`, `4`, and `6`.  
- The curve starts at point `0`, moves upward to point `2`, then diagonally to point `3`.  
- From point `3`, it moves downward diagonally to point `4`, then vertically to point `6`.  
- A note is written near point `3`: `0.5 bar`.  
- Another note is written near point `6`: `clear axial gas flows`.  

An arrow is drawn at the bottom of the second diagram, pointing upward.

The page contains two diagrams:

1. **Top Diagram**:
   - The vertical axis is labeled as "T (K)".
   - The diagram shows a process with six numbered points: 0, 1, 2, 3, 4, and 6.
   - Curved lines connect these points, forming a cycle.
   - The lines appear to represent thermodynamic processes, with transitions between states marked by the numbered points.

2. **Bottom Diagram**:
   - The vertical axis is unlabeled, and the horizontal axis is also unlabeled.
   - The diagram shows a process with six numbered points: 0, 1, 2, 4, 5, and 6.
   - Curved lines connect these points, forming another cycle.
   - The notation "S L B" appears near the diagram, possibly indicating phase regions or labels.

No additional text or equations are visible.