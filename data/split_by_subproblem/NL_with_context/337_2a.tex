Two diagrams are drawn:  

1. **T-S Diagram**:  
   - The x-axis is labeled as 'S [kJ/kg·K]'.  
   - The y-axis is labeled as 'T [K]'.  
   - The diagram shows several processes:  
     - Isobaric lines labeled 'isobar'.  
     - Points labeled '1', '2', '3', '4', '5', and '6'.  
     - Arrows indicating transitions between states.  
     - The process labeled 'isentrop' connects states 5 and 6.  

2. **P-S Diagram**:  
   - The x-axis is labeled as 'S [kJ/kg·K]'.  
   - The y-axis is labeled as 'P [k]'.  
   - The diagram shows several processes:  
     - Isobaric lines labeled 'isobar'.  
     - Points labeled '1', '2', '3', '4', '5', and '6'.  
     - Arrows indicating transitions between states.  
     - The process labeled 'isentrop' connects states 5 and 6.  

A table is drawn next to the diagrams with the following columns:  
- **State**: 1, 2, 3, 4, 5, 6.  
- **P**: 0.191 bar, P2 = P3, 0.5 bar, 0.5 bar, 0.5 bar, 0.191 bar.  
- **T**: -30°C, T2 = T3, 431.9 K, T5 = T6.  
- **V**: w5 = 220 m/s, S5 = S6.  

Below the table, a note shows:  
- \( \dot{m}_{ges} = \dot{m}_M + \dot{m}_K \).  

---