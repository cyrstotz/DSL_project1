\( \dot{m}_R = 0 \)  

\( T_R2 = 70^\circ \text{C} \)  
\( T_R1 = 100^\circ \text{C} \)  
\( T_{ein} = 20^\circ \text{C} \)  
\( Q_{aus} = 35 \, \text{MJ} \)  

It is stated:  
"Da if i: \( h \neq h_F \), \( U = U_F \)"  

Energy balance for a semi-open system:  
\( \Delta U = \Delta m_{ein} \cdot (h_{ein}) + Q_j \)  

\( m_2 \cdot U_2 - m_1 \cdot U_1 = \Delta m_{ein} \cdot (h_{ein}) - Q_{aus} \)  

\( m_2 = m_1 + \Delta m_{ein} \)  

\( m_1 = 5755 \, \text{kg} \)  

\( U_1 = 2506.5 \cdot 0.005 + (1 - 0.005) \cdot 4178.94 = 4129.38 \)  
\( U_2 = U(70^\circ \text{C}) = 292.95 \)  
\( h_{ein} = h(20^\circ \text{C}) = 83.96 \)

(5755 plus delta m) times 232.95 minus 5755 times 423.38 equals delta m times 83.96 minus 35 times 10 cubed.  

- 750154 equals -208.85 delta m.  

Delta m equals 3583.8 kilograms.  

Delta S12 equals S1 minus S2 equals m1 S1 minus m2 S2.  

m1 equals 5755 kilograms.  
m2 equals 9355 kilograms.  

S1 equals 0.005 times 7.3549 plus 0.985 times 1.3068 equals 1.33714.  

S2 equals 6.9549.  

Delta S12 equals -1237.85 kilojoules.