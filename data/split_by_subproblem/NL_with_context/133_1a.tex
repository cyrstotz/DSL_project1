\( 0 = \dot{m} (h_{in} - h_{out}) + \dot{Q}_{out} + Q_R \)  

TAB A-2:  
\( h_{in} (70^\circ C, \text{saturated liquid}) = 292.984 \, \text{kJ/kg} \)  
\( h_{out} (100^\circ C, \text{saturated liquid}) = 419.044 \, \text{kJ/kg} \)  

\( Q_{out} = \dot{m} (h_{out} - h_{in}) - Q_R \)  
\( Q_{out} = 62.182 \, \text{kW} \)

u subscript A equals negative twenty point zero zero nine two eight four seven kilojoules per kilogram.  

u subscript A plus Q subscript A to equals negative two thousand one hundred nine point five one joules.  

u subscript 2 equals u subscript A plus Q subscript A to divided by m equals negative two thousand one hundred nine point five one divided by m equals negative two hundred ten point nine five one kilojoules per kilogram.  

x subscript 2, equilibrium equals u subscript 2 minus u subscript F equilibrium divided by u subscript FG equilibrium equals zero point six three two five zero equals zero point six three three.  

u subscript 2, equilibrium equals negative two hundred ten point nine five one minus three hundred thirty three point seven two kilojoules per kilogram.  

h subscript 2, equilibrium equals negative zero point zero three three six two kilojoules per kilogram.

T subscript i equals negative 10 degrees Celsius minus 6 Kelvin equals negative 16 degrees Celsius.  

O equals  

S subscript 2 equals 0.9285 kilojoules per kilogram Kelvin equals S subscript 3.  

S subscript 3 equals O in T diagram A-9.  

T subscript 4 equals T subscript 3 equals 30 / 33 degrees Celsius.  

ln ln colon T subscript 3 equals 30 plus 0.9555 minus 0.9444 times (35 minus 30) divided by 0.9554 minus 0.9444 equals 30 degrees Celsius.