u subscript 2 equals u subscript f plus x times (u subscript rest minus u subscript f).  

Delta U equals Delta Q.  

u subscript 2 (T subscript 2) minus u subscript 1 (T subscript 1) equals C subscript r times (T subscript 2 minus T subscript 1).  

T subscript 1 equals 0 degrees Celsius.  
T subscript 2 equals 0.003 degrees Celsius.  

u subscript 2 (T subscript 2) minus u subscript 1 (T subscript 1) equals Delta Q divided by m subscript EW.  
u subscript 2 (T subscript 2) minus u subscript 1 (T subscript 1) equals 15 kilojoules divided by 0.1 kilograms equals 15 kilojoules per kilogram.  

u subscript 1 (T subscript 1) equals u subscript f plus x subscript 1 times (u subscript rest minus u subscript f).  
u subscript 1 (T subscript 1) equals negative 0.045 plus 0.6 times (negative 333.458 plus 0.045).  
u subscript 1 (T subscript 1) equals negative 200.082 kilojoules per kilogram.  

(Note: The last section contains crossed-out content and is not transcribed.)

x subscript 2 equals (U subscript 2 minus U subscript f) divided by (U subscript ice minus U subscript f).  

U subscript 2 equals 15 kilojoules per kilogram plus U subscript l(T) equals 15 kilojoules per kilogram minus 200 kilojoules per kilogram equals minus 185.0 kilojoules per kilogram.  

x subscript 2 equals (minus 185 plus 0.033) divided by (minus 333.442 plus 0.033) equals 0.554.  

Theoretisch nicht möglich, weil, wenn Eis gibt, 0°C die Temperatur sollte sein.  
(Theoretically not possible, because if ice exists, the temperature should be 0°C.)