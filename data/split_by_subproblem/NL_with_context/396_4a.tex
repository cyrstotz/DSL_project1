\( T_1 = \)  
\( p_1 = \)  
\( h_1 = 33.42 \, \text{kJ/kg} \)  

\( S_2 = \)  
\( T_2 = \)  
\( p_2 = p_1 \)  
\( x_2 = 1 \)  

\( S_3 = \)  
\( T_3 = \)  
\( p_3 = 8 \, \text{bar} \)  

\( T_4 = 31.33^\circ \text{C} \)  
\( p_4 = 8 \, \text{bar} \)  
\( x_4 = 0 \)  
\( h_4 = 33.42 \, \text{kJ/kg} \)  

\( \dot{m}_{R134a} (h_2 - h_3) = \dot{W}_K \)

The page contains two diagrams:

1. **First Diagram**:
   - The axes are labeled as follows:
     - The vertical axis is labeled "p (bar)".
     - The horizontal axis is labeled "T (°C)".
   - Three regions are marked:
     - "solid" in the lower left.
     - "fluid" in the upper right.
     - "gas" in the lower right.
   - A curve labeled "Tripel" separates the "solid" and "fluid" regions.
   - Another curve separates the "solid" and "gas" regions.
   - A point labeled "1" is marked on the curve between "solid" and "fluid".
   - A point labeled "2" is marked on the curve between "solid" and "gas".

2. **Second Diagram**:
   - The axes are labeled as follows:
     - The vertical axis is labeled "p (bar)".
     - The horizontal axis is labeled "T (°C)".
   - Three regions are marked:
     - "solid" in the lower left.
     - "fluid" in the lower right.
     - "gas" in the upper right.
   - A curve separates the "solid" and "fluid" regions.
   - Another curve separates the "fluid" and "gas" regions.
   - A point labeled "1" is marked on the curve between "solid" and "fluid".
   - A point labeled "2" is marked on the curve between "fluid" and "gas".