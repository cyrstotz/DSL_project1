A graph labeled "T-s" is drawn, with axes marked as temperature (T) and entropy (s). The graph includes six numbered points (1 through 6) connected by curves and arrows indicating transitions between states. The points are associated with different pressures: \( p_0 \), \( p_1 \), \( p_2 \), \( p_3 \), \( p_5 \), and \( p_6 \). The graph visually represents the thermodynamic process, with isobars labeled for each pressure.  

Below the graph, a table is presented with columns labeled "Zst" (state), "T" (temperature), "s" (entropy), and "P" (pressure). The rows are as follows:  
- State 1: \( T = -30^\circ \text{C} \), \( s = s_1 \), \( P = p_0 \).  
- State 2: \( T = s_3 > s_1 \), \( P = p_2 > p_0 \).  
- State 3: \( T = s_2 > s_2 \), \( P = p_2 = p_2 \).  
- State 4: \( T = s_3 > s_3 \), \( P = p_4 < p_3 \).  
- State 5: \( T = 431.9^\circ \text{K} \), \( P = p_4 = p_5 \).  
- State 6: \( T = s_6 = s_5 \), \( P = p_0 \).

h subscript 5 equals 953.26 kilojoules per kilogram  

Isentropic exit condition:  
T subscript 6 equals T subscript 5 multiplied by (p subscript 6 divided by p subscript 5) raised to the power of (n minus 1 divided by n)  

T subscript 6 equals 328.075 Kelvin  

h subscript 6 equals h (328 Kelvin) plus h (330 Kelvin) minus h (325 Kelvin) divided by 5 Kelvin  

h subscript 6 equals 333.93 kilojoules per kilogram  

Zero equals m dot multiplied by (h subscript 5 minus h subscript 6) plus w subscript 5 squared minus w subscript 6 squared divided by 2  

m dot squared multiplied by (h subscript 6 minus h subscript 5) minus m dot squared multiplied by w subscript 5 squared minus w subscript 6 squared divided by 2  

w subscript 6 squared equals 2 multiplied by (h subscript 5 minus h subscript 6) plus w subscript 5 squared  

w subscript 6 equals 998.26 meters per second  

Delta ex subscript str equals ex subscript str,6 minus ex subscript str,0 equals m dot multiplied by (h subscript 6 minus h subscript 0 minus T subscript 0 multiplied by (s subscript 6 minus s subscript 0) plus delta k zero)  

h subscript 6 equals 333.93 kilojoules per kilogram  

h subscript 0 equals h (265.15 Kelvin)  

s subscript 6 equals s (328 Kelvin)  

s subscript 0 equals s (265 Kelvin)  

Delta k zero equals w subscript 6 squared minus w subscript 0 squared divided by 2  

s (328 Kelvin) approximately equals 1.79 kilojoules per kilogram Kelvin  

s (265 Kelvin) approximately equals 1.86 kilojoules per kilogram Kelvin  

h subscript 0 equals 263 kilojoules per kilogram  

Delta ex subscript str equals ex subscript str divided by m dot equals 119.19 kilojoules per kilogram  

Final result: 119.09 kilojoules per kilogram