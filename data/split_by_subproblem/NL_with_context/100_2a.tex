\( p_5 = p_6 \), \( p_c = 1 \)  

\( T_6 / T_5 = (p_6 / p_5)^{0.6 / 1.4} \)  

\( T_6 = (p_6 / p_5)^{0.6 / 1.4} \cdot T_5 \)  

\( T_6 = 328.07 \, K \)  

\( w_{5/6} = R \cdot (T_B - T_5) / -0.4 \)  

\( w_{5/6} = \dots \)  

\( 7 \cdot 71.498 \, \text{kJ/kg} \)  

---

Diagram:  
A graph is drawn with labeled points 1, 2, 3, 4, and 5. The curve moves upward from 1 to 2, then sharply upward to 3, followed by a downward slope to 4, and finally downward to 5. The axis is not labeled with units but appears to represent a thermodynamic process.  

---

\( w_E = h_e - h_a + (w_e^2 - w_a^2) / 2 \)  

\( w_A^2 = 2h_5 - 2h_6 + w_e^2 - 2w_e \)  

\( w_A' = 2c_p \cdot (T_5 - T_6) + (1000 \, \text{m/s})^2 - 2w_e \)  

\( w_s = 3.16 \, \text{m/s} \)  

\( M = 28.07 \, \text{kg/kmol} \)  

\( R = 28 + \dots \)  

\( R = \dots \)  

\( ? \)