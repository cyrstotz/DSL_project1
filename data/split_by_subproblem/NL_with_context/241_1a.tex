\( Q_{\text{in}} \, (\text{haus-hein}) = \dot{Q}_{\text{aus}} + \dot{Q}_R \)  

\( m_{\text{in}} \, (\text{haus-hein}) = \dot{Q}_R = \dot{Q}_{\text{aus}} \)  

\( h_{\text{aus}} \, (x = 0, 100^\circ \text{C}) = h_f = 419.04 \, \text{kJ/kg} \) — TAB A2  

\( h_{\text{in}} \, (x = 0, 70^\circ \text{C}) = h_f = 292.94 \, \text{kJ/kg} \)  

\( \dot{Q}_{\text{aus}} = -62.182 \, \text{kW} \)

h sub g (2.268 bar) equals h sub 2 minus 243.169 kilojoules per kilogram. Refer to Table A-11.  

m dot times (h sub 3 minus h sub 2) equals W dot sub K.  

m dot equals W dot sub K divided by (h sub 2 minus h sub 3).  

s sub 2 at 2.268 bar equals s sub g equals 0.9232 kilojoules per kilogram Kelvin.  

s sub 2 minus s sub 3.  

h sub 3 at 8 bar, 0.9232 kilojoules per kilogram Kelvin:  
equals 264.15 plus (273.66 minus 264.15) divided by (0.9374 minus 0.9066) times (0.9232 minus 0.9066).  

equals 269.276 kilojoules per kilogram.  

m dot equals W dot divided by 236 kilojoules per second equals 3.18 kilograms per hour.