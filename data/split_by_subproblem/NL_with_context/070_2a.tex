Two diagrams are drawn, both labeled with axes and annotations.  

**First Diagram:**  
- The x-axis is labeled as 'S'.  
- The y-axis is labeled as 'T'.  
- Several curves are drawn, representing processes.  
- Points are marked as 0, 1, 2, 3, 5, and 6.  
- Annotations include:  
  - 'adiabatic, reversible → isentropic' near points 0 to 1.  
  - 'isobar' near points 2 and 3.  
  - 'p2' near the upper curve.  
  - 'p0 = 0.191 bar' near the lower curve.  
  - 'p1 = 0.5 bar' near an intermediate curve.  

**Second Diagram:**  
- The x-axis is labeled as 'S [kJ/kgK]'.  
- The y-axis is labeled as 'T'.  
- Similar points are marked as 0, 1, 2, 3, 5, and 6.  
- Annotations include:  
  - 'adiabatic, reversible → isentropic' near points 0 to 1.  
  - 'isobar' near points 2 and 3.  
  - 'p2' near the upper curve.  
  - 'p5 = 0.5 bar' near an intermediate curve.  
  - 'p0 = 0.191 bar' near the lower curve.  
  - 'T5 = 431.9 K' written near point 5.  
  - '293.15' written near the bottom left corner.  

Both diagrams visually represent thermodynamic processes in a T-S diagram format.