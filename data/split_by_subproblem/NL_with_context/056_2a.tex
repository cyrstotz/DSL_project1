\( w_{Luft} = 200 \, \text{m/s} \)  
\( p_0 = 0.191 \, \text{bar} \)  
\( T_0 = -30^\circ \text{C} \)  

\( q_B = \frac{\dot{Q}_B}{\dot{m}_K} \)  

\( \frac{\dot{m}_M}{\dot{m}_K} = 5.293 \)  

\( c_{pL} = c_p = 1.006 \, \text{kJ/kg·K} \)  
\( n = 1.4 \)  

Diagram:  
A T-s diagram is drawn with the y-axis labeled \( T \, (\text{K}) \) and the x-axis labeled \( s \, \left[ \frac{\text{kJ}}{\text{kg·K}} \right] \).  
The diagram shows the following states and processes:  
- State 0 to 1: \( \eta_{V,s} < 1 \)  
- State 1 to 2: Isentropic  
- State 2 to 3: Isobaric (increase in temperature)  
- State 3 to 4: \( \eta_{T,s} < 1 \)  
- State 4 to 5: Isobaric (\( p_4 = p_5 = 0.5 \, \text{bar} \))  
- State 5 to 6: Isentropic  

Dashed lines represent isobars \( p_2 = p_3 \) and \( p_5 \).