u1 equals minus 200.0928 (see last page).  
uFlüssig equals 0.045.  
uFest equals minus 333.458.  
u2 equals uFlüssig plus x2 times (uFest minus uFlüssig).  
x2 equals (u2 minus uFlüssig) divided by (uFest minus uFlüssig).  
x2 equals 0.05662.  
x2 equals 5.662 percent.

**Closed System**  
\( \Delta E = \Sigma Q - \Sigma W \)  
\( \Delta U = \Sigma Q \)  
\( m_2 (u_2 - u_1) = Q \)  

\( p_{g,2} = p_{g,1} = 1.399 \, \text{bar} \)  

At \( p = 1.4 \, \text{bar} \) (Table A1):  
\( u_1 = u_{flüssig} + x_1 (u_{fest} - u_{flüssig}) \)  
\( u_{flüssig} = -200.0928 \)  
\( u_{fest} = -0.045 \)  
\( u_{flüssig} = -333.458 \)  

\( u_2 = u_{flüssig} + x_2 (u_{fest} - u_{flüssig}) \)  

To \( x_2 ??? \)  

To Task 3d.