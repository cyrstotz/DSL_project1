A graph is drawn with the y-axis labeled as 'T [K]' (temperature in Kelvin) and the x-axis labeled as 'S [kJ / kg·K]' (entropy in kilojoules per kilogram Kelvin).  

The graph shows a process diagram with six points labeled sequentially from 0 to 6. The points are connected by lines, indicating different thermodynamic processes:  
- Point 0 is at the bottom left.  
- Point 1 is slightly above and to the right of Point 0.  
- Point 2 is above Point 1.  
- Point 3 is further to the right and slightly higher than Point 2.  
- Point 4 is below Point 3.  
- Point 5 is to the right of Point 4.  
- Point 6 is directly below Point 5.  

Dashed lines represent isobars labeled as \( p_0 \), \( p_1 \), \( p_2 \), and \( p_4 \).  

A dashed vertical line labeled \( T_{HS} \) is drawn from Point 0 upwards.  

Below the graph, the following notes are written:  
- \( p_5 = p_1 = p_4 \)  
- \( 0 \to 1: \eta < 1 \)  
- \( 1 \to 2: \) isentropic  
- \( 2 \to 3: \) isobaric  
- \( 3 \to 4: \) adiabatic non-reversible  
- \( 4 \to 5: \) isobaric  
- \( 5 \to 6: \) isentropic