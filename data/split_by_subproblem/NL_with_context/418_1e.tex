dS over dt equals the sum of m dot i multiplied by s i of t plus the integral of Q dot over T Q plus S dot erz.  

Therefore, Delta S equals the sum of Delta m i multiplied by s i plus the integral of Q over T Q plus S erz.  

Delta S equals Delta m twelve multiplied by (s twenty degrees Celsius minus s seventy degrees Celsius) plus m gen one multiplied by (s one hundred degrees Celsius minus s seventy degrees Celsius) plus S erz.  

s twenty degrees Celsius equals zero point two nine six six kilojoules per kilogram Kelvin.  
s seventy degrees Celsius equals zero point nine five four nine kilojoules per kilogram Kelvin.  
s one hundred degrees Celsius equals (one point three zero six nine plus x D multiplied by (seven point three five four nine minus one point three zero six nine)) kilojoules per kilogram Kelvin.  

Equals one point three three seven one four.  

S erz equals zero (adiabatic).  

h equals Table A-2.