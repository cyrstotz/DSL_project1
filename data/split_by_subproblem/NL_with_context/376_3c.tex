A diagram is drawn showing a horizontal line with an arrow pointing upward labeled \( \dot{Q} \).  

The equation is written as:  
0 equals \( \dot{m}_w \) times \( (h_1 - h_2) \) plus \( \dot{Q} \) minus \( \dot{W} \).  

The term \( \dot{W} \) is expanded as:  
\( \dot{W} \) equals \( R \) times \( (T_2 - T_1) \) plus \( v_2 \) times \( (v_2 - v_1) \).  

This is further simplified to:  
\( R \) times \( (T_2 - T_1) \) divided by \( 1 - n \).  

A note is written: "Druck bleibt gleich" (Pressure remains constant).  

The equation is rewritten:  
\( \dot{m}_g \) times \( (h_1 - h_2) \) minus \( \dot{Q} \).  

Finally, the heat transfer \( \dot{Q} \) is calculated as:  
\( \dot{Q} \) equals \( \dot{m}_g \) times \( c_p \) times \( (T_1 - T_2) \), which equals 7.083 kJ.  

The result is underlined.