\( T_{in} = 101^\circ \text{C} = 373.15 \, \text{K} \)  
\( T_{out} = 70^\circ \text{C} = 343.15 \, \text{K} \)  

\( \Delta m = ? \)  
\( T_{in,12} = 20^\circ \text{C} = 293.15 \, \text{K} \)  
\( Q_{out} = 35 \, \text{MJ} \)  

It is stated:  
"Steady-state operation: \( \Delta \lambda = Q \)"  

\( m_1 = 5755 \, \text{kg} \)  

An energy balance equation is written:  
\( \Delta E = m_2 \cdot u_2 - m_1 \cdot u_1 = \Delta m \cdot h_{in} + Q_{out} \)  

Expanded equation:  
\( (m_1 + \Delta m) \cdot u_f(70^\circ \text{C}) - m_1 \cdot u_f(100^\circ \text{C}) = \Delta m \cdot h_f(20^\circ \text{C}) + Q_{out} \)  

Values from Table A2 are listed:  
\( u_f(70^\circ \text{C}) = 292.95 \, \text{kJ/kg} \)  
\( u_f(100^\circ \text{C}) = 419.94 \, \text{kJ/kg} \)  
\( u_f(20^\circ \text{C}) = 83.96 \, \text{kJ/kg} \)  

Final equation:  
\( \Delta m = \frac{m_1 \cdot (u_f(70^\circ \text{C}) - u_f(100^\circ \text{C})) - Q_{out}}{h_f(20^\circ \text{C}) - u_f(70^\circ \text{C})} \)  

Result:  
\( = 3637.9 \, \text{kg} \)