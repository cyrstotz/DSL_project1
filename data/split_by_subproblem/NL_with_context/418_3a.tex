R equals R bar divided by M sub g equals 8.314 joules per mole kelvin divided by 50 kilograms per kilomole equals 0.1663 kilojoules per kilogram kelvin.  

V sub g,1 equals 3.14 liters equals 3.14 times 10 to the power of negative 3 cubic meters.  

p sub g,1 times V sub g,1 equals R times T sub g,1 times m sub g,1.  

Therefore, p sub g,1 equals 0.1663 kilojoules per kilogram kelvin times (500 plus 273.15) kelvin divided by 3.14 times 10 to the power of negative 3 cubic meters equals 28636 kilopascals.  

p sub g,2 equals T times D squared divided by 4 times (m sub K plus m sub EW) plus p sub amb.  

p sub g equals 9.81 meters per second squared times (0.1 kilograms plus 32 kilograms) divided by pi times (0.1 meters) squared plus 100023.61 pascals equals 111348.6101 pascals.  

m sub g,1 equals p sub g,1 times V sub g,1 divided by R times T sub g,1 equals approximately 2.445 times 10 to the power of negative 4 kilograms divided by 2.72 times 10 to the power of negative 3.