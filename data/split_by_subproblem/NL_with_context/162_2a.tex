z squared over two m minus z squared over two m plus (phi sub v minus phi sub 1) multiplied by w sub 1 minus (phi sub 1 minus phi sub 1) multiplied by w sub 1 minus (phi sub 1 minus phi sub 1) multiplied by phi equals zero.  

(phi sub 1 minus phi sub 1) multiplied by w sub 1 minus (phi sub 1 minus phi sub 1) equals z squared over two m minus z squared over two m.  

z squared over two m minus z squared over two m plus (phi sub v minus phi sub 1) multiplied by w sub 1 minus (phi sub 1 minus phi sub 1) multiplied by w sub 1 minus (phi sub 1 minus phi sub 1) multiplied by phi equals zero.  

Diagrams:  
Two graphs are drawn side by side.  
- The left graph shows a curve labeled "s" intersecting a straight line labeled "l". The curve starts at the origin and bends upward. A point labeled "phi sub 1" is marked on the curve, with an arrow pointing downward.  
- The right graph shows a similar curve labeled "s" intersecting a straight line labeled "l". The curve starts at the origin and bends upward. A point labeled "phi sub 1" is marked on the curve, with multiple arrows pointing downward and to the side. A straight line crosses through the graph diagonally.  

Additional notes and symbols are scattered across the page but are unclear or illegible.

A diagram is drawn with axes labeled as 'T' (temperature) and 's' (entropy). The diagram shows several curves and arrows, including isobars and process paths. One curve is labeled '1 → 2 → 3 → 4 → 5 → 6', indicating a sequence of states. The curve transitions include a compression phase, heat addition, and expansion. There is a green vertical line labeled 's' and a black curve with arrows indicating direction.