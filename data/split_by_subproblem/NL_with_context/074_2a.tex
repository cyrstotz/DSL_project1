A table is presented with two columns labeled "p" and "T". The rows are as follows:  

- Row 0: \( p = 0.191 \, \text{bar} \), \( T = 30^\circ \text{C} \)  
- Row 1: \( p_1 \)  
- Row 2: \( p_2 \)  
- Row 3: \( p_2 \)  
- Row 4: \( p = 0.5 \, \text{bar} \)  
- Row 5: \( p = 0.5 \, \text{bar} \), \( T = 431.9 \, \text{K} \)  
- Row 6: \( p = 0.191 \, \text{bar} \)  

Below the table, a graph is drawn with the axes labeled:  
- Vertical axis: \( T \, (\text{K}) \)  
- Horizontal axis: \( s \, (\text{J/kg·K}) \)  

The graph shows a T-s diagram with six points labeled 0, 1, 2, 3, 4, and 5. The following processes are indicated:  
- From 0 to 1: Isobar  
- From 1 to 2: Isobar  
- From 2 to 3: Isobar  
- From 3 to 4: Isobar  
- From 4 to 5: Isobar  
- From 5 to 6: Isobar  

The points are connected with lines, and the transitions between states are clearly marked.