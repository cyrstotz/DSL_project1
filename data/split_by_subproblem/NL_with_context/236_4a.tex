State Table:  
| Zustand | P               | T       | x |  
|---------|-----------------|---------|---|  
| 1       | p1 = p2 = 1.5748 bar |         |   |  
| 2       | p2 = p1 = 1.5748 bar | 257.15 K | 1 |  
| ges. Dampf |                 |         |   |  
| 3       | 8 bar           |         |   |  
| 4       | 8 bar           |         | 0 |  

Wk = 25 W adiabatic  

T in Verdampfer = Ti minus 6 K = 263.15 K minus 6 K = 257.15 K  

Ti aus Abb 5 abgelesen: Sublimationspunkt = minus 20 degrees Celsius = 253.15 K  

Ti = 253.15 K plus 10 K = 263.15 K

Two diagrams are drawn:  

1. The first diagram shows a pressure-temperature (p-T) curve with labeled regions.  
- The x-axis is labeled as T [°C].  
- The y-axis is labeled as p [mbar].  
- Two points are marked: i) and ii).  
- An isobar is drawn between the two points.  

2. The second diagram also shows a pressure-temperature (p-T) curve.  
- The x-axis is labeled as T [°C].  
- The y-axis is labeled as p [mbar].  
- Two points are marked: i) and ii).  
- A curve is drawn connecting the two points.