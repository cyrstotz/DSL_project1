A table is provided with the following data:  
- State 0: \( p = 0.191 \, \text{bar} \), \( T = -30^\circ \text{C} \)  
- State 1: (no data visible)  
- State 2: (no data visible)  
- State 3: (highlighted, no data visible)  
- State 4: \( p = 0.5 \, \text{bar} \)  
- State 5: \( p = 0.5 \, \text{bar} \), \( T = 431.9 \, \text{K} \)  
- State 6: \( p = 0.191 \, \text{bar} \), \( T = 328.07 \, \text{K} \)  

Below the table, the following calculation is shown:  
"5 → 6 isentrop  
\( T_6 = T_5 \left( \frac{p_6}{p_5} \right)^{\frac{n-1}{n}} \)  
\( T_6 = 328.07 \, \text{K} \)"  

An energy balance equation is written:  
\( 0 = \dot{m} (h_5 - h_6) + \dot{m} \frac{w_5^2 - w_6^2}{2} + Q^0 - W^0 \)  
Simplified to:  
\( 2 (h_6 - h_5) = w_5^2 - w_6^2 \rightarrow w_6^2 - w_5^2 = 2 (h_6 - h_5) \)

w subscript 6 squared minus w subscript 5 squared equals 2 times (h subscript 6 minus h subscript 5).  

0 equals m subscript g times (h subscript 5 minus h subscript 6) plus m subscript g times (w subscript 5 squared minus w subscript 6 squared divided by 2) plus Q superscript o minus W superscript c.  

w subscript 6 squared equals 2 times (h subscript 5 minus h subscript 6) plus w subscript 5 squared.  

w subscript 6 squared equals 2 times c subscript p times (T subscript 5 minus T subscript 6) plus w subscript 5 squared.  

Isobaric: T divided by v equals constant.  

---