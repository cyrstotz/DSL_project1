Energy balance:  
0 equals m-dot times (h-in minus h-out) plus Q-dot-R minus Q-dot-out.  

Q-dot-out equals m-dot times (h-in minus h-out) plus Q-dot-R.  

h-in equals h-f at 70 degrees Celsius equals 292.98 kilojoules per kilogram (Table A-2).  
h-out equals h-f at 100 degrees Celsius equals 419.04 kilojoules per kilogram (Table A-2).  

Q-dot-out equals 0.3 kilograms per second times (292.98 kilojoules per kilogram minus 419.04 kilojoules per kilogram) plus 100 kilowatts.  

Equals 62.182 kilowatts.  

---

A graph is drawn with the y-axis labeled as "P" and the x-axis labeled as "T". The curve starts near the origin and rises smoothly. The curve is divided into regions labeled "solid", "liquid", and "gas". The point where the three regions meet is labeled "Triple". Below the curve, the region is labeled "gas".