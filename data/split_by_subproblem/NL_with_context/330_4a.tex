Energy balance for the refrigerant in the heat exchanger  

[Diagram]  
A rectangular schematic is drawn, representing the heat exchanger. Inside the rectangle, arrows indicate the flow of refrigerant labeled "R134a" and the heat transfer labeled "\( \dot{Q}_K \)".  

SFP:  
Zero equals mass flow rate of refrigerant times \( h_1 \) minus \( h_2 \) plus \( \dot{Q}_K \).  

Entropy balance for the wall:  
[Diagram]  
Another rectangular schematic is drawn, similar to the first, with arrows indicating heat transfer labeled "\( \dot{Q}_K \)".  

\( Q_e = m(s_2 - s_a) = \frac{\dot{Q}_K}{T_i} - \frac{\dot{Q}_K}{T_{i'} } \)  

\( \dot{Q}_K = m(s_2 - s_a) \cdot (-T_{ii} + T_{ai}) \)  

\( \dot{m}_R = -\frac{\dot{Q}_K}{h_1 - h_2} \)  

[Crossed-out equation]  
An equation involving \( h_1 \), \( h_2 \), and mass flow rate is crossed out and ignored.

Two diagrams are drawn:

1. **First Diagram**:  
   - The x-axis is labeled as "T [°K]" (temperature in Kelvin).  
   - The y-axis is labeled as "p [bar]" (pressure in bar).  
   - A curve is shown resembling a phase diagram.  
   - The curve is divided into three regions:  
     - On the left side, labeled "unterkühlte Flüssigkeit" (subcooled liquid).  
     - In the middle, labeled "Nassdampf" (wet steam).  
     - On the right side, labeled "überhitzter Dampf" (superheated steam).  
   - Two points are marked on the curve:  
     - Point "2" is located in the subcooled liquid region.  
     - Point "1" is located in the wet steam region.

2. **Second Diagram**:  
   - The x-axis is labeled as "T [°K]" (temperature in Kelvin).  
   - The y-axis is labeled as "p [bar]" (pressure in bar).  
   - A curve is drawn resembling a phase diagram.  
   - Two points are marked on the curve:  
     - Point "1" is at the peak of the curve.  
     - Point "2" is slightly below the peak.  
   - Two temperature points are labeled on the x-axis:  
     - "T_i - 6K" (temperature inside minus 6 Kelvin).  
     - "T_i" (temperature inside).