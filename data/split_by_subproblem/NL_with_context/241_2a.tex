A table is drawn with the following columns:  
- \( p \) [bar]  
- \( T \) [°C]  
- \( T \) [K]  
- \( s \)  
- Additional notes (e.g., velocities).  

The rows contain the following data:  
1. \( p = 0.191 \), \( T = -30^\circ \), \( T = 243.15 \, \text{K} \), \( s \) is blank, velocity = \( 200 \, \text{m/s} \).  
2. \( p \) and \( T \) are blank, \( T \) is blank, \( s \) is blank.  
3. \( p \) and \( T \) are blank, \( T \) is blank, \( s \) is blank.  
4. \( p \) and \( T \) are blank, \( T \) is blank, \( s \) is blank.  
5. \( p = 0.5 \), \( T = 431.9^\circ \), \( T = 431.9 \, \text{K} \), \( s \) is blank, velocity = \( w_5 = 220 \, \text{m/s} \).  
6. \( p = 0.191 \), \( T \) is blank, \( T = 2340 \, \text{K} \), \( s \) is blank.  

---

O equals m dot multiplied by (h six minus h five plus (w six squared minus w five squared) divided by two).  

W dot equals R multiplied by (T six minus T five) divided by (one minus k) equals seventy-four point zero seven seven kilojoules per kilogram.  

R equals eight point three one four joules per mole Kelvin divided by twenty-eight point ninety-seven kilograms per kilomole equals zero point two eight six eight kilojoules per kilogram Kelvin.  

w six squared divided by two  
phi squared divided by two  

nach w six umstellen

Delta exergy stream equals w six minus w zero minus T zero multiplied by (s zero of T six minus s zero of T zero) plus one-half multiplied by (w six squared minus w zero squared).  

s six minus s zero equals s zero of T six minus s zero of T zero.  

s zero of T six equals 2.106977 kilojoules per kilogram Kelvin.  

s zero of T zero equals 1.47814 plus (1.51917 minus 1.17824 divided by 250 minus 240) multiplied by (243.15 minus 240).  

Equals 1.49114 kilojoules per kilogram Kelvin.  

Table A-22.  

Delta exergy stream equals 109.995 kilojoules per kilogram.  

---