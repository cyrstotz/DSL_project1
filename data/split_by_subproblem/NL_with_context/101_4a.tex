A table is drawn with the following columns:  
- \( p \)  
- \( T \)  
- \( V \)  
- \( w \)  
- \( \omega \)  
- \( \dot{Q} \)  

The rows contain the following data:  
1. \( p_1 < 8 \, \text{bar} \), \( T_i - 6 \, \text{K} \), empty, empty, empty, \( \dot{Q}_K \).  
2. \( p_1 = p_2 \), \( T_i - 6 \, \text{K} \), empty, \( w_K \), empty, empty.  
3. \( 8 \, \text{bar} \), empty, empty, empty, empty, empty.  
4. \( 8 \, \text{bar} \), empty, empty, empty, empty, empty.  

Notes written next to the table:  
- "Gasförmig, gesättigt" (gaseous, saturated).  
- "Flüssig" (liquid).  

Below the table, three diagrams are drawn:  

1. The first diagram is labeled with \( p \, (\text{bar}) \) on the y-axis and \( T \, (\text{kelvin}) \) on the x-axis.  
   - It shows a saturation curve labeled "Sättigungslinie" (saturation line).  
   - Points are marked as "1" and "i" on the curve, connected by a horizontal line.  
   - Labels: \( p_1 = p_2 \), \( S_2 = S_3 \).  

2. The second diagram is labeled with \( p \, (\text{bar}) \) on the y-axis and \( T \, (\text{kelvin}) \) on the x-axis.  
   - It shows a saturation curve with points labeled "1", "2", "3", and "4".  
   - A horizontal line connects points "1" and "2".  

3. The third diagram is labeled with \( T \, (\text{K}) \) on the x-axis and \( p \, (\text{bar}) \) on the y-axis.  
   - It shows a saturation curve with points labeled "1", "2", "3", and "4".  
   - The processes are labeled: "isobar" and "isentrope".