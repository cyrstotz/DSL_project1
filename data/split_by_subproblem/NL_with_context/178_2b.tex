\( w_6 \), \( T_6 \)  

**Equation for conservation:**  
\( \frac{dE}{dt} = \dot{m} (h_5 - h_6 + \frac{1}{2} w_5^2 - \frac{1}{2} w_6^2) \)  

**Simplified:**  
\( w_6^2 = w_5^2 + 2 \cdot c_p \cdot (T_5 - T_6) \)  

**Ideal Gas Law:**  
\( c_p \cdot (T_5 + T_6) = \frac{1}{2} \cdot (w_5^2 - w_6^2) \)  

**Temperature Ratio:**  
\( \frac{T_6}{T_5} = \left( \frac{p_6}{p_5} \right)^{\frac{n-1}{n}} \)  

**Calculation:**  
\( T_6 = T_5 \cdot \left( \frac{p_0}{p_5} \right)^{\frac{n-1}{n}} \)  

\( T_6 = 431.9 \cdot \left( \frac{0.181}{0.5} \right)^{\frac{1.4-1}{1.4}} \)  

\( T_6 = 328.1 \, \text{K} \)  

**Velocity Calculation:**  
\( w_6 = \sqrt{2 \cdot c_p \cdot (T_5 - T_6) + w_5^2} \)  

\( w_6 = \sqrt{2 \cdot 1.006 \cdot (431.9 - 328.1) + 220^2} \)  

\( w_6 = 507.2 \, \text{m/s} \)

A graph is drawn with the y-axis labeled as T in kilojoules and the x-axis labeled as S in kilojoules per kilogram. The graph shows a process diagram with points labeled 2, 3, 4, 5, and 6. The following processes are indicated:  
- Between points 3 and 4: "isobaric"  
- Between points 4 and 5: "isobaric"  
- Between points 5 and 6: "isobare"  
- Between points 6 and 2: "isobare"  

The graph is a qualitative representation of a thermodynamic process.