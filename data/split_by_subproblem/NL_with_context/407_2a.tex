A graph is drawn with labeled curves and points. The graph appears to represent a thermodynamic process in a T-s diagram.  

- The x-axis is labeled "s" (entropy).  
- The y-axis is labeled "T" (temperature).  
- Three curves are drawn, each representing different isobars.  
- Points are labeled as follows:  
  - "0" at the bottom left.  
  - "1" slightly above and to the right of "0".  
  - "2" further up and to the right of "1".  
  - "3" at the top of the middle curve.  
  - "4" slightly to the right of "3".  
  - "5" further down and to the right of "4".  
  - "6" at the bottom right.  

Lines connect the points in sequence, illustrating the process path.  

Below the graph, a table is drawn with columns labeled:  
- "s"  
- "T"  
- "p"  
- "v"  
- "M"  

Rows contain numerical values corresponding to the points labeled in the graph.  

- For point "0":  
  - s = 0.02225  
  - T = 0.15  
  - p = 0.191  
  - v = 0.15  
  - M = 0  

- For point "1":  
  - s = 0.02225  
  - T = 0.15  
  - p = 0.191  
  - v = 0.15  
  - M = 1  

- For point "2":  
  - s = 0.02225  
  - T = 0.15  
  - p = 0.191  
  - v = 0.15  
  - M = 2  

- For point "3":  
  - s = 0.02225  
  - T = 0.15  
  - p = 0.191  
  - v = 0.15  
  - M = 3  

- For point "4":  
  - s = 0.02225  
  - T = 0.15  
  - p = 0.191  
  - v = 0.15  
  - M = 4  

- For point "5":  
  - s = 0.02225  
  - T = 0.15  
  - p = 0.191  
  - v = 0.15  
  - M = 5  

- For point "6":  
  - s = 0.02225  
  - T = 0.15  
  - p = 0.191  
  - v = 0.15  
  - M = 6  

The table and graph are labeled "Aufgabe 2".