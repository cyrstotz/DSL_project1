A graph is drawn with labeled axes:  
- The vertical axis is labeled as "T [K]" (temperature in Kelvin).  
- The horizontal axis is labeled as "s [Q_s / Q_k]" (entropy).  

The graph shows a thermodynamic process with points labeled as 0, 1, 2, 3, 5, and 6.  
- Point 0 is marked with \( p_0 = 0.191 \, \text{bar} \).  
- Point 5 is marked with \( p_5 = 0.5 \, \text{bar} \).  
- Point 3 is connected to point 5 with a dashed line.

325 divided by 1.78243 equals 323.075.  

1.75883 minus 1.73243 divided by 330 minus 325 equals (328.075 minus 325) plus 1.78243.  

S0 equals 1.294524 kilojoules per kilogram Kelvin.  

---

240 divided by 1.47824 equals 243.15.  

1.51547 minus 1.47824 divided by 250 minus 240 equals (243.15 minus 240) plus 1.47824.  

S0 equals 1.49413 kilojoules per kilogram Kelvin.  

---

S erz equals S6 minus S0 equals 0.3008 kilojoules per kilogram Kelvin.  

---

C x, verl equals T0 multiplied by S erz.  

C x, verl equals 243.15 Kelvin multiplied by 0.3008 kilojoules per kilogram Kelvin.  

C x, verl equals 73.14 kilojoules per kilogram.  

(Boxed result: C x, verl equals 73.14 kilojoules per kilogram.)