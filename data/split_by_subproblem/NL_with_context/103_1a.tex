\( \dot{m}(h_{ein} - h_{aus}) + \dot{Q}_R + \dot{Q}_{aus} = 0 \)  

Water:  
\( h_{ein} \) at 70°C  
A-2 @ 70°C  
Siedend → Nassdampf mit \( x_0 = 0.005 \)  

\( h_{ein} = h_f(70^\circ C) + x_0(h_g(70^\circ C) - h_f(70^\circ C)) \)  
\( = 292.98 \, \frac{\text{kJ}}{\text{kg}} + 0.005(2626.8 \, \frac{\text{kJ}}{\text{kg}} - 292.98 \, \frac{\text{kJ}}{\text{kg}}) \)  
\( = 309.65 \, \frac{\text{kJ}}{\text{kg}} \)  

\( h_{aus} = h_f(100^\circ C) + x_0(h_g(100^\circ C) - h_f(100^\circ C)) \)  
\( = 419.04 \, \frac{\text{kJ}}{\text{kg}} + 0.005(2676.1 \, \frac{\text{kJ}}{\text{kg}} - 419.04 \, \frac{\text{kJ}}{\text{kg}}) \)  
\( = 430.33 \, \frac{\text{kJ}}{\text{kg}} \)  

\( \dot{m} = 0.3 \, \frac{\text{kg}}{\text{s}} \)  

\( 0.3 \, \frac{\text{kg}}{\text{s}} (309.65 \, \frac{\text{kJ}}{\text{kg}} - 430.33 \, \frac{\text{kJ}}{\text{kg}}) + 100 \, \text{kW} + \dot{Q}_{aus} = 0 \)  

\( \dot{Q}_{aus} = -62.296 \, \text{kW} \)  

Wenn Pfeil anders herum zeigt →  
\( 62.30 \, \text{kW} \) fließt heraus  
\( = \dot{Q}_{aus} \)  

In der ganzen Aufgabe habe ich für den Begriff siedend mit einem Dampfanteil \( x \) von \( 0.005 \) gerechnet.