An equation is written:  
ρv = p divided by R times T.  

M_L = 28.52 kg/mol (TAB A1).  

A first law of thermodynamics equation is written:  
O = m dot times (h_5 minus h_6 plus w_5 squared minus w_6 squared divided by 2) plus q_B minus L dot.  

It is simplified to:  
O = h_5 minus h_6 plus w_5 squared minus w_6 squared divided by 2 minus R divided by M_L times T_6 minus T_5 divided by 1 minus n.  

T_6 = T_5 times (p_0 divided by p_5) raised to the power of (n minus 1 divided by n) = 328.07 K.  

Another equation is written:  
w_6 = square root of [c_p times (T_5 minus T_6) plus w_5 squared divided by 2 minus 2 times R divided by M_L times (T_6 minus T_5) divided by 1 minus n].  

This is followed by an equals sign, but no further calculation is shown.