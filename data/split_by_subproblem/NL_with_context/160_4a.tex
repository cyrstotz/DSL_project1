Water in food  

| p          | T          |  
|------------|------------|  
| 1 p1 = p2  | > Ti       |  
| 2 p2 = p3  | Ti         |  
| 3 p3 < T(p)| Ti         |  

Graph 1:  
A pressure-temperature (p-T) diagram is drawn.  
- The x-axis is labeled as T (°C).  
- The y-axis is labeled as p (mbar).  
- The graph includes regions labeled "Eis" (ice), "Wasser" (water), and "Dampf" (vapor).  
- The triple point is marked, and lines are drawn to indicate transitions between phases.  

Graph 2:  
Another pressure-temperature (p-T) diagram is drawn.  
- The x-axis is labeled as T (°C).  
- The y-axis is labeled as p (mbar).  
- The graph includes regions labeled "Flüssig" (liquid), "Wasser Dampf" (water vapor), and "Desublimationsphase" (desublimation phase).  
- The triple point is marked, and an isothermal line is drawn.  
- Points 1, 2, and 3 are labeled to indicate different states.