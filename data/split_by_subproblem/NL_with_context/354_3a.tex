p1 equals 1.96 bar.  
m1 equals m2 equals 0.1 kilograms.  

U1 equals U at 0 degrees Celsius, 1.96 bar plus x multiplied by (Ug at 0 degrees Celsius, 1.96 bar minus Uf at 0 degrees Celsius, 1.96 bar).  
equals negative 133.47 kilojoules per kilogram.  

x2 equals (U2 minus Ufest) divided by (Uflüssig minus Ufest).  

U2g equals m multiplied by RT2 divided by p.  
equals 1.11 multiplied by 10 to the power of negative 3 cubic meters.  

W equals p1 multiplied by (V2 minus V1).  
equals 28.27 joules.  

m1 equals m2 equals mw.  

m2 multiplied by U2 minus m1 multiplied by U1 equals Q12 minus W12.  

U2 equals (Q12 minus W12) divided by mw plus U1.  
equals negative 122.59 kilojoules per kilogram.  

x2 equals (U2 minus Ufest) divided by (Uflüssig minus Ufest).  
equals 63.28 percent.

F equals p times A.

ΣF equals p ambient times pi times (d divided by 2) squared plus m times g equals p1 times pi times (d divided by 2) squared.

p1 equals p ambient plus 4 times m times g divided by pi times d squared.

p1 equals 1.20 bar.

m EW times g.

ΣF equals p2 times pi times (d divided by 2) squared plus m EW times g equals p1 times pi times (d divided by 2) squared.

p2 equals p1 plus 4 times m EW times g divided by pi times d squared.

p2 equals 1.40 bar.

m g equals p times V divided by R times T g1.

R equals R divided by M equals 166.28 joules per kilogram kelvin.

m g equals 1.40 bar times 50.10 times 10 to the power of negative 3 cubic meters divided by 166.28 joules per kilogram kelvin times 773.15 kelvin.

m g equals 3.42 times 10 to the power of negative 3 kilograms.