R equals R divided by M sub g equals 786.28 joules per kilogram Kelvin equals 0.786 joules per gram Kelvin.  

p sub g equals m sub EW times g divided by D squared times pi plus p sub 0 plus m sub K times g divided by D squared times pi.  

Equals 1.4 bar.  

p times R equals F.  

p sub g times V sub g,1 equals n sub g times R times T. Therefore, n sub g equals p sub g,1 times V sub g,1 divided by R times T sub g,1 equals 3.42 joules per gram Kelvin.

\( u_{vs} \), \( u_{l} \), \( u_{(new)} \), \( 2 \), \( 2000^\circ \text{C} \)  

\( m_{tot,1} = 100 \, \text{g} \)  

\( V_{2,FW} - V_{1,FW} = -Q_{12} = m_{tot} \left( x_2 u_L + (1 - x_2) u_S \right) - \left( x_1 u_L + (1 - x_1) u_S \right) \)  

\(- \frac{Q_{12}}{m_{tot}} = x_2 (u_L - u_S) + u_S - \left( x_1 (u_L - u_S) + u_S \right)\)  

\( = (x_2 - x_1) (u_L - u_S) \)  

\( x_2 = - \frac{Q_{12}}{m_{tot} (u_L - u_S)} + x_1 = 0.574 \)