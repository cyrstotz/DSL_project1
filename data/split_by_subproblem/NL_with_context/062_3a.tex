R equals 8.314 kilojoules per kilomole per kelvin divided by 50 kilograms per kilomole equals 0.166 kilojoules per kilogram per kelvin.  

p sub g,1 equals p sub EW,1 plus m sub K times g divided by A plus p sub amb.  

A equals pi times (D divided by 2) squared.  

p sub g,1 equals p sub EW,1 plus m sub K times g divided by A plus p sub amb equals 32 kilograms times 9.81 meters per second squared divided by pi times (0.1 meters divided by 2) squared plus 1.105 newtons per square meter equals approximately 1.40 bar.  

pV equals mRT.  
T sub g,1 equals 773.15 kelvin.  
p sub g,1 equals 1.40 times 10 to the power of 5 newtons per square meter.  
V sub g,1 equals 3.14 times 10 to the power of negative 3 cubic meters.  
m sub g equals p sub g,1 times V sub g,1 divided by R times T sub g,1 equals 0.166 times 10 to the power of negative 3 joules per kilogram per kelvin times 773.15 kelvin equals approximately 3.43 grams.

Q twelve equals one thousand three hundred joules  

V one EW equals V two EW implies v one EW equals v two EW  

A graph is drawn with a wavy line crossing the horizontal axis multiple times.  

V EW equals zero point six times v g at zero degrees Celsius plus (one minus zero point six) times v f at zero degrees Celsius equals one hundred twenty-five point ninety-eight cubic meters per kilogram  

x two equals v two minus v f divided by v g minus v f equals u two minus u f divided by u g minus u f  

T EW two: delta U equals Q twelve minus W twelve  

Delta U equals m times (u two minus u one)  

u two minus u one equals i f times (T two minus T one)