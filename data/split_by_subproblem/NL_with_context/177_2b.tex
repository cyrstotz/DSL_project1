The equation for \( Q \) is written as:  
\( Q = \dot{m}_{ges} \left( h_5 - h_6 + \frac{w_5^2 - w_6^2}{2} \right) - \dot{W}_{+sc} \)

The equation for \( \dot{W}_{+sc} \) is written as:  
\( \dot{W}_{+sc} = \dot{m} \left( - \int_{1}^{3} p \, v \, dp \right) \)  
\( = - \dot{m} \left( p_2 - p_1 \right) v \)

The equation \( p v = R T \) is written, followed by:  
\( v_s = \frac{R T_5}{p_5} \)

The equation for \( R \) is written as:  
\( R = c_p^{ig} - c_v^{ig} \)

The equation for \( k \) is written as:  
\( k = \frac{c_p^{ig}}{c_v^{ig}} \), \( p = \frac{c_p^{ig} - c_v^{ig}}{k} \)

The equation for \( R \) is expanded as:  
\( R = c_p^{ig} \left( 1 - \frac{1}{k} \right) \)  
\( = 0.287 \, \frac{kJ}{kg·K} \)

The equation for \( v_s \) is written as:  
\( v_s = \frac{0.287 \, \frac{kJ}{kg·K} \cdot 437.9 \, K}{50,000 \, Pa} \)  
\( = 0.002479 \, \frac{m^3}{kg} \)

\( \dot{W}_{+sc} \) is partially written but not completed.

T6 divided by T5 equals (p6 divided by p5) raised to the power of (n minus 1) divided by n.  
T6 equals T5 multiplied by (p6 divided by p5) raised to the power of (n minus 1) divided by n.  
T6 equals 431.9 Kelvin multiplied by (19100 Pascal divided by 50000 Pascal) raised to the power of 0.4 divided by 1.4.  
T6 equals 32 Kelvin.  

---