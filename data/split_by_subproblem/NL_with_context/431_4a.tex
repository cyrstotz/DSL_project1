Two diagrams are drawn:  
1. The first diagram is labeled "Schritt ii" and shows a pressure-temperature (p-T) graph. It includes a phase boundary labeled "gasförmig" (gaseous) and "flüssig" (liquid). A point labeled "1" is marked on the liquid phase boundary, and an arrow points upward toward the gaseous region.  
2. The second diagram is labeled "Schritt i" and shows another p-T graph with a dome-shaped phase region. The dome is divided into sections labeled "1," "2," "3," and "4." The left side of the dome is labeled "unterkühlte Flüssigkeit" (subcooled liquid), the middle is labeled "Massedampf" (wet steam), and the right side is labeled "überhitzter Dampf" (superheated steam).