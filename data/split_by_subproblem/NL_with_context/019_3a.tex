Requested: \( p_{g,1} \) and \( m_g \)  

Given:  
\( T_{g,1} = 500^\circ \text{C} = 773.15 \, \text{K} \)  
\( V_{g,1} = 3.14 \, \text{L} = 0.00314 \, \text{m}^3 \)  

Diagram: A rectangular box is shown with arrows indicating \( p_0 \), \( m_K g \), and \( m_{EW} g \) acting downward, and \( p_1 \) acting upward.  

Equations:  
\( p_1 \cdot A = p_0 \cdot A + m_K g + m_{EW} g \)  
\( p_1 = p_0 + \frac{m_K g}{A} + \frac{m_{EW} g}{A} \)  

Area calculation:  
\( A = \frac{D^2}{4 \pi} \)  

Substituting values:  
\( p_1 = 100000 \, \text{Pa} + \frac{32 \, \text{kg} \cdot 9.81 \, \text{m/s}^2}{\frac{D^2}{4 \pi}} + \frac{0.1 \, \text{kg} \cdot 9.81 \, \text{m/s}^2}{\frac{D^2}{4 \pi}} \)  

Result:  
\( p_1 = 5.01 \, \text{bar} \)  

Gas mass calculation:  
\( m_g = \frac{p \cdot V}{R \cdot T} \)  

Result:  
\( m_g = 0.255 \, \text{kg} \)  

---