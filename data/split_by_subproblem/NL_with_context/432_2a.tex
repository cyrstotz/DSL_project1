A graph is drawn with the title "T, s". The x-axis is labeled "s [kJ/kg·K]" and the y-axis is labeled "T [K]".  
The graph shows a qualitative T-s diagram with labeled points:  
- Point 0 at the bottom left.  
- Point 1 connected to point 2 via a curve labeled "P2".  
- Point 2 connected to point 3 via a curve.  
- Point 3 connected to point 4 via a straight line.  
- Point 4 connected to point 5 via a curve labeled "P5 = 0.5 bar".  
- Point 5 connected to point 6 via a straight line.  
Dashed lines connect points 0 to 5 and 5 to 6.

Delta ex_str equals ex_str_6 minus ex_str_0 equals h_6 minus h_0 minus T_0 multiplied by (s_6 minus s_0) plus Delta KE.  
Delta KE equals w_6 squared divided by 2 minus w_0 squared divided by 2.  

Delta ex_str equals [c_p multiplied by (T_6 minus T_0) minus T_0 multiplied by (c_p multiplied by ln(T_6 divided by T_0))] plus w_6 squared divided by 2 minus w_0 squared divided by 2.  

Delta ex_str equals [1.006 kilojoules per kilogram Kelvin multiplied by (325.09 Kelvin minus 243.15 Kelvin) minus 243.15 Kelvin multiplied by (1.006 kilojoules per kilogram Kelvin multiplied by ln(325.09 Kelvin divided by 243.15 Kelvin))] plus (220 meters per second squared divided by 2 minus 200 meters per second squared divided by 2).  

Delta ex_str equals 122.2 kilojoules per kilogram.