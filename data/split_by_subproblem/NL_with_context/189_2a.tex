A T-s diagram is drawn with temperature (T in Kelvin) on the vertical axis and entropy (S in kilojoules per kilogram Kelvin) on the horizontal axis. The diagram includes labeled points:  
- Point 0 at the origin.  
- Point 1 labeled as "isentrop".  
- Point 2 labeled as "isobar".  
- Point 3 labeled as "isobar".  
- Point 4 labeled as "isentrop".  
- Point 5 labeled as "isobar".  
- Point 6 labeled as "isentrop".  

The diagram shows isobaric and isentropic processes connecting the points, with arrows indicating the direction of the processes. The pressures \( p_0 = p_6 \), \( p_2 = p_4 \), and \( p_1 \) are marked.  

An annotation states: "adiabatic-reversible = isentrop".  

---

h6: Interpolated from Table A22 (with T6):  

325.31 + (320.214 - 325.31) / (330 - 325) * (328.075 - 325)  

h6 = 325.31 + (320.214 - 325.31) / (330 - 325) * (328.075 - 325)  
= 328.40354 kJ/kg  

=> w6 = sqrt(2 * (h6 - h5) + w5 squared) = 219.52 m/s