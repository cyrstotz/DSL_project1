A diagram is drawn labeled as a T-s diagram. The axes are labeled as follows:  
- The vertical axis is labeled "T [K]" (temperature).  
- The horizontal axis is labeled "s [kJ/kg·K]" (entropy).  

The diagram includes several points labeled:  
- "1" (starting point), "2", "3", "4", "5", and "6".  
- Arrows indicate transitions between these points, such as "1 → 2", "2 → 3", "3 → 4", "4 → 5", and "5 → 6".  

Additional annotations include:  
- "adiabatic" near the transition from point 3.  
- "isobar" near the transition from point 5.

The page contains two diagrams:

1. The first diagram is labeled "a)" and shows a graph with axes labeled "P out" (vertical axis) and "v m^3/kg" (horizontal axis). The graph includes several curves and points labeled "1", "2", "3", and "4". The curves intersect and loop, forming a complex shape.

2. The second diagram shows a graph with axes labeled "P out" (vertical axis) and "v m^3/kg" (horizontal axis). This graph includes a dome-shaped curve with points labeled "1", "2", "3", and "4". The points are connected by lines, forming a cycle within the dome. 

No additional text or equations are present on the page.