Q equals m dot times (h in minus h out) plus Q dot R minus Q dot AB.  

Q dot AB equals Q dot R plus m dot times (h in minus h out) equals 100 kilowatts plus 0.3 kilograms per second times (292.98 kilojoules per kilogram minus 419.04 kilojoules per kilogram) equals negative 62.18 kilowatts.  

h in equals h at 70 degrees Celsius equals 292.98 kilojoules per kilogram.  
h out equals h at 100 degrees Celsius equals 419.04 kilojoules per kilogram.

0 → 1: adiabatic → Venturi tube  
1 → 2: isentropic  
2 → 3: isobaric  
3 → 4: adiabatic, irreversible → turbine  
4 → 5: mixing, isobaric  
5 → 6: isentropic  

\( p_5 = p_6 = p_0 = 0.5 \, \text{bar} \)  

Diagram:  
A T-s diagram is drawn with labeled points 0, 1, 2, 3, 4, 5, and 6.  
- The x-axis is labeled \( s \, [\text{kJ/kg·K}] \).  
- The y-axis is labeled \( T \, [\text{K}] \).  
- Processes are labeled as isentropic, isobaric, and adiabatic.  
- Dotted lines indicate pressure levels \( p_3 = p_2 \), \( p_4 = p_5 \), and \( p_6 = p_0 \).  

---