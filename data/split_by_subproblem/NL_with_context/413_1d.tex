\( T_2 = 70^\circ \text{C}, \dot{m}_{in} = 0 = \dot{m}_{out} \)  

\( \Delta m_{12} @ 20^\circ \text{C} \)  
\( Q_{R,12} = Q_{out,12} = 35 \, \text{MJ} \)  

Energy balance for water, closed system:  

\( E_2 - E_1 = Q_{R,12} = \Delta m_{12} \cdot (u_2 - u_1) \)  

\( \Delta m_{12} = \frac{Q_{R,12}}{u_2 - u_1} \)  

\( u_2 @ 70^\circ \text{C} = 292.95 \, \text{kJ/kg} \rightarrow \text{Table A-2} \)  
\( u_1 @ 100^\circ \text{C} = 418.84 \, \text{kJ/kg} \rightarrow \text{Table A-2} \)  

\( \Delta m_{12} = \frac{-35.00 \, \text{kJ}}{292.95 - 418.84 \, \text{kJ/kg}} = 277.8 \, \text{kg} \)

Weiterführung  

Half-open system:  
m2 u2 minus m1 u1 equals delta m12 (h ein) plus Q aus  

m2 equals m1 plus delta m12  

delta m12 (h ein minus u2) equals m1 u1 plus delta m12 u2 equals m1 u1 plus Q aus plus u2 m1  

delta m12 (h ein minus u2)  

delta m12 equals m1 u1 minus Q aus plus u2 m1 divided by (h ein minus u2)  

h ein equals 83.96 kilojoules per kilogram → Table A-2 at 20 degrees Celsius  
u1 at 100 degrees Celsius equals 418.94 kilojoules per kilogram → Table A2  
u2 at 20 degrees Celsius equals 292.55 kilojoules per kilogram → Table A2  

Q aus equals minus 35 megajoules