A graph is drawn with the x-axis labeled as 'S [u Joules divided by u Kelvin]' and the y-axis labeled as 'T [u Kelvin]'. The graph shows a thermodynamic process with labeled points:  
- Point 1: 's = const'  
- Point 2: 's = const'  
- Point 3: 'Isobare 2, 3'  
- Point 4: 'Isobare 4, 5'  
- Point 5: 's = const'

Two times m dot times (u six minus u five) plus m dot times (w five squared divided by two) equals m dot times w six squared divided by two.  

u five (431.9, 0.5 bar) equals u (435) minus u (430) divided by (435 minus 430) times (431.9 minus 430) plus u (430).  

Equals (444.67 minus 437.43) divided by 10 times (1.9) plus 437.43.  

Equals 433.36.  

u six (-30 degrees Celsius, 0.191 bar) equals u (250) minus u (240) divided by (250 minus 240) times (247.75 minus 240) plus u (240).  

Equals (250.05 minus 240.02) divided by (250 minus 240) times (3.15) plus 240.02.  

Ideal gas: u (T two) minus u (T one) equals c p times (T two minus T one).  

Therefore, w six squared equals two times c p times (T six minus T five) plus w five squared.  

w six equals square root of two times 1.006 times (568.6 minus 431.9) plus (220 squared).  

Equals 226.6 meters per second.  

Equals 568.71 meters per second.