A diagram is shown with a labeled cross-section. The diagram includes:  
- A horizontal rectangular section labeled "H2O" in the center.  
- Above and below the "H2O" section, there are dashed orange lines labeled "Kühlmittel" (coolant).  
- An arrow labeled "Qaus" pointing upwards from the "H2O" section through the top dashed line.  
- The outermost layers are labeled "Wand" (wall).  

Below the diagram, the following text and equations are written:  

**Bilanzgleichung: Entropiebilanz um die Wand:**  
Q equals Qaus divided by T_H2O minus Qaus divided by T_KF plus Ṡ_erz.  

T_H2O equals T_R equals 100 degrees Celsius equals 373.15 Kelvin.  
T_KF equals 295 Kelvin.  

Ṡ_erz equals Qaus multiplied by (1 divided by T_H2O plus 1 divided by T_KF).  

Ṡ_erz equals 65 kilowatts multiplied by (1 divided by 295 Kelvin minus 1 divided by 373.15 Kelvin).  
Ṡ_erz equals 4.694 times 10 to the power of -2 kilowatts per Kelvin.  

---

Below the equations, a table is drawn with the following columns and rows:  

| Q       | W       | Σ       | T       |  
|---------|---------|---------|---------|  
|         |         | 0       |         |  
| Qaus12  |         | 1       | 100°C   |  
| 35 MJ   |         | 2       | 70°C    |  

Δm_12 is written to the left of the table.