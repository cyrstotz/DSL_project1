Calculate \( p_{g,1} \) and \( m_g \):  
The total gas load is calculated as:  
\( m_{EW} + m_{K} + p \cdot \text{membrane} = 0.1 \, \text{kg} + 32 \, \text{kg} = 32.169 \, \text{kg} \)  

\( F_g = m_g \cdot g = 32.169 \, \text{kg} \cdot 9.81 \, \text{m/s}^2 = 319.907 \, \text{N} \)  

\( p = p_0 + \frac{F_g}{A} = p_{\text{amb}} + \frac{F_g}{\pi \cdot r^2} \)  

\( A = \pi \cdot r^2 = \pi \cdot (0.05 \, \text{m})^2 \)  

\( p_{g,1} = 1.401 \, \text{bar} \)  

For an ideal gas:  
\( m_g = \frac{p_{g,1} \cdot V_{g,1}}{R_g \cdot T_g} \)  

With \( R_g = \frac{8.314 \, \text{m}^2 \cdot \text{kg} \cdot \text{s}^{-2} \cdot \text{K}^{-1}}{50 \, \text{kg/kmol}} = 0.16623 \, \text{J/g·K} \):  

\( m_g = \frac{1.401 \cdot 10^5 \, \text{Pa} \cdot 0.00314 \, \text{m}^3}{0.16623 \, \text{J/g·K} \cdot (500 + 273.15) \, \text{K}} = 3.422 \, \text{g} \)  

---