Two diagrams are drawn:  

1. The first diagram shows a pressure-temperature (p-T) graph.  
   - The x-axis is labeled as 'T (Kelvin)'.  
   - The y-axis is labeled as 'p (bar)'.  
   - A curve is drawn, labeled 'Nassdampf' (wet steam).  
   - A point on the curve is labeled 'kritischer Punkt' (critical point).  

2. The second diagram also shows a pressure-temperature (p-T) graph.  
   - The x-axis is labeled as 'T (Kelvin)'.  
   - The y-axis is labeled as 'p (bar)'.  
   - A line labeled 'Gas' is drawn above the triple point.  
   - The triple point is labeled 'Tripelpunkt'.  
   - A region below the triple point is labeled 'Fest' (solid) and 'Subl.' (sublimation).  
   - Another region is labeled 'Flüssig' (liquid).  

---

ε equals Q-dot-ab divided by Q-dot-zul equals 1 divided by W-dot multiplied by 1 divided by Q-dot-zul minus Q-dot-ab divided by Q-dot-zul equals Q-dot-ic divided by Q-dot-ab.  

Q-dot-ic equals m-dot multiplied by [h2 minus h3].  
Q-dot-ab equals m-dot multiplied by [h4 minus h3].