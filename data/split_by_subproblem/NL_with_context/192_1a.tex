A diagram is drawn showing a reactor system with labeled components: \( \dot{Q}_R = 100 \, \text{kW} \), \( \dot{m}_{in} \), \( \dot{m}_{out} \), and \( \dot{Q}_{aus} \).  

The first law of thermodynamics is written as:  
\( \dot{m} (h_1 - h_2 + \frac{v^2}{2} + g z) + \Sigma \dot{Q} - \dot{W} = 0 \) (stationary).  

A condition is noted:  
\( 0 < x_D < 1 \rightarrow \text{ND-Gebiet} \).  

State 1: \( T_1 = 70^\circ \text{C}, x_D = 0.005 \).  
\( h_1 = h_f(70) + x_D (h_g(70) - h_f(70)) \).  
\( h_1 = 304.65 \, \text{kJ/kg} \).  
\( h_f = 292.58 \, \text{kJ/kg}, h_g = 2626.8 \, \text{kJ/kg} \) (from Table A2).  

State 2: \( T_2 = 100^\circ \text{C} \).  
\( h_2 = h_f(100) + x_D (h_g(100) - h_f(100)) \).  
\( h_2 = 430.33 \, \text{kJ/kg} \).  
\( h_f = 419.04 \, \text{kJ/kg}, h_g = 2676.1 \, \text{kJ/kg} \) (from Table A2).  

The energy balance is written:  
\( \dot{m} (h_1 - h_2) + 100 \, \text{kW} - \dot{Q}_{aus} = 0 \).  

Result:  
\( \dot{Q}_{aus} = 62.3 \, \text{kW} \).