Zustand 2  

Gesamte Wärmekapazität bleibt gleich:  
\( m_{ges} \cdot c_p \cdot T_1 + m_{EW} \cdot c_p \cdot T_1 - m_{ges} \cdot c_p \cdot T_2 + m_{EW} \cdot c_p \cdot T_2 \)  

\( c_p = R + c_v \)  
\( = 0.799 \, \text{kJ/kg·K} \)  

\( m_{EW} = m_{Eis} + m_{Wasser} \)  
\( m_{Eis} = 0.06 \, \text{kg} \)  
\( m_{Wasser} = 0.04 \, \text{kg} \)  

Die thermische Masse vom Wasser ist viel höher als die Dichte vom Gas. Daher ist das Gas kurzzeitig bei \( 0^\circ \text{C} \).  

\( p = \frac{p_{amb} T_2}{T_M} = \left( \frac{p_0}{p_{amb}} \right) \cdot \frac{T_2}{T_M} \cdot R \cdot T \)