The following calculations and equations are written:  

w5 = 220 meters per second.  
p5 = 0.5 bar.  

An equation for mass flow rate is written:  
m dot equals rho times v dot equals p0 divided by R times T0 equals p0 divided by R times T.  

The text states:  
"Since the nozzle is adiabatic and reversible, the following applies:  
T6 equals T0 times (p6 divided by p0) raised to the power of (n minus 1) divided by n equals 328.07 Kelvin."  

Another statement:  
"Since the nozzle is adiabatic, Q equals 0."  

An energy balance equation is written:  
0 equals m dot times (h6 minus h5) plus m dot times (w6 squared minus w5 squared divided by 2).  

Further calculations:  
w6 squared divided by 2 equals Cp times (T5 minus T6) plus w5 squared divided by 2.  

Final result:  
w6 equals square root of 2 times Cp times (T5 minus T6) plus w5 squared equals 506.02 meters per second.