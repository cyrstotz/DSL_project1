Cooling cycle:  
- Zustand 1: \( p_1 \), \( x_1 = 0.9 \), indicates expansion.  
- Zustand 2: \( p_2 = p_1 \), \( x_2 = 1 \), indicates fully evaporated as gas.  
- Zustand 3: \( p_3 = 8 \, \text{bar} \), \( x_3 = 2-3 \), indicates adiabatic compression.  
- Zustand 4: \( p_4 = p_3 \), \( x = 0 \), indicates fully condensed at 8 bar.  

\( T_a = -11 \, \text{bis} \, 8 \, \text{bar} \)  
\( T_a = 31.33^\circ \text{C} \)  

\( T_i \) is 10 K above the sublimation point of frozen water and 5 mbar below the triple point.  

\( \Delta T_i = -10^\circ \text{C} \) → \( T_{\text{evaporator}} = -16^\circ \text{C} \) → \( T_i = -7^\circ \text{C} \)  

Zustand 1, 2: \( x = 79 \% \), \( T_2 = -16^\circ \text{C} \), \( T_{\text{ab}} = -10^\circ \text{C} \), \( p_2 = p_1 = 2.5748 \, \text{bar} \).  

We calculate \( p_2 \) and \( p_3 \) and \( W_r \) of the compressor, which is adiabatic-reversible.  

\( W_{23} / \dot{m} = ? \) does not work because no polytropic process is used.  

A small arrow labeled "R134a" is drawn at the bottom right.