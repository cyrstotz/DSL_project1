The equation for the work rate is given as:  
W dot K equals m dot times (h3 minus h1).  
This implies:  
m dot equals W dot K divided by (h3 minus h1).

To calculate, we need h3 and h1 from the tables A-10, A-11, and A-12.  

h2 is interpolated from h2(Ti equals 6), where Ti is determined with the p-T diagram.  
At Ti equals minus 20 degrees Celsius (from Table A-10):  
h2,s equals h2 (T equals minus 26 degrees Celsius) equals 231.62.  
The entropy is s2 equals sg (T equals minus 26 degrees Celsius) equals 0.83980.

Since the transition from 2 to 3 is adiabatic and reversible:  
Delta S equals 0.  

From Table A-12, we find h3:  
h3 equals h3(s3) equals 273.66 plus (258.33 minus 273.66) times (0.83980 minus 0.8397) divided by (0.8391 minus 0.8397), approximately equals 259.14.

The mass flow rate is calculated as:  
m dot equals W dot K divided by (h3 minus h1) equals 22 divided by 33.55 equals 0.658 kilograms per second, approximately equals 2.37 kilograms per liter.

Ti was determined with the diagram at minus 20 degrees Celsius, giving:  
Ti equals minus 20 degrees Celsius minus 6 equals minus 26 degrees Celsius.