Two diagrams are drawn:  

1. **First Diagram**:  
   - The y-axis is labeled as 'T [K]'.  
   - The x-axis is labeled as 'S [kJ/kg·K]'.  
   - Several curves are plotted, including a prominent dome-shaped curve and intersecting lines.  

2. **Second Diagram**:  
   - The y-axis is labeled as 'T [K]'.  
   - The x-axis is labeled as 'S [kJ/kg·K]'.  
   - Six states are marked along the curves, labeled as '0', '1', '2', '3', '4', '5', and '6'.  
   - Two arrows are drawn:  
     - One labeled 's = const' pointing horizontally from state 1 to state 2.  
     - Another labeled 's ≠ const' pointing diagonally from state 0 to state 1.  
   - The curve labeled 'P0' is drawn on the right side.  

Below the second diagram, the text reads:  
"Zustand 4 liegt nicht per se auf der p5 Isobare."  
Translation: "State 4 does not necessarily lie on the p5 isobar."