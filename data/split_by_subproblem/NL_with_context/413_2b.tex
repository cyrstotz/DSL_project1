Given: \( w_6 \), \( T_6 \)  
The equation for steady-state flow is written as:  
"0 = h_in minus h_out plus w_in squared minus w_out squared divided by 2".  

From this, the equation for \( w_out \) is derived:  
\( w_out = \sqrt{2 \cdot (h_in minus h_out) plus w_in squared} \).  

For an ideal gas:  
\( c_p \cdot (T_in minus T_out) plus v squared \cdot (p_3 minus p_1) \).  

The equation for \( T_out \) is written as:  
\( T_out = T_in \cdot \left(\frac{p_0}{p_1}\right)^{\frac{1.4 minus 1}{1.4}} \).  

This simplifies to:  
\( T_out = T_in \).  

A note is written: "Limit is if \( w_out = w_in \)".

Steady flow process in a nozzle:  
Energy balance, the reversible case:  
Adiabatic:  
h6 minus h5 plus w6 squared minus w5 squared divided by 2 equals 0  

w6 squared equals 2 times (h5 minus h6) plus w5 squared  
equals 220 meters squared per second squared  

Tb equals 75 times (p6 divided by p5) raised to the power of (k minus 1 divided by k)  
equals 328.07468 Kelvin  

---