**Header:** Thermo I  
**Subheader:** Aufgabe 2  

**Content:**  
Three diagrams are drawn, labeled with temperature (T) on the y-axis and entropy (S) on the x-axis.  

1. **First Diagram:**  
   - A curved process line is shown, starting at point 0 and moving through points 1, 2, 3, 4, 5, and 6.  
   - The points are labeled as follows:  
     - 0: \( T_0 = -30^\circ \text{C}, p_0 = 0.191 \text{bar} \)  
     - 0 → 1: \( p_0 \), \( p_1 \)  
     - 1 → 2: Isentropic, \( p_1 \), \( p_2 \)  
     - 2 → 3: Isobaric, \( T \), \( p_2 = p_3 \)  
     - 3 → 4: Not isentropic, \( p_4 \), \( T \)  
     - 4 → 5: Isobaric, \( p_4 = p_5 \)  
     - 5 → 6: Isentropic, \( p_6 = p_0 \)  

2. **Second Diagram:**  
   - Another curved process line is drawn, starting at point 0 and moving through points 1, 2, 3, 4, 5, and 6.  
   - The points are connected with annotations:  
     - 0 → 1: Isentropic  
     - 1 → 2: Isentropic  
     - 2 → 3: Isobaric (\( p_2 \))  
     - 3 → 4: Not isentropic  
     - 4 → 5: Isobaric (\( p_4 \))  
     - 5 → 6: Isentropic (\( p_0, p_6 \))  

3. **Third Diagram:**  
   - A zoomed-in view of the process line between points 4, 5, and 6.  
   - The annotations include:  
     - 4 → 5: Isobaric (\( p_4, p_5 \))  
     - 5 → 6: Isentropic  

**Additional Notes:**  
- The diagrams are labeled with process descriptions such as "isobaric" and "isentropic" at various points.  
- Curved lines represent thermodynamic processes between states.  
- The diagrams visually depict the transitions between pressure and temperature states in a jet engine cycle.

\( w_{ex,str} = \dot{m} \left[ h_6 - h_0 - T_0 \left( s_6 - s_0 \right) + \frac{w^2}{2} \right] \)  

\( ex_{str,6} = c_p \left( T_6 - T_0 \right) - T_0 c_p \ln \left( \frac{T_6}{T_0} \right) - R \ln \left( \frac{p_6}{p_0} \right) \)  

\( ex_{str,6} = 1.006 \, \text{kJ/kg·K} \cdot \left( 328.075 - 243.15 \right) \, \text{K} - 243.15 \, \text{K} \cdot 1.006 \, \text{kJ/kg·K} \cdot \ln \left( \frac{328.075}{243.15} \right) \)  

\( ex_{str,6} = h_6 - h_0 - T_0 \left( s_6 - s_0 \right) + \frac{w^2}{2} = 42.45 \, \text{kJ/kg} \)  

---