A diagram is drawn with a vertical axis labeled "T" and horizontal lines representing different temperature levels. The left side of the diagram is labeled "ein" (inlet), and the right side is labeled "sieden" (boiling).  

Below the diagram:  
- \( \dot{m}_{in} = 0.3 \, \text{kg/s} \)  
- \( T_{ein} = 70^\circ \text{C} \rightarrow \text{sieden flüssig} \) (boiling liquid)  
- \( m_{ges,1} = 5755 \, \text{kg} \)  
- \( x_D = 0.005 \)  
- \( T = 100^\circ \text{C} \, \text{immer} \) (always)  

a)  
\( Q_{aus} \)  
\( 0 = \dot{m}_{in} \left( h_e - h_a \right) + \phi + Q_{aus} - \dot{W}_{n} \)  

\( Q_{aus} = \dot{m}_{ein} \left( u_a - h_e \right) = 37.8 \, \text{kW} \)  

\( u_e = h_f(70^\circ \text{C}) = 292.58 \, \frac{\text{kJ}}{\text{kg}} \)  

\( u_a(100^\circ \text{C}) = h_f(100^\circ \text{C}) = 419.04 \, \frac{\text{kJ}}{\text{kg}} \, \text{siedenflüssig} \) (boiling liquid)

The integral of Tds from state e to state a divided by (sa minus se) is equal to Q out divided by m dot.  

Tds equals Q out divided by m dot.  

This is equal to T2 minus T1 divided by cv times the natural logarithm of T2 divided by T1.  

Ideal fluid is noted.  

---