A table is drawn with columns labeled \( 0 \), \( 1 \), \( 2 \), \( 3 \), \( 4 \), \( 5 \), and \( 6 \). The rows are labeled \( T \) and \( S \). The following entries are visible:  
- \( T \):  
  - \( 1 \): \( < p_c \)  
  - \( 3 \): \( > T_2 \)  
  - \( 5 \): \( 0.5 \, \text{bar}, \, 431.9 \, \text{K} \)  
- \( S \):  
  - \( 2 \): \( S_2 = S_1 \)  
  - \( 5 \): \( S_5 = S_6 \)  

Below the table, a graph is drawn with axes labeled \( T \, (\text{K}) \) and \( S \, (\text{kJ/kg·K}) \). The graph includes the following points and annotations:  
- Points labeled \( 1 \), \( 2 \), \( 3 \), \( 4 \), \( 5 \), and \( 6 \).  
- The curve between \( 1 \) and \( 2 \) is labeled "isentrope."  
- The curve between \( 2 \) and \( 3 \) is labeled "isobar."  
- The curve between \( 3 \) and \( 4 \) is labeled "isotherm."  
- The curve between \( 4 \) and \( 5 \) is labeled "isobar."  
- The curve between \( 5 \) and \( 6 \) is labeled "isentrope."  
- The curve between \( 6 \) and \( 1 \) is labeled "steiler als isobar" (steeper than isobar).  

---