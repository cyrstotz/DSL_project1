T-s diagram:  
A graph is drawn with labeled axes. The x-axis is labeled "s" and the y-axis is labeled "T [K]". The diagram includes curved lines representing isobars and numbered states (1, 2, 3, 4, 5, 6). State 1 starts at the bottom left, and the process progresses through states 2, 3, 4, 5, and 6. Another smaller graph is drawn to the right, showing a zoomed-in section with states 1 and 6. The smaller graph includes annotations such as "p0" and "p1" and a label "120 kW".

Ideal gas law:  
p times V equals R times T.  

p zero times V zero equals R times T zero.  
p six times V six equals R times T six.  

m divided by V zero equals p zero divided by R times T zero.  
m divided by V six equals p six divided by R times T six.  

p zero divided by R times T zero equals p six divided by R times T six.  
=> p zero divided by p six equals T zero divided by T six.  

p zero divided by R times T zero equals p six divided by R times T six.  
=> p six equals p zero times T six divided by T zero.  

Since m dot equals rho times A times w and M six equals M zero, A six equals A zero.  

w equals m dot divided by rho times A.  
=> w six equals m dot divided by rho six times A zero.  
=> w zero equals m dot divided by rho zero times A zero.  

=> w zero divided by w six equals p zero divided by p six.  

p zero equals p zero divided by R times T zero equals 0.15916 divided by 0.3844 kilojoules per kilogram Kelvin times 243.15 Kelvin equals 2.7332.  

p six equals p six divided by R times T six equals 0.15916 divided by 0.3844 kilojoules per kilogram Kelvin times 293.15 Kelvin equals 2.0217.  

No further content transcribed for the crossed-out section.