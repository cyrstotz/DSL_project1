A graph is drawn with pressure labeled as "P [bar]" on the vertical axis and temperature labeled as "T [K]" on the horizontal axis. The graph shows a curve with three distinct regions labeled:  
- "sol." (solid phase) on the left,  
- "flu." (fluid phase) in the middle,  
- "gas" (gas phase) on the right.  

Points are marked on the curve:  
- Point "1" is at the lower right,  
- Point "2" is in the middle of the fluid region,  
- Point "3" is at the peak of the curve in the gas region.  

An arrow connects point "2" to point "1" horizontally.  

---

Epsilon k equals the absolute value of Q zu divided by the absolute value of W plus.  

Epsilon L equals the absolute value of Q k divided by the absolute value of W t.