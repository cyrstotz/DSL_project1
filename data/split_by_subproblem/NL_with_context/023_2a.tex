k equals 1.4  
cp equals 1.006 kilojoules per kilogram Kelvin  

Table:  
| p | T | w |  
|---|---|---|  
| 0 | 0.191 bar | 243.15 Kelvin |   |  
| 1 |   |   |   |  
| 2 |   |   |   |  
| 3 |   |   |   |  
| 4 |   |   |   |  
| 5 | 0.5 bar | 431.9 Kelvin | 220 meters per second |  
| 6 |   |   |   |  

Diagrams:  
1. Left diagram:  
- Axes labeled as T (Kelvin) on the vertical axis and s (kilojoules per kilogram Kelvin) on the horizontal axis.  
- Curves are drawn, labeled with points 0, 1, 2, 3, 4, 5, and 6.  
- Isobars are marked, with arrows indicating transitions between states.  

2. Right diagram:  
- Axes labeled as T (Kelvin) on the vertical axis and s (kilojoules per kilogram Kelvin) on the horizontal axis.  
- Curves are drawn, labeled with points 0, 1, 2, 3, 4, 5, and 6.  
- Isobars are marked, with arrows indicating transitions between states.  

Additional labels:  
- "Luft" written at the top right of the table.

\( w_5 = 220 \, \text{m/s} \)

w subscript 6 equals 5.4 times w subscript 5  
T subscript 0 equals 39 a.k.  

c)  
Delta ex subscript str equals ex subscript str,6 minus ex subscript str,0  

Delta ex subscript str equals m dot multiplied by open bracket h subscript 6 minus h subscript 0 minus T subscript 0 multiplied by open parenthesis s subscript 6 minus s subscript 0 close parenthesis plus open parenthesis w subscript 6 minus w subscript 0 close parenthesis squared divided by 2 close bracket  

Delta ex subscript str equals c subscript p multiplied by open parenthesis T subscript 6 minus T subscript 0 close parenthesis minus T subscript 0 multiplied by natural log open parenthesis T subscript 6 divided by T subscript 0 close parenthesis plus open parenthesis w subscript 6 minus w subscript 0 close parenthesis squared  

equals 335.42 kJ divided by kg  

d)  
ex subscript verl equals question mark  

a equals ex subscript str minus W dot equals ex subscript verl