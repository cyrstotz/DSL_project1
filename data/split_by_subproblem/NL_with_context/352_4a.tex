A pressure-temperature diagram is drawn. The diagram includes labeled regions and points:  
- The vertical axis is labeled as "p" (pressure).  
- The horizontal axis is labeled as "T" (temperature).  
- A curve is drawn, representing phase transitions.  
- The "Triple" point is marked on the curve.  
- Three states are labeled:  
  - State 1: "Zustand zu Beginn" (State at the beginning).  
  - State 2: "Zustand nach (i), also nach isobarem Kühlen" (State after (i), i.e., after isobaric cooling).  
  - State 3: "Zustand nach (ii), also nach isothermem Druckablass" (State after (ii), i.e., after isothermal pressure release).  
- The region above the curve is labeled "Fest" (solid).  
- The region below the curve is labeled "Gas" (gas).  
- Horizontal arrows are drawn to indicate processes, labeled "Kühlen isobar" (cooling isobarically) and "Druckablass isotherm" (pressure release isothermally).