h_q = 93.92 kilojoules per kilogram (A-11)  
h_1 = h_q (Drossel ist Enthalpie)  
T_i = -10 degrees Celsius (aus Diagramm)  
T_wz = T_i minus 6 Kelvin = -16 degrees Celsius  
h_2 = 492.54 kilojoules per kilogram minus 294.15 kilojoules per kilogram = 237.71 kilojoules per kilogram  
s_2 = 6.2258 kilojoules per kilogram Kelvin minus s_3  
h_3 = h_3 = 26.41 kilojoules per kilogram plus (273.66 kilojoules per kilogram minus 269.15 kilojoules per kilogram)  
(s_2 = s_3 (s_4))  
s_2 minus s_3 = 279.3 joules per kilogram Kelvin  
(Note: "Auf nächster Seite weitergeführt" is written, indicating continuation on the next page.)

W_k equals m_dot multiplied by (h_3 minus h_2).  

m_dot equals W_k divided by (h_3 minus h_2).  
Equals 0.000839 kilograms per second.  
Equals 3.0035 kilograms per hour (underlined).