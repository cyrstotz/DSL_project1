Further calculations:
- \( h_3 = h_f(8 \, \text{bar}) + s_2 - s_f(8 \, \text{bar}) \times \frac{h_g(8 \, \text{bar}) - h_f(8 \, \text{bar})}{s_g(8 \, \text{bar}) - s_f(8 \, \text{bar})} \)
- \( h_3 = 93.42 \, \text{kJ/kg} + 0.9298 - 0.3459 \times \frac{269.15 \, \text{kJ/kg} - 93.42 \, \text{kJ/kg}}{0.9374 - 0.2666} \)
- \( h_3 = 269.15 \, \text{kJ/kg} \)

h subscript 4 equals h subscript 4.  
h subscript 4 equals AM at 8 bar, x equals 0.  
h subscript 4 equals h subscript f (8 bar) equals 93.42 kilojoules per kilogram.