p1 equals one bar plus thirty-two kilograms multiplied by acceleration due to gravity divided by pi multiplied by the radius squared.  
p1 equals one bar plus atmospheric pressure equals 1.399695 bar.  

M equals p multiplied by V divided by R multiplied by T1 equals 0.009949 kilograms.  

R equals R divided by M equals 0.76638 kilojoules per kilogram Kelvin.  

---

Delta E_EW equals Delta U_EW equals 1 times Q_12.  

U_EW minus U_new plus Q_12 equals 1500 divided by m_EW equals 75 kilojoules per kilogram.  

U_2 equals (x_2 times U_rest [0 degrees] plus 0.4 times U_fluessig [0 degrees]) divided by m_EW equals -200.0928 kilojoules per kilogram.  

U_2 equals (x_2 times U_rest [0 degrees] plus [0.009]) plus w times (1 minus x_2) times U_fluessig [0.009].  

U_rest [0 degrees] equals -333.438 kilojoules per kilogram.  
U_fluessig [0 degrees] equals -0.045 kilojoules per kilogram.  
U_rest [0.009] equals -333.442 kilojoules per kilogram.  
U_fluessig [0.009] equals -0.033 kilojoules per kilogram.  

Minus x_2 times 333.442 kilojoules per kilogram times (7 minus x_2) times 0.033 kilojoules per kilogram equals 75 kilojoules per kilogram minus 200.0928 kilojoules per kilogram equals -195.0928 kilojoules per kilogram.  

x_2 equals (185.0928 kilojoules per kilogram minus 0.033 kilojoules per kilogram) divided by (333.442 kilojoules per kilogram minus 0.033 kilojoules per kilogram) equals 0.555.