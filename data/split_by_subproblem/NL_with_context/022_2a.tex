First law of thermodynamics:  
0 equals m dot times (h sub 5 minus h sub 6) plus (w sub 5 squared minus w sub 6 squared) divided by 2 plus Q minus W e.  

Adiabatic reversible:  
0 equals m dot times (s sub 5 minus s sub 6) plus Q divided by T.  

Entropy balance:  
0 equals m dot times (s sub 5 minus s sub 6) plus Q divided by T.  

s sub 5 equals s sub 6.  

s sub 5 is determined:  
Interpolation in Table A-22.  

Table corrected for interpolation at temperature 273.15 K:  
h sub 5 equals interpolation (431.9 minus 430) divided by (431.9 minus 431.43) times (440 minus 430) plus 433.86 kJ/kg.

s subscript 6 minus s subscript 5 equals s superscript 0 subscript 6 minus s superscript 0 subscript 5 minus R natural logarithm of p subscript 6 divided by p subscript 5.  

s superscript 0 subscript 6 equals s superscript 0 subscript T subscript 6.  
s superscript 0 subscript 5 equals s superscript 0 subscript T subscript 5 plus R natural logarithm of p subscript 0 divided by p subscript 5.  

c subscript p equals k divided by k minus 1 times c subscript v.  
c subscript v equals 0.7186 kilojoules per kilogram Kelvin.  
R equals c subscript p minus c subscript v equals 0.2874 kilojoules per kilogram Kelvin.  

s superscript 0 subscript T subscript 5 equals s superscript 0 subscript T subscript 5 plus R natural logarithm of p subscript 0 divided by p subscript 5.  

Interpolation in A-22:  
T subscript 6 equals 340 minus 330 divided by (1.73321 minus 1.78249) times (1.73321 minus 1.78249) plus 330 equals 336.89 degrees Celsius.  

s superscript 0 subscript 5 equals 2.0887 minus 2.0653 divided by 440 minus 430 times (431.9 minus 430) plus 2.0653 equals 2.08324 kilojoules per kilogram Kelvin.  

w subscript t equals negative integral from 5 to 6 of p dv.  
w subscript t equals negative integral from 5 to 6 of p dv equals negative integral from p subscript min to p subscript max of p divided by v dv.