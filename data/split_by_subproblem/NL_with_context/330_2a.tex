A graph is drawn with the vertical axis labeled `[K] T` and the horizontal axis labeled `[R2 by k] S`. The graph includes several curves labeled `p2 = p3`, `p5 = p4`, and `p0`. Points are marked as `0`, `2`, `3`, `4`, `4.5`, `5`, and `6`, connected by orange lines indicating a process. The label `isobars` is written near the curves.  

Below the graph, the following text and equations are written:  

"Luft als ideales Gas → kein Wandauf Effekt"  

"5 → 6 isentrope Zustandsaenderung"  

\( w_e = w squared divided by 2 \)  
\( i divided by m = g A w \)  

\( k = 1.4 \)  
\( T_6 divided by T_5 = (p_5 divided by p_6) raised to (k minus 1 divided by k) \)  

\( p_0 = p_0 = 0.191 \, \text{bar} \)  
\( T_5 = 431.9 \, \text{K} \)  
\( p_5 = 0.5 \, \text{bar} \)  

\( T_6 = T_5 \cdot (p_5 divided by p_6) raised to (k minus 1 divided by k) = 568.58 \, \text{K} \)  

The final result is underlined: \( T_6 = 568.58 \, \text{K} \).

A table is drawn with multiple columns and rows. The headers of the columns are labeled as follows:  

- **T**  
- **W**  
- **Q**  
- **Z**  

The rows contain the following data:  

Row 1:  
- Column T: "0"  
- Column W: "w = 200 meters per second"  
- Column Q: "adiabatic"  
- Column Z: "ambient"  

Row 2:  
- Column T: "1"  
- Column W: Empty  
- Column Q: Empty  
- Column Z: Empty  

Row 3:  
- Column T: "2"  
- Column W: Empty  
- Column Q: Empty  
- Column Z: Empty  

Row 4:  
- Column T: "3"  
- Column W: Empty  
- Column Q: "q equals 1195 kilojoules per kilogram"  
- Column Z: "T equals 1289 Kelvin"  

Row 5:  
- Column T: "4"  
- Column W: Empty  
- Column Q: "adiabatic"  
- Column Z: "adiabatic"  

Row 6:  
- Column T: "5"  
- Column W: "w equals 220 meters per second"  
- Column Q: Empty  
- Column Z: "T equals 431.9 Kelvin"  

Row 7:  
- Column T: "6"  
- Column W: Empty  
- Column Q: Empty  
- Column Z: Empty  

No additional content is visible.