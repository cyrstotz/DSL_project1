Cp equals 1.006 kilojoules per kilogram Kelvin.  
n equals k equals 1.4.  
PE equals 0.  

| State | p (bar) | T (°C) | s |  
|-------|---------|--------|---|  
| 0     | 0.191   | -30°   |   |  
| 1     | p1 greater than p0, T1 greater than T0 |   | s1 equals s2 |  
| 2     |         |        |   |  
| 3     | p3 equals p2 |   | s3 less than s4 |  
| 4     |         |        |   |  
| 5     | p4 equals p5 equals 0.5 | 431.9 K |   |  
| 6     |         |        | s5 equals s6 |  

a)  
A T-s diagram is drawn with the following features:  
- The y-axis is labeled as T (°C).  
- The x-axis is labeled as s (kilojoules per kilogram Kelvin).  
- Points 0, 1, 2, 3, 4, 5, and 6 are marked along the curve.  
- The curve includes labeled segments:  
  - Isobar between points 3 and 5.  
  - Isentrop between points 2 and 3, and between points 5 and 6.  
  - Isotherm at the bottom of the diagram.  
- Pressure at point 5 is labeled as 0.5 bar.