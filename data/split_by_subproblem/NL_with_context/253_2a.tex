A temperature-entropy (T-s) diagram is drawn with the temperature labeled as T [K] on the vertical axis and entropy labeled as s [kJ/kg·K] on the horizontal axis. The diagram includes six states labeled 0, 1, 2, 3, 4, 5, and 6.  
- State 3 is marked at the top of the curve, with the note \( p_1 = p_3 \).  
- State 5 is marked lower on the curve, with the note \( p_4 = p_5 \).  
- State 0 is at the bottom left corner, and state 6 is at the bottom right corner.  
Arrows indicate the transitions between states, showing the process flow.

Zero equals h subscript 5 minus h subscript 6 plus w subscript 5 minus w subscript 6.  
Equals c subscript p, Luft multiplied by (T subscript 5 minus T subscript 6) plus w subscript 5 minus w subscript 6.  

Reversible, adiabatic subroutine:  
T subscript 6 divided by T subscript 5 equals (p subscript 6 divided by p subscript 5) raised to the power of (1 minus 1 divided by kappa).  
Therefore, T subscript 6 equals (p subscript 0 divided by p subscript 5) raised to the power of (1 minus 1 divided by kappa) multiplied by T subscript 5 equals 328.07 Kelvin.  

h subscript 0 minus h subscript 6 equals negative 85.43 kilojoules per kilogram.  

w subscript 6 equals 510 meters per second.