Next page.

\( T_{g,1} \), \( p_{g,1} \), \( x_{\text{Eis},1} = 0.6 \), \( T_{\text{EW},1} = 0^\circ \text{C} \)  
\( T_{g,2} = T_{\text{EW},2} \)  

Energy balance from 1 → 2:  
\( m_g u_g + m_{\text{EW}} u_{\text{EW}} = m_g u_g + m_{\text{EW}} u_{\text{EW}} \)  

\( \Rightarrow m_g c_V (T_{g,1}) + m_{\text{EW}} c_u = m_g c_V (T_{g,2}) + \)  

The temperature \( T_{g,2} = T_{\text{EW},2} = 0^\circ \text{C} \) since \( x > 0 \).  
If the temperature were above \( 0^\circ \text{C} \), \( x_{\text{Eis}} = 0 \).  

\( \Rightarrow p_g V_g = m R T_g \)  
\( p_{g,2} = \frac{m R T_{g,2}}{V_{g,2}} \)  

\( = 3.6 \cdot 10^{-3} \cdot \frac{8.314}{50 \cdot 10^{-3}} \cdot 273.15 \)  

The ice and water are incompressible, so:  
\( p_{g,2} = p_{g,1} = 1.5 \text{bar} \)