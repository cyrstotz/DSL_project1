A diagram is drawn showing a rectangular chamber with arrows labeled \( p_{\text{EW}} \cdot A \) and \( m_{\text{g}} \cdot g \) pointing downward, and \( p_{\text{g,1}} \cdot A \) pointing upward.  

The equation:  
\( p_{\text{g,1}} = p_{\text{EW}} + \frac{m_{\text{g}} \cdot g}{A} \)  
\( = 1.4 \, \text{bar} \)  

Another diagram is drawn with similar labels, showing \( p_{\text{EW}} \cdot A \) downward and \( m_{\text{g}} \cdot g \) downward, with \( p_{\text{EW}} = p_0 + \frac{m_{\text{g}} \cdot g}{A} = \text{also } 1.4 \, \text{bar} \).  

An additional note: \( \frac{D^2}{4} \cdot \frac{1}{\pi} \).  

The equation:  
\( m_{\text{gas}} = \frac{p_{\text{g,1}} \cdot V_{\text{g,1}}}{R / M \cdot T_{\text{g,1}}} = 3.4224 \, \text{g} \).  

---

Absolute value of Q one-two equals fifteen thousand joules.  

Energy balance for the closed system (ice-water):  
Delta E equals m EW times (U two minus U one) equals Q one-two.  

U one equals U flüssig (zero degrees Celsius) plus x Eis times (U fest (zero degrees Celsius) minus U flüssig (zero degrees Celsius)).  

Equals negative zero point zero four five kilojoules per kilogram plus negative three three three point four five eight kilojoules per kilogram.  

Equals negative two hundred point zero nine two eight.  

U two equals U flüssig (ten degrees Celsius) plus x Eis two times (U fest minus U flüssig).  

Therefore:  
Q one-two divided by m EW plus U one equals U flüssig plus x two times (U f minus U flüssig).  

Q one-two divided by m EW plus U one minus U flüssig divided by U f minus U flüssig equals x two.  

x two equals zero point five five five.