The equation for kinetic energy is crossed out but partially visible:  
ΔKE = Δm multiplied by (h_1 plus w_1 squared divided by 2).  

Another equation for work is written:  
W = square root of (h_6 minus h_5).  

An integral expression for work is written:  
KE = integral from 5 to 6 of p dV.  

An equation for velocity squared divided by 2 is written:  
w squared divided by 2 equals 1 divided by (T_2 minus T_1).  

Another equation is written:  
w squared divided by 2 equals c_p multiplied by ΔT.  
The result is calculated as:  
92.95 kJ/kg.  

The final velocity squared is calculated:  
w squared equals 430 m/s.

Delta ex_str equals ex_str,6 minus ex_str,0.  
ex_str equals h minus h_0 minus T_0 times (s minus s_0) plus ke.  
Delta ex_str,0 equals h_6 minus h_0 minus T_0 times (s_6 minus s_0) plus ke_6 minus ke_0.  

Equals Cp times Delta T_6,0 minus T_0 times Cp times ln (T_6 divided by T_0) minus T_0 times R times ln (p_6 divided by p_0).  

p_6 equals p_0.  

Equals Cp times (T_6 minus T_0) minus T_0 times Cp times ln (T_6 divided by T_0).  

Equals 1.006 kilojoules per kilogram Kelvin times (310 Kelvin minus (273.15 plus 30) Kelvin minus (273.15 plus 30) Kelvin times ln (310 divided by (273.15 plus 30))).  

Equals 67.126 kilojoules per kilogram.  

Equals Cp times (Delta T_6,0 minus T_0 times ln (T_6 divided by T_0)) plus ke.  

Equals 1.006 kilojoules per kilogram Kelvin times (310 Kelvin minus (273.15 plus 30) Kelvin minus (273.15 plus 30) Kelvin times ln (310 divided by (273.15 plus 30))) plus (w_10 squared divided by 2).  

Equals 67.126 kilojoules per kilogram plus (10 meters per second squared divided by 2).  

Equals 145.47 kilojoules per kilogram.  

---