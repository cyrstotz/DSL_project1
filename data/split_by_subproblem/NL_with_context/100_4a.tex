A table is drawn with columns labeled as follows:  
- \( p \) (bar)  
- \( T \) (K)  
- \( Q \)  
- \( W \)  
- \( x \)  

The rows are numbered 1 to 4. The following values are filled in:  

Row 1:  
- \( p = 1210 \), \( p_1 = p_2 \)  
- \( T_i = -6 / (-20^\circ C) = 277.15K \)  

Row 2:  
- \( p = p_2 \)  

Row 3:  
- \( p = 8 \)  

Row 4:  
- \( p = p_4 \), \( p = 8 \)  

Additional notes:  
- \( s_2 = s_3 \)  
- \( s_1 = s_5 \)  
- \( s_5 = s_4 \)  
- \( h_2 = h_4 \), \( h_3 = h_2 \), \( h_3 = 93.42 \, \text{kJ/kg} \)  

---

A graph is drawn on a grid with the y-axis labeled as "p" (pressure) and the x-axis labeled as "T" (temperature).  

The graph depicts a curve resembling a dome shape, with four distinct points labeled:  
- Point 1 is at the lower left of the curve.  
- Point 2 is on the lower right side of the curve.  
- Point 3 is above Point 2, outside the dome, with an arrow pointing upwards and labeled "8 bar."  
- Point 4 is on the upper left side of the curve.  

The peak of the dome is labeled "Tkrit" (critical temperature).  

Lines connect the points, and the curve represents phase regions.