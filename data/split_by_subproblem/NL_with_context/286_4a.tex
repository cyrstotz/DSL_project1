Three diagrams are drawn, each labeled with axes and points:  

1. The first diagram has pressure (p) on the y-axis and temperature (T) on the x-axis. It shows a curve labeled "gas" and "flüssig" (liquid). Points 1, 2, 3, and 4 are marked, with arrows indicating transitions between states. A region is shaded between points 1 and 2, and a vertical line is drawn from point 4.  

2. The second diagram also has pressure (p) on the y-axis and temperature (T) on the x-axis. It shows a rectangular cycle with points 1, 2, 3, and 4 connected by straight lines. The regions are labeled "gas" and "flüssig" (liquid).  

3. The third diagram has pressure (p) on the y-axis and temperature (T) on the x-axis. It shows a straight line connecting points 1, 2, 3, and 4, with the regions labeled "flüssig" (liquid) and "gasförmig" (gaseous).