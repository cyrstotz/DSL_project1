\( \Delta m_{12} \):  
\( (m_1 + \Delta m_{12}) U_2 - m_1 U_1 = \Delta m_{12} h_f(20^\circ C) - Q_{R,12} \)  

\( U_2 = (57.55) + 0.005 (2,465.6 - 252.55) = 303.83 \, \text{kJ/kg} \)  
\( U_1 = (57.55) + 0.005 (2,550.5 - 419.04) = 493.55 \, \text{kJ/kg} \) (TA-2)  

\( h_f(20^\circ C) = 83.56 \, \text{kJ/kg} \)  

\( m_1 U_2 + \Delta m_{12} h_f - m_1 U_1 = \Delta m_{12} h_f - Q_{R,12} \)  
\( m_1 (U_2 - h_f) - Q_{R,12} = m_1 U_1 \)  

\( \Delta m_{12} = \frac{m_1 (U_1 - Q_{R,12}) - m_1 U_2}{U_2 - h_f} \)  

\( \Delta m_{12} = \frac{57.55 \, \text{kg} (493.55 - 303.83) \, \text{kJ/kg} - 35,000 \, \text{kJ}}{(303.83 - 83.56) \, \text{kJ/kg}} \)  

\( \Delta m_{12} = 317 \, \text{kg} \)

Delta S twelve equals Delta M twelve multiplied by S w (twenty degrees Celsius) minus Q out twelve divided by T KF.  

S w (twenty degrees Celsius) equals zero point two five six kilojoules per kilogram Kelvin (from A-2).  

Delta S twelve equals three thousand one hundred twenty-seven kilograms multiplied by zero point two five six kilojoules per kilogram Kelvin minus thirty-five thousand kilojoules divided by two hundred ninety-three point fifteen Kelvin equals eight hundred eight point one kilojoules per Kelvin.