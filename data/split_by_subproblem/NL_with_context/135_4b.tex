The transition from state 2 to state 3 is labeled as "isentropic."  
It is noted that entropy at state 2 (s2) equals entropy at state 3 (s3).

A table is drawn with the following columns:  
- Zustand (State)  
- p (Pressure in bar)  
- T (Temperature)  
- x (Vapor quality)  
- s (Entropy)  
- h (Enthalpy)  

The rows are filled as follows:  
- State 1: Pressure = blank, Temperature = blank, x = blank, s = blank, h = blank.  
- State 2: Pressure = 1 bar, Temperature = blank, x = 1, s = s2 = s3, h = blank.  
- State 3: Pressure = 8 bar, Temperature = blank, x = blank, s = blank, h = blank.  
- State 4: Pressure = 8 bar, Temperature = blank, x = 0, s = blank, h = h4.  

It is noted:  
"x3 = blank."

A calculation is written:  
h1 = hf (at 1 bar) = 93.42 kilojoules per kilogram = h1.