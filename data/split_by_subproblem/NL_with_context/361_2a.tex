A graph is drawn with the y-axis labeled as "T" (temperature) and the x-axis labeled as "s" (entropy). The graph depicts a thermodynamic process with labeled points: 0, 1, 2, 3, 5, and 6. The curve includes isobars and arrows indicating the direction of the process.  

Below the graph, the following values and notes are written:  
\( T_0 = -30^\circ \text{C} \)  
\( p_0 = 0.191 \, \text{bar} \)  
\( p_1 > p_0 \)  
\( p_1 > p_0 \)  

---

Q equals m times (u2 minus u1)  

u1 equals u at 0 degrees Celsius  

u1 equals u at 0 degrees Celsius plus x times (uF minus uEis) equals minus 133.91 kilojoules per kilogram  

Q equals 7.5 kilojoules (as given)  

u2 equals Q divided by m plus u1 equals 7.5 divided by 0.1 plus minus 133.91 equals minus 118.41 kilojoules per kilogram  

u2 equals uEis plus x times (uF minus uEis)  

x equals u2 minus uEis divided by uF minus uEis  

x equals minus 118.41 plus 333.442 divided by minus 9.093 plus 333.442 equals 0.645  

0 degrees Celsius