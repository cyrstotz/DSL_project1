T5 equals 431.91 K, p5 equals 0.5 bar, and w5 equals 220 meters per second.  

The process from state 5 to state 6 is described as "adiabat reversible".  
The polytropic exponent is stated as kappa equals 1.4.  
p0 equals p6.  

The temperature T6 is calculated using the polytropic relation:  
T6 equals (p6 divided by p5) raised to the power of (kappa minus 1 divided by kappa), multiplied by T5.  
T6 equals 328.11 K.  

The energy balance is described:  
W equals 0, Q equals 0, and the process is adiabatic.  

The stationary energy equation is written:  
O equals inlet enthalpy minus outlet enthalpy plus (w e squared minus w a squared) divided by 2.  

Simplified:  
O equals h e minus h a plus w e squared divided by 2 minus w a squared divided by 2.  

Further simplified:  
w e squared divided by 2 equals h e minus h a plus w e squared divided by 2.  

The final velocity w6 is calculated:  
w6 equals the square root of (2 times (h e minus h a) plus w e squared).  
w6 equals 507.2 meters per second.  

Additional notes:  
- cp is constant.  
- cp equals 1.006 kJ/kg·K.  
- cp times (T5 minus T6) equals 1044.32 kJ/kg.