

\subsection*{a)}

\begin{description}
    \item[Graph:] The graph is a pressure-temperature ($p$-$T$) diagram. The $y$-axis represents pressure ($p$) and the $x$-axis represents temperature ($T$). The graph is divided into three regions labeled "Fest" (solid), "Flüssig" (liquid), and "Gas" (gas). There is a curve starting from the origin and rising upwards, separating the "Gas" region from the "Fest" region. This curve is labeled "Tripel" at a point where it intersects with another horizontal line. The horizontal line extends from the left to the right, marking the boundary between the "Flüssig" and "Gas" regions. The intersection point of the horizontal line and the curve is labeled "1". The right end of the horizontal line is labeled "O". The region between the curve and the horizontal line is shaded. Another vertical line extends upwards from the point labeled "1" and intersects the horizontal line at the point labeled "O".
\end{description}