

\subsection*{a)}

\begin{align*}
T_g &= 500^\circ C \\
V_{g1} &= 3.1\,L \\
\frac{96.2\,m}{s^2} &= 32.981\,\frac{kg\,m}{s^2} = 343.52\,N
\end{align*}

\textbf{Diagramm:} Eine schematische Zeichnung eines Zylinders mit zwei horizontalen Linien, die den Zylinder in drei Abschnitte unterteilen. Der obere Abschnitt ist mit $526$ beschriftet. Ein Pfeil zeigt von oben nach unten durch den Zylinder.

\begin{align*}
p_{g1} &= ? \\
m_g &= ? \\
A &= \pi r^2, \quad r = \frac{D}{2} = 0.65\,m \\
A &= 7.853 \cdot 10^{-3}\,m^2 \\
p \text{ durch Gewicht} &= \frac{343.52\,N}{7.853 \cdot 10^{-3}\,m^2} = 0.04\,bar
\end{align*}

\text{Gleichgewichtszustand (GGW), in der ersten Kammer:} $1.4\,bar$, \\
\text{so muss in der Gaskammer auch} $1.4\,bar$ \text{sein}

\begin{align*}
p_{g1} &= 1.4\,bar \\
m_g &= \frac{p_g V_g}{R_g T_g} \\
m_g &= \frac{1.4 \cdot 10^5\,\frac{N}{m^2} \cdot 3.1 \cdot 10^{-3}\,m^3}{8.314\,\frac{J}{mol\,K} \cdot 773.15\,K} \\
m_g &= 3.045 \cdot 10^{-3}\,kg = 3.045\,g
\end{align*}