

\subsection*{c)}

\begin{itemize}
    \item \textbf{Graph 5:} A graph with axes labeled $P$ (vertical axis) and $T$ (horizontal axis). The graph shows a curve starting from a point labeled 1, rising steeply, then curving downwards and leveling off towards the right.
    \item \textbf{Graph 6:} A graph with axes labeled $P$ (vertical axis) and $T$ (horizontal axis). The graph shows a complex curve with points labeled 1, 2, and 3. The curve starts at point 1, moves to point 2, then to point 3, and finally loops back towards point 2.
\end{itemize}

\[
\dot{m} (h_1 - h_2)
\]

\[
\dot{m} (h_2 - h_3) + \dot{Q} - \dot{W}_n > 0
\]

\[
h_2 (x = 1) \Rightarrow h_g
\]

\[
h_3 (8 \text{ bar}) = h_3 (8 \text{ bar}, x = 2)
\]

\[
T_i = 16 h \text{ über Sublinie} \Rightarrow T: 4.5 \text{ bar und Tripel} = 0^\circ C
\]

\[
\dot{m} (h_2 - h_3) + \dot{Q} - \dot{W}_n > 0
\]

\[
\dot{m} (T = 0^\circ C) =
\]

``````latex


\section*{Student Solution}



\subsection*{c)}

\begin{align*}
x_n T_2 &\rightarrow \text{adiabate Drossel} \rightarrow \text{isotrop 20 bar} \\
s_n &= s_u \\
x_n &= \frac{s_u - s_{1t}}{s_{g1} - s_{1t}}
\end{align*}

\noindent
su (8 bar, x_u = 0) = s_{1t} \quad s_{k} (8 bar) = 0.345 \, s_g \\
\Delta T = 31.33^\circ C