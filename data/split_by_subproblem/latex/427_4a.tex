

\item[a)] 
    \begin{itemize}
        \item[1.] Zustand 1:
        \item[2.] Zustand 2:
        \item[3.] Zustand 3: \( p_3 = 8 \text{bar} \)
        \item[4.] Zustand 4: \( p_4 = 8 \text{bar} \)
    \end{itemize}
    
    \( T = 6 \text{K} \)
    
    \( x_1 = 1 \)
    
    \( x_4 = 0 \)
    
    \textbf{Graph Description:}
    
    The graph is a Pressure-Temperature (\( P \)-\( T \)) diagram. The horizontal axis represents the temperature \( T \) in Kelvin (K), and the vertical axis represents the pressure \( P \) in bar. The temperature axis is labeled \( T [K] \) and the pressure axis is labeled \( P [\text{bar}] \). The graph includes the following points and lines:
    
    - A horizontal line at \( P = 8 \text{bar} \) from \( T = 6 \text{K} \) to an unspecified higher temperature.
    - A vertical line from the point where the horizontal line ends, going up to a higher pressure.
    - A diagonal line connecting the end of the vertical line to a lower pressure and higher temperature.
    - Another horizontal line at the lower pressure, going back to the initial temperature \( T = 6 \text{K} \).
    
    The points are labeled as follows:
    
    - Point 1 at the intersection of \( P = 8 \text{bar} \) and \( T = 6 \text{K} \).
    - Point 2 at the end of the horizontal line at higher temperature.
    - Point 3 at the end of the vertical line at higher pressure.
    - Point 4 at the end of the diagonal line at lower pressure and higher temperature.