

\subsection*{b)}

\begin{itemize}
    \item A graph with the vertical axis labeled \( P \) and the unit \([ \text{mbar} ]\).
    \item The horizontal axis is labeled \( T \) with the unit \([ \text{K} ]\).
    \item The graph shows a curve that starts at the origin, rises to a peak, and then falls off to the right.
    \item There are three points labeled 1, 2, and 3.
    \item Point 1 is on the curve at the peak.
    \item Point 2 is to the left of the peak on the curve.
    \item Point 3 is below point 2 on the vertical line.
    \item A horizontal dashed line extends from point 1 to the right, labeled \( 10 \text{mbar} \).
    \item Another horizontal dashed line extends from point 2 to the right, labeled \( 5 \text{mbar} \).
    \item The term "isobare Abkühlung" is written near the top of the graph.
    \item The term "isotherm" is written near the bottom left of the graph.
\end{itemize}

``````latex


\section*{Aufgabe 4}

\subsection*{b) Energiebilanz um Kompressor}

\[
\dot{W}_n = \dot{m} (h_2 - h_3)
\]

```