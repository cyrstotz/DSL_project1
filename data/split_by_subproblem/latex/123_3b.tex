

\section*{b)}

\text{Ges: } T_{g2} + p_{g2}

Der Vorgang ist adiabatisch, daher ist der Vorgang polytrop $pV^n = \text{const.}$

Aus der idealen Gasgleichung $pV = mRT$ folgt, wenn $pV^n = \text{const.}$, muss das ganze auch isotherm sein!

\[
T_{g2} = T_{g1} = 500^\circ C
\]

\[
\left( pV^n = mRT \right)
\]

\[
p_2 = \frac{mRT}{V^n} = \frac{3,1422 \cdot 10^{-3} \cdot 116,28}{(3,14 \cdot 10^{-3})^n}
\]