

\subsection*{b)}

\[
T_{1,2} = T_{\text{Eis}} \rightarrow \text{daher keine neue These.}
\]

\[
T_{Eis} = T_{\text{Eis}}, \, \text{weil kein Eis}
\]

\[
T_{1,2} = 0^\circ \text{C} = 273.15 \, \text{K}
\]

Die Temperatur ist 0 Grad, weil im oberen Behälter immer noch Eis vorhanden ist, das heißt, alles Wasser über das Eis abgesperrt hat, hat zum Eis gesprochen und nicht erwärmt.

\[
\rho_2 = \rho_1 = \frac{1.4 \, \text{g/cm}^3}{1.4 \, \text{g/cm}^3} = 1.4 \, \text{g/cm}^3
\]

Der Druck ändert sich nicht, da das Kräftegleichgewicht immer noch gelten muss.

\subsection*{b) Berechnungen}

\[
\dot{E}_{\text{W}} = \text{kinetische Energie}:
\]

\[
\dot{E}_{\text{W}} = \frac{\dot{m}_1}{A} + \dot{m}_2 = \frac{7.9 \cdot 0.9}{0.6 \cdot 0.7 \cdot 1.1} + \dot{Q} = q_1 + q_2
\]

\[
u_2 = u_{\text{W}} + q_1 + q_2 \left( \frac{u_1}{u_2 - u_1} \right)
\]

\[
\dot{m}_1 = 7.9 \cdot 0.9 = 7.11
\]

\[
u_2 = \frac{Q}{m \cdot u_1}
\]

\[
u_2 = \frac{7.9 \cdot 287 \cdot 10^{-3}}{0.14} + \left( -200.6 \cdot 614 \cdot \frac{kJ}{kg} \right) = -192.300 \frac{kJ}{kg}
\]

\[
\beta = \frac{7.9 \cdot \dot{m}_1}{u_1 \cdot \dot{m}_2} \cdot \left( \frac{u_1}{u_2 - u_1} \right)
\]

\[
X_{\text{W}} = \frac{u_2 - u_{\text{W}}}{u_1 - u_{\text{W}}} = \frac{-192.300 - (-3.33 \cdot 9.8)}{-0.085 - (-7.33 \cdot 5.55)} = 0.42333
\]

\[
X_{\text{Exp}} = 1 - X_{\text{W}} = 0.57663
\]

``````latex