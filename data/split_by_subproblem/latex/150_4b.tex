

\section*{4b)}

\begin{align*}
& s_2 = s_3 \\
& T_i = -10^\circ C \\
& T_{\text{ver}} = -16^\circ C \quad / = T_2 \\
& h_2 = 237.74 \frac{\text{kJ}}{\text{kg}} \\
& s_2 = s_3 = 0.8289 \frac{\text{kJ}}{\text{kg K}} \\
& h_3 = \frac{(273.66 - 264.75) \frac{\text{kJ}}{\text{kg}}}{(0.8374 - 0.8066) \frac{\text{kJ}}{\text{kg K}}} (0.8289 - 0.8066) \frac{\text{kJ}}{\text{kg K}} + 264.75 \frac{\text{kJ}}{\text{kg}} \\
& = 271.31 \frac{\text{kJ}}{\text{kg}} \\
& 0 = \dot{m}_{R12} (h_2 - h_3) + \dot{W}_H \implies \dot{m}_{R12} = \frac{-\dot{W}_H}{h_2 - h_3} = 3.00 \frac{\text{kg}}{\text{h}}
\end{align*}

``````latex



\item[(b)] Die Temperatur würde abnehmen, da durch den abgeführten Wärmestrom an Energie dem System entzogen wird. (Bei konstantem Volumen)
\end{itemize}

\begin{itemize}