

\subsection*{a)}
Energieblianz im Reaktor, stationär mit $\Delta p = \Delta E = W = 0$

\[
0 = \dot{m}_{ein} (h_{ein} - h_{aus}) + \dot{Q}_{Reaktor} - \dot{Q}_{aus}
\]

(mit $\dot{Q}$ aus in Pfeilrichtung positiv definiert)

\[
\dot{Q}_{aus} = \dot{m}_{ein} (h_{r200^\circ C} - h_{w200^\circ C}) + Q_2
\]

Im ganzen Reaktor siedende Flüssigkeit $\Rightarrow$ wir sind im 2-Phasengebiet

\[
h_{ein} = h_{f, Tab = 2} \quad \text{mit} \quad T = 70^\circ C \quad \text{und} \quad x = 0.005
\]

\[
h_{f70} = 293.2 \frac{kJ}{kg} \quad h_{g70} = 2626.8 \frac{kJ}{kg} \quad (bz. 20^\circ)
\]

\[
h_{ein} = h_{f70} + x_p (h_{g70} - h_{f70}) = 309.649 \frac{kJ}{kg} = h_{ein}
\]

\[
h_{aus} = h_{f200} + x_p (h_{g200} - h_{f200}) = 428.23 \frac{kJ}{kg}
\]

\[
Tab A-2: \quad h_{f200} = 849.04 \frac{kJ}{kg} \quad h_{g200} = 2257.0 \frac{kJ}{kg}
\]

\[
\dot{Q}_{aus} = 0.5 \frac{kg}{s} (309.649 + 428.23) \frac{kJ}{kg} + 100 \frac{kJ}{s} = 62.93 \frac{kJ}{s} = \dot{Q}_{aus}
\]