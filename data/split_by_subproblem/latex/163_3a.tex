

\subsection*{a)}

\[
P_{g1} = 
\]

\begin{center}
\begin{tabular}{c}
\begin{picture}(100,100)
\put(50,80){\vector(0,-1){40}}
\put(50,40){\vector(0,1){40}}
\put(50,60){\makebox(0,0){$F_{g1}$}}
\put(50,20){\makebox(0,0){$F_{g2}$}}
\put(50,0){\line(1,0){100}}
\put(50,100){\line(1,0){100}}
\put(50,0){\line(0,1){100}}
\put(150,0){\line(0,1){100}}
\end{picture}
\end{tabular}
\end{center}

\[
P_{g1} = P_{amb} + \frac{m \cdot g}{A_K} + \frac{m_{EW} \cdot g}{A_m}
\]

\[
A_K = A_m = \frac{D^2}{4} \cdot \pi = \left(0,1 \, \text{m}\right)^2 \cdot \pi = 0,00785 \, \text{m}^2
\]

\[
= 1 \, \text{bar} + \frac{32 \, \text{kg} \cdot 9,81 \, \frac{\text{m}}{\text{s}^2}}{0,00785 \, \text{m}^2} \cdot 10^{-5} \, \frac{\text{bar}}{\text{Pa}} + \frac{0,1 \, \text{kg} \cdot 9,81 \, \frac{\text{m}}{\text{s}^2}}{0,00785 \, \text{m}^2} \cdot 10^{-5} \, \frac{\text{bar}}{\text{Pa}}
\]

\[
= 1,1401 \, \text{bar}
\]

\subsection*{1d Gas Gesetz}

\[
pV = mRT \quad \Rightarrow \quad m = \frac{pV_1}{RT_1}
\]

\[
m = \frac{1,1401 \, \text{bar} \cdot 10^5 \, \frac{\text{Pa}}{\text{bar}} \cdot 0,34 \, \text{m}^3 \cdot 10^{-3}}{8,314 \, \frac{\text{J}}{\text{mol} \cdot \text{K}} \cdot 273,15 \, \text{K} + 500 \, \text{K}}
\]

\[
m = 0,0034 \, \text{kg} \approx 3,4 \, \text{g}
\]

\[
R = \frac{R}{M_{CH}} = \frac{8,314 \, \frac{\text{J}}{\text{mol} \cdot \text{K}}}{50 \, \frac{\text{kg}}{\text{mol}}} = 166,3 \, \frac{\text{J}}{\text{kg} \cdot \text{K}}
\]

Der Druck bleibt wie in Zustand 1 da sich das System, wenn es sich ausdehnt nach oben gegen den geringeren Druck ausdehnt zudem bleibt die Masse die selbe.

\[
P_2 = 1,1401 \, \text{bar}
\]

Da das System in einem thermodynamischen Gleichgewicht steht

\[
T_2 = T_E (P_2) = 0^\circ \text{C} = T_{ZG1} = T_{ZEW}
\]

\[
\text{bei } T_E (P_2) \text{ aus Tabelle 1}
\]

``````latex