

\subsection*{c)}

Drossel ist isenthalp und isotherm:

\[
T_3 = T_4
\]

\[
T_4 (8 \text{ bar}, x = 0) = 31.33^\circ \text{C} \quad T_{cb} = A - 11
\]

1. HS stationär um Drossel:

\[
0 = \dot{m} (h_4 - h_3)
\]

\[
h_4 = h_3
\]

\[
T_3 \text{ A + 11} \quad h_4 = 33.42 \, h_3 = h_f + x h_{fg}
\]

Zustand 1: \quad x = ?

\[
x = \frac{(31.33^\circ \text{C}, L_{hf} = 37.42)}{h_f}
\]

\[
T_3 = A + 11 \quad 30^\circ \text{C} = h_f = 94.49
\]

\[
h_f = 263.5 \quad h_g = 263.5
\]

\[
h_g = 263.5 \quad h_f = 94.49
\]

\[
32^\circ \text{C} = h_f = 94.39
\]

\[
h_f = 263.48
\]

``````latex


\section*{Student Solution}

\subsection*{Part c)}

\[
h_f(31.33) = 91.45 + x
\]