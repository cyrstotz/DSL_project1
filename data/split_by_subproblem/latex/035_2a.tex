\subsection*{a) T-s Diagramm}

\begin{itemize}
    \item Graph description:
    \begin{itemize}
        \item The graph is a T-s diagram with temperature \( T \) on the vertical axis and entropy \( s \) on the horizontal axis.
        \item The vertical axis is labeled \( T[K] \).
        \item The horizontal axis is labeled \( s \left[ \frac{kJ}{kg \cdot K} \right] \).
        \item There are several points labeled on the graph: \( O \), \( 1 \), \( 2 \), \( 3 \), \( 4 \), \( 5 \), and \( 6 \).
        \item The path from \( O \) to \( 1 \) is a curved line with a downward slope.
        \item The path from \( 1 \) to \( 2 \) is a vertical line labeled \( s = \text{const} \).
        \item The path from \( 2 \) to \( 3 \) is a line with an upward slope.
        \item The path from \( 3 \) to \( 4 \) is a line with a steeper upward slope.
        \item The path from \( 4 \) to \( 5 \) is a horizontal line labeled \( p = p_4 = p_5 \).
        \item The path from \( 5 \) to \( 6 \) is a vertical line labeled \( s = \text{const} \).
    \end{itemize}
\end{itemize}

\begin{itemize}
    \item \( O \rightarrow 1 \): \( s \downarrow, T \downarrow \)
    \item \( 1 \rightarrow 2 \): isentrope \( s_1 = s_2 \)
    \item \( 2 \rightarrow 3 \): isobar \( T \uparrow \)
    \item \( 3 \rightarrow 4 \): \( s \uparrow \)
    \item \( 4 \rightarrow 5 \): \( p_4 = p_5 \)
    \item \( 5 \rightarrow 6 \): \( s_5 = s_6 \)
\end{itemize}

``````latex


\section*{Aufgabe 2}