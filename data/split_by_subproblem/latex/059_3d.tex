

\subsection*{d)}
\begin{align*}
    u_2 &= u_f + x (u_{est} - u_f) \\
    \Delta U &= \Delta Q \\
    u_2 (T_2) - u_f (T_1) &= C_r (T_2 - T_1) \\
    T_1 &= 0 \degree \text{C} \\
    T_2 &= 0{,}003 \degree \text{C} \\
    u_2 (T_2) - u_f (T_1) &= \frac{\Delta Q}{m_w} = \frac{+1{,}5 \, \text{kJ}}{0{,}1 \, \text{kg}} = 15 \, \frac{\text{kJ}}{\text{kg}} \\
    u_f (T_1) &= u_f + x_1 (u_{est} - u_f) = -0{,}045 + 0{,}6 \cdot (-333{,}158 + 0{```latex


\begin{equation*}
x_2 = \frac{(U_2 - U_x)}{(U_{\text{rest}} - U_y)} = \text{(unreadable scribbles)}
\end{equation*}

\begin{equation*}
U_2 = 15 \frac{\text{kJ}}{\text{kg}} + U_1(T) = 15 \frac{\text{kJ}}{\text{kg}} - 200 \frac{\text{kJ}}{\text{kg}} = -185,0 \frac{\text{kJ}}{\text{kg}}
\end{equation*}

\begin{equation*}
x_2 = \frac{(-185 + 0,033)}{(-333,442 + 0,033)} = 0,554
\end{equation*}

\begin{flushleft}
\text{Theoretisch nicht möglich,}\\
\text{weil, wenn Eis gibt, 0°C}\\
\text{die Temperatur sollte sein.}
\end{flushleft}

``````latex