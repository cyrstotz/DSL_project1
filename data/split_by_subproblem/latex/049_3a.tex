a) Im Zustand 1 haben wir ein Druck bekommen:

\[
\text{Punkt} \quad \frac{F_S}{A} = p_1 \quad \Rightarrow \quad p_1 = \frac{(m_x + m_w)g}{\left(\frac{D}{2}\right)^2 \pi} = \boxed{140089,4 \, \text{Pa}}
\]

Mit \quad \( pV = nRT \)

\[
T = 500 + 273,15 = 773,15 \, \text{K}
\]

\[
\Rightarrow n = \frac{pV}{RT} \approx 0,068
\]

Mit \quad \( M_g = 50 \, \frac{\text{mg}}{\text{mm}} \)

\[
\text{erhalten wir} \quad m_g = \frac{M_g}{1000} \quad n \approx 3,42 \, \text{g}
\]