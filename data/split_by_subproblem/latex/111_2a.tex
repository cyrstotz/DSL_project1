a)

\[
T \left[ K \right]
\]

\begin{description}
    \item[Graph Description:] The graph is a plot with the x-axis labeled \( S \left[ \frac{J}{kg \cdot K} \right] \) and the y-axis labeled \( T \left[ K \right] \). The plot contains several curves and lines:
    \begin{itemize}
        \item There are two horizontal dashed lines labeled \( T = T_3 \) and \( T = T_1 \).
        \item There are three vertical dashed lines labeled \( S_1 \), \( S_2 \), and \( S_3 \).
        \item There is a solid curve starting from the origin, rising steeply, and then leveling off.
        \item Another solid curve starts from the end of the first curve, rises steeply again, and then levels off.
        \item The curves are connected by vertical lines, forming a closed loop.
        \item The region enclosed by the curves and lines is shaded.
        \item The term "Isobare" is written near the top right of the graph.
        \item The term "v_n c" is written near the middle right of the graph.
    \end{itemize}
\end{description}

``````latex