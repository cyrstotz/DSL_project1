

\subsection*{b)}

\textbf{Zustand 2}

Gesamte Wärmehaushalt bleibt gleich

\[
m_{ges} \cdot c_{p} \cdot T_1 + m_{ew} \cdot c_p \cdot T_1 = m_{ges} \cdot c_p \cdot T_2 + m_{ew} \cdot c_p \cdot T_2
\]

\[
c_p = R + c_v
\]

\[
= 0.997 \, \frac{\text{kJ}}{\text{kg} \cdot \text{K}}
\]

\[
m_{ew} = m_{eis} + m_{wasser}
\]

\[
m_{eis} = 0.06 \, \text{kg}
\]

\[
m_{wasser} = 0.04 \, \text{kg}
\]

Die thermische Masse vom Wasser ist viel höher als die vom Gas. Daher ist das Gas kropficker bei 0°C.

\[
p_s = \frac{m \cdot R \cdot T_2}{V} = \left( \frac{p_1 \cdot T_2}{T_1} \right) = \left( \frac{p_1 \cdot 273}{773} \right)
\]

``````latex