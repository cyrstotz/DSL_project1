a) Da der Reaktor bleibt bei \(100^\circ C\) es gilt:
\[
\dot{Q}_{\text{zu}} + \dot{m} (h_{\text{zu}} - h_{\text{ein}}) = \dot{Q}_e \Rightarrow \dot{Q}_{\text{net}} = Q_{\text{zu}} - \dot{m} (h_{\text{aus}} - h_{\text{ein}})
\]
wobei wir können \(h_{\text{zu}}\) und \(h_{\text{ein}}\) von Tabelle A-2 finden:
\[
h_{\text{zu}} = x_b h_g (T = 100) + (1 - x_b) h_f (T = 100) = 730.72
\]
\[
h_{\text{ein}} = x_b (h_f (T = 70)) + (1 - x_b) h_f (T = 70) = 709.6581
\]
\[
\Rightarrow \dot{Q}_{\text{net}} = 100 \times 100 - 37.7012975 = \boxed{62.299 \text{ kW}}
\]