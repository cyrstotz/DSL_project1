\section*{2. a)}

\begin{description}
    \item[Graph 1:] The first graph is a plot with the vertical axis labeled \( T \) [K] and the horizontal axis labeled \( S \) \(\left[\frac{kJ}{kg \cdot K}\right]\). The graph contains several curves:
    \begin{itemize}
        \item A blue curve that starts from the origin, rises to a peak, and then falls back down symmetrically.
        \item Several intersecting lines of different slopes crossing through the blue curve.
    \end{itemize}
    
    \item[Graph 2:] The second graph is a plot with the vertical axis labeled \( T \) [K] and the horizontal axis labeled \( S \) \(\left[\frac{kJ}{kg \cdot K}\right]\). The graph contains several curves and points:
    \begin{itemize}
        \item A series of curves that form a zigzag pattern, labeled with numbers 1 through 6 at each turning point.
        \item A line labeled \( P_0 \) that intersects the zigzag pattern.
        \item Two arrows indicating different processes:
        \begin{itemize}
            \item An arrow labeled \( s = \text{const} \) pointing horizontally to the right.
            \item An arrow labeled \( s \neq \text{const} \) pointing upwards and to the right.
        \end{itemize}
    \end{itemize}
\end{description}

\noindent Below the second graph, there is a handwritten note: \\
\textit{Zustand 4 liegt nicht per se auf der ps Isobare}

``````latex


\begin{itemize}