

\subsection*{c)}

Nehme dass das Teilsystem adiabate, isotherme Umgebung ist:

Entropiebilanz mit \(\dot{T}\):

\[
\dot{S} = \dot{m}_{in} s_{Ein} - \dot{m}_{out} s_{Aus} + \frac{\dot{Q}}{T} + \dot{S}_{ext} = 0 \quad \text{da stationär System}
\]

\[
\Rightarrow \frac{\dot{Q}}{T} = \dot{m}_{in} s_{Ein} - \dot{m}_{out} s_{Aus} = -\```latex


\section*{1)}

\begin{align*}
    \dot{m}_{ein} - \dot{m}_{aus} &= 0 \\
    m_{nass} &= 5.755 \text{ kg} \\
    T_{ein,12} &= 20^\circ \text{C} \\
    T_{aus,1} &= 100^\circ \text{C} \\
    x_{D,1} &= 0.005 \\
    x_{D,2} &= 0 \\
    T_{aus,2} &= 70^\circ \text{C}
\end{align*}

\noindent
\(\Rightarrow\) Wir geben dem Nass sein Feucht \(m_{nass}\) mit 0.005 \(\dot{m}_{nass}\) an Dampf und dann an andere Feuchtigkeit und 70°C entweicht wieder.

\noindent
\(\Rightarrow\) Energiegleichung des geschlossenen Systems:

\begin{align*}
    \dot{E} &= \frac{d}{dt} (KE) + \frac{d}{dt} (PE) + \frac{d}{dt} (u) = \dot{Q} - \dot{W} \\
    \Rightarrow \dot{u} &= \dot{Q}
\end{align*}

\noindent
Innere Energie in Zustand 4 aus A-2:

\begin{align*}
    u_{Dampf}(100^\circ \text{C}) &= 2504.5 \frac{\text{kJ}}{\text{kg}} \\
    u_{Flüssig}(100^\circ \text{C}) &= 419.04 \frac{\text{kJ}}{\text{kg}} \\
    u_{Flüssig}(70^\circ \text{C}) &= 293.95 \frac{\text{kJ}}{\text{kg}}
\end{align*}

\noindent
\(\Rightarrow\)

\begin{align*}
    \Delta m_{12} &= \frac{Q_{12} - m_{nass} \cdot u_{Flüssig}(70^\circ \text{C})}{u_{Flüssig}(70^\circ \text{C})} \\
    &\approx 2794.6 \frac{\text{kJ}}{\text{kg}}
\end{align*}



\section*{c)}

``````latex