\subsection*{a) p-T Diagramm}

\textbf{Graph Description:} The graph is a pressure-temperature ($p$-$T$) diagram. The $y$-axis is labeled $p$ (bar) and the $x$-axis is labeled $T$ (K). There is a curve that starts at a point labeled 1, rises to a peak at point 2, and then falls to point 3. From point 3, there is a line that rises steeply to point 4. The region between points 1, 2, and 3 is shaded. There is a horizontal line extending from point 2 to the right, labeled $T2$.

a) 
\[
\dot{E}_K = \frac{\dot{Q}_{zw}}{\dot{w}_t} = \frac{\dot{Q}_K}{\dot{w}_K} = \frac{\dot{Q}_{zw}}{\dot{Q}_{ab} - \dot{Q}_{zw}} = \frac{\dot{Q}_K}{\dot{Q}_{ab} - \dot{Q}_K}
\]

\[
\dot{Q}_K = \dot{m} \dot{R} s u (h_2 - h_1)
\]

\[
\dot{Q}_K = 205,79 \text{ kJ}
\]

\[
\dot{Q}_{ab} = \dot{m} \dot{R} s u (h_4 - h_3)
\]

\[
\dot{Q}_{ab} = 156,29
\]

\[
\dot{E}_K = \frac{\dot{Q}_K}{\dot{w}_t} = 7349,64
\]

\[
\text{Tab } \Delta 10
\]

\[
h_2 = 234,08
\]

\[
h_1 = h_e + x_1 (h_g - h_f) = 48,87
\]

\[
h_2 = 234,08
\]

\[
h_4 = \text{Tab } \Delta M
\]

\[
h_4 = h_e = 93,42
\]