

\section*{b)}

\[
T_{g,2} = T_{ew,1}
\]

\[
= \underline{0^\circ \text{C}}
\]

\begin{itemize}
    \item Das Eis-Wasser-Gemisch ändert durch den Wärmestrom seine Temperatur nicht, da es bereits im fest-flüssig Gebiet ist.
    \item isotherm
    \item Im GGW haben Gas \& Eis die gleiche Temperatur.
\end{itemize}

\begin{center}
\begin{picture}(100,50)
\put(10,20){\line(1,0){80}}
\put(10,30){\line(1,0){80}}
\put(10,20){\line(0,1){10}}
\put(90,20){\line(0,1){10}}
\put(45,25){\makebox(0,0){\textbf{(scribbled area)}}
\end{picture}
\end{center}

\[
p_{a2} = p_{a1} = \underline{1.4 \, \text{bar}}
\]

\begin{itemize}
    \item Druck bleibt gleich, weil sich am Kräftegleichgewicht nichts ändert, nur weil Wärme übertragen wurde (s. Aufgabe 2).
\end{itemize}

``````latex