c) adiabate Drossel - isenthalp Tabelle A11

\[
h_4 = h_3
\]

\[
h_3 = h_f (8.0 \text{bar}) = 93.92 \frac{\text{kJ}}{\text{kg}}
\]

Der Verdampfer im Wasserdampfgebiet liegt und bei -16°C operiert. Kann man das Dampfgehalt ausrechnen Tabelle A-10

\[
X_m = \frac{h_1 - h_f (-16^\circ \text{C})}{h_g (-16^\circ \text{C}) - h_f (-16^\circ \text{C})} = 30.76\%
\]

``````latex


\section*{Aufgabe 4}