

\subsection*{d)}
\begin{align*}
    \text{Energiebilanz am geschlossenen System:} & \\
    \Delta oE = m_g (u_2 - u_1) = Q_{12} - W_{12} \\
    W_{12} = \int p \, dV = p_a (V_2 - V_1) & \text{ mit } V_{g2} = \frac{m_g}{p_{02}} \\
    V_{g2} = 0.000111 \text{ m}^3 \Rightarrow V_{g2} = \frac{1.401 \cdot 10^5 \text{ Pa}}{0.000111 - 0.00349 \text{ m}^3} = -294.5 \text{ J} \\
    u_2 - u_1 = c_v (T_{g2} - T_g```latex



\section*{d)}

zuerst müssen wir $u_{m}$ berechnen.

\textbf{analogie: um Wasserdampf:} $x_{m}$ gleich wie $x_{gas}$, alle Werte aus Tafel 1 bis 2.1 bar

\[
L_{q} = u_{f} + x \left( u_{eis} - u_{f,gas} \right) = \left( -0.045 + 0.6 \cdot \left( -333.859 + 0.045 \right) \frac{kJ}{kg} \right)
\]

\[
u_{m} = -200.0828 \frac{kJ}{kg}
\]

\textbf{Energieerhaltung am Eis zugeflossen:} $m_{ew} \left( u_{2} - u_{1} \right) = \left| Q_{12} \right| - Q_{2,EW}$

\[
L_{q} u_{2} = u_{1} + \frac{\left| Q_{12} \right|}{m_{ew}} = -200.0828 \frac{kJ}{kg} + \frac{1.3675 \frac{kJ}{kg}}{0.1 \text{kg}} = -186.4 \frac{kJ}{kg} = u_{2}
\]

\textbf{nach $x_{2}$ aufgelöst:} $x_{2} = \frac{u_{2} - u_{f}}{u_{ris} - u_{f}}$

\textbf{Werte in 1.8bar Tafel, da wir immer noch in Wasserdampf sind und $p_{1} = p_{2}$, $T_{1} = T_{2}$}

\[
x_{2} = \frac{-186.4 + 0.045}{-333.859 + 0.045} = 0.555 = x_{2}
\]

``````latex