

\item[b)]
    \[
    T_{g2} = 0^\circ \text{C} = 273.15 \, \text{K}
    \]
    Da \( x_{Ew,2} > 0 \), ist die Temperatur vom EW im Zustand 2 \( 0^\circ \text{C} \). Da Zustand 2 schon der Gleichgewichtszustand (Endzustand) ist, ist die Temperatur vom Gas gleich wie die vom EW, also \( 0^\circ \text{C} \).
    \[
    p_{g2} = p_{g1} = 1.4 \, \text{bar}
    \]
    Da die obige Masse unverändert ist, muss der Druck gleich bleiben, sodass das Gas die Masse "tragen" könnte, d.h. Gleichgewichtszustand bleiben.