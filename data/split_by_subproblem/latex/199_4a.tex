

\subsection*{a)}

\begin{itemize}
    \item Ein Diagramm mit der y-Achse beschriftet als $p(\text{Pa})$ und der x-Achse beschriftet als $T(\text{K})$. 
    \item Es gibt drei Phasenbereiche, die durch Linien getrennt sind:
    \begin{itemize}
        \item Ein Bereich links oben, der als "fest" beschriftet ist.
        \item Ein Bereich rechts oben, der als "flüssig" beschriftet ist.
        \item Ein Bereich unten, der als "gas" beschriftet ist.
    \end{itemize}
    \item Eine grüne vertikale Linie, die von der x-Achse nach oben verläuft und mit "i" markiert ist.
    \item Eine grüne horizontale Linie, die von der y-Achse nach rechts verläuft und mit "ii" markiert ist.
    \item Eine blaue Kurve, die von links unten nach rechts oben verläuft und die Phasenbereiche trennt.
\end{itemize}



\subsection*{a)}
\begin{equation*}
    \epsilon_u = \frac{\dot{Q}_{zu}}{(W+1)} = \frac{\dot{Q}_k}{28W}
\end{equation*}