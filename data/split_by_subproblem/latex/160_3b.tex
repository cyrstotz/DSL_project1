

\subsection*{b)}
\[
x_{eis,2} > 0 \quad x_{eis,1} = \frac{m_{eis}}{m_{ew}} = 0{,}6
\]

\[
\Rightarrow m_{eis} = 0{,}1 \cdot m_{ew} = 0{,}06 \, \text{kg}
\]

\textbf{Gas und EW Thermodyn. GGW}

Weil Dichte von Eis und Wasser gleich sind, verändert sich die Masse (und Volumen) von Eiswasser nicht. Durch das KGW sieht man, dass $p_{g2} \stackrel{!}{=} p_{g1}$

\[
p_{g1,2} = 1{,}4 \, \text{bar}
\]

``````latex

\section*{b) fort.}

\begin{figure}[h!]
    \centering
    \begin{tabular}{c}
        \begin{tabular}{|c|}
            \hline
            EV \\
            \hline
        \end{tabular} \\
        \begin{tabular}{|c|}
            \hline
            Gas \\
            \hline
        \end{tabular} \\
    \end{tabular}
    \caption*{A diagram showing two horizontal sections. The top section is labeled "EV" and the bottom section is labeled "Gas". There is an arrow labeled "$\dot{Q}$" pointing upwards from the "Gas" section to the "EV" section.}
\end{figure}

\[
m_{g1} = m_{g2} = 3,422 \, \text{g}
\]

\noindent
Weil das EV dann wie Wasser im Nassdampf behandelt werden kann und $x_2 > 0$ ist, ist $T_{EV,2} = T_{EV,1} = 0^\circ \text{C}$

\noindent
Für thermodynamisches Gleichgewicht muss $T_{g,2} = T_{EV,2} = 0^\circ \text{C}$ sein