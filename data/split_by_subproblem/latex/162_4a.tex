

\section*{4a)}

\begin{itemize}
    \item There is a graph with the vertical axis labeled \( P \) and the horizontal axis labeled \( T \).
    \item A line starts from the origin and goes upwards to the right, labeled \( T_{\text{inel}} \).
    \item Another line starts from the origin and goes upwards to the right, but at a steeper angle.
    \item A horizontal line intersects the steeper line at point 1 and is labeled \( p_4 \).
    \item A vertical line goes down from point 1 to point 2 on the first line.
    \item A dashed line goes from point 2 downwards to the left, ending at point 3 on the horizontal axis, labeled \( T_z \).
\end{itemize}



\item[(a)] 
    \begin{align*}
        S_4 &= S_{1,4} \\
        S_4 &= S_f + \\
        X &= \frac{S_4 - S_f}{S_9 - S_f}
    \end{align*}