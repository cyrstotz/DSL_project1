

\subsection*{c)}
\[
m_1 = m_2 \quad \text{und} \quad p_1 = p_2 \quad \text{weil gleiches "Gewicht" von dem Druckstück!!}
\]

\[
\Rightarrow \text{isotherme Polytrope Veränderung}
\]

\[
T_2 = T_1 \left( \frac{V_1}{V_2} \right)^{n-1}
\]

\[
p_2 \cdot v_2 = R \cdot T_2 \quad \Leftrightarrow \quad v_2 = \frac{R \cdot T_2}{p_2}
\]

\[
T_2 = T_1 \left( \frac{V_1}{\frac{R T_2}{p_2}} \right)^{n-1} = T_1 \cdot \frac{R T_2}{V_1 p_2}
\]

``````latex

\section*{Problem c}

\begin{align*}
    T_{912} &= 0.003^\circ C \\
    p_2 &= 7.900 \, bar \\
    m_2 &= 3.427 \cdot 10^{-3} \, kg
\end{align*}

1. HS über Gas: \quad \text{nach Lösens S. 5/6m.}

\begin{align*}
    \Delta U_{12} &= m Q_{12} - \cancel{W^0} \\
    & \quad \text{KE + PE vernachlässigbar} \\
    T_2 &= 500^\circ C
\end{align*}

\begin{align*}
    \Delta U_{12}^d &= C_V (T_2 - T_1) = -376.49 \, kJ \\
    & \quad \text{ideales Gas}
\end{align*}

\begin{align*}
    U_{12} = m_2 \cdot \Delta u_{12} &= 1.0824 \, kJ = \underline{Q_{12}}
\end{align*}