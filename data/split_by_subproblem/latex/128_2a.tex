\section*{3. a)}

\begin{equation*}
\begin{array}{l}
\text{Diagram:} \\
\text{A horizontal membrane is shown with arrows pointing upwards labeled } P_0. \\
\text{Above the membrane, there are three blocks labeled } m_{ew}. \\
\text{The membrane is labeled as } \text{Membran}. \\
\end{array}
\end{equation*}

\begin{equation*}
m k g + p_0 A t + m_{ew} g = p_{G1} A
\end{equation*}

\begin{equation*}
\frac{g}{A} (m k + m_{ew}) + p_0 = p_{G1}
\end{equation*}

\begin{equation*}
\frac{3.8 \, \frac{m}{s^2}}{(0.5 \, m)^2 \pi} (32 \, kg + 0.1 \, kg) + 1 \, \text{bar} = 1.9 \, \text{bar} = p_{G1}
\end{equation*}

\begin{equation*}
\text{0.05 m}
\end{equation*}

\begin{equation*}
\text{Diagram:} \\
\text{A horizontal piston-cylinder arrangement is shown. The piston is labeled } m_{G1}. \\
\text{The volume } V_{G1} \text{ is shown to the right of the piston.} \\
\end{equation*}

\begin{equation*}
p_{G1} V_{G1} = \frac{R}{M_{gas}} T_{G1} \cdot m_{G1}
\end{equation*}

\begin{equation*}
m_{G1} = \frac{p_{G1} V_{G1}}{T_{G1}} \cdot \frac{M_{gas}}{R} = \frac{1.9 \, \text{bar} \cdot 3.14 \cdot 10^{-3} \, m^3}{773 \, K} \cdot \frac{50 \, kg}{8.314 \, \frac{kJ}{kmol \cdot K}}
\end{equation*}

\begin{equation*}
= 3.42 \, kg = m_{G1}
\end{equation*}