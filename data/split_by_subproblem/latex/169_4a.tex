

\subsection*{a)}

\begin{description}
    \item[Graph 1:] A pressure-volume (p-v) diagram with a complex curve. The curve starts at the bottom left, rises steeply, then falls, rises again, and falls once more. There are four points marked on the curve: 
    \begin{itemize}
        \item Point 1 is on the first rise.
        \item Point 2 is on the second rise.
        \item Point 3 is on the second fall.
        \item Point 4 is on the first fall.
    \end{itemize}
    There are horizontal lines connecting points 1 to 2 and 3 to 4, labeled as "isobar". The line connecting points 1 to 2 is labeled "x=0" at point 1 and "x=1" at point 2. The line connecting points 3 to 4 is labeled "p=5 bar". There is an arrow indicating the direction of the process from point 1 to point 2 and from point 3 to point 4.

    \item[Graph 2:] A pressure-temperature (p-T) diagram with a dome-shaped curve. The curve starts at the bottom left, rises to a peak, and then falls symmetrically. There are three points marked on the curve:
    \begin{itemize}
        \item Point 1 is on the left side of the dome.
        \item Point 2 is on the right side of the dome.
        \item Point 3 is at the peak of the dome.
    \end{itemize}
    There are horizontal lines connecting points 1 to 2 and 3 to 4, labeled as "isobar = 8 bar". The line connecting points 1 to 2 is labeled "x=0" at point 1 and "x=1" at point 2. There is an arrow indicating the direction of the process from point 1 to point 2.
\end{description}