

\subsection*{b)}
\begin{align*}
X_{\text{Eis2}} &> 0 \rightarrow T_{\text{Ew2}} = T_{\text{Ew1}} = 0^\circ \text{C} \\
\text{Therm. GW} &\Rightarrow Q = 0 \Rightarrow \Delta T_{G2} = 0^\circ \text{C} = T_{\text{Ew}} \\
p_{G2} &\Rightarrow Gewicht auf Kolben ändert sich nicht \\
p_{G2} &= p_{s1, A} = 1.400342 \, \text{bar}
\end{align*}

\begin{quote}
Da der Eisanteil größer als komplett bedeutet, dass die Eisküsse-Temperatur gleich bleibt und zudem im Zustand 2 keine Wärmeströme mehr existieren muss, die zweite Gas-Temperatur 0°C sein. Zudem bleibt der Druck im Gas konstant, da die Kolbenkraft konstant bleibt.
\end{quote}