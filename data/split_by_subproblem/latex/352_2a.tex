\section*{2. a)}

\begin{description}
    \item[Graph Description:] The graph is a Temperature-Entropy (T-s) diagram. The vertical axis is labeled "T" and the horizontal axis is labeled "S [\frac{kJ}{kgK}]". The graph consists of six points labeled from 0 to 5, connected by lines. The points and their connections are as follows:
    \begin{itemize}
        \item Point 0 is at the origin.
        \item Point 1 is above and to the right of Point 0, connected by a curved line.
        \item Point 2 is directly above Point 1, connected by a vertical line labeled "isobar".
        \item Point 3 is above and to the right of Point 2, connected by a line labeled "isobar".
        \item Point 4 is below and to the right of Point 3, connected by a line labeled "isentrop".
        \item Point 5 is directly above Point 4, connected by a vertical line labeled "isobar".
        \item Point 6 is below and to the right of Point 5, connected by a line labeled "isentrop".
    \end{itemize}
\end{description}

\noindent
\textbf{Bei:} \underline{1} teilt sich die Luft auf, m2 durchläuft den Turbinenprozess, während m1 direkt weiter zur Mischeinmer geführt wird.

\section*{1)}

\[
\dot{m} \sum h = \sum \dot{Q} - \sum \dot{W}
\]

\[
0 = \dot{m} (h_{ein} - h_{aus})
\]

``````latex