

\subsection*{a)}

\begin{itemize}
    \item \textbf{Diagram Description:} The diagram shows a rectangular block labeled "EW" with a membrane at the bottom. There are three forces acting on the block: $F_z$ acting upwards, $F_1$ acting downwards on the piston, and $F_3$ acting downwards on the membrane. The forces are labeled as follows:
    \begin{itemize}
        \item $F_z$ (upwards)
        \item $F_1$ (downwards on the piston)
        \item $F_3$ (downwards on the membrane)
    \end{itemize}
    The block is labeled "Kolben" (piston) and "Membran" (membrane).
\end{itemize}

\begin{itemize}
    \item $EW$-Gemisch inkompressibel $\rightarrow$ deshalb als Block betrachten
    \item $F_3 = m_{EW} \cdot g = F_g$ des $EW$
    \item $p = \frac{F}{A} \Rightarrow \left[ \frac{N}{m^2} \right] \Rightarrow F = p \cdot A$
    \item $F_1 = p_{amp} \cdot A \rightarrow A = \pi r^2 \rightarrow r = \frac{D}{2} = 5 \text{cm}$
    \item \textbf{Diagram Description:} A circle with radius $r = 0.05 \text{m}$ is shown, and the area $A = 7.854 \cdot 10^{-3} \text{m}^2$ is calculated.
    \item $F_2 = m \cdot g \rightarrow$ Schwerkraft des Gewichts auf dem Kolben
    \item $F_u = p_{0,1} \cdot A \rightarrow p_{0,1} = 1.4 \text{bar} \rightarrow F_3 = 0.981 \text{N}$
    \item $F_z = 785.398 \text{N}$
    \item $F_3 = 343.92 \text{N}$
\end{itemize}