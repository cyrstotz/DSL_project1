\subsection*{3. a)}

\begin{equation*}
    p_{1,g} = \frac{m_g \cdot R \cdot T_{g,1}}{V_{g,1}}
\end{equation*}

\begin{equation*}
    m_g = \frac{p_{1,g} \cdot V_{g,1}}{R \cdot T_{g,1}}
\end{equation*}

\begin{equation*}
    R = \frac{\overline{R}}{M} \quad \overline{R} = 8.314 \frac{J}{mol \cdot K}
\end{equation*}

\begin{equation*}
    R = 166.28 \frac{J}{kg \cdot K}
\end{equation*}

\begin{equation*}
    p_{1,g} = \frac{\pi}{4} D^2 \cdot (m_k + m_{eV}) + p_{amb}
\end{equation*}

\begin{equation*}
    = 1.00 \, \text{bar}
\end{equation*}

\begin{equation*}
    m_g = 5.84 \, g
\end{equation*}

``````latex


3.

a) vorgehen.

T_{g,2} = T_{g,1}

Da für ideale Gase Temperatur und Druck gekoppelt sind.

``````latex


2.

\section*{3. a)}

\begin{equation*}
    x = \frac{u - u_f}{u_g - u_f}
\end{equation*}

Da die Membran wärmeüberlagert ist:

\begin{equation*}
    T_{g,1} = T_{g,2} = 0,003^\circ C
\end{equation*}

\begin{equation*}
    \Delta E = -Q_{12}
\end{equation*}

\begin{equation*}
    \frac{\Delta E}{m_{ev}} = u = -15 \frac{a^2}{4g}
\end{equation*}

\begin{equation*}
    \Rightarrow x = \frac{T_{g}^b 75 - (-0,077)}{-233,442 + 0,022} = 0,0449
\end{equation*}

``````latex