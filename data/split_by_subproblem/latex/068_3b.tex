\subsection*{b) Sublimation}

\begin{align*}
    (1) \rightarrow (9) & \Rightarrow p = 2 \, \text{mbar} = 2 \cdot 10^3 \, \text{bar} = 2 \cdot 10^3 \cdot 10^{-5} \, \text{bar} \\
    -20^\circ \text{C} & = 0.1 \, \text{bar} \quad \text{= echte während Sublimation!!}
\end{align*}

\text{Aus TAB A-6: Interpolation:}
\begin{align*}
    T(0.21 \, \text{kPa}) & = T(0.20388 \, \text{kPa}) + \frac{T(0.0883 \, \text{kPa}) - T(0.20388 \, \text{kPa})}{0.1635 - 0.0883} \cdot 0.1 \, \text{kPa} \\
    (0.1 - 0.0883) \, \text{kPa} & = -22^\circ \text{C} \approx 20.385^\circ \text{C}
\end{align*}

\text{Diagram description:}
\begin{itemize}
    \item There is a horizontal line labeled "T2 = Sub-punkt".
    \item Below this line, there is a note "20K über sublimationspunkt".
    \item An arrow points from "20K" to "20.385^\circ \text{C}".
    \item Another horizontal line labeled "T2 = Sub-punkt = 20.385^\circ \text{C} \approx 20^\circ \text{C}, wenn man es vom Diagramm abliest".
\end{itemize}

\text{Process 1-2: Verdampfen}
\begin{align*}
    T_{\text{verdampfen}} & = T_{\text{verflüssigen}} + 6 \, \text{K} = 26^\circ \text{C}
\end{align*}

\text{Stationärer Fließprozess:}
\begin{align*}
    0 & = \dot{m} \left( h_1 + \frac{v_1^2}{2} + g z_1 \right) - \left( h_2 + \frac{v_2^2}{2} + g z_2 \right) + \dot{Q} - \dot{W} \\
    \dot{Q}_{\text{zu}} & = \dot{Q}_{\text{K}} + \dot{Q}_{\text{AB}} \\
    \dot{Q}_{\text{K}} & = \dot{Q}_{\text{zu}} - \dot{Q}_{\text{AB}}
\end{align*}

\text{Process 3-4:}
\begin{align*}
    \dot{m} \cdot \Delta h & = \dot{Q}_{\text{zu}} - \dot{Q}_{\text{AB}} \\
    \dot{m} \cdot \Delta h & = \dot{m} \cdot (h_3 - h_4) - \dot{Q}_{\text{AB}} \\
    \dot{m} \cdot \Delta h & = \dot{m} \cdot (h_3 - h_4) - \dot{Q}_{\text{AB}} \\
    \dot{m} \cdot \Delta h & = \dot{m} \cdot (h_3 - h_4) - \dot{Q}_{\text{AB}} \\
    \dot{m} \cdot \Delta h & = \dot{m} \cdot (h_3 - h_4) - \dot{Q}_{\text{AB}} \\
    \dot{m} \cdot \Delta h & = \dot{m} \cdot (h_3 - h_4) - \dot{Q}_{\text{AB}} \\
    \dot{m} \cdot \Delta h & = \dot{m} \cdot (h_3 - h_4) - \dot{Q}_{\text{AB}} \\
    \dot```latex


\begin{equation}
(II) \Rightarrow \dot{m}_e = \frac{\dot{Q}_K - \dot{W}_K}{h_3 - h_4} \Rightarrow \dot{Q}_K = \dot{m}_e (h_3 - h_4) + \dot{W}_K \tag{III}
\end{equation}

\begin{equation}
(III) \Rightarrow (I) \Rightarrow \dot{m}_e = \dot{m}_e \left( \frac{h_3 - h_4}{h_2 - h_1} \right) + \frac{\dot{W}_K}{h_2 - h_1}
\end{equation}

\begin{equation}
\Rightarrow \dot{m}_e \left( 1 + \frac{h_3 - h_4}{h_2 - h_1} \right) = \frac{\dot{W}_K}{h_2 - h_1}
\end{equation}

\begin{equation}
\Rightarrow \dot{m}_e \left( \frac{h_2 - h_1 + h_3 - h_4}{h_2 - h_1} \right) = \frac{\dot{W}_K}{h_2 - h_1}
\end{equation}

\begin{equation}
\Rightarrow \dot{m}_e = \frac{\dot{W}_K}{\frac{h_2 - h_1 + h_3 - h_4}{h_2 - h_1}} = \frac{\dot{W}_K}{h_2 - h_1 + h_3 - h_4}
\end{equation}

\begin{equation}
\dot{m}_e = \frac{\dot{W}_K}{h_2 - h_1 + h_3 - h_4} = \frac{28 \frac{kJ}{s}}{(232,62 - 16,82 + 93,42 - 264,25) \frac{kJ}{kg}}
\end{equation}

\begin{equation}
\Rightarrow \dot{m}_e = 0,6352 \frac{kg}{s}
\end{equation}

Aus TABELLE A-20:

\begin{equation}
@ T_2 = T_3 = 260^\circ C, \quad h_{g_2} = h_{g_3} = 232,62 \frac{kJ}{kg}
\end{equation}

\begin{equation}
h_2 = h_f = 16,82 \frac{kJ}{kg}
\end{equation}

Aus TABELLE A-12:

\begin{equation}
@ p_3 = 3 \text{bar} \quad x = 0,6
\end{equation}

\begin{equation}
h_3 = h_f = 264,25 \frac{kJ}{kg}, \quad h_{fg} = h_{g_2} - h_{f_2} = 93,42 \frac{kJ}{kg}
\end{equation}

\begin{equation}
\Rightarrow \dot{m}_e = 0,6352 \frac{kg}{s} \quad \text{or} \quad 4 \frac{kg}{h}
\end{equation}

\begin{equation}
\Rightarrow \dot{m}_{\text{gesammt}} = 0,6352 \frac{g}{s}
\end{equation}