b)

\[
\begin{array}{ccccc}
\phi & p & T & h & s \\
1 & 1.3748 & -16 & & \\
x=1 & 2 & p1 & -16 & \\
3 & 8 & & & \\
x=0 & 4 & 8 & & \\
\end{array}
\]

\[
T_{1,2} = T_i - 6K = -16^\circ C
\]

\[
h_2 = h_g(-16^\circ C) \rightarrow A10
\]

\[
h_2 = 237.74 \frac{kJ}{kg}
\]

\[
s_2 = s_3 = s_g(-16^\circ C) \quad s_2 = 0.9288 \frac{kJ}{kgK}
\]

\[
p_3 = 8 \text{bar} \rightarrow \text{via A-M sehen wir, wir sind im Dampfgebiet}
\]

\[
\rightarrow A-12
\]

\[
h_3 = h_{sat} + \frac{(h(40) - h_{sat})}{s(40) - s_{sat}} \cdot (s_3 - s_{sat})
\]

\[
h(40) = 273.66
\]

\[
h_{sat} = 269.45
\]

\[
s_{sat} = 0.9066
\]

\[
s(40) = 0.9374
\]

\[
h_3 = 271.3 \frac{kJ}{kg}
\]

``````latex


\begin{align*}
h_2 &= 237.74 \\
h_3 &= 271.3 \\
\end{align*}

\text{2} \rightarrow \text{3} \quad \text{1. HS}

\begin{align*}
0 &= \dot{m}(h_2 - h_3) + \dot{Q} - \dot{W} \\
\frac{\dot{W}}{h_2 - h_3} &= \dot{m}
\end{align*}

\text{adiabat} \quad \dot{W} \downarrow -28 \frac{\text{kW}}{\text{kg}}

\begin{align*}
\dot{m} &= 0.884 \frac{\text{g}}{\text{s}}
\end{align*}

\underline{\dot{m} = 0.884 \frac{\text{g}}{\text{s}}}

\begin{itemize}