

\subsection*{b)}
\begin{align*}
    m_g &= m_g \frac{V_{g1}}{R T_{g1}} = 0,003 \text{kg} \quad R = \frac{1}{M_g} = 0,4 \frac{\text{J}}{\text{kg} \cdot \text{K}} \\
    &= 3,42 \text{g}
\end{align*}



\subsection*{b)}
Im Zustand 2, also im Endzustand, müssen das Eiswasser und das Gas die gleiche Temperatur haben, $T_{g2} = 0^\circ C$, damit keine Wärme mehr ausgetauscht wird und das System "konstant" bleibt.

$p_2$ bleibt gleich wie $p_1$, da sich die äußeren Bedingungen nicht ändern.