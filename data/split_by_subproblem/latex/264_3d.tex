

\subsection*{d)}
$x_{\text{Eis,2}}$ in Zustand 2

dafür benötigen wir diese Formel

\[
\frac{U - U_{\text{Fest}}}{U_{\text{Flüssig}} - U_{\text{Fest}}} = x_{\text{Eis,2}}
\]

\[
\text{unser } U = \frac{U}{m_{\text{sw}}} = \frac{Q}{m_{\text{sw}}}
\]

\[
= \frac{1082}{1.08} = 1000 \frac{\text{kJ}}{\text{kg}}
\]

\[
= 10.82 \frac{\text{kJ}}{\text{kg}}
\]

Da wir noch nicht vollständig flüssig sind, liegen wir bei einer Temperatur von $0^\circ$C aus der Tabelle entnommen

\[
U_{\text{Fest}} = -333.458 \frac{\text{kJ}}{\text{kg}}, \quad U_{\text{Flüssig}} = -0.645 \frac{\text{kJ}}{\text{kg}}
\]

\text{einsetzen}

\[
\frac{10.82 \frac{\text{kJ}}{\text{kg}} - (-333.458) \frac{\text{kJ}}{\text{kg}}}{(-0.645) \frac{\text{kJ}}{\text{kg}} - (-333.458) \frac{\text{kJ}}{\text{kg}}} = 1.032
\]

\text{Daraus folgt:}

\[
x_{\text{Eis,2}} = 1.032
\]

``````latex