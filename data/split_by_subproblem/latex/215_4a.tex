

\subsection*{a)}

\begin{description}
    \item[Graph Description:] The graph is a Pressure-Temperature (P-T) diagram. The vertical axis is labeled \( P \) and the horizontal axis is labeled \( T \). There are three distinct regions labeled "flüssig" (liquid), "gasförmig" (gaseous), and a line separating these regions. The line starts from the origin and curves upwards to the right. There is a horizontal line at \( 5 \text{ mbar} \) intersecting the curve. Two points are marked on the graph: point (i) is on the curve, and point (ii) is below the curve in the gaseous region. The temperature at point (ii) is labeled \( T_1 \) and the pressure at point (ii) is labeled \( 10 \text{ K} \). The region below the curve and above the horizontal line is labeled "test".
\end{description}