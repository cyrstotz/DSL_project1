\subsection*{a) Wasser der Lebensmittel}

\begin{tabular}{|c|c|c|}
\hline
p & T & \\
\hline
1 & $p_1 = p_2$ & $> T_i$ \\
2 & $p_2 = p_3$ & $T_i$ \\
3 & $p_3 < f(T,p)$ & $T_i$ \\
\hline
\end{tabular}

\subsection*{Graphical Descriptions}

\textbf{First Graph:}

The first graph is a pressure-temperature ($p$-$T$) diagram. The x-axis is labeled $T$ [°C] and the y-axis is labeled $p$ [mbar]. The y-axis has a logarithmic scale with values marked at 0.1, 1, and 10. 

- There is a curve starting from the bottom left, labeled "Wasser" (water), which rises steeply and then levels off as it moves to the right.
- Another curve, labeled "Eis" (ice), starts from the bottom left and rises more gently, intersecting the "Wasser" curve at a point labeled "Tripel" (triple point).
- From the "Tripel" point, a line extends horizontally to the right, labeled "Dampf" (steam).
- The graph is divided into three regions: "Eis" (ice) below the "Wasser" curve, "Wasser" (water) between the "Wasser" and "Dampf" curves, and "Dampf" (steam) above the "Dampf" curve.

\textbf{Second Graph:}

The second graph is another pressure-temperature ($p$-$T$) diagram. The x-axis is labeled $T$ [°C] and the y-axis is labeled $p$ [mbar]. 

- The y-axis has values marked at 1 and 5.
- There is a curve starting from the bottom left, labeled "Wasser" (water), which rises steeply and then levels off as it moves to the right.
- Another curve, labeled "Flüssig" (liquid), starts from the bottom left and rises more gently, intersecting the "Wasser" curve at a point labeled "Tripel" (triple point).
- From the "Tripel" point, a line extends horizontally to the right, labeled "Isotherm" (isotherm).
- The graph is divided into three regions: "Wasser" (water) below the "Flüssig" curve, "Flüssig" (liquid) between the "Flüssig" and "Dampf" curves, and "Dampf" (steam) above the "Dampf" curve.
- Points labeled 1, 2, and 3 are marked on the graph, with point 1 in the "Dampf" region, point 2 on the "Isotherm" line, and point 3 in the "Wasser" region.

``````latex