

\section*{3a)}

Zuerst muss der Druck im Gasbehälter berechnet werden: \\
Druck EW + Druck vom Gewicht + Druck Umgebung

\[
P_L = \frac{F}{A} = \frac{32 \, \text{kg} \cdot 9,81 \, \frac{\text{m}}{\text{s}^2}}{\pi \left(5 \cdot 10^{-2} \, \text{m}\right)^2} = 39,96 \, \frac{\text{N}}{\text{m}^2}
\]

\[
P_{EW} = \frac{F}{A} = \frac{0,74 \, \text{kg} \cdot 9,81 \, \frac{\text{m}}{\text{s}^2}}{\pi \left(5 \cdot 10^{-2} \, \text{m}\right)^2} = 724,304 \, \frac{\text{N}}{\text{m}^2}
\]

\[
P_{amb} = 10^5 \, \frac{\text{N}}{\text{m}^2}
\]

\[
P_{g,1} = P_L + P_{EW} + P_{amb} = 740059,44 \, \frac{\text{N}}{\text{m}^2}
\]

\[
R = \frac{\mathcal{R}}{M} = \frac{8,314 \, \frac{\text{J}}{\text{mol} \cdot \text{K}}}{50 \, \frac{\text{g}}{\text{mol}}} = 0,16628 \, \frac{\text{J}}{\text{g} \cdot \text{K}}
\]

\[
P_{g,1} = 1,4 \, \text{bar}
\]

\[
m_g = \frac{P_{g,1} \cdot V_{g,1}}{R \cdot T_{g,1}} = \frac{1,4 \cdot 10^5 \, \frac{\text{N}}{\text{m}^2} \cdot 3,14 \cdot 10^{-3} \, \text{m}^3}{0,16628 \, \frac{\text{J}}{\text{g} \cdot \text{K}} \cdot 7,73,15 \, \text{K}} = 3,419 \, \text{g}
\]

``````latex


36)

Das Eis-Wasser weil die Wassermoleküle im Fest-Flüssig Gebiet befindet oder eine Isotherme bis alles Eis (fest) weggeschmolzen ist. Also ist das EW immer noch $T_{EW,1} = 0^\circ C$. Somit muss auch das Gas $T_{g,2} = 0^\circ C$, thermodynamisches Gleichgewicht.

\[
T_{g,2} = 0^\circ C
\]

Die Massen welche auf das Gas wirken sind ebenfalls die gleichen. Somit $p_{g,2} = 1,46 \text{bar}$

\[
p_{g,2} = \frac{m_{g,1}}{V_{g,1}} \frac{T_{g,2}}{T_{g,1}} \frac{V_{g,2}}{V_{g,2}}
\]