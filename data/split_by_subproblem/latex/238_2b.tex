

\item[b)] 
    \textbf{Stellen wir die Energiebilanz auf:}
    
    \[
    \frac{dE}{dt} = \sum \dot{m} \left( h_i + \frac{ke}{pe} \right) + \sum \dot{Q} - \sum \dot{W}
    \]
    
    \textit{(Note: The term "o, da adiab" and "eine Schaufelzelle verrichtet keine Arbeit" are crossed out.)}
    
    \[
    \frac{T_6}{T_5} = \left( \frac{p_c}{p_s} \right)^{\frac{n-1}{n}} \Rightarrow T_6 = \left( \frac{p_c}{p_s} \right)^{\frac{n-1}{n}} T_5 = 0.7586 \cdot 431.9K \Rightarrow T_6 = 328.07K
    \]
    
    \[
    0 = \dot{m} \left[ h_5 - h_6 + \frac{w_5^2}{2} - \frac{w_6^2}{2} \right]
    \]
    
    \[
    h_5 - h_6
    \]
    
    \[
    h_5 - h_6 = c_p (T_5 - T_6) = 1.006 \frac{kJ}{kg \cdot K} (431.9K - 328.07K) = 104.44825 \frac{kJ}{kg}
    \]
    
    \[
    \frac{w_5^2}{2} = h_5 - h_6 + \frac{w_6^2}{2}
    \]
    
    \[
    w_6^2 = 2 \left( h_5 - h_6 + \frac{w_6^2}{2} \right)
    \]
    
    \[
    w_6 = \sqrt{2 \left( h_5 - h_6 + \frac{w_6^2}{2} \right)} = \sqrt{2 \cdot 104.448} = 20.47 \frac{m}{s}
    \]
    
\end{itemize}

``````latex