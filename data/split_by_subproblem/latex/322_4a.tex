

\subsection*{a)}

\begin{itemize}
    \item The first graph is a plot with the vertical axis labeled \( p \) and the horizontal axis labeled \( T \, [K] \).
    \item There are three curves in the graph:
        \begin{itemize}
            \item The first curve starts from the bottom left, curves upwards, and then flattens out as it moves to the right. This curve is labeled \( p_{L, \text{kur}} \).
            \item The second curve starts from the top left, curves downwards, and intersects the first curve. This curve is labeled \( p_{G, \text{kur}} \).
            \item The third curve starts from the bottom left, curves upwards, and intersects the second curve. This curve is labeled \( p_{G, \text{kur}} \).
        \end{itemize}
\end{itemize}

\begin{itemize}
    \item The second graph is a plot with the vertical axis labeled \( p \) and the horizontal axis labeled \( T \).
    \item There is a dome-shaped curve in the graph:
        \begin{itemize}
            \item The left side of the curve is labeled \( Flüssig \).
            \item The right side of the curve is labeled \( Gas \).
            \item The bottom of the curve is labeled \( Mischung \).
            \item The top of the curve is labeled \( Tripelpunkt \).
            \item There are two points marked on the curve:
                \begin{itemize}
                    \item Point 1 is on the left side of the curve.
                    \item Point 2 is on the right side of the curve.
                \end{itemize}
        \end{itemize}
\end{itemize}