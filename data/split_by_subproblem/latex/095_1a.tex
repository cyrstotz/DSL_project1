a) Ges: \textit{Qaus über Reaktionswand an Kühlflüssigkeit}

\begin{itemize}
    \item Es gibt zwei Diagramme. Das erste Diagramm zeigt einen vertikalen rechteckigen Behälter, der in der Mitte durch eine vertikale Linie in zwei Hälften geteilt ist. Auf der linken Seite steht "T_{F,aus}" und auf der rechten Seite "T_{F,ein}". Ein Pfeil zeigt von links nach rechts durch die Mitte des Behälters und ist mit "Q_{aus}" beschriftet.
    \item Das zweite Diagramm zeigt einen ähnlichen Behälter, aber ohne die vertikale Linie in der Mitte. Auf der linken Seite steht "T_{F,ein}" und auf der rechten Seite "T_{F,aus}". Ein Pfeil zeigt von links nach rechts durch die Mitte des Behälters und ist mit "W" beschriftet.
\end{itemize}

\begin{align*}
    \text{stationärer FP mit 1 Massenstrom} \\
    \dot{m} \cdot (h_{ein} - h_{aus}) + Q_{aus} + Q_R \\
    Q_{aus} = \dot{m} \cdot (h_{aus} - h_{ein}) \\
    Q_{aus} = \dot{m} \cdot (h_{ein} - h_{aus}) + \dot{G} \cdot e
\end{align*}

\[
h_{ein} = \frac{A2}{h_{ein}} \cdot h(70^\circ C)
\]

\[
h_{ein} = \frac{A2}{h_{ein}} \cdot h(100^\circ C)
\]

\[
h_{ein} = \frac{A2}{h_{ein}} \cdot h(70^\circ C) = 292.88 \frac{kJ}{kg}
\]

\[
h_{aus} = h(100^\circ C) = 419.04 \frac{kJ}{kg}
\]

\[
Q_{aus} = 0.3 \cdot (292.88 \frac{kJ}{kg} - 419.04 \frac{kJ}{kg}) = -100 \text{ kW}
\]

\[
Q_{aus} = 62.182 \text{ kW}
\]