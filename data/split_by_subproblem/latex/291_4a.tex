\section*{4(a)}

\[ x_2 = 1 \quad x_c = 0 \quad T_k = T_i - 6K \quad T_f = T_{TOP} \]
\[ T_i = T_{TOP} \text{ unter Sublimationsdruck} \]
\[ T_i @ 100 \text{mbar} \]
\[ T_i @ 5 \text{mbar} \]
\[ T_{Sied} @ 5 \text{mbar} = -1^\circ C \]
\[ T_i = 9^\circ C = 282.15K \]

\subsection*{Graph Description}

The graph is a pressure-temperature ($P$-$T$) phase diagram. The vertical axis is labeled $P$ [mbar] and the horizontal axis is labeled $T$ [°C]. 

- The graph shows three distinct regions labeled "fest" (solid), "flüssig" (liquid), and "gas" (gas).
- The "fest" region is at the top left, the "flüssig" region is at the top right, and the "gas" region is at the bottom.
- There is a curve separating the "fest" and "gas" regions, which starts from the top left and curves downwards to the right.
- The point where the "fest", "flüssig", and "gas" regions meet is labeled "Tripel" (triple point) at $0^\circ C$.
- A horizontal dashed line at approximately 5 mbar extends from the "Tripel" point to the right, indicating the sublimation pressure.
- The temperature at the triple point is marked as $T_{Tripel} = 0^\circ C$.
- The graph also shows a horizontal line at $T_i @ 5 \text{mbar}$, indicating the initial temperature at 5 mbar.
- Another horizontal line is drawn at $T_i = 9^\circ C$.