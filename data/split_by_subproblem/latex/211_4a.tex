

\subsection*{a)}

\begin{itemize}
    \item[(i)] 
    \begin{description}
        \item[Graph Description:] The graph is a Pressure-Temperature (P-T) diagram. The x-axis is labeled "T" and the y-axis is labeled "P". There is a curve that starts from the bottom left and rises to the top right, representing the phase boundary between gas and liquid. The region to the left of the curve is labeled "Flüssig" (liquid), and the region to the right is labeled "Gas". There are four points labeled 1, 2, 3, and 4, with arrows indicating transitions between these points. The text "Alle Zustände sind im Dreiphasengebiet" is written below the graph, indicating that all states are in the three-phase region.
    \end{description}
    
    \item[(ii)] 
    \begin{description}
        \item[Graph Description:] The graph is another Pressure-Temperature (P-T) diagram. The x-axis is labeled "T" and the y-axis is labeled "P". There is a curve that starts from the bottom left and rises to the top right, representing the phase boundary between gas and liquid. The region to the left of the curve is labeled "Flüssig" (liquid), and the region to the right is labeled "Gas". There are two points labeled 1 and 2, with an arrow indicating a transition between these points. The text "ΔT=10K" is written below the graph, indicating a temperature difference of 10K.
    \end{description}
\end{itemize}