

\section*{c)}
\text{Annahme:} \quad \dot{m} = 4 \frac{\text{kg}}{\text{h}}, \quad T_{se} = 52^\circ \text{C} \\
\text{ges} \quad x_1 \\

\text{Aus T\&B A-10} \quad h_{g}(52^\circ \text{C}) = 234,28 \frac{\text{kJ}}{\text{kg}} \\
\text{(wir wissen h}_4 = 93,49 \frac{\text{kJ}}{\text{kg}}) \\
\text{und aus T\&B A-10} \quad p_{4}(52^\circ \text{C}) = 1,24992 \text{bar} = p_2 \\

\text{h}_2 \text{ und h}_3 \text{ aus T\&B A-10} \\
h_{g2} = 234,28 \frac{\text{kJ}}{\text{kg}} \\
h_{f2} = 93,49 \frac{\text{kJ}}{\text{kg}} \\

x_1 = \frac{h_1 - h_{f2}}{h_{g2} - h_{f2}} \quad \text{mit} \quad h_4 = 93,49 \\

\Rightarrow x_1 = 0,3375 \\