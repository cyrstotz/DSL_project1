

\subsection*{a)}

\begin{description}
    \item[Graph Description:] The graph is a Pressure-Temperature ($P$-$T$) diagram. The x-axis is labeled $T$ (Temperature) and the y-axis is labeled $P$ (Pressure). The graph shows a curve starting from the origin and rising upwards. The curve is divided into three regions:
    \begin{itemize}
        \item The region below the curve is labeled "Fest" (solid).
        \item The region on the curve is labeled "isotherm" (isothermal) and "isobar" (isobaric).
        \item The region above the curve is labeled "Flüssig" (liquid).
    \end{itemize}
    There are three points marked on the curve:
    \begin{itemize}
        \item Point 1 is at the top right of the curve.
        \item Point 2 is at the middle of the curve.
        \item Point 3 is at the bottom left of the curve.
    \end{itemize}
    The region between points 2 and 3 is labeled "Gas".
\end{description}