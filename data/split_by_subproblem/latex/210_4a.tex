\section*{4. R 134a}

\subsection*{Graph 1}
The first graph is a phase diagram with pressure \( P \) on the y-axis (labeled in \([ \text{bar} ]\)) and temperature \( T \) on the x-axis (labeled in \([ \text{K} ]\)). The graph shows three distinct regions labeled as "fest" (solid), "flüssig" (liquid), and "gas" (gas). 

- The "fest" region is on the left side of the graph.
- The "flüssig" region is in the middle.
- The "gas" region is on the right side.

There are three lines separating these regions:
- The line between "fest" and "flüssig" is sloped upwards.
- The line between "flüssig" and "gas" is also sloped upwards but less steeply.
- The line between "fest" and "gas" is curved and intersects the other two lines at a point labeled "Tripel" (triple point).

\subsection*{Graph 2}
The second graph is another phase diagram with pressure \( P \) on the y-axis (labeled in \([ \text{bar} ]\)) and temperature \( T \) on the x-axis (labeled in \([ \text{K} ]\)). This graph also shows three regions labeled "fest" (solid), "flüssig" (liquid), and "gasförmig" (gaseous).

- The "fest" region is on the left side of the graph.
- The "flüssig" region is in the middle.
- The "gasförmig" region is on the right side.

There are three lines separating these regions:
- The line between "fest" and "flüssig" is sloped upwards.
- The line between "flüssig" and "gasförmig" is also sloped upwards but less steeply.
- The line between "fest" and "gasförmig" is curved and intersects the other two lines at a point labeled "Tripel" (triple point).

Additionally, there are two points labeled \( x_1 \) and \( x_2 \) on the "fest" to "flüssig" line, with a vertical arrow labeled \( i \) pointing upwards from \( x_2 \) to \( x_1 \). There is also a horizontal arrow labeled \( ii \) pointing from \( x_1 \) to a point labeled \( O \) on the "flüssig" to "gasförmig" line.

\subsection*{Graph 3}
The third graph is another phase diagram with pressure \( P \) on the y-axis (labeled in \([ \text{bar} ]\)) and temperature \( T \) on the x-axis (labeled in \([ \text{K} ]\)). This graph shows three regions but without labels.

- The leftmost region is separated by a line sloped upwards.
- The middle region is separated by a line sloped upwards but less steeply.
- The rightmost region is separated by a curved line.

The lines intersect at a point, forming a triple point similar to the previous graphs.

``````latex