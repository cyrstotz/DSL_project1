

\subsection*{a)}

\begin{center}
\textbf{Verbal Description of the Graph:}

The graph is a phase diagram with pressure \( P \) on the y-axis and temperature \( T \) on the x-axis. The y-axis is labeled with \( P \) in \([ \text{bar} ]\) and \([ \text{Pa} ]\). The x-axis is labeled with \( T \) in \([ K ]\). 

The graph shows three regions labeled "Fest" (solid), "Flüssig" (liquid), and "Gas" (gas). The boundary between the solid and liquid regions is a curve starting from the y-axis and moving upwards to the right. The boundary between the liquid and gas regions is another curve starting from the x-axis and moving upwards to the right, ending at a point labeled "Kritischer Punkt" (critical point). 

There is a point labeled "Tripel Punkt" (triple point) where the three regions meet. 

Two processes are shown on the graph:
1. A horizontal line labeled "IsoTherm" (isothermal) from point I to point II.
2. A vertical line labeled "IsoTherm druckabnahme" (isothermal pressure decrease) from point III to point II.

\end{center}