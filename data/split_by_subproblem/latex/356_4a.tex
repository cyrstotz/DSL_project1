

\subsection*{a)}

\begin{description}
    \item[Graph Description:] The graph is a Pressure-Temperature (P-T) diagram. The vertical axis is labeled \( P \) with the unit \([mbar]\) and ranges from 0 to 10. The horizontal axis is labeled \( T \) with the unit \([^\circ C]\) and ranges from 0 to an unspecified value. The graph contains the following elements:
    \begin{itemize}
        \item A horizontal dashed line at \( P = 10 \) mbar.
        \item A vertical dashed line at \( T = 0 \) \( ^\circ C \).
        \item A point labeled (i) at the intersection of \( P = 10 \) mbar and \( T = 0 \) \( ^\circ C \).
        \item A point labeled (ii) at \( P = 2 \) mbar and \( T = 0 \) \( ^\circ C \).
        \item A point labeled (iii) at \( P = 3 \) mbar and \( T = 0 \) \( ^\circ C \).
        \item A horizontal line labeled "isotherm" at \( P = 3 \) mbar.
        \item A vertical line labeled "isobar" at \( T = 0 \) \( ^\circ C \).
        \item A point labeled "triple point" at the intersection of the isotherm and isobar lines.
        \item The region to the left of the isobar is labeled "Eis" (ice).
        \item The region to the right of the isobar is labeled "Wasser" (water).
        \item The region above the isotherm is labeled "Gas" (gas).
    \end{itemize}
\end{description}

```