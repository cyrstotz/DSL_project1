

\subsection*{a)}

\begin{align*}
X & = 70 \\
P_{\text{Wasser}} & = 139 \cdot 969.54 \, \text{Pa}
\end{align*}

Der Druck wird gleich hoch sein, da die Höhe gleich groß ist vom Einwasser.

Die Temperatur des Gases wird der Temperatur des Einwassers entsprechen, da im thermodynamischen Gleichgewicht Temperaturen gleich sind.

\begin{align*}
T_{\text{sat}} (7.400 \, \text{bar}) & = 0.000 \degree C = T_{\text{a,2}} - T_{\text{em,2}} \\
P_{\text{g,2}} - P_{\text{g,1}} & = -700 \cdot 694.04 \, \text{Pa}
\end{align*}