b) \quad \text{Mit } p_{g1} = 7.5 \, \text{bar und } m_g = 3 \, \text{kg} \text{ weiterrechnen.}

\[
\text{GS: } x_{g1,2} \quad T_{g1,2} \quad p_{g1,2} \quad \text{GG: } x_{g1,2} > 0
\]

\text{Setze voraus, dass das Eis mittel komplett geschmolzen ist, bleibt:}

\[
T_{g1,2} < T_{g1,1} \quad \text{dn Temperaturausgleich bis } T_{ev,1} \approx T_{g1,2} = 0^\circ \text{C}
\]

\[
\text{weil } V_{g1,2} \text{ sinkt, steigt } p_{g1,2} \quad \text{kräften/Drucken}
\]

\[
p_{g1,2} = p_{g1,1} \quad \text{dn keine zusätzlichen Wirkungen bei Zustand 2.}
\]