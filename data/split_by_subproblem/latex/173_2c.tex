c)
\begin{align*}
    T_1 &= 773.15 \, \text{K} \\
    T_2 &= 273.15 \, \text{K} \\
    p_1 &= 101000 \, \text{Pa} \\
    p_2 &= 101000 \, \text{Pa}
\end{align*}

\begin{align*}
    \frac{dU}{dt} &= \sum \dot{m}(i) + \sum \dot{Q} - \sum \dot{w}
\end{align*}

\begin{align*}
    \Delta u &= Q - w
\end{align*}

\begin{align*}
    Q_{12} &= \Delta u + W_{12}
\end{align*}

\begin{align*}
    W_{12} &= \int_{V_1}^{V_2} p \, dV
\end{align*}

\begin{align*}
    pV &= mRT \\
    V_2 &= \frac{mRT_2}{p_2} = \frac{0.2922 \, \text{kg} \cdot 0.16628 \, \frac{\text{kJ}}{\text{kg} \cdot \text{K}} \cdot 273.15 \, \text{K}}{101000 \, \text{Pa}} = 9.474 \cdot 10^{-5} \, \text{m}^3
\end{align*}

\begin{align*}
    W_{12} &= \int_{V_1}^{V_2} p \, dV = p \int_{V_1}^{V_2} dV = p (V_2 - V_1) = 101000 \, \text{Pa} \left(9.474 \cdot 10^{-5} \, \text{m}^3 - 0.00314 \, \text{m}^3 \right)
\end{align*}

\begin{align*}
    W_{12} &= -426.64 \, \text{J}
\end{align*}

\begin{align*}
    \Delta u &= C_v (T_2 - T_1) = 0.633 \, \frac{\text{kJ}}{\text{kg} \cdot \text{K}} (273.15 \, \text{K} - 773.15 \, \text{K}) \\
    \Delta u &= -319.5 \, \text{J}
\end{align*}

\begin{align*}
    Q_{12} &= \Delta u + W_{12} = -319.5 \, \text{J} - 426.64 \, \text{J} = -746.14 \, \text{J}
\end{align*}

\begin{align*}
    W_{12} &= \int_{V_1}^{V_2} p \, dV
\end{align*}

\begin{align*}
    W_{12} &= p (V_2 - V_1) = 101000 \, \text{Pa} \left(9.474 \cdot 10^{-5} \, \text{m}^3 - 0.00314 \, \text{m}^3 \right) \\
    W_{12} &= -426.64 \, \text{J}
\end{align*}

\begin{align*}
    Q_{12} &= \Delta u + W_{12} = -746.14 \, \text{J}
\end{align*}

\underline{Graphical Description:}

There is a diagram of a piston-cylinder device. The cylinder is drawn as a rectangle with a horizontal line at the top representing the piston. Inside the cylinder, there is a shaded area labeled as $V_2$. An arrow pointing upwards next to the piston indicates the work done by the gas, labeled as $w$.

``````latex


\section*{Aufgabe 3}