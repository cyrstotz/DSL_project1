b)

\[
T_{SU,2} = T_{EW,1} = 0^\circ \text{C}
\]

\[
p_{g,1} = p_{g,2} \approx 1.40094 \, \text{bar}
\]

\text{Es herrscht thermisches Gleichgewicht in Zustand 2.}

\text{Es herrscht weiterhin Kräftegleichgewicht.}

``````latex


3.c) Perfektes Gas \& Isobar:

\[
m(h_2 - h_1) = Q_{12}
\]

\[
m(C_v + R) \Delta (T_2 - T_1) = Q_{12}
\]

\[
Q_{12} = -1.367 \, \text{kJ}
\]

\textbf{Systemgrenze Gas:}

There is a rectangular box divided into two horizontal sections. The upper section is labeled "EW" and the lower section is labeled "Gas". An arrow labeled "Q_{12}" points upwards from the lower section (Gas) to the upper section (EW).