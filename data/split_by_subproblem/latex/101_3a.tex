\section*{4 a)}

\begin{tabular}{|c|c|c|c|c|c|c|}
\hline
P & T & V & W & \omega & \Omega & \dot{Q} \\
\hline
1 & $<$ 8 bar & & & & & $\dot{Q}$ \\
\hline
$P_1 = P_2$ & $T_i = T_{-6K}$ & & $\dot{W}$ & & & \\
\hline
8 bar & & & & & & \\
\hline
8 bar & & & & & & \\
\hline
\end{tabular}

\begin{itemize}
    \item Gasförmig, gesättigt
    \item flüssig
\end{itemize}

$p_1 = p_2 \quad S_2 = S_3$

\subsection*{Graph Descriptions}

\textbf{First Graph:}

The first graph is a pressure-temperature ($p$-$T$) diagram. The x-axis is labeled $T$ (kelvin) and the y-axis is labeled $p$ (bar). There is a saturation curve that starts at the origin, rises to a peak, and then falls back down. Two points are marked on the curve, labeled 1 and 2, connected by a horizontal line. The point 1 is on the left side of the peak, and point 2 is on the right side of the peak. The region to the left of the curve is labeled "sättigungslinie".

\textbf{Second Graph:}

The second graph is also a pressure-temperature ($p$-$T$) diagram. The x-axis is labeled $T$ (kelvin) and the y-axis is labeled $p$ (bar). There is a saturation curve similar to the first graph. Four points are marked on the curve, labeled 1, 2, 3, and 4. Points 1 and 2 are connected by a horizontal line labeled "1 bar". Points 3 and 4 are on the left side of the curve, with point 3 below point 4.

\textbf{Third Graph:}

The third graph is a temperature-entropy ($T$-$s$) diagram. The x-axis is labeled $T$ (K) and the y-axis is labeled $s$ (bar). There is a dome-shaped curve. Four points are marked on the curve, labeled 1, 2, 3, and 4. Points 1 and 2 are connected by a horizontal line labeled "isobar". Points 3 and 4 are connected by a vertical line labeled "isentrope". The region under the dome is labeled "isobar" and "isentrope".

``````latex