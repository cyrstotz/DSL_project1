

\item[a)] Von der Aufgabenstellung wissen wir, dass es sich beim Kühlmittel um eine ideale Flüssigkeit handelt.
    
    \[
    \frac{dE'}{dt} = \sum_i \left[ m_i \left( h_i + \frac{ke_i + pe_i}{m_i} \right) \right] + \sum_j Q_j - \sum_k W_k
    \]
    
    \[
    0 = m_e \left[ h_{ein} - h_{aus} \right] + Q_{ab}
    \]
    
    \[
    Q_{ab} = m_{ein} \left[ h_{aus} - h_{ein} \right]
    \]
    
    \[
    h_{ein} = h_f (70^\circ C) = 292.98 \frac{\text{kJ}}{\text{kg}} \quad \text{(von TAB A-2)}
    \]
    
    \[
    h_{aus} = h_f (100^\circ C) = 419.04 \frac{\text{kJ}}{\text{kg}} \quad \text{(von TAB A-2)}
    \]
    
    \[
    Q_{ab} = 0.3 \frac{\text{kg}}{\text{s}} \left[ 419.04 \frac{\text{kJ}}{\text{kg}} - 292.98 \frac{\text{kJ}}{\text{kg}} \right] = \underline{37.82 \text{ kW}}
    \]