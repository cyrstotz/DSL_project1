

\subsection*{d)}

\begin{align*}
    X_{ei} &= \frac{m_{ei}}{m_{tot}} \\
    m_{tot} &= a \cdot l \cdot h \\
    \Delta U_{tot} &= 0 \\
    \Delta U_{1} &= \Delta U_{2} \\
    \Delta U_{1} &= 3,16 \cdot 478 + U_{1,tot} = U_{2,tot} \\
    - 83,442 &= - \Delta q_{1} + U_{EW} = U_{LEW} \\
    q_{2} &= -3,16 \cdot 478 + 7 - 0,045 + 0,61 - 333 \cdot (158 + 0,045) = - 540,56 \frac{kJ}{kg} \\
    U_{2,EW} - U_{1,t} &= X_{ei} \\
    U_{ext} - U_{t} &= - 540,56 + 0,083 \\
    &= - 333,442 + 0,053 \\
    \Delta U &\Rightarrow \\
    \Delta U_{ei} &= G_{12} + \dot{V}_{12} \\
    \Delta \dot{V}_{12} - U_{1} &= \frac{1500J}{m} \Rightarrow U_{2} = \frac{1500J}{m} + U_{1} \\
    U_{2} &= U_{FL} + X_{2} (U_{fg} - U_{FL}) \\
    U_{1} &= - 0,1045 + 0,61 \cdot (-333) \cdot (158 + 0,045) = - 200,083 \\
    X_{2} &= \frac{G}{m} + U_{1} - U_{FL} = \frac{- 1500J}{0,14} - 200,083 + 0,1053 \\
    \frac{U_{E} - U_{FL}}{U_{E} - U_{t}} &= - 333,442 + 0,1053 \\
    &\Rightarrow X_{ei,2} \approx 1 \text{ höher als } 0,6 \text{ sei } U_{1} \text{ da Temperatur steigt.}
\end{align*}

``````latex