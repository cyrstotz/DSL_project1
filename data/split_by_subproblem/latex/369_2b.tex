

\subsection*{b)}

\begin{align*}
    T_S &= 437.9 K \\
    w_s &= 220 \frac{m}{s} \\
    p_S &= 0.5 bar \\
    m_s &= \text{ingeo} \\
    \\
    h_s &= c_p \cdot T_S \quad (\text{ideales gas}) \\
    h_s &= 7.006 \frac{kJ}{kg \cdot K} \cdot 437.9 K = 43.69 \frac{kJ}{K} \\
    \\
    \text{adiabat:} \quad T_0 &= T_S \left( \frac{p_0}{p_S} \right)^{\frac{R}{c_p}} = 437.9 K \left( \frac{0.1}{0.5 bar} \right)^{\frac{0.19 \frac{kJ}{kg \cdot K}}{0.4 \frac{kJ}{kg \cdot K}}} = 325.07 K \\
    \\
    h_0 &= c_p \cdot T_0 = 330 \frac{kJ}{kg} \quad \Rightarrow \quad \text{Gesamtsystem:} \\
    \\
    \text{Für } \dot{Q} &= \dot{m} \left( h_0 - h_6 + \frac{w_2^2 - w_1^2}{2} \right) + Q \\
    \\
    h_0 &= c_p \cdot 263.15 K = 2.4 \frac{kJ}{kg}
\end{align*}

\begin{description}
    \item[Additional Notes:] 
    \begin{itemize}
        \item The student mentions "Schubdüse Adiabat: Stationär" and "reversible Schubdüse: \(\dot{S}_{erzeugt} = 0\)".
        \item There are notes about "adiabat" and "reversible (adiabat)" processes.
        \item The student also writes "V. dp c_v (p_0 - p_3)" and "c_v p_0 - p_3".
    \end{itemize}
\end{description}

``````latex