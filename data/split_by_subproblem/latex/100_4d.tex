\subsection*{Part (d)}

\begin{align*}
    e_{k} &= \frac{\dot{Q}_{zu}}{\dot{Q}_{ab} - \dot{Q}_{zu}} = \frac{Q_{zu}}{W_E} = \frac{Q_K}{W_k} \\
    Q_{k} &= m(h_2 - h_1) \\
    Q_{c} &= m \cdot \frac{G(40)}{60^2 \text{s}} \cdot (251,8 - 95,42) \\
    &= 0,76 \text{ kJ}
\end{align*}

``````latex


\begin{figure}[h]
    \centering
    % Description of the graph
    The graph is a plot on a grid paper with the horizontal axis labeled as \( T \) and the vertical axis labeled as \( P \). The graph contains a curve that starts from the bottom left, rises to a peak labeled \( T_{\text{krit}} \), and then descends towards the bottom right. There are four points marked on the graph, labeled 1, 2, 3, and 4. 
    
    - Point 1 is located on the left side of the curve, below the peak.
    - Point 2 is on the right side of the curve, at the same height as point 1.
    - Point 3 is above point 2, on a smaller curve that branches off from the main curve.
    - Point 4 is on the left side of the curve, at the same height as point 3.
    
    There is a horizontal line passing through points 1 and 2. Another line branches off from point 2 and goes upwards, passing through point 3. The label "150 bar" is written near the line that passes through point 3.
\end{figure}

``````latex


\[
E = \frac{176 \, \text{kWh}}{28 \, \text{W}} = 6,2857
\]

\begin{itemize}