\subsection*{a) T-s Diagramm}

\begin{itemize}
    \item The first graph is a T-s diagram with the y-axis labeled as \( T \) and the x-axis labeled as \( s \left[ \frac{kJ}{kg} \right] \). 
    \item The graph contains a closed loop with six points labeled 0, 1, 2, 3, 5, and 6.
    \item The path from point 0 to point 1 is labeled as "isobar".
    \item The path from point 1 to point 2 is labeled as "isobar".
    \item The path from point 2 to point 3 is labeled as "isobar".
    \item The path from point 3 to point 5 is labeled as "isentrop".
    \item The path from point 5 to point 6 is labeled as "isobar".
    \item The path from point 6 to point 0 is labeled as "isobar".
    \item The graph also contains lines labeled \( p_2 = p_3 \), \( p_5 \), and \( p_0 = p_6 \).
\end{itemize}

\begin{itemize}
    \item The second graph is a T-s diagram with the y-axis labeled as \( T \) and the x-axis labeled as \( s \left[ \frac{kJ}{kg} \right] \). 
    \item The graph contains a closed loop with six points labeled 0, 1, 2, 3, 4.5, and 6.
    \item The path from point 0 to point 1 is labeled as "isobar".
    \item The path from point 1 to point 2 is labeled as "isobar".
    \item The path from point 2 to point 3 is labeled as "isobar".
    \item The path from point 3 to point 4.5 is labeled as "isentrop".
    \item The path from point 4.5 to point 6 is labeled as "isobar".
    \item The path from point 6 to point 0 is labeled as "isobar".
    \item The graph also contains lines labeled \( p_2 = p_3 \), \( p_4 = p_5 \) (which is higher than \( p_0 = p_6 \)), and \( p_0 = p_6 \).
\end{itemize}

\begin{itemize}
    \item The third graph is a simplified version of the second graph with the same points labeled 0, 1, 2, 3, 4.5, and 6.
    \item The paths between the points are the same as in the second graph.
\end{itemize}

``````latex