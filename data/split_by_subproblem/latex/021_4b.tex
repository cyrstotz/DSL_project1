

\subsection*{b)}

\begin{align*}
1 \rightarrow 2 &: \text{isobare Verdampfung} \quad p_1 = p_2 \quad x_1 = 1 \rightarrow \text{komplett verdampft} \\
2 \rightarrow 3 &: \text{adiab. isochor} \quad s_2 = s_3 \\
3 \rightarrow 4 &: \text{isobar} \quad p_3 = p_4 \quad x_4 = 0 \\
0 &= \dot{m} \left( h_2 - h_3 \right) - \dot{W}_k \\
h_3 &= h_2 (p = 8 \, \text{bar}) = 2694{,}15 \, \frac{\text{kJ}}{\text{kg}} \\
s_1 = s_2 &= s_g (p = 8 \, \text{bar}) = 0{,}906 \, \frac{\text{kJ}}{\text{kg} \cdot \text{K}} \\
p_4 &= 8 \, \text{bar} \\
h_4 &= h_f (p = 8 \, \text{bar}) = 53{,}42 \, \frac{\text{kJ}}{\text{kg}} \\
s_f (p = 8 \, \text{bar}) &= s_f = 0{,}3455 \, \frac{\text{kJ}}{\text{kg} \cdot \text{K}}
\end{align*}

```