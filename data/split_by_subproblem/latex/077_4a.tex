

\subsection*{a)}

\begin{itemize}
    \item \textbf{Graph Description:} The graph is a phase diagram with pressure \( p \) on the y-axis and temperature \( T \) on the x-axis. The y-axis is labeled with "mbar" and has a value of 5 mbar marked. The x-axis is labeled with "°C". There are three regions labeled "solid", "liquid", and "gas". The "solid" region is on the left, the "liquid" region is in the middle, and the "gas" region is on the right. The boundary lines between these regions are curved. The triple point is marked where all three regions meet. There are three points labeled (1), (2), and (3) on the graph. Point (1) is in the "solid" region, point (2) is on the boundary between "solid" and "gas", and point (3) is in the "gas" region. Arrows indicate transitions between these points.
\end{itemize}

1) isobar geforen: \( p \) constant \\
   \( T \) geht runter \\
   Zustand: solid

2) isotherm sublimiert: \( p \) geht runter \\
   Zustand: gas \\
   \( T \) ist constant