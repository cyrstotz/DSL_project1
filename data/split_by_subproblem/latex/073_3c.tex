c) 1. HS um das Gas:
\begin{equation}
\frac{dE}{dt} = \sum_i \dot{m}_i (h_i + \frac{p_i}{\rho_i} + \frac{v_i^2}{2}) + \dot{Q} - \dot{V}
\end{equation}

\begin{equation}
\Delta U = Q_{12}
\end{equation}

\begin{equation}
m_{gas} (u_2 - u_1) = Q_{12}
\end{equation}

\begin{equation}
Q_{12} = m_{Cu} c_{u} (T_2 - T_1) \quad \text{(perfekter Gas)}
\end{equation}

\begin{equation}
Q_{12} = 3.6 \cdot 10^3 \, \text{kg} \cdot 0.633 \, \frac{\text{kJ}}{\text{kgK}} (273.15 \, \text{K} - 773.15 \, \text{K})
\end{equation}

\begin{equation}
= -1.139.4 \, \text{kJ} = -1.139.4 \, \text{J}
\end{equation}

\begin{equation}
\Rightarrow \text{Es wurde 1.139.4 J übertragen.}
\end{equation}