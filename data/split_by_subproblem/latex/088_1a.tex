

\subsection*{a)}
Energiegleichung um den Reaktor:
\[
\dot{S}_{1,2} = 0
\]
\[
\frac{dE}{dt} = \sum \dot{m} (h + \frac{pe}{\rho}) + \sum \dot{Q} - \sum \dot{W} = 0
\]
\[
\dot{m} (h_{\text{ein}} - h_{\text{aus}}) + \dot{Q}_2 - \dot{Q}_{\text{aus}} = 0
\]
\[
\dot{Q}_{\text{aus}} = \dot{Q}_2 + \dot{m} (h_{\text{ein}} - h_{\text{aus}}) = 62.182 \frac{\text{kJ}}{\text{kg}}
\]
\[
\dot{Q}_2 = 100 \text{kW}
\]
\[
\begin{cases}
h_{\text{ein}} = h_1(200^\circ \text{C}) = 292.88 \frac{\text{kJ}}{\text{kg}} \\
h_{\text{aus}} = h_1(100^\circ \text{C}) = 418.04 \frac{\text{kJ}}{\text{kg}}
\end{cases}
\]