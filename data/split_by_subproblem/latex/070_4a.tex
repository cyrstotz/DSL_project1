

\subsection*{a)}

\begin{align*}
    T_{\text{tripel}} &= 0^\circ C \\
    T_{\text{sub}} (\text{Motor}) &= -20^\circ C \\
    \text{Somit liegt } T_i \text{ mit 10K über } T_{\text{sub}} \text{ bei } T_i &= -10^\circ C
\end{align*}

\subsubsection*{Graph Description}

The graph is a pressure-temperature ($P$-$T$) diagram. The x-axis is labeled $T$ and the y-axis is labeled $P$. There is a curve starting from the origin and curving upwards to the right, labeled "Fest-Flüssig" (solid-liquid). Another curve starts from the same origin and curves upwards to the left, labeled "Fest" (solid). The point where these two curves meet is labeled "Tripelpunkt" (triple point). Above the triple point, there is a region labeled "Gas" (gas). There is a vertical arrow pointing upwards from the triple point labeled "Sublimation" (sublimation). There is also a horizontal arrow pointing to the right from the triple point labeled "isotherm" (isothermal).