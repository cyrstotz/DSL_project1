

\subsection*{a)}

\begin{center}
\textbf{Graph Description:}
\end{center}

The graph is a plot with the vertical axis labeled \( T(K) \) and the horizontal axis labeled \( S \left( \frac{kJ}{kgK} \right) \). The plot contains a curve that starts at point 0 and moves through points 1, 2, 3, 4, 5, and 6. The curve has the following characteristics:
- From point 0 to point 1, the curve is a straight line.
- From point 1 to point 2, the curve is a steep upward line.
- From point 2 to point 3, the curve is a horizontal line.
- From point 3 to point 4, the curve is a steep upward line.
- From point 4 to point 5, the curve is a horizontal line.
- From point 5 to point 6, the curve is a steep upward line.

The points are labeled as follows:
- Point 0 is at the origin.
- Point 1 is at a higher temperature than point 0.
- Point 2 is at a higher temperature than point 1.
- Point 3 is at a higher temperature than point 2.
- Point 4 is at a higher temperature than point 3.
- Point 5 is at a higher temperature than point 4.
- Point 6 is at a higher temperature than point 5.

Below the graph, there is a table with the following values:

\[
\begin{array}{ccccccc}
0 & 1 & 2 & 3 & 4 & 5 & 6 \\
P & 0.191 & = & = & = & = & = \\
T & -30 & & & & & & \\
\end{array}
\]