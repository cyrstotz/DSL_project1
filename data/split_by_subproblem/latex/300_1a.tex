a) Die Temperatur muss nicht ansteigen, stationärer Fließprozess.

\[
0 = \dot{m} (\text{h}_{\text{ein}} - \text{h}_{\text{aus}}) + \dot{Q} + \dot{Q}_{\text{aus}}
\]

\[
\dot{Q}_{\text{aus}} = \dot{m} (\text{h}_{\text{aus}} - \text{h}_{\text{ein}}) - \dot{Q}
\]

Siedende Flüssigkeit als reine Wasser entnommen

\[
\text{h}_{\text{aus}} = \text{h}_{\text{fg}} (100^\circ \text{C}) = 2257.0 \frac{\text{kJ}}{\text{kg}}
\]

\[
\text{h}_{\text{ein}} = \text{h}_{\text{fg}} (70^\circ \text{C}) = 2333.8 \frac{\text{kJ}}{\text{kg}}
\]

\[
\dot{Q} = 100 \text{kW}
\]

\[
\dot{Q}_{\text{aus}} = -123.04 \text{kW}
\]