d) Im Reaktor: \quad \Delta U = Q_{12} \checkmark \\
Q_{12} = m \cdot (u_2 - u_1) \\

u_1: \quad 100^\circ C, \quad x_0 = 0.005 \quad \text{Tab A2:} \quad (1-x) \cdot u_f + x \cdot u_g \\
\quad = 0.995 \cdot 418.34 \frac{\text{kJ}}{\text{kg}} + 0.005 \cdot 2506.5 \frac{\text{kJ}}{\text{kg}} \\
\quad = 423.38 \frac{\text{kJ}}{\text{kg}} \\

u_2: \quad 70^\circ C, \quad \text{satt. Flüssigkeit} \quad \text{Tab A2:} \quad u_2 = 292.35 \frac{\text{kJ}}{\text{kg}} \\

Q_{12} = 5735 \text{kg} \left( 292.35 \frac{\text{kJ}}{\text{kg}} - 429.38 \frac{\text{kJ}}{\text{kg}} \right) \\
\quad = -785154 \text{kJ} \\

\noindent\rule{8cm}{0.4pt}

\frac{1}{\dot{m}} \cdot -Q_{12} = \Delta u_{12} (u_2 - u_3) \\
\Delta m_{12} = \frac{-Q_{12}}{(u_2 - u_3)} \\

u_3 = \text{siedend} \quad 20^\circ C \quad \text{A2:} \quad 83.85 \frac{\text{kJ}}{\text{kg}} \\

\Delta m_{12} = \frac{-Q_{12}}{(292.35 - 83.85) \frac{\text{kJ}}{\text{kg}}} = \frac{785154 \text{kJ}}{209 \frac{\text{kJ}}{\text{kg}}} = 3756.71 \text{kg} \\