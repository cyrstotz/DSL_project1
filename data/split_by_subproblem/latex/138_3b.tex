

\subsection*{b)}

Das Gas hat Wärme \( Q \) aufgenommen um einen Teil des Eis zu schmelzen. Daher \( T_{g,2} = T_{g,1} \Rightarrow p_{g,2} < p_{g,1} \). Das Volumen und Masse des Gases gleich bleiben. Im Gleichgewichtszustand sind \( T_{eq,1} = T_{eq,2} \).

\[
\dot{E} = \frac{d}{dt} (PE) + \frac{d}{dt} (KE) + \frac{d}{dt} (U) = \dot{Q} - \dot{W} \quad \Rightarrow \quad \Delta U = Q
\]

\noindent
Potential/kinetic energy of the gas is zero. Internal energy depends only on temperature, so the temperature change of the gas is the same as the system. Energy balance of the entire system:

\[
\dot{E} = \frac{d}{dt} (PE) + \frac{d}{dt} (KE) + \frac{d}{dt} (U) = \dot{Q} - \dot{W} \quad \Rightarrow \quad \Delta U = Q - W
\]

\noindent
The change in internal energy of the ice water is equal to the heat plus the work done on the system:

\[
\Delta U = Q + p_{eq,1} \cdot (V_{eq,2} - V_{eq,1})
\]

\noindent
\textbf{Note:} \( V_{eq,2} = V_{eq,1} \) and the ice water is incompressible. Therefore, the heat transferred from the gas to the ice water is:

\[
\Delta U = Q
\]

\noindent
\textbf{System Eis:}
\begin{align*}
U_{eis}(0^\circ C)```latex

\section*{Question 4}

\begin{equation*}
Q_{12} = m_{12} \cdot C_v \cdot (T_2 - T_1) \approx -1092.42 \, J
\end{equation*}

\[
\left[ \frac{kg \cdot \frac{KJ}{kg \cdot K} \cdot K}{J} \right]
\]

\[
\text{with} \quad T_2 = 0.0003^\circ C
\]

``````latex