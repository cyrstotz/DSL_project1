a) \textbf{Energiebilanz um Reaktor}

\[
\frac{dE}{dt} = \dot{m}_{ein}(h_{ein} - h_{aus}) + \dot{Q}_{aus} - \sum \dot{W} + \dot{Q}_R
\]

\[
\dot{Q}_{aus} = \dot{m}_{ein}(h_{aus} - h_{ein})
\]

\[
h_{ein} = h(T=100^\circ C, x=1) = 2626,8 \frac{kJ}{kg}
\]

\[
h_{aus} = h(T=100^\circ C, einp) = 2257,0 \frac{kJ}{kg}
\]

\[
\dot{Q}_{aus} = 0,3 \frac{kg}{s} (2257,0 - 2333,8) + 100 kW = 76,96 kW
\]



\section*{a)}

\begin{equation*}
\frac{dE}{dt} = \dot{m}_{12} (h_{\text{ein}}) + Q_{\text{aus},12}
\end{equation*}

\begin{equation*}
u_2 - u_1 = \dot{m}_{12} (h_{\text{ein}}) - Q_{\text{aus},12}
\end{equation*}

\begin{equation*}
-(m_{\text{ges},1} + \dot{m}_1) (u_2) - m_{\text{ges},1} (u_1) = \dot{m}_{12} h_{\text{ein}} - Q_{\text{aus},12}
\end{equation*}

\begin{equation*}
- \dot{m}_{12} = \frac{m_{\text{ges},1} u_1 - m_{\text{ges},1} u_2 - Q}{u_2 - h_{\text{ein}}}
\end{equation*}

\begin{equation*}
h_{\text{ein}} = h(20^\circ C, x=1) = 2538,1 \frac{\text{kJ}}{\text{kg}}
\end{equation*}

\begin{equation*}
u_1 = h(105^\circ C, x=1) = 2676,1 \frac{\text{kJ}}{\text{kg}}
\end{equation*}

\begin{equation*}
u_2 = h(70^\circ C, x=1) = 2626,8 \frac{\text{kJ}}{\text{kg}}
\end{equation*}

\begin{equation*}
- \dot{m}_{12} = \frac{5755 \cdot 2676,1 - 5755 \cdot 2626,8}{2626,8 - 2538,1} = 100 \frac{\text{kg}}{\text{s}}
\end{equation*}