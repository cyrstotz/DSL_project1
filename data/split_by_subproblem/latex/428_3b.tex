

\subsection*{b)}

Das Gas im Zylinder betrachtet als perfektes Gas, mit $M_g = 10 \, \frac{\text{kg}}{\text{kmol}}$.

\begin{align*}
R^* &= \frac{R}{M} = \frac{8.314 \, \frac{\text{J}}{\text{mol} \cdot \text{K}}}{10 \, \frac{\text{kg}}{\text{kmol}}} = 0.8314 \, \frac{\text{J}}{\text{kg} \cdot \text{K}} \\
    &= 0.8314 \, \frac{\text{J}}{\text{kg} \cdot \text{K}}
\end{align*}

\begin{align*}
M &= \frac{p \cdot V}{R \cdot T} = \frac{3.14 \, \text{L} \cdot 1.46 \, \text{bar}}{166.58 \, \frac{\text{J}}{\text{kg} \cdot \text{K}} \cdot 775.15 \, \text{K}} = 3.14 \, \text{g}
\end{align*}

``````latex



\section*{b)}

\begin{itemize}
    \item \underline{Initial} $T_{Eis} > 0 \Rightarrow$ es gibt noch Eis im Eis-Wasser-Gemisch $\Rightarrow T_{Eis,2} = 0^\circ C$.
    \item $\Rightarrow T_{gz,2} = 0^\circ C$.
    \item \underline{Fazit:} Da die Sättigung mit Kolben bleiben unverändert $\Rightarrow P_{S,1} = P_{S,2} \Rightarrow 7.4 \text{ bar}$.
\end{itemize}