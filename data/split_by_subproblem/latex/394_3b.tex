

\subsection*{b)}

\begin{align*}
T_{a,1}, P_{a,2}
\end{align*}

Zustand 1 \& 2 stehen im Flussgleichgewicht. Da die Massen nicht isoliert ist, wird $T_{a,1}$ \text{gleich} $T_{a,2}$.

Der Druck $P_{a,2}$ ist gleich $P_{a,1}$, da keine Änderung im oben erwähnten Kräfteverhältnis auftritt. Dies liegt an der gleichen Dichte von Wasser und Eis, $V + \text{gleiches Volumen}$.

\begin{align*}
P_{a,2} = P_{a,1} = 1,4 \, \text{bar}
\end{align*}

\begin{align*}
\Delta E &= Q - W \\
\end{align*}

Da das EW-Gemisch im Gleich mit dem hat stets und noch nicht alles geschmolzen ist, muss das hat schon bis auf 0°C heruntergekühlt sein, da dies Taut ist.

``````latex