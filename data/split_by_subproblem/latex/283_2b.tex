\subsection*{b) Energiebilanz um Schaufelöse (Stationär) führt zu}

\[
h_5 + \left( \frac{w_5^2}{2} \right) = h_6 + \left( \frac{w_6^2}{2} \right)
\]

\[
h_5 \text{ (lösbar, 431.9K)} = 
\]

\[
\Rightarrow h_5 = h(T_5) = h(431.9K)
\]

\[
\text{Tabelle A-22} \Rightarrow h(430K) + \frac{h(440K) - h(430K)}{(440 - 430)K} \cdot (431.9 - 430)K
\]

\[
\text{Interpolation} \Rightarrow \frac{h(440K) - h(430K)}{(440 - 430)K} \cdot (431.9 - 430K) + h(430K)
\]

\[
\text{Tabelle A-22} \Rightarrow \frac{(449.61 - 431.93) \frac{kJ}{kg}}{10K} \cdot 1.9K + 431.93 \frac{kJ}{kg}
\]

\[
= 433.37 \frac{kJ}{kg} = h_5
\]

\[
h_6 = h(340K) \Rightarrow \text{Tabelle A-22} = 340.42 \frac{kJ}{kg}
\]

``````latex