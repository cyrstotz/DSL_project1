\section*{4. a)}

\begin{itemize}
    \item The graph is a plot with the y-axis labeled as \( p \, [\text{mbar}] \) and the x-axis labeled as \( T \, [^\circ \text{C}] \).
    \item The y-axis has a logarithmic scale with values marked at 0.1, 1, 10, and 100.
    \item The x-axis ranges from -70 to 20 with increments marked at -70, -20, 0, 10, and 20.
    \item There is a curve starting from the bottom left, labeled as "Sublimationspunkt", and it rises to the right.
    \item The curve intersects the y-axis at around 0.1 mbar and the x-axis at around -70°C.
    \item There is a horizontal line labeled "test" starting from a point on the curve at around -20°C and extending to the right.
    \item The horizontal line ends at a point labeled "i".
    \item From point "i", a vertical line drops down to the curve, ending at a point labeled "ii".
    \item The region above the curve is labeled "flüssig".
    \item The region below the curve is labeled "flüssig gasförmig".
\end{itemize}