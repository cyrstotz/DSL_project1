a) Piston und Masse am Gas im Zylinder ung

\textbf{Zustand:}
\[
T_{EG,1} = 500^\circ C, \quad V_{g,1} = 3,74 \, L = 3,74 \cdot 10^{-3} \, m^3
\]
\[
m_{EG} = 0,5 \, kg, \quad T_{EU,1} = 0^\circ C, \quad x \cdot F_{m,1} = m_{EU} \cdot g \cdot 0,5
\]

\textbf{Graph Description:}
There is a graph with two axes. The horizontal axis is labeled with $V \, \left[\frac{m^3}{s}\right]$ and the vertical axis is labeled with $P \, \left[\frac{N}{m^2}\right]$. There are three lines drawn on the graph:
1. The first line starts at $P_{g,1}$ and $V_{g,1}$ and is labeled $F_{ges} = 10^5 \, \frac{N}{m^2} + \left(0,5 \, kg \cdot 9,81 \, \frac{m}{s^2} + 32 \, kg \cdot 9,81 \, \frac{m}{s^2}\right) \cdot \frac{1}{2,5 \cdot 10^{-3} \, m^2}$.
2. The second line is labeled $F_{ges} = 4,9 \cdot 10^5 \, \frac{N}{m^2}$.
3. The third line is labeled $F_{ges} = 1,1 \cdot 10^6 \, \frac{N}{m^2}$.

\[
\Rightarrow \frac{V}{m^3} = 3,74 \cdot 10^{-3} \, bis
\]

\textbf{Graph Description:}
There is a diagram showing a piston with a force $F$ acting on it. The force is labeled $F = A \cdot \left(m_{EU} + m_{c}\right) \cdot 9,81 \, \frac{m}{s^2}$. The area $A$ is labeled as $A = \pi \cdot \left(\frac{5 \cdot 10^{-2} \, m}{2}\right)^2 = 7,853 \cdot 10^{-3} \, m^2$.

\textbf{Durch das Gesetz:}
\[
F = F_{ges} \quad \text{(im Gleichgewicht)}
\]

\[
F_{ges} = P_{g,1} \cdot A
\]

\[
\Rightarrow P_{g,1} = \frac{F_{ges}}{A} = \frac{10^5 \, \frac{N}{m^2} + \left(32 \, kg \cdot 9,81 \, \frac{m}{s^2}\right)}{7,853 \cdot 10^{-3} \, m^2}
\]

\[
= \frac{10^5 \, \frac{N}{m^2} + \left(32 \, kg \cdot 9,81 \, \frac{m}{s^2}\right)}{7,853 \cdot 10^{-3} \, m^2}
\]

\[
= 1,4003 \cdot 10^5 \, \frac{N}{m^2} = 1,4003 \, bar
\]

``````latex