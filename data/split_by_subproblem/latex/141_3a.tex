

\subsection*{a)}
\begin{itemize}
    \item[$p_{g1}$] $\rightarrow$ ideale Gasgleichung
\end{itemize}

\[
p_{g1} = \frac{R \cdot T}{V}
\]

\[
R = \frac{R_m}{M} = \frac{8.314}{30} = 0.166
\]

\[
p_{g1} = \frac{p \cdot V}{R \cdot T} = \frac{1.40 \, \text{bar} \cdot 3 \, \text{l}}{0.166 \cdot 773.15 \, \text{K}} = 3.42 \, \text{g}
\]



\section*{a)}

\begin{align*}
x_{\text{Eis}} &= \\
h_{\text{Eis}} &= 0.6 \\
h_{\text{Neu}} &= 
\end{align*}

Eiswasser hat die Temperatur von 0.0033°C \\
aufgrund der thermodynamischen AGL.

\begin{align*}
u &= u_{\text{flüssig}} + x \cdot (u_{\text{gesättigt}} - u_{\text{flüssig}}) \\
x &= \frac{u - u_{\text{flüssig}}}{u_{\text{gesättigt}} - u_{\text{flüssig}}} = \frac{-546.1 - (-0.033)}{-333.492 - (-0.033)} \\
u &= u_{\text{flüssig}} + q_{12} = -200.092 \frac{\text{kJ}}{\text{kg}} \quad \leftarrow 316.563 \frac{\text{kJ}}{\text{kg}} = 516.6
\end{align*}

\begin{align*}
u_1 (40^\circ \text{C}) &= 0.6 \cdot -333.458 + (1 - 0.6) \cdot (-0.045) \\
&= -200.092 \frac{\text{kJ}}{\text{kg}}
\end{align*}

``````latex