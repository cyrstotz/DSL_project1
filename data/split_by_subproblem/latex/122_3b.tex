b)

\[
\begin{array}{|c|c|c|c|c|c|c|c|}
\hline
 & P & T & V & \dot{W} & \dot{Q} & \dot{S} & X \\
\hline
1 & p_1 = p_2 & 1.5748 & & & & & \\
\hline
2 & 1.5748 & -10^\circ C & & 28 \, W & 0 & 0 & 1 \\
\hline
3 & 8 \, \text{bar} & & & & & & \\
\hline
4 & 8 \, \text{bar} & & & 0 & 0 & 0 & \\
\hline
\end{array}
\]

\[
p_1 = p_2
\]

\[
T_2 = T_1 - 6K \quad T_{1i} = \text{Abb 5} \left( \frac{p = 1 \, \text{mbar}}{c \, 10^\circ \text{übereinstimmt}} \right) \approx -10^\circ C
\]

\[
T_2 = -10^\circ C
\]

\[
Q_{12} = 0
\]

\[
Q_{23} = 0
\]

\[
S_{23} = \text{adiabat + Reversibel} = 0
\]

\[
p_2 = \text{Tab A - 10} \left( T = -10^\circ C \right) = 1.5748 \, \text{bar} = p_1
\]

``````latex


\begin{itemize}
    \item[e)] Temperatur würde fallen: Sublimation ist ein endothermer Prozess, sprich beim Verdunsten nimmt das Wasser Wärme auf. Ebenso kommt noch die Abg.-Flüssig.-Wärme.
\end{itemize}

\[
\dot{E}_n = \frac{1 \cdot \dot{Q}_{zu}}{1 \cdot \dot{W}_1} = \frac{1 \cdot \dot{Q}_{zu}}{1 \cdot \dot{Q}_{ab} + \dot{Q}_{zu}}
\]

\[
\frac{\dot{Q}_{zu}}{\dot{W}_1} = 8 \epsilon
\]

\[
0 = \dot{m}(h_e - h_a) + \dot{Q}_n
\]

\[
\dot{Q}_n = \dot{m}(h_a - h_e)
\]

\[
h_a = \text{Tab A-10 (T = -76 °)} \left( \text{rein gasförmig, h_g} \right) = 237.14 \frac{w}{kg}
\]

\[
\dot{W}_n = 28 \, W
\]

\begin{itemize}