

\subsection*{(a)}

\[
P_{g,1} \quad \text{Zustand 1}
\]
\[
mg \neq \text{im Zylinder}
\]

\begin{description}
    \item[Figure 1:] A cylinder is drawn with a piston inside it. The piston is shown with an arrow pointing downwards labeled \( mg \) and another arrow pointing downwards labeled \( p_{amb} \). There is an arrow pointing upwards from the gas inside the cylinder labeled \( P_{g,1} \). The piston is also labeled with \( mew \cdot g \).
\end{description}

\[
P_{g,1} = \frac{mew \cdot g}{A} + \frac{mg}{A} + p_{amb}
\]

\[
p = \frac{F}{A} \quad g = 9.81
\]

\subsection*{Zylinder:}

\[
A = \pi r^2 = \pi \left( \frac{D}{2} \right)^2 = \pi \left( \frac{0.11}{2} \right)^2 = A = 0.0095 \, \text{m}^2
\]

\[
P_{g,1} = 170094.92 \, \text{Pa} = \underline{1.7 \, \text{bar}}
\]

\subsection*{mg:}

\[
\rho V = mRT
\]

\[
\frac{\rho V_1}{RT_1} = m_g = 0.00397 \, \text{kg} = 3.97 \, \text{g}
\]

\[
V_1 = 3.14 \, l = 3.14 \cdot 10^{-3} \, \text{m}^3
\]

\[
T_1 = 500^\circ C \Rightarrow 773.15 \, K
\]

\[
R = \frac{R}{M_g} = \frac{8.314 \, \frac{\text{m}^3}{\text{mol} \cdot \text{K}}}{50 \, \frac{\text{kg}}{\text{kmol}}} = 0.16628 \, \frac{\text{m}^3}{\text{kg} \cdot \text{K}} = R
\]

``````latex


\section*{Aufgabe 3}



\subsection*{a)}

\[ x_{E3,1^2} > 0 \]

Da der Zustand 2 ein Gleichgewicht darstellt und immer noch $E3$ vorhanden ist, muss $T_{g12}$ dieselbe Temperatur haben wie das $E3$ als $T = 0^\circ C$.

Da der Kolben gegeben ist, ändert sich keine Temp oder hemmt damit.

\[ T_{g12} = 0^\circ C \]