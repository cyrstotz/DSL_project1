a) $p_{g1} = ?$

\[
C_v = 0.63 \frac{kJ}{kg \cdot K}
\]

\[
M = 50 \frac{kg}{kmol}
\]

\[
p = \frac{mRT}{V}
\]

\[
\Rightarrow p_{g1} \text{ ist Gegendruck zu Gaswasser und Umgebung}
\]

\[
p_g = p_{amb} + mg + m_{ew} g
\]

\[
= 100000 + 32 \cdot 9.81
\]

\[
A = r^2 \cdot \pi
\]

\[
= 0.1^2 \cdot \pi = 0.031416 \, m^2
\]

\[
= \frac{100000 + 10.000}{0.0314} \, Pa
\]

\[
= 1.1 \, bar = p_{g1}
\]

\[
m_g = \frac{pV}{RT}
\]

\[
= \frac{110.000 \cdot 0.0314}{8394 \frac{kJ}{kg \cdot K} \cdot (500 + 273.15 \, K)}
\]

\[
= 0.0.2687 \, kg = m_g
\]

\[
= 0.2687 \, g
\]