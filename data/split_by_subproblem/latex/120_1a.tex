\section*{1) a)}

\[
\dot{Q}_{\text{aus}} = ?
\]

\[
1 \text{ HS}
\]

\[
0 = \dot{m} (\text{hein - haus}) + \dot{Q}_{\text{ars}} + \dot{Q}_R \neq \dot{W}
\]

\[
\dot{Q}_{\text{aus}} = \dot{m} (\text{haus - hein}) - \dot{Q}_R
\]

\[
\text{TAB A2 siedende Flüssigkeit} \quad x = 0
\]

\[
\text{haus (100°C)} = 419{,}04 \, \text{kJ/kg}
\]

\[
\text{hein (70°C)} = 292{,}98 \, \text{kJ/kg}
\]

\[
\dot{Q}_{\text{aus}} = 0{,}3 \cdot (419{,}04 - 292{,}98) - 100 \, \text{kW}
\]

\[
\dot{Q}_{\text{aus}} = -63{,}652 \, \text{kW}
\]