

\subsection*{a)}

\begin{itemize}
    \item The first graph is a $T$-$s$ diagram with $T$ on the vertical axis labeled in Kelvin (K) and $s$ on the horizontal axis labeled in $\left[\frac{kJ}{kg \cdot K}\right]$. The graph contains several isobars and isotherms. The isobars are curved lines, and the isotherms are horizontal lines. The points 0, 1, 2, 3, 4, 5 are marked on the graph, with arrows indicating the direction of the process. The process starts at point 0, moves to point 1, then to point 2, 3, 4, and finally to point 5. The isobars are labeled with their respective pressures, and the isotherms are labeled with their respective temperatures.
    
    \item The second graph is also a $T$-$s$ diagram with $T$ on the vertical axis labeled in Kelvin (K) and $s$ on the horizontal axis labeled in $\left[\frac{kJ}{kg \cdot K}\right]$. This graph contains several isobars and isotherms as well. The points 0, 1, 2, 3, 4, 5 are marked on the graph, with arrows indicating the direction of the process. The process starts at point 0, moves to point 1, then to point 2, 3, 4, and finally to point 5. The isobars are labeled with their respective pressures, and the isotherms are labeled with their respective temperatures. Additionally, there is a label indicating an isochore process.
\end{itemize}

``````latex

\section*{Problem 2}