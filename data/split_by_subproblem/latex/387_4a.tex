

\subsection*{a)}

\begin{description}
    \item[Graph Description:] The graph consists of a horizontal axis labeled \( T(K) \) and a vertical axis labeled \( p(N/m^2) \). There are three curves on the graph:
    \begin{itemize}
        \item The first curve starts at the origin, rises to a peak, and then falls back down.
        \item The second curve starts at the same point as the first curve, rises to a higher peak, falls below the horizontal axis, rises again to a smaller peak, and then falls back down.
        \item The third curve starts at the same point as the first two curves, rises to a peak, falls below the horizontal axis, rises again to a smaller peak, and then falls back down, similar to the second curve but with different amplitudes.
    \end{itemize}
\end{description}



\subsection*{a)}
Er würde Sie würde bis \\
der kondensations temperatur \\
von R134a abkühlen etwa bis zu -22°C \\
und dann würde keine wärme mehr abgegeben \\
weil ein gleichgewicht erreicht ist

\subsection*{a2)}
\[
\begin{array}{c}
\text{Graph description:}
\end{array}
\]

The graph is a plot with the horizontal axis labeled \( T (u) \) and the vertical axis labeled \( p / N_{1/2} \). The plot shows a curve that starts at the origin, rises to a peak, and then falls back down, forming a bell-shaped curve. 

- The curve starts at the origin and rises steeply.
- There is a point labeled \( i \) at the peak of the curve.
- To the left of the peak, there is a point labeled \( i' \).
- To the right of the peak, there is another point labeled \( i'' \).
- The curve then descends symmetrically after the peak.

```