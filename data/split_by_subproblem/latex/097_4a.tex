\section*{4(a)}
\textbf{ges: P-T}

\subsection*{Graphical Descriptions}

\subsubsection*{First Graph}
The first graph is a P-T diagram with the following details:
- The vertical axis is labeled \( P \) with units \([mbar]\).
- The horizontal axis is labeled \( T \) with units \([^\circ C]\).
- The vertical axis has a marking at \( 5 \, mbar \).
- The horizontal axis has markings at \( -50^\circ C \), \( -20^\circ C \), \( 0^\circ C \), and \( 10^\circ C \).
- There are two curves:
  - The first curve starts from the bottom left, rises steeply, and then curves to the right, labeled \( F_{eis1} \).
  - The second curve starts from the bottom left, rises less steeply, and then curves to the right, labeled \( S_{a3} \).
- The region above the second curve is labeled \( Flüssig \).
- The region between the two curves is labeled \( T_{tripel} \).
- The region below the first curve is labeled \( Wasser in Lebensmitteln \).

\subsubsection*{Second Graph}
The second graph is another P-T diagram with the following details:
- The vertical axis is labeled \( P \) with units \([mbar]\).
- The horizontal axis is labeled \( T \) with units \([^\circ C]\).
- The vertical axis has a marking at \( 5 \, mbar \).
- The horizontal axis has a marking at \( T = 10^\circ C \).
- There are two curves:
  - The first curve starts from the bottom left, rises steeply, and then curves to the right, labeled \( F_{eis1} \).
  - The second curve starts from the bottom left, rises less steeply, and then curves to the right, labeled \( S_{a3} \).
- There are two points marked on the second curve:
  - Point 1 is labeled \( 1 \) and is located on the curve.
  - Point 2 is labeled \( 2 \) and is located on the curve, with a horizontal line extending to the right labeled \( T = const \).
- The region above the second curve is labeled \( Flüssig \).
- The region between the two curves is labeled \( T_{tripel} \).

``````latex