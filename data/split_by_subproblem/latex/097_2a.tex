2a) \\
see: \( T - s \) Diagram

\begin{description}
    \item[Graph Description:] 
    The graph is a \( T - s \) diagram with temperature \( T \) on the vertical axis and entropy \( s \) on the horizontal axis. The vertical axis is labeled \( T \) with the unit \([K]\). The horizontal axis is labeled \( s \) with the unit \([kJ/(kg \cdot K)]\). The graph contains several points and lines connecting these points, representing different states and processes. The points are labeled as follows:
    \begin{itemize}
        \item Point 1 at the bottom left corner, labeled \( P_e \).
        \item Point 2 above and to the right of Point 1.
        \item Point 3 above and to the right of Point 2.
        \item Point 4 above and to the right of Point 3.
        \item Point 5 to the right of Point 4.
        \item Point 6 below and to the right of Point 5.
    \end{itemize}
    The lines connecting these points are labeled with different processes:
    \begin{itemize}
        \item Line from Point 1 to Point 2 is labeled \( s = \text{const} \).
        \item Line from Point 2 to Point 3 is labeled \( p = \text{const} \).
        \item Line from Point 3 to Point 4 is labeled \( T = \text{const} \).
        \item Line from Point 4 to Point 5 is labeled \( s = \text{const} \).
        \item Line from Point 5 to Point 6 is labeled \( s = \text{const} \).
    \end{itemize}
    The graph also includes several horizontal dashed lines indicating specific temperature values:
    \begin{itemize}
        \item A line at \( 128.9K \).
        \item A line at \( 431.9K \).
        \item A line at \( -30^\circ \) to \( 243.15K \).
    \end{itemize}
    Additionally, there is a note at the bottom of the graph stating "Luft ideales Gas" and "keine 'Shock'".
\end{description}

``````latex