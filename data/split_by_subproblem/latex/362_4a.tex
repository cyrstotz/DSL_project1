

\subsection*{a)}

\begin{description}
    \item[Graph:] The graph is a Pressure-Temperature (P-T) diagram. The x-axis is labeled with $T$ (Temperature) and the y-axis is labeled with $P$ (Pressure). There are three points labeled 1, 2, and 3. Point 1 is in the liquid region, point 2 is in the vapor region, and point 3 is on the boundary between liquid and vapor. The line connecting points 1 and 2 is horizontal, indicating a constant temperature process. The line connecting points 2 and 3 is curved, indicating a change in both temperature and pressure. The region between points 1 and 2 is labeled "flüssig" (liquid), and the region between points 2 and 3 is labeled "Dampf" (vapor). The line from point 1 to the boundary is labeled "x=0" (quality equals zero), and the line from point 2 to the boundary is labeled "x=1" (quality equals one).
\end{description}