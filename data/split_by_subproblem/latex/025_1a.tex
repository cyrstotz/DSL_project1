\subsection*{a) Quasi is}
\begin{itemize}
    \item Ausgleich an 
    \item $\Delta = \text{mittlerer Wert} + \text{Quasi}$
\end{itemize}

\begin{tabular}{|c|c|c|c|c|c|}
    \hline
    & Zustand & P & V & T & h \\
    \hline
    1 & & & & 700°C & \\
    \hline
    2 & & & & 1000°C & \\
    \hline
\end{tabular}

\begin{itemize}
    \item $\alpha = 0.005 \, \text{h}$
\end{itemize}

``````latex

a) \text{Energieerhaltungsgleichung}!

\[
\dot{m} \cdot \Delta h = \dot{Q} + \dot{m} \cdot \Delta u
\]

\[
0 = \dot{m} \cdot (h_{\text{ein}} - h_{\text{aus}}) + \dot{Q} \quad \Rightarrow \quad \dot{Q}_R = \dot{Q}_{\text{aus}}
\]

\[
h_{\text{ein}} (100^\circ C, \text{Sättigungszustand}) \quad \Rightarrow \quad TAD \quad A \quad Z = 292.38 \frac{kJ}{kg}
\]

\[
h_{\text{aus}} (100^\circ C, h + h_f) \quad \Rightarrow \quad -TAD \quad A \quad Z' = 419.04 \frac{kJ}{kg}
\]

\[
\dot{Q}_{\text{aus}} = \dot{Q}_R + \dot{m}_{\text{ein}} (292.38 - 419.04) \frac{kJ}{kg} = 67.182 \, kW
\]