

\subsection*{b)}

Solange $x_{\text{Eis,2}} > 0$ ist Eis vorhanden, und die Temperatur beträgt 0°C, da sämtliche zugeführte Wärme in den Phasenübergang zu Wasser fließt und keine Temperaturänderung zur Folge hat. Daher muss die Temperatur beim Schmelzpunkt liegen. $T_{g,2} = T_{EW,2}$ weil Thermodynamisches Gleichgewicht!

Der Druck $p_{g,2}$ bleibt gleich wie im Zustand 1, also

\[
p_{g,2} = p_{g,1} = 1{,}9 \, \text{bar}
\]

Das liegt daran, dass sich weder der Außendruck $p_{\text{amb}}$ noch die Masse ($m_k + m_{EW}$) geändert haben.

\[
T_{g,2} = 0^\circ \text{C}
\]

\[
p_{g,2} = 1{,}910 \, \text{bar} = p_{g,1}
\]

\subsection*{Graphical Descriptions}

There is a circle drawn with a radius labeled as 5 cm, and the area is labeled as 0.002854 m².

``````latex