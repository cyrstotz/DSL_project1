

\subsection*{a)}

\begin{itemize}
    \item The first graph is a $P$-$T$ diagram with the following details:
    \begin{itemize}
        \item The x-axis is labeled $T$.
        \item The y-axis is labeled $P$.
        \item There are four points labeled 1, 2, 3, and 4 along the x-axis.
        \item The y-axis has two points labeled $8 \, \text{bar}$ and $0 \, \text{bar}$.
        \item There is a curve starting from the bottom left, rising to a peak, and then descending towards the right. This curve is labeled "unterkühlte Flüssigkeit" (subcooled liquid) in red.
        \item Another curve starts from the bottom left, intersects the first curve at a point labeled "Tripelpunkt" (triple point) in red, and then continues upwards to the right. This curve is labeled "überhitzter Dampf" (superheated steam) in red.
        \item The area between these two curves is labeled "N\_D Gebiet (Nassdampf)" (wet steam region) in red.
    \end{itemize}
    
    \item The second graph is a $P$-$T$ diagram with the following details:
    \begin{itemize}
        \item The x-axis is labeled $T$.
        \item The y-axis is labeled $P$.
        \item There are three regions labeled "Fest" (solid), "Flüssig" (liquid), and "Gas" (gas).
        \item The "Fest" region is at the bottom left, the "Flüssig" region is at the top, and the "Gas" region is at the bottom right.
        \item The boundary between the "Fest" and "Flüssig" regions is a straight line rising from the bottom left to the top right.
        \item The boundary between the "Flüssig" and "Gas" regions is a horizontal line starting from the "Tripelpunkt" (triple point) and extending to the right.
        \item The boundary between the "Fest" and "Gas" regions is a straight line rising from the bottom left to the top right, intersecting the "Tripelpunkt".
    \end{itemize}
\end{itemize}