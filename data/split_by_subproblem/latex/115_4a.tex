

\subsection*{a)}

\begin{itemize}
    \item The graph has a vertical axis labeled "Eis" and a horizontal axis labeled "T".
    \item There is a curve starting from the origin, moving upwards and to the right, labeled "Eis".
    \item The curve then bends downwards and to the right, labeled "Wasser".
    \item There is a horizontal line segment labeled "f" intersecting the curve at a point labeled "C".
    \item The vertical distance from the horizontal axis to the point "C" is labeled "f".
    \item The curve continues to the right and slightly downwards, ending at a point labeled "D".
\end{itemize}



\subsection*{a)}

\begin{description}
    \item[Graph Description:] The graph is a plot with the vertical axis labeled as $p$ and the horizontal axis labeled as $T$. There is a curve starting from the origin and rising upwards to the right, labeled as "Sättigungsdamp". A vertical line is drawn from a point on the curve upwards, labeled as "isotherm". A horizontal line is drawn from the same point to the right, labeled as "isobar". Another horizontal line is drawn from a point on the curve to the left, labeled as "isotherm". The intersection of the isotherm and isobar lines is marked with a dot.
\end{description}