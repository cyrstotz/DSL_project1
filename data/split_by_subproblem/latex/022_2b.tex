

\section*{b)}

\subsection*{1. HS um Schubdüse:}

\[
0 = \dot{m}_{\text{ges}} \left[ h_5 - h_6 + \frac{w_5^2 - w_6^2}{2} \right] + \dot{Q} - W_t
\]

\begin{itemize}
    \item adiabatic reversible $\rightarrow \dot{Q} = 0$
\end{itemize}

\subsection*{Entrop. bilanz:}

\[
0 = \dot{m}_{\text{ges}} \left[ s_5 - s_6 \right] + \frac{\dot{Q}}{T} + S_{\text{erzeugt}}
\]

\begin{itemize}
    \item $S_{\text{erzeugt}} = 0$, adiabatic
    \item $0$, conv. ad.
\end{itemize}

\[
S_5 = S_6
\]

\begin{itemize}
    \item \textit{hin ausrechnen mit $^\circ C$ statt Kelvin in Tabelle gerechnet dementsprechend überxall bei Temperatur +273.15}
\end{itemize}

\subsection*{S_5 bestimmen}

\[
h_5 = \text{interpolation in TAB-A-22}
\]

\[
h_5 = 441.61 - 431.43
\]

\[
\frac{440 - 430}{431.9 - 430} \cdot 431.43
\]

\[
= 433.36 \frac{kJ}{kg}
\]

\subsection*{s_5 interpolieren:}

\[
s_6 - s_5 = s^0 (T_6) - s^0 (T_s) - R \ln \left( \frac{P_6}{P_s} \right)
\]

\begin{itemize}
    \item \textit{falsch}
    \item \textit{finden}
\end{itemize}

\section*{Graphical Description}

\begin{itemize}
    \item The graph is a Temperature-Entropy (T-S) diagram.
    \item The x-axis is labeled $S \, [\frac{kJ}{kg \cdot K}]$.
    \item The y-axis is labeled $T \, [K]$.
    \item There is a dome-shaped curve in the middle of the graph, representing the phase change region.
    \item There are several points labeled 0, 1, 2, 3, 4, 5, and 6.
    \item Point 0 is at the bottom left of the dome.
    \item Point 1 is slightly above point 0.
    \item Point 2 is to the right of point 1, on an isobar line.
    \item Point 3 is above point 2, on another isobar line.
    \item Point 4 is to the left of point 3, on an isobar line.
    \item Point 5 is below point 4, on another isobar line.
    \item Point 6 is to the left of point 5, on an isobar line.
    \item Arrows indicate the direction of the process from point 0 to point 6.
    \item The isobar lines are dashed and labeled as "isobar".
    \item The mass flow rates $\dot{m}_1$, $\dot{m}_2$, and $\dot{m}_{\text{ges}}$ are indicated at various points.
\end{itemize}

``````latex

\[
S_s^0 = \frac{2.0887 - 2.0653}{440 - 430} (431.3 - 430) + 2.0653
\]

\[
= 2.0657
\]

\[
T_6 = T_5 = 431.3 \, K
\]

\[
S_6 - S_s^B = 0 = S^0(T_6) - S^0(T_5) - R \ln \left( \frac{P_6}{P_5} \right)
\]

\[
C_V = \frac{C_P}{k} = 0.7186 \, \frac{kJ}{kg \cdot K}
\]

\[
R = C_P - C_V = 0.2874 \, \frac{kJ}{kg \cdot K}
\]

\[
S^0(T_6) = S^0(T_5) + R \ln \left( \frac{P_6}{P_5} \right)
\]

\[
= 1.7931 \, \frac{kJ}{kg \cdot K}
\]

\[
\text{interpolieren in A-22}
\]

\[
T_6 = \frac{340 - 330}{1.7931 - 1.78249} \cdot (1.7931 - 1.78249) + 330
\]

\[
= 336.81^\circ C
\]

\[
w_t = -n \int_5^6 \rho \, dv
\]

\[
W_t = -m_{ges} \cdot n \int_5^6 \rho \, dv
\]

\[
\left( \frac{P}{P_5} \right)^{\frac{n-1}{n}} = \left( \frac{V_5}{V} \right)^{n-1}
\]

``````latex