a)

\[
\begin{array}{|c|}
\hline
k \\
\hline
\end{array}
\]

\[
\begin{array}{|c|}
\hline
\text{kJ} \\
\text{kg} \cdot \text{K} \\
\hline
\end{array}
\]

\[
\begin{array}{c|c}
 & S \\
\hline
T & \\
\end{array}
\]

\textbf{Graph Description:}

The graph is a plot with the x-axis labeled as \( S \) and the y-axis labeled as \( T \). The x-axis has a label at the end indicating \( S \), and the y-axis has a label at the top indicating \( T \). 

There are six points marked on the graph, labeled from 1 to 6. The points are connected by lines to form a curve. The curve starts at point 1, moves up to point 2, then to point 3, and reaches a peak at point 4. From point 4, the curve descends to point 5 and finally to point 6.

- Point 1 is at the bottom left.
- Point 2 is above and to the right of point 1.
- Point 3 is above point 2.
- Point 4 is at the peak of the curve.
- Point 5 is to the right of point 4.
- Point 6 is below point 5.

There are also some hatching lines between points 2 and 3, indicating a specific region of interest.

``````latex