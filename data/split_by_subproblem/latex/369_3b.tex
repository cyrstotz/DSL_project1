

\subsection*{b)}

\begin{itemize}
    \item Da das Eis nicht geschmolzen ist, war die Wärmeabgabe konstant, da Temp. konstant war.
    \item $\uparrow$ Temperatur und Druck müssen sinken, da Wärme nach $\downarrow$
    \item O.H. ist das Volumen nach dem Warm nach Kalt fließt!
    \item Gas und Eis in thermodynamischen GG d.h. $Q = 0$
    \item und $T = 0^\circ \text{C}$ muss gleiche Temp. haben wie Eiswasser (0°C)
    \item sonst wäre kein Eis mehr vorhanden oder wäre gleichmäßig.
\end{itemize}

\[
\frac{p_1 \cdot V_1}{T_1} = \frac{p_2 \cdot V_2}{T_2}
\]

\[
\frac{500 \text{°C}}{0 \text{°C}} = \frac{p_2 \cdot 2.73 \cdot 10^5 \text{ Pa}}{273.15 \text{ K}}
\]

\[
= 0.49 \text{ bar}
\]

``````latex


\section*{Student Solution}

b) 

\[
\frac{T_2}{T_1} = \left( \frac{p_2}{p_1} \right)^{\frac{\kappa - 1}{\kappa}}
\]

\[
\frac{T_2}{T_1} = \left( \frac{0.2617}{2.837} \right)^{\frac{1.4 - 1}{1.4}}
\]

\[
p_2 = p_1 \Rightarrow \text{Druck durch } p_0 \text{ und Masse hat sich nicht geändert}
\]