

\subsection*{a)}

\[
T [K] \quad \text{(y-axis)}
\]

\[
S \left[ \frac{kJ}{kg \cdot K} \right] \quad \text{(x-axis)}
\]

\begin{itemize}
    \item The graph is a T-S diagram with the y-axis labeled as \( T [K] \) and the x-axis labeled as \( S \left[ \frac{kJ}{kg \cdot K} \right] \).
    \item There are six points labeled 0, 1, 2, 3, 4, 5, and 6.
    \item Point 0 is at the origin.
    \item Point 1 is directly above point 0.
    \item Point 2 is to the right of point 1.
    \item Point 3 is above point 2.
    \item Point 4 is below point 3 and to the right of point 2.
    \item Point 5 is to the right of point 4.
    \item Point 6 is below point 5 and to the right of point 0.
    \item The path from point 0 to point 1 is vertical.
    \item The path from point 1 to point 2 is horizontal.
    \item The path from point 2 to point 3 is curved upwards.
    \item The path from point 3 to point 4 is curved downwards.
    \item The path from point 4 to point 5 is horizontal.
    \item The path from point 5 to point 6 is vertical.
    \item The path from point 6 to point 0 is diagonal.
    \item There are labels along the paths indicating processes:
        \begin{itemize}
            \item \( \dot{m}_R \) along the path from point 0 to point 1.
            \item \( \dot{m}_M \) along the path from point 1 to point 2.
            \item \( \dot{m}_M \) along the path from point 2 to point 3.
            \item \( \dot{m}_M \) along the path from point 3 to point 4.
            \item \( \dot{m}_M \) along the path from point 4 to point 5.
            \item \( \dot{m}_M \) along the path from point 5 to point 6.
            \item \( \dot{m}_{\text{super}} \) along the path from point 6 to point 0.
        \end{itemize}
    \item The label \( \text{isobaren} \) is written near the top of the graph.
    \item The label \( 0.19 \, \text{bar} \) is written near the top right of the graph.
\end{itemize}

\begin{tabular}{|c|c|c|c|c|}
    \hline
    P & T & \omega & & \\
    \hline
    0 & 0.19 \, \text{bar} & = 243.15 \, \text{K} & -30^\circ \text{C} & 200 \, \frac{m}{s} \\
    \hline
    1 & & & & \\
    \hline
    2 & P_2 & & & \\
    \hline
    3 & = P_2 & & & \\
    \hline
    4 & & & & \\
    \hline
    5 & 0.5 \, \text{bar} & 431.9 \, \text{K} & 220 \, \frac{m}{s} & \\
    \hline
    6 & 0.19 \, \text{bar} & & & \\
    \hline
\end{tabular}

\[
Q_{0A} = 0
\]

\[
\dot{m}_R = 5.235
\]

\[
q_B = 1.955 \, \frac{kJ}{kg}
\]

\[
\Delta S = 0
\]

``````latex


\section*{Aufgabe 2}



\subsection*{a)}
\begin{equation*}
0 = \dot{m} (s_0 - s_6) + \frac{\dot{Q}_B}{T_B} + \dot{S}_{erzeugt} \quad \text{1. Linie}
\end{equation*}

\begin{equation*}
\dot{S}_{erzeugt} = s_6 - s_0 - \frac{q_6}{T_B} = c_p \ln \left( \frac{T_6}{T_0} \right) - \frac{q_6}{T_B} = -625.73 \frac{kJ}{kg \cdot K}
\end{equation*}

\begin{equation*}
e_{x,verl} = s_{erzeugt} \cdot T_0 = -152.144 \frac{kJ}{kg}
\end{equation*}

``````latex