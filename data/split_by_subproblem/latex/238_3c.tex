c) Um diese Aufgabe zu lösen brauchen wir die Energiebilanz
\[
\frac{dE}{dt} = \sum_i \dot{m}_i \left( h_i + \frac{c_i^2}{2} + \frac{g z_i}{2} \right) + \sum_j \dot{Q}_j - \sum_k \dot{W}_k
\]

\textbf{System:}

\begin{description}
    \item[Description of the diagram:] The diagram shows a rectangular box with a horizontal line at the top representing the system boundary. Inside the box, there is a small circle at the top center with a vertical line extending downwards, representing a piston or similar mechanism. Below the piston, there are two small rectangles side by side, representing different phases or components within the system.
\end{description}

\[
m_{u1} - m_{u2} = Q_{u2}
\]

\[
u_1 = u_{\text{flüssig}} (1 \text{ bar}) + x_{\text{eis,1}} \left( u_{\text{fest}} (1 \text{ bar}) - u_{\text{flüssig}} (1 \text{ bar}) \right)
\]

\[
u_{\text{flüssig}} (1 \text{ bar}) = -0.045 \frac{\text{kJ}}{\text{kg}}
\]

\[
u_{\text{fest}} (1 \text{ bar}) = 333.458 \frac{\text{kJ}}{\text{kg}}
\]

\[
u_1 = -200.0928 \frac{\text{kJ}}{\text{kg}}
\]

\[
u_2 = 
\]

\[
m \cdot (u_2 - u_1) = \dot{Q} \quad \text{(nach Energiebilanz; gelöst mit System Gas)}
\]

\[
u_{2,\text{gas}} - u_1 = c_v (T_2 - T_1) = c_v (0^\circ \text{C} - 500^\circ \text{C}) = 0.632 \frac{\text{kJ}}{\text{kgK}} \cdot (-500 \text{K}) = -331.5 \frac{\text{kJ}}{\text{kg}}
\]

\[
\dot{m}_{\text{gas}} (u_2 - u_1) = Q
\]

\[
\dot{Q} = -1.134 \text{ kJ} \quad \text{(so viel Wärme wird dem Gas entzogen)}
\]