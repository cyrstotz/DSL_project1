

\subsection*{b)}

\begin{align*}
\left\{ \frac{dE}{dt} = \dot{Q}(\cdots) + \dot{Q}_2^0 - \dot{W}_u \right\} \\
\implies \Delta E = \Delta U = -W_u \\
\frac{dE}{dt} &= \text{lin} \\
p_{g12} - p_{Ev12}
\end{align*}

\subsection*{Graphical Description}

The graphical content consists of several parts:

1. A wavy line that starts from the left and oscillates up and down across the page. This line represents a function or a process over time.
2. There are annotations along the wavy line, indicating changes in energy or other quantities. These annotations include:
   - $\Delta E$
   - $\Delta U$
   - $\Delta E_{int} = \text{cif}$

3. To the right of the wavy line, there is a vertical double line, which might represent a boundary or a separation between different regions or phases.

4. The text "for gas" is written next to the vertical double line, indicating that the equations or processes described are specific to a gas.

``````latex

\section*{Student Solution}



\subsection*{b)}

\begin{align*}
\text{p = const} & \implies T_{3,2} = T_{3,1} \left( \frac{V_{1g}}{V_{1f}} \right) \\
& \\
& \frac{d}{dt} \left( V_{1g} \right) = - \frac{W_v}{\rho} + V_r \\
& \\
\frac{dE}{dt} & = \dot{m} (\cdots) + \dot{Q}_{in} - \dot{W}_v \\
& \\
\Delta \text{u}_{\text{in}} = \dot{Q}_{in} - \dot{W}_v & \quad \text{and} \quad \Delta \text{u}_{\text{ing}} = \dot{Q}_{in} - \dot{W}_v \\
& \implies \dot{W}_v = 
\end{align*}