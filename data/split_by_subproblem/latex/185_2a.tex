\subsection*{a) T-s Diagramm}

\[
\begin{array}{c}
\text{T} [\text{K}] \\
\uparrow \\
\text{T}_0 \\
\end{array}
\]

\noindent
The diagram is a Temperature-Entropy (T-s) diagram with the y-axis labeled as \( T [\text{K}] \) and the x-axis labeled as \( S [\frac{\text{kJ}}{\text{kg K}}] \). The origin is marked as \( T_0 \) on the y-axis. 

There are several curves and points marked on the diagram:
- A curve labeled "isotherm" starting from the origin and moving upwards.
- Another curve labeled "isotherm" starting from a point on the x-axis and moving upwards.
- A curve labeled "isobar" starting from a point on the y-axis and moving to the right.
- Points labeled 1, 2, 3, 4, 5, and 6 are marked on the diagram.
- Point 1 is at the intersection of the first isotherm and the y-axis.
- Point 2 is on the second isotherm.
- Point 3 is on the isobar.
- Point 4 is on the second isotherm.
- Point 5 is on the isobar.
- Point 6 is on the x-axis.

The points are connected by lines indicating processes:
- From point 1 to point 2, the process is labeled "isentrop".
- From point 2 to point 3, the process is labeled "isotherm".
- From point 3 to point 4, the process is labeled "isentrop".
- From point 4 to point 5, the process is labeled "isotherm".
- From point 5 to point 6, the process is labeled "isobar".

There are also labels indicating pressures:
- \( p_1 \)
- \( p_2 = p_3 \)
- \( p_0 \times 10^2 \)

\[
\begin{array}{c}
\text{T} [\text{K}] \quad p [\text{bar}] \\
1 \quad \text{isentrop} \quad \quad 242.10 \quad 0.151 \\
2 \quad \text{isentrop} \quad \quad \\
3 \quad \text{isobar} \quad \quad \\
4 \quad \quad \quad \quad \\
5 \quad \quad \quad \quad 431.9 \quad 0.5 \\
6 \quad \quad \quad \quad 316.48 \quad 0.191 \\
\end{array}
\]

\[
\varphi = 1.006
\]
\[
\mu = 1.4
\]

``````latex