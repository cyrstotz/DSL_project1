

\subsection*{b)}

\begin{align*}
    w_s &= 220 \frac{m}{s} \\
    p_5 &= 0.8 \text{ bar} \\
    T_5 &= 431.8 \text{ K} \\
    p_6 &= 0.191 \text{ bar} \\
    Q &= \dot{m}_{ges} \left( h_5 - h_6 + \frac{w_5^2}{2} - \frac{w_6^2}{2} \right) - \dot{W}_{rev}
\end{align*}

\begin{itemize}
    \item Another graph is drawn with the x-axis labeled as $S \left[ \frac{u}{kg} \right]$ and the y-axis labeled as $T \left[ K \right]$.
    \item Several curves are drawn, each representing different pressures.
    \item Points are labeled as 1, 2, 3, 4, 5, and 6.
    \item Point 1 is at the intersection of the lowest pressure curve and the y-axis.
    \item Point 2 is on the next higher pressure curve, connected to point 1 by a line labeled "reversibel".
    \item Point 3 is on the same pressure curve as point 2, connected by an isobaric line.
    \item Point 4 is on a higher pressure curve, connected to point 3 by a line labeled "reversibel".
    \item Point 5 is on the same pressure curve as point 4, connected by an isobaric line.
    \item Point 6 is on the lowest pressure curve, connected to point 5 by a line labeled "reversibel".
\end{itemize}

\begin{align*}
    S_5 &= S_6 \quad \text{weil isentrop} \\
    \frac{T_6}{T_5} &= \left( \frac{p_6}{p_5} \right)^{\frac{n-1}{n}} \\
    T_6 &= T_5 \left( \frac{p_6}{p_5} \right)^{\frac{n-1}{n}} \\
    T_6 &= 431.8 \left( \frac{0.191}{0.8} \right)^{\frac{2}{7-1}} = \underline{328.675 K}
\end{align*}

``````latex


\begin{align*}
Q &= m_{ges} \cdot Cp\delta (T_5 - T_6) + m_{ges} \frac{\omega_5^2}{2} - m_{ges} \frac{\omega_6^2}{2} \\
\omega_{ges} &= \frac{\sqrt{2 \cdot Cp \cdot \delta (T_5 - T_6)}}{n} - 74.66 \\
Q &= Cp \cdot w = Cp \cdot \frac{c_p}{n} = 0.2854 \frac{\omega_5}{\delta K} \\
E_v &= \frac{Cp}{n} \\
\omega_6 &= \sqrt{2 \left( Cp (T_5 - T_6) + \frac{\omega_5^2}{2} \right)} = 507.2434 \frac{m}{s}
\end{align*}

\begin{align*}