\subsection*{a) gegeben: \( \dot{m} \)}

Energiehaltung im Reaktor:
\[
\frac{d}{dt} \sum_i (h_i + \frac{c_i^2}{2} + \frac{g z_i}{2})^0 + \sum \dot{Q} = \sum \dot{W}
\]
\[
0 = \dot{m} (\text{rein} - \text{raus}) + \dot{Q}_{\text{zu}} + \dot{Q}_R
\]
\[
\dot{Q}_{\text{zu}} = \dot{m} (\text{h}_{\text{aus}} - \text{h}_{\text{rein}}) - \dot{Q}_R
\]
\[
\dot{Q}_R < 0 \quad \dot{m} = 3 \, \text{kg/s}
\]

Aus T\&A A-2:
\[
\text{h}_{\text{rein}} (\text{10°C}) = 42.98 \, \frac{\text{kJ}}{\text{kg}}
\]
\[
\text{h}_{\text{aus}} (\text{100°C}) = 419.04 \, \frac{\text{kJ}}{\text{kg}}
\]

\[
\Rightarrow \dot{Q}_{\text{zu}} = 1375.82 \, \text{kW}
\]