

\subsection*{a)}
ges: $p_{g1}$, $m_g$

\[
T_{g1} = 500^\circ C = 773.15 \, K
\]

\[
V_{g1} = 3.14 \, L = 0.00314 \, m^3
\]

\[
R_g = \frac{\overline{R}}{M} = 0.1663 \, \frac{kJ}{kg \cdot K}
\]

\[
p_{g1} V_{g1} = m_g R_g T_{g1}
\]

\[
\begin{array}{|c|}
\hline
\text{Diagram: A vertical arrow pointing downwards labeled } (m_k + m_{EW})g \\
\text{A horizontal arrow pointing to the right labeled } p_{amb} A \\
\text{A rectangle representing a piston with the label } p_{amb} A \text{ on the left side} \\
\hline
\end{array}
\]

\[
p_{amb} A + (m_k + m_{EW})g = p_{1} A
\]

\[
p_{amb} + \frac{g}{A} (m_k + m_{EW}) = p_{1} = 1.4 \, bar = p_{g1}
\]

\[
\frac{p_{g1} V_{g1}}{R_g T_{1}} = m_g
\]