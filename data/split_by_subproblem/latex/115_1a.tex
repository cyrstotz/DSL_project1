

\subsection*{a)}

\begin{align*}
\text{ges:} \quad \dot{Q}_{\text{aus}} \quad \text{die Flüssigkeit} \\
\text{Energiegleichung einer } \text{\sout{Turbine}} \text{Strömung, Heizung} \text{mit dem} \\
\dot{Q} = \dot{m} [h_2 - h_a] + \sum \dot{Q}_i - \sum \dot{E}_{\text{kin}} - \sum \dot{E}_{\text{pot}} \text{ Vernachlässigbar}
\end{align*}

\begin{align*}
\dot{m} = \dot{m}_{\text{an}} - \dot{m}_{\text{aus}} = 0.3 \frac{\text{kg}}{\text{s}}, \quad \sum \dot{W}_{\text{ein}}: \text{laser} \rightarrow \dot{E}_{\text{kin}} = 0
\end{align*}

\begin{align*}
\text{zD} \quad h_2 = h_{\text{wasser, siedend (700°C)}}, \quad h_a = h_{\text{wasser, siedend (200°C)}}
\end{align*}

\begin{align*}
\Rightarrow \quad \dot{Q}_a = \sum \dot{Q}_i - \dot{Q}_R - \dot{Q}_{\text{aus}}
\end{align*}

\begin{align*}
\Rightarrow \quad \dot{Q}_{\text{aus}} = \dot{Q}_R + \dot{m} (h_{\text{wasser, siedend (700°C)}} - h_{\text{wasser, siedend (200°C)}})
\end{align*}

\begin{align*}
\text{Tabelle A2:} \quad h_{\text{wasser, siedend (200°C)}}: \quad h_f = 292.98 \frac{\text{kJ}}{\text{kg}} \\
h_{\text{wasser, siedend (700°C)}}: \quad h_f = 419.04 \frac{\text{kJ}}{\text{kg}}
\end{align*}

\begin{align*}
\Rightarrow \quad \dot{Q}_{\text{aus}} = 700 \frac{\text{kJ}}{\text{s}} + 0.3 \frac{\text{kg}}{\text{s}} \cdot (292.98 \frac{\text{kJ}}{\text{kg}} - 419.04 \frac{\text{kJ}}{\text{kg}}) \\
= 62.192 \frac{\text{kJ}}{\text{s}} = \dot{Q}_{\text{aus}}
\end{align*}

``````latex


\section*{Problem 1}



\subsection*{a)}

\text{ges:} \quad \overline{T_{KF}}

\[
\Rightarrow \overline{T} = \frac{\int_{s_a}^{s_e} T \, dS}{s_a - s_e}
\]

\text{Totaldifferential mit idealem Gas:}

\[
T dS = dH - v dP = dH, \text{ da } v dP = 0
\]

\[
\Rightarrow \overline{T} = \frac{h_a - h_e}{s_a - s_e}
\]

\[
\Delta h = \int_{T_1}^{T_2} C_p(T) \, dT + c_p(T_2 - T_1), \quad p_2 - p_1 = 0
\]

\[
\Delta s = \int_{T_1}^{T_2} \frac{C_p(T)}{T} \, dT, \quad \text{c_p ist konstant}
\]

\[
\overline{T} = \frac{c_p \cdot \Delta T}{c_p \cdot \left( \ln \left( \frac{T_2}{T_1} \right) \right)} = \frac{\Delta T}{\ln \left( \frac{T_2}{T_1} \right)}
\]

\[
\overline{T} = \frac{(298.15 - 288.15) \, K}{\ln \left( \frac{298.15}{288.15} \right)} = 293.121 \, K = \overline{T_{KF}}
\]



\subsection*{a)}

\[
\dot{S}_{eq} = \dot{m} \cdot (s_{a} - s_{e}) - \frac{\dot{Q}_{12}}{T_{12}} - \frac{\dot{Q}_{aus}}{T_{KF}}
\]

\[
= 0.3 \, \frac{\text{kg}}{\text{s}} \cdot (1.3009 - 0.9549) \, \frac{\text{kJ}}{\text{kgK}} - \frac{100 \, \text{kW}}{337.75 \, \text{K}} - \frac{(-162.782 \, \text{kW})}{793.122 \, \text{K}}
\]

\[
= 0.00497 \, \frac{\text{kJ}}{\text{K s}} = \dot{S}_{eq}
\]



\subsection*{a)}
\[
\Delta m = \frac{5.455 \, \text{kg} \cdot \left( 292.95 - 229.3778 \right) \, \frac{\text{kJ}}{\text{kg}}}{83.96 \left( 1 - \frac{292.95}{83.96} \frac{\text{kJ}}{\text{kg}} \right)}
\]
\[
= 3.756.84 \, \text{kg} = \Delta m_{12}
\]