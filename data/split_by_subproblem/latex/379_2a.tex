

\subsection*{a)}

\begin{itemize}
    \item The first graph is a pressure-volume ($P$-$V$) diagram with several curves. The curves start at a high pressure and low volume, then decrease in pressure as the volume increases, forming a series of downward-sloping curves.
    \item The second graph is a pressure-temperature ($P$-$T$) diagram. The y-axis is labeled $P$ and the x-axis is labeled $T$. There are two curves: one labeled $T_{krit}$ and another labeled $T_{tripel}$. The $T_{krit}$ curve is higher and more to the right than the $T_{tripel}$ curve. The x-axis has markings at $-10^\circ$, $0^\circ$, $10^\circ$, and $20^\circ$. There is a point labeled $X$ on the $T_{tripel}$ curve.
\end{itemize}

$T_1 = -20^\circ C = 253.15K$