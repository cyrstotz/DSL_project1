

\subsection*{a)}

\begin{itemize}
    \item A graph is drawn with pressure \( P \) on the y-axis and temperature \( T \) on the x-axis.
    \item The y-axis is labeled with \( P \) and the unit \([ \text{bar} ]\).
    \item The x-axis is labeled with \( T \) and the unit \([ \text{K} ]\).
    \item There are three curves on the graph:
    \begin{itemize}
        \item The first curve, labeled "Isotherme Expansion", starts from the origin and curves upwards steeply.
        \item The second curve, labeled "Isobare Abkühlung", starts from a point on the y-axis and curves upwards more gently.
        \item The third curve, labeled "Tripelpunkt", is a horizontal line intersecting the second curve.
    \end{itemize}
    \item Points 1, 2, and 3 are marked on the graph:
    \begin{itemize}
        \item Point 1 is on the first curve.
        \item Point 2 is on the second curve.
        \item Point 3 is on the third curve.
    \end{itemize}
    \item The region between the first and second curves is labeled "Fest" (solid).
    \item The region between the second and third curves is labeled "Flüssig" (liquid).
    \item The region to the right of the third curve is labeled "Gas" (gas).
    \item The temperature difference \(\Delta T = 10K\) is marked on the x-axis.
\end{itemize}