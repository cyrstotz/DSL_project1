

\subsection*{b)}

\begin{itemize}
    \item The process from state 2 to state 3 is adiabatic and reversible.
    \item There is a circular diagram with arrows pointing into and out of it, labeled as state 3.
    \item The energy balance equation is given by:
    \[
    \frac{dE}{dt} = 0 = \dot{m}(h_2 - h_3) + \dot{Q}^{\text{in}} - \dot{W}
    \]
    \item The enthalpies are given as:
    \[
    h_2 \left( x = 1 \right)
    \]
    \[
    h_3 \left( 8 \text{ bar}, \right)
    \]
    \item The work done is given as:
    \[
    W_K = -28 \text{ W}
    \]
\end{itemize}



\subsection*{b)}
\subsubsection*{Graph 1:}
The graph is a plot with pressure \(P\) on the vertical axis and temperature \(T\) on the horizontal axis. The curve starts from the origin and rises non-linearly, representing the gas phase. There is a label "compressed" pointing to a point on the curve.

\subsubsection*{Graph 2:}
The graph is a plot with pressure \(P\) on the vertical axis and temperature \(T\) on the horizontal axis. There are two lines intersecting. The first line, labeled \(1\), starts from a higher pressure and decreases linearly. The second line, labeled \(2\), starts from a lower pressure and increases linearly. The region between the lines is shaded and labeled "blow".

\subsubsection*{Graph 3:}
The graph is a plot with pressure \(P\) on the vertical axis and temperature \(T\) on the horizontal axis. There are four points labeled \(1\), \(2\), \(3\), and \(4\) forming a cycle. The process between points \(1\) and \(2\) is labeled "isobar", between points \(2\) and \(3\) is labeled "revers. ad", between points \(3\) and \(4\) is labeled "isobar", and between points \(4\) and \(1\) is labeled "entropy changes".