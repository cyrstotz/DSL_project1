\section*{Problem 2a}

\begin{figure}[h]
    \centering
    % Detailed verbal description of the graph
    The graph is a Temperature-Entropy (T-S) diagram with the x-axis labeled as $S \left( \frac{kJ}{kg \cdot K} \right)$ and the y-axis labeled as $T \left( ^\circ C \right)$. The graph contains several curves and points labeled from 0 to 6. The points are connected by different types of processes, which are labeled as follows:
    
    - Point 0 to Point 1: Labeled as "Kompression" and "isotrop".
    - Point 1 to Point 2: Labeled as "isotrop" and "adiabat".
    - Point 2 to Point 3: Labeled as "Isobar".
    - Point 3 to Point 4/5: Labeled as "isobar" and "adiabat".
    - Point 4/5 to Point 6: Labeled as "adiabat, reversibel".
    
    The graph also includes the following labels:
    
    - Between Point 0 and Point 1: $p_0$
    - Between Point 1 and Point 2: $p_1$
    - Between Point 2 and Point 3: $p_2$
    - Between Point 3 and Point 4/5: $p_3$
    
    The graph shows the different thermodynamic processes and their respective paths on the T-S diagram.

\end{figure}

``````latex


\section*{Problem 2}



\subsection*{a)}

\begin{itemize}
    \item The first graph is a $T$ vs $s$ diagram. The $x$-axis is labeled $s \left[\frac{kJ}{kg \cdot K}\right]$ and the $y$-axis is labeled $T \left[K\right]$. There are several curves drawn, representing different isobars. Points 5 and 6 are marked on the graph, connected by a line. The isobar for $0.5 \, \text{bar}$ is labeled.
    
    \item The second graph is a $T$ vs $s$ diagram. The $x$-axis is labeled $s \left[\frac{kJ}{kg \cdot K}\right]$ and the $y$-axis is labeled $T \left[K\right]$. There are several curves drawn, representing different isobars. Points 1, 5, and 6 are marked on the graph, connected by lines. The isobar for $p_0$ is labeled.
    
    \item The third graph is a 3D plot with $T$, $s$, and $p$ axes. Points 1, 2, 5, and 6 are marked on the graph, connected by lines. The isobars and isotherms are labeled.
    
    \item The fourth graph is a $T$ vs $s$ diagram. The $x$-axis is labeled $s \left[\frac{kJ}{kg \cdot K}\right]$ and the $y$-axis is labeled $T \left[K\right]$. There are several curves drawn, representing different isobars. Points 0, 1, 2, 4, and 5 are marked on the graph, connected by lines. The isobar for $p_0 = 0.191 \, \text{bar}$ is labeled.
\end{itemize}