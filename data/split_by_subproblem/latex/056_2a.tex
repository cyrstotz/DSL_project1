

\subsection*{a)}

\textbf{Graph Description:}

The graph is a Temperature-Entropy (T-s) diagram. The x-axis is labeled as \( s \left[ \frac{kJ}{kg \cdot K} \right] \) and the y-axis is labeled as \( T \, (K) \).

There are six points labeled from 0 to 6, connected by different types of lines indicating different processes:

- From point 0 to point 1, the line is curved upwards, labeled as "isentrop".
- From point 1 to point 2, the line is horizontal, labeled as "isotrop".
- From point 2 to point 3, the line is curved upwards, labeled as "isobar" with an increase in temperature (\(+T\)).
- From point 3 to point 4, the line is curved downwards, labeled as "isentrop".
- From point 4 to point 5, the line is horizontal, labeled as "isobar".
- From point 5 to point 6, the line is curved downwards, labeled as "isentrop".

Additional labels on the graph:
- \( p_2 = p_3 \)
- \( p_5 \)
- \( p_0 \)

\subsection*{Process Descriptions:}

\begin{align*}
0 \rightarrow 1 &: \eta_{\text{vs}} < 1 \\
1 \rightarrow 2 &: \text{isotrop} \\
2 \rightarrow 3 &: \text{isobar} \, (+T) \\
3 \rightarrow 4 &: \eta_{\text{ts}} < 1 \\
4 \rightarrow 5 &: \text{isobar} \quad p_4 = p_5 = 0.5 \, \text{bar} \\
5 \rightarrow 6 &: \text{isentrop}
\end{align*}

``````latex