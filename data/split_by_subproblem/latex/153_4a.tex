

\section*{4a)}

\subsection*{Graph 1: p(T) Diagram}
The graph shows a plot of pressure \( p \) (in bar) on the vertical axis versus temperature \( T \) (in Kelvin) on the horizontal axis. The curve starts at the origin, rises to a peak, and then falls back down, forming a bell shape. The peak of the curve is labeled as "kritischer Punkt" (critical point). To the left of the peak, the region is labeled "Nassdampf" (wet steam). Several lines are drawn from the bottom left to the top right, intersecting the curve.

\subsection*{Graph 2: T(p) Diagram}
The graph shows a plot of temperature \( T \) (in Kelvin) on the horizontal axis versus pressure \( p \) (in bar) on the vertical axis. The curve starts at the origin and rises steeply, then levels off slightly, and continues to rise. The curve is labeled "Flüssig" (liquid) on the lower part and "Gas" (gas) on the upper part. A horizontal line is drawn from the curve to the right, labeled "Tripelpunkt" (triple point). Another line is drawn vertically from the horizontal line, labeled "Sublim." The point where the horizontal line meets the curve is labeled "T = 401K".



\section*{a)}

\begin{align*}
\epsilon &= \frac{\dot{Q}_{\text{ab}}}{\dot{Q}_{2 \rightarrow 1}} = \frac{\left| \dot{Q}_{1c} \right|}{\left| \dot{W}_1 \right|} = \frac{\left| \dot{Q}_{1c} \right|}{\dot{Q}_{\text{ab}} - \dot{Q}_{2 \rightarrow 1}} \\
\dot{Q}_{1c} &= \dot{m} \left[ h_2 - h_3 \right] \\
\dot{Q}_{\text{ab}} &= \dot{m} \left[ h_4 - h_3 \right]
\end{align*}

```