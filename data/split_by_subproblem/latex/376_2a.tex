\section*{2) a)}

\begin{itemize}
    \item The graph is a plot with the vertical axis labeled as \( T \) with units \([K]\) and the horizontal axis labeled as \( s \) with units \(\left[\frac{kJ}{kg \cdot K}\right]\).
    \item There are three curved lines representing different pressures: \(0.5 \, \text{bar}\), \(P_1\), and \(0.197 \, \text{bar}\).
    \item The graph contains five points labeled \(1\), \(2\), \(3\), \(4\), and \(5\).
    \item Point \(1\) is at the intersection of the \(0.5 \, \text{bar}\) line and a vertical line.
    \item Point \(2\) is on the \(P_1\) line, slightly to the right of point \(1\).
    \item Point \(3\) is at the peak of the \(P_1\) line.
    \item Point \(4\) is on the \(P_1\) line, slightly to the right of point \(3\).
    \item Point \(5\) is on the \(0.197 \, \text{bar}\) line, directly below point \(4\).
    \item There are arrows indicating the direction of processes between the points: 
        \begin{itemize}
            \item From \(1\) to \(2\) (labeled as "isentrop").
            \item From \(2\) to \(3\) (labeled as "isentrop").
            \item From \(3\) to \(4\).
            \item From \(4\) to \(5\).
            \item From \(5\) to \(1\).
        \end{itemize}
    \item The points \(K\) and \(M\) are marked on the graph, with \(K\) being between points \(1\) and \(2\), and \(M\) being between points \(4\) and \(5\).
    \item The point \(n\) is marked on the graph, slightly below point \(1\).
\end{itemize}

``````latex


\section*{2)}

\begin{tabular}{|c|c|c|}
\hline
 & $p$ & $T$ \\
\hline
0 & \sout{0.19 bar} & -30$^\circ$C \\
  & 0.19 bar & 293.15 K $T_0$ \\
\hline
1 & 0.5 bar & \\
\hline
2 & & \\
\hline
3 & \sout{} & \\
\hline
4 & 0.5 bar & \\
\hline
5 & 0.5 bar & 293.15 K \\
\hline
6 & 0.19 bar & \\
\hline
\end{tabular}

\vspace{1cm}

$T_s, p_s, w_s$ gegeben

``````latex

28)

\[
0 = \dot{m}(h_e - h_a + \frac{\omega_e^2 - \omega_a^2}{2}) + \dot{Q} - \dot{W}
\]

Isentrope Schubdüse

\[
\Rightarrow \nu = \mathcal{H} = 1.4
\]

\[
\frac{T_2}{T_1} = \left( \frac{\nu_1}{\nu_a} \right)^{\frac{\nu - 1}{\nu}}
\]

\text{hier} = \frac{T_2}{T_5} = \left( \frac{\nu_0}{\nu_5} \right)^{\frac{K - 1}{K}}

\[
\frac{T_6}{\underline{T_5}} = T_5 \left( \frac{\nu_0}{\nu_5} \right)^{\frac{K - 1}{K}} = \underline{328.07 \, K}
\]

``````latex


\[
\dot{m} c_p (T_5 - T_6) + \dot{m} \left( \frac{\omega_3^2 - \omega_6^2}{2} \right) + \dot{\phi} - \dot{m} \dot{W}_t^+
\]

\[
W_t = -\frac{R(T_6 - T_5)}{n - 1} - \Delta h
\]

\[
\dot{W}_t^{rot} = \dot{m} \dot{\phi} + \Delta E
\]

\[
\text{Luft} \Rightarrow \text{Ideales Gas} \Rightarrow \int_{s} - \frac{R(T_6 - T_5)}{n - 1} \, ds
\]

\[
\dot{m} c_p (T_5 - T_6) + \dot{m} (\omega_5^2 - \omega_6^2) - \dot{m} \frac{R(T_6 - T_5)}{n - 1}
\]

\[
\sqrt{c_p (T_5 - T_6) + \omega_5^2 - \frac{R(T_6 - T_5)}{n - 1}} = \omega_6 = 350 \, \frac{m}{s}
\]

\[
c_v = \frac{c_p}{1.4} = 0.7936 \, \frac{kJ}{kgK}
\]

\[
R = c_p - c_v = 289 \, \frac{J}{kgK}
\]

``````latex