

\subsection*{c)}
\begin{align*}
    \text{Zustand 1:} \quad x_1 &= 0 \\
    p_1 &= p_2 = 8 \, \text{bar}
\end{align*}

\text{Prozess ist adiabatisch:}

\begin{align*}
    \text{Energiegleichung:} \quad \dot{Q} &= \dot{m} \left( h_{k2} - h_{k1} \right) + \dot{Q} \\
    \text{TAB:} \quad u_1 &= u_2 = 53,42 \, \frac{\text{kJ}}{\text{kg}}
\end{align*}

\text{Zustand 2:} \quad \text{Adiabatische Linie:} \quad p_2 = p_1

\begin{align*}
    \text{und es gilt:} \quad T_2 &= -20 \, \text{kJ/kg} \quad \text{gleich wie} \\
    \text{TAB:} \quad T_1 &= 470 \, \text{K}
\end{align*}

\begin{align*}
    x_1 &= \frac{h_1 - u_1}{u_g - u_f} \\
    h_f &= \frac{3,35 + 3,56}{2} \\
    \frac{u_2}{u_f} &= 36,365 \quad \frac{u_2}{u_g} \\
    u_2 &= 26,345 \quad \frac{u_2}{u_f}
\end{align*}

\begin{align*}
    \Rightarrow x_2 &= 0,2762
\end{align*}