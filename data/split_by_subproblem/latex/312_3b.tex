(b) Die Temperatur und der Druck bleiben konstant:

\[
T_{EW,2} = T_{EW,1} = 0^\circ C
\]

\[
p_{G,2} = p_{G,1} = p_{amb} + \frac{M_{kolben} \cdot g}{\frac{\pi \left( \frac{D}{2} \right)^2}{}} = 133970 \, Pa = 1.34 \, bar
\]

Dies liegt daran, dass wir uns im Zweiphasengebiet befinden, bei der eine Wärmezufuhr erst dann zu einer Temperatur- und Druckerhöhung führt, wenn das gesamte Eis geschmolzen ist. Dies ist bei \( x_{Eis,2} > 0 \) nicht der Fall.

Die Temperatur beträgt \( 0^\circ C \), da das Eis nicht komplett schmilzt (\( x_{Eis,2} > 0 \)) und die Temperatur des Eises daher konstant bleibt, \( T_{G,2} = T_{Eis,2} = 0 \).

Damit keine Wärme mehr übertragen wird, muss \( T_{G,2} = T_{Eis,2} = 0 \) gelten. Der Druck bleibt ebenfalls konstant, da das Atmosphärendruck und das Gewicht des Kolbes sowie der Wasser-Eis-Mischung konstant bleiben.

``````latex


3.(c) Gasegemisch Energiebilanz:

\[
\Delta E = M_{Gas,2} \cdot U_{Gas,2} - M_{Gas,1} \cdot U_{Gas,1} = -Q_{12} - W_{12}
\]

Wenn \( T_{EW,2} = T_{G,2} \), dann wird keine Wärme mehr übertragen.

Daher: \( T_{G,2} = T_{EW,2} = 0^\circ C = 273.15 K \)

Zudem bleibt der Druck konstant: \( p_{G,2} = p_{G,1} = 1.4 \text{bar} \)

Die Masse ebenfalls: \( M_{G,2} = M_{G,1} = 0.00342 \text{kg} \)

\[
V_{G,2} = \frac{M_{G,2} \cdot R_G \cdot T_{G,2}}{p_{G,2}} = 0.001109 \, m^3
\]

\[
W_{12} = p_{G,1} \left( V_{G,2} - V_{G,1} \right) = -284.48 \, J
\]

\[
Q_{12} = M_{Gas} \left( U_{G,1} - U_{G,2} \right) - W_{12} = M_G \cdot C_V \cdot (T_{G,1} - T_{G,2}) - W_{12}
\]

\[
= 1367.5 \, J
\]