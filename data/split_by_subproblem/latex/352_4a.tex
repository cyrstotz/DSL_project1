a)

\begin{figure}[h!]
\centering
\begin{minipage}{0.8\textwidth}
\textbf{Description of the graph:}

The graph is a pressure ($p$) vs. temperature ($T$) diagram. The $p$-axis is vertical and the $T$-axis is horizontal. 

- There is a curve starting from the origin and rising upwards, representing the phase boundary between different states of matter.
- The curve has a point labeled "Tripel" (triple point) where three phases coexist.
- To the left of the curve, the region is labeled "Fest" (solid).
- To the right of the curve, the region is labeled "Flüssig" (liquid).
- Below the curve, the region is labeled "Gas" (gas).
- There are three points marked on the graph:
  - Point 1 is on the solid side of the curve.
  - Point 2 is on the liquid side of the curve, vertically above point 1.
  - Point 3 is on the gas side of the curve, horizontally to the right of point 2.
- A horizontal arrow labeled "isobar" points from point 2 to point 3.
- A vertical arrow labeled "isochor" points from point 1 to point 2.

\end{minipage}
\end{figure}

\begin{itemize}
    \item[1] Zustand zu Beginn
    \item[2] Zustand nach (i), also nach isochorem Kühlen
    \item[3] Zustand nach (ii), also nach isobarem Druckabbau
\end{itemize}