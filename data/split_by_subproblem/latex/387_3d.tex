

\subsection*{d)}
\begin{align*}
    Q_{max} &= m_e (u_f(1.45) - u_e(1.45)) = -20 kJ
\end{align*}

\subsection*{e)}
\begin{align*}
    \text{Da weniger Q benötigt wird um auf des} \\
    \text{Eis auf 0 grad zu kühlen als um} \\
    \text{das Eis zu schmelzen bleibt die Temp bei 0}^\circ C \\
    T_{g2} &= 0^\circ C \\
    P_{g2} &= P_{g1} = 1.1 \cdot 0.15
\end{align*}

\subsection*{Figures and Graphs}
\begin{itemize}
    \item There is a graph with the following details:
    \begin{itemize}
        \item The x-axis is labeled with $x$.
        \item The y-axis is labeled with $y$.
        \item There are three points plotted on the graph:
        \begin{itemize}
            \item Point 1: $(1, 1)$
            \item Point 2: $(2, 4)$
            \item Point 3: $(3, 9)$
        \end{itemize}
    \end{itemize}
\end{itemize}

``````latex


\section*{Problem A3.5 (cont.)}

\noindent
durch \underline{langhalt weil ausdruck \& masse gleich bleibt}

\subsection*{d) $X_{\text{Eis}} 2:$}

\[
- m_{\text{EW}} \cdot 0.6 \left( U_{\text{frisch}} (T_2) + m_{\text{EW}} \cdot X_{\text{Eis}2} \cdot U_{\text{frisch}} (T_2) \right) + m_{\text{MEW}} (1 - X_{\text{Eis}2}) U_{\text{flüssig}} (T_2) - 0.6 \cdot m_{\text{EW}} U_{\text{flüssig}} (T_2)
\]

\[
= Q_{1 \rightarrow 2} - \rightarrow 0
\]

``````latex


\section*{A2b cont}

\begin{align*}
Q_{12} + M_{EW} \cdot 0.6 \cdot U_{est} + 0.4 \cdot M_{EW} \cdot U_{crisis} - M_{EW} \cdot U_{est} \\
M_{EW} \cdot U_{est} - M_{EW} \cdot U_{crisis} \\
= K_{E:3} \cdot \frac{P}{2}
\end{align*}

\begin{align*}
Q_{12} = \Delta M_{E:i} \left( \Delta U \right) \\
-\Delta M_{E:i} = \frac{C_{x}}{\Delta U} = -0.0058 \frac{kg}{s} \rightarrow 5.8 \frac{g}{s}
\end{align*}

\begin{align*}
\frac{0.6 \cdot 0.1 + \Delta M_{E:i}}{0.1 \cdot U} = 0.5615 = x_{E:i:2}
\end{align*}

``````latex