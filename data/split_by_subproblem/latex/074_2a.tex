

\subsection*{a)}

\begin{tabular}{|c|c|c|}
\hline
 & $P$ & $T$ \\
\hline
Zustand 0 & 0,1 \, \text{bar} & 30 \, ^\circ \text{C} \\
\hline
1 & & \\
\hline
2 & $P^2$ & \\
\hline
3 & $P^2$ & \\
\hline
4 & 0,5 \, \text{bar} & \\
\hline
5 & 0,5 \, \text{bar} & 437,13 \, \text{K} \\
\hline
6 & 0,1 \, \text{bar} & \\
\hline
\end{tabular}

\subsection*{Graph Description}

The graph is a plot with the vertical axis labeled $T \, (\text{K})$ and the horizontal axis labeled $s \, \left( \frac{J}{\text{kg} \cdot \text{K}} \right)$. 

- Point 0 is at the origin.
- Point 1 is above and to the right of point 0.
- Point 2 is above and to the right of point 1, connected by a line labeled "isobar".
- Point 3 is above and to the right of point 2, connected by a line labeled "isobar".
- Point 4 is below and to the right of point 3, connected by a line labeled "isobar".
- Point 5 is below and to the left of point 4, connected by a line labeled "isobar".
- Point 6 is below and to the left of point 5, connected by a line labeled "isobar".

The lines connecting the points form a closed loop, with the following labels:
- The line from point 0 to point 1 is labeled "isobar".
- The line from point 1 to point 2 is labeled "isobar".
- The line from point 2 to point 3 is labeled "isobar".
- The line from point 3 to point 4 is labeled "isobar".
- The line from point 4 to point 5 is labeled "isobar".
- The line from point 5 to point 6 is labeled "isobar".

``````latex