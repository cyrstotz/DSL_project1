

\subsection*{b)}

\begin{itemize}
    \item \textbf{Kräfte GGW:}
    \begin{align*}
        F_z + F_1 + F_3 &= F_u \\
        F_z + F_1 + F_3 &= p_0 \cdot A \\
        p_0 &= \frac{F_z + F_1 + F_3}{A} \\
        p_0 &= 1.4005 \cdot 10^5 \text{Pa} \\
        p_{0,1} &= 1.4 \text{bar}
    \end{align*}
    \item \textbf{Ideale Gas Gleichung:} $p \cdot V = m \cdot R \cdot T$
    \begin{align*}
        m_g &= \frac{p_1 \cdot V_1}{R \cdot T_1} \rightarrow V_1 = 3.14 \text{L} = 3.14 \text{dm}^3 = 3.14 \cdot 10^{-3} \text{m}^3 \\
        T_1 &= 773.15 \text{K}
    \end{align*}
    \item $R = \frac{\bar{R}}{M_g} \frac{J}{K \cdot mol} = 8.314 \frac{J}{mol \cdot K} \cdot \frac{0.05 \text{kg}}{mol} = 166.28 \frac{J}{kg \cdot K}$
    \item $m_g = 0.0034217 \text{kg}$
\end{itemize}

``````latex


\section*{Aufgabe 3}



\subsection*{b)}

Da das EK-Gemisch inkompressibel ist, bleibt das Kräfte GGW gleich.

\[
\begin{array}{c}
\begin{array}{|c|}
\hline
\text{Fläche} \\
\hline
\end{array}
\begin{array}{|c|}
\hline
\text{Fläche} \\
\hline
\end{array}
\end{array}
\]

\[
F_u = \frac{p_1 A_1 + p_2 A_2 + p_3 A_3}{A}
\]

\[
p_{1,2} = \frac{F_1 + F_2 + F_3}{A} = p_{tot} = 1.1 \text{ bar}
\]

Allerdings ändern das Volumen und die Temperatur.

\[
p \cdot V = mRT \Rightarrow T = T_{GG}
\]

\[
T_{um} = 0^\circ \text{C}
\]

\[
p_{1} = 1.1 \text{ bar} \Rightarrow \text{aus Tab. 1}
\]

Im WW 1. HS: 

\[
m(u_2 - u_1) = Q_{12} - \dot{W}_{12}^{0}
\]

\[
u_2 - u_1 = \frac{Q_{12}}{m} \Rightarrow u_2 - u_1 = c_v (T_2 - T_1)
\]

\[
c_v (T_2 - T_1) = \frac{Q_{12}}{m}
\]

\[
T_2 - T_1 = \frac{Q_{12}}{m \cdot c_v}
\]

\[
T_2 = T_1 + \frac{Q_{12}}{m \cdot c_v}
\]

``````latex


\section*{A3}