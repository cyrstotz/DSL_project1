

\subsection*{c)}
\[
\text{mit} \quad T_2 = -22^\circ \text{C}
\]
\[
P_2 = P_1 = 1.2132 \text{bar} \quad \text{aus TAB A-10}
\]

\[
\frac{h - h_f}{h_g - h_f} = x_1 = \underline{0.3347}
\]

``````latex


\section*{Student Solution}

\subsection*{Graph 1}

The first graph is a phase diagram with pressure \( p \) in bar on the vertical axis and temperature \( T \) in degrees Celsius on the horizontal axis. The graph contains three regions labeled "solid", "fluid", and "gas". 

- The "solid" region is located in the upper left part of the graph.
- The "fluid" region is located in the upper right part of the graph.
- The "gas" region is located in the lower part of the graph.

There are three lines separating these regions:

1. A line starting from the origin and curving upwards to the right, separating the "solid" and "gas" regions. This line is labeled with points \( 1 \) and \( 2 \).
2. A line starting from the origin and curving upwards to the right, separating the "solid" and "fluid" regions.
3. A line starting from the origin and curving upwards to the right, separating the "fluid" and "gas" regions. This line intersects the other two lines at a point labeled "Tripel".

\subsection*{Graph 2}

The second graph is another phase diagram with pressure \( p \) in bar on the vertical axis and temperature \( T \) in degrees Celsius on the horizontal axis. The graph contains three regions labeled "solid", "fluid", and "gas". 

- The "solid" region is located in the lower left part of the graph.
- The "fluid" region is located in the lower part of the graph.
- The "gas" region is located in the upper right part of the graph.

There are two lines separating these regions:

1. A line starting from the origin and curving upwards to the right, separating the "solid" and "fluid" regions. This line is labeled with points \( 1 \) and \( 2 \).
2. A line starting from the origin and curving upwards to the right, separating the "fluid" and "gas" regions. This line intersects the other line at a point.

```