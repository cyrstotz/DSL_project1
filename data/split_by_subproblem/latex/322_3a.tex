

\subsection*{a)}
\begin{align*}
    P_{st} \quad \text{und} \quad \text{Masse des Gases im Zylinder} \\
    \text{Druck:} \quad P_G &= P_0 + P_{\text{innen}} + P_{\text{ew}} \\
    &= 1 \, \text{bar} + \frac{M \cdot g}{A} + \frac{M \cdot g}{A} \\
    & \\
    \Rightarrow A &= \pi r^2 = \pi \cdot 5 \, \text{cm}^2 \\
    &= \pi \cdot (0.05 \, \text{m})^2 = 0.00785 \, \text{m}^2 \\
    & \\
    P_{\text{innen}} &= \frac{32 \, \text{kg} \cdot 9.81 \, \frac{\text{m}}{\text{s}^2}}{0.00785 \, \text{m}^2} = \frac{0.01 \, \text{kg} \cdot 9.81 \, \frac{\text{m}}{\text{s}^2}}{0.00785 \, \text{m}^2} \\
    &= 10^5 \, \text{Pa} + 39.36 \cdot 10^2 \, \text{Pa} + 124.1 \, \text{Pa} \\
    &= 1.4 \cdot 10^5 \, \text{Pa} \approx 1.4 \, \text{bar} \\
    & \\
    \text{Masse (ideales Gasgesetz)} \\
    m_G &= \frac{P_G V_1}{R T_1} \\
    D V &= n R T \quad (V = 3.4 \, \text{L} = 0.0034 \, \text{m}^3) \\
    & \\
    \Rightarrow R_3 = \frac{R}{M} &= \frac{8.314 \, \frac{\text{m}^3 \text{bar}}{\text{kmol} \cdot \text{K}}}{50 \, \frac{\text{kmol}}{\text{kmol}}} = 0.16628 \, \frac{\text{m}^3 \text{bar}}{\text{kg} \cdot \text{K}} \\
    & \\
    \Rightarrow m_G &= \frac{1.4 \cdot 10^5 \, \text{Pa} \cdot 0.0034 \, \text{m}^3}{0.16628 \, \frac{\text{m}^3 \text{bar}}{\text{kg} \cdot \text{K}} \cdot 273.15 \, \text{K}} \\
    &= 0.10342 \, \text{kg} \\
    & \\
    & \\
    3.42 \, \text{s} \\
\end{align*}