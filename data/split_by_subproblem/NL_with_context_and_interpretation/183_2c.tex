T subscript 6 equals T subscript 5 times (p subscript 6 divided by p subscript 5) raised to the power of (n minus 1 divided by n).  
T subscript 6 equals 328.1 Kelvin.  

Energy balance for the nozzle (process 5 to 6):  
0 equals m dot times the quantity (h subscript 5 minus h subscript c plus w subscript 5 squared divided by 2 minus w subscript 6 squared divided by 2) minus W dot subscript 5.

The equation for the work done is:  
\( \dot{W}_s = - \int_5^6 v \, dp + \frac{w_6^2 - w_5^2}{2} \)  

Expanding:  
\( = - n \cdot R \cdot (T_6 - T_5) \cdot \frac{1}{1-n} + \frac{(w_6^2 - w_5^2)}{2} \)  

Simplified further:  
\( = n \cdot R \cdot \frac{(T_5 - T_6)}{1-n} + \frac{(w_5^2 - w_6^2)}{2} \)  

The expression for \( O \):  
\( O = h_6 - h_5 + \frac{w_5^2 - w_6^2}{2} - n \cdot R \cdot \frac{(T_6 - T_5)}{1-n} \)  

---