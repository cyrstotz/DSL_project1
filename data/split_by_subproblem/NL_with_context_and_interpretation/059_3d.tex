u2 = uf plus x times (u rest minus uf)  

Delta U = Delta Q  

u2(T2) minus u1(T1) = Cr times (T2 minus T1)  
T1 = 0 degrees Celsius  
T2 = 0.003 degrees Celsius  
u2(T2) minus u1(T1) = Delta Q divided by mEW = plus 15 kilojoules divided by 0.1 kilograms = 15 kilojoules per kilogram  

u1(T1) = uf plus x1 times (u rest minus uf)  
u1(T1) = negative 0.045 plus 0.6 times (negative 333.458 plus 0.045)  
u1(T1) ≈ negative 200.082 kilojoules per kilogram  

(Note: Some parts of the content in this subtask are crossed out and unclear.)

x subscript 2 equals open parenthesis U subscript 2 minus U subscript f close parenthesis divided by open parenthesis U subscript ice plus U subscript f close parenthesis.  

U subscript 2 equals 15 kilojoules per kilogram plus U subscript f open parenthesis T close parenthesis equals 15 kilojoules per kilogram minus 200 kilojoules per kilogram equals negative 185.0 kilojoules per kilogram.  

x subscript 2 equals open parenthesis negative 185 plus 0.033 close parenthesis divided by open parenthesis negative 333.442 plus 0.033 close parenthesis equals 0.554.  

Theoretically not possible because if ice exists, the temperature should be 0 degrees Celsius.