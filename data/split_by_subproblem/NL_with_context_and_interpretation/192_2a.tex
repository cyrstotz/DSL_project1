h2 equals 237.74  
h3 equals 271.3  

2 → 3 First Law of Thermodynamics  
0 equals m-dot times (h2 minus h3) plus Q-dot minus W-dot  
adiabatic  

W-dot divided by (h2 minus h3) equals m-dot  

m-dot equals 0.834 grams per second (crossed out)  
m-dot equals 0.834 grams per second (underlined)

The diagram is a T-s (temperature-entropy) graph with labeled axes:  
- The vertical axis is labeled as T [K].  
- The horizontal axis is labeled as S [kJ/kg·K].  

The graph shows six states connected by arrows:  
- State 0 is at the bottom left, labeled as \( p_0 \) (ambient pressure) and \( T_0 = -30^\circ \text{C} \).  
- State 1 is above state 0, labeled as \( p_1 \) (pre-compressor pressure).  
- State 2 is further up, labeled as "isentrop \( p_2 \)".  
- State 3 is higher up, labeled as \( p_c \) (high-pressure compressor pressure, much greater than \( p_0 \)).  
- State 4 is to the right of state 3, labeled as \( \eta < 1 \).  
- State 5 is below state 4, labeled as "isotrop \( p_5 \)".  
- State 6 is to the left of state 5, labeled as \( p_6 \).  

Additional notes:  
- \( p_4 = p_5 = p_6 \).  
- \( p_0 = 0.191 \text{ bar} \), \( T_0 = -30^\circ \text{C} \).  
- \( p_1 \) is the pre-compressor pressure.  
- \( p_2 \) is the isentropic pressure.  
- \( p_3 \) is the isobaric pressure.