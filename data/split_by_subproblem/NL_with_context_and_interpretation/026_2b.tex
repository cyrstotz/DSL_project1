The process from state 5 to state 6 is described as isentropic.  

The temperature \( T_6 \) is calculated using the formula:  
\( T_6 = T_5 \cdot \left( \frac{p_6}{p_5} \right)^{\frac{n-1}{n}} \)  

Substituting values:  
\( T_6 = 431.9 \, \text{K} \cdot \left( \frac{0.191}{0.5} \right)^{\frac{1.4-1}{1.4}} \)  

Result:  
\( T_6 = 328.07 \, \text{K} \)  

Energy balance equation:  
\( 0 = \dot{m} \cdot (h_5 - h_6) + \dot{m} \cdot \frac{w_5^2 - w_6^2}{2} + Q^0 - W^0 \)  

Simplified:  
\( 2 \cdot (h_6 - h_5) = w_5^2 - w_6^2 \)  

Final form:  
\( w_6^2 = w_5^2 - 2 \cdot (h_6 - h_5) \)

w subscript 6 squared minus w subscript 5 squared equals 2 times (h subscript 6 minus h subscript 5).  
0 equals m subscript g times (h subscript 5 minus h subscript 6) plus m subscript g times (w subscript 5 squared minus w subscript 6 squared divided by 2) plus Q dot minus W dot.  
w subscript 6 squared equals 2 times (h subscript 5 minus h subscript 6) plus w subscript 5 squared.  
w subscript 6 squared equals 2 times c subscript p times (T subscript 5 minus T subscript 6) plus w subscript 5 squared.