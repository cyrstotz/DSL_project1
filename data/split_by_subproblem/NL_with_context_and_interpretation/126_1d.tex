First law of thermodynamics:  
Delta E equals Delta m subscript 12 times h plus Q dot.  
m subscript 2 times u subscript 2 minus m subscript 1 times u subscript 1 equals Delta m subscript 12 times h plus Q dot.  
Delta m subscript 12 equals 1 divided by h times (m subscript 2 times u subscript 2 minus m subscript 1 times u subscript 1 minus Q dot).  

h equals h subscript f at 20 degrees Celsius equals 83.96 kilojoules per kilogram (from Table A2).  
Q dot equals negative 35 megajoules.  

m subscript 2 times u subscript 2 minus m subscript 1 times u subscript 1:  
T subscript 1 equals 100 degrees Celsius, T subscript 2 equals 70 degrees Celsius.  
m subscript 1 equals 5755 kilograms.  
m subscript 2 equals m subscript 1 plus Delta m.  

u subscript 2 equals u subscript f at T subscript 2 equals 292.15 kilojoules per kilogram (from Table A2).  
u subscript 1 equals u subscript f at T subscript 1 equals 418.94 kilojoules per kilogram (from Table A2).  

Therefore, Delta m subscript 12 equals 1 divided by h times (m subscript 2 times u subscript 2 minus m subscript 1 times u subscript 1 minus Q dot).  
Delta m equals 1 divided by h times (m subscript 1 times u subscript 2 minus m subscript 1 times u subscript 1 minus Q dot divided by h).  

Delta m equals 1 divided by 83.96 times (5755 times 292.15 minus 5755 times 418.94 minus negative 35 times 10 to the power of 3).  
Delta m equals 3969.9 kilograms.