\( c_p = 1.006 \, \text{kJ/kg·K} \)  
\( n = k = 1.4 \)  
\( PE = 0 \)  

| \( p \, (\text{bar}) \) | \( T \, (^\circ \text{C}) \) | \( s \) |  
|-------------------------|-----------------------------|-------|  
| 0 | 0.191 | -30° |  
| 1 | \( p_1 > p_0 \), \( T_1 > T_0 \) | \( s_1 = s_2 \) |  
| 2 | | |  
| 3 | \( p_3 = p_2 \) | \( s_3 < s_4 \) |  
| 4 | | |  
| 5 | \( p_4 = p_5 = 0.5 \), \( T_5 = 431.9 \, \text{K} \) | |  
| 6 | | \( s_5 = s_6 \) |  

Diagram:  
A graph is drawn with the y-axis labeled \( T \, [^\circ \text{C}] \) and the x-axis labeled \( s \, [\text{kJ/kg·K}] \).  
- The graph shows a qualitative T-s diagram with points labeled 0, 1, 2, 3, 4, 5, and 6.  
- The process includes an isobar labeled \( 0.5 \, \text{bar} \), isotherm, and isentropic sections.  
- The curve starts at point 0, rises to point 1, and continues through points 2, 3, 4, 5, and 6.  
- Isentropic sections are marked between points 2 and 3, and points 5 and 6.  
- Isobaric and isothermal processes are labeled clearly.