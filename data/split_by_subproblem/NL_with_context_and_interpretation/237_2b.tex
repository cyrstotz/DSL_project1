The equation for \( T_6 \) is written as:  
\( T_6 = T_5 \left( \frac{p_6}{p_5} \right)^{\frac{n-1}{n}} \).  

It is calculated as:  
\( T_6 = T_5 \left( \frac{p_6}{p_5} \right)^{\frac{n-1}{n}} \).  
\( T_6 = 328.07 \, K \).  

Constants are defined:  
\( n = 1.4 \).  

Heat flow and work are stated as:  
\( \dot{Q} = 0 \), \( \dot{W} = 0 \).  

The energy balance equation is written as:  
\( 0 = \dot{m} \left( h_5 - h_6 + \frac{w_5^2 - w_6^2}{2} \right) \).  

The work term \( W_{56} \) is calculated:  
\( W_{56} = - \dot{m} \frac{w_6^2 - w_5^2}{2} \).  

The enthalpy difference is expressed as:  
\( h_5 - h_6 - W_{56} \).  

The velocity term is simplified:  
\( W_6 = \sqrt{2 \left( h_5 - h_6 - W_{56} \right)} \).  

Further simplification is shown:  
\( W_6 = \sqrt{2 c_p \left( T_5 - T_6 \right) - 2 W_{56} + \frac{w_5^2}{2}} \).  

The final result is:  
\( W_6 = 2299 \).  

Additional calculations:  
\( W_{56} = \frac{R \left( T_5 - T_6 \right)}{1 - n} \).  
\( W_{56} = -74.6 \, kJ \).  

Constants are defined:  
\( R = c_p - c_v \).  
\( R = c_p \left( 1 - \frac{1}{k} \right) = 0.2874 \, kJ/kgK \).