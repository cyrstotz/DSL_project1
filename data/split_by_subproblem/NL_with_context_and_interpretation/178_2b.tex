\( w_6 \), \( T_6 \)  

Conservation of energy:  
\( \frac{dE}{dt} = \dot{m} \left( h_5 - h_6 + \frac{1}{2} w_5^2 - \frac{1}{2} w_6^2 \right) \)  

\( \Rightarrow w_6 = \sqrt{2 \cdot \left( h_5 - h_6 \right) + w_5^2} \)  

Ideal gas:  
\( c_p \left( T_5 - T_6 \right) = \frac{1}{2} \cdot \left( w_5^2 - w_6^2 \right) \)  

\( \frac{T_6}{T_5} = \left( \frac{p_6}{p_5} \right)^{\frac{n-1}{n}} \)  

\( \Rightarrow T_6 = T_5 \cdot \left( \frac{p_0}{p_5} \right)^{\frac{n-1}{n}} \)  

\( \Rightarrow T_6 = 431.9 \cdot \left( \frac{0.181}{0.5} \right)^{\frac{1.4-1}{1.4}} \)  

\( = 328.1 \, \text{K} \)  

\( w_6 = \sqrt{2 \cdot c_p \left( T_5 - T_6 \right) + w_5^2} \)  

\( w_6 = \sqrt{2 \cdot 1.006 \cdot \left( 431.9 - 328.1 \right) + 220^2} \)  

\( = 507.2 \, \text{m/s} \)

A graph is drawn with the y-axis labeled as T in kilojoules and the x-axis labeled as S in kilojoules per kilogram.  
The graph shows a cycle with the following points and processes:  
- Point 2 to Point 3: Isobaric process.  
- Point 3 to Point 4: Isentropic process.  
- Point 4 to Point 5: Isobaric process.  
- Point 5 to Point 6: Isentropic process.  
- Point 6 to Point 2: Isobaric process.