F equals open parenthesis m subscript K plus m subscript EW close parenthesis multiplied by g.  
F equals open parenthesis 32 kilograms plus 0.1 kilograms close parenthesis multiplied by 9.81 newtons per kilogram.  
F equals 314.9 newtons.  

F subscript atm equals p subscript amb multiplied by A.  
F subscript atm equals 10 to the power of 5 pascals multiplied by open parenthesis pi multiplied by open parenthesis 0.05 meters close parenthesis squared close parenthesis.  
F subscript atm equals 1.5 newtons per square meter multiplied by open parenthesis pi multiplied by 0.05 meters squared close parenthesis.  
F subscript atm equals 785.4 newtons.  

Force equilibrium:  
F plus F subscript atm equals F subscript n.  
Therefore, p subscript g equals F subscript n divided by A.  
p subscript g equals open parenthesis 314.9 newtons plus 785.4 newtons close parenthesis divided by open parenthesis pi multiplied by open parenthesis 0.05 meters close parenthesis squared close parenthesis.  
p subscript g equals 1.4 times 10 to the power of 5 pascals.  

Ideal gas law:  
p multiplied by V equals m multiplied by R multiplied by T.  
m equals open parenthesis p subscript g multiplied by V subscript g close parenthesis divided by open parenthesis R multiplied by T subscript g close parenthesis.  

R equals R subscript u divided by M.  
R equals 8.314 joules per mole per kelvin divided by 50 kilograms per kilomole.  
R equals 0.166 joules per kilogram per kelvin.

\( m_g = \frac{V_g \cdot p_{g,1}}{R \cdot T_{g,1}} = \frac{3.14 \cdot 5}{0.466 \cdot 500} = 3.43 \, g \)  
\( = 3.43 \cdot 10^{-3} \, kg \)  

\( V_g = 3.14 \, L \)  
\( = 3.14 \cdot 10^{-3} \, m^3 \)  

\( T_{g,1} = 500^\circ \, C \)  
\( = 773.15 \, K \)