w6: reversible and adiabatic nozzle.  
Energy balance around the nozzle:  
0 = ṁ (he - ha + we squared minus wa squared divided by 2) [I]  

Entropy balance:  
0 = ṁ (Se - Sa) → Se = Sa  

For ideal polytropic state behavior:  
T6 = Ts (P6 divided by Ps) raised to the power of (n minus 1 divided by n)  
T6 = 328.07 K  

Equation I:  
2 (h6 minus h5) = w5 squared minus w6 squared divided by 2  

Ideal gas:  
Cp (T6 minus T5)  
Cp = 1.006 kJ/kg·K  
T6 = 328.07 K  
T5 = 431.9 K  

w6 squared = w5 squared minus 2 (h6 minus h5)  
w6 = square root of w5 squared minus 2 (h6 minus h5)  
w6 = 507.24 m/s  

Units are consistently converted:  
kJ/kg = J/kg divided by 1000  
m squared/s squared = J/kg  

Final velocity:  
w6 = 507.24 m/s