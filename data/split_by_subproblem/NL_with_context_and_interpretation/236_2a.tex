A T-s diagram is drawn with labeled points 0, 1, 2, 3, 4, and 5. The diagram includes the following features:  
- The x-axis is labeled as entropy (S) in kilojoules per kilogram Kelvin (kJ/kg·K).  
- The y-axis is labeled as temperature (T) in Kelvin (K).  
- A curve connects the points, with arrows indicating the direction of the process.  
- The segment from point 0 to point 1 is labeled as "isentrop."  
- The segment from point 1 to point 2 is labeled as "isobar p2 = p3."  
- The segment from point 4 to point 5 is labeled as "isobar p0 = p5."  

A smaller inset diagram is drawn to the right, showing a simplified T-s process with points 0, 1, and 2 labeled.  

Below the diagram, a table titled "Zustandstabelle" (State Table) is provided. The columns are labeled as follows:  
- Zustand (State)  
- p [bar] (Pressure in bar)  
- T [K] (Temperature in Kelvin)  
- \( \dot{m} \) [kg/s] (Mass flow rate in kilograms per second)  
- s (Entropy)  
- h (Enthalpy)  
- w [m/s] (Velocity in meters per second)  

The rows contain the following information:  
- State 1: No data provided.  
- State 2: p2 = p3.  
- State 3: p3 = p2.  
- State 4: No data provided.  
- State 5: p = 0.5 bar, T = 431.9 K, w = 220 m/s.  
- State 6: No data provided.  
- State 0: p = 0.191 bar, T = 243.15 K, s0 = s1, w = 200 m/s.  

No additional information is provided in the table.