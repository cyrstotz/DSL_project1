A12 @ 8 bar  
h₃ = hₓ (sat) + s₂ - sₓ (sat) / (sₓ (90°C) - sₓ (sat)) * (hₓ (90°C) - hₓ (sat))  
h₃ = 269.15 kJ/kg + 0.9298 - 0.7046 / (0.9374 - 0.7046) * (273.6 kJ/kg - 269.15 kJ/kg)  
h₃ = 271.31 kJ/kg

m equals W subscript K divided by h subscript 3 minus h subscript 1 equals negative 28 watts divided by (237.74 kilojoules per kilogram minus 271.31 kilojoules per kilogram) equals 0.000835 kilograms per second.  

c)  
h subscript 4 equals h subscript 4.  
h subscript 4 at AM at 8 bar, x equals 0.  
h subscript f (8 bar) equals 93.42 kilojoules per kilogram.  

d equals epsilon subscript K equals absolute value of Q dot subscript zu divided by absolute value of W dot subscript H equals negative 28 watts.  
Q dot subscript zu equals Q dot subscript K.  
m dot times (h subscript 1 minus h subscript 4) plus Q dot subscript K equals 0.  
0.000835 kilograms per second times (93.42 kilojoules per kilogram minus 237.74 kilojoules per kilogram) plus Q dot subscript K equals 0.  
Q dot subscript K equals 119.8 watts.  
epsilon subscript K equals 4.27.  

T subscript 1 equals negative 16 degrees Celsius.  
A10 at negative 16 degrees Celsius.  
h subscript 1 equals h subscript f (negative 16 degrees Celsius) plus x subscript 1 times (h subscript g (negative 16 degrees Celsius) minus h subscript f (negative 16 degrees Celsius)).  
93.42 kilojoules per kilogram minus 29.3 kilojoules per kilogram divided by (237.79 kilojoules per kilogram minus 29.3 kilojoules per kilogram) equals x subscript 1 equals 0.31.  

T subscript 1 equals T subscript 2.