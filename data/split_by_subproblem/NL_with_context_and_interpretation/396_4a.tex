\( T_1 = \)  
\( p_1 = \)  
\( h_1 = 93.42 \, \text{kJ/kg} \)  

\( S_2 = \)  
\( T_2 = \)  
\( p_2 = p_1 \)  
\( x_2 = 1 \)  

\( S_3 = \)  
\( T_3 = \)  
\( p_3 = 8 \, \text{bar} \)  

\( T_4 = 31.33^\circ \text{C} \)  
\( p_4 = 8 \, \text{bar} \)  
\( x_4 = 0 \)  
\( h_4 = 93.42 \, \text{kJ/kg} \)  

\( \dot{m}_{R134a} (h_2 - h_3) = \dot{W}_K \)

Two diagrams are drawn, both showing phase regions in a pressure-temperature (p-T) graph.

1. **First Diagram**:
   - The x-axis is labeled as 'T (°C)' for temperature.
   - The y-axis is labeled as 'p (bar)' for pressure.
   - Three phase regions are marked: 'solid', 'fluid', and 'gas'.
   - A line labeled 'Tripel' (triple point) intersects the three regions.
   - Two points are marked: '1' and '2', positioned along the curve separating the solid and fluid regions.

2. **Second Diagram**:
   - The x-axis is labeled as 'T (°C)' for temperature.
   - The y-axis is labeled as 'p (bar)' for pressure.
   - Three phase regions are marked: 'solid', 'fluid', and 'gas'.
   - A curve separates the solid and fluid regions, and another curve separates the fluid and gas regions.
   - Two points are marked: '1' and '2', positioned along the curve separating the solid and fluid regions.