A graph is drawn with temperature (T) on the vertical axis and entropy (S) on the horizontal axis. The graph represents a thermodynamic process with labeled points:  
- Point 1 to Point 2: Isobar (constant pressure).  
- Point 2 to Point 3: Isentropic (constant entropy).  
- Point 3 to Point 4: Isobar (constant pressure).  
- Point 4 to Point 5: Isentropic (constant entropy).  
- Point 5 to Point 6: Isobar (constant pressure).  
The axes are labeled as follows:  
- Vertical axis: "T K".  
- Horizontal axis: "S kJ/kg".

0 equals m multiplied by (h zero minus h six plus w zero squared minus w six squared divided by 2).  

w six squared divided by 2 equals h zero minus h six plus w zero squared divided by 2.  

w six equals the square root of 2 multiplied by (h zero minus h six) plus w zero squared.  

w six equals 193 meters per second.  

w zero equals the square root of 2 multiplied by c p multiplied by (T zero minus T six) plus w zero squared.  

T zero equals negative 30 degrees Celsius.  

T six equals 323.074 Kelvin.  

c p equals 1.006 kilojoules per kilogram Kelvin.  

---