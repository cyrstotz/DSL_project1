A table is drawn with columns labeled \( p \) (pressure) and \( T \) (temperature). The rows contain the following data:  

- Row labeled "Zustand 0": \( p = 0.191 \, \text{bar} \), \( T = -30^\circ \text{C} \).  
- Row labeled "1": \( p_1 \), \( T_1 \).  
- Row labeled "2": \( p_2 \), \( T_2 \).  
- Row labeled "3": \( p_2 \), \( T_3 \).  
- Row labeled "4": \( p = 0.5 \, \text{bar} \), \( T_4 \).  
- Row labeled "5": \( p = 0.5 \, \text{bar} \), \( T = 431.9 \, \text{K} \).  
- Row labeled "6": \( p = 0.191 \, \text{bar} \), \( T_6 \).  

Below the table, a diagram is drawn.  

The diagram is a \( T \)-\( s \) graph (temperature vs. entropy).  
- The vertical axis is labeled \( T \, (\text{K}) \).  
- The horizontal axis is labeled \( s \, (\text{J/kg·K}) \).  

The graph shows a series of points connected by lines:  
- Point 0 to Point 1 is labeled "Isobar".  
- Point 1 to Point 2 is labeled "Isobar".  
- Point 2 to Point 3 is labeled "Isobar".  
- Point 3 to Point 4 is labeled "Isobar".  
- Point 4 to Point 5 is labeled "Isobar".  
- Point 5 to Point 6 is labeled "Isobar".  

The graph visually represents the thermodynamic process described in the task.