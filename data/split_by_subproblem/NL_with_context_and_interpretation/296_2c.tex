The kinetic energy (KE) equation is written as:  
\( KE = W = \int_{5}^{6} p \, dV \).  
Another equation is written:  
\( \frac{w^2}{2} = \frac{1}{\kappa - 1} (T_2 - T_1) \).  
Substituting values:  
\( \frac{w^2}{2} = c_p \cdot \Delta T = 92.95 \, \text{kJ/kg} \).  
Finally, \( w_6 = 430 \, \text{m/s} \).

Delta ex_str equals ex_str,6 minus ex_str,0.  
ex_str equals h minus h_0 minus T_0 times (s minus s_0) plus ke.  
ex_str,0 equals h_6 minus h_0 minus T_0 times (s_6 minus s_0) plus ke_6 minus ke_0.  

Equals Cp times Delta T_6,0 minus T_0 times Cp times ln(T_6 divided by T_0) minus T_0 times 2 times ln(p_6 divided by p_0).  
p_1, p_6 equals p_0.  

Equals Cp times Delta T_6,0 minus T_0 times ln(T_6 divided by T_0) plus ke.  
Equals 1.006 kilojoules per kilogram Kelvin times (310 Kelvin minus (273.15 plus 30 Kelvin) minus (273.15 plus 30 Kelvin) times ln(310 divided by (273.15 plus 30 Kelvin))) plus (w_6 squared minus w_0 squared divided by 2).  

Equals 67.126 kilojoules per kilogram.