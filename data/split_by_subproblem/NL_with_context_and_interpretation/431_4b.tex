The equation is written:  
0 equals m-dot times (h2 minus h3) plus Q-dot minus W-dot.  

Below this, calculations are shown:  
- h2 equals hfg plus hf at T equals 27.7 Kelvin, taken from Table A-10.  
- h2 equals 295.55 kilojoules per kilogram.  
- s2 equals s3 equals sg at T equals 27 degrees Celsius, equals 0.9163 kilojoules per kilogram Kelvin.  

Ti equals 10 degrees Celsius equals 283.15 Kelvin.  

Pressure is constant:  
p1 equals 1 millibar equals p2.  

T2 equals 283.15 Kelvin minus 6 Kelvin equals 277.15 Kelvin.  

Further calculations are partially crossed out and unclear:  
h equals hf plus hfg equals sf minus sfg times (hfg minus hf) plus hg equals ...  

No further clear content is visible.