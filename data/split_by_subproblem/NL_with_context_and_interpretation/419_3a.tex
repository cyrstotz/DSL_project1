The pressure of the gas \( p_g^1 \) is calculated using the formula:  
\( p_g = \frac{m_K \cdot g}{A} + \frac{m_{EW} \cdot g}{A} \).  

Substituting values:  
\( p_g = \frac{32 \cdot 9.81}{0.00785} + \frac{0.1 \cdot 9.81}{0.00785} = 140144 \).  
Thus, \( p_g^1 = 1.401 \, \text{Bar} \).  

The ideal gas law \( pV = mRT \) is used to calculate the mass of the gas:  
\( R = \frac{\hat{R}}{M} = \frac{8.314}{50} = 0.16628 \).  

The mass of the gas \( m_g \) is calculated as:  
\( m_g = \frac{p \cdot V}{R \cdot T} = \frac{1.404 \cdot 10^5 \cdot 3.14 \cdot 10^{-3}}{0.16628 \cdot (773.15)} = 0.00342 \, \text{kg} \).  
This corresponds to \( 3.42 \, \text{g} \).  

---