Two diagrams are drawn:

1. The first diagram is a pressure-temperature (P-T) graph.  
   - The graph has a curved line with labeled points 1, 2, 3, and 4 forming a closed loop.  
   - The curve appears to represent a thermodynamic cycle.  
   - The axes are labeled: the vertical axis is pressure (P), and the horizontal axis is temperature (T).  

2. The second diagram is another pressure-temperature (P-T) graph.  
   - The graph includes a curved line separating phase regions.  
   - The regions are labeled as "fest" (solid), "flüssig" (liquid), and "gas/förmig" (gas/vapor).  
   - A vertical line labeled "i" is drawn within the liquid region, and another vertical line labeled "ii" is drawn within the gas/vapor region.  
   - The axes are labeled: the vertical axis is pressure (P), and the horizontal axis is temperature (T).