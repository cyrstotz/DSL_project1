The temperature \( T_0 \) is given as \( -30^\circ \text{C} = 243.15 \, \text{K} \).  

The following equations are written:  
1. \( p_0 V_0 = R T_0 \) implies \( V_0 = \frac{R T_0}{p_0} \)  
2. \( R = c_p - c_v = c_p \frac{\kappa - 1}{\kappa} = 0.287429 \, \text{kJ}/\text{kg·K} \)  
3. \( V_0 = 3.65908 \, \text{m}^3/\text{kg} \)  

Additional derivations are shown:  
4. \( T_6 = T_0 \left( \frac{p_6}{p_0} \right)^{\frac{\kappa - 1}{\kappa}} \)  
5. \( T_6 = \frac{R T_6}{p_6 V_0} \)  
6. \( T_6 = \frac{R}{p_6 V_0} T_6^{\kappa - 1} \)  

Several crossed-out sections are present, but the above equations remain visible and intact.

\( T_6 / T_5 = (p_6 / p_5)^{(k-1)/k} \)  
Therefore, \( T_6 = T_5 \cdot (p_6 / p_5)^{(k-1)/k} = 328.0747 \, \text{K} \)  

\( S_5 = S_6 \)  

\( 0 = \dot{m}_{\text{gas}} \left[ h_5 - h_6 + \frac{w_5^2 - w_6^2}{2} \right] - \dot{W}_{\text{turbine}} \)