The equation for the first law of thermodynamics is written:  
0 equals m dot times (h four minus h one) plus epsilon Q minus epsilon W.  

Below it, the mass flow rate equation is written:  
m dot equals W dot divided by (h three minus h two).

\( T_i = -10^\circ \text{C} \)  
\( T_2 = T_i \)  
Interpolated from Table A-10:  
\( h_{12} = \frac{h_g(-8^\circ \text{C}) - h_g(-12^\circ \text{C})}{-8^\circ \text{C} - (-12^\circ \text{C})} \cdot (-8^\circ \text{C} - (-10^\circ \text{C})) + h_g(-12^\circ \text{C}) = 264.15 \, \text{kJ/kg} \)  

\( h_g \) from Table A-12:  
\( p(8 \, \text{bar}) \, h_g(8 \, \text{bar}) = h_{12} = 264.15 \)  

\( \dot{m} = \frac{28 \, \text{g}}{5 \, \text{s}} = \frac{28 \times 10^{-3} \, \text{kg}}{5 \, \text{s}} = 1.4 \times 10^{-2} \, \text{kg/s} \)  
\( \dot{m} = \frac{1.4 \times 10^{-2} \, \text{kg/s}}{264.15 - 243.7} = 7.368 \times 10^{-4} \, \text{kg/s} = 4.9742 \, \text{kg/h} \)