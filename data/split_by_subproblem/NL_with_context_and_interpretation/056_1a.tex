ΔE equals m2 times u2 minus m1 times u1 equals Δm12 times h_in plus ΣQ.  

Δm12 equals m1 plus m2 minus m1 equals Δm12 times h_in plus ΣQ.  

ΣQ equals Q_out equals -Q_out.  

Δm12 equals m2 times u2 minus m1 times u1 minus ΣQ divided by h_in.  

h_in equals h_f at 20 degrees Celsius equals 2538.4 kilojoules per kilogram.  

Δm12 equals (Δm12 plus m1) times u2 minus m1 times u1 minus ΣQ divided by h_in.  

m1 equals 5755 kilograms.  

m2 equals Δm12 plus m1.  

ΣQ equals -35 megajoules equals -Q_out.  

Δm12 equals u2 plus m1 times u2 minus m1 times u1 divided by h_in minus u2.  

Δm12 equals 17.1955 kilograms (??).  

u2 equals u_f at 70 degrees Celsius equals 246.96.  

u_f at 100 degrees Celsius equals 419.38.  

u1 equals u_f plus x times (u_g minus u_f) at 100 degrees Celsius.  

u_g equals 2506.5.  

u_f equals 468.94.  

u1 equals 429.38.

The inlet mass flow rate \( \dot{m}_{in} \) is equal to 0.3 kilograms per second.  
The system is stationary (steady-state flow).  

The energy balance equation is:  
\( \Sigma \dot{Q} = \dot{Q}_R - \dot{Q}_{out} \).  
(Note: \( \dot{Q}_{out} \) is negatively defined.)  

The heat flow equation is:  
\( \dot{Q}_{out} = \dot{m}_{in} (h_{in} - h_{out}) + \Sigma \dot{Q} \).  

The enthalpy values are determined as follows:  
\( h_{in} = h_g(70^\circ C) = 2676.8 \, \text{kJ/kg} \).  
\( h_{out} = h_g(100^\circ C) = 2676.1 \, \text{kJ/kg} \) (from Table A-2).  

The heat flow \( \dot{Q}_{out} \) is calculated as:  
\( \dot{Q}_{out} = 85.24 \, \text{kW} \).  

---