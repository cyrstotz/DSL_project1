A diagram is drawn showing a cylinder with two chambers separated by a membrane. The upper chamber contains an insulating piston labeled with downward forces \( p_{\text{atm}} \) and \( m \cdot g \), and upward forces labeled \( p_1 \). The cylinder diameter is labeled as \( D = 10 \, \text{cm} \).  

Equations:  
\( F_g = m \cdot g = 31.319 \, \text{N} \)  
\( F_{\text{atm}} = p \cdot (5 \, \text{cm})^2 \cdot \pi = p \cdot (0.005 \, \text{m})^2 \cdot \pi = 7.854 \, \text{N} \)  

Force equilibrium:  
\( F_{p_1} = F_g + F_{\text{atm}} \Rightarrow p_1 = \frac{F_g + F_{\text{atm}}}{(0.005 \, \text{m})^2 \cdot \pi} = 40.96 \cdot 10^5 \, \text{Pa} \)  
\( p_1 = 40.96 \, \text{bar} \)  

Gas mass calculation:  
\( m_g = \frac{R}{M} = \frac{8.314 \, \text{J/mol·K}}{50 \, \text{kg/kmol}} = 0.16628 \, \text{kJ/kg·K} \)  

Using \( pV = mRT \):  
\( m = \frac{p_1 \cdot V_1}{R \cdot T_1} = \frac{40.96 \cdot 10^5 \, \text{Pa} \cdot 3.14 \cdot 10^{-3} \, \text{m}^3}{0.16628 \, \text{kJ/kg·K} \cdot 773.15 \, \text{K}} = 99.27 \, \text{kg} \)  

Note:  
"Since I thought my pressure was incorrect, I continued calculating with the given values."