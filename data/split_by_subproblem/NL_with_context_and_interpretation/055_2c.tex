The equation is rewritten for the entire system (states 0-6):  
\( 0 = \dot{m} \left[ h_e - h_a + \frac{w_e^2 - w_a^2}{2} + g(z_e - z_a) \right] + \dot{Q} - \sum \dot{W} \).  

Substituting:  
\( h_6 = h_5 - h_c(T_6) = c_p \cdot (T_0 - T_6) = 1.006 \, \text{kJ/kg·K} \cdot (243.15 - T_6) \).  

Final equation:  
\( 0 = 1.006 \cdot (243.15 - T_6) + \frac{200^2 - w_6^2}{2} \).  
\( 0 = h \cdot 1.006 \cdot (431.9 - T_6) + \frac{220^2 - w_6^2}{2} \).

Delta exergy equals exergy6 minus exergy0.  
Delta exergy equals h6 minus h0 minus T0 multiplied by (s6 minus s0) plus kinetic energy plus potential energy minus w6 squared minus w0 squared divided by 2.  

p equals p0 isobar, adiabatic.  
s6 minus s0 equals Cp multiplied by ln(T6 divided by T0) minus R multiplied by ln(p6 divided by p0).  
s6 minus s0 equals 1.006 multiplied by ln(328.0747 divided by 243.15) minus 0.287278 multiplied by ln(1 divided by 0.191).  
s6 minus s0 equals 0.337278 kilojoules per kilogram kelvin.  

h6 minus h0 equals Cp multiplied by (T6 minus T0) equals 1.006 multiplied by (340 minus 243.15).  
h6 minus h0 equals 97.4311 kilojoules per kilogram.  

Delta exergy equals 97.4311 kilojoules per kilogram minus (243.15K multiplied by 0.337278 kilojoules per kilogram kelvin) plus 510 squared minus 200 squared divided by 2 divided by 1000.  
Delta exergy equals 100.85 plus 15.42 plus 100.05.  
Delta exergy equals 125.47 kilojoules per kilogram.