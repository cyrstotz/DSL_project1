Two times (h six minus h five) minus w five squared equals minus w six squared.  
Minus two hundred eight minus two hundred twenty squared divided by two equals minus w six squared.  
w six equals four hundred eighty-one point six meters per second.

A table is drawn with columns labeled \( P \), \( U \), \( T \), and \( T \). The rows contain the following data:  
- Row 0: \( 0.191 \, \text{bar} \), \( -30^\circ \text{C} \).  
- Row 5: \( 0.5 \, \text{bar} \), \( 431.9 \, \text{K} \), \( w_5 = 220 \, \text{m/s} \).  
- Row 6: \( p_6 \).  

Below the table, the following equations and calculations are written:  

"Stationäre 1HS um Schubdüse:"  
\( w_6^2 = w_5^2 + 2(h_5 - h_6) \).  

"Reversible adiabate Schubdüse:"  
\( \frac{T_6}{T_5} = \left( \frac{p_6}{p_5} \right)^{\frac{n-1}{n}} \), where \( n = 1.4 \).  

\( h_5 - h_6 = c_p(T_5 - T_6) \).  
\( = 1.006 (431.5 - 320) \).  
\( = 104 \, \text{kJ/kg} \).  

\( T_6 = T_5 \cdot \left( \frac{p_6}{p_5} \right)^{\frac{n-1}{n}} \).  
\( = 320 \, \text{K} \).