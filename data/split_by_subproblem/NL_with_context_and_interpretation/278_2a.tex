0 → 1: adiabatic → Venturi tube  
1 → 2: isentropic  
2 → 3: isobaric  
3 → 4: adiabatic, irreversible  
4 → 5: mixing, isobaric  
5 → 6: isentropic  

\( p_5 = p_6 = p_0 = 0.5 \, \text{bar} \)  

Diagram:  
A T-s diagram is drawn with labeled points 0, 1, 2, 3, 4, 5, and 6.  
- 0 → 1: adiabatic  
- 1 → 2: isentropic  
- 2 → 3: isobaric  
- 3 → 4: adiabatic, irreversible  
- 4 → 5: mixing, isobaric  
- 5 → 6: isentropic  

The axes are labeled:  
- Vertical axis: \( T \, [\text{K}] \)  
- Horizontal axis: \( s \, [\text{kJ/kg·K}] \)

The first diagram is labeled "a)" and shows a qualitative T-s diagram. The x-axis is labeled as "T [K]" and the y-axis is labeled as "p [bar]". The diagram consists of multiple curved lines, resembling cycles or loops, with no additional annotations or labels on the curves themselves.