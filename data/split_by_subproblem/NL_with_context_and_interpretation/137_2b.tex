The first law of thermodynamics is applied:  
"1HS: Stationary = \( Q = 0 \), \( W = 0 \)"  
The equation is written as:  
\( 0 = \dot{m} \cdot (h_5 - h_6 + w_5^2 / 2 - w_6^2 / 2) \)  
This is expanded to:  
\( = \dot{m} \cdot c_p \cdot (T_5 - T_6) + w_5^2 / 2 - w_6^2 / 2 \)  
\( w_6 = \sqrt{w_5^2 + 2 \cdot c_p \cdot (T_5 - T_6)} \)

Additional notes and equations:  
- \( p_5 \cdot V = m \cdot R \cdot T_5 \)  
- \( T_5 = 431.9 \, K \)  
- \( p_5 = 0.5 \, bar \)  
- \( p_5 \cdot V_5 = R \cdot T_5 \)  
- \( V_5 = R \cdot T_5 / p_5 \)  
- \( R = Q / M \)  
- \( M = 28.97 \, kg/kmol \)  
- \( R = 0.287 \, kJ/kg·K \)  
- \( c_p = 1.006 \, kJ/kg·K \)  
- \( k = 1.4 \)

The temperature difference is calculated:  
\( T_6 = T_5 \cdot \left( \frac{p_6}{p_5} \right)^{\frac{k-1}{k}} \)  
\( T_6 = 431.9 \cdot \left( \frac{0.191}{0.5} \right)^{0.4/1.4} \)  
\( T_6 = 328.07 \, K \)

The velocity \( w_0 \) is calculated:  
\( w_0 = \sqrt{(200 \, m/s)^2 + 1.006 \, kJ/kg·K \cdot (431.9 - 328.07) \, K} \)  
\( w_0 = 507.25 \, m/s \)