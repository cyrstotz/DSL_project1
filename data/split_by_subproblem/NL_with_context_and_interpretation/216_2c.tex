A T-s diagram is drawn with the y-axis labeled as \( T \) (temperature) and the x-axis labeled as \( s \) (entropy). Several isobars are drawn and labeled:  
- \( p_0 = p_6 \)  
- \( p_1 \)  
- \( p_2 = p_3 \)  
- \( p_4 = p_5 \)  

The diagram includes numbered points (0, 1, 2, 3, 4, 5, 6) connected by arrows to represent the thermodynamic processes.

Δexstr = h6 - h0 - T0(s6 - s0) + w6 squared / 2 - w0 squared / 2  
p6 = p0  

Δexstr = cp(T6 - T0) - T0(cp ln(T6 / T0)) + w6 squared / 2 - w0 squared / 2  
Δexstr = 46.533 kilojoules per kilogram