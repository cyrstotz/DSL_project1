A rectangular control volume is drawn with arrows indicating flow entering at point 0 and exiting at point 6.  

The following equation is written:  
dE/dt = Σiṁi(h0 + k0 + p0 + trel(0)) + Σiṁj(h1 + k1 + p1 + trel(1)) - Σiṁu(h1)  

The following derivation is shown:  
O = vges(h0 + k0 - h6 - k6) => h6 = h0 - k0 - h6  
k0 = w squared divided by 2 = (200 meters per second) squared divided by 2 = 20,000 meters squared per second squared  
k6 = w squared divided by 2 = w6 squared divided by 2  

h0 - h6 = integral from T0 to T6 of cp dT = 7.066 kilojoules per kilogram Kelvin (-243.15 Kelvin + 328.07 Kelvin) + 85.925 kilojoules per kilogram  

T6/T0 = (p6/p0) raised to the power of (kappa minus 1 divided by kappa)  
=> T6 = 431.9 Kelvin (0.196 bar divided by 0.5 bar) raised to the power of (0.4 divided by 1.4) = 328.07 Kelvin  

328.07 Kelvin is underlined.

\( w_6 = \sqrt{2 \cdot (h_0 - h_6 + h_w)} \)  
\( = \sqrt{2 \cdot (85.4285 \, \text{J/kg} + 20,000 \, \text{m}^2/\text{s}^2)} \)  
\( = 49.19 \, \text{m/s} \)  

\( m^2/\text{s}^2 = \text{N·m}/\text{kg} = \text{J}/\text{kg} \)  

---