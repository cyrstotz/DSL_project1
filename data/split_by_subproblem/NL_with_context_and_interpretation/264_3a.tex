The equation for pressure is given as:  
pV = mRT  
Where R = R divided by M.  
R = 8.314 kilojoules per kilomole Kelvin divided by 50 kilograms per kilomole = 0.16628 kilojoules per kilogram Kelvin.  

The area is calculated as:  
A = πr squared = π(5 × 10^-2 meters) squared = 7.853981 × 10^-3 square meters.  

The pressure p_s,1 is calculated as:  
F divided by A plus p_amb.  
F = 32 kilograms × 9.81 meters per second squared = 32 kilograms × 9.81 meters per second squared divided by (7.853981634 × 10^-3 square meters) plus 100,000 newtons per square meter.  
= 139,969.538 newtons per square meter.  
= 1.3967 bar.  

The mass of the gas is calculated as:  
pV divided by RT = m_g.  
(139,969.538 pascals × (3.14 × 10^-3 cubic meters)) divided by (0.16628 kilojoules per kilogram Kelvin × (500 + 273.15) Kelvin).  
= 3.418687423 newtons meters divided by newtons meters per kilogram.  
m_g = 3.41879 kilograms.