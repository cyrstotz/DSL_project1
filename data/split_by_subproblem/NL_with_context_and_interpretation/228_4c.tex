The title "TAS AM" is written. A table is drawn with columns labeled as:  
- P  
- T  
- s  
- h  
- x  

The rows contain the following data:  
1. \( 1.2132 \)  
2. \( -22^\circ \text{C} \)  
3. \( 8 \text{bar} \)  
4. \( 8 \text{bar}, 81.33, 0.3458, 83.42 \)  

Additional notes:  
- \( x_2 = 1 \)  
- Processes labeled as:  
  - "isobar"  
  - "isochor"  
  - "adiabatic"  
  - "isobar"  

No further content is visible.

x2 equals 1.  
T2 equals minus 22 degrees Celsius.  

From Table A-10:  
p2 equals 1.2192 bar.  
hg equals 234.08 kilojoules per kilogram.  
sg2 equals 0.8351 kilojoules per kilogram per Kelvin.  

T:  
For isothermal process, p2 equals p1 equals 1.2192 bar.  

xA equals (s minus sf) divided by (sg minus sf), which equals (s minus sf) divided by (sg minus sf).