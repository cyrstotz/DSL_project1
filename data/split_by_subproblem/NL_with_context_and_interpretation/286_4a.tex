Three diagrams are drawn, each labeled with axes and points:  

1. The first diagram has pressure (p) on the vertical axis and temperature (T) on the horizontal axis. It shows a curve with labeled points 1, 2, 3, and 4. The regions are marked as "gas" and "flüssig" (liquid). There is a line labeled "subl." (sublimation) and another labeled "verd." (evaporation).  

2. The second diagram also has pressure (p) on the vertical axis and temperature (T) on the horizontal axis. It shows a rectangular cycle with points labeled 1, 2, 3, and 4. The regions are marked as "gas" and "flüssig" (liquid).  

3. The third diagram has pressure (p) on the vertical axis and temperature (T) on the horizontal axis. It shows a straight line connecting points 1, 2, 3, and 4. The regions are marked as "gasförmig" (gaseous) and "flüssig" (liquid).