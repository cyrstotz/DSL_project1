Determine \( p_{g,1} \) and \( m_g \):  
The gas pressure is calculated using the total mass:  
\( m_{EW} + m_K + m_{piston} = 0.1 \, \text{kg} + 32 \, \text{kg} = 32.1 \, \text{kg} \).  

Force \( F_g = m_g \cdot g = 32.1 \, \text{kg} \cdot 9.81 \, \text{m/s}^2 = 314.901 \, \text{N} \).  
Pressure \( p = p_0 + \frac{F_g}{A} = p_{\text{atm}} + \frac{F_g}{\pi \cdot (0.05 \, \text{m})^2} \).  

\( p_{g,1} = 1.401 \, \text{bar} \).  

For an ideal gas:  
\( m_g = \frac{p_{g,1} \cdot V_{g,1}}{R_g \cdot T_{g,1}} \).  

With \( R_g = \frac{8.314 \, \text{m}^3 \cdot \text{Pa}/\text{mol} \cdot \text{K}}{50 \, \text{kg}/\text{kmol}} = 0.16623 \, \text{J}/\text{g} \cdot \text{K} \),  
\( m_g = \frac{1.401 \cdot 10^5 \, \text{Pa} \cdot 0.00314 \, \text{m}^3}{0.16623 \, \text{J}/\text{g} \cdot \text{K} \cdot (500 + 273.15) \, \text{K}} = 3.422 \, \text{g} \).  

---