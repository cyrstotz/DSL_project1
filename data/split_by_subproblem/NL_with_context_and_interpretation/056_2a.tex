\( w_{\text{Luft}} = 200 \, \text{m/s} \)  
\( p_0 = 0.191 \, \text{bar} \)  
\( T_0 = -30^\circ \text{C} \)  

\( q_B = \frac{\dot{Q}_B}{\dot{m}_K} \)  

\( \frac{\dot{m}_M}{\dot{m}_K} = 5.293 \)  

\( c_{pL} = c_{pL} = 1.006 \, \text{kJ/kg·K} \)  
\( n = 1.4 \)  

### Diagram Description:  
A T-s diagram is drawn with labeled isobars and processes. The x-axis is labeled \( s \, \left[ \frac{\text{kJ}}{\text{kg·K}} \right] \), and the y-axis is labeled \( T \, (\text{K}) \).  
The following states and processes are marked:  
- \( 0 \to 1 \): \( \eta_{V,s} < 1 \)  
- \( 1 \to 2 \): Isentropic  
- \( 2 \to 3 \): Isobaric (increase in temperature)  
- \( 3 \to 4 \): \( \eta_{T,s} < 1 \)  
- \( 4 \to 5 \): Isobaric (\( p_4 = p_5 = 0.5 \, \text{bar} \))  
- \( 5 \to 6 \): Isentropic  

The isobars \( p_2 = p_3 \), \( p_4 = p_5 \), and \( p_0 \) are clearly labeled.