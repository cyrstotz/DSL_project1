T-s diagram:  
The diagram shows a qualitative representation of the thermodynamic process in a jet engine. It includes labeled isobars and process states:  
- States 1, 2, 3, 4, 5, and 6 are marked along the curve.  
- The axes are labeled as T (temperature in Kelvin) and s (entropy).  
- The diagram includes arrows indicating the direction of the process and isobaric lines.  
- A smaller inset diagram is drawn to the right, showing a detailed section with labeled states and heat transfer (\( \dot{Q}_B \)) at state 3.

Ideal gas law:  
p times V equals R times T.  

From this:  
p subscript 0 times V subscript 0 equals R times T subscript 0.  
p subscript 6 times V subscript 6 equals R times T subscript 6.  

This leads to:  
m equals p subscript 0 divided by V subscript 0.  
m equals p subscript 6 divided by V subscript 6.  

From this:  
p subscript 0 divided by p subscript 6 equals T subscript 0 divided by T subscript 6.  
Rearranging:  
p subscript 0 divided by R times T subscript 0 equals p subscript 6 divided by R times T subscript 6.  
p subscript 0 divided by T subscript 0 equals p subscript 6 divided by T subscript 6.  

Since mass flow rate dot m equals rho times A times w, and rho subscript 6 equals rho subscript 0, A subscript 6 equals A subscript 0.  

Thus:  
w equals m divided by rho times A.  
w subscript 6 equals dot m divided by rho subscript 6 times A subscript 6.  

This leads to:  
w subscript 0 equals dot m divided by rho subscript 0 times A subscript 0.  

Calculations:  
p subscript 0 equals p subscript 0 divided by R times T subscript 0 equals 0.1916 bar divided by (0.3844 kilojoules per kilogram Kelvin times 243.15 Kelvin) equals 2.7332.  

p subscript 6 equals p subscript 6 divided by R times T subscript 6 equals 0.1916 bar divided by (0.3844 kilojoules per kilogram Kelvin times 238.15 Kelvin) equals 2.0217.  

No further content found.