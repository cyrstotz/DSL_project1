T subscript G equals T subscript 5 multiplied by the ratio of p subscript c to p subscript 5 raised to the power of (kappa minus 1 divided by kappa).  
T subscript G equals 328.05 K.  

w subscript G equals square root of (2 multiplied by (p subscript 6 minus p subscript 5) divided by rho subscript 5).  

v subscript G equals R multiplied by T subscript 0 divided by p subscript 0.  
v subscript G equals 3.66 cubic meters per kilogram.  

c subscript p, Luft equals 1.006 kilojoules per kilogram Kelvin.  
kappa equals 1.4.  
R equals c subscript p minus c subscript v.  
R equals 0.257 kilojoules per kilogram Kelvin.  

Negative 1.1 multiplied by 10 to the power of 5 multiplied by p subscript 0 multiplied by (9.81 cubic meters per kilogram minus 3.66 cubic meters per kilogram) minus w subscript 5 squared divided by 2 multiplied by w subscript 2000 squared divided by 2 equals negative 1.5 multiplied by ln (4.56 cubic meters per kilogram minus 3.66 cubic meters per kilogram).  

w subscript 2000 squared equals 51,7829.  
w subscript c equals 321.8 meters per second.  

---