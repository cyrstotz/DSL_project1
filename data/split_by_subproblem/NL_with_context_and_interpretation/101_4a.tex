A table is drawn with the following columns:  
- \( p \)  
- \( T \)  
- \( V \)  
- \( w \)  
- \( \omega \)  
- \( \dot{Q} \)  

The rows are labeled as follows:  
1. \( p_1 < 8 \, \text{bar} \), \( T_i - 6 \, \text{K} \), \( \dot{Q}_K \), gaseous, saturated.  
2. \( p_1 = p_2 \), \( T_i - 6 \, \text{K} \), \( \dot{Q}_K \), gaseous, saturated.  
3. \( 8 \, \text{bar} \), liquid.  
4. \( 8 \, \text{bar} \), liquid.  

Below the table, diagrams are drawn:  

1. The first diagram is a pressure-temperature (\( p \)-\( T \)) graph.  
   - The x-axis is labeled \( T \, (\text{kelvin}) \).  
   - The y-axis is labeled \( p \, (\text{bar}) \).  
   - A saturation curve is drawn, and points labeled "1" and "2" are marked on the curve.  
   - A horizontal line connects points "1" and "2".  

2. The second diagram is another \( p \)-\( T \) graph.  
   - The x-axis is labeled \( T \, (\text{kelvin}) \).  
   - The y-axis is labeled \( p \, (\text{bar}) \).  
   - A saturation curve is drawn, and points labeled "1", "2", "3", and "4" are marked.  
   - A horizontal line connects points "1" and "2".  

3. The third diagram is a \( T \)-\( s \) graph.  
   - The x-axis is labeled \( T \, (\text{K}) \).  
   - The y-axis is labeled \( s \).  
   - A dome-shaped curve is drawn.  
   - Points labeled "1", "2", "3", and "4" are marked.  
   - The processes are labeled as follows:  
     - "1 to 2" is isobaric.  
     - "2 to 3" is isentropic.  
     - "3 to 4" is isobaric.