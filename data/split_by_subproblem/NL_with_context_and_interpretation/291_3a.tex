The pressure of the gas \( p_{g1} \) and the mass of gas \( m_g \) are calculated as follows:  
\( M_g = 50 \, \text{kg/kmol} \)  
\( A = R_{CC} = D^2 \pi / 4 \)  
\( p_{g1} = m_g R T_g \)  
\( p_{g1} = \frac{M_K \cdot g}{A} + \frac{M_{EW} \cdot g}{A} + p_{amb} = \frac{g}{D^2 \pi / 4} (M_K + M_{EW}) + p_{amb} = 14.0094 \)  
\( p_{g1} = 14 \, \text{bar} \)  

The mass of the gas \( m_g \) is calculated as:  
\( m_g = \frac{p_1 V_1}{R T_1} \)  
\( R = \frac{R}{M_g} = \frac{8.314 \, \text{kJ/kmol·K}}{50 \, \text{kg/kmol}} = 0.16628 \, \text{kJ/kg·K} \)  
\( m_g = 0.003422 \, \text{kg} = 3.422 \, \text{g} \)  

---