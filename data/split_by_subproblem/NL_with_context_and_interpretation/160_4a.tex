Water in food  

| p          | T          |  
|------------|------------|  
| 1: p1 = p2 | T > Ti     |  
| 2: p2 = p3 | T = Ti     |  
| 3: p3 < T(p) | T = Ti   |  

Graph 1:  
A pressure-temperature (p-T) diagram is drawn. The y-axis is labeled as "p (mbar)" and the x-axis as "T [°C]". The diagram includes the following:  
- A curve labeled "Eis" (ice) transitioning into "Wasser" (water) and then "Dampf" (vapor).  
- A point labeled "Tripel" (triple point).  
- Three states labeled "1", "2", and "3".  

Graph 2:  
Another p-T diagram is drawn with the y-axis labeled "p (mbar)" and the x-axis labeled "T [°C]". The diagram includes:  
- A horizontal line at "5 mbar" labeled "Flüssig Eis" (liquid ice).  
- A curve labeled "Wasser Dampf" (water vapor).  
- A point labeled "Tripel" (triple point).  
- Three states labeled "1", "2", and "3".  
- State "2" is marked as "Isotherm" (isothermal).