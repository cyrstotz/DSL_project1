Process: adiabatic/reversible  
h2 = h1 + wK → h1 siehe b)  
h1 = hg (at -16°C) = 203.3 kJ/kg  
h2 = hg (at 6°C) = 237.7 kJ/kg

Rotational flow process:  
O = m dot times (h2 minus h1 plus w2 squared minus w1 squared divided by 2) plus Q dot minus W dot.  

From 2 to 3:  
O = m dot times (h2 minus h3).  

h2 = h1 (from table A-11 for 8 bar).  
h3 = h4 (from table A-11 for 8 bar).

b) Determine sublimation point: Triple point of water.  
Inside dryer: \( p_{amb} = 5 \, \text{mbar} \), \( T_i = 10^\circ \text{C} \).  
\( T_{subl} = T_{A-6} \).  
\( T_{subl} = -160^\circ \text{C} \).  

a) \( h_2 = h_g(160^\circ \text{C}) \) → in Table A-10.  
\( h_2 = 2337.94 \, \text{kJ/kg} \).  

b) \( h_1 = h_f(160^\circ \text{C}) \) → in Table A-10.  
\( h_1 = 273.66 \, \text{kJ/kg} \).  

b) \( x_1 = \frac{\dot{m}_R}{\dot{m}_K} = 0.906 \, \text{kJ/kg} \).  
\( x_1 = \frac{h_g(133.3^\circ \text{C}) - h_f(133.3^\circ \text{C})}{h_g(160^\circ \text{C}) - h_f(160^\circ \text{C})} \).  

c) \( h_3 = \text{interpolieren} \).  
\( h_3 = h_g(133.3^\circ \text{C}) + h_f(133.3^\circ \text{C}) \).  
\( h_3 = 2743.43 \, \text{kJ/kg} \).