The temperature and pressure remain constant:  

\( T_{EW,2} = T_{EW,1} = 0^\circ \text{C} \)  

\( p_{G,2} = p_{G,1} = p_{amb} + \frac{M_{kolben} \cdot g}{\pi \cdot \left(\frac{D}{2}\right)^2} = 133570 \, \text{Pa} = 1.3357 \, \text{bar} \)  

This is because we are operating in a two-phase region, where heat addition leads to a temperature and pressure increase only when all the ice has melted. This is not the case for \( x_{Eis,2} > 0 \).  

The temperature remains at \( 0^\circ \text{C} \) because the ice does not completely melt (\( x_{Eis,2} > 0 \)), and the temperature of the ice therefore remains constant.  

To ensure no heat is transferred, \( T_{G,2} = T_{EW,2} = 0 \).  

The pressure also remains constant because the atmospheric pressure and the weight of the piston, as well as the water-ice mixture, remain constant.