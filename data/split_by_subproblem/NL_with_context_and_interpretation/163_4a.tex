Energy balance:  
Mass balance: \( \dot{m} \) equals zero.  

0 equals \( \dot{m} \) times [h2 minus h3] plus \( \Delta x \) times \( q_{subl} \) plus \( \dot{Q}_K \) minus \( \dot{W}_K \).  

\( \dot{W}_K \) equals \( \dot{m}_{R134a} \) times [h2 minus h3].  

Assumption: \( p_1 \) equals \( p_2 \).  

\( T_{KF} \) equals \( T_i \). Assumption 1: \( x \) greater than 0.  

Throttle isenthalpic:  
Table A-11:  
\( p_4 \) (18 bar), \( h_4 \) (18 bar) equals \( h_f \) (18 bar) equals 83.142 kilojoules per kilogram.  

\( h_1 \) equals \( h_4 \) equals 83.142 kilojoules per kilogram.  

\( T_2 \) from solution:  
\( T_2 \) equals minus 72 degrees Celsius.  

From Table A-10:  
\( h_2 \) equals \( h_g \) (minus 72 degrees Celsius) equals 734.08 kilojoules per kilogram.  

\( h_3 \) (S3 equals S2 equals Sg minus 72 degrees Celsius, 8 bar).  

\( S_2 \) equals \( S_3 \) equals \( S_g \) (minus 72 degrees Celsius) equals 0.8351 kilojoules per kilogram Kelvin.  

Linear interpolation: Table A-12:  
For 8 bar superheated vapor:  

\( h_3 \) (0.8351 kilojoules per kilogram Kelvin) equals \( h \) (0.8374) minus \( h \) (0.8066) divided by (0.8374 minus 0.8066) times (0.8351 minus 0.8066) plus \( h \) (0.8066).  

\( h_3 \) equals 727.95 kilojoules per kilogram.  

Diagram:  
A p-T diagram is drawn with labeled points 1, 2, 3, and 4. The curve includes isobaric and isentropic processes. \( T_{CK} \) is shown on the x-axis, and \( p \) (bar) is shown on the y-axis.

A p-T diagram is drawn with labeled phase regions. The diagram includes:  
- Isobaric processes labeled as '1 → 2 → 3 → 4 → 1'.  
- Isothermal processes labeled as '1 → 2' and '3 → 4'.  
- The axes are pressure (p) and temperature (T).