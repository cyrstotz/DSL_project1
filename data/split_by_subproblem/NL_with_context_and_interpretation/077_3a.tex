\( c_V = 0.633 \, \text{kJ/kg·K} \)  
\( M_g = 50 \, \text{kg/kmol} \)  
\( T = 500^\circ \text{C} = 773 \, \text{K} \)  
\( V = 3.14 \, \text{L} = 3.14 \cdot 10^{-3} \, \text{m}^3 \)  

\( pV = mRT \)  
Unknown variables: \( p \) and \( m \).  

Steps:  
1) Find \( p \).  
2) Use \( m = \frac{pV}{RT} \) to find \( m \).  

---

\( p_{g,1} \):  
\( p = p_{\text{amb}} + \frac{32 \, \text{kg} \cdot g}{A_0} \)  

\( A_0 = \pi r^2 = \pi \left( \frac{40}{2} \right)^2 = \pi \cdot 20^2 = 2511 \, \text{cm}^2 \)  

\( 2511 \, \text{cm}^2 = 2511 \cdot 0.01^2 \, \text{m}^2 \)  

---

\( p = 1 \, \text{bar} + \frac{32 \cdot 9.81}{2511 \cdot 10^{-4}} \, \text{Pa} \)  

\( 1 \, \text{bar} + 39,971 \, \text{N/m}^2 = 1 \, \text{bar} + 0.3997 \, \text{bar} \)  

\( p = 1.3997 \, \text{bar} \)

m = ?  

m = (1.3997 (bar) × 3.14 × 10^-3 (m³)) / (R × 773 (K))  

R = 8.314 (kJ / kmol·K) / 50 (kg / kmol) = 0.1663 (kJ / kg·K)  

m = (1.3997 × 10^5 (kg·m) / (m²·s²)) × 3.14 × 10^-3 (m³) × (kg) × (K)) / (0.1663 (kJ / kg·K) × 773 (K))  

kJ = (10^3 kg·m² / s²)  
=> m = 3.92 × 10^-3 (kg)  

m = 3.92 × 10^-3 kg = 3.92 g  

---