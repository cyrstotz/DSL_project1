\( w_{\text{Luft}} = 200 \, \text{m/s} \)  
\( p_0 = 0.191 \, \text{bar} \)  
\( T_0 = -30^\circ \text{C} \)  
\( q_B = 1195 \, \text{kJ/kg} \)  
\( T_B = 1289 \, \text{K} \)  

A graph is drawn with the y-axis labeled \( T(K) \) and the x-axis labeled \( s(\text{kJ/kg·K}) \). The graph includes several curves representing isobars, labeled \( 0.5 \, \text{bar} \) and \( 0.191 \, \text{bar} \). Points labeled 1, 2, 3, 4, 5, and 6 are marked along the curves, connected by arrows indicating transitions between states. The graph is labeled with \( T_s \), \( h_s \), \( p_v \), and \( T_v \) at the bottom right.