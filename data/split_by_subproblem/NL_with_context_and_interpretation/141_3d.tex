\( x_{Eis} = \)  

\( m_{Eis} = 0.6 \)  
\( m_{EW} \)  

Eiswasser hat die Temperatur von 0.003°C aufgrund der thermodynamischen AGLn.  

\( u = u_{flüssig} + x \cdot (u_{gesamt} - u_{flüssig}) \)  

\( x_2 = \frac{u - u_{flüssig}}{u_{fest} - u_{flüssig}} = \frac{-546.1 - (-0.033)}{-333.492 - (-0.033)} \)  

\( u = u_1 + q_{12} = -200.092 \, \text{kJ/kg} \quad \rightarrow \quad 316.508 \, \text{kJ/kg} = 516.6 \, \text{kJ/kg} \)  

Also:  
\( u_1 (0^\circ C) = 0.6 \cdot -333.458 + (1 - 0.6) \cdot (-0.045) \)  
\( = -200.092 \, \text{kJ/kg} \)