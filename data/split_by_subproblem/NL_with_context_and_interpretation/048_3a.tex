The area \( A \) is calculated as:  
\( A = \left( \frac{D}{2} \right)^2 \pi = 5 \, \text{cm}^2 \pi = 7.85 \times 10^{-4} \, \text{m}^2 \).  

The pressure \( p_{g,1} \) is calculated as:  
\( p_{g,1} = \frac{m_K + m_{EW}}{A} + p_{amb} = \frac{(0.1 \, \text{kg} + 32 \, \text{kg}) 9.81 \, \text{m/s}^2}{7.85 \times 10^{-4} \, \text{m}^2} + 1 \, \text{bar} = 9.40 \, \text{bar} \).  

The ideal gas law is used:  
\( pV = mRT \).  

The mass of the gas \( m_g \) is calculated as:  
\( m_g = \frac{p_{g,1} V_{g,1}}{R T_{g,1}} = \frac{1.406 \, \text{bar} \cdot 3.14 \, \text{L}}{8.17 \, \text{J/mol·K} \cdot 773.15 \, \text{K}} = 3.42 \, \text{g} \).  

The specific gas constant \( R \) is calculated as:  
\( R = \frac{\bar{R}}{M_g} = \frac{8.314 \, \text{J/mol·K}}{50 \, \text{kg/kmol}} = 0.17 \, \text{kJ/kg·K} \).  

---