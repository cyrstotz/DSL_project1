The equation begins with:  
m dot times c sub p times (T sub 5 minus T sub 6) plus m dot times (w sub 5 squared minus w sub 6 squared) divided by 2 plus phi minus W dot plus.  

Next:  
W sub t equals negative R times (T sub 6 minus T sub 5) divided by n minus 1 minus A k e.  

W dot reaction equals v dot times rho plus k e.  

Luft implies ideal gas implies negative R times (T sub 6 minus T sub 5) divided by n minus 1.  

Continuing:  
m dot times c sub p times (T sub 5 minus T sub 6) plus m dot times (w sub 5 squared minus w sub 6 squared) minus m dot times R times (T sub 6 minus T sub 5) divided by n minus 1.  

The integral:  
c sub p times (T sub 5 minus T sub 6) plus w sub 5 squared minus R times (T sub 6 minus T sub 5) divided by n minus 1.  

Equals w sub 6 equals 350 meters per second.  

R equals c sub v equals c sub p divided by 1.4 equals 0.7936 kilojoules per kilogram Kelvin.  

R equals c sub p minus c sub v equals 289 joules per kilogram Kelvin.

A T-s diagram is drawn with the temperature (T) on the vertical axis labeled in Kelvin (K) and entropy (s) on the horizontal axis labeled in kilojoules per kilogram Kelvin (kJ/kg·K).  

The diagram includes several labeled points and curves:  
- Point "0" is at the bottom left.  
- Point "1" is slightly higher and to the right of point "0".  
- Point "2" is higher than point "1" and connected by a vertical line.  
- Point "3" is at the peak of the diagram, connected to point "2" by a steep curve.  
- Point "5" is lower than point "3" and connected by a downward curve.  
- Point "6" is at the bottom right, connected to point "5" by a steep curve.  

The diagram includes labeled isobars:  
- "0.191 bar" is written near the curve at the bottom left.  
- "0.5 bar" is written near the curve at the top right.  

Two regions are labeled as "saturated" near the curves.  

Arrows indicate the direction of the process between points.

A table is presented with columns labeled \( p \) and \( T \).  

Row 0:  
\( p \): 0.191 bar  
\( T \): -30 degrees Celsius  

Row 1:  
\( p \): 0.5 bar  
\( T \): 293.15 K  

Row 2:  
\( p \): 0.5 bar  

Row 3:  
\( p \): 0.5 bar  

Row 5:  
\( p \): 0.5 bar  
\( T \): 431.9 K  

Row 6:  
\( p \): 0.191 bar  

Below the table, the text reads:  
\( T_s, p_5, w_5 \) given