A table is presented with the following columns: \( p \) (pressure), \( T \) (temperature), \( s \) (entropy), and velocity values. The rows are labeled from 0 to 6:  

- Row 0: \( p = 0.191 \, \text{bar} \), \( T = -30^\circ \text{C} \), \( s \) is blank, velocity = \( 200 \, \text{m/s} \).  
- Row 1: \( p \) and \( T \) are blank, \( s \) is blank.  
- Row 2: \( p \) and \( T \) are blank, \( s \) is blank.  
- Row 3: \( p \) and \( T \) are blank, \( s \) is blank.  
- Row 4: \( p \) and \( T \) are blank, \( s \) is blank.  
- Row 5: \( p = 0.5 \, \text{bar} \), \( T = 431.9 \, \text{K} \), \( s \) is blank, velocity = \( w_5 = 220 \, \text{m/s} \).  
- Row 6: \( p = 0.191 \, \text{bar} \), \( T = 2340 \, \text{K} \), \( s \) is blank.  

---

O equals m dot multiplied by (h6 minus h5 plus (w6 squared minus w5 squared) divided by 2).  

W dot equals R multiplied by (T6 minus T5) divided by (1 minus k), equals 74.077 kilojoules per kilogram.  

R equals 8.314 kilojoules per kilomole Kelvin divided by 28.97 kilograms per kilomole, equals 0.28863 kilojoules per kilogram Kelvin.  

w6 squared divided by 2.  
phi p squared divided by 2.  

nach w6 umstellen.