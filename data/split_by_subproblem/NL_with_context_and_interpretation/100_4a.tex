A table is drawn with columns labeled \( p \) (bar), \( T \) (K), \( Q \), \( W \), and \( x \).  

Row 1:  
- \( p = 1210 \)  
- \( p_1 = p_2 \)  

Row 2:  
- \( p = p_2 \)  
- \( T_i = -6 / (-22^\circ C) = 277.15 \, K \)  

Row 3:  
- \( p = 8 \)  

Row 4:  
- \( p = p_4 \)  
- \( p = 8 \)  

Additional notes:  
- \( s_2 = s_3 \)  
- \( s_2 = s_5 \)  
- \( s_1 = s_3 \)  
- \( s_5 = s_4 \)  
- \( h_1 = h_4 \)  
- \( h_2 = h_4 \)  
- \( h_3 = 93.42 \, \text{kJ/kg} \)

A graph is drawn with pressure (p) on the vertical axis and temperature (T) on the horizontal axis. The graph depicts a phase diagram with labeled points and curves:  

- Point 1 is on the left side of the curve.  
- Point 2 is on the right side of the curve.  
- Point 3 is above point 2, connected by a vertical line labeled "8 bar."  
- Point 4 is near the top of the curve, which is labeled "Tkrit" (critical temperature).  
- The curve represents phase regions, with the peak indicating the critical point.  

The diagram visually represents the freeze-drying process and phase transitions.