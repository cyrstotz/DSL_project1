Two pressure-temperature (P-T) diagrams are drawn.  
- The first diagram shows a curve labeled "1500 mbar" with two points marked: point "2" on the curve and a dashed line extending horizontally.  
- The second diagram shows a similar curve with point "2" marked, and a dashed line extending horizontally.  

A table is provided with the following columns: P, T, and additional variables.  
- Row 1: P = pu, T = blank, additional variable: a_41 = 0.  
- Row 2: P = 1500 mbar, T = -22°C, additional variable: x_2 = 1.  
- Row 3: P = 8 bar, T = blank, additional variable: x_3 = 1, S_3 = S_2.  
- Row 4: P = 8 bar, T = blank, additional variable: x_4 = 0, T_i - T_3 = 6 K.  

Equations and notes:  
- p_u = S_mbar + p_tp.  
- T_i - 10 K = T_sublimationspunkt.  

Another P-T diagram is drawn with labeled regions:  
- Isobaric lines are shown, and points "2," "3," and "4" are marked.  
- The curve is labeled "P(T)."