A p-T diagram is drawn with labeled phase regions. The diagram includes:  
- A diagonal line separating liquid and vapor phases.  
- Four states labeled as 1, 2, 3, and 4.  
- State 1 and State 2 connected by an isobar labeled "p1/p2".  
- State 3 and State 4 connected by an isobar labeled "p3/p4".  
- State 2 and State 3 connected by an isentropic process.  
- State 4 and State 1 connected by a boiling/isenthalpic process.  
The x-axis is labeled as "T (K)", and the y-axis is labeled as "p (bar)".  

A schematic of the freeze-drying process is drawn, showing the refrigerant cycle with components labeled:  
- Compressor (adiabatic and reversible).  
- Expansion valve (adiabatic).  
- Two heat exchangers (isobaric processes).  
- States 1, 2, 3, and 4 are marked, with x1 = 0 and x2 = 1.  
- Heat flows \( \dot{Q}_K \) and \( \dot{Q}_{ab} \) are indicated.  
- Work \( \dot{W}_K = 28 \, \text{kW} \) is noted.