A diagram is drawn with labeled horizontal lines representing temperature (T). The labels are:  
- "ein" (inlet)  
- "sieden" (boiling)  

Below the diagram, the following values and equations are written:  
- \( \dot{m}_{in} = 0.3 \, \text{kg/s} \)  
- \( T_{ein} = 70^\circ \text{C} \rightarrow \text{sieden flüssig} \) (boiling liquid)  
- \( m_{ges,1} = 5755 \, \text{kg} \)  
- \( x_0 = 0.005 \)  
- \( T = 100^\circ \text{C} \, \text{immer} \) (always)

The integral of Tds from state e to state a is equal to the heat flow divided by the mass flow rate.  

Tds equals Q divided by m dot.  
This is equal to cv multiplied by the natural logarithm of T2 divided by T1.  

Ideal fluid is assumed.  

---