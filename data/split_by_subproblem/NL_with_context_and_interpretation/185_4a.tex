Two diagrams are drawn:  
1. The first diagram is a p-T graph showing phase regions. A curve is labeled with "Sublimation," "Fusion," and "Vaporization."  
2. The second diagram is a p-h graph with points labeled "1," "2," "3," and "4." Arrows indicate transitions between these points, and "dp" is written near the transition from "2" to "3."

Two diagrams are drawn:  
1. The first diagram is labeled with axes \( p \) (pressure) and \( s \) (entropy). It shows curves labeled "isoch" (likely isochoric) and "isobar" (likely isobaric). Points \( 1 \), \( 2 \), and \( A \) are marked on the curves.  
2. The second diagram is labeled with axes \( T \) (temperature) and \( h \) (enthalpy). It shows similar curves labeled "isoch" and "isobar". Points \( 1 \), \( 2 \), and \( A \) are marked here as well.