\( \Delta m_{12} \):  
\((m_1 + \Delta m_{12}) U_2 - m_1 U_1 = \Delta m_{12} h_{in}(20^\circ C) - Q_{R,12}\)  

\( U_2 = (5755 \cdot 55 + 0.005 \cdot (2466.6 - 252.55)) = 303.85 \, \text{kJ/kg}\)  
\( U_1 = (5755 \cdot 54 + 0.005 \cdot (2500.5 - 419.55)) = 493.55 \, \text{kJ/kg}\) (TA-2)  
\( h_{in}(20^\circ C) = 83.56 \, \text{kJ/kg}\)  

\( m_1 U_1 + \Delta m_{12} h_{in} - m_1 U_2 = \Delta m_{12} - Q_{R}\)  
\( m_1 (U_1 - h_{in}) = m_1 U_2 - Q_{R} - m_1 U_1\)  
\( \Delta m_{12} = \frac{m_1 U_1 - Q_{R} - m_1 U_2}{U_2 - h_{in}}\)  

\(\Delta m_{12} = \frac{5755 \, \text{kg} \cdot (493.55 - 303.85) \, \text{kJ/kg} - 35,000 \, \text{kJ}}{(493.55 - 83.56) \, \text{kJ/kg}}\)  

\(\Delta m_{12} = 317 \, \text{kg}\)

Delta S subscript 12 equals Delta M subscript 12 multiplied by S subscript W (20 degrees Celsius) minus Q subscript out,12 divided by T subscript KF.  

S subscript W (20 degrees Celsius) equals 0.256 kilojoules per kilogram Kelvin (from Table A-2).  

Delta S subscript 12 equals 3127 kilograms multiplied by 0.256 kilojoules per kilogram Kelvin minus 35000 joules divided by 293.15 Kelvin equals 808.1 kilojoules per Kelvin.