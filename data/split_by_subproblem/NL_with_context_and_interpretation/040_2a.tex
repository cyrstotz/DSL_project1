A T-s diagram is drawn with labeled isobars \( p_0 \), \( p_1 \), \( p_2 \), and \( p_s \). The diagram shows six states connected by arrows indicating the process flow.  
- State 1 starts at \( p_0 \) and moves upward.  
- State 2 is at \( p_2 > p_0 \).  
- State 3 is at \( p_3 > p_2 \).  
- State 4 is at \( p_a < p_3 \).  
- State 5 is labeled as \( p_a = p_s \).  
- State 6 returns to \( p_0 \).  

A table is provided with columns labeled \( T \), \( s \), and \( p \):  
- State 1: \( T = -30^\circ C \), \( s = s_1 \), \( p = p_0 \).  
- State 2: \( s_3 > s_1 \), \( p_2 > p_0 \).  
- State 3: \( s_2 > s_2 \), \( p_3 > p_2 \).  
- State 4: \( s_4 > s_3 \), \( p_a < p_3 \).  
- State 5: \( s_5 = s_5 \), \( p_a = p_s \).  
- State 6: \( s_6 = s_5 \), \( p = p_0 \).

\( h_5 = 953.26 \, \text{kJ/kg} \)  

Isentropic exit conditions:  
\( T_6 = T_5 \left( \frac{p_6}{p_5} \right)^{\frac{n-1}{n}} \)  

\( T_6 = 328.075 \, \text{K} \)  

\( h_6 = h(328 \, \text{K}) = h(333 \, \text{K}) - h(353 \, \text{K}) \cdot \frac{5 \, \text{K}}{5 \, \text{K}} \)  

\( h_6 = 333.93 \, \text{kJ/kg} \)  

\( 0 = \dot{m} \cdot (h_5 - h_6) + \frac{w_5^2 - w_6^2}{2} \)  

\( \dot{m} \cdot \frac{(h_5 - h_6)}{w_5^2 - w_6^2} \)  

\( w_6^2 = 2(h_5 - h_6) + w_5^2 \)  

\( w_6 = 998.26 \, \text{m/s} \)  

---