A p-T diagram is drawn with labeled phase regions. The diagram includes the following:  
- A curve labeled "Flüssig" (Liquid).  
- A curve labeled "Fest" (Solid).  
- A curve labeled "Gasförmig" (Gaseous).  
- Points labeled 1, 2, 3, and 4 along the curves.  
- The triple point is marked.  
- The transitions between phases are indicated.

T2 equals negative 10 degrees Celsius minus 6 Kelvin equals negative 16 degrees Celsius.  

O equals  

S2 equals 0.9855 kilojoules per kilogram Kelvin equals S3.  

S3 equals 0 in Table A-9.  

T4 equals approximately 30 degrees Celsius.  

ln of inlet temperature T3 equals 30 plus 0.98555 minus 0.9444 times (35 minus 30) divided by 0.9854 minus 0.9444 equals 30 degrees Celsius.