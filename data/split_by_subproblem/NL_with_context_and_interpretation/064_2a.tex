A T-s diagram is drawn with labeled isobars and states. The diagram includes the following:  
- State 0 at the bottom left.  
- State 1 marked on the curve.  
- State 2 labeled above state 1.  
- State 3 labeled higher up.  
- State 5 marked near the middle right.  
- State 6 labeled at the far right.  
Arrows indicate transitions between states, and the process flow is shown.

A graph is drawn with the y-axis labeled as "T [K]" and the x-axis labeled as "S [kJ/kg·K]". The graph represents a temperature-entropy (T-S) diagram.  

The diagram includes several curves and points:  
- A curved line labeled "isobar" connects points 1, 2, and 3.  
- Another curved line labeled "isobar" connects points 5 and 6.  
- Points 0, 1, 2, 3, 5, and 6 are marked on the graph.  
- The curve between points 3 and 5 appears to represent a transition or mixing process.  

The graph visually represents thermodynamic processes in a jet engine, as described in the problem setup.

Two diagrams are drawn:

1. The first diagram is labeled "a)" and shows a pressure-volume (P-v) graph.  
   - The vertical axis is labeled "P" with "P_0" marked at the top.  
   - The horizontal axis is labeled "v" with "m_dot / rho" written underneath.  
   - The graph contains several curves and points labeled "1", "2", "3", and "4".  
   - The curves intersect and loop around these points, forming a complex shape.  

2. The second diagram is a pressure-specific volume (P-v) graph.  
   - The vertical axis is labeled "P" with "P_out" marked at the top.  
   - The horizontal axis is labeled "v" with "m^3 / kg" written underneath.  
   - The graph shows a dome-shaped curve with points labeled "1", "2", "3", and "4".  
   - Point "4" is at the bottom left, "1" is at the peak of the dome, "2" is on the right side of the dome, and "3" is along the horizontal line connecting "4" and "2".  
   - Arrows indicate the direction of the process between these points.