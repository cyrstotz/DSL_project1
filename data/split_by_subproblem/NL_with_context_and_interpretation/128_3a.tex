The diagram shows a cylinder with two chambers separated by a membrane. The upper chamber contains the ice-water mixture (m_EW), and the lower chamber contains gas. The piston exerts a force on the ice-water mixture.  

The equation for pressure is:  
\( m_g + p_0 \cdot A + m_{EW} \cdot g = p_g \cdot A \)  

The pressure \( p_g \) is calculated as:  
\( \frac{g}{A} \cdot (m_K + m_{EW}) + p_0 = p_g \)  

Substituting values:  
\( \frac{9.81 \, \text{m/s}^2}{(0.05 \, \text{m})^2 \cdot \pi} \cdot (32 \, \text{kg} + 0.1 \, \text{kg}) + 1 \, \text{bar} = 1.9 \, \text{bar} = p_g \)  

The mass of the gas \( m_g \) is calculated using:  
\( p_g \cdot V_{g,1} = \frac{R \cdot T_{g,1}}{M_{gas}} \cdot m_g \)  

Substituting values:  
\( m_g = \frac{p_g \cdot V_{g,1} \cdot M_{gas}}{R \cdot T_{g,1}} \)  

\( m_g = \frac{1.9 \, \text{bar} \cdot 3.14 \cdot 10^{-3} \, \text{m}^3 \cdot 50 \, \text{kg/kmol}}{8.314 \, \text{kJ/kmol·K} \cdot 773 \, \text{K}} \)  

Result: \( m_g = 3.42 \, \text{kg} \)  

---

d1 m subscript g plus p subscript 0 multiplied by A plus m subscript EW multiplied by g equals p subscript g multiplied by A