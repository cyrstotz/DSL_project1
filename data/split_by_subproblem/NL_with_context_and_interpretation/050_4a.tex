Two diagrams are drawn:  
- The first diagram is labeled "i)" and shows a p-T graph. It includes regions labeled "compressed," "ND," and "SH (superheated)." The graph has arrows indicating transitions, and points are marked with "St1" and "x." The pressure axis is labeled with "8 bar."  
- The second diagram is labeled "ii)" and also shows a p-T graph. It includes similar regions labeled "compressed," "ND," and "SH (superheated)." Arrows indicate transitions, and the graph has a curved line representing the phase boundary.

The first diagram shows a pressure-temperature (p-T) graph with a curve labeled "compressed" and a region labeled "gas phase."  

The second diagram shows another p-T graph with lines labeled "sublimation" and "isobar." The phase regions are marked, but some parts are crossed out.  

The third diagram shows a p-T graph with points labeled "1," "2," "3," and "4." The processes are labeled as "isobar," "reversible adiabatic," and "isothermal."