S subscript 12 equals m subscript 1 multiplied by s subscript f (100 degrees Celsius) plus m subscript 2 multiplied by s subscript f (70 degrees Celsius).  

Using Table A-2:  
S subscript 12 equals m subscript 1 multiplied by s subscript f (100 degrees Celsius) plus m subscript 2 multiplied by s subscript f (70 degrees Celsius).  

This expands to:  
m subscript 1 multiplied by (s subscript f (100 degrees Celsius) plus x multiplied by (s subscript g (100 degrees Celsius) minus s subscript f (100 degrees Celsius))) plus m subscript 2 multiplied by s subscript f (70 degrees Celsius).  

Further simplified:  
m subscript 1 multiplied by (1.337 kilojoules per kilogram Kelvin) plus m subscript 2 multiplied by (0.9599 kilojoules per kilogram Kelvin).  

S subscript f (T, p) equals s subscript f (T) equals 0.2560 kilojoules per kilogram Kelvin.  

S subscript n equals 8.715 megajoules per Kelvin.  

S subscript 2 equals s subscript f (70 degrees Celsius) equals 0.9599 kilojoules per kilogram Kelvin.  

S subscript 2 equals m subscript g multiplied by s subscript 2 equals 8.753 megajoules per Kelvin.  

Delta S subscript 12 equals S subscript 2 minus S subscript n equals 74 kilojoules per Kelvin.