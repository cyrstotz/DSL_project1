The first diagram is a T-s diagram (temperature versus entropy). The axes are labeled as follows:  
- The vertical axis is labeled "T [K]" (temperature in Kelvin).  
- The horizontal axis is labeled "S [kJ/kg·K]" (entropy in kilojoules per kilogram per Kelvin).  

The diagram shows several curves and lines:  
- A dome-shaped curve representing a phase boundary.  
- Various intersecting lines, including isobars and other thermodynamic paths.  

The second diagram is also a T-s diagram, with the same axis labels:  
- The vertical axis is labeled "T [K]" (temperature in Kelvin).  
- The horizontal axis is labeled "S [kJ/kg·K]" (entropy in kilojoules per kilogram per Kelvin).  

This diagram includes labeled points (0, 1, 2, 3, 4, 5, 6) connected by curves and lines, representing different states in a thermodynamic process.  
- A pink arrow labeled "s = const" indicates an isentropic process.  
- Another pink arrow labeled "s ≠ const" indicates a process where entropy is not constant.  
- The curve labeled "P0" represents the atmospheric pressure line.  

Below the second diagram, there is a handwritten note:  
"Zustand 4 liegt nicht per se auf der p5 Isobare."  
Translation: "State 4 does not necessarily lie on the p5 isobar."