Ideal Gas Law:  
\( pV = mRT \) → \( m = \frac{pV}{RT} \)  

\( R = \frac{R_u}{M_g} = \frac{8.314 \, \text{m}^3 \, \text{Pa}/\text{mol} \, \text{K}}{50 \, \text{kg}/\text{kmol}} = 166.3 \, \text{m}^3 \, \text{Pa}/\text{kg} \, \text{K} \)  

\( m = \frac{1.1401 \, \text{bar} \cdot 3.14 \, \text{L} \cdot 10^{-3} \, \text{m}^3/\text{L}}{166.3 \, \text{m}^3 \, \text{Pa}/\text{kg} \, \text{K} \cdot (273.15 + 500) \, \text{K}} \)  

\( m = 0.00346 \, \text{kg} \approx 3.46 \, \text{g} \)  

---

The pressure remains as in state 1 because the system, when it expands, balances against the lower pressure. Therefore, the mass remains the same.  

\( p_2 = 1.1401 \, \text{bar} \)  

Since the system is in thermodynamic equilibrium:  
\( T_2 = T_E(p_2) = 0^\circ \text{C} = T_{G,2} = T_{EW} \)  

Refer to \( T_E(p_2) \) from Table 1.

\( U_{1EW} = U_{id(EW)} + X_1 \cdot U_{id(EW-ice)} \)  

\( (P_{EW} = P_{amb} + AK = 1.14 \, \text{bar}) \)  

\( U_{1EW} = -0.0165 \, \frac{\text{kJ}}{\text{kg}} + 0.6 \cdot (-333.458 \, \frac{\text{kJ}}{\text{kg}} + 0.033 \, \frac{\text{kJ}}{\text{kg}}) \)  

\( = -200.08 \, \frac{\text{kJ}}{\text{kg}} \)