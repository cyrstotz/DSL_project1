p1 equals 1.49 bar.  
m1 equals m2 equals 0.1 kilograms.  

U1 equals U at 0 degrees Celsius, 1.49 bar equals Uf at 0 degrees Celsius, 1.49 bar plus x multiplied by (Ug at 0 degrees Celsius, 1.49 bar minus Uf at 0 degrees Celsius, 1.49 bar).  
Equals negative 133.47 kilojoules per kilogram.

F equals p times A.  

A diagram is drawn showing three sections of a cylinder labeled as follows:  
- Top section: Contains a mass labeled "m" and atmospheric pressure labeled "p amb".  
- Middle section: Contains "EW" and pressure labeled "p1".  
- Bottom section: Contains pressure labeled "p2".  

Another diagram is drawn showing the top section with "m" and "p amb". Forces are labeled, including arrows pointing downward and upward.  

The equation for force equilibrium is written:  
Sum of forces equals p amb times pi times (d divided by 2) squared plus m times g equals p1 times pi times (d divided by 2) squared.  

The pressure p1 is calculated:  
p1 equals p amb plus 4 times m times g divided by pi times d squared.  
p1 equals 1.20 bar.  

A third diagram is drawn showing the middle section labeled "m EW" and "p1", with arrows indicating forces acting upward and downward.  

The equation for force equilibrium is written:  
Sum of forces equals p1 times pi times (d divided by 2) squared plus m EW times g equals p2 times pi times (d divided by 2) squared.  

The pressure p2 is calculated:  
p2 equals p1 plus 4 times m EW times g divided by pi times d squared.  
p2 equals 1.40 bar.  

The mass of the gas m g is calculated:  
m g equals p2 times V divided by R times T g1.  

The gas constant R is calculated:  
R equals universal gas constant divided by molar mass.  
R equals 166.28 joules per kilogram Kelvin.  

Substituting values:  
m g equals 1.40 bar times 50.10 times 10 to the power of minus 3 cubic meters divided by 166.28 joules per kilogram Kelvin times 773.15 Kelvin.  

Result:  
m g equals 3.42 times 10 to the power of minus 3 kilograms.