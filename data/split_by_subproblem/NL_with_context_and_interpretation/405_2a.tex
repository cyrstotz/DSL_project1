A table is presented with columns labeled 1, 2, 3, 4, 5, 6, and 0. Rows are labeled as follows:  

- **T**:  
  - Column 5: 431.8 K  
  - Column 6: 318.075 K  
  - Column 0: 293.15 K / 30  

- **p**:  
  - Column 5: 5.10^4 Pa  
  - Column 6: 1.91^4 Pa  
  - Column 0: 1.91^4 Pa  

- **h**: (empty)  

- **s**: (empty)  

- **w**:  
  - Column 5: 220 meters per second  
  - Column 6: 502.2 grams per second  
  - Column 0: 200  

---

Below the table, the following information is written:  

- **Luft**:  
  - \( w_{Luft} = 200 \, \text{m/s} \)  
  - \( \dot{m}_{geo} = ? \, \text{kg/s} \)  
  - \( n = 1.4 \)  
  - \( c_p = 1.006 \, \text{kJ/kg·K} = 1006 \, \text{J/kg·K} \)  

---

A graph is drawn with the label **a)**.  

- The graph is a T-s diagram with the vertical axis labeled \( T \, [K] \) and the horizontal axis labeled \( s \, [\text{kJ/kg·K}] \).  
- Points are marked as 0, 2, G, S, 5, and 6.  
- The curve starts at point 0, rises to point 2, then transitions to point G, moves to point S, and finally descends to points 5 and 6.  
- The isobars \( p_0 \), \( p_2 \), and \( p_5 \) are labeled along the graph.