\( p_{g,1} = p_{amb} + \frac{F_G}{A} \)  

\( F_G = 32 \, \text{kg} \cdot 9.81 \, \text{m/s}^2 = 313.92 \, \text{N} \)  

\( A = \pi \cdot \left(\frac{D}{2}\right)^2 = 0.007853982 \, \text{m}^2 \)  

\( p_{g,1} = 1 \, \text{bar} + \frac{F_G}{A} = 10^5 \, \text{Pa} + 39963.59 \, \text{Pa} = 1.4 \, \text{bar} \)  

To determine the mass of the gas, we can apply the ideal gas law:  

\( m_g = \frac{p_{g,1} \cdot V_{g,1}}{R \cdot T_{g,1}} = \frac{1.4 \, \text{bar} \cdot 3.14 \cdot 10^{-3} \, \text{m}^3}{0.16625 \, \frac{\text{kJ}}{\text{kg·K}} \cdot 773.15 \, \text{K}} = 0.00342 \, \text{kg} \)  

\( R = \frac{R_{\text{gas}}}{M_{\text{gas}}} = 0.16625 \, \frac{\text{kJ}}{\text{kg·K}} \)  

\( m_g = 3.42 \, \text{g} \)  

---