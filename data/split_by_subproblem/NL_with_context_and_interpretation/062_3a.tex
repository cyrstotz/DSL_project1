R equals 8.314 kilojoules per kilomole per kelvin divided by 50 kilograms per kilomole equals 0.166 kilojoules per kilogram per kelvin.  

p subscript g,1 equals p subscript EW,1 plus m subscript g times g divided by A plus p subscript amb.  

A equals pi times (D divided by 2) squared.  

p subscript g,1 equals p subscript EW,1 plus m subscript K times g divided by A plus p subscript amb equals 32 kilograms times 9.81 meters per second squared divided by pi times (0.1 meters divided by 2) squared plus 1.105 newtons per square meter equals approximately 1.40 bar.  

pV equals mRT.  
T subscript g,1 equals 773.15 kelvin.  
p subscript g,1 equals 1.40 times 10 to the power of 5 newtons per square meter.  
V subscript g,1 equals 3.14 times 10 to the power of negative 3 cubic meters.  
m subscript g equals p subscript g,1 times V subscript g,1 divided by R times T subscript g,1 equals 0.166 times 10 to the power of negative 3 joules per kilogram per kelvin times 773.15 kelvin equals approximately 3.43 grams.

Q12 equals 1300 joules.  

V1EW equals V2EW, therefore v1EW equals v2EW.  

---