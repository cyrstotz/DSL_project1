Sublimation: \( T_i \to T_q \) → \( p = 2 \, \text{mbar} \to 1.2 \times 10^3 \, \text{bar} \to 2 \times 10^3 \, 20 \, T_i \)  
= -30 Pa = 0.1 Pa = value during sublimation!!  

From Table A-6: Interpolation. \( T(0.21 \, \text{Pa}) = T(0.20388 \, \text{Pa}) - T(0.0883 \, \text{Pa}) \cdot \frac{0.1635 - 0.0883 \, \text{Pa}}{0.21 - 0.0883 \, \text{Pa}} \)  
\( T_i = -22^\circ \text{C} \approx 20.385^\circ \text{C} \) → Sub-point  

\( T_2 \) = Sub-point \( T_i \) = 20.385 (??) from diagram  
\( T_i \) = 20 K above sublimation point  

\( T_{\text{Sub-point}} = 20.385^\circ \text{C} \approx 20^\circ \text{C} \), if taken from the diagram.  

---

The mass flow rate \( \dot{m}_{R} \) is calculated using the equation:
\( \dot{Q}_K - W_K = \dot{m}_R (h_3 - h_4) + W_K \).

Rearranging:
\( \dot{Q}_K = \dot{m}_R (h_3 - h_4) + W_K \).

From the equation:
\( \dot{m}_R = \frac{W_K}{h_3 - h_4 + h_2 - h_1} \).

Substituting values:
\( \dot{m}_R = \frac{28 \, \text{kJ/s}}{(232.62 - 16.82 + 93.42 - 264.25) \, \text{J/g}} \).

From Table A-20:
At \( T_2 = T_{evap} = 260^\circ \text{C} \), \( h_f = h_2 = 232.62 \, \text{J/g} \), \( h_2 = h_f = 16.82 \, \text{J/g} \).

From Table A-12:
At \( p_3 = 8 \, \text{bar} \), \( h_3 = h_g = 264.25 \, \text{J/g} \), \( h_{fg} = 93.42 \, \text{J/g} \).

Thus:
\( \dot{m}_R = 0.6352 \, \text{g/s} \), equivalent to \( 4 \, \text{kg/h} \).

Final boxed result:
\( \dot{m}_{R, result} = 0.6352 \, \text{g/s} \).