A T-s diagram is drawn with labeled isobars and states:  
- State 1: \( s_1 = s_2 \), \( p = p_1 \), \( T = 243.15 \, \text{K} \).  
- State 2: \( p_2 = p_3 \).  
- State 3: \( p_4 = p_3 \).  
- State 4: \( p = 0.5 \, \text{bar} \).  
- State 5: \( p = 0.5 \, \text{bar} \).  
- State 6: \( p = 0.191 \, \text{bar} \).  

The diagram includes arrows indicating transitions between states, with annotations such as "isentropic" and "adiabatic."  

A table is partially filled with columns labeled \( p \), \( T \), \( v \), \( h \), \( s \), \( Q \), and \( w \).  
- Row 1: \( p = 0.191 \, \text{bar} \), \( T = -30 \, \text{°C} \), \( w_{\text{Luft}} = 200 \, \text{m/s} \), \( \eta_{V,s} < 1 \).  
- Row 2: \( p_2 = p_3 \).  
- Row 5: \( p = 0.5 \, \text{bar} \), \( T = 431.9 \, \text{K} \).