Two diagrams are drawn:  

1. **Temperature vs. Entropy (T-S Diagram)**:  
   - The y-axis is labeled as 'T [K]'.  
   - The x-axis is labeled as 'S [kJ/kg·K]'.  
   - The diagram shows several labeled curves:  
     - 'isobar' and 'isentrop' are marked along the curves.  
     - Points labeled as '1', '2', '3', '4', '5', and '6' are connected by lines.  
     - The direction of the process is indicated with arrows.  

2. **Pressure vs. Entropy (P-S Diagram)**:  
   - The y-axis is labeled as 'P [k]'.  
   - The x-axis is labeled as 'S [kJ/kg·K]'.  
   - The diagram shows similar labeled curves:  
     - 'isobar Pu=Pa' and 'isentrop' are marked.  
     - Points labeled as '1', '2', '3', '4', '5', and '6' are connected by lines.  
     - The direction of the process is indicated with arrows.  

A table is drawn next to the diagrams:  
- Columns are labeled as 'P', 'T', and 'V'.  
- Rows contain numerical values:  
  - Row 1: P = 0.191 bar, T = -30°C, V = blank.  
  - Row 2: P2 = P3, T = blank, V = blank.  
  - Row 3: P = blank, T = blank, V = blank.  
  - Row 4: P = 0.5 bar, T = blank, V = blank.  
  - Row 5: P = 0.5 bar, T = 431.9 K, V = w5 = 220 m/s.  
  - Row 6: P = 0.191 bar, T = blank, V = blank.  

Below the table, a flow ratio is written:  
- \( \dot{m}_{ges} = \dot{m}_M + \dot{m}_K \).  

---