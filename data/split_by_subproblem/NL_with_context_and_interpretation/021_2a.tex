A graph is drawn with the x-axis labeled as "s [kJ/kg·K]" and the y-axis labeled as "T [K]". The graph represents a T-s diagram with several points labeled as follows:  
- Point 0 at the bottom left.  
- Point 1 slightly above and to the right of point 0.  
- Point 2 horizontally to the right of point 1.  
- Point 3 above point 2, with a steep curve connecting them.  
- Point 5 below point 3, with a steep downward curve connecting them.  
- Point 6 at the bottom right, connected to point 5.  

The following pressures are noted:  
- \( p_0 = 0.191 \, \text{bar} \) near point 0.  
- \( p_5 = 0.5 \, \text{bar} \) near point 5.

325, 1.78243  
323.075, x  
330, 1.75835  

1.75835 minus 1.78243 divided by 330 minus 325 equals (323.075 minus 325) plus 1.78243.  

s_0 equals 1.294524 (units: kilojoules per kilogram Kelvin).  

---

240, 1.47824  
243.15, x  
250, 1.54547  

1.54547 minus 1.47824 divided by 250 minus 240 equals (243.15 minus 240) plus 1.47824.  

s_0 equals 1.49413 (units: kilojoules per kilogram Kelvin).  

---

s_erz equals s_6 minus s_0 equals 0.3008 (units: kilojoules per kilogram Kelvin).  

---

ex_ver equals T_0 times s_erz.  

T_0 equals 243.15 Kelvin.  

s_erz equals 0.3008 (units: kilojoules per kilogram Kelvin).  

ex_ver equals 73.14 (units: kilojoules per kilogram).  

Boxed result: ex_ver equals 73.14 (units: kilojoules per kilogram).