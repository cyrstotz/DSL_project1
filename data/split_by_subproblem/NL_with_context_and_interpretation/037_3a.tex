A diagram is drawn showing two chambers labeled "Keller" (lower chamber) and "Membran" (membrane). The pressure in the upper chamber is labeled as \( P_{EW} \), and the pressure in the lower chamber is labeled as \( P_{gas,1} \). Both pressures are equal to 1 bar:  
\( 1 \, \text{bar} = P_{EW} = P_{gas,1} = 1.01325 \, \text{bar} \).  

The equation for the gas pressure is written as:  
\( P_{gas,1} V_{g,1} = m_{g,1} R T_{g,1} \).  

The gas constant \( R \) is calculated as:  
\( R = \frac{\overline{R}}{M_g} = \frac{0.76628 \, \text{kJ/kg·K}}{50 \, \text{kg/kmol}} = 0.76628 \, \text{kJ/kg·K} \).  

The mass of the gas \( m_{g,1} \) is calculated using the formula:  
\( m_{g,1} = \frac{P_{g,1} V_{g,1}}{R T_{g,1}} \).  

Values are substituted:  
\( V_{g,1} = 3.14 \, \text{L} = 3.14 \, \text{dm}^3 = 3.14 \cdot 10^{-3} \, \text{m}^3 \),  
\( T_{g,1} = 500^\circ \text{C} = 773.15 \, \text{K} \).  

The calculation yields:  
\( m_{g,1} = 2.4745 \, \text{kg} \),  
\( m_{g,1} = 2.47488 \, \text{kg} \) (rounded).