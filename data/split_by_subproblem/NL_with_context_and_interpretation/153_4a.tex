Two diagrams are drawn:  
1. The first diagram is labeled with "p (bar)" on the vertical axis and "T (Kelvin)" on the horizontal axis. It shows a curve with a peak labeled "Critical Point." The region beneath the curve is labeled "Nassdampf" (wet steam).  
2. The second diagram is also labeled with "p (bar)" on the vertical axis and "T (Kelvin)" on the horizontal axis. It includes a line labeled "Gas" and another labeled "Flüssig" (liquid). A point is marked as "Triple Point," and the region below the line is labeled "I Sublim." The temperature difference is noted as "10K."

Epsilon equals Q-dot-ab divided by Q-dot-zyl equals 1 divided by W-dot-v multiplied by 1 divided by Q-dot-ab minus Q-dot-zyl equals Q-dot-k divided by Q-dot-ab.  

Q-dot-k equals m-dot multiplied by [h2 minus h3].  
Q-dot-ab equals m-dot multiplied by [h4 minus h3].