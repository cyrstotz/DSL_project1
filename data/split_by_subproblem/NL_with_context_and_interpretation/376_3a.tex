m sub g equals p times V divided by R times T.  
R equals R divided by mu equals 8.314 joules per mole Kelvin divided by 50 kilograms per kilomole.  

R equals 0.16628 joules per kilogram Kelvin equals 166.28 joules per kilogram.  
Equals 0.003922 kilograms.  

m sub g equals 3.422 grams (underlined).

\( c_V = 0.633 \, \text{kJ/kg·K} \), \( M_g = 50 \, \text{kg/kmol} \)  

\( p_{g,1}, m_g = ? \)  

\( pV = mRT \)  

\( 0.1 \, \text{kg} + 32 \, \text{kg} \)  

(Diagram of a cylinder with two chambers, separated by a horizontal line.)  

\( m_g = F \)  

\( \frac{m_g}{0.05 \, \text{m}^2} = p = 40.094 \, \text{bar} \)  

\( p_{g,1} = p_{\text{atm}} + p_{\text{Gewicht}} = 1.40094 \, \text{bar} \)  

\( 1.40 \, \text{bar} \)