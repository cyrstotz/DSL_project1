Gas Mass Calculation:  
\( pV = mRT \)  
\( m_g = \frac{pV}{RT} = \frac{0.003921 \, \text{kg}}{3.921 \, \text{g}} \)  

Volume Conversion:  
\( V_1 = 3.14 \, \text{L} = 3.14 \cdot 10^{-3} \, \text{m}^3 \)  

Temperature Conversion:  
\( T_1 = 500^\circ \text{C} = 773.15 \, \text{K} \)  

Gas Constant Calculation:  
\( R = \frac{R}{M_g} = \frac{8.314 \, \text{J/mol·K}}{50 \, \text{kg/kmol}} = 0.16628 \, \text{m}^3 \cdot \text{Pa}/\text{K} \)

\( x_{EW,2} > 0 \)  

Since state 2 represents an equilibrium state and the EW is still present, \( T_{g,2} \) must have the same temperature as the EW, which is \( T = 0^\circ C \).  

Because the piston is given as frictionless, it does not affect the temperature or hinder equilibrium.  

\( T_{g,2} = 0^\circ C \)  

---

From above:  
\( Q_{ab} \rightarrow \text{We take } 1500 \, \text{kJ} \text{ from the task.} \)  

Energy:  
\( \Delta U_{21} = m_{EW} \cdot f \cdot T + Q_{12} - W_n \)  

Released heat:  
\( Q_{12} = -Q_{12} \)  

\( \Delta U_{21} = -Q_{12} \)  

\( m_{EW} (U_2 - U_1) = -Q_{12} \)  

\( U_2 = -\frac{Q_{12}}{m_{EW}} + U_1 \)  

\( U_1 = U_{fest} + x (U_{flüssig} - U_{fest}) \)  

Values from the given table:  
\( x = 0.6 \)  

\( U_1 = -13.94102 \, \text{kJ/kg} \)  

\( U_2 = \frac{-1500 \cdot 10^3}{m_{EW}} + U_1 = -28.94102 \, \text{kJ/kg} \)