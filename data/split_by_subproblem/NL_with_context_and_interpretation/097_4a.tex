Two diagrams are drawn to represent the freeze-drying process in a pressure-temperature (P-T) diagram.  

**First Diagram:**  
- The vertical axis is labeled as "P [mbar]" and the horizontal axis as "T [°C]".  
- A curve is drawn starting from the solid phase ("Fest") region, passing through the triple point ("T_tripel"), and extending into the liquid ("Flüssig") and gas ("Gas") regions.  
- The triple point is marked, and the pressure at this point is labeled as "5 mbar".  
- The temperature axis includes markings at -50°C, -20°C, 0°C, and 10°C.  
- A horizontal line is drawn at "T_i = 10°C".  
- The label "Wasser in Lebensmittel" (Water in food) is written near the curve.  

**Second Diagram:**  
- Similar axes are used: "P [mbar]" on the vertical axis and "T [°C]" on the horizontal axis.  
- A curve is drawn from the solid phase ("Fest") through the triple point ("T_tripel") into the gas phase ("Gas").  
- Two points are marked on the curve:  
  - Point "1" is labeled near the solid phase.  
  - Point "2" is labeled near the gas phase, with the note "T = const" (Temperature is constant).  
- A horizontal line is drawn at "T_i = 10°C".