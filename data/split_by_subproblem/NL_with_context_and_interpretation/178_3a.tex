Pressure \( p \) is calculated as:  
\( p = p_{amb} + \frac{F}{\pi r^2} \)  
\( p = p_{amb} + \frac{m_K \cdot g}{\pi r^2} \)  
\( p = 10^5 + \frac{32 \cdot 9.81}{\pi \cdot (5 \cdot 10^{-2})^2} \)  
\( p = 1.4 \, \text{bar} \)  

Radius \( r \):  
\( D = 2r \)  
\( r = 5 \, \text{cm} \)  

Mass \( m_g \):  
\( m_g = \frac{p \cdot V}{R \cdot T} \)  
Gas constant \( R \):  
\( R = \frac{R_u}{M_g} = \frac{8.314}{50 \cdot 10^{-3}} = 166.28 \, \text{J/kg·K} \)  

Substituting values:  
\( m_g = \frac{1.4 \cdot 3.14 \cdot 10^5}{166.28 \cdot (500 + 273.15)} \)  
\( m_g = 0.34 \, \text{kg} \)  

Volume \( V \):  
\( V = 3.14 \cdot 10^{-3} \, \text{m}^3 \)  

Pressure \( p \):  
\( p = 1.4 \cdot 10^5 \, \text{Pa} \)  

---