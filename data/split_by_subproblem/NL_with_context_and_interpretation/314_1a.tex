Thermodynamics 1 Exam  
30.01.2024  

a)  
Q_out = m_in (h2 - h1) - Q_R = 0.3 kilograms per second times (419.04 kilojoules per kilogram - 292.98 kilojoules per kilogram) - 100 kilowatts  

h2 = h_f (100 degrees Celsius) = 419.04 kilojoules per kilogram (Table A-2)  
h1 = h_f (70 degrees Celsius) = 292.98 kilojoules per kilogram (Table A-2)  

Q_out = 62.182 kilowatts → Q_out = 62.182 kilowatts  

---

b)  
T_KF = integral Q_out divided by T_ds = Q_out divided by s2 minus s1 = u2 minus u1 divided by s2 minus s1 = integral (T2 minus T1) divided by ln (T2 divided by T1)  

= (293.15 Kelvin - 288.15 Kelvin) divided by ln (293.15 Kelvin divided by 288.15 Kelvin)  

= 293.122 Kelvin  

---

c)  
S_erz = m_in (s2 - s1) times Q_out divided by T_KF  

Water:  
S_erz = m_in (s_out - s_in) = Q_out divided by T_reactor + Q_out divided by T_KF = 0.1036 kilowatts divided by 100 degrees Celsius - 0.2699 kilowatts divided by Kelvin = 0.403 kilowatts divided by Kelvin  

s_out = 1.3029 kilojoules per kilogram Kelvin  
s_in = 0.5549 kilojoules per kilogram Kelvin (Table A-2)  

---

d)  
-m_gas u2 + (m_gas + Delta m12) u2 = Delta m12 (h2) + Q_R12 - Q_out12  

Delta m12 u2 - Delta m12 h2 = m_gas u1 - m_gas u2  

Delta m12 = m_gas (u1 - u2) divided by u2 - h2  

u1 = 292.95 kilojoules per kilogram  
u2 = 419.04 kilojoules per kilogram  
h2 = 53.96 kilojoules per kilogram  

Delta m12 = 2031.2 kilograms  

---

e)  
Verify with given value:  

Delta S12 = m2 s2 - m1 s1  

m2 = m1 + Delta m12 = 9355 kilograms  

Delta S12 = 9355 kilograms times 0.954 kilojoules per kilogram Kelvin - 5755 kilograms times 1.332 kilojoules per kilogram Kelvin  

s2 = s_f (70 degrees Celsius) = 0.954 kilojoules per kilogram Kelvin  
s1 = x_D times s_g (100 degrees Celsius) + (1 - x_D) times s_f (100 degrees Celsius) = 1.332 kilojoules per kilogram Kelvin  

Delta S12 = 9355 kilograms times 1.33 kilojoules per kilogram Kelvin - 15.18 kilojoules per Kelvin  

= 15.18 kilojoules per Kelvin