Equations and calculations are written to determine the temperature \( T_6 \):  

1. The relationship between \( T_6 \) and \( T_5 \) is given:  
   \( T_6 / T_5 = (p_0 / p_5)^{(n-1)/n} \), where \( n = \kappa = 1.4 \).  

2. Substituting values:  
   \( T_6 = T_5 \cdot (p_0 / p_5)^{(n-1)/n} \).  
   \( T_6 = 431.9 \, \text{K} \cdot (0.191 \, \text{bar} / 0.5 \, \text{bar})^{0.4/1.4} \).  

3. The calculated result:  
   \( T_6 = 328.075 \, \text{K} \).  

Additional equations are provided:  
- \( 0 = \dot{m}_{\text{g}} \cdot (h_5 - h_6) + w_5^2 / 2 - w_6^2 / 2 \).  
- \( h_6 - h_5 = w_5^2 / 2 - w_6^2 / 2 \).  
- \( c_p (T_0 - T_5) = w_0 = 507.24 \, \text{m}^2 / \text{s} \).  

Constants and values used:  
- \( c_p = 1.006 \, \text{kJ/kg·K} \).  
- \( T_0 = 328.075 \, \text{K} \).  
- \( T_5 = 431.9 \, \text{K} \).  
- \( w_5 = 220 \, \text{m/s} \).