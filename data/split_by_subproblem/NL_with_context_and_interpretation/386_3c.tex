The volume of the gas in state 2 is calculated as:  
\( V_2 = \frac{m \cdot R \cdot T_2}{p_2} = 0.0011 \, \text{m}^3 \)  

The volume in state 1 is:  
\( V_1 = 0.0039 \, \text{m}^3 \)  

The work done is calculated as:  
\( W = \int_{1}^{2} p \, dV \)  
For an isobaric process:  
\( W = p \cdot (V_2 - V_1) \)  
Result: \( W = +0.42 \cdot 2 (9.187) \)  

The heat transfer is calculated as:  
\( \Delta U = U_2 - U_1 = Q - W \)  
\( Q = U_2 - U_1 + W \)  
For \( u_2 - u_1 = m \cdot c_V \cdot (T_2 - T_1) \):  
\( Q = -850.521 \)  

Final result:  
\( Q_{12} = -Q = 1.0929 \, \text{kJ} \)