U subscript 2, EW minus U subscript 1, EW equals Q subscript 12 equals m subscript EW times (x subscript 2 times u subscript f plus (1 minus x subscript 2) times u subscript ice) minus (x subscript 1 times u subscript f plus (1 minus x subscript 1) times u subscript ice).  
Q subscript 12 equals m subscript EW times (x subscript 1 times (u subscript f minus u subscript ice) plus u subscript ice) minus (x subscript 1 times (u subscript f minus u subscript ice) minus u subscript ice).  
Equals (x subscript 1 minus x subscript 2) times (u subscript f minus u subscript ice) plus x subscript 1 times u subscript ice.  
u subscript ice equals u subscript ice at 0 degrees Celsius.

\( u_{vs} \), \( u_{vl} \)  
\( u_{(Eis)} \), \( u_{(Wass)} \) at \( 2000^\circ \text{C} \)  

\( V_{2,EW} - V_{1,EW} = -Q_{12} = m_{EW} \left( x_2 u_L + (1 - x_2) u_S \right) - \left( x_1 u_L + (1 - x_1) u_S \right) \)  

\( -\frac{Q_{12}}{m_{EW}} = x_2 (u_L - u_S) + u_S - \left( x_1 (u_L - u_S) + u_S \right) \)  

\( = (x_2 - x_1) (u_L - u_S) \)  

\( x_2 = -\frac{Q_{12}}{m_{EW} (u_L - u_S)} + x_1 = 0.574 \)