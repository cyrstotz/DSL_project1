\( \Delta U = \dot{m}_{ges} \cdot u_1 + (\dot{m}_{ges} + \Delta m_{12}) \cdot u_2 \).  

\( h_{in} \) at \( T_{in,12} = 20^\circ C \): \( h_{in} = 83.96 \, \text{kJ/kg} \) (from Table A2).  

\( u_1 \) at \( T_{Reactor,1} = 100^\circ C \): \( u_1 = 429.3778 \, \text{kJ/kg} \) (from Table A2).  
\( u_2 \) at \( T_{Reactor,2} = 70^\circ C \): \( u_2 = 429.2925 \, \text{kJ/kg} \) (from Table A2).  

\( \Delta U = 0 \).  

\( Q_{R,12} = Q_{out,12} = 35 \, \text{MJ} \).

m subscript gas multiplied by open parenthesis u subscript i minus u subscript 2 close parenthesis minus Delta m subscript 12 multiplied by u subscript 2 equals Delta m subscript 12 multiplied by h superscript in subscript e comma n.  

Delta m subscript 12 equals m subscript gas multiplied by open parenthesis u superscript in subscript e comma n minus u subscript 2 close parenthesis divided by h superscript in subscript e comma n plus u subscript 2 equals 6.408 T.