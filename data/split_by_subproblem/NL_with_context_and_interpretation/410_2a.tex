A graph is drawn with the y-axis labeled as 'T [K]' and the x-axis labeled as 'S [kJ]'. The graph represents a thermodynamic process with points labeled as 1, 2, 3, 4, and 5.  
- Between points 1 and 2, the process is marked as 'isentrop'.  
- Between points 2 and 3, the process is marked as 'isentrop'.  
- Between points 4 and 5, the process is marked as 'isentrop'.  
- Between points 3 and 4, the process is labeled as 'isobar p3/p4'.

w equals R multiplied by (T6 minus T5), divided by 1 minus n.  

wc equals 259.56 kilojoules.  

wc equals the square root of 2 multiplied by (h6 minus h5 minus w dot divided by w5 squared divided by 2).  

w5 equals 220 meters per second.  

(h5 minus h6) equals Cp multiplied by (T6 minus T5).  

wc equals the square root of 2 multiplied by (Cp multiplied by (T6 minus T5) minus R multiplied by (T6 minus T5) divided by 1 minus n), divided by w5 squared divided by 2.  

wc equals 551.5 meters per second.  

---