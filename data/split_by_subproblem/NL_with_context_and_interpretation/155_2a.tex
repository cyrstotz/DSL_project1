0-1: Adiabatic compression, \( p = 0.191 \, \text{bar} \)  
1-2: Adiabatic-reversible compression (isentropic)  
2-3: Isobaric heat addition  
3-4: Adiabatic-irreversible turbine (entropy generation)  
4-5: Isobaric mixing chamber, \( p = 0.5 \, \text{bar} \)  
5-6: Reversible adiabatic nozzle (isentropic), \( p = 0.191 \, \text{bar} \)  

Graph description:  
The graph is a T-s diagram with temperature \( T \) in Kelvin on the vertical axis and entropy \( s \) on the horizontal axis.  
Key points are labeled:  
- \( T = 1289 \, \text{K} \), \( T = 431.9 \, \text{K} \), \( T = 243.15 \, \text{K} \).  
- Processes are marked with lines connecting states 0, 1, 2, 3, 4, 5, and 6.  
- Isobaric and isentropic processes are indicated.  
- The label "Auf anderen Seite" (on the other side) appears near the graph.

\( s_5 - s_6 = 0 \)  
\( s_5 - s_6 = cp \int_{T_5}^{T_6} \frac{1}{T} dt - R \ln \left( \frac{p_5}{p_6} \right) \)  

\( 0 = 1.006 \, \text{kJ/kg·K} \, \text{(integral crossed out)} - R \ln \left( \frac{p_5}{p_6} \right) \)  

\( R = cp - \frac{cp}{n} = 0.2874 \, \text{kJ/kg·K} \)  
\( p_5 = 0.5 \, \text{bar}, \, p_6 = 0.191 \, \text{bar} \)  

\( 0.276575 \, \text{kJ/kg·K} = 1.006 \, \text{kJ/kg·K} \ln \left( \frac{431.9 \, \text{K}}{T_6} \right) \)  

\( 0.274925 = \ln \left( \frac{431.9 \, \text{K}}{T_6} \right) \)  
\( 1.316433 = \frac{431.9 \, \text{K}}{T_6} \)  
\( \Rightarrow T_6 = 328.08 \, \text{K} \)  

\( h_5 - h_6 = cp (T_5 - T_6) \)  
\( h_5 - h_6 = 1.006 \, \text{kJ/kg·K} \, (431.9 \, \text{K} - 328.08 \, \text{K}) \)  
\( h_5 - h_6 = 104.44 \, \text{kJ/kg} \)  

\( -104.44 \, \text{kJ/kg} = \frac{w_5^2}{2} - \frac{w_6^2}{2} \)  
\( w_6^2 = 2 \left( \frac{w_5^2}{2} + 104.44 \, \text{kJ/kg} \right) \)  
\( w_6^2 = 48608.88 \)  
\( \Rightarrow w_6 = 220.47 \, \text{m/s} \)

A graph is drawn with the y-axis labeled as 'T [K]' and the x-axis labeled as 'S [kJ/kg·K]'. The graph shows several curves and points labeled as follows:  
- The y-axis has values marked at 1289, 431.9, 322.08, and 243.15.  
- The curve starts at point '0', moves to '1', then '2', and continues to '3', '4', '5', and '6'.  
- Two pressure levels are indicated: 'P = P2' and 'P = P0'.  
- The graph qualitatively represents the process in a T-S diagram with labeled isobars.  

---