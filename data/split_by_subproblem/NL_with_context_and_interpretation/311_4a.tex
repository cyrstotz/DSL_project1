The page contains three diagrams labeled with pressure (P) and volume (V) axes.  

1. The first diagram shows a curve labeled "Flüssig" (liquid) transitioning into "Nassdampf" (wet steam) and then into "Dampf" (steam). Points are marked as 4 and 1 along the curve.  

2. The second diagram also shows a curve labeled "Flüssig," transitioning into "Nassdampf" and "Dampf." Points 4 and 1 are marked again, with arrows indicating the direction of the process. The term "Isobare" (isobar) is written near the curve.  

3. The third diagram shows a similar curve with points labeled 4, 1, 2, and 3. The regions are labeled "Flüssig," "Nass Dampf," and "Dampf." The term "Isobare" is written near the curve, and arrows indicate the direction of the process.  

These diagrams appear to represent the freeze-drying process in a pressure-volume (P-V) diagram, showing phase transitions and isobaric processes.