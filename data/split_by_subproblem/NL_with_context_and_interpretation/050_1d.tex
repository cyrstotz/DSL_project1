d)  
\( T_{KF} = 70^\circ C \)  
\( \Delta m_{12} \)  
\( T_{in,12} = 20^\circ C \)  

State transition: \( 1 \to 2 \)  

Equation:  
\( m_2 u_2 - m_1 u_1 - \Delta m h_{sp} = Q \)  
\( (m + \Delta m) u_2 - m u_1 - \Delta m h_{sp} = Q \)  

\( m_2 = m_1 + \Delta m \)

m subscript 2 u subscript 2 plus m subscript e (u subscript 2 minus h subscript e) minus m subscript 1 u subscript 1 equals Q.  

m subscript e equals (Q plus m subscript 1 (u subscript 1 minus u subscript 2)) divided by (u subscript 2 minus h subscript e).  

Q equals negative 35 megajoules.  
m subscript 1 equals 5755 kilograms.  

T subscript 1 equals 100 degrees Celsius.  
T subscript 2 equals 70 degrees Celsius.  
T subscript e equals 20 degrees Celsius.  

From Table A32:  
u subscript 1 (100 degrees Celsius, saturated fluid) equals 427.38 kilojoules per kilogram.  
u subscript 2 (70 degrees Celsius, saturated fluid) equals 419.44 plus 0.005 (2050.45).  
u subscript e (20 degrees Celsius, saturated fluid) equals h subscript e (20 degrees Celsius, saturated fluid).