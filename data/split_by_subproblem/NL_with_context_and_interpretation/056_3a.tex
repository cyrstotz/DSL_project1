Delta U equals Q.  
Mass times (u2 minus u1) equals Q.  
Mass times cv times (T2 minus T1) equals Q.  
Q12 equals minus 1.469 times 10^3.

A diagram is drawn showing the pressure \( p_{amb} \) acting on the piston, the mass \( m_g \) of the gas, and the ice-water mixture (EW). Below the EW, the pressure \( p_{gas} \) is indicated.  

The equation for \( p_{gas} \) is given as:  
\( p_{gas} = p_{amb} + \frac{m_g}{A} \)  

The pressure \( p_{gas} \) is calculated as:  
\( p_{gas} = 1.399 \, \text{bar} \approx 1.4 \, \text{bar} \)  

The area \( A \) is calculated using the formula:  
\( A = \frac{D^2}{4} \pi \)  

Given values:  
\( p_{amb} = 1 \, \text{bar} \), \( m_K = 32 \, \text{kg} \), \( D = 10 \, \text{cm} \).