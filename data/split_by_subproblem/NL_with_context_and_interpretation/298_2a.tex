A graph is drawn with the y-axis labeled as 'T (K)' and the x-axis labeled as 'S (kJ/kg·K)'. The graph represents a T-s diagram with several labeled points and processes:  

- Point 1: \( T_0, p_0 \)  
- Point 2: \( p_1 > p_0 \)  
- Point 3: High-pressure compressor  
- Point 4: Adiabatic, irreversible turbine  
- Point 5: Mixing chamber with \( T_5, w_5, p_5 \)  
- Point 6: Reversible, adiabatic nozzle with \( p_0 \), \( w_6 \), \( T_0 \)  

The processes are labeled as follows:  
- Between points 1 and 2: Adiabatic, irreversible compression.  
- Between points 2 and 3: High-pressure compressor, adding heat to reach \( p_2, T_2 \).  
- Between points 3 and 4: Adiabatic, irreversible turbine.  
- Between points 4 and 5: Mixing chamber.  
- Between points 5 and 6: Reversible, adiabatic nozzle.  

Additional notes on the graph:  
- \( \dot{m}_M \) and \( \dot{m}_K \) are indicated near point 2.  
- The curve between points 3 and 4 is labeled as 'irreversible entropy increase'.  
- The curve between points 5 and 6 is labeled as 'reversible entropy decrease'.  

Below the graph, the following steps are written:  
1. \( T_0, p_0 \): Adiabatic, irreversible \( w_L < \) compression.  
2. \( p_1 > p_0 \): \( \dot{m}_M \) flows into the mantle without mixing. \( \dot{m}_K \) flows into the core, adding heat to reach \( p_2, T_2 \).  
3. From 2 to 3: Mixing chamber, mantle not heated.  
4. From 3 to 4: Adiabatic, irreversible turbine.  
5. At 5: Mixing chamber with \( T_5, w_5, p_5 \).  
6. From 5 to 6: Reversible, adiabatic nozzle \( \rightarrow p_0, w_6, T_0 \).

Exergy loss equals exergy balance.  

Zero equals the sum of exergy streams in minus the sum of exergy streams out plus exergy heat minus exergy work minus exergy loss.  

Exergy loss equals the sum of exergy streams in minus the sum of exergy streams out divided by mass flow rate.  

Zero equals the sum of exergy streams in minus the sum of exergy streams out plus exergy heat minus work minus exergy loss.  

Exergy loss equals exergy heat divided by thermodynamic mean temperature minus exergy streams in plus one minus T zero divided by T B times Q B minus work.  

Exergy loss minus delta exergy streams plus one minus T zero divided by T B times Q B equals 100 kilojoules per kilogram times one minus 243.15 kelvin divided by 1289 kelvin minus 1.45 kilojoules per kilogram.  

Approximately equals 106.72 kilojoules per kilogram.