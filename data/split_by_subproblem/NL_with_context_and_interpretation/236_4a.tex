State Table:  

| Zustand | P           | T       | x   |  
|---------|-------------|---------|-----|  
| 1       | p1 = p2     | 1.5748 bar |       |  
| 2       | p2 = p1     | 1.5748 bar | 257.15 K | 1   |  
| ges. Dampf |           |         |     |  
| 3       | 8 bar       |         |     |  
| 4       | 8 bar       |         | 0   |  

Wk = 25 W adiabatic  

T in Verdampfer = Ti - 6 K = 263.15 K - 6 K = 257.15 K  
Ti aus Abb 5 abgelesen: Sublimationspunkt -20°C = 253.15 K  
Ti = 253.15 K + 10 K = 263.15 K

Two diagrams are drawn:  

1. A pressure-temperature (\( p \)-\( T \)) diagram:  
   - The vertical axis is labeled \( p \, [\text{mbar}] \).  
   - The horizontal axis is labeled \( T \, [^\circ \text{C}] \).  
   - Two processes are shown:  
     - Process i: Isobaric.  
     - Process ii: Sublimation.  

2. A second \( p \)-\( T \) diagram:  
   - The vertical axis is labeled \( p \, [\text{mbar}] \).  
   - The horizontal axis is labeled \( T \, [^\circ \text{C}] \).  
   - The phase regions are illustrated with points for processes i and ii.