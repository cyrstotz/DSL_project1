Q equals m dot times (h two minus h one)  

Adiabatic → h two equals h one (8 bar) equals 93.42 kilojoules per kilogram  

h one equals h f plus x one times (h g minus h f)  

p one equals p two  

T equals minus 22 degrees Celsius → h g (minus 22 degrees Celsius) equals 233.08 kilojoules per kilogram  

h f equals 25.77 minus u 32 equals 21.472 kilojoules per kilogram  

u f equals 1.2192 times 10 to the power of 4  

h g equals 236.04 minus 233.86 equals (1.2192 times 10 to the power of 4 minus u 2) plus 233.86 minus 233.06 kilojoules per kilogram  

x one equals 0.358