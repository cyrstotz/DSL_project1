The pressure of the gas in state 1 is calculated as:  
\( p_{g,1} = p_{amb} + \frac{m_K \cdot g}{A} + \frac{m_{EW} \cdot g}{A} \)  
The area \( A \) is determined as:  
\( A = \pi \cdot r^2 = 0.03142 \, \text{m}^2 \)  
Result: \( p_{g,1} = 1.1 \, \text{bar} \)  

The mass of the gas \( m \) is calculated using the ideal gas law:  
\( p_1 \cdot V_1 = m \cdot R \cdot T_1 \)  
Substituting values:  
\( p_1 = 1 \, \text{bar}, V_1 = 3.14 \, \text{L}, R = 8.314 \, \text{J/mol·K}, T_1 = 500^\circ \text{C} \)  
\( m = \frac{p_1 \cdot V_1}{R \cdot T_1} = \frac{1 \cdot 3.14}{8.314 \cdot 500} \)  
Result: \( m = 2.687 \, \text{g} \)  

---