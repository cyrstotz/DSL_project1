A table is drawn with columns labeled as:  
- 'P' (pressure)  
- 'T' (temperature)  
- 'V' (volume)  
- 'W' (work)  
- 'Q' (heat transfer)  
- 'S' (entropy)  
- 'X' (vapor quality)  

The rows are labeled as states 1, 2, 3, and 4:  
- State 1:  
  - P = P2  
  - T = 1.5748 bar  
- State 2:  
  - P = 1.5748 bar  
  - T = -16°C  
  - W = 28 W  
  - Q = 0  
  - S = 0  
  - X = 1  
- State 3:  
  - P = 8 bar  
  - Q = 0  
  - X = 0  
- State 4:  
  - P = 8 bar  
  - Q = 0  
  - X = 0  

Additional notes:  
- P1 = P2  
- T2 = T1 - 6K  
- T_i = Abb 5 (p = 1 bar)  
- T2 = -16°C  
- Q44 = 0  
- Q23 = 0  
- S23 = adiabatic + reversible = 0  
- P2 = Tab A-10 (T = -16°C) = 1.5748 bar = P1

Heat flow equals mass flow rate multiplied by the difference between enthalpy at the exit and enthalpy at the inlet, plus the heat removed.  
Heat flow equals mass flow rate multiplied by the difference between enthalpy at the inlet and enthalpy at the exit.

Enthalpy at the inlet equals the value from Table A-10 at temperature T equals negative 76 degrees Celsius (for gaseous state, h_g).  
Enthalpy at the inlet equals 237.14 kilojoules per kilogram.

Work equals 28 watts.