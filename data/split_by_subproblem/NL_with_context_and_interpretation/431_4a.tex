Two diagrams are drawn:  

1. The first diagram is labeled "Schritt ii" and shows a pressure-temperature (p-T) graph.  
   - The vertical axis is labeled "p(R134a)" (pressure of R134a).  
   - The horizontal axis is labeled "T(K)" (temperature in Kelvin).  
   - A curve is drawn, with regions labeled "gasförmig" (gaseous) and "flüssig" (liquid).  
   - A point is marked on the curve, labeled "1".  

2. The second diagram is labeled "Schritt i" and shows another pressure-temperature (p-T) graph.  
   - The vertical axis is labeled "p(R134a)" (pressure of R134a).  
   - The horizontal axis is labeled "T(K)" (temperature in Kelvin).  
   - A dome-shaped curve is drawn, representing phase regions.  
   - The regions are labeled "überhitzter Dampf" (superheated vapor), "Masse Dampf" (mass vapor), and "unterkühlte Flüssigkeit" (subcooled liquid).  
   - Points labeled "1", "2", "3", and "4" are marked along the curve.  

---