The goal is to calculate \( w_6 \) and \( T_6 \).  
Given:  
\( w_5 = 220 \, \text{m/s} \), \( p_5 = 0.5 \, \text{bar} \), \( T_5 = 431.9 \, \text{K} \).  
\( S_5 = S_6 \) because the process is isentropic.  
Air is treated as an ideal gas.  
\( c_p = 1.006 \, \text{kJ/kg·K} \).  

Since the process is adiabatic:  
\( \frac{T_6}{T_5} = \left( \frac{p_6}{p_5} \right)^{\frac{n-1}{n}} \).  

Substituting values:  
\( T_6 = 431.9 \, \text{K} \cdot \left( \frac{0.191 \, \text{bar}}{0.5 \, \text{bar}} \right)^{\frac{1.4-1}{1.4}} = 328.075 \, \text{K} \).  

The equation \( u_6 - u_5 = c_v \Delta T \) is written, where \( c_v = c_p - R \).