The cross-sectional area \( A \) is calculated as:  
\( A = 5 \, \text{cm}^2 \cdot \pi = 0.00785 \, \text{m}^3 \).  

The force \( F \) is determined using:  
\( F = m \cdot g = 32 \, \text{kg} \cdot 9.81 \, \text{m/s}^2 = 314.9 \, \text{N} \).  

The pressure \( p_e \) is calculated as:  
\( p_e = \frac{317.9 \, \text{N}}{7.85 \cdot 10^{-3} \, \text{m}^2} = 0.84 \, \text{bar} \).  

The total pressure \( P \) is:  
\( P = 1 \, \text{bar} + 0.9 \, \text{bar} = 1.9 \, \text{bar} \).  

The molar mass \( M \) is given as:  
\( M = 50 \, \text{kg/kmol} \).  

The mass \( m \) is calculated using the ideal gas law:  
\( m = \frac{p \cdot V}{R \cdot T} = 0.0034 \, \text{kg} = 3.4 \, \text{g} \).  

The temperature \( T \) is:  
\( T = 773.15 \, \text{K} \).  

The pressure \( \phi \) is:  
\( \phi = 7.9 \cdot 10^3 \, \text{Pa} \).  

The gas constant \( R \) is calculated as:  
\( R = \frac{\bar{R}}{M} = \frac{8.314 \, \text{kJ/kmol·K}}{50 \, \text{kg/kmol}} \cdot 10^3 = 766.28 \, \text{J/kg·K} \).  

The volume \( V \) is:  
\( V = 3.14 \cdot 10^{-3} \, \text{m}^3 \).  

---