a) Exergy balance equation:  
\( \dot{m} \cdot (h_{in} - h_{out}) + \dot{Q} = 0 \)  
\( \dot{m} \cdot (h_{in} - h_{out}) = \dot{Q} \)

A table is provided with the following columns and values:  

Columns:  
- Zustand (State)  
- P [bar]  
- T [°C]  
- w [m/s]  
- s [kJ/kg·K]  
- h [kJ/kg]  

Values:  
- State 0: P = 0.191, T = -30, w = 200, s = -, h = -  
- State 1: P = -, T = -, w = -, s = -, h = -  
- State 2: P = -, T = -, w = -, s = -, h = -  
- State 3: P = -, T = -, w = -, s = -, h = -  
- State 4: P = -, T = -, w = -, s = -, h = -  
- State 5: P = 0.5, T = 431.9, w = 220, s = 6.274, h = 476.1  
- State 6: P = 0.5, T = -, w = -, s = -, h = -  

Additional notes:  
- \( w_6 = \sqrt{2 \cdot \Delta h \cdot 1000} \)  
- \( \Delta h = h_5 - h_6 = 476.1 - 0.27 \)  
- \( w_6 = \sqrt{2 \cdot 475.83 \cdot 1000} \)  
- \( w_6 = 975.8 \, \text{m/s} \)  

Handwritten label: "Aufgabe 2"

The diagram is a T-s (temperature-entropy) graph.  

- The x-axis is labeled as 's [kJ/kg·K]'.  
- The y-axis is labeled as 'T [K]'.  

The graph includes several curves and points:  
1. Point '0' is marked at the bottom left.  
2. A curve labeled 'isobar (ambient)' starts from point '0' and extends upward.  
3. Point '3' is marked along the curve labeled 'isobar (ambient)'.  
4. A curve labeled 'isobar (high pressure)' is drawn parallel to the ambient isobar.  
5. Point '6' is marked along the high-pressure isobar.  
6. A connecting line between points '3' and '6' is labeled 'isentropic compression'.  
7. A line extending from point '6' downward is labeled 'isentropic expansion'.  

Additional annotations:  
- 'Isobar (ambient)' and 'Isobar (high pressure)' are clearly labeled near their respective curves.  
- The graph visually represents the thermodynamic processes described in the jet engine task.