
``````latex


\section*{Aufgabe 4}

\subsection*{a)}
\begin{description}
    \item[Graph 1:] A graph with pressure \( p \) on the y-axis and temperature \( T \) on the x-axis. There is a closed loop with four points labeled 1, 2, 3, and 4. The loop is shaded.
    \item[Graph 2:] A graph with pressure \( p \) on the y-axis and temperature \( T \) on the x-axis. There is a dome-shaped curve with points labeled 1, 2, 3, and 4. The left side of the dome is labeled "Flüssig" and the right side is labeled "Gas". The region under the dome is labeled "Nassdampf".
\end{description}

\subsection*{b)}
\begin{align*}
    s_2 &= s_3 \\
    0 &= \dot{m} (h_2 - h_3) - \dot{W}_K \\
    T_i \\
    p &= 1 \text{mbar}
\end{align*}

\subsection*{c)}
\begin{equation*}
    h_1 = h_4
\end{equation*}

\subsection*{e)}
Sie würde größer den mehr Exergieverlust

\subsection*{d)}
\begin{equation*}
    \epsilon_K = \frac{\dot{Q}_K}{\dot{W}_K}
\end{equation*}

```