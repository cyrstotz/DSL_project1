
``````latex


\section*{Aufgabe 3}

\subsection*{a)}

\begin{align*}
\text{cgs} \quad Mg & \implies p_m = \frac{m_a \cdot R_a \cdot T_1}{V_a} \\
R &= \frac{E}{Mg} = 166.28 \, \frac{J}{kg \cdot K} \\
m_a &= \frac{p_a \cdot V_a}{R_a \cdot T_a} \\
A_2 &= T_1 \cdot r^2 = \frac{\Pi \cdot r^2}{9} = \frac{\Pi \cdot r}{400} \\
p &= \frac{m_E \cdot V}{A_2} + \frac{32 \, kg \cdot g}{A_2} + p_m = 160 \, kPa - \text{Atmos} \, 1.6 \, bar \\
m_a &= 3.4217 \, g
\end{align*}

\subsection*{b)}

\begin{align*}
\left\{ \frac{dE}{dt} = \dot{Q}(\cdots) + \dot{Q}_2^0 - \dot{W}_u \right\} \\
\implies \Delta E = \Delta U = -W_u \\
\frac{dE}{dt} &= \text{lin} \\
p_{g12} - p_{Ev12}
\end{align*}

\subsection*{Graphical Description}

The graphical content consists of several parts:

1. A wavy line that starts from the left and oscillates up and down across the page. This line represents a function or a process over time.
2. There are annotations along the wavy line, indicating changes in energy or other quantities. These annotations include:
   - $\Delta E$
   - $\Delta U$
   - $\Delta E_{int} = \text{cif}$

3. To the right of the wavy line, there is a vertical double line, which might represent a boundary or a separation between different regions or phases.

4. The text "for gas" is written next to the vertical double line, indicating that the equations or processes described are specific to a gas.

``````latex

\section*{Student Solution}

\subsection*{b)}

\begin{align*}
\text{p = const} & \implies T_{3,2} = T_{3,1} \left( \frac{V_{1g}}{V_{1f}} \right) \\
& \\
& \frac{d}{dt} \left( V_{1g} \right) = - \frac{W_v}{\rho} + V_r \\
& \\
\frac{dE}{dt} & = \dot{m} (\cdots) + \dot{Q}_{in} - \dot{W}_v \\
& \\
\Delta \text{u}_{\text{in}} = \dot{Q}_{in} - \dot{W}_v & \quad \text{and} \quad \Delta \text{u}_{\text{ing}} = \dot{Q}_{in} - \dot{W}_v \\
& \implies \dot{W}_v = 
\end{align*}

\subsection*{c)}

\begin{align*}
T_{3,2} &= 0.005^\circ C \\
& \\
\frac{dE}{dt} & = \dot{m} (\cdots) + \dot{Q}_{in} - \dot{W}_v \\
& \\
dE - m_g \cdot \Delta \text{u}_{\text{ing}} & = m_g \cdot c_v (T_2 - T_1) = \\
& \\
\text{W}_v = \text{p const} & \implies \frac{T_2}{T_1} = \frac{V_2}{V_1} \implies V_2 = 1.109 L \\
& \\
W_v & = (V_2 - V_1) \cdot \rho = -284.28 J \\
& \\
\implies \dot{Q}_{in} & = -1082.96 \text{ kJ} + 284.28 = -798.68 \text{ J} - 1367.24 \text{ J} \\
& \\
\implies |\dot{Q}_{in}| & = 137.26 \text{ J}
\end{align*}

``````latex


d) $p = p_{s,4} = 7.4 \, \text{bar}$ \hspace{1cm} $Q = 15000$

\[
\frac{\Delta E}{dt} = \dot{m}_{2v} \cdot \left( \frac{h_{2v}}{v} \right) + \dot{Q}_{2v} - \dot{W}
\]

\[
\Delta E = \dot{m}_{2v} \cdot \left| Q_{2v} \right|
\]

\[
\Delta u \cdot m_{2v} = \left| Q_{2v} \right| \Rightarrow \frac{15000}{h_{g}} \Rightarrow \delta u
\]

\[
\Delta u = \text{(some scribbles)}
\]

\[
\Rightarrow \dot{x}_{\text{Flüssig}} + \Delta x - \Delta u = 0.04 \Rightarrow \text{(some scribbles)} \hspace{1cm} x_{d,3} = 0.9330 = 1 - x_{\text{Flüssig}}
\]

\[
\frac{h_{g} (7.4 \, \text{bar}) - h_{g} (1.4 \, \text{bar})}{\text{(some scribbles)}}
\]

Below this, there are two small sketches labeled "Dampf" and "Flüssig". 

- The sketch labeled "Dampf" shows a vertical arrow pointing upwards with a horizontal line at the top. 
- The sketch labeled "Flüssig" shows a vertical arrow pointing downwards with a horizontal line at the bottom.

``````latex


