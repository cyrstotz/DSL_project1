
``````latex


\section*{A4) a)}

\begin{description}
    \item[Figure 1:] The first graph is a plot with the vertical axis labeled as \( p \) and the horizontal axis labeled as \( T \). The graph contains three lines:
    \begin{itemize}
        \item A straight line starting from the origin and going upwards to the right.
        \item Another straight line starting from the origin and going upwards to the left, forming an 'X' shape with the first line.
        \item A curved line starting from the bottom left, rising to a peak, and then descending to the bottom right. This curved line is labeled as "NO".
    \end{itemize}
    The intersection of the straight lines forms a triangular region, and the curved line passes through this region.
    
    \item[Figure 2:] The second graph is a plot with the vertical axis labeled as \( p \) and the horizontal axis labeled as \( T \). The graph contains a single straight line:
    \begin{itemize}
        \item A straight horizontal line starting from the left and extending to the right.
    \end{itemize}
    The horizontal axis has a point labeled \( T_i \) near the origin.
\end{description}

``````latex


A4) b)

\[
\begin{array}{ccccc}
\phi & p & T & h & s \\
1 & 1.3748 & -16 & & \\
x=1 & 2 & p1 & -16 & \\
3 & 8 & & & \\
x=0 & 4 & 8 & & \\
\end{array}
\]

\[
T_{1,2} = T_i - 6K = -16^\circ C
\]

\[
h_2 = h_g(-16^\circ C) \rightarrow A10
\]

\[
h_2 = 237.74 \frac{kJ}{kg}
\]

\[
s_2 = s_3 = s_g(-16^\circ C) \quad s_2 = 0.9288 \frac{kJ}{kgK}
\]

\[
p_3 = 8 \text{bar} \rightarrow \text{via A-M sehen wir, wir sind im Dampfgebiet}
\]

\[
\rightarrow A-12
\]

\[
h_3 = h_{sat} + \frac{(h(40) - h_{sat})}{s(40) - s_{sat}} \cdot (s_3 - s_{sat})
\]

\[
h(40) = 273.66
\]

\[
h_{sat} = 269.45
\]

\[
s_{sat} = 0.9066
\]

\[
s(40) = 0.9374
\]

\[
h_3 = 271.3 \frac{kJ}{kg}
\]

``````latex


\begin{align*}
h_2 &= 237.74 \\
h_3 &= 271.3 \\
\end{align*}

\text{2} \rightarrow \text{3} \quad \text{1. HS}

\begin{align*}
0 &= \dot{m}(h_2 - h_3) + \dot{Q} - \dot{W} \\
\frac{\dot{W}}{h_2 - h_3} &= \dot{m}
\end{align*}

\text{adiabat} \quad \dot{W} \downarrow -28 \frac{\text{kW}}{\text{kg}}

\begin{align*}
\dot{m} &= 0.884 \frac{\text{g}}{\text{s}}
\end{align*}

\underline{\dot{m} = 0.884 \frac{\text{g}}{\text{s}}}

\begin{itemize}
\item[c)] \text{ich nehme} \quad \dot{m} = 4 \frac{\text{kg}}{\text{h}} \quad \text{for safety} = \frac{4}{3600} \frac{\text{kg}}{\text{s}}
\end{itemize}

\begin{align*}
0 &= \dot{m}(h_1 - h_2) + \dot{Q}_K \\
-\frac{\dot{Q}_K}{\dot{m}} + h_2 &= h_1
\end{align*}

\begin{align*}
h_3 &= 271.3 \frac{\text{kJ}}{\text{kg}}
\end{align*}

\text{3} \rightarrow \text{4} \quad \text{ischor} \quad 2 \nu: \quad 8 \text{bar} \quad x = 0 \quad h_4 = h_f(8 \text{bar})

\begin{align*}
h_4 &= 93.42 \frac{\text{kJ}}{\text{kg}}
\end{align*}

\begin{align*}
\dot{m}(h_3 - h_4) + \dot{Q} &= 0 \\
Q &= 0.198 \text{kW}
\end{align*}

\text{Aufgrund Kreisprozess:} \quad W + \dot{E}_{\text{ab}} = 0

``````latex

4c)

\begin{align*}
-28W + (-197.6W) + \dot{Q}_c &= 0 \\
\dot{Q}_c &= 225.6W
\end{align*}

\begin{align*}
\dot{m}(h_1 - h_2) + \dot{Q} &= 0 \\
h_1 &= -\frac{\dot{Q}}{m} + h_2
\end{align*}

```