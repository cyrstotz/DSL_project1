
``````latex


\section*{Aufgabe 2}

\subsection*{a)}

\begin{description}
    \item[Graph Description:] The graph is a Temperature-Entropy (T-s) diagram. The x-axis is labeled \( s \, (\frac{\text{kJ}}{\text{kg} \cdot \text{K}}) \) and the y-axis is labeled \( T \, (\text{K}) \). There are two isobaric lines: one labeled \( 0.5 \, \text{bar} \) and the other labeled \( 0.9316 \, \text{bar} \). The graph shows two vertical arrows indicating processes at constant entropy. The first arrow starts at point \( O \) and ends at point \( 5 \) on the \( 0.5 \, \text{bar} \) line. The second arrow starts at point \( O \) and ends at point \( L \) on the \( 0.9316 \, \text{bar} \) line. Both arrows are labeled as "adiabatic".
\end{description}

\subsection*{b)}

\begin{align*}
    \omega_{\text{up}} &= 200 \, \frac{\text{m}}{\text{s}} \\
    p_0 &= 0.9316 \, \text{bar} \\
    T_0 &= -30^\circ \text{C} = 270.15 \, \text{K} \\
    S_0 &= \\
    \dot{m}_{\text{ges}} &= \dot{m}_n + \dot{m}_k
\end{align*}

\begin{align*}
    S_1 &= S_0 \\
    T_1 &= \\
    p_1 &= \\
    S_5 &= S_0 \\
    T_5 &= 431.8 \, \text{K} \\
    p_5 &= 0.5 \, \text{bar} \\
    \omega_5 &= 200 \, \frac{\text{m}}{\text{s}}
\end{align*}

\begin{align*}
    T_2 &= \\
    p_2 &= p_3 \\
    T_3 &= \\
    p_3 &= p_2 \\
    T_4 &= \\
    p_4 &= 
\end{align*}

\begin{align*}
    S_6 &= S_5 \quad \text{(isentrop)} \\
    T_6 &= T_5 \left( \frac{p_6}{p_5} \right)^{\frac{n-1}{n}} = 328.0747 \, \text{K}
\end{align*}

\begin{align*}
    Q &= \dot{m}_{\text{ges}} \left( h_5 - h_6 + \frac{\omega_5^2 - \omega_6^2}{2} \right) \\
    &= c_p (T_5 - T_6) + \frac{\omega_5^2}{2} = \frac{\omega_6^2}{2}
\end{align*}

\begin{align*}
    \omega_6 &= \sqrt{2 c_p (T_5 - T_6) + \omega_5^2} = 507.244 \, \frac{\text{m}}{\text{s}}
\end{align*}

\begin{align*}
    S_0^\circ &= S_n^\circ = 270.26 \, \frac{\text{kJ}}{\text{kg} \cdot \text{K}}
\end{align*}

``````latex


