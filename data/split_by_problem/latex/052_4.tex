
``````latex


\section*{Aufgabe 4}

\subsection*{a)}

\begin{itemize}
    \item \textbf{Graph I:} 
    \begin{itemize}
        \item The graph is a PV diagram with the y-axis labeled as \( p(\text{bar}) \) and the x-axis labeled as \( V \).
        \item There is a closed loop with four points labeled 1, 2, 3, and 4.
        \item The path goes from 1 to 2, 2 to 3, 3 to 4, and 4 back to 1.
        \item The path from 1 to 2 is horizontal, from 2 to 3 is vertical, from 3 to 4 is horizontal, and from 4 to 1 is vertical.
    \end{itemize}
    
    \item \textbf{Graph II:} 
    \begin{itemize}
        \item The graph is a PT diagram with the y-axis labeled as \( p(\text{bar}) \) and the x-axis labeled as \( T(\text{K}) \).
        \item There are two curves, one labeled \( \text{liquid} \) and the other \( \text{vapor} \).
        \item There are two points labeled 1 and 2 on the liquid curve.
        \item There is an arrow pointing from point 1 to point 2.
    \end{itemize}
\end{itemize}

\subsection*{b)}

\begin{align*}
    x_2 &= 1 \\
    x_4 &= 0 \\
    s_4 &= s_f = 0.349 \frac{\text{kJ}}{\text{kgK}} \\
    s_4 &= 83.42 \frac{\text{kJ}}{\text{kg}} \\
    T_i &= 10^\circ \text{C} \\
    \text{Trockendampf} &= 40^\circ \text{C} \\
    s_2 &= s_3 = 0.9163 \frac{\text{kJ}}{\text{kgK}} \\
    h_2 &= 249.58 \frac{\text{kJ}}{\text{kg}} \\
    0 &= \dot{m} (h_2 - h_3) + \dot{W}_i \\
    h_3 &= ? \\
    h_3 &= h_{\text{in}} + s_{30} - s_{31} \\
    h_{40} - h_{41} &= 264.15 \frac{\text{kJ}}{\text{kg}}
\end{align*}

\begin{itemize}
    \item \textbf{Table:}
    \begin{itemize}
        \item The table has two columns and five rows.
        \item The first column is labeled with numbers 1 to 4.
        \item The second column is labeled with \( p \).
        \item The values in the second column are 8, 8, 8, and 8.
        \item The last row has \( x = 0 \).
    \end{itemize}
\end{itemize}

``````latex


\[
\dot{m} = \frac{\dot{W}_u}{h_2 - h_3} = \frac{0.00 \frac{kJ}{s}}{6.894 \frac{kJ}{kg}}
\]

\[
T_2 = \infty
\]

\[
c) \quad h_{a} \rightarrow x > 0 \quad \text{und} \quad \delta_{ev} \rightarrow h_{a} = 93,42 \frac{kJ}{kg}
\]

\[
Q = \dot{m} (h_a - h_e) \rightarrow h_a - h_e
\]

\[
p = 4.2391
\]

\[
x_a = \frac{h_a - h_f}{h_g - h_f} = 0.337
\]

\[
\underline{156.28 \, W}
\]

\[
d) \quad \dot{E}_u = \frac{\dot{Q}_{zu}}{\dot{m} h_f} \rightarrow \dot{E}_u = \frac{156.28 \, W}{6.894 \frac{kJ}{kg}} = 5.581
\]

\[
\underline{156.28 \, W}
\]

\[
Q = \dot{m} (h_2 - h_1) + \dot{Q}_{zu} \rightarrow \dot{Q}_{zu} = \dot{m} (h_2 - h_1) = 156.28 \, W
\]

\[
\underline{156.28 \, W}
\]

\[
e) \quad \text{Die Temperatur würde weiter sinken und sich tiefer einpendeln, da weiter hin 28 \, W Leistung in den Kühlkreislauf gegeben wird.}
\]

```