
``````latex


\section*{A3}
\subsection*{a)}

\begin{itemize}
    \item \textbf{Diagram Description:} The diagram shows a rectangular block labeled "EW" with a membrane at the bottom. There are three forces acting on the block: $F_z$ acting upwards, $F_1$ acting downwards on the piston, and $F_3$ acting downwards on the membrane. The forces are labeled as follows:
    \begin{itemize}
        \item $F_z$ (upwards)
        \item $F_1$ (downwards on the piston)
        \item $F_3$ (downwards on the membrane)
    \end{itemize}
    The block is labeled "Kolben" (piston) and "Membran" (membrane).
\end{itemize}

\begin{itemize}
    \item $EW$-Gemisch inkompressibel $\rightarrow$ deshalb als Block betrachten
    \item $F_3 = m_{EW} \cdot g = F_g$ des $EW$
    \item $p = \frac{F}{A} \Rightarrow \left[ \frac{N}{m^2} \right] \Rightarrow F = p \cdot A$
    \item $F_1 = p_{amp} \cdot A \rightarrow A = \pi r^2 \rightarrow r = \frac{D}{2} = 5 \text{cm}$
    \item \textbf{Diagram Description:} A circle with radius $r = 0.05 \text{m}$ is shown, and the area $A = 7.854 \cdot 10^{-3} \text{m}^2$ is calculated.
    \item $F_2 = m \cdot g \rightarrow$ Schwerkraft des Gewichts auf dem Kolben
    \item $F_u = p_{0,1} \cdot A \rightarrow p_{0,1} = 1.4 \text{bar} \rightarrow F_3 = 0.981 \text{N}$
    \item $F_z = 785.398 \text{N}$
    \item $F_3 = 343.92 \text{N}$
\end{itemize}

\subsection*{b)}

\begin{itemize}
    \item \textbf{Kräfte GGW:}
    \begin{align*}
        F_z + F_1 + F_3 &= F_u \\
        F_z + F_1 + F_3 &= p_0 \cdot A \\
        p_0 &= \frac{F_z + F_1 + F_3}{A} \\
        p_0 &= 1.4005 \cdot 10^5 \text{Pa} \\
        p_{0,1} &= 1.4 \text{bar}
    \end{align*}
    \item \textbf{Ideale Gas Gleichung:} $p \cdot V = m \cdot R \cdot T$
    \begin{align*}
        m_g &= \frac{p_1 \cdot V_1}{R \cdot T_1} \rightarrow V_1 = 3.14 \text{L} = 3.14 \text{dm}^3 = 3.14 \cdot 10^{-3} \text{m}^3 \\
        T_1 &= 773.15 \text{K}
    \end{align*}
    \item $R = \frac{\bar{R}}{M_g} \frac{J}{K \cdot mol} = 8.314 \frac{J}{mol \cdot K} \cdot \frac{0.05 \text{kg}}{mol} = 166.28 \frac{J}{kg \cdot K}$
    \item $m_g = 0.0034217 \text{kg}$
\end{itemize}

``````latex


\section*{Aufgabe 3}

\subsection*{b)}

Da das EK-Gemisch inkompressibel ist, bleibt das Kräfte GGW gleich.

\[
\begin{array}{c}
\begin{array}{|c|}
\hline
\text{Fläche} \\
\hline
\end{array}
\begin{array}{|c|}
\hline
\text{Fläche} \\
\hline
\end{array}
\end{array}
\]

\[
F_u = \frac{p_1 A_1 + p_2 A_2 + p_3 A_3}{A}
\]

\[
p_{1,2} = \frac{F_1 + F_2 + F_3}{A} = p_{tot} = 1.1 \text{ bar}
\]

Allerdings ändern das Volumen und die Temperatur.

\[
p \cdot V = mRT \Rightarrow T = T_{GG}
\]

\[
T_{um} = 0^\circ \text{C}
\]

\[
p_{1} = 1.1 \text{ bar} \Rightarrow \text{aus Tab. 1}
\]

Im WW 1. HS: 

\[
m(u_2 - u_1) = Q_{12} - \dot{W}_{12}^{0}
\]

\[
u_2 - u_1 = \frac{Q_{12}}{m} \Rightarrow u_2 - u_1 = c_v (T_2 - T_1)
\]

\[
c_v (T_2 - T_1) = \frac{Q_{12}}{m}
\]

\[
T_2 - T_1 = \frac{Q_{12}}{m \cdot c_v}
\]

\[
T_2 = T_1 + \frac{Q_{12}}{m \cdot c_v}
\]

``````latex


\section*{A3}

\subsection*{c)}
$T_{2,1G} = 0.003^\circ C = 273.153K$

1. HS für geschlossenes System:

\[
\Delta E = Q_{12} - W_{v,12}
\]

\[
m(u_2 - u_1) = Q_{12} - W_{v,12}
\]

\[
Q_{12} = m(u_2 - u_1) + W_{PR}
\]

\[
u_2 - u_1 = c_v (T_2 - T_a) \rightarrow \text{perf. Gas}
\]

\[
\Delta u = -3.6 \times 1.98 \frac{kJ}{kg}
\]

\[
W_{v,12} = \int_{1}^{2} p dv = p_a (v_2 - v_1)
\]

\[
v_2 \text{ berechnen}
\]

\[
p_a v_2 = m R T_2
\]

\[
v_2 = \frac{m R T_2}{p_a}
\]

\[
v_2 = 0.001111 \frac{m^3}{kg}
\]

\[
\rightarrow W_{v,12} = (p_{c1} - p_{a1}) (v_2 - v_1)
\]

\subsection*{c)}
\[
W_{v,12} = p_a (u_2 - u_1)
\]

\[
= -28u.2 \frac{kJ}{kg}
\]

\[
W_{v,12} = -0.2842 \frac{kJ}{kg}
\]

\[
Q_{12} = m_2 (u_2 - u_1) + W_{v,12}
\]

\[
Q_{12} = -1.3624u
\]

\section*{A3}

\subsection*{d)}
\[
T_2 = T_{2,1G} = 0.003^\circ C = 273.15K
\]

1. HS:

\[
\Delta E = m(u_2 - u_1) = Q_{12} - W_{v,12}
\]

\[
u_2 - u_1 = \frac{Q_{12}}{m}
\]

\[
u_2 = u_1 + \frac{Q_{12}}{m}
\]

\[
u_2 = u_1 + Q_{12} \rightarrow Q_{12} = 1500J
\]

\[
Q_{12} = -1.5u
\]

\[
u_2 \text{ kann aus Tab.1 ermittelt werden}
\]

\[
u_2 \text{ bei } 0^\circ C = -333.05 \frac{kJ}{kg}
\]

\[
u_2 \text{ bei } 0^\circ C = -0.045 \frac{kJ}{kg}
\]

``````latex

\section*{A3}

\begin{equation*}
u_1 = (1 - x_1) u_{\text{flüssig}} + x_1 u_{\text{fest}}
\end{equation*}

\begin{equation*}
u_1 = -200.0528 \frac{\text{kJ}}{\text{kg}}
\end{equation*}

\begin{equation*}
u_2 = u_1 + \frac{Q_{12}}{m_{\text{Wasser}}} \quad \rightarrow \quad Q_{12} = 1.5 \text{kJ}
\end{equation*}

\begin{equation*}
u_2 = -145.0528 \frac{\text{kJ}}{\text{kg}}
\end{equation*}

\begin{equation*}
T_{2, \text{is}} = 0.003^\circ \text{C} \quad \rightarrow \quad u_{\text{fest}} (0.003^\circ \text{C}) = -333.442 \frac{\text{kJ}}{\text{kg}}
\end{equation*}

\begin{equation*}
u_{\text{flüssig}} (0.003^\circ \text{C}) = -0.033 \frac{\text{kJ}}{\text{kg}}
\end{equation*}

\begin{equation*}
u_2 = (1 - x_2) u_{\text{flüssig}} + x_2 u_{\text{fest}}
\end{equation*}

\begin{equation*}
u_2 - u_{\text{flüssig}} = x_2 (u_{\text{fest}} - u_{\text{flüssig}})
\end{equation*}

\begin{equation*}
x_2 = \frac{u_2 - u_{\text{flüssig}}}{u_{\text{fest}} - u_{\text{flüssig}}}
\end{equation*}

\begin{equation*}
x_2 = 0.555
\end{equation*}

``````latex


