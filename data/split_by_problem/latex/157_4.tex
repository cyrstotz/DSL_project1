
``````latex


4) 

a)

\begin{description}
    \item[Graph 1:] The graph is a phase diagram with three regions labeled "Fest", "Flüssig", and "Gas". The x-axis is labeled "T" and the y-axis is labeled "P". There are three points labeled 1, 2, and 3. Point 1 is in the "Gas" region, point 2 is in the "Flüssig" region, and point 3 is in the "Fest" region. There are several lines connecting these points, indicating phase transitions.
    \item[Graph 2:] The graph is a simplified phase diagram with the x-axis labeled "T" and the y-axis labeled "P". There are three points labeled 1, 2, and 3. Point 1 is in the "Flüssig" region, point 2 is in the "Fest" region, and point 3 is in the "Gas" region. The lines connecting these points indicate phase transitions.
\end{description}

b)

\begin{align*}
T_i &= -10^\circ C \\
T_{\text{Verdampf}} &= T_i - 6^\circ C = -16^\circ C \\
p_s &= 8 \text{ bar} = p_a \\
x_2 &= 1 \\
x_4 &= 0 \\
O &= \dot{m}_{\text{R134a}} (h_3 - h_4) + \dot{Q}_{\text{de}} \\
O &= \dot{m}_{\text{R134a}} (h_2 - h_3) - \dot{W}_k \\
\dot{m}_{\text{R134a}} &= \frac{\dot{W}_k}{h_2 - h_3} \\
h_3 &= h(8 \text{ bar}) = 264.15 \frac{\text{kJ}}{\text{kg}} \\
h_2 &= h(-16^\circ C) = 237.74 \frac{\text{kJ}}{\text{kg}} \\
&\left\{
\begin{aligned}
\dot{m}_{\text{R134a}} &= 3.8167 \frac{\text{kg}}{\text{h}}
\end{aligned}
\right.
\end{align*}

``````latex


4)

c) 
\[
p_4 = 8 \, \text{bar}
\]
Drossel ist isenthalp: \( h_1 = h_4 \)

\[
h_4 = h_f
\]

d) 
\[
\epsilon_L = \frac{\dot{Q}_{ab}}{|W_{b}|}
\]

\[
\dot{Q}_{ab} =
\]

```