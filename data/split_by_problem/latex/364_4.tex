
``````latex


\section*{Thermo I}
\subsection*{Aufgabe 4}

\subsubsection*{a)}
\textbf{isobarer Prozess}

\begin{itemize}
    \item 1-2: isobar, $T \uparrow$
    \item 2-3: $p,t$, isentrop
    \item 3-4: isobar
    \item 4-1: $p,t$, isenthalp $\Rightarrow$ isotherm
\end{itemize}

\begin{itemize}
    \item 4: $x=0$ (flüssig)
    \item 2: $x=1$ (gas)
    \item 1:
\end{itemize}

\subsubsection*{Graphical Descriptions}

\textbf{Graph 1:} A $p$-$T$ diagram with a complex, looping curve. The x-axis is labeled $T$ and the y-axis is labeled $p$. The curve starts at a point, loops around several times, and ends at a different point.

\textbf{Graph 2:} A $p$-$v$ diagram with a dome-shaped curve. The x-axis is labeled $v$ and the y-axis is labeled $p$. The curve starts at point 1, goes up to point 2, then to point 3, and finally to point 4. The following labels are present:
\begin{itemize}
    \item Between points 1 and 2: "isotherm"
    \item Between points 2 and 3: "isentrop"
    \item Between points 3 and 4: "isobar"
    \item Between points 4 and 1: "gerade (flüssig)"
    \item Near point 4: "ND"
\end{itemize}

\subsubsection*{b)}
\[
\dot{W}_u = 28 \text{W}, \quad \text{isentrop} \Rightarrow s_2 = s_3
\]

\[
\text{St. FP:} \quad 0 = \dot{m} (h_2 - h_3) + \dot{W}_u
\]

\[
\dot{W}_u = \dot{m} (h_2 - h_3)
\]

\[
\dot{m} = \frac{\dot{W}_u}{h_2 - h_3}
\]

\[
h_3 = h \left( 85 \frac{\text{kJ}}{\text{kg}} \right) = 264.15 \frac{\text{kJ}}{\text{kg}}
\]

\[
h_2 = h \left( 277.15 \text{K} \right) = 249.53 \frac{\text{kJ}}{\text{kg}}
\]

\[
\dot{m} = \frac{28 \frac{\text{J}}{\text{s}}}{\left( 249.53 - 264.15 \right) \frac{\text{kJ}}{\text{kg}}}
\]

\[
\dot{m} = 1.915 \frac{\text{g}}{\text{s}}
\]

\[
\frac{3}{5} \frac{h_3}{h_5} = \frac{3}{5}
\]

\[
T_1 = 0^\circ \text{C} + 40^\circ \text{C}
\]

\[
T_1 = 283.15 \text{K} + 40^\circ \text{C}
\]

\[
T_v = T_1 - 6 \text{K} = 277.15 \text{K} = 40^\circ \text{C}
\]

``````latex


\begin{itemize}
    \item[d)] \[
    E_u = \frac{|Q_{zu}|}{|W_{el}|} = \frac{|Q_{ab}| - |Q_{u}|}{|W_{el}|}
    \]
    \item[e)] sie würde sinken
\end{itemize}

```