
``````latex


\section*{Problem 2}

\subsection*{a)}

\[
T \, [K]
\]

\[
T_0 = 243.15
\]

\[
s \, \left[ \frac{kJ}{kg \cdot K} \right]
\]

\begin{description}
    \item[Graph Description:] 
    The graph is a Temperature-Entropy (T-s) diagram with the y-axis labeled as \( T \, [K] \) and the x-axis labeled as \( s \, \left[ \frac{kJ}{kg \cdot K} \right] \). The graph consists of six points labeled 0, 1, 2, 3, 4, 5, and 6. The points are connected by lines representing different thermodynamic processes:
    \begin{itemize}
        \item Point 0 to Point 1: A vertical line labeled "adiabat" and "isentrop".
        \item Point 1 to Point 2: A vertical line labeled "adiabat" and "isentrop".
        \item Point 2 to Point 3: A horizontal line labeled "isobar".
        \item Point 3 to Point 4: A curved line labeled "adiabat irreversibel".
        \item Point 4 to Point 5: A horizontal line labeled "isobar".
        \item Point 5 to Point 6: A vertical line labeled "adiabat, reversibel" and "isentrop".
    \end{itemize}
    The points are annotated with the following values:
    \begin{itemize}
        \item Point 0: \( T_0 = 243.15 \)
        \item Point 2: \( T_s = 1753 \)
        \item Point 5: \( 6.5 \, \text{bar} \)
        \item Point 6: \( 0.19 \, \text{bar} \)
    \end{itemize}
\end{description}

\[
\begin{array}{|c|c|c|c|}
\hline
 & P & T & \text{Notes} \\
\hline
0 & 0.6591 & 243.15 & \\
1 & \text{isotherm} & T_s = 1753 & \text{adiabat, reversibel} \\
2 & & & \text{isentrop} \\
3 & & & \text{isobar} \\
4 & & & \text{adiabat, irreversibel} \\
5 & 6.5 \, \text{bar} & (B) 3K & \text{isobar} \\
6 & 0.19 \, \text{bar} & & \text{adiabat, reversibel} \\
\hline
\end{array}
\]

``````latex

\section*{b)}

\[
w_s = 220 \frac{\text{m}}{\text{s}}, \quad p_5 = 6.56 \text{bar}, \quad T_5 = 431.5 \text{K}
\]

Schubdüse adiabatic reversibel \(\Rightarrow s_5 = s_6\)

Ideales Gas:

\[
\left( \frac{p_6}{p_5} \right)^{\frac{n-1}{n}} = \frac{T_6}{T_5} \Rightarrow T_6 = T_5 \left( \frac{p_6}{p_5} \right)^{\frac{n-1}{n}}
\]

\[
\Rightarrow T_6 = 431.9 \text{K} \cdot \left( \frac{4.56 \text{bar}}{0.5 \text{bar}} \right)^{0.4/1.4} = 328.07 \text{K}
\]

\textbf{St. FP: Um ganze Turbine}

\[
0 = \dot{m} (h_2 - h_1 + \frac{w_e^2 - w_a^2}{2}) + q - \dot{W}_t
\]

\[
0 = \dot{m} (h_2 - h_1 + \frac{w_e^2 - w_a^2}{2})
\]

\textit{(Water auf anderen Blatt)}

\section*{c)}

\[
\Delta e_{x, \text{str}} = e_{x, \text{sto}} - e_{x, \text{sk}}
\]

\[
\Rightarrow \Delta e_{x, \text{str}} = \left( h_6 - h_c - T_0 (s_6 - s_0) + t_e + p_e \right) - \left( h_0 - h_c - T_0 (s_0 - s_0) + t_e + p_e \right)
\]

\[
\Rightarrow \Delta e_{x, \text{str}} = h_6 - h_0 - T_0 (s_6 - s_0) + t_e + p_e - t_e - p_e
\]

\[
= c_p (T_6 - T_0) - T_0 \left( c_p \ln \frac{T_6}{T_0} - R \ln \frac{p_6}{p_0} \right) + \frac{w_6^2}{2} - \frac{w_0^2}{2}
\]

\[
= 1.006 \frac{\text{kJ}}{\text{kg K}} \left( 360 \text{K} - 293.15 \text{K} \right) - 293.15 \text{K} \left( 1.006 \frac{\text{kJ}}{\text{kg K}} \ln \frac{360}{293.15} - \frac{1.006}{1.4} \ln \frac{5.6}{1} \right) + \frac{510^2}{2} - \frac{200^2}{2}
\]

\[
\Delta e_{x, \text{str}} = \boxed{40.66 \frac{\text{kJ}}{\text{kg}}}
\]

``````latex


\section*{2}
\subsection*{b) water}

\[
\dot{m}(h_a + \frac{v_a^2}{2}) = \dot{m} (h_e + \frac{v_e^2}{2})
\]

\[
h_a - h_e = \frac{v_e^2 - v_a^2}{2}
\]

\[
v_{e, \text{isuff}}^2 = 2(h_a - h_e) = v_e^2
\]

\[
\Rightarrow v_e = \sqrt{v_{e, \text{isuff}}^2 - 2(h_a - h_b)}
\]

\[
v_0 = \sqrt{v_{e, \text{isuff}}^2 - 2c_p(T_0 - T_0)}
\]

\[
v_0 = \sqrt{\frac{200 - 7 \cdot v_e^2}{5} - 7 \cdot 1.000 \frac{v_e^2}{5} \cdot (325.69K - 293.15)}
\]

\[
v_0 = 195.07 \frac{m}{s}
\]

\subsection*{d)}

\[
\dot{Q}_{x, \text{verl}} = \dot{c} \cdot \frac{\dot{S}_{erz}}{\dot{m}_{ges}} \Rightarrow e_{x, \text{verl}} = \frac{T_0 \cdot S_{erz}}{\dot{m}_{ges}}
\]

\[
S_{erz} = \dot{m} \cdot (s_2 - s_1), \quad T_0 = 293.15K
\]

\[
\Rightarrow e_{x, \text{verl}} = \frac{T_0 \cdot \dot{m} \cdot (s_2 - s_1)}{\dot{m}_{ges}}
\]

\[
s_2 - s_1 =
\]

``````latex


