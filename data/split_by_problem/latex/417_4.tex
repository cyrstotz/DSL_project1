
``````latex


\section*{4.a}

\begin{description}
    \item[Graph Description:] The graph is a plot on a grid paper with the vertical axis labeled as \( p \) and the horizontal axis labeled as \( T \). The graph consists of a curve that starts at point 1, rises to point 4, dips slightly, and then rises again to point 3. There is a horizontal line connecting points 1 and 2. The curve intersects this horizontal line at point 2 and continues to rise to point 3. The points are labeled as follows:
    \begin{itemize}
        \item Point 1: At the intersection of the horizontal line and the curve.
        \item Point 2: On the horizontal line, to the right of point 1.
        \item Point 3: At the peak of the curve, to the right of point 2.
        \item Point 4: At the peak of the initial rise of the curve, above point 1.
    \end{itemize}
\end{description}

\section*{4.b}

\[
\frac{dE}{dt} = \sum_i \dot{m}_i (h_{2} - h_{1}) - \sum_i Q_i - \sum_i \dot{m}_i
\]

\[
Q = \dot{m}_i (h_{3} - h_{2}) + \dot{W}_k
\]

\[
\dot{m}_{2+3+4} = -\frac{\dot{W}_k}{h_{3} - h_{2}}
\]

\[
h_{3} \rightarrow \text{const} \Delta Z
\]

\[
h_{3} (8 \text{bar})
\]

\section*{4.c}

\[
u = u_f + x (u_g - u_f)
\]

\[
\Rightarrow x = \frac{u - u_f}{u_g - u_f}
\]

``````latex


\section*{4.d}
\[
E_{\text{n}} = \frac{| \vec{B}_{\text{n}} |}{| \vec{B}_{\text{n}} |} = \frac{| \vec{B}_{\text{n}} |}{| \vec{B}_{\text{n}} |}
\]

\begin{itemize}
    \item Sie würde steigen, da die Festigkeit nur noch gedämpft wäre und ihr ganzes Potential nicht mehr genutzt werden könnte.
\end{itemize}

```