
``````latex


\section*{Thermo Aufgabe 2}

\begin{align*}
\dot{V}_{\text{Luft}} &= 200 \frac{m}{s} \\
p_0 &= 0.15 \, \text{bar} \\
T_0 &= -30^\circ \text{C} \\
q_0 &= 1195 \frac{\text{kJ}}{\text{kg}} \\
T_B &= 1283 \text{K}
\end{align*}

\subsection*{a)}

\textbf{Graph Description:} The graph is a Temperature-Entropy (T-S) diagram. The y-axis is labeled $T(\text{K})$ and the x-axis is labeled $S \left( \frac{\text{kJ}}{\text{kg K}} \right)$. There are three isobars drawn as curved lines, labeled 0.15 bar, 0.5 bar, and an unlabeled isobar. There are six points labeled 1 through 6, connected by lines indicating processes. The points are connected as follows: 1 to 2, 2 to 3, 3 to 4, 4 to 5, and 5 to 6. The points 1 and 6 are on the 0.15 bar isobar, points 2 and 5 are on the 0.5 bar isobar, and points 3 and 4 are on the unlabeled isobar.

\subsection*{b)}

\begin{align*}
&\text{Wo?}, \, T_2? \\
&\text{da adiabatisch, isentrop:} \\
&\left( \frac{p_2}{p_1} \right)^{\frac{n-1}{n}} = \frac{T_2}{T_1} \Rightarrow T_2 = 434.9 \cdot \left( \frac{0.15 \cdot 10^5}{0.5 \cdot 10^5} \right)^{\frac{1.4-1}{1.4}} = 328.075 \text{K} \\
&\text{wobei} \quad T_2 = T_6 = 328.075 \text{K} \\
&\text{wir nehmen} \quad \frac{w^2}{2} = \text{verrichtete Arbeit} = \int vdp = RT \ln \left( \frac{p_2}{p_1} \right) = \frac{8.314}{28.97 \cdot 10^{-3}} \cdot 434.9 \text{K} \left( \frac{0.15 \text{bar}}{0.5 \text{bar}} \right) \\
&= \frac{R}{M} \cdot T \ln \left( \frac{p_2}{p_1} \right) = -115.280 \, \text{kJ} \\
&\Rightarrow \text{verrichtet} \\
&\sqrt{-115.280 \cdot (-1) \cdot 2} = w = 488.423 \frac{m}{s}
\end{align*}

``````latex


