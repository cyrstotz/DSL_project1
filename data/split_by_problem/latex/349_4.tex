
``````latex


\section*{4.a)}

\begin{itemize}
    \item The graph is a phase diagram with pressure \( p \) on the y-axis and temperature \( T \) on the x-axis.
    \item The y-axis is labeled \( p \, [\text{bar}] \) and the x-axis is labeled \( T \, [\text{K}] \).
    \item There are three regions labeled:
    \begin{itemize}
        \item "Fest" (solid) in the lower left region.
        \item "Flüssig" (liquid) in the upper region.
        \item "Gasförmig" (gaseous) in the lower right region.
    \end{itemize}
    \item The boundary lines between these regions are drawn as follows:
    \begin{itemize}
        \item A curve separating the "Fest" and "Flüssig" regions.
        \item A curve separating the "Flüssig" and "Gasförmig" regions.
        \item A curve separating the "Fest" and "Gasförmig" regions.
    \end{itemize}
    \item The point where all three curves meet is labeled "Tripel" (triple point).
    \item There are two arrows labeled "i" and "ii" indicating transitions:
    \begin{itemize}
        \item Arrow "i" starts in the "Fest" region, moves vertically upwards, and then horizontally to the right into the "Flüssig" region.
        \item Arrow "ii" starts in the "Fest" region, moves horizontally to the right, and then vertically upwards into the "Gasförmig" region.
    \end{itemize}
\end{itemize}

``````latex


4. b) stationärer Fließprozess

\[
0 = \dot{m}_{\text{misch}} (h_2 - h_3) + \dot{V}_K
\]

\[
\dot{m}_{\text{misch}} = - \frac{\dot{V}_A}{h_2 - h_3}
\]

\[
s_2 = s_3 \quad p_3 = 8 \, \text{bar} \quad x_2 = 1
\]

\[
p_2 = p_4
\]

\[
T_2 = T_i - 6 \, K
\]

``````latex


4. c) \quad x_A = \frac{\phi - \phi_{\epsilon}}{\phi_{\beta} - \phi_{\epsilon}} \quad \phi = v, a, l, s

``````latex


4. \\
a) \\
\[
\mathcal{E}_K = \frac{|\dot{Q}_{ab}|}{|\dot{W}_e|} = \frac{-\dot{Q}_{ab}}{-\dot{W}_K}
\]

```