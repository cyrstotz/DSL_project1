
``````latex


\section*{Thermo I}
\subsection*{Aufgabe 2}

\subsubsection*{c)}

\begin{itemize}
    \item \textbf{Graph 1:}
    
    The graph is a plot of \( T \) (temperature) on the y-axis versus \( S \) (entropy) on the x-axis. The plot starts at the origin (0,0) and follows a complex, wavy path. Points are marked along the path as follows:
    \begin{itemize}
        \item Point 0 at the origin.
        \item Point 1 is slightly to the right and above point 0.
        \item Point 2 is further to the right and above point 1.
        \item Point 3 is to the left and above point 2.
        \item Point 4 is to the right and below point 3.
        \item Point 5 is to the right and above point 4.
        \item Point 6 is to the right and below point 5.
    \end{itemize}
    The path between points 5 and 6 is labeled "isobar". The path between points 2 and 3 is labeled "nicht isentrop".
    
    \item \textbf{Graph 2:}
    
    The graph is a plot of \( T \) (temperature) on the y-axis versus \( S \) (entropy) on the x-axis. The plot starts at the origin (0,0) and follows a complex, wavy path. Points are marked along the path as follows:
    \begin{itemize}
        \item Point 0 at the origin.
        \item Point 1 is slightly to the right and above point 0.
        \item Point 2 is further to the right and above point 1.
        \item Point 3 is to the left and above point 2.
        \item Point 4 is to the right and below point 3.
        \item Point 5 is to the right and above point 4.
        \item Point 6 is to the right and below point 5.
    \end{itemize}
    The path between points 1 and 2 is labeled "isobar". The path between points 2 and 3 is labeled "isobar". The path between points 3 and 4 is labeled "isobar". The path between points 4 and 5 is labeled "isobar". The path between points 5 and 6 is labeled "isobar".
    
    \item \textbf{Graph 3:}
    
    The graph is a plot of \( T \) (temperature) on the y-axis versus \( S \) (entropy) on the x-axis. The plot starts at the origin (0,0) and follows a complex, wavy path. Points are marked along the path as follows:
    \begin{itemize}
        \item Point 0 at the origin.
        \item Point 1 is slightly to the right and above point 0.
        \item Point 2 is further to the right and above point 1.
        \item Point 3 is to the left and above point 2.
        \item Point 4 is to the right and below point 3.
        \item Point 5 is to the right and above point 4.
        \item Point 6 is to the right and below point 5.
    \end{itemize}
    The path between points 1 and 2 is labeled "isobar". The path between points 2 and 3 is labeled "isobar". The path between points 3 and 4 is labeled "isobar". The path between points 4 and 5 is labeled "isobar". The path between points 5 and 6 is labeled "isobar".
\end{itemize}

\begin{itemize}
    \item \textbf{Table:}
    
    \begin{tabbing}
    0: \= \( T_0 = 30^\circ \), \( p_0 = 0.95 \, \text{bar} \) \\
    0-1: \> \( p \uparrow \), \( p_1 \) \\
    1-2: \> isotherm, \( p \uparrow \), \( p_1 \rightarrow p_2 \) \\
    2-3: \> isobar, \( T \uparrow \), \( p_2 = p_3 \) \\
    3-4: \> nicht isentrop,```latex

\section*{b) $w_6$, $T_6$}

\begin{align*}
T_5 &= 431.8\,K, \quad p_5 = 0.5\,bar, \quad p_6 = 0.491\,bar \\
w_5 &= 220\,\frac{m}{s} \\
s_5 &= s_6 \quad \Rightarrow \quad s_6 - s_5 = 0 \\
\frac{T_6}{T_5} &= \left(\frac{p_6}{p_5}\right)^{\frac{n-1}{n}} \\
T_6 &= T_5 \left(\frac{p_6}{p_5}\right)^{\frac{n-1}{n}} = 431.8\,K \left(\frac{0.491}{0.5}\right)^{\frac{1.4-1}{1.4}} \\
T_6 &= 328.075\,K
\end{align*}

\section*{c)}

\begin{align*}
w_{ex,st,0} &= \dot{m} \left[ h_6 - h_0 - T_0 (s_6 - s_0) + \frac{v_0^2}{2} - \frac{v_6^2}{2} \right] \\
e_{ex,st,6} &= c_p (T_6 - T_0) - T_0 c_p \ln \left(\frac{T_6}{T_0}\right) - R \ln \left(\frac{p_6}{p_0}\right) \\
e_{ex,st,6} &= 1.006\,\frac{kJ}{kgK} \left( 328.075\,K - 243.15\,K \right) - 243.15\,K \cdot 1.006\,\frac{kJ}{kgK} \ln \left( \frac{328.075\,K}{243.15\,K} \right) \\
e_{ex,st,6} &= h_6 - h_0 - T_0 (s_6 - s_0) + \frac{v_0^2}{2} - \frac{v_6^2}{2} = 12.45\,\frac{kJ}{kg}
\end{align*}

\section*{d)}

\begin{align*}
e_{x,vel} &= T_0 s_{02} - m_{iges} \\
U &= \dot{m} \dot{v}
\end{align*}

``````latex


