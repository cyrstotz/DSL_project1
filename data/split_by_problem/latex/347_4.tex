
``````latex


\section*{Aufgabe 1}

\subsection*{a)}

\begin{itemize}
    \item The graph is a pressure-temperature ($p$-$T$) diagram.
    \item The x-axis is labeled $T$ [°C] and the y-axis is labeled $p$ [bar].
    \item The graph shows a dome-shaped curve with three distinct regions: 
        \begin{itemize}
            \item The left side of the curve is labeled "flüssig" (liquid).
            \item The right side of the curve is labeled "Dampf" (vapor).
            \item The area under the curve is labeled "Nassdampf" (wet steam).
        \end{itemize}
    \item There are three points marked on the graph:
        \begin{itemize}
            \item Point 1 is at the peak of the dome.
            \item Point 2 is on the left side of the dome.
            \item Point 3 is on the right side of the dome.
        \end{itemize}
    \item The temperature at the left intersection of the dome with the x-axis is labeled $T_1$.
    \item The temperature at the right intersection of the dome with the x-axis is labeled $T_2$.
    \item The pressure at the right end of the graph is labeled $p_D$.
\end{itemize}

\subsection*{b)}

\begin{itemize}
    \item \textbf{Kältekreisprozess:}
    \item \textbf{Zustand 1:} $p_1 = p_4 = 9.1$ bar, adiabate Entspannung
    \item \textbf{Zustand 2:} $p_2 = p_3$, $x = 1 \Rightarrow$ gerade vollständig verdampft als Gas
    \item \textbf{Zustand 3:} $p_3 = 8$ bar, $x = 3$ adiabant - reversibel
    \item \textbf{Zustand 4:} $p_4 = p_3$, $x = 0 \Rightarrow$ gerade kondensiert bei 8 bar
    \item $T_a = 11$ bei 8 bar
    \item $T_a = 3.1 - 33^\circ$C
    \item $T_1$ ist 10 K über Sublimationspunkt, adiabant reversibel und sinkt unter Tripelpunkt über 1m/s
    \item $\Delta T_1 = -10^\circ$C $\Rightarrow$ Tverdampfer $= -16^\circ$C $\Rightarrow -7^\circ$C $= T_2$
    \item Zustand 1, 2: $x = 1 \Rightarrow T_2 = -16^\circ$C $\Rightarrow T_{ab} = -10^\circ$C $p_2 = p_3 = 2.5748$ bar
    \item wir haben $p_2$ und $p_3$ und $W_k$ des Kompressors, der adiabant - reversibel ist:
    \item $\frac{W_{23}}{\dot{m}} = ?$ geht nicht, da wir keine Polytrope haben
\end{itemize}

``````latex


\section*{Student Solution}

\subsection*{c)}
\begin{align*}
\text{rechne mit} \quad \dot{m}_p &= 4 \frac{\text{kg}}{\text{h}} \quad , \quad T_2 = -22^\circ \text{C} \\
\text{ges: } x_2 &= ?
\end{align*}

\subsection*{d)}
\begin{align*}
\varepsilon_c &= \left| \frac{\dot{Q}_1}{\dot{W}_t} \right| = \frac{|\dot{Q}_1|}{\dot{W}_t} \\
\text{von E-Bilanz am isobaren Verdampfer, D:} \quad 0 &= \dot{m} (h_1 - h_2) + \dot{Q}_k - \dot{W}_k \\
\Rightarrow \varepsilon_c &= \frac{\dot{m} (h_2 - h_1)}{\dot{W}_k}
\end{align*}

\subsection*{e)}
Die Lebensmittel können nie ins Nassdampfgebiet, weshalb das Wasser bei immer weiter werdender Temperatur gefroren bleibt.

```