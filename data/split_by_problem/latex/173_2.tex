
``````latex


\section*{Aufgabe 3}

\subsection*{a) Druck $p_{g,1}$ und Masse Gas $m_g$}

\begin{equation*}
\begin{aligned}
    &\text{EV Kolben Atmos} \\
    &\downarrow \quad \downarrow \quad \downarrow \\
    &p_{g,1}
\end{aligned}
\end{equation*}

\begin{equation*}
    p_{g,1} \cdot A_{zy1} = m_{ev} \cdot g + m_{k} \cdot g + p_{atmos} \cdot A_{zy1}
\end{equation*}

\begin{equation*}
    A_{zy1} = \pi r^2 = \pi (0,05 \, \text{m})^2 = \pi \cdot 7,853982 \cdot 10^{-3} \, \text{m}^2
\end{equation*}

\begin{equation*}
    p_{g,1} = \frac{m_{ev} \cdot g + m_k \cdot g}{A_{zy1}} + p_{atmos}
\end{equation*}

\begin{equation*}
    p_{g,1} = \frac{0,1 \, \text{kg} \cdot 9,81 \, \text{m/s}^2 + 32 \cdot 9,81 \, \text{m/s}^2}{7,853982 \cdot 10^{-3} \, \text{m}^2} + 10^5 \, \text{Pa}
\end{equation*}

\begin{equation*}
    p_{g,1} = 140094,4137 \, \text{Pa}
\end{equation*}

\begin{equation*}
    pV = mRT \quad T_1 = 500^\circ \text{C} = 773,15 \, \text{K}
\end{equation*}

\begin{equation*}
    p_1 = 140000 \, \text{Pa} \quad V_1 = 3,141 \cdot 0,0037 \, \text{m}^3
\end{equation*}

\begin{equation*}
    m_g = \frac{p_1 V_1}{RT_1}
\end{equation*}

\begin{equation*}
    R = \frac{R}{M} = \frac{8314 \, \text{J/(kmol K)}}{56 \, \text{kg/kmol}} = 0,16628 \, \text{KJ/(kg K)}
\end{equation*}

\begin{equation*}
    m_g = \frac{0,0037 \cdot 140000}{0,16628 \cdot 773,15} = 0,2922 \, \text{kg}
\end{equation*}

\subsection*{b)}

\begin{equation*}
    T_{g2} \text{ ist } 0^\circ \text{C weil immer noch } Es \text{ vorhanden ist und im Gleichgewicht ist } T_{ev} = T_g
\end{equation*}

\begin{equation*}
    p_{g,2} = p_{g,1} = 140100 \, \text{Pa weil die Kraft welche der Druck } p_{g,2} \text{ entgegenwirkt}
\end{equation*}

\begin{equation*}
    \text{da immer noch derselbe ist (} p_{atmos}, m_k, m_{ev} \text{ sind noch gleich)}
\end{equation*}

``````latex


c)
\begin{align*}
    T_1 &= 773.15 \, \text{K} \\
    T_2 &= 273.15 \, \text{K} \\
    p_1 &= 101000 \, \text{Pa} \\
    p_2 &= 101000 \, \text{Pa}
\end{align*}

\begin{align*}
    \frac{dU}{dt} &= \sum \dot{m}(i) + \sum \dot{Q} - \sum \dot{w}
\end{align*}

\begin{align*}
    \Delta u &= Q - w
\end{align*}

\begin{align*}
    Q_{12} &= \Delta u + W_{12}
\end{align*}

\begin{align*}
    W_{12} &= \int_{V_1}^{V_2} p \, dV
\end{align*}

\begin{align*}
    pV &= mRT \\
    V_2 &= \frac{mRT_2}{p_2} = \frac{0.2922 \, \text{kg} \cdot 0.16628 \, \frac{\text{kJ}}{\text{kg} \cdot \text{K}} \cdot 273.15 \, \text{K}}{101000 \, \text{Pa}} = 9.474 \cdot 10^{-5} \, \text{m}^3
\end{align*}

\begin{align*}
    W_{12} &= \int_{V_1}^{V_2} p \, dV = p \int_{V_1}^{V_2} dV = p (V_2 - V_1) = 101000 \, \text{Pa} \left(9.474 \cdot 10^{-5} \, \text{m}^3 - 0.00314 \, \text{m}^3 \right)
\end{align*}

\begin{align*}
    W_{12} &= -426.64 \, \text{J}
\end{align*}

\begin{align*}
    \Delta u &= C_v (T_2 - T_1) = 0.633 \, \frac{\text{kJ}}{\text{kg} \cdot \text{K}} (273.15 \, \text{K} - 773.15 \, \text{K}) \\
    \Delta u &= -319.5 \, \text{J}
\end{align*}

\begin{align*}
    Q_{12} &= \Delta u + W_{12} = -319.5 \, \text{J} - 426.64 \, \text{J} = -746.14 \, \text{J}
\end{align*}

\begin{align*}
    W_{12} &= \int_{V_1}^{V_2} p \, dV
\end{align*}

\begin{align*}
    W_{12} &= p (V_2 - V_1) = 101000 \, \text{Pa} \left(9.474 \cdot 10^{-5} \, \text{m}^3 - 0.00314 \, \text{m}^3 \right) \\
    W_{12} &= -426.64 \, \text{J}
\end{align*}

\begin{align*}
    Q_{12} &= \Delta u + W_{12} = -746.14 \, \text{J}
\end{align*}

\underline{Graphical Description:}

There is a diagram of a piston-cylinder device. The cylinder is drawn as a rectangle with a horizontal line at the top representing the piston. Inside the cylinder, there is a shaded area labeled as $V_2$. An arrow pointing upwards next to the piston indicates the work done by the gas, labeled as $w$.

``````latex


\section*{Aufgabe 3}
\subsection*{d)}

\begin{center}
\begin{tabular}{c}
\framebox{
\begin{minipage}{0.2\textwidth}
\begin{center}
\includegraphics[width=0.8\textwidth]{house.png} \\
$Q$
\end{center}
\end{minipage}
}
\end{tabular}
\end{center}

\[
\frac{dE}{dt} = \dot{Q} + \sum Q - \sum W
\]

\[
\Delta U = Q
\]

\[
\Delta u = q
\]

\[
q = \frac{Q_{12}}{m} = \frac{1500 \, \text{J}}{0.1 \, \text{kg}} = 15000 \, \frac{\text{J}}{\text{kg}} = 15 \, \frac{\text{kJ}}{\text{kg}}
\]

\[
u_2 - u_4 = 15 \, \frac{\text{kJ}}{\text{kg}}
\]

\[
u_2 = 15 \, \frac{\text{kJ}}{\text{kg}} + u_4
\]

\[
u_1: \quad p = 1.1 \, \text{bar} \quad T_f = 0^\circ \text{C} \quad x_1 = 0.6
\]

\[
u_1 = x_1 u_{\text{Fest}} + (1 - x_1) u_{\text{Flüssig}} = 0.6 \cdot (-333.458) + 0.4 \cdot (-3.045) \quad \text{(Tab. 1)}
\]

\[
u_1 = -200.0928 \, \frac{\text{kJ}}{\text{kg}}
\]

\[
u_2 = 15 \, \frac{\text{kJ}}{\text{kg}} - 200.0928 \, \frac{\text{kJ}}{\text{kg}} = -185.0928 \, \frac{\text{kJ}}{\text{kg}}
\]

\[
X_{Eis,2}
\]

\[
X_{Eis} \cdot u_{\text{Fest}} + (1 - X_{Eis}) u_{\text{Flüssig}} = -185.0928
\]

\[
X \cdot u_{\text{Fest}} + u_{\text{Fl}} - X \cdot u_{\text{Fl}} = -185
\]

\[
X (u_{\text{Fest}} - u_{\text{Fl}}) - 185 = u_{\text{Fl}}
\]

\[
X_{Eis} = \frac{-185.0928 - u_{\text{Flüssig}}}{u_{\text{Fest}} - u_{\text{Flüssig}}} = \frac{-185.0928 + 0.045}{-333.458 + 0.045}
\]

\[
X_{Eis,2} = 0.555
\]

``````latex


