
``````latex


2. \(\odot\)

\[
T = -30^\circ C
\]
\[
p = 0.194 \, \text{bar}
\]

\[
\begin{array}{cccccc}
1 & \rightarrow & 2 & \rightarrow & 3 & \rightarrow & 4 & \rightarrow & 5 & \rightarrow & 6 \\
 & & & & & & & & 431.3 \, K \\
 & & & & & & & & 0.5 \, \text{bar} & 0.151 \, \text{bar} \\
 & & & & & & & & 270 \, \text{m/s}
\end{array}
\]

\[
\begin{array}{c}
\text{Graph:} \\
\text{The graph is a T-s diagram with the y-axis labeled as T (temperature) and the x-axis labeled as s (entropy).} \\
\text{There are two isobars and two isentropes drawn.} \\
\text{The isobars are labeled as follows:} \\
\text{1. p = 0.5 bar (upper isobar)} \\
\text{2. p = 0.151 bar (lower isobar)} \\
\text{The isentropes are labeled as follows:} \\
\text{1. Isentrop (left isentrope)} \\
\text{2. Isentrop (right isentrope)} \\
\text{Points 1 through 6 are marked on the graph, with arrows indicating the transitions between these points.} \\
\text{The transitions are as follows:} \\
\text{1 to 2 (along the lower isobar)} \\
\text{2 to 3 (along the left isentrope)} \\
\text{3 to 4 (along the upper isobar)} \\
\text{4 to 5 (along the right isentrope)} \\
\text{5 to 6 (along the lower isobar)} \\
\end{array}
\]

b) 
\[
T_6 = T_5 \left( \frac{p_6}{p_5} \right)^{\frac{\kappa - 1}{\kappa}}
\]
\[
= 328.67 \, K
\]

\[
\begin{array}{l}
\text{1. HS 5} \rightarrow \text{6:} \\
0 = \dot{Q} (h_5 - h_6 + \frac{\omega_5^2}{2} - \frac{\omega_6^2}{2}) \\
\Rightarrow \omega_6^2 = \omega_5^2 + 2 c_{p, \text{Luft}} (T_5 - T_6) = 25730 \, \frac{m}{s} \\
\omega_6 = 507.24 \, \frac{m}{s}
\end{array}
\]

\[
n = k \quad (\text{isentrop})
\]

``````latex


c)
\begin{align*}
e_{x,st,c} &= h_c - h_0 - T_c(s_6 - s_c) + \frac{\omega_c^2}{2} \\
e_{x,st,c} &= \cancel{h_c} - \cancel{h_0} - T_c(\cancel{s_6} - \cancel{s_c}) + \cancel{\frac{\omega_c^2}{2}} \\
\Delta e_{x,st} &= \cancel{h_c} - \cancel{h_0} - T_c(\cancel{s_6} - \cancel{s_c}) + \cancel{\frac{\omega_c^2}{2}} \\
R &= c_p - c_v \\
&= c_p - \frac{c_p}{\kappa} = 0.2874 \frac{\text{kJ}}{\text{kg K}} \\
e_{x,st,0} &= \frac{1}{2} \omega_0^2 \\
\Delta e_{x,st} &= h_6 - h_0 - T_c(s_6 - s_0) + \frac{\omega_6^2}{2} - \frac{\omega_0^2}{2} \\
&= c_{p,luft}(T_6 - T_0) - T_0 \left( c_{p,luft} \ln \left( \frac{T_6}{T_0} \right) - R \ln \left( \frac{p_6}{p_0} \right) \right) + \frac{\omega_6^2}{2} - \frac{\omega_0^2}{2} \\
&= 85.43 \frac{\text{kJ}}{\text{kg}} - 73.276 \frac{\text{kJ}}{\text{kg}} + 108.65 \frac{\text{kJ}}{\text{kg}} \\
&= 120.81 \frac{\text{kJ}}{\text{kg}}
\end{align*}

d)
\begin{align*}
2. HS: \quad 0 &= s_6 - s_{\alpha} + \cancel{c, adiab} + s_{\omega z} \\
&= s_{e z} = \Delta s_{0 6} = c_{p,luft} \ln \left( \frac{T_6}{T_0} \right) - R \ln \left( \frac{p_6}{p_0} \right) \\
&= 0.3014 \frac{\text{kJ}}{\text{kg K}} \\
e_{x,vel} &= T_0 s_{e z} \\
&= 73.276 \frac{\text{kJ}}{\text{kg}}
\end{align*}

\begin{description}
\item[Graph:] The graph is a plot with the x-axis labeled as $s$ and the y-axis labeled as $T$. There are two curves on the graph. The first curve starts at the origin (0,0), rises steeply, and then levels off. The second curve starts at the same point but rises more gradually and then also levels off. The first curve is labeled with a point $1$ at the start, a point $2$ at the peak, and a point $3$ where it levels off. The second curve is not labeled.
\end{description}

``````latex


