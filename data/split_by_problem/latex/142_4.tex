
``````latex


\section*{Student Solution}

\begin{itemize}
    \item The graph is drawn on a grid paper.
    \item The x-axis is labeled as \( T \, [^\circ C] \) and the y-axis is labeled as \( \rho \, [\text{kg/m}^3] \).
    \item The x-axis has a point marked at \( 0 \) and another point marked at \( X_2 \).
    \item The y-axis has points marked at \( 0.7 \), \( 0.1 \), \( 1 \), and \( 10 \).
    \item There is a curve starting from the origin (0,0) and rising upwards in a concave manner, passing through the points marked on the y-axis.
    \item The curve is labeled as "flüssig".
    \item There is a horizontal line at the top right of the graph, labeled as "gas".
    \item There are two vertical lines intersecting the horizontal line at the top right, labeled as "1" and "2".
    \item There is a point marked on the horizontal line between the two vertical lines, labeled as "3".
    \item The point "3" is connected to the curve with a dashed line.
    \item The point "3" is also connected to the y-axis with a dashed line, labeled as \( \rho = 0.8 \).
    \item The point "3" is connected to the x-axis with a dashed line, labeled as \( p_3 = 8 \text{bar} \).
\end{itemize}

```