
``````latex


\section*{4}

\subsection*{a)}

\begin{description}
    \item[Graph Description:] The graph is a Pressure-Temperature ($p$-$T$) diagram. The x-axis is labeled $T [K]$ and the y-axis is labeled $p [bar]$. The graph shows three distinct regions: solid, liquid, and gas. The solid region is on the left, the liquid region is in the middle, and the gas region is on the right. The boundaries between these regions are marked by lines. The line separating the solid and liquid regions is labeled "schmelzen" (melting), the line separating the liquid and gas regions is labeled "verdampfen" (evaporation), and the line separating the solid and gas regions is labeled "sublimieren" (sublimation). There is a point labeled "ND" (triple point) where all three regions meet. The line above the liquid region is labeled "isobar" (constant pressure), and the line below the liquid region is labeled "isotherm" (constant temperature).
\end{description}

\subsection*{b)}

\[
q_1 = h_{g}(8 \, bar) - h_f(8 \, bar) = 0.2455 \, \frac{kJ}{kg} \quad \text{Tabelle A.11}
\]

\[
h_4 = h_f(8 \, bar) = 53.42 \, \frac{kJ}{kg} \quad \text{Tabelle A.11}
\]

\[
4 \rightarrow 1 \quad \text{isenthalp} \rightarrow h_4 = h_c = 53.42 \, \frac{kJ}{kg}
\]

\subsection*{c)}

\[
\dot{m}_{R134a} = \frac{6 \, kJ}{h} \quad T_2 = -22^\circ C
\]

\[
h_4 = h_2 \rightarrow h_2 = h_f(8 \, bar) = 53.42 \, \frac{kJ}{kg} = h_1
\]

\subsection*{d)}

```