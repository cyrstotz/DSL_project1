
``````latex


\section*{A7}

\subsection*{a)}

\begin{description}
    \item[Graph Description:] The graph is a plot with the x-axis labeled as $s$ and the y-axis labeled as $T$. There are several lines and curves drawn on the graph. The origin is marked as $0$. There are six points labeled from $0$ to $6$ along the curves. The points are connected by lines, and there are annotations such as $p_2 = p_3$, $p_4 < p_3$, $p_4 = p_6$, $0.5$, $4.3$, $1.9K$, and $0.1971$. There are also labels such as $adiab.$, $isobar$, and $isochor$ along the curves. The temperature $T$ is marked as $-30^\circ$ at the bottom left of the graph.
\end{description}

\subsection*{b)}

\begin{equation*}
\frac{dE}{dt} = 0 = \dot{m} (h_0 - h_6) + \left( \frac{w_0^2 - w_6^2}{2} \right) + \frac{q_{13}}{\dot{m}_K} + \dot{W}
\end{equation*}

\begin{equation*}
0 = \dot{m} c_p (T_0 - T_6) + \left( \frac{w_0^2 - w_6^2}{2} \right) = - \frac{q_{13}}{\dot{m}_K}
\end{equation*}

\begin{equation*}
5.273 \cdot \dot{m}_K = \dot{m} = \dot{m} - \dot{m}_K
\end{equation*}

\begin{equation*}
\dot{m}_K = \frac{\dot{m}}{6.273}
\end{equation*}

\subsection*{c)}

\begin{description}
    \item[Graph Description:] The graph is a plot with the x-axis labeled as $s$ and the y-axis labeled as $T$. There are several curves drawn on the graph. The origin is marked as $0$. There are six points labeled from $0$ to $6$ along the curves. The points are connected by lines, and there are annotations such as $0.5 \, \text{bar}$ and $0.1971 \, \text{m}$. The points $2$, $3$, $4$, and $5$ are connected by arrows indicating transitions between these points. The labels $m_K$ and $m_n$ are also present along the curves.
\end{description}

``````latex


\section*{A2}

\subsection*{b)}

\[
\frac{dE}{dt} = 0 = \dot{m} \left( h_0 - h_6 + \frac{w_0^2 - w_6^2}{2} \right) + q_{B} - \dot{W}
\]

\[
0 = h_0 - h_6 + \frac{w_0^2 - w_6^2}{2} + q_{B}
\]

\[
w_6 = \sqrt{2 \left( h_0 - h_6 + q_{B} \right) + w_0^2}
\]

\[
= \sqrt{2 \left[ c_p (T_0 - T_6) + q_{B} \right] + w_0^2}
\]

\[
q_{B} = 1195 \frac{kJ}{kg}
\]

\[
T_0 = 243.15 \, K
\]

\[
T_6 = \ldots
\]

\[
w_0 = 200 \frac{m}{s}
\]

\subsection*{c)}

\[
\Delta e_{x,str} = e_{x,str,6} - e_{x,str,0} = h_6 - h_0 - T_0 (s_6 - s_0) + \frac{w_6^2 - w_0^2}{2}
\]

\[
= c_p \left[ T_0 - T_6 - T_0 \ln \left( \frac{T_6}{T_0} \right) \right] + \frac{w_6^2 - w_0^2}{2}
\]

\[
= c_p \left[ 340K - 243.15K - 243.15K \ln \left( \frac{340}{243.15} \right) \right] + \frac{340^2 - 200^2}{2}
\]

\[
= 25.5 \frac{kJ}{kg}
\]

\subsection*{Graphical Content Description}

There is a diagram of a nozzle. The nozzle is depicted as a converging-diverging shape with an inlet on the left side labeled as $r$ and an outlet on the right side labeled as $b$. The flow direction is indicated by an arrow pointing from left to right.

Next to the nozzle diagram, there are several equations and expressions:

\[
0 = \dot{S}_{irr} \neq \dot{m} (s_6 - s_0) + \frac{\dot{Q}}{T_0}, ad
\]

\[
s_6 - s_0 = c_p \ln \left( \frac{T_6}{T_0} \right) - R \ln \left( \frac{p_6}{p_0} \right) = 0
\]

\[
\frac{T_6}{T_0} = \left( \frac{p_6}{p_0} \right)^{\frac{\kappa - 1}{\kappa}}
\]

\[
T_6 = T_0 \left( \frac{p_6}{p_0} \right)^{\frac{\kappa - 1}{\kappa}} \times 328.1 \, K
\]

\[
\Rightarrow w_6 = 20
\]

``````latex

\section*{A2 Ff.}

\subsection*{d)}

\begin{align*}
c_{\text{kurvel}} &= c_{\text{aster}} + t \cdot e \cdot Q \\
&= T_0 \cdot \text{sin} z \\
&= \\
&= -\Delta c_{\text{aster}} + \left( 1 - \frac{T_0}{9} \right) q \\
&= -100 \frac{kJ}{kg} + \left( 1 - \frac{243.15 K}{278 K} \right) 11495 \\
&= 869.16 \frac{kJ}{kg}
\end{align*}

``````latex


