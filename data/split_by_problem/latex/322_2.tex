
``````latex


\section*{Aufgabe 2}

\subsection*{a)}

\begin{description}
    \item[Graph Description:] The graph is a plot with the x-axis labeled \( s \left[ \frac{kJ}{kg} \right] \) and the y-axis labeled \( T [K] \). There are three curves on the graph:
    \begin{itemize}
        \item The first curve starts at the origin, rises steeply, peaks, and then falls back down, resembling a bell curve. This curve is labeled as "Isobare".
        \item The second curve is a straight line with a positive slope, labeled as "Isotherm".
        \item The third curve is another straight line with a positive slope, but steeper than the second curve, labeled as "Isotherm".
    \end{itemize}
\end{description}

\subsection*{b)}

\begin{align*}
    w_0 \quad \text{und} \quad T_6 \quad \text{am Austritt} \\
    \rightarrow \text{Energie Gleich für stationären Fließprozess} \\
    0 &= h \left[ h_5 - h_6 - \frac{w_2^2 - w_1^2}{2} \right] + \sum \frac{\dot{Q}_j}{\dot{m}} - \sum \frac{\dot{W}_k}{\dot{m}} \\
    \rightarrow 2 (h_6 - h_5) + w_5^2 - w_6^2 \\
    w_6 &= \sqrt{2 \cdot (h_5 - h_6) + w_5^2} \\
    h_6 - h_5 &= \int_{T_5}^{T_6} c_p dT = c_p (T_6 - T_5) \\
    \rightarrow T_5 - T_6 \quad \text{adiabat - reversibel} \rightarrow \text{isentrop} \\
    \rightarrow \text{polytrope Temperaturverhältnis mit} \quad n = k = 1.4 \\
    \frac{T_6}{T_5} &= \left( \frac{p_6}{p_5} \right)^{\frac{n-1}{n}} \rightarrow T_6 = T_5 \left( \frac{p_6}{p_5} \right)^{\frac{n-1}{n}} \\
    \rightarrow T_6 &= 431.9 \, K \quad \left( 0.193 \frac{bar}{0.1 \, bar} \right)^{\frac{0.4}{1.4}} \\
    &= 326.07 \, K \\
    \rightarrow h_6 - h_5 &= c_p (T_6 - T_5) \\
    &= 1.006 \frac{kJ}{kgK} (326.07 K - 431.9 K) \\
    &= -106.45 \frac{kJ}{kg} \\
    \rightarrow w_6 &= \sqrt{2 (h_5 - h_6) + w_5^2} \\
    &= \sqrt{2 \cdot 106.45 \frac{kJ}{kg}} \\
    &= 243.54 \frac{m}{s}
\end{align*}

\subsection*{c)}

\begin{align*}
    \Delta s_{ges} &= s_{ex, ström} - s_{ex, ström, 0} \\
    s_{ex, ström} &= (h_6 - h_6 - T_0 (s_6 - s_0))
\end{align*}

``````latex


\section*{Solution}

\subsection*{a)}

\begin{description}
    \item[Graph Description:] The graph is a Temperature-Entropy (T-S) diagram. The vertical axis is labeled as $T$ (Temperature) and the horizontal axis is labeled as $S$ (Entropy). There are three isobaric lines (constant pressure) drawn diagonally from the bottom left to the top right. The lines are labeled as "Isobar". There are three points labeled 1, 2, and 3. Point 1 is at the intersection of the lowest isobar and the vertical axis. Point 2 is on the highest isobar. Point 3 is on the middle isobar. There is a vertical line connecting points 1 and 3, and a horizontal line connecting points 3 and 2. There is also a curved line connecting points 2 and 1. The region between the vertical and horizontal lines is labeled as "Verflüssigen" (liquefaction), and the region between the horizontal and curved lines is labeled as "Verdampfen" (evaporation).
\end{description}

\[
\begin{array}{c}
\text{Verflüssigen} \quad \text{sind} \quad \text{adiab. reversib.} \\
\text{isotherm} \quad \text{isotherm} \quad \text{isotherm} \quad \text{isotherm} \quad \text{isotherm} \quad \text{isotherm} \quad \text{isotherm} \quad \text{isotherm} \quad \text{isotherm} \quad \text{isotherm} \quad \text{isotherm} \quad \text{isotherm} \quad \text{isotherm} \quad \text{isotherm} \quad \text{isotherm} \quad \text{isotherm} \quad \text{isotherm} \quad \text{isotherm} \quad \text{isotherm} \quad \text{isotherm} \quad \text{isotherm} \quad \text{isotherm} \quad \text{isotherm} \quad \text{isotherm} \quad \text{isotherm} \quad \text{isotherm} \quad \text{isotherm} \quad \text{isotherm} \quad \text{isotherm} \quad \text{isotherm} \quad \text{isotherm} \quad \text{isotherm} \quad \text{isotherm} \quad \text{isotherm} \quad \text{isotherm} \quad \text{isotherm} \quad \text{isotherm} \quad \text{isotherm} \quad \text{isotherm} \quad \text{isotherm} \quad \text{isotherm} \quad \text{isotherm} \quad \text{isotherm} \quad \text{isotherm} \quad \text{isotherm} \quad \text{isotherm} \quad \text{isotherm} \quad \text{isotherm} \quad \text{isotherm} \quad \text{isotherm} \quad \text{isotherm} \quad \text{isotherm} \quad \text{isotherm} \quad \text{isotherm} \quad \text{isotherm} \quad \text{isotherm} \quad \text{isotherm} \quad \text{isotherm} \quad \text{isotherm} \quad \text{isotherm} \quad \text{isotherm} \quad \text{isotherm} \quad \text{isotherm} \quad \text{isotherm} \quad \text{isotherm} \quad \text{isotherm} \quad \text{isotherm} \quad \text{isotherm} \quad \text{isotherm} \quad \text{isotherm} \quad \text{isotherm} \quad \text{isotherm} \quad \text{isotherm} \quad \text{isotherm} \quad \text{isotherm} \quad \text{isotherm} \quad \text{isotherm} \quad \text```latex


