
``````latex


\section*{Aufgabe 4}

\subsection*{a)}

\begin{center}
\textbf{Graph Description:}

The graph is a plot with the vertical axis labeled as $\rho$ and the horizontal axis labeled as $T$. There is a curve starting from the bottom left and curving upwards to the right. Three points are marked on the graph:

- Point 1 is located near the bottom left of the graph.
- Point 2 is located higher up and to the right of Point 1.
- Point 3 is located directly above Point 2.

Arrows indicate the transitions between these points:
- An arrow points from Point 1 to Point 2.
- Another arrow points from Point 2 to Point 3.

The points are labeled as follows:
- Point 1: "1"
- Point 2: "2,4"
- Point 3: "3"
\end{center}

\subsection*{b)}

1. HS am Verdichter
\[
\dot{W}_K = \dot{m}_{R134a} \cdot (h_2 - h_3)
\]

\[
h_2 = h \left( x = 1, T = T_i - 6 \right)_{\text{Tab A-10}} = 237.79 \frac{\text{kJ}}{\text{kg}}
\]

\[
h_3 = h \left( s = s_2, 8 \text{bar} \right) = \text{adiabat, rev.}
\]

\[
\Rightarrow \dot{m}_{R134a} = \frac{\dot{W}_K}{h_2 - h_3}
\]

\begin{flushright}
\begin{tabular}{rl}
$T_i$ & $= -10^\circ C$ \\
$s_2$ & $= 0.9298 \frac{\text{kJ}}{\text{kg K}}_{\text{Tab A-10}}$ \\
\end{tabular}
\end{flushright}

``````latex


\section*{Aufgabe 4, Seite 2}

\subsection*{c)}
Adiabate Drossel: \( h_4 = h_1 \) aus 1. HS.

\[
x_4 = 0 \quad , \quad p_4 = 8 \text{ bar} \quad (\text{isochor 3} \rightarrow 4)
\]

\[
\text{Tab A-11} \implies h_4 = 93,42 \frac{\text{kJ}}{\text{kg}} = h_1
\]

\[
p_1 = p_2 \quad \text{(isobare)}
\]

\[
x_2 = 1 \quad , \quad T_2 = -22^\circ \text{C} \implies p_2 = 1,21926 \text{ bar} = p_1
\]

\[
x_A = \frac{h_1 - h_f}{h_g - h_f}
\]

\[
= 0,337
\]

\subsection*{d)}
\[
\varepsilon_K = \frac{\dot{Q}_{zu}}{|\dot{W}_e|}
\]

\[
\dot{Q}_{zu} = \dot{Q}_K = \dot{m}_{13 \text{ in}} \cdot (h_2 - h_1)
\]

\[
= 0,156 \text{ kW}
\]

\[
\implies \varepsilon_K = 5,58 \cdot 10^{-3}
\]

\subsection*{e)}
Temperatur würde wachsen.

```