
``````latex


\section*{4(a)}

\[ x_2 = 1 \quad x_c = 0 \quad T_k = T_i - 6K \quad T_f = T_{TOP} \]
\[ T_i = T_{TOP} \text{ unter Sublimationsdruck} \]
\[ T_i @ 100 \text{mbar} \]
\[ T_i @ 5 \text{mbar} \]
\[ T_{Sied} @ 5 \text{mbar} = -1^\circ C \]
\[ T_i = 9^\circ C = 282.15K \]

\subsection*{Graph Description}

The graph is a pressure-temperature ($P$-$T$) phase diagram. The vertical axis is labeled $P$ [mbar] and the horizontal axis is labeled $T$ [°C]. 

- The graph shows three distinct regions labeled "fest" (solid), "flüssig" (liquid), and "gas" (gas).
- The "fest" region is at the top left, the "flüssig" region is at the top right, and the "gas" region is at the bottom.
- There is a curve separating the "fest" and "gas" regions, which starts from the top left and curves downwards to the right.
- The point where the "fest", "flüssig", and "gas" regions meet is labeled "Tripel" (triple point) at $0^\circ C$.
- A horizontal dashed line at approximately 5 mbar extends from the "Tripel" point to the right, indicating the sublimation pressure.
- The temperature at the triple point is marked as $T_{Tripel} = 0^\circ C$.
- The graph also shows a horizontal line at $T_i @ 5 \text{mbar}$, indicating the initial temperature at 5 mbar.
- Another horizontal line is drawn at $T_i = 9^\circ C$.

\section*{4(b)}

\[
\text{(b) siehe}
\]

``````latex


\section*{4(b)}

\text{MiR134a adiabatic}

\[
2 \rightarrow 3 \text{ isotrop } \quad W_k = 28W \quad P_3 = 8 \text{bar} \quad x_3 = 1 \quad P_2 = P_1
\]

\[
0 = \dot{m} \left[ h_2 - h_3 + \frac{c}{k} + \frac{pe}{k} \right] + \dot{Q} - W_k \quad \text{adiabatic}
\]

\[
W_k = \dot{m} \left[ h_2 - h_3 \right] \quad \rightarrow \quad \dot{m} \text{R134a} = \frac{W_k}{h_2 - h_3}
\]

\[
S_2 = S_3
\]

\[
h_2 @ x_2 = 1 \quad T_2 = 9^\circ C
\]

\[
h_2 = \frac{254,03 - 251,80}{12 - 8} (9 - 8) + 251,80 = 252,358 \frac{kJ}{kg} \quad \Delta 10
\]

\[
S_3 = S_2 = 0,9 \frac{232 - 99,150}{12 - 8} (9 - 8) + 99,150 = 99,1455 \frac{kJ}{kgK}
\]

\[
@ 8 \text{bar}, 5,53 \quad \Delta 12
\]

\[
h_3 = 273,66 - 266,15 (0,9 (1455,8 - 9906) + 266,15
\]

\[
0,9374 - 0,9065
\]

\[
h_3 = 266,605 \frac{kJ}{kg}
\]

\[
\dot{m} \text{R134a} = \frac{-1 W_k}{h_2 - h_3} = \frac{4,965 \frac{kg}{s}}{1,965 \frac{g}{s}}
\]

\section*{c)}

\[
h_1 = h_4 \quad x_4 = 0 \quad P_4 = 8 \text{bar}
\]

\[
h_2 = h_4 = 93,62 \frac{kJ}{kg} \quad \Delta 11
\]

\[
x_1 = ??? \quad h_1 = h_f + x_1 (h_g - h_f)
\]

\[
\rightarrow \quad \frac{h_1 - h_f}{h_g - h_f} = x_1 = \frac{46}{0,1647}
\]

\[
h_g @ 9^\circ C \quad \Delta 10
\]

\[
h_g = \frac{254,03 - 251,80}{12 - 8} (9 - 8) + 251,80 = 252,34 \frac{kJ}{kg}
\]

\[
h_f @ 9^\circ C \quad \Delta 10
\]

\[
h_f = \frac{66,18 - 60,73}{12 - 8} (9 - 8) + 60,73 = 62,093 \frac{kJ}{kg}
\]

\section*{d)}

\[
\boxed{3}
\]

``````latex

4) \\
d) \\
\[
E_K = \frac{|Q_{\text{zu}}|}{|W_{\text{el}}|}
\]

```