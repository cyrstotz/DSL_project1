
``````latex


\section*{Aufg 9:}

\subsection*{a)}

\begin{description}
    \item[Graph Description:] The graph is a Pressure-Temperature (P-T) diagram. The y-axis is labeled with "P" and the x-axis is labeled with "T". The graph contains a closed loop with four points labeled 1, 2, 3, and 4. The path from point 1 to point 2 is labeled "isobar", the path from point 2 to point 3 is labeled "isentrop", the path from point 3 to point 4 is labeled "isobar", and the path from point 4 to point 1 is labeled "isenthalp". The pressure at points 1 and 4 is labeled as "3 bar" and the pressure at points 2 and 3 is labeled as "8 bar".
\end{description}

\subsection*{b)}

\begin{align*}
    \dot{m} \, \text{R134a} \\
    1HS \, \text{verdichter:} \\
    0 &= \dot{m} \left[ h_2 - h_3 \right] + \dot{W}_k \\
    \text{adiabat und rev:} \, s_2 = s_3 \\
    h_2 &= \\
    h_3 &= \\
    T_2 &= T_i + 6h = 9h + 7su6 \\
    T_i &= 10k + T_smb
\end{align*}

``````latex

c) \quad h_1 = h_a

\vspace{2cm}

d) \quad \frac{\dot{Q}_7 - 1}{|W| + 1} = \frac{\dot{Q}_m}{28 \omega}

\vspace{2cm}

e)

```