
``````latex


\section*{Aufgabe 4:}

\subsection*{a)}

\begin{itemize}
    \item The first graph is a $P$ vs $T$ diagram.
    \item The y-axis is labeled $P$ and the x-axis is labeled $T$.
    \item There are three regions labeled "Eis", "Wasser", and "Dampf".
    \item The boundary between "Eis" and "Wasser" is a line starting from the origin and sloping upwards.
    \item The boundary between "Wasser" and "Dampf" is a line starting from the origin and sloping upwards more steeply.
    \item The boundary between "Eis" and "Dampf" is a line starting from the origin and sloping upwards even more steeply.
    \item There is a point labeled "Tripelpunkt" where the three lines intersect.
    \item There is a rectangular region labeled "1", "2", "3", and "4" with the top side labeled "lil" and the bottom side labeled "u".
\end{itemize}

\begin{itemize}
    \item The second graph is a $P$ vs $T$ diagram.
    \item The y-axis is labeled $P$ and the x-axis is labeled $T$.
    \item There are three regions labeled "Eis", "Wasser", and "Dampf".
    \item The boundary between "Eis" and "Wasser" is a line starting from the origin and sloping upwards.
    \item The boundary between "Wasser" and "Dampf" is a line starting from the origin and sloping upwards more steeply.
    \item The boundary between "Eis" and "Dampf" is a line starting from the origin and sloping upwards even more steeply.
    \item There is a point labeled "1" on the boundary between "Eis" and "Wasser".
    \item There is a point labeled "2" on the boundary between "Wasser" and "Dampf".
    \item There is a point labeled "3" on the boundary between "Eis" and "Dampf".
    \item There is a point labeled "4" in the "Wasser" region.
\end{itemize}

\subsection*{b)}

\begin{align*}
    T_i &= -10^\circ C \\
    \overline{T_v} &= -16 K
\end{align*}

```