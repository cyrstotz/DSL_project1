
``````latex


\section*{2)}

\begin{itemize}
    \item A graph is drawn with the y-axis labeled $T[K]$ and the x-axis labeled $s$. The graph shows a curve starting from the origin and moving upwards to the right. There are several points marked on the graph:
    \begin{itemize}
        \item Point 0 at the bottom left.
        \item Point 1 slightly above and to the right of point 0.
        \item Point 2 further up and to the right.
        \item Point 3 even further up and to the right.
        \item Point 4 further up and to the right.
        \item Point 5 at the top right.
        \item Point 6 directly below point 5.
    \end{itemize}
    \item The graph has several annotations:
    \begin{itemize}
        \item $P_5 = P_4$ near point 5.
        \item $0.5 \text{ bar}$ near point 5.
        \item $P_2 = P_3$ near point 2.
        \item $P_6 = P_4$ near point 6.
        \item $0.1916 \text{ bar}$ near point 6.
    \end{itemize}
\end{itemize}

\begin{tabular}{|c|c|c|}
    \hline
    0 & 0.1916 \text{ bar} & -30^\circ C \\
    \hline
    1 & & \\
    \hline
    2 & & \\
    \hline
    3 & & \\
    \hline
    4 & 0.5 \text{ bar} & \\
    \hline
    5 & 0.5 \text{ bar} & 431.9 \text{ K} \\
    \hline
    6 & 0.1916 \text{ bar} & 328.07 \text{ K} \\
    \hline
\end{tabular}

\subsection*{b)}

$5 \rightarrow 6 \quad \text{isentrop}$

\[
T_6 = T_5 \left( \frac{P_6}{P_5} \right)^{\frac{n-1}{n}}
\]

\[
= 328.07 \text{ K} = T_6
\]

\subsection*{Energiebilanz}

\[
0 = \dot{m}_{\text{ges}} (h_5 - h_6) + \dot{m}_1 \left( \frac{w_5^2 - w_6^2}{2} \right) + \dot{Q}^0 - \dot{W}^0
\]

\[
2 (h_6 - h_5) = w_5^2 - w_6^2 \rightarrow w_6^2 - w_5^2 = 2 (h_6 - h_5)
\]

``````latex

\begin{align*}
&\omega_G^2 - \omega_0^2 = 2(h_G - h_0) \\
&0 = mg(h_5 - h_G) + \dot{m}g \left( \frac{\omega_5^2 - \omega_G^2}{2} \right) + \cancel{Q} - \cancel{W} \\
&\omega_G^2 = 2(h_5 - h_G) + \omega_5^2 \\
&= 2c_p(T_5 - T_6) + \omega_5^2
\end{align*}

\text{Diagram: A diagram with three arrows labeled as follows:}
\begin{itemize}
    \item The first arrow is labeled $Q$.
    \item The second arrow is labeled $W$.
    \item The third arrow is labeled $Q - W$.
\end{itemize}

\text{Additionally, there is a note:}
\begin{itemize}
    \item \text{isobar:} \quad \frac{T}{V} = \text{const}
\end{itemize}

\section*{c)}
\begin{align*}
&\Delta ex_{str,G} = (h_G - h_0 - T_0(s_G - s_0) + \frac{\omega_G^2}{2}) \\
&ex_{str,0} = 0 \\
&\text{Ideal Gas:} \quad ex_G = (c_p(T_G - T_0) - T_0 c_p \left( \frac{T_G}{T_0} \right) + R \ln \left( \frac{P_G}{P_0} \right)) \\
&ex_{str,G} = h_1 - h_0 - T_0(s_1 - s_0) + \frac{\omega_G^2}{2} = \frac{\omega_{u,ft}^2}{2} = 
\end{align*}

\section*{d)}
\begin{align*}
Q &= \sum ex_{str} + \sum ex_q - \sum W - \frac{av}{P_0 a_0} - ex_{verl} \\
0 &= \dot{m}(s_0 - s_a) + \frac{\dot{Q}}{T} + S_{elvz} \\
S_{elvz} &= S_6 - S_0 \quad \text{da ideales gas} \\
&= S^0(T_6) - S^0(T_0) - R \ln \left( \frac{P_6}{P_0} \right) \\
&\quad \text{, da } P_6 = P_0 \\
\dot{e}_{ex,verl} &= T_0 \cdot S_{elvz}
\end{align*}

``````latex

2. d)

\[
s(T_1) = s^0(T_1)
\]

\[
\begin{aligned}
s^0(1240 \, \text{K}) & \quad \text{interpolieren auf} \quad s^0(1243.15 \, \text{K}) = 243.17 \, \frac{\text{kJ}}{\text{kg} \cdot \text{K}} \\
s^0(1250 \, \text{K}) & \\
\\
s^0(325 \, \text{K}) & \quad \text{interpolieren auf} \quad s^0(328.71 \, \text{K}) = 328.39 \, \frac{\text{kJ}}{\text{kg} \cdot \text{K}} \\
s^0(330 \, \text{K}) & \\
\end{aligned}
\]

\[
\dot{s}_{erz} = 85.22 \, \frac{\text{kJ}}{\text{kg} \cdot \text{K}}
\]

\[
\Rightarrow \, ex_{ver1} = T_0 \cdot \dot{s}_{erz} = 20.723 \, \frac{\text{MJ}}{\text{kg}}
\]

``````latex


