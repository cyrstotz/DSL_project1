
``````latex


\section*{A4}

\subsection*{a)}

\begin{itemize}
    \item 1 $\rightarrow$ 2: isobar
    \item 2 $\rightarrow$ 3: adiabatic rev $\rightarrow$ isentrop
    \item 3 $\rightarrow$ 4: isobar
    \item 4 $\rightarrow$ 1: isentrope
\end{itemize}

\subsection*{b)}

\begin{itemize}
    \item GSI: $\dot{m}_{1234}$
\end{itemize}

\[
0 = \dot{m}_{1234} (h_3 - h_2) + \dot{Q}_{ab}
\]

\[
0 = \dot{m}_{1234} (h_2 - h_3) - \dot{W}_K \Rightarrow \dot{m}_{1234} = \frac{\dot{W}_K}{h_2 - h_3}
\]

\[
\Rightarrow h_2 = h_g, \quad s_2 = s_3
\]

\[
h_3 (8 \text{bar}, T_3)
\]

\subsection*{Graph Description}

The graph is a pressure-volume (p-V) diagram with the following details:

- The x-axis is labeled as $T [K]$.
- The y-axis is labeled as $p [\text{bar}]$.
- There are four points labeled 1, 2, 3, and 4.
- The process from point 1 to point 2 is a horizontal line indicating an isobaric process.
- The process from point 2 to point 3 is a curved line indicating an adiabatic reversible process leading to an isentropic process.
- The process from point 3 to point 4 is another horizontal line indicating an isobaric process.
- The process from point 4 to point 1 is a curved line indicating an isentropic process.

``````latex

\begin{itemize}
    \item[(c)] 
    \[
    \dot{m}_{R134A} = \frac{4t\theta}{h}, \quad T_2 = -22^\circ C
    \]
    \[
    G5: x_1
    \]
    \[
    S_4 = S_1
    \]
    \[
    S_4 = S_f
    \]
    \[
    x_1 = \frac{S_3 - S_f}{S_g - S_f}
    \]
\end{itemize}

```