
``````latex


\section*{4) a)}

\begin{itemize}
    \item A graph is drawn with the y-axis labeled as \( (P) \) in bar and the x-axis labeled as \( T_i \) in Kelvin.
    \item The graph has a line starting from the origin and going upwards to the right, labeled as "Dampf".
    \item There are two horizontal lines intersecting the main line, labeled as "Nassdampf".
    \item The intersection points are labeled as 1 and 2.
    \item A vertical line is drawn from point 2 downwards, intersecting the x-axis at point 3.
    \item The temperature \( T_i = 10^\circ C \) is noted on the right side of the graph.
\end{itemize}

\section*{b)}

\begin{itemize}
    \item \( 1 \, \text{HS am Verdichter} \)
    \item \( 0 = \dot{m} (h_1 - h_2) + \dot{Q} - W \)
    \item \( 2: x_2 = 0 \, \text{bei} \, 8 \, \text{bar} \)
    \item \( \text{TAB A11} \, h_2 = 93,42 \, \frac{\text{kJ}}{\text{kg}} \)
    \item \( \text{Zustand 2} = 4 \, \text{K} \)
\end{itemize}

\section*{c)}

\begin{itemize}
    \item \( x_1 \, T_2 = -22^\circ C \, x_1 = 4 \, \frac{\text{kJ}}{\text{kg}} \)
    \item \( \text{Verdampfer} = \text{isobar} \)
    \item \( 0 = \dot{m} (h_2 - h_3) \)
    \item \( 1 \, \text{HS an der Drossel} \)
    \item \( 0 = \dot{m} (h_4 - h_3) + \dot{Q} - \dot{Q} \)
    \item \( h_3 = 81 \, \frac{\text{kJ}}{\text{kg}} \, x = 0 \)
    \item \( \text{TAB A11} \, h_2 = 93,42 \, \frac{\text{kJ}}{\text{kg}} \)
    \item \( h_2 = h_4 \, \text{da isenthalp} \)
    \item \( h_2 = 93,42 \, \frac{\text{kJ}}{\text{kg}} \)
    \item \( \text{interpolieren bei Ausgangsdruck mit} \, h_1 \)
\end{itemize}

\section*{d)}

\begin{itemize}
    \item \( E_k = \frac{1 \, \text{kJ}}{1} = \frac{1 \, \text{kJ}}{28 \, \text{W}} \)
    \item \( h_2 = h_4 \, \text{da isenthalp} \)
    \item \( h_2 = 93,42 \, \frac{\text{kJ}}{\text{kg}} \)
    \item \( \text{interpolieren bei Ausgangsdruck mit} \, h_1 \)
\end{itemize}

\section*{e)}

\begin{itemize}
    \item Die Temperatur wurde natürlich sicher \textit{gestrichen} zu bekommen.
\end{itemize}

```