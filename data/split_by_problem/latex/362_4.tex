
``````latex


\section*{Aufgabe 4}

\subsection*{a)}

\begin{description}
    \item[Graph:] The graph is a Pressure-Temperature (P-T) diagram. The x-axis is labeled with $T$ (Temperature) and the y-axis is labeled with $P$ (Pressure). There are three points labeled 1, 2, and 3. Point 1 is in the liquid region, point 2 is in the vapor region, and point 3 is on the boundary between liquid and vapor. The line connecting points 1 and 2 is horizontal, indicating a constant temperature process. The line connecting points 2 and 3 is curved, indicating a change in both temperature and pressure. The region between points 1 and 2 is labeled "flüssig" (liquid), and the region between points 2 and 3 is labeled "Dampf" (vapor). The line from point 1 to the boundary is labeled "x=0" (quality equals zero), and the line from point 2 to the boundary is labeled "x=1" (quality equals one).
\end{description}

\subsection*{b)}

Energiebilanz über Verdichter $\rightarrow 2 \rightarrow 3$

\[
0 = \dot{m} \left( h_{\text{out}} - h_{\text{in}} \right) - W_t
\]

\[
\Rightarrow W_t = \dot{m} \left( h_{\text{out}} - h_{\text{in}} \right)
\]

\[
T_2 - T_1 = T_{\text{out}} + 10\,K
\]

\[
T_2 - T_1 = 6\,K
\]

\[
T_2 = T_1 + 10\,^\circ C \quad \text{aus Tabelle}
\]

\[
T_2 = T_1 + 6\,K
\]

\[
T_1 = 10\,^\circ C \quad \text{aus Tabelle}
\]

\[
T_2 = 16\,^\circ C \quad x = 1 \quad \text{gesättigter Dampf}
\]

\[
T_{AB} = A - 10 \quad h_2 = h_{\text{out}} = 237,74 \frac{\text{kJ}}{\text{kg}}
\]

``````latex


\section*{Aufgabe b)}

\subsection*{b.2)}
\begin{align*}
    T_2 &= -16^\circ C \\
    h_2 &= 237.74 \frac{\text{kJ}}{\text{kg}} \\
    p_3 &= 8 \text{bar}
\end{align*}

\[
\Rightarrow \text{Kreisel}
\]

\[
s_2 = s_3
\]

\[
s_2 = 0.8328 \frac{\text{kJ}}{\text{kg} \cdot \text{K}} = s_3
\]

\[
\Rightarrow \text{TAB A-12, 8 bar}
\]

\[
\text{S}_{\text{sat}} = 0.92066 \frac{\text{kJ}}{\text{kg} \cdot \text{K}}, \text{ at } 31.33^\circ C
\]

\[
s(20^\circ C) = 0.8374 \frac{\text{kJ}}{\text{kg} \cdot \text{K}}
\]

\[
\Rightarrow \text{interpolieren für } T_3:
\]

\[
\Rightarrow T_3 =
\]

\subsection*{d)}

```