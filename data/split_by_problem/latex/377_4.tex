
``````latex


\section*{Übung}

\subsection*{b) 1/15 um verdichtet}

\[
Q = \dot{m}(h_{2} - h_{3}) - (-\dot{W})
\]

\[
\dot{m} = \frac{-\dot{W}}{h_{2} - h_{3}} = \left| \text{TAB} \quad h_{2} = \quad h_{3} = \right.
\]

\[
\begin{array}{c|ccccc}
 & T & P \, [\text{bar}] & V & x & \dot{W} \, [\text{W}] & S \\
\hline
1 & & & & & & \\
2 & & 1 & & -28\dot{W} & 0 \\
3 & & 8 & & & & \\
4 & & & & 0 & & \\
\end{array}
\]

\[
S_{23} = 0 \quad (\text{adiab. rev.})
\]

\subsection*{a)}

\textbf{Description of the graph:}

The graph is a Pressure-Temperature (P-T) diagram. The x-axis is labeled \( T \, [^\circ C] \) and the y-axis is labeled \( P \, [\text{bar}] \). The graph contains the following points and lines:

- Point 1 is at the intersection of the isobaric line at 1 bar and the adiabatic curve.
- Point 2 is on the isobaric line at 1 bar, to the right of point 1.
- Point 3 is on the isobaric line at 8 bar, below point 2.
- Point 4 is on the isobaric line at 8 bar, to the left of point 3.

The lines connecting these points are as follows:

- A horizontal line (isobar) connects points 1 and 2.
- A curved line (adiabatic) connects points 2 and 3.
- A horizontal line (isobar) connects points 3 and 4.
- A curved line (adiabatic) connects points 4 and 1.

The labels on the graph are:
- "isobare" on the horizontal lines.
- "adiabate" and "adiabate (reversibel)" on the curved lines.

```