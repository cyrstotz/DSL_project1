
``````latex


\section*{4(a)}
\textbf{ges: P-T}

\subsection*{Graphical Descriptions}

\subsubsection*{First Graph}
The first graph is a P-T diagram with the following details:
- The vertical axis is labeled \( P \) with units \([mbar]\).
- The horizontal axis is labeled \( T \) with units \([^\circ C]\).
- The vertical axis has a marking at \( 5 \, mbar \).
- The horizontal axis has markings at \( -50^\circ C \), \( -20^\circ C \), \( 0^\circ C \), and \( 10^\circ C \).
- There are two curves:
  - The first curve starts from the bottom left, rises steeply, and then curves to the right, labeled \( F_{eis1} \).
  - The second curve starts from the bottom left, rises less steeply, and then curves to the right, labeled \( S_{a3} \).
- The region above the second curve is labeled \( Flüssig \).
- The region between the two curves is labeled \( T_{tripel} \).
- The region below the first curve is labeled \( Wasser in Lebensmitteln \).

\subsubsection*{Second Graph}
The second graph is another P-T diagram with the following details:
- The vertical axis is labeled \( P \) with units \([mbar]\).
- The horizontal axis is labeled \( T \) with units \([^\circ C]\).
- The vertical axis has a marking at \( 5 \, mbar \).
- The horizontal axis has a marking at \( T = 10^\circ C \).
- There are two curves:
  - The first curve starts from the bottom left, rises steeply, and then curves to the right, labeled \( F_{eis1} \).
  - The second curve starts from the bottom left, rises less steeply, and then curves to the right, labeled \( S_{a3} \).
- There are two points marked on the second curve:
  - Point 1 is labeled \( 1 \) and is located on the curve.
  - Point 2 is labeled \( 2 \) and is located on the curve, with a horizontal line extending to the right labeled \( T = const \).
- The region above the second curve is labeled \( Flüssig \).
- The region between the two curves is labeled \( T_{tripel} \).

``````latex

\section*{4d)}
\[
\eta_{ic} = \frac{|\dot{Q} \cdot \tau|}{|\dot{W}|} = \frac{\text{nutz}}{\text{Aufwand}}
\]
\[
\text{25 W}
\]

\section*{4b)}
\[
0 = \dot{m} (h_2 - h_3) + \dot{Q} - \dot{W}
\]
\[
\frac{\dot{W}_{ic}}{h_2 - h_3} = \dot{m}
\]
\[
\Rightarrow \dot{m} = \frac{W_{ic}}{h_2 - h_3}
\]
\[
h_2 @ x_2 = 1 \quad \text{(T_2)}
\]
\[
h_3 @ p_3 = 8 \text{bar} \quad \text{(T_3)}
\]

\section*{4c)}
\[
\dot{\Phi} = \dot{\Phi}_f + x (\dot{\Phi}_g - \dot{\Phi}_f)
\]
\[
x = \frac{q - \dot{\Phi}_f}{\dot{\Phi}_g - \dot{\Phi}_f}
\]

\section*{Figure Description}
There is a diagram labeled "Ü1b" with the following details:
- A rectangular box with three points labeled "1", "2", and "3".
- Point "1" is on the left side of the box, point "2" is on the top side, and point "3" is on the right side.
- Inside the box, there is a circular shape with an arrow pointing from the center to the top right, labeled "adiabat".
- The box is labeled "Sas".

``````latex


4e)

Die Temperatur würde langsam sinken

Gase lassen sich leichter kühlen als Flüssigkeiten

Falls der Druck

```