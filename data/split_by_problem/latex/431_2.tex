
``````latex


2. a)

\begin{description}
    \item[Graph Description:] The graph is a Temperature-Entropy (T-S) diagram. The x-axis is labeled $s \left( \frac{kJ}{kg \cdot K} \right)$ and the y-axis is labeled $T \left[ K \right]$. There are three isobars labeled $p_0 = p_6$, $p_4 = p_5$, and $p_2 = p_3$. The graph shows a cycle with points labeled 1 through 6. The cycle starts at point 1, moves to point 2, then to point 3, point 4, point 5, and finally to point 6, which connects back to point 1.
\end{description}

b)

\[
5-6 \text{ isentrop} \Rightarrow \Delta s = 0
\]

\[
0 = c_p \ln \left( \frac{T_6}{T_5} \right) - R \ln \left( \frac{p_6}{p_5} \right)
\]

\[
\eta = \frac{c_p}{c_v} \Rightarrow c_v = \frac{c_p}{\eta} = 0.7186 \frac{kJ}{kg \cdot K}
\]

\[
R = c_p - c_v = 0.2874 \frac{kJ}{kg \cdot K}
\]

\[
T_6 = T_5 e^{\frac{R}{c_p} \ln \left( \frac{p_6}{p_5} \right)}
\]

\[
T_6 = 328.084 K
\]

\[
0 = \dot{m} \left( h_5 - h_6 + \frac{w_5^2 - w_6^2}{2} \right) + \dot{Q} - \dot{W} = 0
\]

\[
w_6 = \sqrt{2 h_5 - 2 h_6 + w_5^2 - \sqrt{2 \left( c_p \left( T_5 - T_6 \right) + w_5^2 \right)}}
\]

\[
= 507.25 \frac{m}{s}
\]

``````latex


\section*{Student Solution}

\subsection*{c)}

\begin{align*}
e_{k,st,c} - e_{k,st,0} &= (h_c - h_0 - T_0(s_c - s_0) + k_e) \\
&= C_p(T_c - T_0) - T_0 \left( C_p \ln \left( \frac{T_c}{T_0} \right) - R \left( \ln \left( \frac{p_c}{p_0} \right) \right) \right) \\
&\quad - \frac{\omega_c^2 - \omega_0^2}{2} \\
&= 120.802 \frac{kJ}{kg}
\end{align*}

\subsection*{d)}

\begin{align*}
0 &= (e_{2,st,c} + e_{2,st,0}) + \left( 1 - \frac{T_0}{T_e} \right) \dot{Q} - \dot{e}_{k,rev} = 0 \\
\dot{e}_{k,rev} &= e_{k,st,0} - e_{k,st,c} + \left( 1 - \frac{T_0}{T_e} \right) \left( \frac{263.15 kJ}{128.5 kJ} \right) \\
&= 848.78 \frac{kJ}{kg}
\end{align*}

\subsection*{Graphical Content}

There is a small diagram in part (d) that appears to be a sketch of a thermodynamic cycle or process. The diagram consists of a closed loop with arrows indicating the direction of the process. The axes are not labeled, and there are no numerical values or specific points marked on the diagram. The loop is drawn in a rough, freehand style, suggesting it is a conceptual illustration rather than a precise graph.

``````latex


