
``````latex


\section*{Problem 4}

\subsection*{a)}

\begin{description}
    \item[Graph Description:] The graph is a Pressure-Temperature ($P$-$T$) diagram. The x-axis is labeled $T$ (Temperature) and the y-axis is labeled $P$ (Pressure). There is a curve starting from the origin and rising upwards to the right. Points 1, 2, and 3 are marked on the curve. Point 1 is at the lower left, point 2 is at the middle, and point 3 is at the upper right. There is a horizontal line labeled "isobar" connecting points 1 and 2. Another horizontal line labeled "isobar" connects points 2 and 3. There is a vertical line labeled "isotherm" connecting points 2 and 2'. Another vertical line labeled "isotherm" connects points 3 and 3'. The points 2' and 3' are on the curve.
\end{description}

\subsection*{b)}

\[
Q = \dot{m}_2 \left[ h_2 - h_3 \right] - \dot{W}_K
\]

\[
T_1 = -10^\circ C
\]

\[
T_2 = T_1 - 6K = -16^\circ C
\]

\[
\Rightarrow h_2 = h_3 = 237.74 \frac{kJ}{kg} \quad (\text{Tab A.10})
\]

\[
p_2 = 1.5748 bar \quad (\text{Tab A.10})
\]

\subsection*{c)}

\[
p_1 = p_2 = 1.5748 bar
\]

\[
T_1 = T_1 = -10^\circ C
\]

\[
x_1 =
\]

\subsection*{d)}

\[
\epsilon_K = \frac{\dot{Q}_{zu}}{\dot{W}_K + 1}
\]

\[
\dot{Q}_{zu} = \dot{Q}_K
\]

\[
\dot{W}_K = \dot{W}_K = 286W
\]

\subsection*{e)}

```