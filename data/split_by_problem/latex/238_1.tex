
``````latex


\section*{Thermodynamik 1}

\subsection*{Aufgabe 1}

\begin{itemize}
    \item[a)] Von der Aufgabenstellung wissen wir, dass es sich beim Kühlmittel um eine ideale Flüssigkeit handelt.
    
    \[
    \frac{dE'}{dt} = \sum_i \left[ m_i \left( h_i + \frac{ke_i + pe_i}{m_i} \right) \right] + \sum_j Q_j - \sum_k W_k
    \]
    
    \[
    0 = m_e \left[ h_{ein} - h_{aus} \right] + Q_{ab}
    \]
    
    \[
    Q_{ab} = m_{ein} \left[ h_{aus} - h_{ein} \right]
    \]
    
    \[
    h_{ein} = h_f (70^\circ C) = 292.98 \frac{\text{kJ}}{\text{kg}} \quad \text{(von TAB A-2)}
    \]
    
    \[
    h_{aus} = h_f (100^\circ C) = 419.04 \frac{\text{kJ}}{\text{kg}} \quad \text{(von TAB A-2)}
    \]
    
    \[
    Q_{ab} = 0.3 \frac{\text{kg}}{\text{s}} \left[ 419.04 \frac{\text{kJ}}{\text{kg}} - 292.98 \frac{\text{kJ}}{\text{kg}} \right] = \underline{37.82 \text{ kW}}
    \]
    
    \item[b)] Die thermodynamische Mitteltemperatur lässt sich wie folgt bestimmen:
    
    \[
    Q = \dot{m}_{ein} \left( s_{ein} - s_{aus} \right) + \frac{Q_{ab}}{T}
    \]
    
    \[
    T = \int_{s_a}^{s_e} T \, ds
    \]
    
    \[
    s_a - s_e = \int_{T_1}^{T_2} \frac{c}{T} \, dT = c \cdot \ln \left( \frac{T_{aus}}{T_{ein}} \right) = c \cdot 0.0341155
    \]
    
    \[
    T = \frac{s}{c \cdot 0.0341155}
    \]
    
\end{itemize}

``````latex


\section*{c)}

\begin{equation}
\frac{d\dot{S}}{dt} = \sum_i \dot{m}_i s_i + \sum_i \frac{\dot{Q}_i}{T_i} + \dot{S}_{erz}
\end{equation}

\begin{equation}
\dot{S}_{erz} = \dot{m} (s_{aus} - s_{ein}) + \frac{Q_{ab}}{T_{job}}
\end{equation}

\begin{equation}
= \dot{m} \, c^{i} \ln \left( \frac{T_{aus}}{T_{ein}} \right) + \frac{65 \, \text{kW}}{295 \, \text{K}}
\end{equation}

\begin{equation}
\dot{S}_{erz} = \dot{m}_{uml} \, c^{i}_{uml} \ln \left( \frac{288.15 \, \text{K}}{285.15 \, \text{K}} \right) + 0.22 \, \frac{\text{kJ}}{\text{kg} \cdot \text{K}}
\end{equation}

``````latex


