
``````latex


\section*{2.}
\subsection*{a)}

\begin{center}
\textbf{Graph Description:}
\end{center}

The graph is a plot with the vertical axis labeled \( T \) in units of \( [K] \) and the horizontal axis labeled \( s \) in units of \( \left[ \frac{kJ}{kg \cdot K} \right] \). 

There are three isobaric lines labeled as follows:
- The topmost isobar is labeled \( \text{isobar @ } p_3 = 0.5 \, \text{bar} \).
- The middle isobar is labeled \( \text{isobar @ } p_2 \).
- The bottom isobar is labeled \( \text{isobar @ } p_0 = 0.797 \, \text{bar} \).

The graph contains six points labeled 0 through 6, connected by arrows indicating the direction of processes:
- Point 0 is at the bottom left.
- Point 1 is above and to the right of point 0.
- Point 2 is above and to the right of point 1, on the middle isobar.
- Point 3 is above and to the right of point 2, on the top isobar.
- Point 4 is to the right of point 3, connected by a dashed line.
- Point 5 is below point 4, on the middle isobar.
- Point 6 is below point 5, on the bottom isobar.

\subsection*{b)}

\[
T_0 = -30^\circ C = 243.15 \, K
\]

\[
p_0 V_0 = R T_0 \implies V_0 = \frac{R T_0}{p_0}
\]

\[
R = c_p - c_v = c_p - \frac{c_p}{k} = 0.287 \, \frac{kJ}{kg \cdot K}
\]

\[
\left( V_0 = 3.65908 \, \frac{m^3}{kg} \right)
\]

\[
\frac{T_6}{T_0} = \left( \frac{V_6}{V_0} \right)^{k-1} \implies \frac{T_6}{T_0} = \left( \frac{R T_6}{p_0 V_0} \right)^{k-1}
\]

\[
p_0 V_6 = R T_6 \implies V_6 = \frac{R T_6}{p_0}
\]

\[
T_0 = \left( \frac{R}{p_0 \cdot V_0} \right)^{k-1} T_6^{k-1} = \left( \frac{R}{p_0 \cdot V_0} \right)^{k-1} T_6^{k}
\]

``````latex


\section*{2 b)}

\begin{equation*}
\frac{T_6}{T_5} = \left( \frac{p_6}{p_5} \right)^{\frac{k-1}{k}} \implies T_6 = T_5 \left( \frac{p_6}{p_5} \right)^{\frac{k-1}{k}} = \boxed{328,0747 \text{ K}}
\end{equation*}

\begin{equation*}
S_5 = S_6
\end{equation*}

\begin{equation*}
0 = \dot{m}_{\text{ges}} \left[ h_5 - h_6 + \frac{v_5^2 - v_6^2}{2} \right] - \dot{W}_{\text{turbine}}
\end{equation*}

\section*{Description of Figures}

The page contains a series of equations and a diagram. The diagram is located at the top of the page and consists of several arrows and labels. Here is a detailed description of the diagram:

- The diagram shows a process involving temperatures and pressures.
- There are two main states labeled as \( T_6 \) and \( T_5 \).
- \( T_6 \) is connected to \( T_5 \) by an arrow pointing from \( T_5 \) to \( T_6 \).
- Above the arrow, there is a fraction \(\frac{p_6}{p_5}\) raised to the power of \(\frac{k-1}{k}\).
- The labels \( p_6 \) and \( p_5 \) are written next to the respective states.
- The diagram is surrounded by several curved lines and arrows indicating the flow and transformation of the states.

The rest of the page contains the equations written above.

``````latex


