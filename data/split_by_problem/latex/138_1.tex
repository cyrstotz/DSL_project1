
``````latex


\section*{Aufgabe 1}

Stelle Energiebilanz für offenes System des Reaktors auf:
\[
\dot{E} = \dot{m}_{in} \left( h_{Ein} + \frac{v_{Ein}^2}{2} + g z_{Ein} \right) - \dot{m}_{out} \left( h_{Aus} + \frac{v_{Aus}^2}{2} + g z_{Aus} \right) + \dot{Q} - \dot{W} = \dot{m}_{in} \left( h_{Ein} + \frac{v_{Ein}^2}{2} + g z_{Ein} \right) - \dot{m}_{out} \left( h_{Aus} + \frac{v_{Aus}^2}{2} + g z_{Aus} \right) + \dot{Q} - \dot{W} = 0 \quad \text{da stationär}
\]

Die potentielle und kinetische Energie vernachlässigt werden können also
\[
\dot{Q} = \dot{m}_{in} \left( h_{Ein} - h_{Aus} \right) \Rightarrow \dot{Q} = \dot{m} \left( h_{Ein} - h_{Aus} \right)
\]

Nun mit Tabelle für siedende Flüssigkeit bei 100°C also \( A = 2 \) mit \( x = 0 \):

\[
h_{Ein}(70°C) = h_f(70°C) = 293,98 \frac{\text{kJ}}{\text{kg}} \quad \text{und} \quad h_{Aus}(100°C) = h_f(100°C) = 419,04 \frac{\text{kJ}}{\text{kg}}
\]

\[
\Rightarrow \dot{Q} = 100 \frac{\text{kJ}}{\text{s}} \cdot 0,3 \frac{\text{kg}}{\text{s}} \cdot (293,98 \frac{\text{kJ}}{\text{kg}} - 419,04 \frac{\text{kJ}}{\text{kg}}) \approx 6,2182 \frac{\text{kW}}{\text{kg}}
\]

\subsection*{b)}

Wärmeübertragbarkeit des Systems, offen, stationär. Nimmt nur Wärme auf, ideale Flüssigkeit.

\[
T_{in} = 288,15 \, \text{K} \quad T_{out} = 293,15 \, \text{K} \quad P_1 = P_2
\]

\[
\dot{T} = \frac{\dot{Q}}{\dot{m} \cdot c_p} \quad \Lambda \quad S(T_1) - S(T_2) = \int_{T_1}^{T_2} \frac{c_p(T)}{T} \, dT \quad \text{wenn} \quad c_p \quad \text{als konstant an also}
\]

\[
\Rightarrow \dot{T} = \frac{T_2 - T_1}{\ln \left( \frac{T_2}{T_1} \right)} \quad \text{Einheit} \left[ \frac{\text{kJ}}{\text{kg K}} \right] \quad \text{A minus Quellfähigkeit als} \quad \dot{Q} \Rightarrow \dot{T} = \frac{(293,15 - 288,15 \, \text{K})}{\ln \left( \frac{293,15}{288,15} \right)} \approx 293,12 \, \text{K}
\]

\subsection*{c)}

Nehme dass das Teilsystem adiabate, isotherme Umgebung ist:

Entropiebilanz mit \(\dot{T}\):

\[
\dot{S} = \dot{m}_{in} s_{Ein} - \dot{m}_{out} s_{Aus} + \frac{\dot{Q}}{T} + \dot{S}_{ext} = 0 \quad \text{da stationär System}
\]

\[
\Rightarrow \frac{\dot{Q}}{T} = \dot{m}_{in} s_{Ein} - \dot{m}_{out} s_{Aus} = -\```latex


\section*{1)}

\begin{align*}
    \dot{m}_{ein} - \dot{m}_{aus} &= 0 \\
    m_{nass} &= 5.755 \text{ kg} \\
    T_{ein,12} &= 20^\circ \text{C} \\
    T_{aus,1} &= 100^\circ \text{C} \\
    x_{D,1} &= 0.005 \\
    x_{D,2} &= 0 \\
    T_{aus,2} &= 70^\circ \text{C}
\end{align*}

\noindent
\(\Rightarrow\) Wir geben dem Nass sein Feucht \(m_{nass}\) mit 0.005 \(\dot{m}_{nass}\) an Dampf und dann an andere Feuchtigkeit und 70°C entweicht wieder.

\noindent
\(\Rightarrow\) Energiegleichung des geschlossenen Systems:

\begin{align*}
    \dot{E} &= \frac{d}{dt} (KE) + \frac{d}{dt} (PE) + \frac{d}{dt} (u) = \dot{Q} - \dot{W} \\
    \Rightarrow \dot{u} &= \dot{Q}
\end{align*}

\noindent
Innere Energie in Zustand 4 aus A-2:

\begin{align*}
    u_{Dampf}(100^\circ \text{C}) &= 2504.5 \frac{\text{kJ}}{\text{kg}} \\
    u_{Flüssig}(100^\circ \text{C}) &= 419.04 \frac{\text{kJ}}{\text{kg}} \\
    u_{Flüssig}(70^\circ \text{C}) &= 293.95 \frac{\text{kJ}}{\text{kg}}
\end{align*}

\noindent
\(\Rightarrow\)

\begin{align*}
    \Delta m_{12} &= \frac{Q_{12} - m_{nass} \cdot u_{Flüssig}(70^\circ \text{C})}{u_{Flüssig}(70^\circ \text{C})} \\
    &\approx 2794.6 \frac{\text{kJ}}{\text{kg}}
\end{align*}

\section*{c)}

``````latex


