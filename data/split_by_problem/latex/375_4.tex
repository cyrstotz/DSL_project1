
``````latex


\section*{Aufgabe 4}

\subsection*{a)}

\begin{description}
    \item[Graph:] The graph is a plot with the vertical axis labeled as \( p \) and the horizontal axis labeled as \( T \). The plot shows a single peak, resembling a bell curve. The peak is located towards the center of the graph. There are three points marked on the curve: one at the left base labeled as \( 3 \), one at the peak labeled as \( 2 \), and one at the right base labeled as \( 1 \). There is a horizontal arrow pointing from point \( 3 \) to point \( 1 \), indicating a process or transition between these points. The word "adiabat" is written near the curve.
\end{description}

\subsection*{b)}

\[
0 = \dot{m} (h_2 - h_3) + \dot{Q} + \dot{W}_k
\]

\begin{align*}
p_3 &= 8 \text{ bar} \\
T_2 &\text{ ist 6K unter } T_i \\
T_i &= 273K + 10K = 283K \\
T_2 &= 283K - 6K = 277K \quad (5^\circ C)
\end{align*}

\[
h_2 \text{ interpoliert} \quad y = \frac{x - x_1}{x_2 - x_1} (y_2 - y_1) + y_1
\]

\begin{align*}
h_{40^\circ C} &= 249.53 \frac{kJ}{kg} \\
h_{80^\circ C} &= 251.80 \frac{kJ}{kg}
\end{align*}

\[
h_{50^\circ C} = \frac{1}{4} \frac{T_2 - 40^\circ}{80^\circ - 40^\circ} (h_{80^\circ} - h_{40^\circ}) + h_{40^\circ}
\]

\[
h_2 = 250.087 \frac{kJ}{kg}
\]

\begin{align*}
h_3 &\text{ (5 bar)} \\
s_2 &= s_3 \\
s_2 \text{ interpoliert}
\end{align*}

``````latex


\begin{itemize}
    \item[c)] \(Z4 \quad 8 \text{bar} \quad x_4 = 0\)
    
    \textit{Adiabate Drossel:} \(h_2 = h_1\)
    
    \(h_4 \quad (8 \text{bar} \quad x = 0) \quad TAB A11 \quad 93.42 \frac{kJ}{kg}\)
    
    \(\quad (1-x) \cdot h_f + x \cdot h_g = h_2\)
    
    \(p_2 \quad @-22^\circ C = 1.2452 \text{bar}\)
    
    \(p_4 + p_2 = 9.2452 \text{bar}\)
    
    \(93.42 \frac{kJ}{kg} = (1-x)\)
    
    \item[d)] \(\sum_k = \quad \cancel{\text{illegible}} \quad \frac{\dot{Q}_k}{\dot{V}_k} = \text{Nutzen Aufz.}\)
\end{itemize}

``````latex


\section*{Aufgabe 4 weiter}

e)

```