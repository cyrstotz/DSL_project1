
``````latex


\section*{Aufgabe 2}

\subsection*{a)}

\begin{description}
    \item[Graph Description:] The graph is a Temperature-Entropy (T-S) diagram. The x-axis is labeled as $T$ (Temperature) and the y-axis is labeled as $s$ (Entropy). There are several curves and points marked on the graph:
    \begin{itemize}
        \item A curve labeled "adiabat $s = \text{const}$" starting from the bottom left and moving upwards to the right.
        \item Another curve labeled "adiabat $s = \text{const}$" parallel to the first one but starting higher on the y-axis.
        \item A curve labeled "isobar $p_0 = p_6$" starting from the bottom left and moving upwards to the right, intersecting the adiabats.
        \item Points labeled 1 through 6, with arrows indicating transitions between these points.
        \item A curve labeled "isotherm" starting from point 1 and moving horizontally to the right.
        \item A curve labeled "isotherm" starting from point 4 and moving horizontally to the right.
        \item A curve labeled "isotherm" starting from point 5 and moving horizontally to the right.
        \item A curve labeled "isotherm" starting from point 6 and moving horizontally to the right.
    \end{itemize}
\end{description}

\subsection*{b) Energiegleichung}

\begin{equation}
0 = \dot{m} \left( h_e - h_a - \frac{w_e^2 - w_a^2}{2} + g(z_e - z_a) \right) + \dot{Q} - \frac{V_f}{t}
\end{equation}

\begin{equation}
q_{13} = \frac{\dot{Q}_{13}}{\dot{m}_k}
\end{equation}

\begin{equation}
q_{B} = \frac{\dot{Q}_{13}}{\dot{m}_k} = q_{13}
\end{equation}

\begin{equation}
\frac{\dot{m}_m}{\dot{m}_k} = \frac{5.293}{7} = 7 \cdot \dot{m}_k = \left( \frac{5.293}{7} \right) \cdot \dot{m}_m
\end{equation}

\subsection*{c) ex_{str}}

\begin{equation}
\alpha_{ex_{str}} = \dot{m} (h_e - h_a - T_0 (s_e - s_a) + ke + pe)
\end{equation}

\begin{equation}
\alpha_{E_{x_{str}}} = \dot{m} (h_e - h_a - T_0 (s_e - s_a) + ke + pe) = 0
\end{equation}

\subsection*{d) kalorisches System}

\begin{equation}
\Delta E = \sum \dot{m}_i \left( h_i + \frac{w_i^2}{2} \right) + Q_{13} = 0, \text{da gleichen Volumen im Anfang und Ende}
\end{equation}

\begin{equation}
0 = \dot{m} (h_e - h_a - \frac{w_e^2 - w_a^2}{2} + g(z_e - z_a)) + \dot{Q}_{13}
\end{equation}

\begin{equation}
0 = \dot{m} (h_e - h_a - \frac{200^2 - w_a^2}{2}) + \dot{Q}_{13}
\end{equation}

\begin{equation}
\sqrt{\left( \frac{\dot{Q}_{13}}{\dot{m}} - (h_e - h_a) \cdot 2 - 200^2 \right)} = w_a
\end{equation}

\begin{equation}
\text{Ableitung von } T_6
\end{equation}

``````latex


2) b)

\[
T_0 : \quad p v = R T \quad \rightarrow \frac{p}{R} = \text{const} = \frac{T}{v}
\]

\[
T_0 = T_s, \quad \text{da adiabater Prozess}
\]

\[
\frac{T_1}{v_1} = \frac{T_0}{v_0}
\]

d)

\[
\frac{\Sigma_i e_{i, \text{real}}}{m_{\text{gas}}} = e_{i, \text{real}} - \frac{T_0 \cdot \dot{S}_{\text{gen}}}{m_{\text{gas}}}
\]

``````latex


