
``````latex


\section*{Problem 2}

\begin{tabular}{|c|c|c|c|}
\hline
 & T & P & |h| \\
\hline
1 & -20 & & \\
\hline
2 & & $P_2 = P_4$ & \\
\hline
3 & & $P_3 = P_4$ & \\
\hline
4 & & $P_4 = P_5$ & \\
\hline
5 & 437,9 kJ & 0,5 bar & \\
\hline
6 & & 0,1516 bar & \\
\hline
\end{tabular}

\bigskip

\noindent
To the right of the table, there are three annotations:
\begin{itemize}
    \item $ \frac{s_1}{s_2} \Rightarrow \eta_{vs} < 1 $
    \item \textit{isentropic}
    \item An arrow pointing from $P_4 = P_5$ to the right
\end{itemize}

\bigskip

\noindent
Below the table, there are several equations and values:
\begin{align*}
    w_{14} &= 200 \frac{m}{s} \\
    w_{j} &= 220 \frac{m}{s} \\
    p_0 &= (0,14912 \text{m}) \\
    \text{mm} &= 5,292 \text{m} \\
    c_p &= 1006
\end{align*}

``````latex


\section*{a)}

\begin{itemize}
    \item The graph is a plot with the x-axis labeled as \( S \left[ \frac{kJ}{kg \cdot K} \right] \) and the y-axis labeled as \( T [K] \).
    \item There are four points labeled 1, 2, 3, and 4.
    \item Point 1 is at the bottom left, point 2 is directly above point 1, point 3 is to the right of point 2, and point 4 is below point 3 but to the right of point 2.
    \item The curve from point 1 to point 2 is labeled "isentrop".
    \item The curve from point 2 to point 3 is labeled "isentrop (reversibel)".
    \item The curve from point 3 to point 4 is labeled "isentrop".
    \item There is a horizontal line from point 1 to the right, labeled "isochore".
    \item There is a curve from point 1 to the right, labeled "isochore".
    \item There is a curve from point 3 to the right, labeled "isochore".
    \item There is a curve from point 4 to the right, labeled "isochore".
    \item There is a curve from point 1 to the right, labeled "isochore".
    \item There is a curve from point 3 to the right, labeled "isochore".
    \item There is a curve from point 4 to the right, labeled "isochore".
    \item There is a curve from point 1 to the right, labeled "isochore".
    \item There is a curve from point 3 to the right, labeled "isochore".
    \item There is a curve from point 4 to the right, labeled "isochore".
    \item There is a curve from point 1 to the right, labeled "isochore".
    \item There is a curve from point 3 to the right, labeled "isochore".
    \item There is a curve from point 4 to the right, labeled "isochore".
    \item There is a curve from point 1 to the right, labeled "isochore".
    \item There is a curve from point 3 to the right, labeled "isochore".
    \item There is a curve from point 4 to the right, labeled "isochore".
    \item There is a curve from point 1 to the right, labeled "isochore".
    \item There is a curve from point 3 to the right, labeled "isochore".
    \item There is a curve from point 4 to the right, labeled "isochore".
    \item There is a curve from point 1 to the right, labeled "isochore".
    \item There is a curve from point 3 to the right, labeled "isochore".
    \item There is a curve from point 4 to the right, labeled "isochore".
    \item There is a curve from point 1 to the right, labeled "isochore".
    \item There is a curve from point 3 to the right, labeled "isochore".
    \item There is a curve from point 4 to the right, labeled "isochore".
    \item There is a curve from point 1 to the right, labeled "isochore".
    \item There is a curve from point 3 to the right, labeled "isochore".
    \item There is a curve from point 4 to the right, labeled "isochore".
    \item There is a curve from point 1 to the right, labeled "isochore".
    \item There is a curve from point 3 to the right, labeled "isochore".
    \item There is a curve from point 4 to the right, labeled "isochore".
    \item There is a curve from point 1 to the right, labeled "isochore".
    \item There is a curve from point 3 to the right, labeled "isochore".
    \item There is a curve from point 4 to the right, labeled "isochore".
    \item There is a curve from point 1 to the right, labeled "isochore".
    \item There is a curve from point 3 to the right, labeled "isochore".
    \item There is a curve from point 4 to the right, labeled "```latex


\section*{Problem 2d}

\begin{equation*}
O = \dot{m} \dot{e}
\end{equation*}

\begin{equation*}
O = \dot{m} (\dot{e}_{str}) + (1 - \frac{T_0}{T_j}) \dot{Q}_j - \dot{W}_{t,u} - \dot{E}_{verl}
\end{equation*}

\begin{equation*}
\dot{e}_{verl} = \dot{e}_{str} - \dot{W}_{t,u}
\end{equation*}

\begin{equation*}
= 100 \frac{h}{g}
\end{equation*}

\begin{equation*}
\dot{W}_{t,u} = \dot{m} \dot{e}
\end{equation*}

\textbf{Description of the diagram:} There is a large arrow pointing from the equation $\dot{W}_{t,u} = \dot{m} \dot{e}$ to the text "Lagerort arbeitet Turbine an" and "Verlustleiter".

\begin{equation*}
\text{Energie bilanz}
\end{equation*}

\begin{equation*}
\frac{\dot{W}_{turbine}}{\dot{m}} = \frac{(h_3 - h_4)}{c_v (T_3 - T_4)} + \dot{Q}
\end{equation*}

``````latex


\[
w_c = \dot{m}g \left( c_p (T_3 - T_0) + \frac{w_5^2}{2} \right)
\]

\[
\dot{m}g = \dots
\]

\[
\dot{m}_k + \dot{m}_m = \dot{m}_k (1 + 5.1, 2.53)
\]

\[
0 = \dot{m}_k (h_2 - h_3) + \dot{m}_k \cdot q_b
\]

\[
-q_b = \dot{m}_k c_p (T_2 - T_3)
\]

\[
\text{Turbine:}
\]

\[
0 = \dot{m}_k c_p (T_1 - T_2) + Q_{kW}^0 \frac{W_t}{kW}
\]

\[
\text{(c)}
\]

\[
\Delta \text{ex}_{str} = \text{ex str}_6 \cdot \text{ex}_0
\]

\[
= \left( \frac{h_6 - h_0}{T_0} \right) - T_0 (s_6 - s_0) + p_0 \left( \frac{V_6 - V_0}{V_0} \right)
\]

\[
= c_p (T_6 - T_0) - T_0 \left( c_p \ln \left( \frac{T_6}{T_1} \right) - R \ln \left( \frac{p_6}{p_0} \right) \right) + p_0 (V_6 - V_0)
\]

\[
V_0 = \frac{R \cdot T}{p_0} \quad V_6 = \frac{R \cdot T}{p_0}
\]

\[
R = c_p \left( 1 - \frac{1}{r_c} \right)
\]

\[
c_v = \frac{c_p}{r_c}
\]

``````latex


