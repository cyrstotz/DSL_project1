
``````latex


4) \text{Ebene}

a)

\begin{center}
\textbf{Graph Description:}
\end{center}

The graph is a phase diagram with temperature \( T \) (in Kelvin) on the x-axis and an unspecified variable on the y-axis. The graph is divided into three regions labeled "Fest" (solid), "Flüssig" (liquid), and "Gasförmig" (gaseous). 

- The "Fest" region is on the left side of the graph.
- The "Flüssig" region is on the top right side of the graph.
- The "Gasförmig" region is in the middle and bottom right side of the graph.

There are two curves:
- The first curve, labeled "isotherm," is a horizontal line starting from the "Fest" region and extending into the "Gasförmig" region.
- The second curve, labeled "isobar," starts from the "Flüssig" region and extends horizontally into the "Gasförmig" region.

There are three points labeled 1, 2, and 3:
- Point 1 is on the "isotherm" line in the "Fest" region.
- Point 2 is on the "isobar" line in the "Flüssig" region.
- Point 3 is in the "Gasförmig" region, below the "isobar" line.

b)

\[
\frac{d}{dt} = \sum \dot{m}_i (h_i + \frac{c_i^2}{2} + g z_i) + \dot{Q} - \dot{W}
\]

\[
Q = \dot{m} (h_e - h_a) + \dot{Q} - \dot{W}
\]

\[
\dot{m} = \frac{\dot{W} - \dot{Q}}{h_e - h_a}
\]

\[
\begin{array}{ccc}
\rho & T & x \\
1 & 0 & 0 \\
2 & 1 & 1 \\
3 & 8 \text{ bar} & 1 \\
4 & 8 \text{ bar} & 0 \\
\end{array}
\]

\text{Wärmeübertrager}

\[
1 \rightarrow 2 \quad \dot{m} (h_e - h_a) + \dot{Q} = 0 \Rightarrow \dot{Q} = \dot{m} (h_a - h_e)
\]

``````latex


\section*{c)}

\section*{d)}
\[
E_k = \frac{\dot{Q}_{\text{zul}}}{\dot{Q}_{\text{ab}} - \dot{Q}_{\text{zul}}} = \frac{\dot{Q}_{\text{zul}}}{\dot{Q}_{\text{ab}} - \dot{Q}_{\text{id}}}
\]

\section*{e)}
Temperatur würde sich senken bis ein thermisches Gleichgewicht erreicht wäre.

```