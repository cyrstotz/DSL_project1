
``````latex


1. a) Energiesbilanz Wasser

\[
\Rightarrow 0 = \dot{m} (h_{\text{ein}} - h_{\text{aus}}) + Q_R - Q_{\text{aus}}
\]

\[
h_{\text{ein}} = h_f(70^\circ C) = 232,98 \frac{\text{kJ}}{\text{kg}} \quad \text{(A-2)}
\]

\[
h_{\text{aus}} = h_f(80^\circ C) = 419,04 \frac{\text{kJ}}{\text{kg}} \quad \text{(A-2)}
\]

\[
\Rightarrow |Q_{\text{aus}}| = \dot{m} (h_{\text{ein}} - h_{\text{aus}}) + Q_R
\]

\[
= 62,13 \frac{\text{kJ}}{\text{s}}
\]

\[
\Rightarrow \text{da realisiert} \quad Q_{\text{aus}} = -62,13 \frac{\text{kJ}}{\text{s}} \quad \text{(-62,13 kW)}
\]

b)

\[
\overline{T}_{\text{KF}} = \frac{\int_{s_a}^{s_e} T ds}{s_a - s_e} \Rightarrow \int dH = T ds + V dp \quad \text{weil isobar}
\]

\[
= \frac{\int_{s_a}^{s_e} dH}{s_a - s_e} = \frac{h_a - h_e}{s_a - s_e} = \frac{cp_l (T_e - T_a)}{cp_l \frac{T_e}{T_e} - R \ln \left( \frac{p_e}{p_e} \right) \rightarrow 0}
\]

\[
= \frac{T_a - T_e}{\ln \left( \frac{T_e}{T_a} \right)} = 283,124 \overline{T}_{\text{KF}}
\]

c) Satz aus Wärmelehre

\[
\text{Wand zweiter Rechtsatz der Lehre}
\]

\[
\Rightarrow \text{Entropiebilanz} \quad 0 = \sum \dot{m}_i s_i + \frac{\dot{Q}_{\text{aus}}}{T_R} - \frac{Q_{\text{aus}}}{T_{\text{KF}}} + S_{\text{erw}}
\]

\[
\Rightarrow S_{\text{erw}} = \frac{Q_{\text{aus}}}{T_{\text{KF}}} - \frac{Q_{\text{aus}}}{T_R} \quad T_R: \text{isotherm da Nass-Dampf} = 373,15 K
\]

\[
= \frac{62,13}{5,46} \frac{2}{5 K} = 45,46 \frac{\text{kJ}}{\text{K}}
\]

``````latex


\section*{1d)}

\begin{align*}
\text{Zustand 1:} & \quad 100^\circ C \\
\text{Zustand 2:} & \quad 70^\circ C \quad \Rightarrow \quad 0 \text{MJ} + Q_{\text{rein}} = Q_{\text{aus}} + W
\end{align*}

\begin{align*}
Q_{\text{zu}} &= 2 \text{MJ} \\
Q_{\text{aus}} &= 35 \text{MJ}
\end{align*}

\text{Energiegleichung:}
\begin{align*}
Q_{\text{zu}} &= Q_{\text{zu}} + Q_{\text{rein}} + Q_{\text{aus}} - Q_{\text{aus}} + W
\end{align*}

\begin{align*}
\Delta U &= m_2 u_2 - m_1 u_1 \\
Q_{\text{zu}} &= m_2 - m_1
\end{align*}

\begin{align*}
u_2 &= u_f(70^\circ C) = 202,95 \frac{\text{kJ}}{\text{kg}} \\
u_1 &= u_f(100^\circ C, 0,005) = u_f(100^\circ C) + 0,005 (u_g(100^\circ C) - u_f(100^\circ C)) = 423,4 \frac{\text{kJ}}{\text{kg}}
\end{align*}

\begin{align*}
h_{\text{ein}} &= h_f(70^\circ C) = 83,06 \frac{\text{kJ}}{\text{kg}}
\end{align*}

\begin{align*}
m_1 &= 5755 \text{kg}
\end{align*}

\begin{align*}
\Rightarrow m_2 &= m_1 + Q_{\text{zu}} \\
\Rightarrow (m_1 + Q_{\text{zu}}) u_2 - m_1 u_1 &= Q_{\text{zu}} m_{\text{ein}} \\
\Rightarrow m_1 (u_2 - u_1) &= Q_{\text{zu}} (m_{\text{ein}} - u_1) \\
\Rightarrow Q_{\text{zu}} &= \frac{m_1 (u_2 - u_1)}{m_{\text{ein}} - u_2} = 3757,45 \text{kg}
\end{align*}

\section*{1e)}

\begin{align*}
\Delta S_{12} &= m_2 s_2 - m_1 s_1 \\
\Rightarrow m_2 &= m_1 + Q_{\text{zu}} = 8572,45 \text{kg}
\end{align*}

\begin{align*}
s_2 &= s_f(70^\circ C) = 0,95 \frac{\text{kJ}}{\text{kg K}} \\
s_1 &= s_f(100^\circ C) + 0,005 (s_g(100^\circ C) - s_f(100^\circ C)) = 1,3374 \frac{\text{kJ}}{\text{kg K}}
\end{align*}

\begin{align*}
\Rightarrow \Delta S_{12} &= 1388,2 \frac{\text{kJ}}{\text{K}}
\end{align*}

``````latex


