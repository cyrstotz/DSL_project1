
``````latex


\section*{Aufgabe 1}

\[ p = p_{\text{amb}} + \frac{F}{S} = p_{\text{amb}} + \frac{m \cdot g}{\pi r^2} \]
\[ = 10^5 + \frac{52 \cdot 9,81}{\pi \cdot (5 \cdot 10^{-2})^2} \]
\[ = 1,4 \, \text{bar} \]

\[ n = \frac{pV}{RT} \]
\[ R = \frac{R}{M} = \frac{8,314}{50 \cdot 10^{-3}} = 166,28 \, \frac{J}{\text{kg} \cdot \text{K}} \]

\[ n = \frac{1,4 \cdot 3,12 \cdot 10^{-3}}{166,28 \cdot (500 + 273,15)} \]
\[ = 3,4 \cdot 10^{-5} \]

\[ V = 3,12 \cdot 10^{-3} \, \text{m}^3 \]
\[ p = 1,4 \cdot 10^5 \, \text{Pa} \]

\subsection*{b)}

\[ x_{\text{eis}} > 0 \]

Die Grenze die das Gas verläuft ist die gleiche wie das Eis. \( p_{3,2} = 1,4 \cdot 10^5 \, \text{Pa} = 1,4 \, \text{bar} \). Die Temperatur ist also \( 0,000^\circ \text{C} \) nach der Tabelle.

\subsection*{c)}

\[ \frac{dQ}{dt} = \dot{m} (h_2 - h_1) + Q \]

\[ \Delta h = \frac{q_m}{n} = \]

\[ (u_2 - u_1) = q_m \]

\[ c_v (T_2 - T_1) = q_m \]

\[ q_m = 0,633 \cdot (0,005 - 500) \]
\[ = -316,5 \, \frac{\text{kJ}}{\text{kg}} \]

\[ T_L = 0,005^\circ \text{C} \]

\[ Q = -q_m = 36,6 \cdot 10^3 - 316,5 \]
\[ = -1440 \, \text{J} \]

``````latex


d) \quad x_{eis}

\[
\frac{dE}{dt} = \dot{m} (h_1 - h_2) + Q
\]

\[
\Delta u = q
\]

\[
u_1 (t_1) - u_1 (t_2) = q_{12}
\]

\[
x_1 \cdot u_{F, Eis} (0^\circ C) + (1 - x) \cdot u_{F, Flüssig} (0^\circ C) - x_2 \cdot u_{2, Fest} (0.005^\circ C) - (1 - x_2) u_{Flüssig} (0.005^\circ C) = q_{12}
\]

\[
x_1 \cdot u_{FH} + (1 - x) \cdot u_{FA} - x_L \cdot u_{H2} - (1 - x_L) u_{F2} = q_{12}
\]

\[
q_{1L} - x_1 \cdot u_{H2} - (1 - x) u_{FA} + u_{F2} - u_{H2} + u_{F2} = x_L
\]

\[
\frac{-316.5 + 0.6 \cdot (-333.650) - (1 - 0.6) \cdot (-0.045) - 0.031}{3.3342 - 0.053}
\]

\[
= 0.548
\]

Die Werte sind alle aus der Tabelle 1

``````latex


