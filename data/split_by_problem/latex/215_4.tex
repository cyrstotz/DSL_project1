
``````latex


\section*{Aufgabe 4}

\subsection*{a)}

\begin{description}
    \item[Graph Description:] The graph is a Pressure-Temperature (P-T) diagram. The vertical axis is labeled \( P \) and the horizontal axis is labeled \( T \). There are three distinct regions labeled "flüssig" (liquid), "gasförmig" (gaseous), and a line separating these regions. The line starts from the origin and curves upwards to the right. There is a horizontal line at \( 5 \text{ mbar} \) intersecting the curve. Two points are marked on the graph: point (i) is on the curve, and point (ii) is below the curve in the gaseous region. The temperature at point (ii) is labeled \( T_1 \) and the pressure at point (ii) is labeled \( 10 \text{ K} \). The region below the curve and above the horizontal line is labeled "test".
\end{description}

\subsection*{b)}

\begin{align*}
    &\text{En. Bilanz um Verdichter:} \\
    &0 = \dot{m} (h_2 - h_3) - \dot{W}_{\text{tn}} \\
    &\dot{m} = \frac{\dot{W}_{\text{tn}}}{h_2 - h_3} = \frac{-28 \text{ W}}{h_2 - h_3} \\
    &\text{adiabat \& reversibel:} \quad S_2 = S_3 \text{ aus Ent. Bil.}
\end{align*}

``````latex


\begin{itemize}
    \item[c)] En. Bilanz um Drossel:
    \[
    h_1 = h_4
    \]
    \[
    h_1 \text{ interpolieren mit } T_1 = T_2 \text{ und Table A-10}
    \]
    \[
    93.42 \frac{\text{kJ}}{\text{kg}}
    \]

    \item[d)] \[
    E_K = \frac{\dot{Q}_{zu}}{W_K} \quad \text{aus En. Bilanz um Kühlkreislauf:}
    \]
    \[
    28W \quad \dot{Q}_{zu} = \dot{m} (h_2 - h_1)
    \]

    \item[e)] Die Temperatur würde weiter abnehmen, da wenn Wärme rausfließt diese nicht mehr in Form innerer $E$ vorhanden ist. 
\end{itemize}

```