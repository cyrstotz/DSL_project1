
``````latex


\section*{Aufgabe 2}

\subsection*{(a)}

\begin{itemize}
    \item The graph is a Temperature-Entropy (T-s) diagram.
    \item The y-axis is labeled \( T (\degree C) \).
    \item The x-axis is labeled \( s \left( \frac{kJ}{kgK} \right) \).
    \item There are several curves and points labeled on the graph:
        \begin{itemize}
            \item A curve starting from the bottom left, labeled "Vorkompressor" and "Verdichter".
            \item A point labeled "0" at the bottom left.
            \item A point labeled "1" slightly above and to the right of point "0".
            \item A point labeled "2" further up and to the right of point "1".
            \item A point labeled "3" at the peak of the curve, labeled "Brennkammer".
            \item A point labeled "4" to the right of point "3", labeled "Turbine".
            \item A point labeled "5" below point "4", labeled "Schubdüse".
            \item A point labeled "6" to the left of point "5", labeled "0.197 bar".
            \item A horizontal line from point "6" to the left, labeled "-30 \degree C".
        \end{itemize}
    \item The graph also includes several annotations:
        \begin{itemize}
            \item "kompression" with an arrow pointing downwards.
            \item "verdichter" with an arrow pointing to the right.
            \item "isentrop" with an arrow pointing to the right.
            \item "Brennkammer" with an arrow pointing upwards.
            \item "Turbine" with an arrow pointing downwards.
            \item "adiabat" with an arrow pointing downwards.
        \end{itemize}
\end{itemize}

\subsection*{(b)}

\begin{itemize}
    \item The graph is a Pressure-Temperature (P-T) diagram.
    \item The y-axis is labeled \( P \).
    \item The x-axis is labeled \( T (\degree C) \).
    \item There are several curves and points labeled on the graph:
        \begin{itemize}
            \item A curve starting from the bottom left, labeled "Vorkompressor" and "Verdichter".
            \item A point labeled "0" at the bottom left.
            \item A point labeled "1" slightly above and to the right of point "0".
            \item A point labeled "2" further up and to the right of point "1".
            \item A point labeled "3" at the peak of the curve, labeled "Brennkammer".
            \item A point labeled "4" to the right of point "3", labeled "Turbine".
            \item A point labeled "5" below point "4", labeled "Schubdüse".
            \item A point labeled "6" to the left of point "5", labeled "0.197 bar".
            \item A horizontal line from point "6" to the left, labeled "-30 \degree C".
        \end{itemize}
    \item The graph also includes several annotations:
        \begin{itemize}
            \item "kompression" with an arrow pointing downwards.
            \item "verdichter" with an arrow pointing to the right.
            \item "isentrop" with an arrow pointing to the right.
            \item "Brennkammer" with an arrow pointing upwards.
            \item "Turbine" with an arrow pointing downwards.
            \item "adiabat" with an arrow pointing downwards.
        \end{itemize}
\end{itemize}

``````latex

\begin{itemize}
    \item[b)] 
    \begin{align*}
        W_s &= 220 \frac{m}{s} \\
        p_s &= 0.3 \, \text{bar} \quad p_6 = 0.791 \, \text{bar} \\
        T_s &= 437.9 \, K \\
        S_s &= S_6 \quad \text{da rev. adiabant.} \\
        W_{s6} &= \dot{m} (h - h_0 - T_0 (s - s_0) + ke) \\
        0S &= 0 = cp \ln \left( \frac{T_6}{T_s} \right) - R \ln \left( \frac{p_6}{p_s} \right)
    \end{align*}
    
    \item[c)] 
    \begin{align*}
        O_{exstr} &= [h_0 - h_0 - T_0 (s - s_0) + ke] - exstr.6
    \end{align*}
\end{itemize}

``````latex


