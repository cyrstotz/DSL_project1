
``````latex


4\textsuperscript{th} \textsuperscript{0} \textsubscript{1} \textsubscript{2}

\begin{itemize}
    \item[a)] 
    \begin{verbatim}
    (Graph Description)
    The graph is a Temperature-Entropy (T-s) diagram. The x-axis is labeled "s in kJ/kgK" and the y-axis is labeled "T in K". The graph consists of a closed loop with four points labeled 1, 2, 3, and 4. Point 1 is at the bottom right, point 2 is at the bottom left, point 3 is at the top left, and point 4 is at the top right. The curve from point 1 to point 2 is a horizontal line, the curve from point 2 to point 3 is a rising curve, the curve from point 3 to point 4 is a horizontal line, and the curve from point 4 to point 1 is a falling curve.
    \end{verbatim}
    
    \item[b)] 
    \begin{align*}
    T_1 &= -26 \degree C \\
    \rightarrow T_4 = T_2 &= -32 \degree C
    \end{align*}
    
    \begin{tabular}{c|c|c|c|c}
    T & P & V & x & Q & W \\
    \hline
    1 & T_1: -6 & P & & & \\
    2 & T_1: -6 & P & 1 & & (w_{23} = 28W) \\
    3 & & 8 & & & \\
    4 & & 8 & & 0 & \\
    \end{tabular}
    
    \item[c)] 
    \textbf{Energie bilanz}
    
    \begin{align*}
    \text{st. FP:} \quad Q &= \dot{m} [h_c - h_a + \frac{1}{2} (\Delta PE)] + \sum \dot{Q} - \sum W \\
    \rightarrow \dot{m} &= \frac{1 \cdot w_{31}}{h_2 - h_3} \\
    &= \frac{-28W}{227.9 - 276.27} \\
    &= -0.579 \times 10^{-3} \\
    &= 2039.4 \frac{kg}{h}
    \end{align*}
    
    \text{A-12 interpolation}
    
    \begin{align*}
    h_3 &= \frac{0.9456 - 0.9374}{0.9711 - 0.9374} \cdot (h(s: 0.9711, 8bar) - h(0.9374, 8bar)) + h(0.9374, 8bar) \\
    &= 276.27
    \end{align*}
    
    \begin{align*}
    h_2 &= h_f(T_1 = -32 \degree C) - 9.52 \\
    &= h_g(T_1 = -32 \degree C) - 227.90 \\
    h_1 &= h_f(p: 8bar) - 204.15 \frac{kg}{h} \\
    &= h_f(8bar)
    \end{align*}
    
    \begin{align*}
    s_4 = s_1 &= s_{g(8bar)} = 0.9066 \\
    \text{adi. \& rev.} \rightarrow s_2 &= s_3 \\
    s_2 &= s_f(T_1 = -32 \degree C) + 0.0401 \\
    &= s_g(T_1 = -32 \degree C) = 0.9456 \\
    h_3 &= (s_3: 8bar) \cdot (p_3: 8bar) \\
    &= 0.9456
    \end{align*}
\end{itemize}

``````latex

\section*{A.11}

\subsection*{c)}

\begin{align*}
h_2 &= h_f(\text{Blaul}) = 93.42 \\
s_4 &= 0.3459 = s_1 \\
s_f(\text{3bar}) \\
T_1 &= -22^\circ C \\
x &= \frac{s_1 - s_f}{s_g - s_f} = \underline{0.303}
\end{align*}

\begin{align*}
T_1 &= -22^\circ C \\
s_f(-22^\circ C) &= 0.0897 \\
s_g(-22^\circ C) &= 0.9351
\end{align*}

```