
``````latex


\section*{Aufgabe 4}

\subsection*{a)}

\begin{description}
    \item[Graph 1:] A pressure-volume (p-v) diagram with a complex curve. The curve starts at the bottom left, rises steeply, then falls, rises again, and falls once more. There are four points marked on the curve: 
    \begin{itemize}
        \item Point 1 is on the first rise.
        \item Point 2 is on the second rise.
        \item Point 3 is on the second fall.
        \item Point 4 is on the first fall.
    \end{itemize}
    There are horizontal lines connecting points 1 to 2 and 3 to 4, labeled as "isobar". The line connecting points 1 to 2 is labeled "x=0" at point 1 and "x=1" at point 2. The line connecting points 3 to 4 is labeled "p=5 bar". There is an arrow indicating the direction of the process from point 1 to point 2 and from point 3 to point 4.

    \item[Graph 2:] A pressure-temperature (p-T) diagram with a dome-shaped curve. The curve starts at the bottom left, rises to a peak, and then falls symmetrically. There are three points marked on the curve:
    \begin{itemize}
        \item Point 1 is on the left side of the dome.
        \item Point 2 is on the right side of the dome.
        \item Point 3 is at the peak of the dome.
    \end{itemize}
    There are horizontal lines connecting points 1 to 2 and 3 to 4, labeled as "isobar = 8 bar". The line connecting points 1 to 2 is labeled "x=0" at point 1 and "x=1" at point 2. There is an arrow indicating the direction of the process from point 1 to point 2.
\end{description}

\subsection*{b)}

\begin{equation*}
0 = \dot{m}_{\text{el}} \left( h_2 - h_3 \right) - \dot{W}_K
\end{equation*}

\begin{equation*}
\dot{m}_{\text{el}} = \frac{\dot{W}_K}{h_2 - h_3}
\end{equation*}

\begin{equation*}
h_2 = h_f + x \left( h_g - h_f \right) = h_f + h_{fg}
\end{equation*}

\begin{equation*}
\text{TABELLE A-12: } h_3 \left( p_3 = 8 \text{ bar} \right) = 
\end{equation*}

\begin{equation*}
T_3 = ?
\end{equation*}

\begin{equation*}
\frac{h_2 \left( T = -22^\circ C \right)}{h_f}
\end{equation*}

\begin{equation*}
\text{TABELLE A-10: } h_2 \left( T_2 = -22^\circ C \right) = 21.77 + 212.32 = 234.09 \frac{\text{kJ}}{\text{kg}}
\end{equation*}

``````latex


\begin{itemize}
    \item[c)] 
    \[
    h_1 = h_f + x_1 (h_g - h_f)
    \]
    \[
    \Rightarrow x_1 = \frac{h_1 - h_f}{h_g - h_f}
    \]
    \[
    T_i = 10 \cdot 10 - 273.15 - 20 = -283.15 \degree C
    \]
    \[
    T_{\text{ABH}} = h_1 (T = -283.15 \degree C)
    \]
    
    \item[d)] 
    \[
    \varepsilon_k = \frac{|\dot{Q}_{zu}|}{|\dot{W}|} = \frac{|\dot{Q}_{zu}|}{|\dot{Q}_{ab}| - |\dot{Q}_{zu}|}
    \]
    \[
    \varepsilon_k = \frac{|\dot{Q}_k|}{|\dot{Q}_{ab}| - |\dot{Q}_k|}
    \]
    
    \item[e)] 
    die Temperatur würde immer mehr sinken
\end{itemize}

```