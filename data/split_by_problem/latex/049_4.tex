
``````latex


\section*{Aufgabe 4}

\subsection*{a)}

\begin{description}
    \item[Graph Description:] The graph is a Pressure-Temperature (P-T) diagram. The x-axis is labeled \( T \) and the y-axis is labeled \( P \). There are four points labeled 1, 2, 3, and 4 forming a closed loop. The path from 1 to 2 is a vertical line, from 2 to 3 is a horizontal line, from 3 to 4 is a vertical line, and from 4 to 1 is a horizontal line. The area inside the loop is shaded.
\end{description}

\subsection*{b)}

\[
\dot{W}_k = \dot{m} \left( h_3 - h_4 \right) \implies \dot{m} = \frac{\dot{W}_k}{h_3 - h_4}
\]

\noindent
Wobei wir nun \( h_3 \) und \( h_4 \) von die Tabellen \textbf{A-10, A-11, A-12}.

\noindent
\( h_2 \) können wir interpolieren aus \( h_f (T_1 = 6) \), wobei \( T_1 \) mit den \( p-T \) diagram auf \(-20^\circ C\) gefunden (Tab \textbf{A-10})

\[
h_{2g} = h_f (T = -20^\circ C) = 231.62 \quad \text{und fast entropie} \quad s_2 = s_f (T = -20^\circ C) = 0.8390
\]

\noindent
Da \( 2 \rightarrow 3 \) is adiabatic reversible \(\implies \Delta s = 0 \implies s_2 = s_3 \)

\noindent
aus Tabelle \textbf{A-12} finden wir \( h_3 \) bei \( s_3 \)

\[
h_3 = 273.66 + \left( \frac{(298.33 - 273.66)}{(891.11 - 0.8390)} \right) (0.8390 - 0.5930) \approx 279.14
\]

\[
\implies \dot{m} = \frac{\dot{W}_k}{h_3 - h_4} = \frac{2.6}{47.55} = 0.055 \, \text{kg/s} \approx 2.37 \, \text{kg/h}
\]

\noindent
\( T_1 \) war mit den binären und \(-20^\circ C\) gefunden \(\implies T_2 = -20 - 6 = -26^\circ C \)

``````latex


\section*{d)}
Den Leistungszahl ist
\[
\epsilon_K = \frac{\left| \dot{Q}_K \right|}{\left| \dot{W}_{\text{tot}} \right|} = \frac{\left| \dot{Q}_K \right|}{\left| \dot{W}_K \right|}
\]

\section*{e)}
Die Temperatur wurde höher werden, da das Kreislaus normal sein wurde, und der Umwälzstrom wurde zu helfen um es kälter zu machen.

```