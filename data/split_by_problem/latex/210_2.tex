
``````latex


\section*{2. a)}

\begin{description}
    \item[Graph 1:] The first graph is a plot with the vertical axis labeled \( T \) [K] and the horizontal axis labeled \( S \) \(\left[\frac{kJ}{kg \cdot K}\right]\). The graph contains several curves:
    \begin{itemize}
        \item A blue curve that starts from the origin, rises to a peak, and then falls back down symmetrically.
        \item Several intersecting lines of different slopes crossing through the blue curve.
    \end{itemize}
    
    \item[Graph 2:] The second graph is a plot with the vertical axis labeled \( T \) [K] and the horizontal axis labeled \( S \) \(\left[\frac{kJ}{kg \cdot K}\right]\). The graph contains several curves and points:
    \begin{itemize}
        \item A series of curves that form a zigzag pattern, labeled with numbers 1 through 6 at each turning point.
        \item A line labeled \( P_0 \) that intersects the zigzag pattern.
        \item Two arrows indicating different processes:
        \begin{itemize}
            \item An arrow labeled \( s = \text{const} \) pointing horizontally to the right.
            \item An arrow labeled \( s \neq \text{const} \) pointing upwards and to the right.
        \end{itemize}
    \end{itemize}
\end{description}

\noindent Below the second graph, there is a handwritten note: \\
\textit{Zustand 4 liegt nicht per se auf der ps Isobare}

``````latex


\begin{itemize}
    \item[(b)] $1G \quad c_p = 1.006 \frac{J}{kgK}, \quad n = k = 1.4$
\end{itemize}

\textbf{Gegeben:} $w_e$ und $T_s \quad p_e = p_0 = 0.191 \, bar = 19.1 \, kPa$

$w_s = 220 \, \frac{m}{s} \quad p_s = 0.5 \, bar \quad T_s = 431.8 \, K$

\[
\Rightarrow T_c \, \text{über Adiabatenkoeff}
\]

\[
\frac{T_c}{T_s} = \left( \frac{p_c}{p_s} \right)^{\frac{n-1}{n}} \Rightarrow T_c = T_s \left( \frac{p_c}{p_s} \right)^{\frac{0.4}{1.4}}
\]

\[
T_c = 431.8 \, K \cdot \left( \frac{0.191 \, bar}{0.5 \, bar} \right)^{\frac{0.4}{1.4}} = 328.07 \, K
\]

\underline{$w_e$:}

\textbf{St. Fließ Prozess am adiabaten reversiblen Düse}

\[
0 = \dot{m} \left[ h_s - h_e + \frac{(w_s)^2 - (w_e)^2}{2} + \frac{p_e}{\rho_e} - \frac{p_s}{\rho_s} \right]
\]

\[
0 = h_s - h_e + \frac{w_s^2 - w_e^2}{2}
\]

\[
1G: \quad h_s - h_e = c_p \cdot \Delta T = c_p (T_s - T_c)
\]

\[
c_p (T_s - T_c) = \frac{w_s^2}{2} - \frac{w_e^2}{2}
\]

\[
c_p (T_s - T_c) + \frac{(w_e)^2}{2} = \frac{w_s^2}{2}
\]

\[
w_e^2 = 2 c_p (T_s - T_c) + (w_s)^2
\]

\[
w_e = \sqrt{2 \cdot 1.006 \left( 431.8 - 328.1 \right) + 220^2} \, \frac{m}{s}
\]

\[
w_e \approx \sqrt{\boxed{}} = \underline{507.2 \, \frac{m}{s}}
\]

``````latex


\section*{2. c)}

\begin{align*}
\Delta e_{x,str} &= e_{x,str6} - e_{x,h0} \\
e_{x,str6} &= h_6 - h_0 - T_0 (s_6 - s_0) \\
e_{x,str6} - e_{x,h0} &= h_6 - h_0 - T_0 (s_6 - s_0) = \Delta \dot{e}_{x,str}
\end{align*}

\[
\frac{h_6}{h_0} = \frac{(h_6 - h_0)}{c_p (T_6 - T_0)}
\]

\[
s_6 - s_0 = c_p \ln \left( \frac{T_6}{T_0} \right) - R \ln \left( \frac{p_6}{p_0} \right) \quad \text{, } p_6 = p_0
\]

\[
\Rightarrow \Delta \dot{e}_{x,str} = c_p \left( T_6 - T_0 - T_0 \ln \left( \frac{T_6}{T_0} \right) \right)
\]

\[
= 1.006 \cdot \left( 340 \text{K} - 243.15 \text{K} - 243.15 \text{K} \cdot \ln \left( \frac{340 \text{K}}{243.15 \text{K}} \right) \right) \frac{\text{kJ}}{\text{kg K}}
\]

\[
= \underline{115.42 \frac{\text{kJ}}{\text{kg}}}
\]

\section*{d)}

\[
e_{x,verl} = \Delta e_{x,str} + \frac{\dot{E}_x Q}{\dot{m}} = \Delta e_{x,str} + e_{x,q}
\]

\[
e_{x,q} \text{ im Brennkammer:}
\]

\[
e_{x,q} = \left( 1 - \frac{243.15 \text{K}}{1289 \text{K}} \right) \cdot 9 \text{B}
\]

\[
e_{x,q} = 969.58 \frac{\text{kJ}}{\text{kg}}
\]

\[
\Rightarrow e_{x,verl} = 100 \frac{\text{kJ}}{\text{kg}} + 969.58 \frac{\text{kJ}}{\text{kg}} = 1069.58 \frac{\text{kJ}}{\text{kg}}
\]

``````latex


