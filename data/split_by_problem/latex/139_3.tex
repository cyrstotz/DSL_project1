
``````latex


3(2)

a) \quad \rho V = mRT

\[
m_{00} = \frac{\rho V_{1}}{R T_{1}} = \frac{\frac{E}{A} \cdot V_{1}}{R T_{1}} = \frac{\left( \frac{32,1 \cdot 40080}{8,314} \cdot 3,14 \right)}{50 \cdot (500 + 273,15)} = 9,982539 \approx 9,983 \, \text{g}
\]

\[
\rho = \frac{E}{A} = \frac{F_{K} + E_{W}}{A \cdot r_{2}} = \frac{4087,059939}{40080} \approx 4087 \, \frac{N}{m^2}
\]

b) \quad \text{Monobrom ist Wärmeübertragend, somit hat das EW das gleiche Temperatur.}

\text{Das gleiche gilt für den Druck, denn Monobrom ist isotherm.}

\[
\rho_{2} = \frac{F_{K}}{r_{2}} = \frac{32 \, \text{kg} \cdot 9,8065 \, \frac{m}{s^2}}{0,005 \, \text{m} \cdot \pi} = 3955,9987 \, \frac{N}{m^2} \approx 3956 \, \frac{N}{m^2}
\]

c) \quad \text{Zerbrochenes System}

\[
dE = \sum Q_{i} - \sum W_{n}
\]

\[
E = U + \cancel{KE} + \cancel{PE} \quad \left\{ \begin{array}{l}
\Delta U = Q_{12} - W_{n} \\
Q_{12} = \Delta U + W_{n}
\end{array} \right.
\]

d) \quad E = U + KE + PE

``````latex


