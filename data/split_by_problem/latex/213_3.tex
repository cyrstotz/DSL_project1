
``````latex


3. a)

\[
R_g = \frac{R}{M_g} = 166.289 \frac{J}{kg \cdot K}
\]

\[
A = \pi \left( \frac{D}{2} \right)^2 = \pi \frac{D^2}{4} = 78.5338 \, \text{cm}^2 = 78.5338 \cdot 10^{-4} \, \text{m}^2
\]

\text{Kräftegleichgewicht}

\[
(m_{EW} + m_K) g + p_0 \cdot A = p_{SU} \cdot A
\]

\[
\frac{(m_{EW} + m_K) g}{A} + p_0 = p_{g,1} \quad \Rightarrow \quad \frac{32.1 + 9.31}{A} + 1 \, \text{bar} = 1.40094 \, \text{bar}
\]

\text{Ideales Gasgesetz:}

\[
p_{SU} V_{SU} = m_g R T_{g,1}
\]

\[
\frac{p_{SU} V_{SU}}{R \cdot T_{SU}} = m_g = 3.4215 \, \text{g}
\]

b)

\[
T_{SU,2} = T_{EW,1} = 0^\circ \text{C}
\]

\[
p_{g,1} = p_{g,2} \approx 1.40094 \, \text{bar}
\]

\text{Es herrscht thermisches Gleichgewicht in Zustand 2.}

\text{Es herrscht weiterhin Kräftegleichgewicht.}

``````latex


3.c) Perfektes Gas \& Isobar:

\[
m(h_2 - h_1) = Q_{12}
\]

\[
m(C_v + R) \Delta (T_2 - T_1) = Q_{12}
\]

\[
Q_{12} = -1.367 \, \text{kJ}
\]

\textbf{Systemgrenze Gas:}

There is a rectangular box divided into two horizontal sections. The upper section is labeled "EW" and the lower section is labeled "Gas". An arrow labeled "Q_{12}" points upwards from the lower section (Gas) to the upper section (EW).

d) 

\[
m_{EW} (u_2 - u_1) = Q_{12}
\]

\[
u_1 = u_f(0^\circ C) + x (u_{fg}(0^\circ C) - u_f(0^\circ C))
\]

\[
= -0.085 + 0.6 (-333.858 + 0.085) = -200.0928 \, \text{kJ}
\]

\[
u_2 = \frac{Q_{12}}{m_{EW}} + u_1 = -186.448 \frac{\text{kJ}}{\text{kg}}
\]

\[
u_2 = u_f + x_2 (u_{fg} - u_{gf})
\]

\[
\frac{u_2 - u_f}{u_{fg} - u_f} = x_2 = \frac{-186.448 + 0.085}{-333.858 + 0.085} = 0.559
\]

``````latex


