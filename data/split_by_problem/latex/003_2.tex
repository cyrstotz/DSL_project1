
``````latex


\section*{Aufgabe 2}

\subsection*{a)}

\begin{equation*}
\frac{\partial E}{\partial t} = \sum_i \dot{m}_i (h_i(t) + \frac{v_i^2}{2} + g z_i(t)) + \sum_j \dot{Q}_j(t) - \sum_n \dot{W}_{in}(t)
\end{equation*}

\begin{equation*}
0 = m_{ges} \left[ h_5 - h_6 + \frac{w_5^2}{2} - \frac{w_6^2}{2} \right] - \int_1^2 \rho \, dV
\end{equation*}

\begin{equation*}
h_5 - h_6 = \int_{T_6}^{T_5} c_p(T) \, dT = c_p \cdot (T_5 - T_6)
\end{equation*}

\begin{equation*}
m_{ges}
\end{equation*}

\subsection*{Graph Description}

The graph is a plot with the x-axis labeled as $s$ and the y-axis labeled as $T$. The graph contains several curves and points:

- There is a curve starting from the bottom left, rising to a peak, and then falling again, labeled as $NS$.
- There are six points labeled 1 through 6.
- Point 1 is at the bottom left.
- Point 2 is above point 1, connected by a curve labeled "isentrop".
- Point 3 is to the right of point 2, connected by a curve labeled "reversible adiabatic".
- Point 4 is above point 3, connected by a curve labeled "isotherm".
- Point 5 is to the right of point 4, connected by a curve labeled "p_0 = p_6".
- Point 6 is below point 5, connected by a curve labeled "isotherm".

The graph also includes arrows indicating the direction of the processes between the points.

``````latex


\section*{Student Solution}

\subsection*{(c)}
\[
\dot{E}_{\text{ex,sr}} = \dot{m} \, e_{\text{ex,sr}} = \dot{m} \left[ h - h_0 - T_0 \cdot (s - s_0) + \frac{v^2}{2} + \frac{pe}{\rho} \right]
\]

\subsection*{(d)}
\[
\dot{E}_{\text{w,ex}} = T_0 \dot{S}_{\text{ez}}
\]

``````latex


