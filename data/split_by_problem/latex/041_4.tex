
``````latex


\section*{Aufgabe 4}

\subsection*{a)}

\begin{itemize}
    \item The first graph is a pressure ($p$) vs. temperature ($T$) diagram. The $p$-axis is vertical and the $T$-axis is horizontal. The graph shows a complex curve with multiple peaks and valleys, and a straight line intersecting the curve.
    \item The second graph is also a pressure ($p$) vs. temperature ($T$) diagram. The $p$-axis is vertical and the $T$-axis is horizontal. The graph shows three distinct regions labeled "fest" (solid), "flüssig" (liquid), and "Gas" (gas). There is a point labeled $T_P$ on the curve separating the solid and liquid regions, and another point labeled $T_i$ on the curve separating the liquid and gas regions.
\end{itemize}

\subsection*{b)}

\begin{align*}
    &2 \rightarrow 3 \quad \text{isotrop} \\
    &0, \text{sich. FP} \quad \text{adiabar} \\
    &\frac{d}{dt} = \dot{m}_i [h_i] + \sum \dot{Q} = \sum \dot{W} \\
    &0 = \dot{m}_{R134a} [h_2 - h_3] + \dot{W}_K \\
    &\dot{m}_{R134a} = \frac{\dot{W}_K}{h_2 - h_3} \\
    &h_2 = h_g(T_2) \quad T_2 = ? \\
    &h_3 = h_f(8 \text{bar}) + x (h_g(8 \text{bar}) - h_f(8 \text{bar})) \quad x = ? \\
    &\text{isotrop} \Rightarrow s_2 = s_3 \quad s_2 = s_g(T_2)
\end{align*}

``````latex


\section*{Student Solution}

\[
X_3 = \frac{S_3 - S_f}{S_g - S_f} \quad \text{(bei 8 bar)}
\]
\[
S_3 = S_2 \quad S_2 = S_g(T_2)
\]
\[
T_2 = ?
\]

\text{brauche } T_2, \text{ aber wie mach ich das?}

\subsection*{c)}
\[
p_4 = p_3 = 8 \text{ bar}
\]
\text{adiabate Drossel} \(\Rightarrow\) \text{isotherm} \(\quad T_3 = T_4\)
\[
X_4 = 0 \Rightarrow \text{gesättigt bei 8 bar}
\]
\[
T_u = 31,33^\circ C \quad \text{TABA-11} \quad \text{liquid-vapor at 8 bar}
\]

\subsection*{d)}
\[
\epsilon_K = \frac{|\dot{Q}_{zu}|}{|\dot{W}_{el}|} = \frac{|\dot{Q}_{zu}|}{|\dot{Q}_{ab}| - |\dot{Q}_{zu}|} = \frac{|\dot{Q}_{k}|}{|\dot{Q}_{ab}| - |\dot{Q}_{k}|}
\]
\[
=
\]

```