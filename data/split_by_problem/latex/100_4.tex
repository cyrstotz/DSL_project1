
``````latex


\section*{Nr. 4}

\begin{tabular}{|c|c|c|c|c|c|}
\hline
 & $p \, [\text{bar}]$ & $T \, [K]$ & $Q$ & $W$ & $x$ \\
\hline
1 & 1210 & & & & \\
 & $p = p_2$ & & & & \\
\hline
2 & $p = p_2$ & $T_i = \frac{-6 / (-22^\circ C)}{277.15 K}$ & & & \\
\hline
3 & 8 & & & 28 W & \\
\hline
4 & $p = p_4$ & 8 & & & \\
\hline
\end{tabular}

\begin{itemize}
    \item $s_1 = s_4$
    \item $h_1 = h_4$
    \item $s_2 = s_3$
    \item $s_3 = s_5$
\end{itemize}

\begin{itemize}
    \item $s_4 = s_4$
    \item $h_2 = h_4$
    \item $93,42 \, \frac{kJ}{kg}$
\end{itemize}

\begin{itemize}
    \item Tripelpunkt
    \item $x_4 = 0$
\end{itemize}

\begin{itemize}
    \item $T_{AB}$
    \item $h_4 = 93,42 \, \frac{kJ}{kg}$
\end{itemize}

\begin{itemize}
    \item Tripelpunkt $= 10 K$ in der Sublimationsspalte $= 10$
\end{itemize}

\[
T_1 = T_{\text{tripel}} + 10 K = 283,15 K
\]

\[
T_2 = T_{\text{tripel}} + 10 K = 283,15 K - 6 K \quad T_2 = 277,15 K
\]

\begin{itemize}
    \item nehme $T_2$ aus $t_2 = -22^\circ C$
\end{itemize}

\[
s_2 = 0,9351 \, \frac{kJ}{kg \cdot K} \quad h_2 = 25,18 \, \frac{kJ}{kg}
\]

\[
s_3 = 0,5351 \, \frac{kJ}{kg \cdot K}
\]

``````latex


\section*{Student Solution}

\subsection*{Part (a)}

\begin{align*}
    h_3 & \text{ inter polin} \\
    p &= 8 \text{ bar} \\
    h_3 &= h(sat) + \frac{h(40) - h(sat)}{s(r0) - s(sat)} \cdot (s s_2 - s(sat)) \\
    h_3 &= 272,95
\end{align*}

\subsection*{Part (b)}

\begin{align*}
    -W_{12} &= m(h_2 - h_3) \\
    \dot{m} &= \frac{-W_{12}}{h_2 - h_3} = 1,52 \frac{\text{kg}}{\text{s}}
\end{align*}

\subsection*{Part (c)}

\begin{align*}
    p_{vk} &= p_1 = 1,2 \cdot 10 \text{ and } T A 10 \\
    h_1 &= h_{ey} \\
    x &= 1,2 \\
    h_u &= 95,42 \frac{\text{kJ}}{\text{kg}} \\
    x &= \frac{h_4 - h_f}{h_g - h_f} = 0,3 \text{ or } 0,39
\end{align*}

\subsection*{Part (d)}

\begin{align*}
    e_{k} &= \frac{\dot{Q}_{zu}}{\dot{Q}_{ab} - \dot{Q}_{zu}} = \frac{Q_{zu}}{W_E} = \frac{Q_K}{W_k} \\
    Q_{k} &= m(h_2 - h_1) \\
    Q_{c} &= m \cdot \frac{G(40)}{60^2 \text{s}} \cdot (251,8 - 95,42) \\
    &= 0,76 \text{ kJ}
\end{align*}

``````latex


\begin{figure}[h]
    \centering
    % Description of the graph
    The graph is a plot on a grid paper with the horizontal axis labeled as \( T \) and the vertical axis labeled as \( P \). The graph contains a curve that starts from the bottom left, rises to a peak labeled \( T_{\text{krit}} \), and then descends towards the bottom right. There are four points marked on the graph, labeled 1, 2, 3, and 4. 
    
    - Point 1 is located on the left side of the curve, below the peak.
    - Point 2 is on the right side of the curve, at the same height as point 1.
    - Point 3 is above point 2, on a smaller curve that branches off from the main curve.
    - Point 4 is on the left side of the curve, at the same height as point 3.
    
    There is a horizontal line passing through points 1 and 2. Another line branches off from point 2 and goes upwards, passing through point 3. The label "150 bar" is written near the line that passes through point 3.
\end{figure}

``````latex


\[
E = \frac{176 \, \text{kWh}}{28 \, \text{W}} = 6,2857
\]

\begin{itemize}
    \item[(c)] \(\Rightarrow\) Würde das was man versucht zu umdulen, da die Wärme entzogen wird, wird das es schwerer. Es wird die in die Gasphase zu überführen.
\end{itemize}

```