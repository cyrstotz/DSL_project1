
``````latex


\section*{Aufgabe 2}

\subsection*{a) T [K]}
\begin{description}
    \item[Graph Description:] 
    The graph is a plot with the vertical axis labeled \( T \, [K] \) and the horizontal axis labeled \( S \left[ \frac{kJ}{kg \cdot K} \right] \). The plot starts at the origin and initially rises steeply, then levels off to a horizontal line at point 2. From point 2, it rises again to point 3, forming a peak at point 4. After point 4, the graph descends to point 5 and then sharply rises again. The points are labeled as follows:
    \begin{itemize}
        \item Point 2: Located on the horizontal line.
        \item Point 3: Located at the peak of the curve.
        \item Point 4: Located at the descending part of the curve.
        \item Point 5: Located at the sharp rise after the descent.
    \end{itemize}
    There is a horizontal line connecting points 2 and 4, and a vertical line connecting point 2 to the horizontal axis.
\end{description}

\subsection*{b) Result: \( w_e, T_6 \)}

``````latex


\section*{Student Solution}

\subsection*{c)}
\begin{align*}
\text{Weitergabe mit } \dot{m} &= 520 \frac{\text{kg}}{\text{h}} \quad T_0 = 340 \text{K} \\
\Delta e_{\text{ex,sr}} &= h_e - h_a - T_0 (s_e - s_a) + \Delta e
\end{align*}

\subsection*{d)}
\begin{align*}
\text{1. H.P.} \quad 0 &= -\Delta e_{\text{ex,sr}} + e_{\text{ex,2}} - \dot{W}_t - e_{\text{ex,vel}} \\
e_{\text{ex,vel}} &= -100 \frac{\text{kJ}}{\text{kg}} + e_{\text{ex,2}} - \dot{W}_t
\end{align*}

``````latex


