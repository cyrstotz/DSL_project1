
``````latex


\section*{Aufgabe 2: Energie am Triebwerk}

\subsection*{a)}

\begin{description}
    \item[Graph Description:] The graph is a pressure-volume (p-v) diagram with the x-axis labeled as $s \, (\text{kJ}/\text{kg} \cdot \text{K})$ and the y-axis labeled as $T \, (\text{K})$. The graph consists of a closed loop with six points labeled 1 through 6. The process between points 1 and 2 is labeled as "adiabat s=konst reversibel". The process between points 2 and 3 is labeled as "isobar". The process between points 3 and 4 is labeled as "adiabat ir". The process between points 4 and 5 is labeled as "adiabat irreversibel". The process between points 5 and 6 is labeled as "isobar". The process between points 6 and 1 is labeled as "s=konst". The pressures at points 1, 2, 3, 4, 5, and 6 are labeled as $p_0 = p_6 = 0.1916 \, \text{bar}$, $p_2 = p_3$, and $p_5 = 0.5 \, \text{bar}$.
\end{description}

\subsection*{b)}

\begin{align*}
    &\text{aus der Energiebilanz} \\
    &0 = \dot{m} \left[ h_e - h_a + \frac{(w_e^2 - w_a^2)}{2} \right] I + \sum_j \dot{Q}_j - \sum_n \dot{W}_{en} = 0 \\
    &\dot{Q}_j \, \text{irren co.} \\
    &\dot{W}_t \leq 0 \\
    &q = \frac{h_a - h_e}{\dot{m}} \\
    &w_e = w_{inft} > 200 \, \text{m/s} \\
    &\frac{(w_e^2 - w_a^2)}{2} = h_a - h_e - \frac{q}{\dot{m}} \\
    &= C_p (T_a - T_e) - \frac{q}{\dot{m}} \\
    &5 \rightarrow 6 \, \text{isentrop} \rightarrow \frac{T_6}{T_5} = \left( \frac{p_6}{p_5} \right)^{\frac{n-1}{n}} = \left( \frac{p_0}{p_5} \right)^{0.4/1.4} = \left( \frac{0.191}{0.5} \right)^{0.4/1.4} \\
    &= 0.76 \\
    &T_6 = T_5 \cdot 0.76 = 0.76 \cdot 423 \, \text{K} = 318.7 \, \text{K}
\end{align*}

``````latex


\[
\left( \frac{w_5^2 - w_6^2}{2} \right) = C_p (T_6 - T_5)
\]
\[
= 1.006 \frac{\text{kJ}}{\text{kg} \cdot \text{K}} (328.07 - 431.91) \text{K}
\]
\[
= -104.45 \frac{\text{kJ}}{\text{kg}}
\]
\[
w_6 = \sqrt{2 \cdot (-104.45 \frac{\text{kJ}}{\text{kg}}) + w_5^2}
\]
\[
= 510 \frac{\text{m}}{\text{s}}
\]

\[
T_6 = 340 \text{K}, \quad w_6 = 510 \frac{\text{m}}{\text{s}}
\]

\[
\Delta E_{x, \text{is}} = h_6 - h_0 - T_0 (s_6 - s_0) + \frac{w_6^2 - w_0^2}{2}
\]
\[
= C_p (T_6 - T_0) - T_0 C_p \ln \left( \frac{T_6}{T_0} \right) - R \ln \left( \frac{p_6}{p_0} \right) + \frac{w_6^2 - w_0^2}{2}
\]

\[
\Phi = \frac{C_p - C_v}{C_v} = \frac{C_p}{T} = 1.006 - \frac{1.006}{1.4} = 0.287
\]
\[
T_0 = 273.15 - 30 = 243.15 \text{K}
\]

\[
\Delta E_{x, \text{is}} = 1.006 \left( 340 - 243.15 \right) + 243.15 \cdot 1.006 \ln \left( \frac{340}{243.15} \right) - 0.287 \ln \left( 1 \right) + \frac{200^2 - 510^2}{2}
\]
\[
= -117.8 \frac{\text{kJ}}{\text{kg}}
\]

\[
\Delta E_{x, \text{rev}} = 100 \frac{\text{kJ}}{\text{kg}}
\]

\[
E_{x, \text{rev}} = T_0 S_{\text{rev}} = T_0 C \ln (s_1 - s_0) - \frac{Q_{12}}{T}
\]

\[
\dot{m} = \frac{E_{x, \text{rev}}}{T_0 (s_1 - s_0)} = \frac{E_{x, \text{rev}}}{T_0 \cdot C_p \ln \left( \frac{T_6}{T_0} \right)} = \frac{110 \frac{\text{kJ}}{\text{kg}}}{243.15 \cdot 1.006 \ln \left( \frac{340}{243.15} \right)} = 1.27 \frac{\text{kg}}{\text{s}}
\]

``````latex


