
``````latex


\section*{Aufgabe 2}

\subsection*{a)}

\begin{tabular}{c|c|c|c|c}
Zustand & $p \, [\text{bar}]$ & $T \, [^\circ \text{C}]$ & $W \, [\frac{\text{kJ}}{\text{kg}}]$ & $s \, [\frac{\text{kJ}}{\text{kg K}}]$ \\
\hline
0 & & & & \\
1 & 0,191 & & & \\
2 & & 243,15 & & \\
3 & & & & \\
4 & & & & \\
5 & 0,5 & 431,9 & 220 & \\
6 & & & & \\
\end{tabular}

\[
P_1 = P_5
\]

\[
s_1 = s_2
\]

\[
s_2 = s_1
\]

\[
s_5 = s_6
\]

\[
s_6 = s_5
\]

\subsection*{Graphical Description}

The graph is a Temperature-Entropy ($T$-$s$) diagram. The x-axis is labeled $s \, [\frac{\text{kJ}}{\text{kg K}}]$ and the y-axis is labeled $T \, [K]$. 

- There are two isobars drawn: one at 0.5 bar and another at 0.191 bar.
- The isobar at 0.5 bar is drawn as a curve starting from the bottom left and curving upwards to the right.
- The isobar at 0.191 bar is drawn similarly but is positioned to the left of the 0.5 bar isobar.
- Points 1, 2, 3, 4, 5, and 6 are marked on the graph.
- Point 1 is at the intersection of the 0.191 bar isobar and a vertical line.
- Point 2 is slightly to the right of Point 1 on the same isobar.
- Point 3 is on the 0.5 bar isobar, directly above Point 2.
- Point 4 is to the right of Point 3 on the same isobar.
- Point 5 is on the 0.191 bar isobar, directly below Point 4.
- Point 6 is to the right of Point 5 on the same isobar.
- Arrows indicate the transitions between these points: from 1 to 2, 2 to 3, 3 to 4, 4 to 5, and 5 to 6.

\subsection*{b)}

Energiebilanz um die Schubdüse:

\[
0 = \dot{m} \left[ h_5 - h_6 + \frac{w_5^2 - w_6^2}{2} \right]
\]

\[
h_5 - h_6 = \dots
\]

\[
\Rightarrow w_6 = \sqrt{2 \left( h_5 - h_6 \right) + w_5^2}
\]

\[
= \sqrt{2 \cdot 104,5 \frac{\text{kJ}}{\text{kg}} + (220 \frac{\text{m}}{\text{s}})^2} = 220,57 \frac{\text{m}}{\text{s}}
\]

``````latex


\section*{Student Solution}

\subsection*{c)}

\begin{align*}
\Delta e_{x,\text{istr}} &= \frac{\dot{E}_{x,\text{istr}}}{\dot{m}_{ges}} = \left[ h_6 - h_0 - T_0 (s_6 - s_0) \right] \\
&= \\
s_6 - s_0 &= s^0(T_6) - s^0(T_0) - R \ln \left( \frac{P_6}{P_0} \right) \\
&= 0 \\
h_6 - h_0 &= 189.88 \frac{\text{kJ}}{\text{kg}} \quad (\text{aus Teilaufgabe b})
\end{align*}

\subsection*{d)}

\begin{align*}
e_{x,\text{verl}} &= \frac{\dot{E}_{x,\text{verl}}}{\dot{m}_{ges}} = T_0 \dot{S}_{e, \text{verl}} \\
\end{align*}

\text{Entropiebilanz:} \quad 0 = \dot{m}_g s_0 - \dot{m}_k s_6 + \frac{\dot{Q}}{T_0} + \dot{S}_{e, \text{verl}}

\begin{align*}
\dot{S}_{e, \text{verl}} &= \frac{\dot{m}_g}{\dot{m}_{ges}} s_0 - s_0 \\
&= \\
\end{align*}

\text{Massenstrombilanz am Verdichter:} \quad \dot{m}_g (h_0 - h_{ns}) = \dot{m}_k (h_0 - h_1)

\begin{align*}
\Rightarrow \eta_{ns} &= \frac{h_0 - h_{ns}}{h_0 - h_1} = \frac{\dot{m}_g}{\dot{m}_k}
\end{align*}

\subsection*{b)}

\begin{align*}
\frac{T_6}{T_5} &= \left( \frac{P_6}{P_5} \right)^{\frac{n-1}{n}} \\
C_p &= 1.006 \frac{\text{kJ}}{\text{kg K}} \\
n &= k = 1.4 \\
\Rightarrow T_6 &= T_5 \left( \frac{P_6}{P_5} \right)^{\frac{n-1}{n}} = 437.9 \text{K} \left( \frac{0.191}{0.15} \right)^{\frac{1-1.4}{1.4}} = 328 \text{K} \\
\Rightarrow h_5 - h_6 &= C_p (T_5 - T_6) = 104.95 \frac{\text{kJ}}{\text{kg}}
\end{align*}

``````latex


