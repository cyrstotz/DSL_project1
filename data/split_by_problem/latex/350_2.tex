
``````latex


\section*{Problem 2a}

\begin{figure}[h]
    \centering
    % Detailed verbal description of the graph
    The graph is a Temperature-Entropy (T-S) diagram with the x-axis labeled as $S \left( \frac{kJ}{kg \cdot K} \right)$ and the y-axis labeled as $T \left( ^\circ C \right)$. The graph contains several curves and points labeled from 0 to 6. The points are connected by different types of processes, which are labeled as follows:
    
    - Point 0 to Point 1: Labeled as "Kompression" and "isotrop".
    - Point 1 to Point 2: Labeled as "isotrop" and "adiabat".
    - Point 2 to Point 3: Labeled as "Isobar".
    - Point 3 to Point 4/5: Labeled as "isobar" and "adiabat".
    - Point 4/5 to Point 6: Labeled as "adiabat, reversibel".
    
    The graph also includes the following labels:
    
    - Between Point 0 and Point 1: $p_0$
    - Between Point 1 and Point 2: $p_1$
    - Between Point 2 and Point 3: $p_2$
    - Between Point 3 and Point 4/5: $p_3$
    
    The graph shows the different thermodynamic processes and their respective paths on the T-S diagram.

\end{figure}

``````latex


\section*{Problem 2}

\subsection*{a)}

\begin{itemize}
    \item The first graph is a $T$ vs $s$ diagram. The $x$-axis is labeled $s \left[\frac{kJ}{kg \cdot K}\right]$ and the $y$-axis is labeled $T \left[K\right]$. There are several curves drawn, representing different isobars. Points 5 and 6 are marked on the graph, connected by a line. The isobar for $0.5 \, \text{bar}$ is labeled.
    
    \item The second graph is a $T$ vs $s$ diagram. The $x$-axis is labeled $s \left[\frac{kJ}{kg \cdot K}\right]$ and the $y$-axis is labeled $T \left[K\right]$. There are several curves drawn, representing different isobars. Points 1, 5, and 6 are marked on the graph, connected by lines. The isobar for $p_0$ is labeled.
    
    \item The third graph is a 3D plot with $T$, $s$, and $p$ axes. Points 1, 2, 5, and 6 are marked on the graph, connected by lines. The isobars and isotherms are labeled.
    
    \item The fourth graph is a $T$ vs $s$ diagram. The $x$-axis is labeled $s \left[\frac{kJ}{kg \cdot K}\right]$ and the $y$-axis is labeled $T \left[K\right]$. There are several curves drawn, representing different isobars. Points 0, 1, 2, 4, and 5 are marked on the graph, connected by lines. The isobar for $p_0 = 0.191 \, \text{bar}$ is labeled.
\end{itemize}

\subsection*{b)}

\textbf{Energiebilanz am 5-6:}

\[ T_6 \text{ bestimmen via Polytropenexponent} \]

\[ p_6 - p_0 = 0.191 \, \text{bar} \]

\[ T_6 = T_5 \left(\frac{p_6}{p_5}\right)^{\frac{1.9-1}{1.9}} = 328.07 \, K = T_6 \]

\textbf{Energiebilanz über 5-6; Stat FP:}

\[ m \left( h_5 - h_6 + \frac{220^2 - u_6^2}{2} \right) = 0 \quad \text{in streichen} \]

\[ h_5 - h_6 + \frac{220^2}{2} - \frac{u_6^2}{2} = 0 \]

\[ u_6 = \sqrt{2 \cdot 128.653} \]

\[ u_6 = 507.25 \, \frac{m}{s} \]

\[ h_5 - h_6 + \frac{220^2}{2} = \frac{u_6^2}{2} \]

\[ 1.006 \cdot 1000 \cdot (431.9 - 328.07) + \frac{220^2}{2} = \frac{u_6^2}{2} = 128.651.98 \]

``````latex


\section*{Problem c}

\begin{align*}
e_{x,5\rightarrow 6} - e_{x,5\rightarrow 0} &= \dot{m} \left( h_6 - h_0 - T_0 (s_6 - s_0) + \frac{ke}{\dot{m}} \right) \\
h_6 - h_0 &= c_p (T_6 - T_0) = 1.006 \frac{kJ}{kgK} (328.07 - 30) \\
&= 83.43 \frac{kJ}{kg} \\
T_0 (s_6 - s_0) &= 293.15 K \left( c_p \ln \left( \frac{T_6}{T_0} \right) - R \ln \left( \frac{v_1}{v_0} \right) \right) \\
&= 293.15 \left( 1.006 \ln \left( \frac{328.07}{293.15} \right) \right) = 73.27 \frac{kJ}{kg} \\
\Delta e_x &= \dot{m} (85.43) \\
ke &= \frac{\omega_6^2 r^2}{2} = \frac{50^2 \cdot 2.5^2}{2} = 128857.3 \, W \\
\text{Total } \Delta E_{x,1\rightarrow 6} &= -85.139 + 73.27 + 128.65 \, (kW) \\
&= 140.82 \, kW = \Delta E_x
\end{align*}

\section*{Problem d}

\text{Exergetic balance for turbine: steady state, adiabatic process}

\begin{align*}
0 &= \dot{m} \Delta E_{x,1\rightarrow 6} + \left( 1 - \frac{T_0}{T} \right) \dot{Q} - \dot{E}_{x,\text{vel}} \\
\left( \frac{1}{\dot{m}} \right) \dot{m} \Delta E_{x,\text{str}} &= \Delta e_{x,\text{str}} + \left( 1 - \frac{T_0}{T} \right) \dot{Q} \\
\dot{E}_{x,\text{vel}} &= 100 \frac{kJ}{kg} + \left( 1 - \frac{293.15}{1289} \right) 1.195 \frac{kJ}{kg} = 1009.58 \frac{kJ}{kg}
\end{align*}

\text{Note: Calculation continued with given values as they are uncertain.}

``````latex


