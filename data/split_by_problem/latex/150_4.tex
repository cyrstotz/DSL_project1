
``````latex


\section*{4a)}

\begin{center}
\textbf{Graph Description:}
\end{center}

The graph is a Pressure-Temperature ($P$-$T$) diagram. The x-axis is labeled $T$ (Temperature) and the y-axis is labeled $P$ (Pressure). There is a curve starting from the origin and rising upwards, which is labeled "Fest" (solid). At a certain point on the curve, a horizontal line extends to the right, labeled "isotherm". From the end of this horizontal line, a vertical line extends upwards, labeled "isobar". The intersection of the isotherm and isobar lines is labeled "Tripel" (triple point). Above the isotherm line and to the right of the isobar line, the region is labeled "flüssig" (liquid). Below the isotherm line and to the right of the curve, the region is labeled "gasförmig" (gaseous).

\begin{center}
\textbf{Abb. 5}
\end{center}

\section*{4b)}

\begin{align*}
& s_2 = s_3 \\
& T_i = -10^\circ C \\
& T_{\text{ver}} = -16^\circ C \quad / = T_2 \\
& h_2 = 237.74 \frac{\text{kJ}}{\text{kg}} \\
& s_2 = s_3 = 0.8289 \frac{\text{kJ}}{\text{kg K}} \\
& h_3 = \frac{(273.66 - 264.75) \frac{\text{kJ}}{\text{kg}}}{(0.8374 - 0.8066) \frac{\text{kJ}}{\text{kg K}}} (0.8289 - 0.8066) \frac{\text{kJ}}{\text{kg K}} + 264.75 \frac{\text{kJ}}{\text{kg}} \\
& = 271.31 \frac{\text{kJ}}{\text{kg}} \\
& 0 = \dot{m}_{R12} (h_2 - h_3) + \dot{W}_H \implies \dot{m}_{R12} = \frac{-\dot{W}_H}{h_2 - h_3} = 3.00 \frac{\text{kg}}{\text{h}}
\end{align*}

``````latex


\section*{4c}

\begin{itemize}
    \item[(a)] \text{(Graphical content: A diagram with several arrows and labels, including $Q_{zu}$, $Q_{ab}$, $W_{t}$, and $W_{v}$, indicating various energy flows and work done in a thermodynamic system.)}
\end{itemize}

\begin{itemize}
    \item[(b)] Die Temperatur würde abnehmen, da durch den abgeführten Wärmestrom an Energie dem System entzogen wird. (Bei konstantem Volumen)
\end{itemize}

\begin{itemize}
    \item[(c)] 
    \begin{align*}
        h_4 &= 93.42 \frac{kJ}{kg} \\
        0 &= \dot{m}_{12} (h_4 - h_7) \quad \text{(exdibt)} \\
        0 &< \dot{m}_{12} (h_4 - h_7) \\
        \Rightarrow h_7 &= h_4 = 93.42 \frac{kJ}{kg} \\
        h_{f} \text{ at } -76^\circ C \quad h_{g} \text{ at } -76^\circ C \\
        h_7 &= (1 - x) h_f + x h_g \Rightarrow h_f - x h_f + x h_g \\
        \Rightarrow x (h_f + h_g) &= h_7 - h_f \Rightarrow x = \frac{h_7 - h_f}{-h_f + h_g} = 0.308
    \end{align*}
\end{itemize}

``````latex

4d)

\[
0 = \dot{m}_{\text{ab}} (h_1 - h_2) + Q_k
\]

\[
Q_k = -\dot{m}_{\text{ab}} (h_1 - h_2) = 0.0002 \times 744.3 \frac{\text{kJ}}{\text{h}} = 120.3 \frac{\text{W}}{}
\]

\[
\epsilon_k = \frac{|Q_k|}{|W_A|} = 4.3
\]

```