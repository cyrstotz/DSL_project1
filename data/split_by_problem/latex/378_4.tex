
``````latex


\section*{Aufgabe 4}

\subsection*{a)}

\begin{description}
    \item[Graph 1:] A graph with the vertical axis labeled \( T \) and the horizontal axis labeled \( p \). The graph contains a square cycle with four points labeled 1, 2, 3, and 4 in a clockwise direction. The cycle starts at point 1 at the bottom left, moves up to point 2, then right to point 3, down to point 4, and left back to point 1.
\end{description}

\begin{description}
    \item[Graph 2:] A graph with the vertical axis labeled \( T \) and the horizontal axis labeled \( p \). The graph contains a triangular cycle with three points labeled. The cycle starts at the bottom left, moves up to the top vertex labeled "isotherm", then down to the bottom right vertex labeled "isotherm", and finally back to the starting point. The left side of the triangle is labeled "adiabat", and the right side is labeled "adiabat". The bottom side is labeled "isochor".
\end{description}

\subsection*{b)}

\[
\frac{d\dot{m}}{dt} = \dot{m} (h_2 - h_3) + \frac{\dot{Q}_{ev}}{T_{ev}} - \dot{V}_k
\]

\[
h_2 = h(p_f, h_g, (s_2 = s_1))
\]

\[
h_3 = h(s_3 = s_2, 8 \text{ bar})
\]

\[
\dot{V}_k = -28 \text{ W}
\]

``````latex

\section*{4}

c) \[ x_a = \frac{S_2 - S_f}{S_3 - S_f} \]

d) \[ q_K = \frac{\dot{Q}_K}{\dot{V}_t} = \frac{\dot{Q}_K}{\dot{V}_t} = \frac{\dot{Q}_K}{28W} \]

\[
\dot{Q}_K \Rightarrow \frac{d\dot{T}}{dT} = \dot{m}(h_2 - h_1) + \dot{Q}_{2v} - \frac{\dot{V}_v}{V}
\]

\[
\dot{Q}_{2v} = \dot{m}(h_2 - h_1)
\]

\[
\Rightarrow \varepsilon_K = \frac{\dot{m}(h_2 - h_1)}{28W}
\]

e)

```