
``````latex


4. (a)

\begin{itemize}
    \item There is a graph with the y-axis labeled as $p$ (pressure) in $\text{[bar]}$ and the x-axis labeled as $T$ (temperature) in $\text{[K]}$. 
    \item The graph shows a curve starting from the origin and rising upwards, representing the saturation line.
    \item The region to the left of the curve is labeled as "Solid", and the region to the right of the curve is labeled as "Fluid".
    \item There are three points marked on the graph: $1$, $2$, and $3$.
    \item Point $1$ is at a higher pressure and temperature, connected to point $2$ by a vertical line labeled "Isobar".
    \item Point $2$ is connected to point $3$ by a horizontal line labeled "Isotherm".
    \item Point $3$ is on the saturation line.
    \item The distance between the points on the x-axis is marked as $10K$.
\end{itemize}

(b)

\begin{align*}
    p_3 &= p_f \\
    T_f &= 31.33^\circ C \\
    p_1 &= p_2 \\
    h_f &= \\
    \text{Prozess isenthalp:} \\
    h_1 &= h_f = h_3 (8 \text{bar}) = 93.42 \frac{\text{kJ}}{\text{kg}} \\
    T_2 &= T_1
\end{align*}

``````latex


\section*{Aufgabe 9.e}

Es würde irgendwann ein thermisches Gleichgewicht eintreffen und der Innenraum die Temperatur im Kondensator einnehmen.

\subsection*{c)}

\[ T_2 = -22^\circ C \]

\[
\dot{m}(h_2 - h_1) + \dot{Q_k} = 0
\]

\[ T_2 = T_1 \]

\[ \implies h_2 = h_1 \]

\[ h_2 = h_f(-22^\circ C) + x \cdot h_{fg}(-22^\circ C) \]

\[
\frac{h_2 - h_f}{h_{fg}} = x_2 = \frac{43.42 - 21.77}{22.32} = 0.3374
\]

\subsection*{d)}

\[ \varepsilon_k = \frac{\dot{Q_k}}{W_k} = 1.27 \cdot 1.35 \]

\[
\dot{Q_k} = \dot{m}_R \cdot (h_2 - h_1)
\]

\[ h_2 = 234.08 \frac{kJ}{kg} \]

\[ \dot{Q_k} = 0.1562 \, kW \]

```