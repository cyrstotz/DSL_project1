
``````latex


\section*{Aufgabe 4}

\subsection*{a)}

\begin{description}
    \item[Graph Description:] The graph is a pressure-temperature ($P$-$T$) diagram. The x-axis is labeled $T$ (temperature) and the y-axis is labeled $P$ (pressure). The graph consists of four points labeled (1), (2), (3), and (4) connected by lines forming a closed loop. The path from (1) to (2) and from (3) to (4) is labeled "isobar" (constant pressure) in orange. The path from (2) to (3) is labeled "isentrop" (constant entropy) in orange. The path from (4) to (1) is a curved line labeled "gesättigte Flüssigkeit" (saturated liquid) and "nassdampfgebiet" (wet steam area) in blue. The point (2) is labeled "P=12".
\end{description}

\subsection*{b)}

\begin{align*}
    &\text{Energieerhaltung um Kompressor} \\
    &0 = \dot{m}_{\text{Kreislauf}} (h_2 - h_3) + \dot{Q}^0 - \dot{W}_{\text{u}} \\
    &\frac{\dot{W}_{\text{u}}}{h_2 - h_3} = \dot{m}_{\text{Kreislauf}} \\
    &h_2 = h_{\text{2ig}}(T_2) \\
    &T_2 = T_1 - 6K \\
    &\text{Mit Bedingung aus ii) lässt sich } T_1 \text{ bei üben das } p-T \text{ Diagramm bestimmen: } T_1 = 10^\circ C \\
    &T_1 = 283.15 K
\end{align*}

``````latex


c) \quad T_2 = -22^\circ \text{C} \quad \quad T_1 = -16^\circ \text{C}

\begin{itemize}
    \item $x_1$
    \item $p_3 = p_4$
    \item $S_3 = S_F \quad (S_{ice}) = C_{3.5} \frac{L_T}{4.5L} \quad \text{TABA}$
    \item $x_2$
    \item $S_2 = S_3$
\end{itemize}

```