
``````latex


\section*{Student Solution}

\subsection*{a)}

\begin{description}
    \item[Graph Description:] The graph is a plot with a vertical axis labeled "Ausgabe" and a horizontal axis labeled "t". The vertical axis has a value of "8" marked near the top. The horizontal axis has a value of "1" marked near the right end. There is a curve that starts at the origin, rises to a peak, and then falls back down. The peak is labeled "Tiefpunkt" and has a value of "-1" marked next to it. There are two points marked on the curve, one labeled "1" and another labeled "2". The point labeled "1" is at the peak of the curve, and the point labeled "2" is slightly to the right of the peak.
\end{description}

\subsection*{b) Energiebilanz}

\begin{equation}
0 = \dot{m} \cdot \left( h_{2} - h_{3} \right) + \dot{Q} - \dot{W}_{k}
\end{equation}

\begin{equation}
h_{v2} = h_{vR134a} \left( p = T = T_{i} = 6K \right) \quad \text{und} \quad x = a
\end{equation}

\begin{equation}
h_{v3} = h_{vR134a} \left( p = 0bar \right)
\end{equation}

\begin{equation}
s_{2} = s_{3}
\end{equation}

\subsection*{c) Energiebilanz}

\begin{equation}
\dot{m} \cdot \left( \dot{h}_{2} - \dot{h}_{3} \right) + \dot{Q} - \dot{W}
\end{equation}

\subsection*{d)}

\begin{equation}
\dot{E}_{k} = \frac{\dot{Q}_{k}}{\dot{W}_{k}}
\end{equation}

``````latex


\section*{Solution}

\subsection*{a)}

```