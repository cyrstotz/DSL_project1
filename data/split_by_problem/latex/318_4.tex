
``````latex


\section*{4. a}

\begin{itemize}
    \item The graph is a pressure-temperature ($p$-$T$) diagram.
    \item The y-axis is labeled as $p$ with the unit in brackets [bar].
    \item The x-axis is labeled as $T$ with the unit in brackets [K].
    \item The graph shows a phase diagram with three regions: solid (Fest), liquid (Flüssig), and gaseous (Gasförmig).
    \item The curve separating the liquid and gaseous regions is labeled as "isotherm".
    \item The curve separating the solid and liquid regions is labeled as "isobar".
    \item The point where all three phases meet is labeled as "Tripel".
    \item Points i and ii are marked on the graph, with i in the liquid region and ii in the gaseous region.
\end{itemize}

\section*{4. b}

\begin{align*}
    \dot{m} \text{ ges} \\
    T_{\text{in Verdampfer}} &= 9^\circ C = T_i \\
    T_{\text{in Verdampfer}} &= 9^\circ C - 6 K = 3^\circ C \\
    0 &= \dot{m} \left[ h_2 - h_3 \right] + \dot{W}_k \\
    \dot{W}_k &= \dot{m} \frac{\dot{W}_k}{h_2 - h_3} \\
    h_3 \text{ aus Tabelle A-12 über interpolieren} \\
    y &= \frac{x - x_1}{x_2 - x_1} (y_2 - y_1) + y_1 \\
    h_2 \text{ bei } T_2 &= \text{ bei } -22^\circ C \text{ vollständig verdampft} \\
    h_2 \text{ bei } -22^\circ C &= 234.08 \frac{kJ}{kg} \\
    \text{Interpolation zwischen 0°C und 6°C} \\
    y &= \frac{x - x_1}{x_2 - x_1} (y_2 - y_1) + y_1 \\
    y &= 2.94 \frac{kJ}{kg} \\
    y_2 &= 247.23 \frac{kJ}{kg} \\
    x_1 &= 0^\circ C \\
    x_2 &= 6^\circ C \\
    h_3 &= 
\end{align*}

``````latex


\section*{4.c}

\[
T_2 = -22^\circ C
\]

\[
S_2 = S_f + x \left( S_{g} - S_{f} \right)
\]

\[
S = S_f
\]

\[
S - S_f = x \left( S_{g} - S_{f} \right)
\]

\[
x = \frac{S - S_f}{S_{g} - S_f}
\]

\[
S_3 = S_2 \quad \text{da: Verdichtung isentrop}
\]

\[
S_2 \text{ herausfinden durch Interpolation zwischen 20 an}
\]

\[
S_2 = 0.9351 \frac{kJ}{kg \cdot K}
\]

\[
Drossel = \text{isenthalp}
\]

\[
u_7 = u_2
\]

\[
u_2 = 93.42 \frac{kJ}{kg}
\]

\[
u = u_f + x \left( u_{g} - u_{f} \right)
\]

\[
x = \frac{u - u_f}{u_{g} - u_f}
\]

\[
u_f \text{ und } u_g \text{ bei Ausgangsdruck}
\]

\section*{d)}

\[
ek = \frac{\dot{Q}_{zu}}{\dot{W}_{t}} = \frac{\dot{Q}_{k}}{2 \dot{W}}
\]

\section*{e)}

Die Lebensmittel würden weiter runtergekühlt werden, wenn der Wärmestrom \(\dot{Q}_{k}\) immer konstant bleiben würde, da mehr wäre es am Ende OK kühl.

```