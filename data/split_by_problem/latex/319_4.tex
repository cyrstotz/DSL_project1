
``````latex


4) a)

\begin{description}
    \item[Graph 1:] The graph is a plot with the x-axis labeled as $T_{ink}$ and the y-axis labeled as $p_{char}$. There are two curves intersecting each other. One curve starts from the top left and goes downwards to the bottom right, while the other curve starts from the bottom left and goes upwards to the top right. There are annotations indicating "wie in (*) gezeigt" and "p_{ink} geht nach rechts hinab".
    \item[Graph 2:] The graph is a plot with the x-axis labeled as $T_{ink}$ and the y-axis labeled as $p_{ink}$. There are three vertical lines labeled as "fest", "flüssig", and "gas". The "fest" line is annotated with "sauer", "amber", and "sauer". The "flüssig" line is annotated with "H_2O" and "Li". The "gas" line is annotated with "H_2".
\end{description}

b)

\[
\frac{dE_{p}}{dt} = \sum \dot{m}_i (h_i + \frac{ke_i + pe_i}{0}) + \sum \dot{Q} - \sum \dot{W}
\]

\[
0 = \dot{m}_{Rohm} (h_2 - h_3) + 28W
\]

\[
h_2 = \overline{h_3} (T = T_i - 6K) =
\]

\[
h_3^0 = s_2 = s_3 \Rightarrow nicht nötig zu interpolieren für h_3
\]

\[
s_3 = s_{ink} (hoch = runter)
\]

c)

\[
p_1 = p_2
\]

\[
p_3 = p_4 = 8 bar
\]

\[
\frac{dE_{p}}{df} = \sum \dot{m}_i (h_i) + \sum \dot{Q}^0 - \sum \dot{W}
\]

\[
0 = \dot{m} (h_1 - h_1) - \dot{W}
\]

d)

\[
E_k = \frac{\dot{Q}_k}{\dot{W}_k} = \frac{\dot{Q}_k}{\dot{W}_k} =
\]

\[
\dot{Q}_k = \dot{m} (h_2 - h_1)
\]

\begin{description}
    \item[Graph 3:] The graph is a plot with an x-axis and y-axis. There is a curve starting from the bottom left and going upwards to the top right. The curve is annotated with a wavy line at the top. The graph is labeled with "(*)".
\end{description}

```