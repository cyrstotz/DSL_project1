
``````latex


\section*{Aufgabe 3}

\subsection*{a)}

\[
pV = m \left( \frac{R}{M} \right) T
\]

\[
R = \frac{R}{M} = \frac{8.314 \, \frac{\text{kJ}}{\text{kmol K}}}{50 \, \frac{\text{kg}}{\text{kmol}}} = 0.16628 \, \frac{\text{kJ}}{\text{kg K}}
\]

\[
R = 0.1663 \, \frac{\text{kJ}}{\text{kg K}}
\]

\[
p_{s,1} = \frac{F}{A} + p_{amb}
\]

\[
A = \pi r^2 = \pi (5 \times 10^{-2} \, \text{m})^2 = 7.853981 \times 10^{-3} \, \text{m}^2
\]

\[
p_{s,1} = \frac{32 \, \text{kg} \cdot 9.81 \, \frac{\text{m}}{\text{s}^2}}{7.853981634 \times 10^{-3} \, \text{m}^2} + 100,000 \, \text{N m}^{-2}
\]

\[
= 139,969.538 \, \text{N m}^{-2}
\]

\[
= 1.3967 \, \text{bar}
\]

\[
\frac{pV}{RT} = m_g = \frac{(139,969.538 \, \text{Pa}) (3.14 \times 10^{-3} \, \text{m}^3)}{0.16628 \, \frac{\text{kJ}}{\text{kg K}} (500 + 273.15) \, \text{K}}
\]

\[
= 3.418687423 \, \frac{\text{N m}}{\frac{\text{N m}}{\text{kg}}}
\]

\[
M_g = \underline{3.41879}
\]

\subsection*{b)}

\subsection*{c)}

mit $T_{1,2} = 0.063^\circ \text{C}$ weiter gerechnet.

\[
\frac{dE}{dt} = \dot{Q}_{12} - \dot{W} \implies \Delta E = Q_{12} - W
\]

Der Kolben führt Arbeit in das System hinein,

wir können die Formel $Q_{12} = m_s C (\Delta T)$ benutzen

\[
Q_{12} = 3.4187 \times 10^3 \, \text{kg} \cdot 0.623 \, \frac{\text{kJ}}{\text{kg K}} \cdot (500 - 0.003) \, \text{K}
\]

\[
= 1.08200 \, \text{kJ}
\]

\[
= 1082 \, \text{J}
\]

``````latex


\section*{Aufgabe 3}

\subsection*{d)}
$x_{\text{Eis,2}}$ in Zustand 2

dafür benötigen wir diese Formel

\[
\frac{U - U_{\text{Fest}}}{U_{\text{Flüssig}} - U_{\text{Fest}}} = x_{\text{Eis,2}}
\]

\[
\text{unser } U = \frac{U}{m_{\text{sw}}} = \frac{Q}{m_{\text{sw}}}
\]

\[
= \frac{1082}{1.08} = 1000 \frac{\text{kJ}}{\text{kg}}
\]

\[
= 10.82 \frac{\text{kJ}}{\text{kg}}
\]

Da wir noch nicht vollständig flüssig sind, liegen wir bei einer Temperatur von $0^\circ$C aus der Tabelle entnommen

\[
U_{\text{Fest}} = -333.458 \frac{\text{kJ}}{\text{kg}}, \quad U_{\text{Flüssig}} = -0.645 \frac{\text{kJ}}{\text{kg}}
\]

\text{einsetzen}

\[
\frac{10.82 \frac{\text{kJ}}{\text{kg}} - (-333.458) \frac{\text{kJ}}{\text{kg}}}{(-0.645) \frac{\text{kJ}}{\text{kg}} - (-333.458) \frac{\text{kJ}}{\text{kg}}} = 1.032
\]

\text{Daraus folgt:}

\[
x_{\text{Eis,2}} = 1.032
\]

``````latex


