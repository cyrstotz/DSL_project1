
``````latex


\section*{Aufgabe 4}

\subsection*{a)}

\begin{description}
    \item[Graph Description:] The graph is a Pressure-Temperature (P-T) diagram. The y-axis is labeled \(P (\text{bar})\) and the x-axis is labeled \(T (K)\). The graph shows a curve with two peaks, forming a closed loop. The curve starts at a lower point, rises to the first peak, then dips down to a valley, and rises again to the second peak before descending. There are three points marked on the curve: point 1 at the first peak, point 2 at the valley, and point 3 at the second peak. Arrows indicate transitions between these points: from point 1 to point 2, and from point 2 to point 3. There is a shaded area under the curve between points 1 and 2, labeled as region II. Another shaded area is between points 2 and 3, labeled as region I.
\end{description}

\subsection*{b) Energiebilanz über den Verdichter}

\[
\dot{W}_k = \dot{m}_{\text{da}} (h_e - h_a)
\]

\[
h_e = h_2 = h(T_2) = h(-18^\circ C) = 236,53 \frac{\text{kJ}}{\text{kg}}
\]

\[
h_a = h(p = 8 \text{ bar}) = 264,15 \frac{\text{kJ}}{\text{kg}}
\]

\[
T_2 = T_i - 6 K
\]

\[
\frac{\dot{W}_k}{h(T_2) - h(p = 8 \text{ bar})} = \frac{0,088 \text{ kW}}{28 \text{ W}} = \frac{0,088 \text{ kW}}{(236,53 - 264,15) \frac{\text{kJ}}{\text{kg}}} = -10,00101 \frac{\text{kg}}{\text{s}}
\]

\[
T_i = -18^\circ C
\]

``````latex


\section*{Student Solution}

\subsection*{c)}
\[
x = \frac{h - h_f}{h_g - h_f}
\]

\subsection*{d)}
\[
E_k = \frac{1}{2} \frac{\dot{Q} \tau}{\dot{V} + 1}
\]
\[
\dot{V}_4 = 28 \omega
\]

\subsection*{e)}
Das Wasser würde wieder gefrieren und weiter abkühlen.

```