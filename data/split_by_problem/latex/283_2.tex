
``````latex


\section*{Aufgabe 2}

\subsection*{a)}

\begin{description}
    \item[Graph 1:] The graph is a plot with the vertical axis labeled \( T \, [K] \) and the horizontal axis labeled \( S \, \left[ \frac{kJ}{kgK} \right] \). There are two curves on the graph. The first curve starts at point labeled \( 0 \) at the bottom left, moves upwards in a zigzag pattern, and ends at a point labeled \( 1 \) at the top right. The second curve starts at a point labeled \( 1 \) near the top left, moves downwards in a zigzag pattern, and ends at a point labeled \( 2 \) at the bottom right.
    
    \item[Graph 2:] The graph is a plot with the vertical axis labeled \( T \, [K] \) and the horizontal axis labeled \( S \, \left[ \frac{kJ}{kgK} \right] \). There are six points labeled \( 0, 1, 2, 3, 4, 5, \) and \( 6 \). The curve starts at point \( 0 \) at the bottom left, moves upwards to point \( 1 \), then continues upwards to point \( 2 \) labeled "isobare", then reaches a peak at point \( 3 \). From point \( 3 \), the curve moves downwards to point \( 4 \), then slightly upwards to point \( 5 \) with a horizontal dashed line labeled \( T_5 = 431.0 \, K \), and finally moves downwards to point \( 6 \).
\end{description}

``````latex


\section*{Aufgabe 2}

\subsection*{b) Energiebilanz um Schaufelöse (Stationär) führt zu}

\[
h_5 + \left( \frac{w_5^2}{2} \right) = h_6 + \left( \frac{w_6^2}{2} \right)
\]

\[
h_5 \text{ (lösbar, 431.9K)} = 
\]

\[
\Rightarrow h_5 = h(T_5) = h(431.9K)
\]

\[
\text{Tabelle A-22} \Rightarrow h(430K) + \frac{h(440K) - h(430K)}{(440 - 430)K} \cdot (431.9 - 430)K
\]

\[
\text{Interpolation} \Rightarrow \frac{h(440K) - h(430K)}{(440 - 430)K} \cdot (431.9 - 430K) + h(430K)
\]

\[
\text{Tabelle A-22} \Rightarrow \frac{(449.61 - 431.93) \frac{kJ}{kg}}{10K} \cdot 1.9K + 431.93 \frac{kJ}{kg}
\]

\[
= 433.37 \frac{kJ}{kg} = h_5
\]

\[
h_6 = h(340K) \Rightarrow \text{Tabelle A-22} = 340.42 \frac{kJ}{kg}
\]

``````latex


