
``````latex


3.(a)
\[
p_{G,1} = p_{amb} + \frac{M_{kolben} \cdot g}{\frac{\pi \left( \frac{D}{2} \right)^2}{}} + \frac{m_{EW} \cdot g}{\frac{\pi \left( \frac{D}{2} \right)^2}{}}
\]

\[
A = \pi \left( \frac{D}{2} \right)^2 = 0.00785 \, m^2
\]

\[
p_{G,1} = 10 \, Pa + \frac{32 \, kg \cdot 9.81 \, \frac{m}{s^2}}{0.00785 \, m^2} + \frac{0.1 \, kg \cdot 9.81 \, \frac{m}{s^2}}{0.00785 \, m^2} = 140090 \, Pa = 1.4 \, bar
\]

\[
p_{G,1} \cdot V_{G,1} = m_{G,1} \cdot R_G \cdot T_{G,1} \rightarrow m_{G,1} = \frac{p_{G,1} \cdot V_{G,1}}{R_G \cdot T_{G,1}}
\]

\[
R_G = \frac{R}{M_G} = 166.3 \, \frac{J}{kg \cdot K}, \quad V_{G,1} = 3.14 \, L = 0.00314 \, m^3
\]

\[
m_{G,1} = 0.00342 \, kg = 3.42 \, g
\]

(b) Die Temperatur und der Druck bleiben konstant:

\[
T_{EW,2} = T_{EW,1} = 0^\circ C
\]

\[
p_{G,2} = p_{G,1} = p_{amb} + \frac{M_{kolben} \cdot g}{\frac{\pi \left( \frac{D}{2} \right)^2}{}} = 133970 \, Pa = 1.34 \, bar
\]

Dies liegt daran, dass wir uns im Zweiphasengebiet befinden, bei der eine Wärmezufuhr erst dann zu einer Temperatur- und Druckerhöhung führt, wenn das gesamte Eis geschmolzen ist. Dies ist bei \( x_{Eis,2} > 0 \) nicht der Fall.

Die Temperatur beträgt \( 0^\circ C \), da das Eis nicht komplett schmilzt (\( x_{Eis,2} > 0 \)) und die Temperatur des Eises daher konstant bleibt, \( T_{G,2} = T_{Eis,2} = 0 \).

Damit keine Wärme mehr übertragen wird, muss \( T_{G,2} = T_{Eis,2} = 0 \) gelten. Der Druck bleibt ebenfalls konstant, da das Atmosphärendruck und das Gewicht des Kolbes sowie der Wasser-Eis-Mischung konstant bleiben.

``````latex


3.(c) Gasegemisch Energiebilanz:

\[
\Delta E = M_{Gas,2} \cdot U_{Gas,2} - M_{Gas,1} \cdot U_{Gas,1} = -Q_{12} - W_{12}
\]

Wenn \( T_{EW,2} = T_{G,2} \), dann wird keine Wärme mehr übertragen.

Daher: \( T_{G,2} = T_{EW,2} = 0^\circ C = 273.15 K \)

Zudem bleibt der Druck konstant: \( p_{G,2} = p_{G,1} = 1.4 \text{bar} \)

Die Masse ebenfalls: \( M_{G,2} = M_{G,1} = 0.00342 \text{kg} \)

\[
V_{G,2} = \frac{M_{G,2} \cdot R_G \cdot T_{G,2}}{p_{G,2}} = 0.001109 \, m^3
\]

\[
W_{12} = p_{G,1} \left( V_{G,2} - V_{G,1} \right) = -284.48 \, J
\]

\[
Q_{12} = M_{Gas} \left( U_{G,1} - U_{G,2} \right) - W_{12} = M_G \cdot C_V \cdot (T_{G,1} - T_{G,2}) - W_{12}
\]

\[
= 1367.5 \, J
\]

(d) \( m_{EW} \left( U_{EW,2} - U_{EW,1} \right) = Q_{12} \quad \text{(Arbeit 0, da isochor)} \)

\[
\Delta E = \Delta U = \cancel{\Delta p_{EW}}
\]

\[
U_{EW,1} = U_{flüssig} \left( 0^\circ C, 1.4 \text{bar} \right) + X_{Eis,1} \cdot \left( U_{fest} \left( 0^\circ C, 1.4 \text{bar} \right) - U_{flüssig} \left( 0^\circ C, 1.4 \text{bar} \right) \right)
\]

\[
= -200.117 \, \frac{kJ}{kg}
\]

\[
U_{EW,2} = U_{flüssig} \left( 0^\circ C, 1.4 \text{bar} \right) + X_{Eis,2} \cdot \left( U_{fest} \left( 0^\circ C, 1.4 \text{bar} \right) - U_{flüssig} \left( 0^\circ C, 1.4 \text{bar} \right) \right)
\]

\[
m_{EW} \left( U_{EW,2} - U_{EW,1} \right) = m_{EW} \left( X_{Eis,2} \cdot \left( U_{fest} \left( 0^\circ C, 1.4 \text{bar} \right) - U_{flüssig} \left( 0^\circ C, 1.4 \text{bar} \right) \right) - X_{Eis,1} \cdot \left( U_{fest} \left( 0^\circ C, 1.4 \text{bar} \right) - U_{flüssig} \left( 0^\circ C, 1.4 \text{bar} \right) \right) \right) = Q_{12}
\]

``````latex


Fortsetzung 3.d.:

\[
X_{\text{Eis,2}} = \frac{\frac{Q_{12}}{m_{\text{EN}}} + X_{\text{Eis,1}} \left( U_{\text{Eis}} (0^\circ C, 1.46 \text{bar}) - U_{\text{flüssig}} (0^\circ C, 1.46 \text{bar}) \right)}{U_{\text{Eis}} (0^\circ C, 1.46 \text{bar}) - U_{\text{flüssig}} (0^\circ C, 1.46 \text{bar})}
\]

\[
= 0.559
\]

``````latex


