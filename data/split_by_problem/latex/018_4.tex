
``````latex


\section*{a) R134a}

\begin{tabular}{|c|c|c|}
\hline
 & P & T \\
\hline
1 & & 50.33$^\circ$C \\
\hline
2 & 8 bar & \\
\hline
3 & 8 bar & 31.33$^\circ$C \\
\hline
4 & & \\
\hline
\end{tabular}

\[
T_1 - 6K = T_2
\]

\[
x_2 = 1
\]

\[
x_4 = 0
\]

\[
\text{gas} \quad \text{isotherm} \quad \text{isobar}
\]

\[
\text{flüssig} \quad \text{isentrop} \quad \text{isotherm}
\]

\[
T_1 = 10K + T_{ges}
\]

\[
= 10K + 273.15
\]

\[
= 283.15
\]

\[
T_2 = T_1 - 6K = 277.15
\]

\subsection*{a2)}

\textbf{Graph Description:} The graph is a pressure-volume (P-V) diagram. The x-axis is labeled $v [\text{m}^3/\text{kg}]$ and the y-axis is labeled $p [\text{bar}]$. There are four points labeled 1, 2, 3, and 4. Point 1 is at the bottom left, point 2 is directly to the right of point 1, point 3 is above point 2, and point 4 is to the left of point 3. The process from 1 to 2 is labeled as "isotherm", from 2 to 3 as "isobar", from 3 to 4 as "isotherm", and from 4 to 1 as "isentrop".

\section*{b) $\dot{m}$?}

\[
\text{stat.} \quad \text{1. H.S.}
\]

\[
0 = \dot{m} (h_e - h_a) + \sum \dot{Q} - \sum \dot{W}
\]

\section*{c)}

\[
\dot{m} \quad T_2 =
\]

\[
x_1 = \frac{h_A - h_g}{h_g - h_f} \quad \rightarrow T_1 = T_a = 31.33^\circ C
\]

\section*{d)}

\[
E_u = \frac{\dot{Q}_{zu}}{\dot{W}_t} = \frac{|\dot{Q}_{zu}|}{\dot{Q}_{abl} - |\dot{Q}_{zu}|} =
\]

\[
\rightarrow \dot{Q}_{zu} = \dot{Q}_i = \dot{m} (h_2 - h_1)
\]

\[
\rightarrow \dot{Q}_{ab} = \dot{m} (h_4 - h_3)
\]

\[
h_1 = h(T_1, p_1)
\]

\[
h_2 = h_2(T_2 - 22^\circ C) = 236.08
\]

\[
h_3 (8 \text{bar}, T_3)
\]

\[
h_4 = h_f (8 \text{bar}) = 93.62
\]

``````latex


d) $T_p$ würde immer niedriger werden \\
bis es "imaginärerweise" zum absoluten \\
Nullpunkt erreicht

```