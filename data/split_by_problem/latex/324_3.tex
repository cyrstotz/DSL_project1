
``````latex


\section*{Aufgabe 3: Gas}

\begin{tabular}{|c|c|c|c|c|c|}
\hline
Zustand & P & V & T & X & m \\
\hline
1. & & 3.14L & 50°C & & \\
\hline
2. & & & & & \\
\hline
\end{tabular}

\begin{tabular}{|c|c|c|c|c|c|}
\hline
 & P & V & T & X & m \\
\hline
 & & & 0°C & 0.6 & 0.1kg \\
\hline
\end{tabular}

\subsection*{a)}

\[
p_{g1} = \frac{m_g}{A}
\]

Da EW gemisch inkompressibel vernachlässiger bei Druck:

\[
p_{g1} = \frac{m_g}{A} + p_{atm}
\]

\[
A = \text{Fläche von Zylinder}
\]

\[
A = \pi \left( \frac{d}{2} \right)^2 = \pi \left( \frac{0.1}{2} \right)^2 = \frac{\pi}{400} \, m^2
\]

\[
p_{g1} = \frac{32 \, kg}{\frac{\pi}{400} \, m^2} \cdot 9.81 \, \frac{m}{s^2} + 100000 \, Pa = 131396.54 \, Pa
\]

\[
p_{g1} V_1 = R m_g T
\]

\[
m_g = \frac{p_1 V_1}{R \cdot T}
\]

\[
R = \frac{8.314 \, \frac{J}{mol \cdot K}}{\frac{50 \, kg}{kmol}} = \frac{R}{M} = 166.28 \, \frac{J}{kg \cdot K}
\]

\[
m_g = \frac{131396.54 \cdot 0.00314 \, m^3}{166.28 \cdot 273.15 \, K} = 0.0034 \, kg = 3.4 \, g
\]

\subsection*{b)}

\[
p_{g2} = p_{g1} \Rightarrow p \, da \, das \, EW-Gemischen \, inkompressibel \, ist.
\]

Die Temperatur wird bei 0°C bleiben, da das Eis nur ein gewisser Anteil des Eises schmilzt, da x2 > 0.

``````latex


\section*{Aufgabe 3:}

\subsection*{c)}
\textit{1. HS um Gasgemisch: Geschlossenes System}

\[
\Delta U = Q_{12} - W_{12}
\]

\[
\Delta U = m c_V (T_2 - T_1)
\]

\[
\Delta U = 0.0034 \cdot 0.633 \cdot (0 - 500)
\]

\[
\Delta U = -1.076.1
\]

\textit{W12: isochore Kompression}

\[
W_{12} = p_1 \left( V_{2J} - V_{1J} \right) = 285.54 J
\]

\[
p_2 V_{2J} = R T m
\]

\[
V_{2J} = \frac{R T m}{p_2} = 0.0011 \, m^3
\]

\[
T_2 = 0^\circ C
\]

\[
m = m_2
\]

\[
p_2 = p_A
\]

\[
Q_{12} = \Delta U + W_{12}
\]

\[
Q_{12} = -1.076.1 J + 285.54 J = -1.361.54 J
\]

\subsection*{d)}
\textit{1. HS um EW-Gemisch}

\[
x_2 = x
\]

\textit{0 inkompressibel}

\[
\Delta U = Q_{12} - W_{12}
\]

\[
m_{EW} (u_2 - u_1) = Q_{12}
\]

\[
u_2 = \frac{Q_{12}}{m_{EW}} + u_1
\]

\[
u_2 = -1.361.54 J
\]

\[
u_2 = 0.1 kg \cdot -200
\]

\[
u_2 = -186.3836 \, kJ/kg
\]

\[
u_2 (T = 0) = -186.3836
\]

\[
u_2 (T = 0) = -186.3836 = u_{Flüssig} + x \left( u_{RST} - u_{Flüssig} \right)
\]

\[
x = \frac{u_2 - u_{Flüssig}}{u_{RST} - u_{Flüssig}} = 0.553
\]

\textit{Q12 wird zugeführt daher > 0. 1361.64}

\textit{wenn alles Gas dann x = 1}

\[
u_1 (T = 0, x = 0.6) = u_{Flüssig} + x \left( u_{RST} - u_{Flüssig} \right)
\]

\textit{Tab. 1. Interpolieren}

\[
u_{A} = -0.045 + 0.6 \left( -333.458 + 0.045 \right)
\]

\[
u_1 = -200 \, kJ/kg
\]

\[
u_{Flüssig} = -0.045 \, kJ/kg
\]

\[
u_{RST} = -333.458 \, kJ/kg
\]

``````latex


