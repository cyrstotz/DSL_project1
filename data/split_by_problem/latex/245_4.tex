
``````latex


\section*{Aufgabe 1}

\subsection*{a)}

\textbf{Description of the first graph:} 

The first graph is a plot with the x-axis labeled as \( T [K] \) and the y-axis labeled as \( p [mbar] \). There are four points labeled 1, 2, 3, and 4. The points are connected to form a closed loop. The path from point 1 to point 2 is labeled as "isobar", the path from point 2 to point 3 is labeled as "adiabat", the path from point 3 to point 4 is labeled as "isobar", and the path from point 4 to point 1 is labeled as "adiabat".

\textbf{Description of the second graph:} 

The second graph is also a plot with the x-axis labeled as \( T [K] \) and the y-axis labeled as \( p [mbar] \). There are four points labeled 1, 2, 3, and 4. The points are connected to form a closed loop. The path from point 1 to point 2 is labeled as "isobar", the path from point 2 to point 3 is labeled as "isobar", the path from point 3 to point 4 is labeled as "isobar", and the path from point 4 to point 1 is labeled as "isobar".

\subsection*{b)}

\begin{align*}
T_2 &= T_i - 6K \\
T_i &= T_{\text{tripel}} + 10K \\
T_2 &= 0^\circ C + 10K - 6K = 4^\circ C \\
\text{Zustand 2-3} \\
0 &= \dot{m} (h_2 - h_3) + \dot{Q}^{\circ} - \dot{W} \\
\dot{m} &= \frac{\dot{W}}{h_2 - h_3}
\end{align*}

\subsection*{c)}

\begin{align*}
\text{Zustand 4-1} \\
0 &= \dot{m} (h_4 - h_1) + \dot{Q}^{\circ} - \dot{W}^{\circ} \\
h_4 &= h_1
\end{align*}

``````latex


d) \[ E_k = \frac{|Qzu|}{|W|} \]

e) Die Temperatur würde weiter sinken, aber immer langsamer

```