
``````latex


\section*{4.}

\subsection*{a)}

\begin{description}
    \item[Graph 1:] A plot with the y-axis labeled \( p(\text{bar}) \) and the x-axis labeled \( T(\degree C) \). The graph shows a dome-shaped curve labeled "flüssig" on the left side, "Nass-Dampf" in the middle, and "Dampf" on the right side.
    \item[Graph 2:] A plot with the y-axis labeled \( p(\text{bar}) \) and the x-axis labeled \( T(\degree C) \). The graph shows a curve starting from the origin and rising upwards. The curve is labeled "Flüssig" on the lower part and "Dampf" on the upper part. A horizontal line intersects the curve at a point labeled \( T_i \). The intersection point is labeled "Flüssig" on the left and "Dampf" on the right.
\end{description}

\subsection*{b)}

\begin{align*}
    \text{1. HS der Verdichter} \\
    0 &= \dot{m} (h_2 - h_1) - \dot{V}_u \\
    \dot{m} &= \frac{\dot{V}_u}{h_2 - h_1} \\
    T_2 - T_1 &= 6 = 273.15 - 10 - 6 = 259.15 \text{K} = -76 \degree C \\
    h_2 &= h(x=0, T_2) = 234.08 \frac{\text{kJ}}{\text{kg}} \\
    x &= 0.33845
\end{align*}

\subsection*{c)}

\begin{align*}
    h_1 - h_2 &= h \left( 5 \frac{\text{bar}}, x=0 \right) = 31.92 \frac{\text{kJ}}{\text{kg}} \\
    T_2 &= T_1 \\
    p_1 = p_2 &= 1.57 \text{bar}
\end{align*}

\subsection*{d)}

\begin{align*}
    \eta_{\text{th}} &= \frac{\dot{Q}_{zu}}{\dot{Q}_{zu} - \dot{Q}_{ab}} = \frac{\dot{Q}_{zu}}{\dot{V}_u}
\end{align*}

\subsection*{e)}

Die Fernwärme würde sich in einer Pfütze gut auflösen.

```