
``````latex


\section*{Aufgabe 4a}

\begin{description}
    \item[Graph Description:] The graph is a phase diagram with the y-axis labeled as "P (Druck)" and the x-axis labeled as "T (°C)". The y-axis represents pressure and the x-axis represents temperature. The graph shows three distinct regions labeled "fest", "flüssig", and "gas" which correspond to solid, liquid, and gas phases respectively.
    
    The boundary between the solid and liquid phases is marked by a curve that starts from the origin and curves upwards. The boundary between the liquid and gas phases is a horizontal line extending from the right end of the solid-liquid boundary. The boundary between the solid and gas phases is a curve that starts from the origin and curves downwards to the right.
    
    There is a point labeled "Tripel" where all three phases meet. Additionally, there are two points marked on the graph: one in the solid region and one in the liquid region. These points are connected by a vertical line indicating a phase transition from solid to liquid.
    
    The graph also includes an arrow pointing from the liquid region to the gas region, labeled "kritischer Punkt", indicating the critical point.
\end{description}

``````latex


\section*{Aufgabe 4}

\subsection*{b)}

\begin{align*}
    p_1 &= p_2 \\
    T &= 6k = T_1 = T_2
\end{align*}

7. HS über Verdichter, \( V_p = 250 \, W \)

\[
O = \dot{m}_{R_{134}} \left( h_2 - h_3 \right) + \dot{m}_{R_{134}} \cdot \dot{m}_{R_{134}} + \frac{W_k}{h_2 - h_3}
\]

\[
\dot{m}_{R_{134}} = \frac{-W_k}{h_2 - h_3}
\]

\[
T_1 = 40 \, ^\circ C
\]

\subsection*{Graphical Description}

There is a wavy line drawn horizontally across the page, representing some form of boundary or separation.

\[
h_2 = 249.53 \, \text{kJ/kg}
\]

\[
h_2 = 587.36 \, \text{kJ/kg}
\]

\[
s_2 = s_3 = 0.9 \, \text{kJ}
\]

\[
s_3 = s_f + x_3 (s_g - s_f) \implies x_3 = \frac{s_3 - s_f}{s_g - s_f}
\]

\[
h_3 = h_f + x_3 (h_g - h_f)
\]

\[
\dot{m}_{R_{134}} = \frac{W_k}{h_4 + x_3 (h_g - h_f) - h_2}
\]

``````latex


\section*{Aufgabe 4 (C)}

\subsection*{a)}
1. HS über total

\[
\boxed{\dot{Q}_k}
\]

\subsection*{b)}
\[
0 = \dot{m} \cdot R_{12} \left[ h_{2} - h_{2} \right] + \dot{Q}_k
\]

\[
h_2 = -\frac{\dot{Q}_k}{\dot{m} \cdot R_{12}} + h_2
\]

\[
\text{Bei } p_2 = p_1 = \frac{p_2}{p_1} = 3.3756 \, \text{bar}
\]

\subsection*{c)}
\[
h_2 \Rightarrow h_{tp} \quad h_1 (3.3756 \, \text{bar}) = h_f + x_n (h_{fg} - h_f)
\]

\[
x_n = \frac{h_2 - h_f}{h_{fg} - h_f}
\]

\subsection*{d)}
\[
\epsilon_k = \frac{\dot{Q}_k \cdot x_n}{1 \cdot \left( \dot{Q}_{ub} - \dot{Q}_k \right)} = \frac{\dot{Q}_k}{\dot{Q}_{ub} - \dot{Q}_k}
\]

\subsection*{e)}

```

**Verbal Description of Graphical Content:**

- There is a rectangular box drawn with an arrow pointing to the right side of the box. The arrow is labeled with \(\dot{Q}_k\).

This LaTeX document includes all the text and formulas from the handwritten student solution, as well as a detailed verbal description of the graphical content.