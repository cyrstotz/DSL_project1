
``````latex


4) a)

\textbf{Graphical Description:}

The first graph is a Pressure-Temperature (P-T) diagram. The y-axis is labeled as \( P \) and the x-axis is labeled as \( T \). There are two curves intersecting each other. The first curve starts from the y-axis, goes downwards, and then upwards, forming a V-shape. The second curve starts from the y-axis, goes upwards, and then downwards, forming an inverted V-shape. The intersection point of these two curves is labeled as "Tripelpunkt". The region to the left of the first curve is labeled as "solid", the region between the two curves is labeled as "liquid", and the region to the right of the second curve is labeled as "vapor". There are four points labeled 1, 2, 3, and 4 forming a cycle in the liquid region.

The second graph is a Temperature-Entropy (T-s) diagram. The y-axis is labeled as \( T \) and the x-axis is labeled as \( s \). There are two curves forming a dome shape. The left curve starts from the y-axis and goes upwards, and the right curve starts from the y-axis and goes upwards. The region inside the dome is labeled as "liquid vapor". There are four points labeled 1, 2, 3, and 4 forming a cycle inside the dome.

The small graph on the top right is a magnified view of the cycle formed by points 1, 2, 3, and 4. The points are connected by arrows indicating the direction of the cycle.

b)

\text{gegeben:} \dot{m} = 0.7437 \, \text{kg/s}

\text{Zustandsgleichung:}
\[
0 = \dot{m}(h_2 - h_3) - \dot{W}_k
\]
\[
\Rightarrow \dot{W}_k = \frac{\dot{m}}{h_2 - h_3} = 0.7437 \, \text{kg/s}
\]

1. \( p_1 = p_2 = p_3 = p_4 = 85 \, \text{bar} \)
2. \( s_2 = s_3 \)
3. \( s_1 = s_4 \)
4. \( x = 0, p_4 = 85 \, \text{bar} \)

\text{mit} \quad T_2 = -m_1 92^\circ C \quad \text{aus Tabelle}

\[
h_2 = h_g = 254.08 \, \text{kJ/kg}
\]
\[
s_3 = s_2 = 0.855 \, \text{kJ/kgK}
\]
\[
h_3 = h(40^\circ C, 85 \, \text{bar}) - h(\text{Sat}, 85 \, \text{bar})
\]
\[
\text{interp.} \quad s(40^\circ C, 85 \, \text{bar}) - s(\text{Sat}, 85 \, \text{bar})
\]
\[
= 274.72 \, \text{kJ/kg}
\]

``````latex


\section*{Student Solution}

\subsection*{c)}

\[
X_\lambda
\]

\[
P_1 = P_2 = \frac{A^{10}}{A^{10}} = 1.2932 \, \text{bar}
\]

\[
S_4 = S_3 = S_F = 0.3459 \, \frac{L^3}{\text{kgK}}
\]

\[
S_3 = S_4 = S_F = \cancel{0.3459 \, \frac{L^3}{\text{kgK}}}
\]

\[
b
\]

\[
X_\lambda = \frac{S_4 - S_P}{S_g - S_f} = \frac{A^{10}}{A^{10}} = 0.3031
\]

\subsection*{d)}

\[
\varepsilon_k
\]

\[
\varepsilon_k = \frac{\dot{E}_{X,\text{Nutz}}}{\dot{E}_{X,\text{ZU}}}
\]

\[
L_3 \dot{E}_{X,\text{ZU}} = \left(1 - \frac{T_0}{T}\right) \cdot \dot{Q}_k
\]

\[
\dot{E}_{X,\text{Nutz}} = \left(1 - \frac{T_0}{T}\right) \cdot \dot{Q}_{ab}
\]

\subsection*{e)}

Wenn nicht isotherm würde der Druck immer steigen.

```