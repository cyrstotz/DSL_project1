
``````latex


\section*{Aufgabe 3}

\subsection*{a)}

\begin{itemize}
    \item $p_1$ entsteht durch $p_{atm} + F_G \frac{m}{A} + F_{OBV} \frac{V}{A}$
\end{itemize}

\[
A = 8,95 \, \text{cm}^2 \cdot \pi = 28 \pi \, \text{cm}^2
\]

\[
p_1 = 1 \, \text{bar} + 32 \, \text{kg} \cdot 9,81 \, \frac{\text{m}}{\text{s}^2} \cdot \frac{1}{28 \pi \cdot 10^{-4} \, \text{m}^2} + 0,12 \, \text{kg} \cdot 9,81 \, \frac{\text{m}}{\text{s}^2} \cdot \frac{1}{28 \pi \cdot 10^{-4} \, \text{m}^2}
\]

\[
= 1 \cdot 10^5 \, \frac{\text{N}}{\text{m}^2} + 35966,3 \, \frac{\text{N}}{\text{m}^2} + 42,3 \, \frac{\text{N}}{\text{m}^2}
\]

\[
= 1,40 \, \text{bar}
\]

\begin{itemize}
    \item $m_g$: $pV = mRT \rightarrow m_g = \frac{p_3 V_3}{R T_{g_3}}$
\end{itemize}

\[
-m_g = \frac{p_3 V_3}{R T_{g_3}} = \frac{p_3 V_3}{R T_{g_3}}
\]

\[
= \frac{1,4 \cdot 10^5 \, \frac{\text{N}}{\text{m}^2} \cdot 3,74 \cdot 10^{-3} \, \text{m}^3}{8,314 \, \frac{\text{J}}{\text{mol} \cdot \text{K}} \cdot 50 \, \text{J} / \text{mol} \cdot \text{K}}
\]

\[
= 0,003479 \, \text{kg} = 3,479 \, \text{g}
\]

\subsection*{b)}

Die Masse des Zylos und des Gehäuses sowie $p_0$ haben sich nicht geändert. Somit ist $p_2$ immernoch $p_1$.

\[
p_2 = p_1 = 1,40 \, \text{bar}
\]

``````latex


c) 1. HS von Gas, geschlossenes System

\[
\Delta U = Q_{12} - W_{12} = Q_{12} - W_V
\]

\[
Q_{12} = Q_2 - Q_1 = W_V
\]

\[
Q_{12} = Q_2 - Q_1 + W_V
\]

``````latex


