
``````latex


\section*{Aufgabe 4}

\subsection*{a)}

\begin{description}
    \item[Graph Description:] The graph is a Pressure-Temperature (P-T) diagram. The y-axis is labeled with \( p \) and the x-axis is labeled with \( T \). There are three points labeled 1, 2, and 3. Point 1 is at the intersection of the isentrope and the isobar at 8 bar. Point 2 is on the isentrope curve below point 1. Point 3 is on the isobar line at 8 bar, to the right of point 1. The isentrope curve is concave downwards, and the isobar line is horizontal. The isentrope curve is labeled "isentrope" and the isobar line is labeled "isobar".
\end{description}

\subsection*{b)}

\[
\dot{m} \, R134a
\]

\[
\text{stationärer Prozess} \quad (\text{Bilanz um den Kompressor})
\]

\[
0 = \dot{m} \left[ h_e - h_a \right] - \dot{W}_k
\]

\[
\dot{m} = \frac{\dot{W}_k}{h_e - h_a} = \frac{28 \omega}{h_e - h_a}
\]

\[
\dot{W}_k = 28 \omega
\]

\[
h_e \, \text{und} \, h_a \, \text{aus Tabelle A-11 auslesen}
\]

\[
h_e =
\]

``````latex


\begin{itemize}
    \item[c)] \( x_a \) direkt nach der Drossel
    \[
    \frac{-1 \text{kg}}{1 \text{s}} = 1 \frac{\text{g}}{\text{s}}
    \]
    \[
    x_a = \frac{mg}{m}
    \]
    \[
    x_a = \frac{Mg + Mp}{}
    \]
    
    \item[d)] 
    \[
    E_k = \frac{\dot{Q}_{zu}}{\left| \dot{w}_s \right|} = \frac{\dot{Q}_k}{\dot{Q}_k} - \frac{}{25 \text{kw}}
    \]
    \[
    \dot{Q}_k = -\dot{m} (\text{the - ha}) = -4 \frac{\text{kg}}{\text{h}}
    \]
    
    \item[e)] Man hatte dann einen Übergang von fest zu flüssig und wenn man dann \textit{was?} nach Kontakt weiter abgekühlt wurde, blieb die Temperatur so lange konstant, bis der Phasenwechsel vollzogen ist. Erst dann würde sie wieder sinken.
\end{itemize}

```