
``````latex


\section*{Aufgabe 3}

\subsection*{a)}

\begin{align*}
R &= \frac{\bar{R}}{M_g} = \frac{8,314 \frac{J}{mol \cdot K}}{50 \frac{kg}{kmol}} = 0,1663 \frac{kJ}{kg \cdot K} \\
V_{g,1} &= 3,14 L = 3,14 \cdot 10^{-3} m^3 \\
p_{g,1} \cdot V_{g,1} &= R \cdot T_{g,1} \cdot m_{g,1} \\
\Rightarrow p_{g,1} &= \frac{0,1663 \frac{kJ}{kg \cdot K} \cdot (500 + 273,15) K}{3,14 \cdot 10^{-3} m^3} = \\
&= 2863,6 \, kPa
\end{align*}

\begin{align*}
p_{m,g} \cdot \pi D^2 &= g \left( m_m + m_{m,w} \right) + p_{amb} \\
p_{m,g} &= \frac{9,81 \frac{m}{s^2} (0,14 kg + 32 kg)}{\pi \cdot (0,1 m)^2} = 10023 \, Pa = 11,1348,6101 \, Pa
\end{align*}

\begin{align*}
m_{g,1} &= \frac{p_{g,1} \cdot V_{g,1}}{R \cdot T_{g,1}} = \frac{2863,6 \cdot 10^3 \, Pa \cdot 3,14 \cdot 10^{-3} \, m^3}{0,1663 \cdot 10^3 \, \frac{J}{kg \cdot K} \cdot 773,15 \, K} \approx 2,445 \cdot 10^{-4} \, kg
\end{align*}

\subsection*{b)}

\begin{quote}
$p_{g,2}$ ist unverändert, weil der durch das Schmelzen verursachte Temperaturunterschied ausgeglichen wird, indem sich das Volumen vergrößert. Von oben drückt ja unverändert die gleiche Gewichtskraft.
\end{quote}

``````latex

\section*{c)}

\begin{equation}
\dot{Q}_{12} = +\dot{W}_v = +m_{gas} \cdot \dot{P}_{gas} \left( V_2 - V_1 \right)
\end{equation}

\section*{d)}

``````latex


