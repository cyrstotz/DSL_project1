
``````latex


\section*{Aufgabe 3}

\subsection*{a)}

\textbf{Kräftegleichgewicht:}

\[
p_{g,1} \cdot \left(\frac{D}{2}\right)^2 \pi = p_{amb} \left(\frac{D}{2}\right)^2 \pi + m_k \cdot g
\]

\[
\Rightarrow p_{g,1} = p_{amb} + \frac{m_k \cdot g}{\left(\frac{D}{2}\right)^2 \pi} = 100000 \, \text{Pa} + \frac{32 \, \text{kg} \cdot 9.81 \, \frac{\text{N}}{\text{kg}}}{(0.05 \, \text{m})^2 \cdot \pi}
\]

\[
= \boxed{1.3997 \, \text{bar} = p_{g,1}}
\]

\textbf{Nach idealem Gasgesetz gilt:}

\[
m_g = \frac{p_{g,1} \cdot V_{g,1}}{R_g \cdot T_{g,1}}, \quad R_g = \frac{R}{M_g} \quad \Rightarrow \quad R_g = \frac{8.314 \, \frac{\text{kJ}}{\text{kmol} \cdot \text{K}}}{50 \, \frac{\text{kg}}{\text{kmol}}} = 166.3 \, \frac{\text{J}}{\text{kg} \cdot \text{K}}
\]

\[
\Rightarrow m_g = \frac{139970 \, \text{Pa} \cdot 3.14 \cdot 10^{-3} \, \text{m}^3}{166.3 \, \frac{\text{J}}{\text{kg} \cdot \text{K}} \cdot 773.15 \, \text{K}} = \boxed{3.42 \, \text{g} = m_g}
\]

``````latex


\section*{Aufgabe 3}

\subsection*{b)}
$p_{g12}$ muss gleich bleiben wegen Kräftegleichgewicht \\
(Kraft die das Gas zusammendrückt bleibt gleich)

für $T_{g12}$ muss wegen thermod. Gleichgewicht gelten

\subsection*{c)}
Tabelle für Gas (m = 3.92g)

\begin{tabular}{|c|c|c|c|}
\hline
 & P & V & T \\
\hline
1 & 1.3997 \, \text{bar} & 3.19 \, L & 773.15 \, K \\
\hline
2 & 1.3997 \, \text{bar} & 0.0011 \, m^3 & 273.153 \, K \\
\hline
\end{tabular}

\subsection*{Ideales Gas-Gesetz}
\[
V_z = \frac{R_g \cdot m_g \cdot T_2}{p_2} = \frac{166.3 \, \frac{J}{kg \cdot K} \cdot 0.00392 \, kg \cdot 273.153 \, K}{139970 \, Pa} = 0.0011 \, m^3
\]

``````latex


\section*{Aufgabe 3}

d) es muss gelten: 
\[
U_2^{\text{EW}} = \frac{U_{12}}{m_{\text{EW}}}(0.003^\circ \text{C}) + x \left( U_{\text{fst}}(0.003^\circ \text{C}) - U_{\text{ff}}(0.003^\circ \text{C}) \right)
\]

\[
\Rightarrow x = \frac{U_2^{\text{EW}} - U_{\text{ff}}}{U_{\text{fst}} - U_{\text{ff}}}
\]

Energiebilanz des cms Eiswasser

\[
U_2^{\text{EW}} - U_1^{\text{EW}} = \frac{Q_{12}}{m_{\text{EW}}}
\]

\[
U_1^{\text{EW}} = U_{\text{ff}}(0^\circ \text{C}) + x_{\text{Eis},1} \left( U_{\text{fst}}(0^\circ \text{C}) - U_{\text{ff}}(0^\circ \text{C}) \right)
\]

\[
U_1^{\text{EW}} = -0.045 \frac{\text{kJ}}{\text{kg}} + 0.6 \left( -333.458 + 0.045 \right) \frac{\text{kJ}}{\text{kg}}
\]

\[
= -200.09 \frac{\text{kJ}}{\text{kg}} \Rightarrow U_1^{\text{EW}}
\]

einsetzen in

\[
U_2^{\text{EW}} = \frac{Q_{12}}{m_{\text{EW}}} + U_1^{\text{EW}} = \frac{1500 \text{J}}{0.1 \text{kg}} - 200.09 \frac{\text{kJ}}{\text{kg}}
\]

\[
= -185.09 \frac{\text{kJ}}{\text{kg}} = U_2^{\text{EW}}
\]

einsetzen in

\[
x = \frac{U_2^{\text{EW}} - U_{\text{ff}}(0.003^\circ \text{C})}{U_{\text{fst}}(0.003^\circ \text{C}) - U_{\text{ff}}(0.003^\circ \text{C})}
\]

``````latex


\section*{Aufgabe 3}

\subsection*{d)}

\begin{itemize}
    \item[$\rightarrow$] EW: Tab. 1
\end{itemize}

\[
x = \frac{-185.09 \frac{\text{kJ}}{\text{kg}} + 0.033 \frac{\text{kJ}}{\text{kg}}}{-333.492 + 0.033 \frac{\text{kJ}}{\text{kg}}}
\]

\[
= 0.555 = x
\]

The final result is boxed.

``````latex


