
``````latex


\section*{4)}

\subsection*{a)}
\begin{itemize}
    \item A graph is drawn with the y-axis labeled as \(P\) and the x-axis labeled as \(T\).
    \item There is a curve starting from the origin and rising upwards, labeled as "2-phase".
    \item Another curve starts from the top left and goes downwards, labeled as "triple".
    \item A horizontal line intersects the "2-phase" curve and is labeled as "sos".
    \item The point of intersection of the "sos" line and the "2-phase" curve is marked.
    \item The region below the "2-phase" curve is labeled as "flüssig".
\end{itemize}

\subsection*{b)}
\begin{align*}
    \text{Energieblance} \\
    0 &= \dot{m} (h_2 - h_3) - \dot{W}_K \\
    \frac{\dot{W}_K}{h_2 - h_3} &= \dot{m} \\
    &\rightarrow 1.145 \, \text{kg/s} \\
    T_H &= \text{Sublim. -10k} \\
    &\rightarrow 263.15 \, \text{K} \\
    h_2 &= 257.154 \\
    &\rightarrow 237.74 \\
    s_2 &= s_3 \\
    &\rightarrow 0.5258 \\
    h_3 &= 257.3 \\
    &\rightarrow 0.96 \\
    X &= \frac{s_3 - s_f}{s_g - s_f} \\
    &\rightarrow 0.96
\end{align*}

\subsection*{c)}
\begin{align*}
    s_4 &= s_1 \\
    p_4 &= p_1 \\
    s_4 \, \text{at} \, 8 \, \text{kPa} \, x = 0 &\rightarrow 0.3455 \\
    X_1 &= \frac{s_1 - s_f}{s_g - s_f} \\
    T_1 &= 263.15 \\
    &\rightarrow X = 0.773
\end{align*}

``````latex


\begin{itemize}
    \item[d)] 
    \begin{equation*}
        E_u = \frac{Q_{zu}}{W_t} = \frac{Q_{u}}{W}
    \end{equation*}
    
    \begin{equation*}
        Q = \dot{m} (h_1 - h_2) + Q
    \end{equation*}
    
    \begin{equation*}
        Q = \dot{m} (h_2 - h_1) \Rightarrow
    \end{equation*}
    
    \begin{equation*}
        h_1 = h_4 + x (h_5 - h_4)
    \end{equation*}
    
    \begin{equation*}
        h_2 = h_5 - 237,74
    \end{equation*}
    
    \item[e)] 
    \begin{equation*}
        Würde kälter werden bis ein Gleichgewicht erreicht
    \end{equation*}
\end{itemize}

```