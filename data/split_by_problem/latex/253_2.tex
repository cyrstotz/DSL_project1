
``````latex


\section*{A2, Seite 1}

\subsection*{Aufgabe 2}

\subsubsection*{a)}

\begin{description}
    \item[Graph:] The graph is a Temperature-Entropy (T-S) diagram. The x-axis is labeled $S \left[ \frac{kJ}{kg \cdot K} \right]$ and the y-axis is labeled $T \left[ K \right]$. The graph consists of a closed loop with six points labeled 1 through 6. The points are connected by arrows indicating the direction of the process. The points are connected as follows:
    \begin{itemize}
        \item Point 1 to Point 2: Upward sloping line.
        \item Point 2 to Point 3: Vertical line upwards.
        \item Point 3 to Point 4: Downward sloping line.
        \item Point 4 to Point 5: Vertical line downwards.
        \item Point 5 to Point 6: Downward sloping line.
        \item Point 6 to Point 1: Upward sloping line.
    \end{itemize}
    Points 2 and 5 are labeled with $P_2 = P_5$, and Points 1 and 3 are labeled with $P_1 = P_3$. Point 0 is labeled at the bottom left corner of the graph.
\end{description}

\subsubsection*{b)}

\textbf{1. HS Am Gesamtsystem:}

\[
0 = \dot{m}_0 \left( h_0 + \frac{\omega_{\text{Luft}}^2}{2} \right) - h_6 - \frac{\omega_6^2}{2}
\]

\[
\Rightarrow 2 \left( h_0 - h_6 + \frac{\omega_{\text{Luft}}^2}{2} \right) = w_6
\]

\[
h_0 - h_6 = c_p, \text{Luft} \left( T_0 - T_6 \right)
\]

\textbf{1. HS an der Schubdüse:}

\[
\cancel{w_s} = \cancel{w_s} \Rightarrow T_5 = T_6
\]

``````latex

\[
0 = h_5 - h_6 + w_s - w_6
\]

\[
= c_{p, \text{luft}} (T_5 - T_6) + w_s - w_6
\]

\text{reversible, adiabate Schubdüse}

\[
\Rightarrow \frac{T_6}{T_5} = \left( \frac{p_6}{p_5} \right)^{1 - \frac{1}{\kappa}}
\]

\[
\Rightarrow T_6 = \left( \frac{p_0}{p_5} \right)^{1 - \frac{1}{\kappa}} \cdot T_5 = 328{,}07 \, \text{K}
\]

\[
\Rightarrow h_0 - h_6 = -85{,}43 \, \frac{\text{kJ}}{\text{kg}}
\]

\[
\Rightarrow w_6 = 510 \, \frac{\text{m}}{\text{s}}
\]

c) \(\Delta e_{\text{xs,irr}}\)

\[
\Delta e_{\text{xs,irr}} = (h_6 - h_0 - T_0 (s_6 - s_0) + \Delta ke) \Rightarrow 0
\]

\[
= \left( c_p (T_6 - T_0) - T_0 \left( c_p \ln \left( \frac{T_6}{T_0} \right) - R \ln \left( \frac{p_6}{p_0} \right) \right) + \Delta ke \right)
\]

\[
\approx 125{,}97 \, \frac{\text{kJ}}{\text{kg}}
\]

``````latex


\section*{Aufgabe 2}

\subsection*{d)}
Entropiebilanz um Brennkammer um $\dot{m}_K$ auszurechnen

\[
0 = \dot{m}_K (s_2 - s_3) + \frac{\dot{Q}_B}{T_B} + \dot{S}_{erz}
\]

``````latex


