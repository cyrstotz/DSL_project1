
``````latex


\section*{Aufgabe 4}

\subsection*{a)}

\begin{description}
    \item[Graph Description:] The graph is a phase diagram with pressure (P) on the y-axis and temperature (T) on the x-axis. The y-axis is labeled "P [mBAR]" and the x-axis is labeled "T [°C]". The graph shows three regions labeled "FEST" (solid), "FLÜSSIG" (liquid), and "GAS" (gas). The "FEST" region is on the left, the "FLÜSSIG" region is on the top right, and the "GAS" region is on the bottom right. The graph includes a line that starts from the origin and curves upwards, representing the sublimation point. There is a point labeled "TRIPEL" at 5 mBAR and 0°C. Two vertical dashed lines are drawn at -20°C and 0°C. There are two horizontal arrows labeled "I" and "II" indicating transitions between phases. The arrow "I" points downwards from the "FLÜSSIG" region to the "GAS" region, and the arrow "II" points downwards from the "FEST" region to the "GAS" region. The temperature difference between the two arrows is labeled "10K".
\end{description}

\subsection*{b)}

\[
\dot{m}_{R134A} = 4 \frac{kg}{h}, \quad T_2 = -22^\circ C
\]

\[
p_2 = p(-22^\circ C) = 1.2192 \, \text{BAR}
\]

\[
h_2 = h_2 \quad \Rightarrow \quad T_1 = T_2
\]

\[
h_4 = h_1 \quad \text{DROSSEL IST ISENTHALP} \quad p_3 = p_4
\]

\[
h_{4} = h_{f} (8 \, \text{BAR}) = 93.42 \, \frac{kJ}{kg}
\]

\[
x_1 = \frac{h_1 - h_f (-22^\circ C)}{h_{fg} (-22^\circ C)} \approx 0.337
\]

\subsection*{d)}

\[
\epsilon_K = \frac{\left| \dot{Q}_K - \dot{Q}_{AB} \right|}{\dot{W}_K}
\]

``````latex


\section*{Problem b)}

\begin{align*}
    T_1 &= -20^\circ C \\
    \dot{Q} &= T_1 \rightarrow T_2 = -16^\circ C \\
    s_2 &= 0.9298 \frac{kJ}{kg \cdot K} \\
    s_2 &= s_3 \\
    h_2 &= h_G (-16^\circ C) = 237.74 \frac{kJ}{kg} \\
    h_3 &= h(8 \text{BAR}, 40^\circ C) - h(8 \text{BAR}, 31.33^\circ C) \\
    &\quad - (s_2 - s(8 \text{BAR}, 31.33^\circ C)) + h(8 \text{BAR}, 31.33^\circ C) \\
    &\approx 271.31 \frac{kJ}{kg}
\end{align*}

\section*{Energy Balance Compressor}

\begin{align*}
    \dot{m}_{R134A} &= \frac{-28 \dot{W}}{h_2 - h_3} \approx 3 \frac{kg}{h}
\end{align*}

\section*{Part c)}

\begin{quote}
    Die Temperatur wird weiter sinken bis $\dot{W}_k$ nicht mehr reicht um die Wärmeabfuhr weiter zu erhalten.
\end{quote}

```