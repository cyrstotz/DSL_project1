
``````latex


\section*{Aufgabe 2:}

\subsection*{a) T-s-Diagramm}

\begin{description}
    \item[Graph Description:] The graph is a T-s diagram with the y-axis labeled as $T [K]$ and the x-axis labeled as $s \left[ \frac{kJ}{kgK} \right]$. The graph contains several curves and points labeled from 0 to 6. The points are connected by lines indicating different processes. The points are labeled as follows:
    \begin{itemize}
        \item Point 0 is at the bottom left.
        \item Point 1 is directly above point 0.
        \item Point 2 is above point 1, with a vertical line connecting them.
        \item Point 3 is to the right of point 2, connected by a curve.
        \item Point 4 is below point 3, connected by a vertical line.
        \item Point 5 is to the right of point 4, connected by a curve.
        \item Point 6 is below point 5, connected by a vertical line.
    \end{itemize}
    The graph also includes annotations such as $p_2 = p_3$, $0.5 \text{ bar}$, and $1.5$.
\end{description}

\begin{table}[h!]
    \centering
    \begin{tabular}{|c|c|c|c|c|}
        \hline
        Zust. & P & T [K] & S & h \\
        \hline
        1 & 0.5 bar & & & \\
        \hline
        2 & & & $s_1 = s_2$ & \\
        \hline
        3 & & & & \\
        \hline
        4 & 0.5 bar & & & \\
        \hline
        5 & 0.5 bar & 43.9 & & \\
        \hline
        6 & 0.19 & 328.07 & $s_6 = s_5$ & \\
        \hline
        0 & 0.19 & & & \\
        \hline
    \end{tabular}
\end{table}

\noindent Ideales Gas: Isentrope $K = 1.4$

\[
T_6 = T_5 \cdot \left( \frac{p_6}{p_5} \right)^{\frac{0.4}{1.4}}
\]

``````latex

b) $W_0, T_0$

1. HS am Turbinen: \\
stationär: $dE = 0$ \\
keine Arbeit: $Q = 0$ \\
\[
0 = \dot{m} \left( h_a - h_c + \frac{w_e^2 - w_0^2}{2} \right) \Rightarrow 2(h_a - h_c) + \frac{w_e^2}{2} = \frac{w_0^2}{2}
\]
\[
-2 \dot{m} \left[ h_a - h_c \right] + w_e^2 = w_0^2
\]
\[
W_0 = \sqrt{+2 \dot{m} \left[ h_a - h_c \right] + w_e^2}
\]
\[
h_a = \\
h_c = \left\{ h_0 - h_c = c_p \left[ T_0 - T_c \right] = -85.43 \right\}
\]
\[
T_0 = -30^\circ
\]
\[
T_5:
\]
\[
1. HS am Schubdüse:
\]
\[
T_c = T_5 \left( \frac{p_c}{p_5} \right)^{\frac{\gamma - 1}{\gamma}} = 328.075 K = 54.925^\circ C
\]
\[
p_6 = 0.95 bar
\]
\[
p_5 = 0.85 bar
\]
\[
h_5 - h_c = c_p \left[ T_5 - T_c \right] = 109.418
\]

b) 1 HS am Schubdüse:
\[
0 = \dot{m} \left[ h_5 - h_c + \frac{w_5^2 - w_6^2}{2} \right]
\]
\[
h_0 - h_5 = \frac{w_5^2}{2} - \frac{w_6^2}{2} \Rightarrow w_6 = \sqrt{2(h_5 - h_c) + w_5^2}
\]
\[
= 160 \frac{m}{s}
\]

``````latex


\section*{Aufg 2:}

\subsection*{c)}
\[
0 = h_6 - h_0 - T_0 (s_6 - s_0) + A_{he}
\]

\subsection*{d)}
\[
\Delta x_{re} = \Delta x_{estr.}
\]
\[
\Delta x_{ev} = \Delta x_{extr.} = 100 \frac{kJ}{kg}
\]

``````latex


