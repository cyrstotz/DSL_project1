
``````latex


3. 

a) $p_1 = 1.9 \text{bar}$

$m_1 = m_2 = 0.1 \text{kg}$

\[
U_1 = U(0^\circ C, 1.9 \text{bar}) = U_{fg}(0^\circ C, 1.9 \text{bar}) + x(U_g(0^\circ C, 1.9 \text{bar}) - U_f(0^\circ C, 1.9 \text{bar}))
\]

\[
= -133.4 \frac{kJ}{kg}
\]

\[
x_2 = \frac{U_2 - U_{est}}{U_{Düssig} - U_{est}}
\]

\[
U_{2g} = \frac{m \cdot R \cdot T_2}{p} = 1.11 \cdot 10^{-3} \text{m}^3
\]

\[
W = p_1 (V_2 - V_1) = 28.27J
\]

\[
m_1 = m_2 = m_w
\]

\[
m_2 U_2 - m_w U_w = Q_{12} - W_{12}
\]

\[
U_2 = \frac{Q_{12} - W_{12}}{m_w} + U_w
\]

\[
= -122.5 \frac{kJ}{kg}
\]

\[
x_2 = \frac{U_2 - U_{est}}{U_{Düssig} - U_{est}} = 63.28\%
\]

``````latex


\section*{Aufgabe 3: 1.}

\begin{equation*}
    F = p \cdot A
\end{equation*}

\[
\begin{array}{c}
\text{Graphical description:} \\
\text{A vertical rectangle divided into three sections. The top section contains a mass labeled "M"} \\
\text{with an arrow pointing downwards labeled "mg". The middle section is labeled "E w P_1"} \\
\text{and the bottom section is labeled "P_2".}
\end{array}
\]

\[
\begin{array}{c}
\text{Graphical description:} \\
\text{A horizontal rectangle with a mass "M" on top. The top of the rectangle is labeled "p_{amb}".} \\
\text{There are arrows pointing upwards from the bottom of the rectangle labeled "P_1".}
\end{array}
\]

\begin{equation*}
    \sum F = p_{amb} \pi \left( \frac{d}{2} \right)^2 + mg = P_1 \pi \left( \frac{d}{2} \right)^2
\end{equation*}

\begin{equation*}
    P_1 = p_{amb} + \frac{4mg}{\pi d^2}
\end{equation*}

\begin{equation*}
    P_1 = 1.20 \, \text{bar}
\end{equation*}

\[
\begin{array}{c}
\text{Graphical description:} \\
\text{A horizontal rectangle with a mass "m_{EW}" on top. The top of the rectangle is labeled "P_1".} \\
\text{There are arrows pointing upwards from the bottom of the rectangle labeled "P_2".}
\end{array}
\]

\begin{equation*}
    \sum F = P_1 \pi \left( \frac{d}{2} \right)^2 + m_{EW} g = P_2 \pi \left( \frac{d}{2} \right)^2
\end{equation*}

\begin{equation*}
    P_2 = P_1 + \frac{4 m_{EW} g}{\pi d^2}
\end{equation*}

\begin{equation*}
    P_2 = 1.40 \, \text{bar}
\end{equation*}

\begin{equation*}
    m_g = \frac{P_2 V}{R T_{g1}}
\end{equation*}

\begin{equation*}
    R = \frac{R}{M} = 166.28 \, \frac{J}{kg \cdot K}
\end{equation*}

\begin{equation*}
    m_g = \frac{1.40 \, \text{bar} \cdot 50 \cdot 10^{-3} \, \text{m}^3}{166.28 \, \frac{J}{kg \cdot K} \cdot 773.15 \, \text{K}}
\end{equation*}

\begin{equation*}
    = 3.42 \cdot 10^{-3} \, \text{kg}
\end{equation*}

``````latex


b) Da der Zustand immer noch im Nassdampfgebiet liegt bleibt \\
die Temperatur und der Druck konstant. $T_2 = 0^\circ C \quad p_2 = 1.40 \, \text{bar}$

c) 
\[
C_{p, \text{gas}} = \frac{R}{M} + c_v = 7.99 \cdot 28 \, \frac{\text{J}}{\text{kg} \cdot \text{K}}
\]

\[
m_f (h_{f2} - h_{f1}) = m_f (h_{m2} - h_{m1}) \quad \text{(delta Enthalpie von Temperatur ab)}
\]

\[
m_s c_p (T_{m2} - T_2) = m_s c_p (T_{m2} - T_2)
\]

Da im Zustand 2 Wasser immer noch bei $0^\circ C$ ist und Zustand 2 \\
der Gaszustand ist, ist das Gas auch bei $0^\circ C$

\[
Q_{12} = m_g (h_{2g} - h_{2f})
\]

\[
= m_g C_{p,g} (T_2 - T_m)
\]

\[
= 1366.77 \, \text{J}
\]

\[
T_2 - T_m = 500 \, \text{K}
\]

``````latex


