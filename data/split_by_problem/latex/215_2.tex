
``````latex


\section*{Aufgabe 2}

\subsection*{a)}

\[
\begin{array}{c}
\begin{array}{c|c}
T \, [K] & \\
\hline
& P_2 = P_3 \\
& P_4 = P_5 = P_1 = 0.5 \, \text{bar} \\
& \text{isobaren} \\
& P_0 = P_6 = 0.191 \, \text{bar} \\
\end{array}
\end{array}
\]

\noindent
\textbf{Verbal Description of the Graph:}

The graph is a Temperature (T) vs. Entropy (S) diagram. The vertical axis is labeled \( T \, [K] \) and the horizontal axis is labeled \( S \, \left[ \frac{\text{kJ}}{\text{kg} \cdot \text{K}} \right] \). 

There are six points labeled 1 through 6. The points are connected as follows:
- Point 1 is connected to Point 2 with a vertical line.
- Point 2 is connected to Point 3 with a line that slopes upwards to the right.
- Point 3 is connected to Point 4 with a horizontal line.
- Point 4 is connected to Point 5 with a line that slopes downwards to the right.
- Point 5 is connected to Point 6 with a vertical line.
- Point 6 is connected to Point 1 with a curved line.

The points are positioned such that:
- \( S_1 = S_2 \)
- \( S_5 = S_6 \)

The pressures at various points are indicated as:
- \( P_2 = P_3 \)
- \( P_4 = P_5 = P_1 = 0.5 \, \text{bar} \)
- \( P_0 = P_6 = 0.191 \, \text{bar} \)

\subsection*{b)}

\[
w_6: \text{reversible \& adiabate Schubdüse:}
\]

\[
\text{En. Bil. um Schubdüse:}
\]

\[
0 = \dot{m} (h_e - h_a + \frac{w_e^2 - w_a^2}{2}) \quad I
\]

\[
\text{Entr. Bil.:} \quad 0 = \dot{m} (s_e - s_a) \Rightarrow s_e = s_a
\]

\[
\Rightarrow \text{für } ig. \text{ polyt. Zust. verh.:}
\]

\[
T_6 = T_s \left( \frac{P_6}{P_s} \right)^{\frac{n-1}{n}}
\]

\[
328.07 \, K = T_6
\]

\[
I: \quad 2 \left( h_6 - h_5 \right) = \frac{w_5^2 - w_6^2}{2}
\]

\[
w_6^2 = w_5^2 - 2 \left( h_6 - h_5 \right)
\]

\[
\Rightarrow w_6 = \sqrt{w_5^2 - 2 \left( h_6 - h_5 \right)}
\]

\[
w_6 = \sqrt{507.24 \, \frac{\text{kJ}}{\text{kg}}}
\]

\[
\text{Cp:} \quad (T_6 - T_5)
\]

\[
\left[ \frac{\text{kJ}}{\text{kg} \cdot \text{K}} \right] = 104.44 \, \frac{\text{kJ}}{\text{kg}}
\]

``````latex


\section*{c)}

\[
\Delta ex_{str} = ex_{str6} - ex_{str0}
\]

\[
= h_6 - h_0 - T_0 \cdot (S_6 - S_0) + \Delta ke
\]

\[
h_6 - h_0 = c_p \cdot (T_6 - T_0)
\]

\[
S_6 - S_0 = c_p \cdot \ln \left( \frac{T_6}{T_0} \right) - R \cdot \ln \left( \frac{p_6}{p_0} \right)
\]

\[
\begin{array}{c}
c_p \cdot (T_6 - T_0) \\
85.43 \\
\left[ \frac{kJ}{kg} \right] \\
\text{c}
\end{array}
\]

\[
\begin{array}{c}
c_p \cdot \ln \left( \frac{T_6}{T_0} \right) \\
73.28 \\
\left[ \frac{kJ}{kg} \right] \\
\text{b}
\end{array}
\]

\[
\frac{W_6^2 - W_0^2}{2} \\
\frac{500 \left[ \frac{J}{kg} \right] - 200 \left[ \frac{J}{kg} \right]}{2} \cdot \frac{1}{1000} \\
217.297 \left[ \frac{kJ}{kg} \right] \\
\Delta ke \\
\text{a}
\]

\[
\boxed{\Delta ex_{str} = 229.455 \left[ \frac{kJ}{kg} \right]}
\]

\section*{d)}

\text{Exergibilit\"at: rechne mit } 100 \left[ \frac{kJ}{kg} \right] \text{ weiter f\"ur } \Delta ex_{str}

\[
ex_{el} = -\Delta ex_{str} + \left( 1 - \frac{T_0}{\overline{T}} \right) \cdot \dot{Q} \cdot 9
\]

\[
\text{aus c } 100 \left[ \frac{kJ}{kg} \right]
\]

\[
T_0 = 243.15 K \\
\overline{T} = 128.9 K
\]

\[
\dot{Q} = 1195 \left[ \frac{kJ}{kg} \right]
\]

\[
\text{Achtung es wird nur auf } \frac{1}{1 + 5.293} = a \text{ die W\"arme } \dot{Q} \text{ \"ubertragen}
\]

\[
\Rightarrow ex_{el} = -\Delta ex_{str} + \left( 1 - \frac{T_0}{\overline{T}} \right) \cdot \frac{\dot{Q}}{1 + 5.293}
\]

\[
\boxed{ex_{el} = 54.073}
\]

``````latex


