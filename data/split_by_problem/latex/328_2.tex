
``````latex


\section*{Student Solution}

\begin{itemize}
    \item The first graph is a plot of \( T \) (temperature) versus \( s \) (entropy). The vertical axis is labeled \( T \) and the horizontal axis is labeled \( s \). The graph starts at the origin, rises to a peak, and then falls back down, forming a single peak. There is a diagonal line that seems to represent a reference or boundary.
    
    \item The second graph is also a plot of \( T \) (temperature) versus \( s \) (entropy). The vertical axis is labeled \( T \) and the horizontal axis is labeled \( s \). The graph consists of four points connected by lines: 
        \begin{itemize}
            \item Point 1 is at the bottom left.
            \item Point 2 is directly above Point 1, forming a vertical line.
            \item Point 3 is to the right of Point 2, forming a horizontal line.
            \item Point 4 is below Point 3, forming a vertical line.
        \end{itemize}
        There is a diagonal line that seems to represent a reference or boundary.
    
    \item The third graph is a more complex plot of \( T \) (temperature) versus \( s \) (entropy). The vertical axis is labeled \( T \) and the horizontal axis is labeled \( s \). The graph consists of six points connected by lines:
        \begin{itemize}
            \item Point 1 is at the bottom left.
            \item Point 2 is directly above Point 1, forming a vertical line.
            \item Point 3 is to the right of Point 2, forming a horizontal line labeled "isobar".
            \item Point 4 is below Point 3, forming a vertical line.
            \item Point 5 is to the right of Point 4, forming a downward-sloping line.
            \item Point 6 is below Point 5, forming another downward-sloping line.
        \end{itemize}
        There is a diagonal line that seems to represent a reference or boundary. Additionally, there is a label \( T_3 = 431.94 \) near the vertical axis.
\end{itemize}

``````latex


