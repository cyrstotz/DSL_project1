
``````latex

3) a) $p_c, n, m_c$ gegeben

\[
p \cdot V = m \cdot R \cdot T \quad \Rightarrow \quad m_c = \frac{p \cdot V}{R \cdot T}
\]

\[
R = \frac{R}{M} = 166.28 \, \frac{J}{kg \cdot K}
\]

\[
= 0.050 \, \frac{kg}{mol}
\]

\[
0.00314L
\]

\[
T = 738.15K
\]

\[
m = 2.687g
\]

\[
p_6 \cdot A = p_0 \cdot A + m_c \cdot g \quad \Rightarrow \quad p_G = 1.100 \, bar
\]

\[
A = (0.1m)^2 \cdot \pi
\]

b) $p_a = 1.5 \, bar, \quad m_c = 3.6 \, g$

\[
p \cdot V = m \cdot R \cdot T
\]

\[
p_2 = p_1 \quad \Rightarrow \quad \text{Der Druck muss gleich bleiben, da das System im Gleichgewicht ist und sich weder das Gewicht des Kolbens noch der Luftdruck verändert.}
\]

\[
T = \frac{p \cdot V}{m \cdot R}
\]

\[
\text{Temperatur gleich bleiben.}
\]

``````latex


