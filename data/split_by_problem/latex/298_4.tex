
``````latex


\section*{Aufgabe 4}

\subsection*{a)}

\begin{itemize}
    \item The graph is a Pressure-Temperature (P-T) diagram.
    \item The y-axis is labeled as $p$ (pressure) and the x-axis is labeled as $T$ (temperature in $^\circ$C).
    \item The graph shows a curve starting from the origin (0,0) and rising upwards.
    \item There is a horizontal line labeled "Schmelzkurve" (melting curve) intersecting the y-axis.
    \item The curve intersects this horizontal line at a point labeled "Tripelpunkt" (triple point).
    \item To the left of the curve, the region is labeled "Fest" (solid).
    \item To the right of the curve, the region is labeled "Gas".
    \item The curve continues upwards and to the right, labeled "Siedekurve" (boiling curve).
    \item The intersection of the curve with the horizontal line is marked with a vertical line labeled "Tripel".
    \item The region between the horizontal line and the curve is labeled "Flüssig" (liquid).
    \item The point where the curve ends on the right is labeled $T_c$.
    \item There are arrows indicating the direction of phase transitions: solid to liquid (left to right) and liquid to gas (upwards).
\end{itemize}

\subsection*{b)}

\begin{itemize}
    \item $m_{\text{Kühl}} = m_{\text{K}}$
    \item $W_{\text{K}} = 28 \, \text{W} \, \text{ohne} \, \text{gedampf} \, \text{oder} \, \text{ist} \, \text{verzogen} \, \text{auf} \, 85 \, \text{sec} \, \text{zu} \, \text{lange}.$
    \item $25 \, \text{U}: \, \text{voll} \, \text{kondensator} \, \beta = 8 \, \text{sec}$
    \item $1 \rightarrow 2 \, \text{indoc}$
    \item $E_{\text{Kühl}} = \dot{Q}_{\text{Kühl}}$
    \item $m_{\text{Kühl}} = \frac{-Q_{\text{Kühl}}}{h_2 - h_1}$
    \item $\frac{dE}{dt} = m_{\text{Kühl}} (h_2 - h_1) + Q_{\text{AS}}$
    \item $h_3 = h(8 \, \text{sec})$
    \item $h_4 = h(\text{st.} \, t, \, 8 \, \text{sec})$
    \item $p_3 - p_1 \, \text{da} \, \text{lindert} \, \text{wieder}$
\end{itemize}

``````latex


\section*{c)}

\[
\dot{m} = \frac{\dot{Q}_u}{\Delta h_{u/s}}
\]

\[
T_2 = -22^\circ C
\]

\[
X_1 \text{ des Kältemittels}
\]

\[
X_1 = \frac{S_{1u} - S_{f1}}{S_{g1} - S_{f1}}
\]

\[
S_{1u} = S_{4}, \text{ da die Drossel adiabatisch-isentrop}
\]

\[
S_{f1} = S_{f} (T_2) = 0.3463 \frac{kJ}{kgK}
\]

\[
S_{f} = S_{f} (T_2 = -22^\circ C)
\]

\[
S_{4} = S_{1} \text{ ist gleich der in Zeile 2, da die Kompression 1 $\rightarrow$ 2.}
\]

\section*{d)}

\[
E_k = \frac{\dot{Q}_u}{W_{t}} = \frac{\dot{Q}_k}{W_{k}}
\]

\section*{e)}

Würde der Druck nicht geändert und dann die Wärmeübertragung abgestellt, würde der Zustand immer weiter in Zweistoffgebiet bleiben und die Temperatur würde weiterhin absinken.

```