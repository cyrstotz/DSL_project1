
``````latex


3. $\overline{P}_{S7}$, $m_S$

a)
\[
\frac{m_k S}{A} + \frac{m_{E V} S}{A} + P_o = P_{S7}
\]
\[
A = \frac{dE D^2}{4} \cdot \pi
\]
\[
\Rightarrow P_{S7} = P_{emo} + \frac{3A}{A} (m_k + m_{EV}) = 760 d
\]
\[
P_{S7} = 1.401 \text{bar}
\]
\[
m_S = \frac{\rho V}{R T} = \frac{P_{S7} V_{S7}}{R T_{S7}} = \frac{1.401 \cdot 3.14 \cdot 10^{-3} \cdot 50 \text{g/mol}}{9.81 \text{g} / \text{cm}^3 \cdot V_k \cdot (50 + 273.15) \text{K}}
\]
\[
\Rightarrow m_S = 3.42 \text{g}
\]

b) $\overline{T}_2$, $P_2$
\[
P_2 \overset{!}{=} P_1 \text{ da das Kräfte/Druck-Gleichgewicht immer nur das gleiche besetzt.}
\]
\[
P_{S2} = 1.401 \text{bar}
\]
\[
\text{Falls das } X_{S2} > 0, \text{ muss die Temperatur von EV immer noch bei 0°C liegen.}
\]
\[
\text{Da bei Zustand 2 } Q_{ab} = 0 \text{ ist, muss auch } T_{S2} = 0°C \text{ bleiben.}
\]

c)
\[
Q_{12} = C_p \cdot m_S \cdot (T_2 - T_1) \quad \left( C_p = \frac{\pi}{M_S} + C_v \right)
\]
\[
= \left( \frac{\pi}{M_S} + C_v \right) \cdot (T_2 - T_1) = \left( \frac{8.31 \cdot V_{S7} \cdot T_{S7}}{5 \cdot V_k \cdot m_S} + 0.63 \cdot V_{S7} / \text{K} \right) \cdot 3.42 \text{g} \cdot (50 \text{J} / \text{K})
\]
\[
= 1366.9 \text{J}
\]

d)
\[
U_{JE} M_{S7} + 0.6 \cdot M_{S7} + 0.4 \cdot M_{S7} - Q_{12} = Q_{12} = U_{JE} \cdot X \cdot M_{S7} + U_{JE} \cdot (7 - X) \cdot M_{S7}
\]
\[
0°C \text{ (Zustand 1)}
\]
\[
\Rightarrow U_{JE} - 0.6 + 0.4 \cdot U_{JE} = Q_{12} / m_{S7} = X_2 \cdot U_{JE} + U_{JE} \cdot (4 - X_1) - X_{S7}
\]
\[
\Rightarrow X_2 = \frac{0.6 \cdot U_{JE} + 0.4 \cdot U_{JE} - Q_{12} / m_{S7} - U_{JE}}{U_{JE} - U_{JE}}
\]
\[
X_2 = 0.560
\]

``````latex


