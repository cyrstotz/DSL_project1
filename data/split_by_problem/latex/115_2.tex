
``````latex


\section*{Aufgabe 21}

\subsection*{a)}

\textbf{Graph 1:} 

The graph is a plot with the vertical axis labeled \( T \) and the horizontal axis labeled \( S \). The plot starts at the origin and rises steeply, then curves back down, forming a loop. The loop is labeled with points 1, 2, and 3, with arrows indicating the direction of movement through these points. The graph then continues to rise again after the loop.

\textbf{Graph 2:}

The second graph is a plot with the vertical axis labeled \( T \) with units in Kelvin \([K]\) and the horizontal axis labeled \( S \) with units in \(\left[\frac{K}{\log K}\right]\). The plot starts at the origin and rises steeply to point 1, then curves back down to point 2, rises again to point 3, and then falls to point 4. The graph then rises slightly to point 5 and falls again to point 6. The points are labeled sequentially from 1 to 6, with arrows indicating the direction of movement through these points. There is a note below the graph stating "1 - höhere Temperatur als 4".

``````latex


\section*{1. Aufgabe: Wo: T0}

\textbf{Energiegleichung: Stationärer Flussprozess, 1. Hauptsatz}

\[
0 = \dot{m} \left[ h_2 - h_1 \right] + \frac{w_s^2}{2} - \frac{w_6^2}{2} + \sum \dot{Q} - \sum \dot{W}_{ein}
\]

\[
\sum \dot{Q} = 0, \text{ da adiabatisch}
\]

\[
\sum \dot{W}_{ein} - \dot{W}_{ver} = \dot{m} \left( \int_{1}^{2} v dp \right)_{n=n} - \int_{1}^{2} \left( \dot{m} \cdot n \cdot \left( \int_{1}^{2} p dv \right) \right)
\]

\[
= - \dot{m} \cdot \frac{R \cdot n \left( T_2 - T_1 \right)}{1 - n}
\]

\[
n = k = 1.4
\]

\[
\frac{w_s}{w_{ein}} = \frac{w_{ver}}{w_{12}} - 1 \Rightarrow \eta_{th}
\]

\[
\boxed{}
\]

``````latex


