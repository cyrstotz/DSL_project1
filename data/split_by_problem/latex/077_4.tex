
``````latex


\section*{Aufgabe 4}

\subsection*{a)}

\begin{itemize}
    \item \textbf{Graph Description:} The graph is a phase diagram with pressure \( p \) on the y-axis and temperature \( T \) on the x-axis. The y-axis is labeled with "mbar" and has a value of 5 mbar marked. The x-axis is labeled with "°C". There are three regions labeled "solid", "liquid", and "gas". The "solid" region is on the left, the "liquid" region is in the middle, and the "gas" region is on the right. The boundary lines between these regions are curved. The triple point is marked where all three regions meet. There are three points labeled (1), (2), and (3) on the graph. Point (1) is in the "solid" region, point (2) is on the boundary between "solid" and "gas", and point (3) is in the "gas" region. Arrows indicate transitions between these points.
\end{itemize}

1) isobar geforen: \( p \) constant \\
   \( T \) geht runter \\
   Zustand: solid

2) isotherm sublimiert: \( p \) geht runter \\
   Zustand: gas \\
   \( T \) ist constant

\subsection*{b)}

\[
\dot{m}_{R134a} = ?
\]

Zustand (2-3): isentrop, weil es reversibel \& adiabatisch ist.

\[
\dot{m}_{R134a} (h_e - h_a) + \sum_i \dot{Q}_i - \sum_n \dot{W}_n = 0 \quad \text{(Der Prozess ist stationär)}
\]

Energiebilanz bei der Drossel (isentrop).

\[
\dot{m}(0-) = 0 \quad \text{Energiebilanz beim Verdichter}
\]

\[
\dot{m}(h_2 - h_3) = 28W
\]

``````latex


\[
\dot{m} (h_2 - h_3) = 28 \, \text{W} \Rightarrow \dot{m} = \frac{28 \, \text{W}}{h_2 - h_3}
\]

\[
x_2 = 1 \Rightarrow \text{vollständig Dampf (sättigt)}
\]

\text{Was ist } T_i \, ?

\[
\text{5 mbar unter Tripel- und 10 K über Sublim.}
\]

\[
\Rightarrow p = 1 \, \text{mbar} \quad \text{und} \quad T_i = -10^\circ \text{C} = 263 \, \text{K}
\]

\[
\Rightarrow T_{\text{Verdampfer}} = 257 \, \text{K}
\]

\[
h_2 = h_{2g} \quad h_3 = h_{3,f} \, (\text{vollständig komprimiert})
\]

\[
\text{bei 8 bar.}
\]

\[
h_{3,f} = h_3 = 93.42 \, \text{(A.11)}
\]

\text{Die Entropie von 2-3 bleibt gleich} \Rightarrow s_2 = s_3

\[
s_3 = s_{3,f} = 0.3459
\]

\text{bei} \quad x_2 = 1 \Rightarrow s_{3,f} = s_{2,g} = s_2 = 0.3459 \Rightarrow p_2 = ?
\]

``````latex


\section*{4c)}

\begin{align*}
x_1 & \text{?} \\
\dot{m}_{\text{p,ab}} & = \dot{m} \frac{h_2}{h_1} \\
T_2 & = -22^\circ \text{C}
\end{align*}

\begin{align*}
x_1 & = 0 \Rightarrow \text{vollständig flüssig.} \\
x_1 & = \frac{s_1 - s_{1,f}}{s_{1,g} - s_{1,f}}
\end{align*}

Die Drossel ist adiabatisch mit \( p = \text{Außendruck} \).

\textbf{Entropiebilanz} \( \dot{S} \)

\begin{align*}
\dot{m} (s_4 - s_1) + \sum \frac{\dot{Q}_i}{T_i} + \dot{S}_{\text{erzeugt}}
\end{align*}

``````latex


\section*{4d)}
\begin{equation*}
E_K = \left| \frac{\dot{Q}_{zu}}{\dot{W}_K} \right|
\end{equation*}

\section*{2)}
Die Temperatur wurde senken.

```