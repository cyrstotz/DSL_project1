
``````latex


\section*{Aufgabe 4}

\subsection*{a)}

\begin{description}
    \item[Graph Description:] The graph is a Pressure-Temperature ($p$-$T$) diagram. The x-axis is labeled $T$ [°C] and ranges from -50 to 10. The y-axis is labeled $p$ [bar] and ranges from 0.01 to 10. The graph shows a phase diagram with a curve separating the liquid and gas phases. The curve starts at the bottom left and curves upwards to the right. The region below the curve is labeled "gas" and the region above the curve is labeled "flüssig". There are several points and lines marked on the graph:
    \begin{itemize}
        \item Point 1 is on the curve and is labeled "1".
        \item Point 2 is above the curve and is labeled "2".
        \item Point 3 is below the curve and is labeled "3".
        \item A vertical line labeled "isobar" connects points 2 and 3.
        \item A horizontal line labeled "isotherm" connects points 1 and 2.
        \item A vertical line labeled "GK" extends upwards from point 2.
    \end{itemize}
\end{description}

\subsection*{b)}

\begin{align*}
    \dot{m} R134a \\
    \text{Zustand 2: gesättigter, nasser Dampf} \rightarrow h_2 = \\
    \text{Zustand 3: } p_3 = 8 \text{ bar} \quad h_3 = 84.74 \frac{\text{kJ}}{\text{kg}} \quad (\text{Tabelle A-11}) \\
    \dot{m} [h_2 - h_3] = \dot{W}_k \\
    \text{Zustand 4: gerade vollständig kondensiertes Kältemittel, isobar: } p_3 = p_4 \\
    \rightarrow h_4 = 93.42 \frac{\text{kJ}}{\text{kg}} \\
    \dot{m} [h_3 - h_4] = \dot{Q}_{ab}
\end{align*}

``````latex


\begin{itemize}
    \item[c)] 
    \[
    x_1 = \frac{h_1 - h_f}{h_{2g} - h_f}
    \]
    \[
    x_4 = 0
    \]
    \[
    p_4 = h_4
    \]
    
    \item[d)] 
    \[
    \epsilon_{k} = \frac{\dot{Q}_{zu}}{\dot{Q}_{t}} = \frac{\dot{Q}_{zu}}{\dot{Q}_{zu} - \dot{Q}_{ab}} = \frac{\dot{Q}_{k}}{\dot{Q}_{ab} - \dot{Q}_{k}}
    \]
    
    \item[e)] 
    Sie würde solange sinken bis die Temperatur im Gefrierschrank bei 0K ist
\end{itemize}

```