
``````latex


\section*{4}
\subsection*{a)}
\begin{itemize}
    \item The graph has a title at the top left corner: \(h(T_2)\).
    \item The x-axis is labeled \([T/K]\) and the y-axis is labeled \(h\).
    \item There are two points marked on the y-axis: one at the origin (0) and another at a higher value (1).
    \item There are two points marked on the x-axis: one at the origin (0) and another at a higher value (2).
    \item The graph contains a curve that starts from the origin, oscillates, and then follows a roughly linear path upwards.
\end{itemize}

\subsection*{b)}
\begin{align*}
    &\text{d} \dot{V} = \text{stark} \\
    &\frac{dE}{dt} = \text{kinetik} + \dot{Q} + \dot{W} \\
    &0 = \dot{m}(h_2 - h_3) + \dot{W} \\
    &h_2 - h_3(T_2) \\
    &T_2 = -6^\circ C \\
    &\mu_3 = 813 \\
    &s_2 = s_3 \quad 100 \frac{kJ}{kgK} \\
    &\log(4) = 244.9 \quad \frac{kJ}{kg} \\
    &\log(8) = 292.54 \quad \frac{kJ}{kg} \\
    &h_2 = 292.54 + \frac{8 + 8}{-4.8} (244.9 - 292.54) \\
    &\quad = 243.72 \quad \frac{kJ}{kg} \\
    &s_2 = 0.925 \frac{kJ}{kgK} + \frac{8 - 6}{8 - 4} (0.925 - 0.9239) \\
    &\quad = 0.9226 \quad \frac{kJ}{kgK}
\end{align*}

\subsection*{h_3 @ 8 bar}
\begin{align*}
    &\text{TAB A-12 interpolation} \\
    &s_3 = 0.9226 \\
    &s_{sat} = 0.8066 \quad \frac{kJ}{kgK} \\
    &s_{40} = 0.9376 \quad \frac{kJ}{kgK} \\
    &h_3 = 264.75 + \frac{0.9276 - 0.8066}{0.9376 - 0.8066} (273.15 - 264.75) \\
    &\quad = 267.82 \quad \frac{kJ}{kg}
\end{align*}

``````latex


\[
W = \dot{m} (h_2 - h_1)
\]

\[
\dot{m} = \frac{W}{h_2 - h_1} = \frac{28 \cdot 10^{-3} \frac{kJ}{s}}{268.82 - 243.72} = 0.00 \overline{1}5 \frac{kg}{s}
\]

c)

\[
Y_t = \frac{s_1 - s_f}{s_2 - s_f}
\]

d)

\[
\eta_{II} = \frac{| \dot{Q}_1 |}{| \dot{W} |} = \frac{1 \cdot | \dot{Q}_1 |}{| \dot{Q}_1 | - | \dot{Q}_2 |}
\]

e) Wäre ungünstig, da man aus den gewonnenen gaslich versteht.
```