
``````latex


\section*{Problem 1}

\subsection*{a)}

\begin{description}
    \item[Graph Description:] The graph is a plot with the y-axis labeled as \( P \) and the x-axis labeled as \( T \). There are four regions marked on the graph:
    \begin{itemize}
        \item The top left region is labeled as "fest".
        \item The top right region is labeled as "flüssig".
        \item The bottom right region is labeled as "gasförmig".
    \end{itemize}
    There are two curves on the graph:
    \begin{itemize}
        \item The first curve starts from the "fest" region and curves upwards towards the "flüssig" region.
        \item The second curve starts from the "flüssig" region and curves downwards towards the "gasförmig" region.
    \end{itemize}
    There are also several arrows and annotations indicating transitions between the states.
\end{description}

\subsection*{b)}

\begin{align*}
    dE_{A}^0 \text{ der stationär} \quad & : \quad \sum_i \dot{m}_i \left( h_i + \frac{v_i^2}{2} + g z_i \right)^0 + \sum_j \dot{Q}_j \text{ der innen} \\
    \frac{dE^0}{dt} & = \dot{m}_i (h_e - h_a) + \dot{Q}_K - \dot{W}_K \\
    \Delta E & = \Delta U + \Delta KE + \Delta PE \\
    \dot{W}_K - \dot{Q}_K & = \dot{m} (h_e - h_a) \\
    \dot{W}_K & = 26 W \quad (\text{aus Aufgabe})
\end{align*}

\subsection*{c)}

\begin{equation*}
    x = \frac{\Phi_e - \Phi_f}{q_f}
\end{equation*}

\subsection*{d)}

\begin{equation*}
    \eta_{K} = \frac{\dot{Q}_{zu}}{\dot{W}_{t}} = \frac{[\dot{Q}_{ab} - \dot{Q}_{zu}]}{[\dot{Q}_{ab}]} = \left( \frac{\dot{Q}_K}{\dot{W}_K} \right)
\end{equation*}

\begin{equation*}
    \dot{Q}_K = \text{"Energieübergang"}
\end{equation*}

\begin{equation*}
    \dot{Q}_K = \dot{m} (h_e - h_a + \frac{v_e^2 - v_a^2}{2} + g(z_e - z_a)) \dot{Q}_K - Q_e
\end{equation*}

\begin{equation*}
    \frac{dE}{dt} = \sum_i Q_i - \sum W_n
\end{equation*}

\begin{equation*}
    E = U - KE - PE \rightarrow \Delta E = \Delta U
\end{equation*}

\begin{equation*}
    \Delta V = \dot{Q}_K
\end{equation*}

\subsection*{e)}

Die Temperatur würde weiter fallen, bis der Nullpunkt erreicht ist, da immer Energie entzogen wird.

\subsection*{f)}

\begin{equation*}
    \text{inter} \quad n = 0 \quad \frac{T_2}{T_1} = \left( \frac{p_{a1}}{p_{a2}} \right)^{\frac{\gamma - 1}{\gamma}} \rightarrow 1 \rightarrow T_1 \frac{v_2}{v_1} = T_2
\end{equation*}

```