
``````latex


\section*{Aufgabe 2}

\subsection*{a)}

\begin{description}
    \item[Graph 1:] The first graph is a plot with the vertical axis labeled as \( T \cdot s/k \) and the horizontal axis unlabeled. The graph contains a wavy line that starts from the origin, rises to a peak, falls, rises again to a higher peak, and then falls again.
    
    \item[Graph 2:] The second graph is a plot with the vertical axis labeled as \( T \cdot s/k \) and the horizontal axis labeled as \( s \left[ \frac{kJ}{kgK} \right] \). The graph contains several points labeled from 0 to 6. The points are connected by lines as follows:
    \begin{itemize}
        \item Point 0 to Point 1: A steep upward line.
        \item Point 1 to Point 2: A less steep upward line.
        \item Point 2 to Point 3: A horizontal line.
        \item Point 3 to Point 4: A downward line.
        \item Point 4 to Point 5: A wavy line.
        \item Point 5 to Point 6: A vertical downward line.
    \end{itemize}
    The points are connected by smooth curves or straight lines, and the graph includes several horizontal lines indicating different levels of \( T \cdot s/k \). The point 5 is labeled with \( P_s = P_u \) and the point 6 is labeled with \( P_o \).
\end{description}

``````latex


b) 5-6 isotherm nic 7.4

\[
T_6 = T_5 \left( \frac{P_6}{P_5} \right)^{\frac{\gamma-1}{\gamma}} = 328.02797 \, K
\]

7. HS Durch:

\[
0 = \dot{m} \left( h_5 - h_6 + \frac{w_5^2 - w_6^2}{2} \right) + \dot{Q} \quad \dot{Q} = 0
\]

\[
2(h_6 - h_5) = w_5^2 - w_6^2
\]

\[
\Rightarrow w_6^2 = w_5^2 - 2(h_6 - h_5) \Rightarrow w_6 = \sqrt{w_5^2 - 2(h_6 - h_5)} = \sqrt{507.2 \, \frac{m^2}{s^2}} \Rightarrow w_6 = 200 \, \frac{m}{s}
\]

\[
h_6 - h_5 = c_p (T_6 - T_5) = -10 \, \frac{G_{45} \, kJ}{kg}
\]

c)

\[
\Delta ex_{5 \rightarrow 6} = (h_6 - h_0 - T_0 (s_6 - s_0) + \frac{w_6^2 - w_0^2}{2})
\]

\[
h_6 - h_0 = c_p (T_6 - T_0) = 95.434 \, \frac{kJ}{kg}
\]

\[
s_6 - s_0 = c_p \ln \left( \frac{T_6}{T_0} \right) - R \ln \left( \frac{P_6}{P_0} \right) = 0.30136 \, \frac{kJ}{kg \cdot K}
\]

\[
c_V = \frac{c_p}{n} = 7.4 \cdot 5.2 \, \frac{kJ}{kg \cdot K}
\]

\[
R = c_p - c_V = 287.025 \, \frac{kJ}{kg \cdot K} \quad (2)
\]

\[
\Rightarrow \Delta ex_{5 \rightarrow 6} = -1.408 \cdot 10^5 \, \frac{J}{kg} = -140.8 \, \frac{kJ}{kg}
\]

``````latex


\section*{2d) Stationärer Fließprozess}

\[
\dot{P}_{\text{ex,ver}} = \Delta e_{\text{ex,ist}} + \Delta e_{\text{ex,iq}} - \frac{\dot{W}_t}{\dot{m}_s} \rightarrow 0, \text{ da gesamte Leistung der Turbine in Verdichter geht!}
\]

\[
e_{\text{ex,iq}} = \left(1 - \frac{T_0}{T_B}\right) q_B = 969,58 \frac{kJ}{kg}
\]

\[
e_{\text{in,ver}} = 1710,38 \frac{kJ}{kg}
\]

\section*{Aufgabe 4}

\subsection*{a) Phasentest}

\begin{description}
    \item[Graph Description:] The graph is a complex, hand-drawn diagram with multiple intersecting lines and annotations. The x-axis is labeled with "T" (temperature) and the y-axis is labeled with "p" (pressure). There are several regions and points marked on the graph:
    \begin{itemize}
        \item A point labeled "Tripel" (triple point).
        \item Regions labeled "sublimieren" (sublimation) and "kondensieren" (condensation).
        \item Multiple phases are indicated, including "fest" (solid), "flüssig" (liquid), and "gas" (gas).
        \item The graph includes several curves and lines that intersect at various points, indicating phase transitions.
        \item There are arrows and numbers (1, 2, 3, etc.) indicating specific points or paths on the graph.
    \end{itemize}
\end{description}

``````latex


\section*{Problem 2}

\subsection*{Part (a)}

\textit{2.45 Heat transfer system:}

\[
\Delta S_{12} = \delta m_2 \cdot s_2 + \frac{Q_R}{T} + \dot{\xi}_m
\]

``````latex


