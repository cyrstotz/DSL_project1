
``````latex


\section*{Aufgabe 4}

\subsection*{b) Sublimation}

\begin{align*}
    (1) \rightarrow (9) & \Rightarrow p = 2 \, \text{mbar} = 2 \cdot 10^3 \, \text{bar} = 2 \cdot 10^3 \cdot 10^{-5} \, \text{bar} \\
    -20^\circ \text{C} & = 0.1 \, \text{bar} \quad \text{= echte während Sublimation!!}
\end{align*}

\text{Aus TAB A-6: Interpolation:}
\begin{align*}
    T(0.21 \, \text{kPa}) & = T(0.20388 \, \text{kPa}) + \frac{T(0.0883 \, \text{kPa}) - T(0.20388 \, \text{kPa})}{0.1635 - 0.0883} \cdot 0.1 \, \text{kPa} \\
    (0.1 - 0.0883) \, \text{kPa} & = -22^\circ \text{C} \approx 20.385^\circ \text{C}
\end{align*}

\text{Diagram description:}
\begin{itemize}
    \item There is a horizontal line labeled "T2 = Sub-punkt".
    \item Below this line, there is a note "20K über sublimationspunkt".
    \item An arrow points from "20K" to "20.385^\circ \text{C}".
    \item Another horizontal line labeled "T2 = Sub-punkt = 20.385^\circ \text{C} \approx 20^\circ \text{C}, wenn man es vom Diagramm abliest".
\end{itemize}

\text{Process 1-2: Verdampfen}
\begin{align*}
    T_{\text{verdampfen}} & = T_{\text{verflüssigen}} + 6 \, \text{K} = 26^\circ \text{C}
\end{align*}

\text{Stationärer Fließprozess:}
\begin{align*}
    0 & = \dot{m} \left( h_1 + \frac{v_1^2}{2} + g z_1 \right) - \left( h_2 + \frac{v_2^2}{2} + g z_2 \right) + \dot{Q} - \dot{W} \\
    \dot{Q}_{\text{zu}} & = \dot{Q}_{\text{K}} + \dot{Q}_{\text{AB}} \\
    \dot{Q}_{\text{K}} & = \dot{Q}_{\text{zu}} - \dot{Q}_{\text{AB}}
\end{align*}

\text{Process 3-4:}
\begin{align*}
    \dot{m} \cdot \Delta h & = \dot{Q}_{\text{zu}} - \dot{Q}_{\text{AB}} \\
    \dot{m} \cdot \Delta h & = \dot{m} \cdot (h_3 - h_4) - \dot{Q}_{\text{AB}} \\
    \dot{m} \cdot \Delta h & = \dot{m} \cdot (h_3 - h_4) - \dot{Q}_{\text{AB}} \\
    \dot{m} \cdot \Delta h & = \dot{m} \cdot (h_3 - h_4) - \dot{Q}_{\text{AB}} \\
    \dot{m} \cdot \Delta h & = \dot{m} \cdot (h_3 - h_4) - \dot{Q}_{\text{AB}} \\
    \dot{m} \cdot \Delta h & = \dot{m} \cdot (h_3 - h_4) - \dot{Q}_{\text{AB}} \\
    \dot{m} \cdot \Delta h & = \dot{m} \cdot (h_3 - h_4) - \dot{Q}_{\text{AB}} \\
    \dot```latex


\begin{equation}
(II) \Rightarrow \dot{m}_e = \frac{\dot{Q}_K - \dot{W}_K}{h_3 - h_4} \Rightarrow \dot{Q}_K = \dot{m}_e (h_3 - h_4) + \dot{W}_K \tag{III}
\end{equation}

\begin{equation}
(III) \Rightarrow (I) \Rightarrow \dot{m}_e = \dot{m}_e \left( \frac{h_3 - h_4}{h_2 - h_1} \right) + \frac{\dot{W}_K}{h_2 - h_1}
\end{equation}

\begin{equation}
\Rightarrow \dot{m}_e \left( 1 + \frac{h_3 - h_4}{h_2 - h_1} \right) = \frac{\dot{W}_K}{h_2 - h_1}
\end{equation}

\begin{equation}
\Rightarrow \dot{m}_e \left( \frac{h_2 - h_1 + h_3 - h_4}{h_2 - h_1} \right) = \frac{\dot{W}_K}{h_2 - h_1}
\end{equation}

\begin{equation}
\Rightarrow \dot{m}_e = \frac{\dot{W}_K}{\frac{h_2 - h_1 + h_3 - h_4}{h_2 - h_1}} = \frac{\dot{W}_K}{h_2 - h_1 + h_3 - h_4}
\end{equation}

\begin{equation}
\dot{m}_e = \frac{\dot{W}_K}{h_2 - h_1 + h_3 - h_4} = \frac{28 \frac{kJ}{s}}{(232,62 - 16,82 + 93,42 - 264,25) \frac{kJ}{kg}}
\end{equation}

\begin{equation}
\Rightarrow \dot{m}_e = 0,6352 \frac{kg}{s}
\end{equation}

Aus TABELLE A-20:

\begin{equation}
@ T_2 = T_3 = 260^\circ C, \quad h_{g_2} = h_{g_3} = 232,62 \frac{kJ}{kg}
\end{equation}

\begin{equation}
h_2 = h_f = 16,82 \frac{kJ}{kg}
\end{equation}

Aus TABELLE A-12:

\begin{equation}
@ p_3 = 3 \text{bar} \quad x = 0,6
\end{equation}

\begin{equation}
h_3 = h_f = 264,25 \frac{kJ}{kg}, \quad h_{fg} = h_{g_2} - h_{f_2} = 93,42 \frac{kJ}{kg}
\end{equation}

\begin{equation}
\Rightarrow \dot{m}_e = 0,6352 \frac{kg}{s} \quad \text{or} \quad 4 \frac{kg}{h}
\end{equation}

\begin{equation}
\Rightarrow \dot{m}_{\text{gesammt}} = 0,6352 \frac{g}{s}
\end{equation}

c) \quad \text{Verlust: adiabate reversible Dampf Expansions} \Rightarrow \Delta s = 
``````latex


\section*{Aufgabe 4}

\subsection*{a) Skizze 1 von Kühlmittel}

\textbf{Graph Description:}

The graph is a simple Cartesian coordinate system. The horizontal axis (x-axis) is labeled with the letter "T" at the far right end, indicating temperature. The vertical axis (y-axis) is labeled with the letter "P" at the top, indicating pressure. Both axes have arrows at their ends pointing in the positive direction. The origin of the graph is at the bottom left corner where the two axes intersect.

``````latex


\begin{itemize}
    \item[c)] \( x_2 = \)
\end{itemize}

``````latex


\section*{Aufgabe 4}

\subsection*{d)}

\[
\dot{Q}_{\text{ab}} = \dot{m} R-134a (h_{2k} - h_1) \quad \text{(aus 4. b)}
\]

\[
\dot{m}_e = \dot{m} \frac{v_2}{v_1} = \frac{v_2 \cdot 20}{v_1 \cdot 20} = \frac{2}{3} = 1,3333 \frac{g}{s}
\]

\[
\dot{Q}_{\text{zu}} = \dot{Q}_k = \dot{W}_k + \dot{Q}_{\text{ab}}
\]

\[
\epsilon_k = \frac{|\dot{Q}_{\text{zu}}|}{|\dot{Q}_{\text{ab}} - \dot{Q}_{\text{zu}}|}
\]

\[
= \frac{|\dot{W}_k + \dot{m} (h_{2k} - h_1)|}{|\dot{m} (h_{2k} - h_1) - \dot{W}_k + \dot{m}_e (h_2 - h_1)|}
\]

\[
= \frac{128 \frac{J}{s} + 1,3333 \frac{g}{s} \cdot (264,25 - 93,42) \frac{J}{g}}{1,3333 \frac{g}{s} \cdot (264,25 - 93,42) \frac{J}{g}}
\]

\[
= \frac{1,3333 \frac{g}{s} \cdot (264,25 - 93,42) \frac{J}{g} + 128 \frac{J}{s}}{1,3333 \frac{g}{s} \cdot (264,25 - 93,42) \frac{J}{g}} = 1
\]

\[
= 1,228
\]

\subsection*{e)}

\[
\dot{E}_{fp} \rightarrow 0 \approx \dot{m} \left( \frac{h_e}{h_1} - h_1 \right) \Theta W_{v12}
\]

\[
W_{v12} = \int_{4}^{2} p dV = m_e \frac{R}{\frac{R}{h} (T_1 - T_2)}
\]

\[
= 1 - h
\]

\[
\text{(siehe Box 11)}
\]

\[
\text{Box 1 = cp für R-134a}
\]

\[
\text{Cp}
\]

```