
``````latex


\section*{Aufgabe 2}

\subsection*{(a) T-S-Diagramm}

\[
T_0 = 293.15 \, K
\]

Das Diagramm zeigt ein T-S-Diagramm mit der Temperatur $T$ auf der y-Achse und der Entropie $S$ auf der x-Achse. Die y-Achse ist mit $T (K)$ beschriftet und die x-Achse mit $S$. Der Ursprung ist mit $T_0$ und $S_0$ markiert. 

Es gibt mehrere Punkte und Linien im Diagramm:

- Punkt 0 ist am Ursprung $(T_0, S_0)$.
- Punkt 1 ist direkt über Punkt 0 auf der gleichen Entropie-Linie.
- Punkt 2 ist direkt über Punkt 1 auf der gleichen Entropie-Linie.
- Punkt 3 ist rechts von Punkt 2 auf einer höheren Entropie-Linie.
- Punkt 4 ist rechts von Punkt 3 auf einer noch höheren Entropie-Linie.
- Punkt 5 ist unter Punkt 4 auf der gleichen Entropie-Linie.
- Punkt 6 ist unter Punkt 5 auf der gleichen Entropie-Linie.

Die Linien zwischen den Punkten sind wie folgt beschriftet:

- Die Linie von Punkt 0 zu Punkt 1 ist als "reversibel, isotherm" beschriftet.
- Die Linie von Punkt 1 zu Punkt 2 ist als "reversibel, adiabatisch" beschriftet.
- Die Linie von Punkt 2 zu Punkt 3 ist als "isobar" beschriftet.
- Die Linie von Punkt 3 zu Punkt 4 ist als "adiabatisch, irreversibel" beschriftet.
- Die Linie von Punkt 4 zu Punkt 5 ist als "isobar" beschriftet.
- Die Linie von Punkt 5 zu Punkt 6 ist als "reversibel" beschriftet.

Rechts unten im Diagramm ist eine kleine Skizze, die $\Delta S$ und $kJ/K$ zeigt.

``````latex


\section*{Problem 6}

\begin{align*}
\omega_6 &= ? \\
T_6 &= ? \\
\end{align*}

\begin{align*}
n &= k = 1.4 \\
\end{align*}

\text{ideale Gas:}

\begin{align*}
\left( \frac{T_6}{T_5} \right) &= \left( \frac{p_6}{p_5} \right)^{\frac{n-1}{n}} \\
\end{align*}

\begin{align*}
T_6 &= T_5 \cdot \left( \frac{p_6}{p_5} \right)^{\frac{n-1}{n}} \\
\end{align*}

\begin{align*}
T_6 &= ? \\
p_6 &= p_0 = 0.1916 \, \text{bar} \\
p_5 &= 0.5 \, \text{bar} \\
T_5 &= 437.19 \, \text{K} \\
\end{align*}

\begin{align*}
T_6 &= T_5 \cdot \left( \frac{p_6}{p_5} \right)^{\frac{n-1}{n}} = \underline{328.07 \, \text{K} = T_5}
\end{align*}

\begin{align*}
\omega_6 &= ? \\
\end{align*}

\text{5 $\rightarrow$ 6 adiabatic reversible!}

\begin{align*}
\Rightarrow Q &= 0 \\
\end{align*}

\text{Energieblanz:}

\text{stationär:}

\begin{align*}
0 &= \dot{m} \left[ h_5 - h_6 + \frac{\omega_5^2 - \omega_6^2}{2} + g(z_2 - z_1) \right] + \dot{E}_Q - \dot{E}_W \\
\end{align*}

\begin{align*}
0 &= \dot{m} \left[ h_5 - h_6 + \frac{\omega_5^2 - \omega_6^2}{2} \right] + \dot{m} \left( \frac{n \cdot R (T_6 - T_5)}{n-1} \right) \\
\end{align*}

\begin{align*}
h_5 - h_6 + \frac{\omega_5^2 - \omega_6^2}{2} &= \frac{n \cdot R (T_6 - T_5)}{n-1} \\
\end{align*}

\begin{align*}
\dot{m} \left( \frac{n \cdot R (T_6 - T_5)}{n-1} \right) &= \dot{W}_t \\
\end{align*}

\text{Verbal description of graphical content:}

The page contains handwritten text and formulas related to a physics problem. The text is written in a mix of German and mathematical notation. There are no explicit graphs, figures, or diagrams on the page. The content includes equations for temperature and pressure relationships in an ideal gas, energy balance equations, and adiabatic processes. The equations are written clearly with variables and constants defined. The text also includes some annotations and corrections, such as crossing out terms and adding arrows to indicate steps in the derivation.

``````latex


\section*{2 (b)}

\[
w_6 = \sqrt{\left( \frac{n \cdot R \cdot (T_6 - T_5)}{n - 1} \right) + h_c - h_s + \left( \text{scribble} \right) + w_5^2} \cdot 2
\]

\[
R = \frac{\overline{R}}{M_{\text{Luft}}} = \frac{\frac{\overline{R}}{\text{mol}}}{\frac{h_g}{R_{\text{mol}}}} = \frac{0.296 \frac{hJ}{h_g}}{\text{ngk}}
\]

\[
w_6 = 310.61 \frac{m}{s}
\]

\[
\text{Cp} = 1.006 \frac{kJ}{kgK}
\]

\section*{(C) Exergie:}

\[
\Delta e_{\text{str}} = (h - h_0 - T_0 (s - s_0) + ke + pe)
\]

\[
\Delta e_{\text{str}} = \left[ h_6 - h_0 - T_0 (s_6 - s_0) + ke + pe \right] = 12.22 \frac{kJ}{kg}
\]

\[
h_6 (p_6 = 0.191, T_6 = 328.07) = 
\]

\[
s_6 (p_6 = 0.191, T_6 = 328.07) = 
\]

\[
h_6 - h_0 = \text{Cp} \cdot \overline{s}(T) \cdot (T_6 - T_6)
\]

\[
s_6 - s_0 = \text{cp} \cdot \left( \ln \left( \frac{T_6}{T_6} \right) - R \cdot \ln \left( \frac{p_6}{p_0} \right) \right)
\]

\[
\text{wie bei aufgabe 1 angeleitet}
\]

\[
h_6 - h_0 = \text{cp} \cdot \overline{g} (T_6 - T_6) = 85.43 \frac{kJ}{h_g}
\]

\[
s_6 - s_0 = 0.30 \frac{kJ}{kgK} \Rightarrow T_0 = 73.22 \frac{kJ}{kg}
\]

\[
ke = \text{scribble} \quad \text{w_6^2 - w_{\text{Luft}}^2} = 110 \frac{kJ}{kg}
\]

``````latex


\section*{1.}

\textit{ex,verl}

\textbf{Masse}

\textbf{Stationär Energie Bilanz:}

\[
0 = \dot{m} \left( h_{e} - \ldots \right) + \dot{Q} e_{x,1Q} - W_{t,n} - \dot{e}_{x,verl}
\]

\[
e_{x,verl} = \Delta e_{x,stro,16} + e_{x,0,16} - W_{t,n}
\]

\textit{ganz ist adiabot?}

\[
e_{x,verl} = \Delta e_{x,stro,16} - W_{t,n}
\]

\textit{Turbine:}

\textit{aufgabe (b)} $\Rightarrow$ \[
W_{t} = -n \cdot R \frac{(T_{6} - T_{8})}{1 - n}
\]

\[
\Delta e_{x,verl} = 100 \frac{kJ}{kg} + 109260.9 \frac{kJ}{kg}
\]

\[
= \underline{204.26 \frac{kJ}{kg}}
\]

``````latex


