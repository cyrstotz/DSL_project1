
``````latex


\section*{Aufgabe 4:}

\subsection*{a)}
\begin{itemize}
    \item A graph is drawn with the x-axis labeled as $T [^\circ C]$ and the y-axis labeled as $p (\text{bar})$.
    \item The graph shows a curve labeled as "Tripelpunkt" which intersects the y-axis at a point labeled "i)" and the x-axis at a point labeled "ii)".
    \item Another curve labeled as "Phasengrenze" starts from the point "i)" and moves towards the right, approaching the x-axis asymptotically.
    \item The point "i)" is marked on the curve "Phasengrenze" and the point "ii)" is marked on the x-axis.
    \item The region to the left of the curve "Phasengrenze" is labeled as "(fest)" and the region to the right is labeled as "(flüssig)".
    \item The region below the curve "Tripelpunkt" is labeled as "(gasförmig)".
\end{itemize}

\subsection*{b)}
1. HS im Verdampfer, stationärer Fließprozess, kin und pot vernachlässigbar \\
\[
0 = \dot{m}_{R1244} \left[ h_2 - h_3 \right] + \dot{Q}_{ev} - \dot{W}_e
\]
\[
\dot{m}_{R1244} = \frac{\dot{W}_e}{h_2 - h_3}
\]
\[
= 0{,}0000667 \frac{kg}{s}
\]
\[
= 2{,}39 \frac{kg}{h}
\]

\begin{itemize}
    \item Tafel A-D:
    \[
    h_2 = h_g (-22^\circ C) = 234{,}02 \frac{kJ}{kg}
    \]
    \item Tafel A-11:
    \[
    h_3 = h_g (30^\circ C) = 264{,}18 \frac{kJ}{kg}
    \]
    \item 
    \[
    \dot{W}_e = -\dot{Q}_{k} = -20 W = -20 \frac{J}{s} = -20 \cdot 10^{-3} \frac{kJ}{s}
    \]
\end{itemize}

\subsection*{c)}
\subsection*{d)}
\[
\epsilon_K = \frac{|\dot{Q}_{2v}|}{|\dot{W}_e|} = \frac{|\dot{Q}_{2v}|}{|\dot{Q}_{a4} - \dot{Q}_{2v}|}
\]

``````latex


e) es würde dann kaum kaum so lange abkühlen, bis die Temperatur im Innenraum gleich der Temperatur des Kühlmittels im Wärmeübertrager ist.

Dann ist kein Wärmeübergang mehr möglich im Wärmeübertrager.

```