
``````latex


\section*{Aufgabe 4}

\subsection*{a)}

\begin{itemize}
    \item A graph is drawn with pressure \( p \) on the y-axis and temperature \( T \) on the x-axis.
    \item The graph shows three phases: "Fest" (solid), "Flüssig" (liquid), and "Gas" (gas).
    \item There is a curve labeled "isotherm" that separates the solid and gas phases.
    \item The point where the solid, liquid, and gas phases meet is labeled "Tripel" (triple point).
    \item Points are marked on the graph:
        \begin{itemize}
            \item Point 1 is in the liquid region.
            \item Point 2 is in the solid region.
            \item Point 3 is in the gas region.
        \end{itemize}
    \item The x-axis is labeled \( T \) and the y-axis is labeled \( p \).
\end{itemize}

\subsection*{b)}

\[
\begin{array}{|c|c|c|c|c|c|c|}
\hline
\text{d} & p & T & x & w & q & s \\
\hline
1 & p_1 & T_{\text{iso}} & 0 & & & \\
\hline
2 & p_2 & 90^\circ & 1 & 2.8 \, \text{W} & q_{23}=0 & \\
\hline
3 & p_3 & & 0 & & & \\
\hline
\end{array}
\]

\begin{itemize}
    \item \( T_i = 10 \, \text{K} \) über subpunkt \( t = 10^\circ \)
    \item \( \overline{T}_{\text{verd}} = \overline{T} - \Delta t_k = 4^\circ \)
    \item \( s_2 < s_3 \)
    \item 1. HS 2. nach 3 statt f. proz
    \item 0 = \dot{m}_{23} \left[ h_2 - h_3 \right] + \dot{Q} - \dot{W}_k \quad \text{adiab}
    \item \frac{\dot{v}_k}{h_2 - h_3} = \dot{m}_R
\end{itemize}

``````latex


1) \quad \epsilon_K = \frac{\dot{Q}_K}{\dot{W}_T} = \frac{\dot{Q}_K}{\dot{W}_K}

Für \quad \dot{Q}_K \quad 1 \quad tis \quad von \quad A \quad nach \quad Z:

0 = \dot{m} \cdot [h_1 - h_2] + \dot{Q}_K - \dot{W}^0

\dot{Q}_K = \dot{m} \cdot [h_2 - h_1]

\dot{Q}_K = 0.27 \cdot 6 \quad kJ/s = 2.96 \quad kW

\epsilon_K = 8.79

``````latex


\section*{Aufgabe 4 b) weiter}

\begin{align*}
h_2 &= h_g(T_2 = 47^\circ) = 242.53 \frac{\text{kJ}}{\text{kg}} \\
s_2 &= s_g(R = 40) = 0.9169 \frac{\text{kJ}}{\text{kgK}} \\
h_3 &= (s \text{ bei } s_2 = s_3 \text{ an } \text{Mischv.}) \\
h_{3m} &= (s \text{ bei } s_2 = 0.90 \frac{\text{kJ}}{\text{kgK}}) = 249.15 \frac{\text{kJ}}{\text{kg}} \\
h_{3c} &= (s \text{ bei } s_2 = 0.937 \frac{\text{kJ}}{\text{kgK}}) = 243.66 \frac{\text{kJ}}{\text{kg}} \\
h_3 &= \frac{s_2 - s_{3c}}{s_{3m} - s_{3c}} (h_{3m} - h_{3c}) + h_{3c} = 264.33 \frac{\text{kJ}}{\text{kg}} \\
\dot{m}_R &= \frac{\dot{A} s_4}{s_3 - s_5} = 1.543 \cdot 10^{-3} \frac{\text{kg}}{\text{s}}
\end{align*}

\section*{c)}

\begin{align*}
1 - h_5 &= 2.1 \\
h_4 &= h_f + x_4 (h_g - h_f) = 0 \\
h_m &= h_f \text{ an } \text{drosseln} \text{ ans } 1. HS \quad 0 = \dot{m} (h_m - h_5) + Q \\
h_4 &= (s \text{ bei } 1, x = 0) = h_f (s \text{ bei } 1) = 93.42 \frac{\text{kJ}}{\text{kg}} \\
P_1 &= P_2 \quad P_2 (47^\circ) = 3.3465 \text{ Bar} \\
h_f (P_2) &= 55.35 \frac{\text{kJ}}{\text{kg}} \\
h_g (P_2) &= 279.53 \frac{\text{kJ}}{\text{kg}} \\
h_4 &= h_f + x_4 (h_g - h_f) \\
\frac{h_4 - h_f}{h_5 - h_f} &= x_4 = 0.196
\end{align*}

```