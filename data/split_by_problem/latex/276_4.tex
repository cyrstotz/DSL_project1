
``````latex


\section*{Aufgabe 4}

\begin{itemize}
    \item[(a)] es würde immer kälter werden.
    \item[(b)] Eine isotherme, Energiebilanz
    \[
    0 = \dot{m}_{R1234a} + \dot{Q}_{R}
    \]
    \[
    X_{2, R1234} = 0
    \]
    \item[(c)] \(X_1\)
\end{itemize}

\subsection*{Graph Description}
The graph is a plot with the vertical axis labeled \(P\) and the horizontal axis labeled \(T\). The graph shows a curve that starts at the origin, rises to a peak labeled \(KP\), and then falls back down. There are two points labeled \(i\) and \(ii\) on the curve. Point \(i\) is on the rising part of the curve, and point \(ii\) is on the falling part of the curve. There is a dashed line from the origin to point \(i\), and a wavy line from point \(i\) to point \(ii\). The peak \(KP\) is marked with an 'x'. The curve represents a process that starts at the origin, rises to a peak, and then falls back down.

```