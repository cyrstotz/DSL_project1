
``````latex


\section*{Aufgabe 3}

\subsection*{a) Ideales Gasgesetz: \( pV = nRT \)}

Kräftegleichgewicht Zustand 1:}

\begin{center}
\begin{tabular}{c}
\begin{minipage}{0.4\textwidth}
\begin{center}
\begin{picture}(100,150)
\put(50,130){\vector(0,-1){20}}
\put(50,110){\vector(0,1){20}}
\put(50,110){\line(1,0){50}}
\put(50,110){\line(-1,0){50}}
\put(50,90){\line(1,0){50}}
\put(50,90){\line(-1,0){50}}
\put(50,90){\vector(0,-1){20}}
\put(50,70){\vector(0,1){20}}
\put(50,70){\line(1,0){50}}
\put(50,70){\line(-1,0){50}}
\put(50,50){\line(1,0){50}}
\put(50,50){\line(-1,0){50}}
\put(50,50){\vector(0,-1){20}}
\put(50,30){\vector(0,1){20}}
\put(50,30){\line(1,0){50}}
\put(50,30){\line(-1,0){50}}
\put(50,10){\line(1,0){50}}
\put(50,10){\line(-1,0){50}}
\put(50,10){\vector(0,-1){20}}
\put(50,-10){\vector(0,1){20}}
\put(50,-10){\line(1,0){50}}
\put(50,-10){\line(-1,0){50}}
\put(50,-30){\line(1,0){50}}
\put(50,-30){\line(-1,0){50}}
\put(50,-30){\vector(0,-1){20}}
\put(50,-50){\vector(0,1){20}}
\put(50,-50){\line(1,0){50}}
\put(50,-50){\line(-1,0){50}}
\put(50,-70){\line(1,0){50}}
\put(50,-70){\line(-1,0){50}}
\put(50,-70){\vector(0,-1){20}}
\put(50,-90){\vector(0,1){20}}
\put(50,-90){\line(1,0){50}}
\put(50,-90){\line(-1,0){50}}
\put(50,-110){\line(1,0){50}}
\put(50,-110){\line(-1,0){50}}
\put(50,-110){\vector(0,-1){20}}
\put(50,-130){\vector(0,1){20}}
\put(50,-130){\line(1,0){50}}
\put(50,-130){\line(-1,0){50}}
\put(50,-150){\line(1,0){50}}
\put(50,-150){\line(-1,0){50}}
\put(50,-150){\vector(0,-1){20}}
\put(50,-170){\vector(0,1){20}}
\put(50,-170){\line(1,0){50}}
\put(50,-170){\line(-1,0){50}}
\put(50,-190){\line(1,0){50}}
\put(50,-190){\line(-1,0){50}}
\put(50,-190){\vector(0,-1){20}}
\put(50,-210){\vector(0,1){20}}
\put(50,-210){\line(1,0){50}}
\put(50,-210){\line(-1,0){50}}
\put(50,-230){\line(1,0){50}}
\put(50,-230){\line(-1,0){50}}
\put(50,-230){\vector(0,-1){20}}
\put(50,-250){\vector```latex


\section*{Aufg. 3}

\subsection*{a)}

\[
M_g = \frac{140,14 \, \text{kPa} \cdot 3,14 \cdot 10^{-3} \, \text{m}^3}{0,166 \, \frac{\text{kJ}}{\text{kg K}} \cdot 773,15 \, \text{K}} = 3,43 \cdot 10^{-3} \, \text{kg} = 3,43 \, \text{g}
\]

\subsection*{c) Energiebilanz (EB) an einem Kolben:}

\[
\frac{dE}{dt} = \sum_j \dot{Q}_j - \sum_n \dot{W}_n
\]

\[
\Delta E = E_2 - E_1 = \sum_j Q_j - \sum_n W_n
\]

\[
E_2 - E_1 = Q_{12} - W_{12} \Rightarrow Q_{12} = E_2 - E_1 + W_{12}
\]

\textbf{Systemgrenze: nur Gas}

\[
\Delta E = Q_{12} - W_{12}
\]

\subsection*{b)}

\[
pV = mRT \Rightarrow p = \frac{mRT}{V} = \frac{RT}{\nu}
\]

\[
\nu = \frac{V}{m} \quad \nu_1 = \frac{3,14 \cdot 10^{-3} \, \text{m}^3}{3,43 \cdot 10^{-3} \, \text{kg}} = 0,915 \, \frac{\text{m}^3}{\text{kg}}
\]

\[
\text{Gas isobar:} \quad p_2 = \frac{RT}{\nu_1} = 49,56 \, \text{kPa}
\]

\subsection*{c)}

\[
E = U + \cancel{E^0} + \cancel{E^0} \quad \text{vernachlässigbar}
\]

\[
\Delta U \, \text{perfektes Gas:} \quad \Delta U = C_v (T_2 - T_1)
\]

\[
= 0,633 \, \frac{\text{kJ}}{\text{kg K}} (273,15 \, \text{K} - 773,15 \, \text{K})
\]

\[
= -316,5 \, \frac{\text{kJ}}{\text{kg}}
\]

\[
Q_{12} = \Delta U + W_{12} = W_{12} - 316,5 \, \frac{\text{kJ}}{\text{kg}}
\]

\subsection*{Graphical Description}

There is a simple diagram showing a piston with a gas inside. The diagram is a rectangle representing the piston, with a horizontal line inside it representing the membrane. Below the membrane, the word "Gas" is written. The label "Systemgrenze: nur Gas" is written next to the diagram.

``````latex


\begin{itemize}
    \item[a)] $\dot{x}_{\text{Eis,1}} = 0{,}6$
\end{itemize}

\textit{aus b):} $P_{g2} = 49{,}56 \text{ kPa} = 0{,}4956 \text{ bar} \approx 0{,}5 \text{ bar}$

\textbf{Energiebilanz um das Eis:}

\textit{Geschlossenes System:}

\[
\frac{dE}{dt} = Q - \dot{W} \rightarrow \text{Volumen ändert sich nicht}
\]

\[
\Delta E = \Delta U + \Delta KE + \Delta PE = U_2 - U_1
\]

\textit{U}_1 \text{ Eis bei } x_1 = 0{,}6 \text{ \& T=0 aus Tab 1 ablesen:}

\[
U_1 = U_f + x \cdot (U_g - U_f)
\]

\[
U_g = U_{\text{gas}} \quad U_f = U_{\text{flüssig}}
\]

\[
= U_f + x U_g - x U_f
\]

\[
= [0{,}1045 + 0{,}6 \cdot (-333{,}458 + 0{,}045)] \frac{\text{kJ}}{\text{kg}}
\]

\[
= -200{,}093 \frac{\text{kJ}}{\text{kg}} \approx -200{,}1 \frac{\text{kJ}}{\text{kg}}
\]

\[
U_2 - U_1 = Q_{12} \Rightarrow U_2 = Q_{12} + m \cdot U_{\text{1 spez}}
\]

\[
U_2 = 1500 \text{ J} + 0{,}1 \text{ kg} \cdot (-200{,}1 \frac{\text{kJ}}{\text{kg}})
\]

\[
= 1500 \text{ J} - 20{,}01 \text{ kJ} = 1500 \text{ J} - 20010 \text{ kJ}
\]

\[
= -18510 \text{ J}
\]

\[
U_2 = \frac{-18510 \text{ J}}{0{,}1 \text{ kg}} = -185100 \frac{\text{J}}{\text{kg}} = -185{,}100 \frac{\text{kJ}}{\text{kg}}
\]

``````latex


\section*{1 d)}
\begin{equation}
    U_2 = U_{gf} + x_2 (U_{fen} - U_{pe})
\end{equation}
\begin{equation}
    \Rightarrow x_2 = \frac{U_2 - U_{gf}}{U_{fen} - U_{pe}}
\end{equation}

``````latex


