
``````latex


\section*{Aufgabe 3}

\subsection*{a)}
\begin{equation*}
P_{g,1} = 100 \text{Pa} + \frac{v \cdot g}{\pi} \left( \frac{D}{2} \right)^2 \pi + \frac{v \cdot E \cdot W}{3} \left( \frac{D}{2} \right)^2 \pi
\end{equation*}

\begin{equation*}
100.027 \text{Pa} \approx 1 \text{bar}
\end{equation*}

\begin{equation*}
102.7 \text{Pa} \approx 1.02 \text{Bar}
\end{equation*}

\begin{equation*}
P_{s,1} \quad 3.192 = 0.00319 \cdot 10^{-3}
\end{equation*}

\begin{equation*}
m = \frac{pV}{\frac{K}{T} \cdot M_{g,1}} = 2.5 \text{g}
\end{equation*}

\begin{equation*}
500^\circ \text{C} = 773.15 \text{K}
\end{equation*}

\subsection*{b)}
Der Druck an Gas rechnet sich zusammen aus pump, dem Gewicht von Kolben sowie dem Gewicht des Einwassers; welche sich alle nicht ändern in Zustand 2.

\begin{equation*}
\Rightarrow P_{g,1}^n = P_{s,1}^2 = 1.02 \text{Bar}
\end{equation*}

\subsection*{c)}
\begin{equation*}
\text{---}
\end{equation*}

\subsection*{d)}
\begin{equation*}
T_{s,2} = T_{s,1}
\end{equation*}

\begin{equation*}
\frac{U_{fest} + U_{flüss}}{h} = U_{fest} + x_{Ein,2} \left( \frac{U_{flüss}}{U_{flüss} - U_{fest}} \right)
\end{equation*}

\begin{equation*}
\frac{U_{flüss}}{U_{flüss} - U_{fest}} = x_{Ein,2} = 9.90 \cdot 10^{-3}
\end{equation*}

``````latex


