
``````latex


\section*{3a)}

Zuerst muss der Druck im Gasbehälter berechnet werden: \\
Druck EW + Druck vom Gewicht + Druck Umgebung

\[
P_L = \frac{F}{A} = \frac{32 \, \text{kg} \cdot 9,81 \, \frac{\text{m}}{\text{s}^2}}{\pi \left(5 \cdot 10^{-2} \, \text{m}\right)^2} = 39,96 \, \frac{\text{N}}{\text{m}^2}
\]

\[
P_{EW} = \frac{F}{A} = \frac{0,74 \, \text{kg} \cdot 9,81 \, \frac{\text{m}}{\text{s}^2}}{\pi \left(5 \cdot 10^{-2} \, \text{m}\right)^2} = 724,304 \, \frac{\text{N}}{\text{m}^2}
\]

\[
P_{amb} = 10^5 \, \frac{\text{N}}{\text{m}^2}
\]

\[
P_{g,1} = P_L + P_{EW} + P_{amb} = 740059,44 \, \frac{\text{N}}{\text{m}^2}
\]

\[
R = \frac{\mathcal{R}}{M} = \frac{8,314 \, \frac{\text{J}}{\text{mol} \cdot \text{K}}}{50 \, \frac{\text{g}}{\text{mol}}} = 0,16628 \, \frac{\text{J}}{\text{g} \cdot \text{K}}
\]

\[
P_{g,1} = 1,4 \, \text{bar}
\]

\[
m_g = \frac{P_{g,1} \cdot V_{g,1}}{R \cdot T_{g,1}} = \frac{1,4 \cdot 10^5 \, \frac{\text{N}}{\text{m}^2} \cdot 3,14 \cdot 10^{-3} \, \text{m}^3}{0,16628 \, \frac{\text{J}}{\text{g} \cdot \text{K}} \cdot 7,73,15 \, \text{K}} = 3,419 \, \text{g}
\]

``````latex


36)

Das Eis-Wasser weil die Wassermoleküle im Fest-Flüssig Gebiet befindet oder eine Isotherme bis alles Eis (fest) weggeschmolzen ist. Also ist das EW immer noch $T_{EW,1} = 0^\circ C$. Somit muss auch das Gas $T_{g,2} = 0^\circ C$, thermodynamisches Gleichgewicht.

\[
T_{g,2} = 0^\circ C
\]

Die Massen welche auf das Gas wirken sind ebenfalls die gleichen. Somit $p_{g,2} = 1,46 \text{bar}$

\[
p_{g,2} = \frac{m_{g,1}}{V_{g,1}} \frac{T_{g,2}}{T_{g,1}} \frac{V_{g,2}}{V_{g,2}}
\]

3c)

\[
\begin{array}{|c|c|}
\hline
\text{U2} & \text{Q12} \\
\hline
\text{U1} & \text{W12} \\
\hline
\end{array}
\]

Energie Bilanz um Systemgrenze:

\[
\frac{dE}{dt} = \sum_j \dot{Q}_j - \sum_n \dot{W}_n
\]

\[
U_2 - U_1 = Q_{12} - W_{12}
\]

Da wir keine Reibung haben ist $W_{12} = reversibel$

\[
W_{12} = \int_1^2 p dV
\]

\[
v_2 = \frac{v_{g,1}}{v_{g,1}} \frac{v_{g,1}}{v_{g,1}} = \frac{3.14 \cdot 10^{-3} \text{m}^3}{3.429 \cdot 10^{-3} \text{kg}} = 0.918397 \frac{\text{m}^3}{\text{kg}}
\]

``````latex


Für $v_{g2}$:

\[
v_{g2} = \frac{m_{12}}{p} = \frac{3{,}449 \cdot 10^{-3} \, \text{kg} \cdot 0{,}76628 \, \frac{\text{m}^3}{\text{kg}} \cdot 273{,}15 \, \text{K}}{1{,}4 \cdot 10^2 \, \frac{\text{J}}{\text{m}^3}}
\]

\[
\approx 1{,}103 \cdot 10^{-2} \, \text{m}^3
\]

\[
v_{g2} = \frac{1{,}103 \cdot 10^{-2} \, \text{m}^3}{3{,}449 \cdot 10^{-3} \, \text{kg}} = 0{,}324364 \, \frac{\text{m}^3}{\text{kg}}
\]

\[
W_{12} = \int_{1}^{2} p \, dv = R \bar{T} \, p (v_2 - v_1) = 1{,}4 \cdot 10^2 \, \frac{\text{J}}{\text{m}^3} \left( 0{,}324364 \, \frac{\text{m}^3}{\text{kg}} - 0{,}948837 \, \frac{\text{m}^3}{\text{kg}} \right) = -83{,}164 \, \frac{\text{J}}{\text{kg}}
\]

\[
W_{12} = W_{12} \cdot m_{12} = -83{,}164 \, \frac{\text{J}}{\text{kg}} \cdot 3{,}449 \cdot 10^{-3} \, \text{kg} = -0{,}286434 \, \text{J}
\]

\[
u_2 - u_1 = u_{sg2} - u_{sg1} = c_v (T_2 - T_1) \cdot m_{12} = 0{,}683 \, \frac{\text{J}}{\text{kgK}} \cdot (273{,}15 \, \text{K} - 77{,}15 \, \text{K}) \cdot 3{,}449 \cdot 10^{-3} \, \text{kg}
\]

\[
= -710827 \, \text{J}
\]

\[
Q_{12} = u_2 - u_1 + W_{12} = -1{,}3661{,}4 \,text{J}
\]

\[
|Q_{12}| = 1{,}3661{,}4 \, \text{J}
\]

``````latex


3c)

\[
\begin{array}{|c|}
\hline
\epsilon_w \\
\hline
Q_{12} \\
\hline
\end{array}
\]

\[
\frac{dE}{dt} = \sum \dot{Q}_j - \sum \dot{W}_n
\]

\[
0, \text{ da keine Volumenarbeit}
\]

\[
U_2 - U_1 = Q_{12}
\]

\[
U_1 = U_f + x (u_g - u_f)
\]

\[
\text{Druck im EW bei Zustand 1:} \\
\text{aus Tabelle: Massenanteil} \\
\text{damit } p = \eta_1 U_{EW} \text{ (gegeben)}
\]

\[
\text{Aus Tab 1:}
\]

\[
u_f = -0,045 \frac{kJ}{kg} + 0,6 \left( -33,3 \frac{kJ}{kg} + 0,045 \frac{kJ}{kg} \right)
\]

\[
= -200,09 \frac{kJ}{kg}
\]

\[
U_1 = u_1 m_{EW} = -200 \frac{kJ}{kg}
\]

\[
U_2 = Q_{12} + U_1 = 1366,4 \, kJ - 200 \, kJ = -78,6336 \, kJ
\]

\[
x = \frac{U_2 - U_f}{u_g - u_f} = \frac{-78,6336 \, kJ + 0,045 \frac{kJ}{kg} \cdot 0,1 \, kg}{0,1 \, kg \left( -33,3 \frac{kJ}{kg} + 0,045 \frac{kJ}{kg} \right)}
\]

\[
= 0,5587
\]

\[
m_2 E_{1,0} = x_1 E_{1,1} \cdot m_{EW} = 0,5587 \cdot 0,1 \, kg = 55,87 \, g
\]

``````latex


