
``````latex


\section*{4 a)}

\begin{itemize}
    \item A graph is drawn with the x-axis labeled $T \, [^\circ C]$ and the y-axis labeled $p \, [\text{bar}]$.
    \item There is a curve starting from the origin and bending upwards to the right.
    \item Three points are marked on the curve: point 1 at the lower left, point 2 at the upper right, and point 3 at the middle of the curve.
    \item A horizontal line is drawn from point 2 to the right, and a vertical line is drawn from point 3 upwards to intersect the horizontal line at point 2.
    \item The region above the curve is labeled "flüssig" and the region below the curve is labeled "gas".
\end{itemize}

\section*{4) Verdichter:}

\begin{itemize}
    \item A circular diagram is drawn with three points labeled 1, 2, and 3.
    \item Point 1 is at the left, point 2 is at the top, and point 3 is at the right.
    \item An arrow labeled $W_k$ points upwards from the bottom of the circle.
    \item Another arrow labeled $m$ points to the right from point 1 to point 3.
\end{itemize}

\[
\Rightarrow 0 = \dot{m} (h_2 - h_3) - W_k
\]

\[
\frac{W_k}{h_2 - h_3} = \dot{m}
\]

\[
h_2 - h_3 (T_1 - t_0)
\]

\[
T_1 = -20^\circ C = 253.15 \, K
\]

\[
\Rightarrow T_2 = 247.15 \, K \quad (-26^\circ C)
\]

\[
\Rightarrow h_2 = h_3 (-26^\circ C) = 231.62 \, \frac{kJ}{kg}
\]

\[
h_3, s: \text{isotrop:} \quad s_3 = s_2 = s_0 (-26^\circ C) = 0.8930 \, \frac{kJ}{kgK}
\]

\[
\Rightarrow \varphi h_3 \text{ integration}
\]

\[
h_3 = h(40^\circ C, 85 \, m) + \frac{h(50^\circ C, 80 \, m) - h(40^\circ C, 80 \, m)}{0.8930 \, \frac{kJ}{kgK} - 0.8930 \, \frac{kJ}{kgK}} \cdot (0.8930 \, \frac{kJ}{kgK} - 0.8930 \, \frac{kJ}{kgK})
\]

\[
= 274.17 \, \frac{kJ}{kg}
\]

\[
\Rightarrow \dot{m} = \frac{-28 \, \frac{kJ}{s}}{h_2 - h_3} = 0.658 \cdot 10^{-4} \, \frac{kJ}{s} = 2.4 \, \frac{kJ}{s}
\]

``````latex


c) \quad x_{T_1}: \quad h_{u_1} - h_{q_1} = h_{1} + (185 - w) = 0.32 \cdot 42 \frac{w}{s}

\[
\Rightarrow \quad h_1 = h_{u_1} (1.0 + 0.89 \cdot w) + x_q \left( h_{q_2} (1.0 + 0.89 \cdot w) - h_q (1.0 + 0.89 \cdot w) \right)
\]

\[
\Rightarrow \quad x_{T_2} = \frac{h_{u_2} - h_{q_2}}{h_{q_2} - h_q} = 0.7566
\]

\[
\boxed{35.66\% = x_{T_1}}
\]

d) \quad E_{zu} = \frac{Q_{zu}}{W_{t}} \quad Q_{zu} = Q_u

\[
\Rightarrow \quad \text{(Diagram: A vertical rectangle with labels $Q_u$ entering from the left, $Q_{zu}$ exiting from the right, and $W_t$ exiting from the bottom)}
\]

\[
\Rightarrow \quad 0 = \dot{m} (h_2 - h_1) + Q_u
\]

\[
Q_u = \dot{m} (h_2 - h_1) =
\]

\[
h_2 = h_1 + 35.66\% (h_{q_2} - h_q) = 93.42 \frac{w}{s}
\]

\[
h_q = 23.1 \cdot 62 \frac{w}{s} = 0.084 \cdot 42 \frac{w}{s} = 88.91 W
\]

\[
\Rightarrow \quad E_k = \frac{Q_{zu}}{W_t} = \frac{91 W}{28 W} = \boxed{3.25}
\]

e) \quad die Temperatur zum würde sich -26°C annehmen, weil das eben die Temperatur des Kaltmittels ist

```