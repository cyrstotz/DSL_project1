
``````latex


\section*{Aufgabe 4}

\subsection*{a)}

\begin{itemize}
    \item[(i)] 
    \begin{description}
        \item[Graph Description:] 
        The graph is a Pressure-Temperature ($P$-$T$) diagram. The vertical axis is labeled $P$ with the unit $[bar]$. The horizontal axis is labeled $T$ with the unit $[U]$. There is a curve that starts from the bottom left, rises to a peak, and then falls back down to the bottom right, forming a bell shape. This curve is labeled as $NS$. There are four points marked on the graph: point 1 at the left base of the curve, point 2 at the peak, point 3 at the right base of the curve, and point 4 slightly above point 1 on the left side of the curve. The segment connecting points 1 and 2 is labeled "adiabat", the segment connecting points 2 and 3 is labeled "isobar", and the segment connecting points 3 and 4 is labeled "adiabat reversible".
    \end{description}

    \item[(ii)] 
    \begin{description}
        \item[Graph Description:] 
        The graph is a Pressure-Temperature ($P$-$T$) diagram. The vertical axis is labeled $P$ with the unit $[bar]$. The horizontal axis is labeled $T$ with the unit $[U]$. There is a curve that starts from the bottom left, rises to a peak, and then falls back down to the bottom right, forming a bell shape. This curve is labeled as $NS$. There is a point marked at the peak of the curve, labeled as point 2. The segment above point 2 is labeled "isotherm", and the point itself is labeled "Triplepunkt".
    \end{description}
\end{itemize}

\subsection*{b)}

\begin{equation*}
\frac{\partial E}{\partial t} = \sum_i \dot{m}_i(t) \left[ h_i(t) + \frac{v_i^2(t)}{2} + g z_i(t) \right] + \sum_j \dot{Q}_j(t) - \sum_k \dot{W}_k(t)
\end{equation*}

\begin{equation*}
\dot{W}_u = \dot{m} \left[ h_2 - h_3 \right]
\end{equation*}

\begin{equation*}
T_2 = T_1 - C_u
\end{equation*}

\begin{equation*}
P_1 = P_2
\end{equation*}

\begin{equation*}
P_3 = P_4 = \dot{m} \dot{v}
\end{equation*}

``````latex


\begin{enumerate}
    \item[(c)] 
    \[
    h = h_d + x \left( h_y - h_d \right)
    \]

    \item[(d)] 
    \[
    \epsilon_a = \left| \frac{\dot{Q}_{zu}}{\dot{W}_t} \right| - \frac{\left| \dot{Q}_{zu} \right|}{\dot{Q}_{ab} - \left( \dot{Q}_{zu} \right)}
    \]

    \item[(e)] 
    T: sind nach dem alles Sediment wurde
\end{enumerate}

```