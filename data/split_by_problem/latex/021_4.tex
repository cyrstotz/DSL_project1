
``````latex


\section*{Aufgabe 4}

\subsection*{a)}

\begin{description}
    \item[Graph Description:] The graph is a phase diagram with the x-axis labeled as $T \, [K]$ and the y-axis labeled as $p \, [\text{bar}]$. The graph shows a curve starting from the origin and rising upwards, representing the phase boundary between solid and vapor. The region below the curve is labeled "Dampf" (vapor), and the region above the curve is labeled "Solid + Dampf" (solid + vapor). There is a point on the curve labeled "Tripel" (triple point).
\end{description}

\subsection*{b)}

\begin{align*}
1 \rightarrow 2 &: \text{isobare Verdampfung} \quad p_1 = p_2 \quad x_1 = 1 \rightarrow \text{komplett verdampft} \\
2 \rightarrow 3 &: \text{adiab. isochor} \quad s_2 = s_3 \\
3 \rightarrow 4 &: \text{isobar} \quad p_3 = p_4 \quad x_4 = 0 \\
0 &= \dot{m} \left( h_2 - h_3 \right) - \dot{W}_k \\
h_3 &= h_2 (p = 8 \, \text{bar}) = 2694{,}15 \, \frac{\text{kJ}}{\text{kg}} \\
s_1 = s_2 &= s_g (p = 8 \, \text{bar}) = 0{,}906 \, \frac{\text{kJ}}{\text{kg} \cdot \text{K}} \\
p_4 &= 8 \, \text{bar} \\
h_4 &= h_f (p = 8 \, \text{bar}) = 53{,}42 \, \frac{\text{kJ}}{\text{kg}} \\
s_f (p = 8 \, \text{bar}) &= s_f = 0{,}3455 \, \frac{\text{kJ}}{\text{kg} \cdot \text{K}}
\end{align*}

```