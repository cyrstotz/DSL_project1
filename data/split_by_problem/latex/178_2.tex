
``````latex


\section*{Aufgabe 2}

\subsection*{Graphical Descriptions}

\subsubsection*{Top Graph}

The top graph is a plot with the x-axis labeled \( s \, [\frac{kJ}{kg}] \) and the y-axis labeled \( T \, [K] \). The graph contains a curve starting from the origin and moving upwards to the right. There are six points labeled 1 through 6. The points are connected by lines as follows:
- Point 1 to Point 2: a straight line.
- Point 2 to Point 3: a straight line labeled "isotherm".
- Point 3 to Point 4: a straight line labeled \( p_2 = p_3 \).
- Point 4 to Point 5: a straight line.
- Point 5 to Point 6: a straight line.

\subsubsection*{Middle Left Graph}

The middle left graph is a plot with the x-axis labeled \( s \) and the y-axis labeled \( T \, [K] \). The graph contains a curve starting from the origin and moving upwards to the right. There are six points labeled 1 through 6. The points are connected by lines as follows:
- Point 1 to Point 2: a straight line.
- Point 2 to Point 3: a straight line.
- Point 3 to Point 4: a straight line.
- Point 4 to Point 5: a straight line.
- Point 5 to Point 6: a straight line.

\subsubsection*{Middle Right Graph}

The middle right graph is a plot with the x-axis labeled \( s \, [\frac{kJ}{kg}] \) and the y-axis labeled \( T \, [K] \). The graph contains a curve starting from the origin and moving upwards to the right. There are six points labeled 1 through 6. The points are connected by lines as follows:
- Point 1 to Point 2: a straight line labeled "isotherm".
- Point 2 to Point 3: a straight line labeled "isotherm".
- Point 3 to Point 4: a straight line labeled "isotherm".
- Point 4 to Point 5: a straight line labeled "isotherm".
- Point 5 to Point 6: a straight line labeled "isotherm".

``````latex


b) \quad w_6, \; i_6

\text{Gleichung:}
\[
\frac{d}{dt} \left( \dot{m} \left[ h_5 + \frac{v_5^2}{2} \right] \right) = \dot{m} \left[ h_6 + \frac{v_6^2}{2} \right]
\]

\text{(ggf. Ideales Gas)}
\[
c_p \left[ T_5 + T_6 \right] = \frac{1}{2} \left[ v_5^2 - w_6^2 \right]
\]

\[
\frac{T_6}{T_5} = \left( \frac{p_6}{p_5} \right)^{\frac{n-1}{n}}
\]

\[
\Rightarrow \quad T_6 = T_5 \left( \frac{p_6}{p_5} \right)^{\frac{n-1}{n}}
\]

\[
\Rightarrow \quad T_6 = 493.15 \cdot \left( \frac{0.184}{0.5} \right)^{\frac{1.4-1}{1.4}}
\]

\[
= 328.1 \, \text{K}
\]

\[
w_6 = \sqrt{2 \, c_p \left[ T_5 - T_6 \right] + v_5^2}
\]

\[
w_6 = \sqrt{2 \cdot 1.006 \left( 493.15 - 328.1 \right) + v_5^2}
\]

\[
= 507.2 \, \frac{m}{s}
\]

``````latex


\section*{c) Aufgabe}

\subsection*{Ansatz 1}

\begin{equation*}
\frac{d\tilde{x}}{dt} = \sum \tilde{E}_{k, str, i}(t) + \sum \tilde{E}_{k, aq, i}(t) - \sum \tilde{\psi}_{n+1}(t) - \rho_0 \frac{d\tilde{U}(t)}{dt} - \tilde{E}_{k, vert}(t)
\end{equation*}

\subsection*{E_k, str}

\begin{equation*}
E_{k, str} = \tilde{m} (\tilde{h} - h_0 - \tilde{r}_0 (s_i - s_0) tke)
\end{equation*}

\subsection*{E_b}

\begin{equation*}
e_{k, str} = \tilde{h}_1 - h_0 - \tilde{r}_0 (s_i - s_0) tke
\end{equation*}

\begin{equation*}
= c_p (\tilde{T}_1 - \tilde{T}_0) - \tilde{r}_0 c_p \ln \left( \frac{\tilde{T}_1}{\tilde{T}_0} \right) - R \ln \left( \frac{\tilde{p}_1}{p_0} \right) + \frac{1}{2} \tilde{v}^2
\end{equation*}

``````latex


d)

\[
\frac{d\cancel{S}}{dt} = \sum \dot{m}_i s_i(t) + \int \frac{\dot{Q}(t)}{T_G(t)} + \dot{S}_{erzeug}(t)
\]

a)

\[
\begin{array}{c}
\text{Graph of } T \text{ vs. } S \\
\text{Vertical axis: } T \, [\text{kJ}] \\
\text{Horizontal axis: } S \, [\text{kJ/kg}] \\
\end{array}
\]

The graph is a closed loop with the following points and segments:
- Point 1 at the origin (0,0).
- Point 2 is above point 1, connected by a vertical line.
- Point 3 is to the right of point 2, connected by a line labeled "isobare".
- Point 4 is below point 3, connected by a vertical line.
- Point 5 is to the left of point 4, connected by a line labeled "isokare".
- Point 6 is below point 5, connected by a vertical line labeled "isobare".
- Point 6 connects back to point 1, completing the loop.

The points are labeled as follows:
- Point 1: 1
- Point 2: 2
- Point 3: 3
- Point 4: 4
- Point 5: 5
- Point 6: 6

``````latex


