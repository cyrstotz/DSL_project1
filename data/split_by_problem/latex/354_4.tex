
``````latex


4.

\begin{itemize}
    \item A graph is drawn with the y-axis labeled as $Q_{abgegeben}$ and the x-axis labeled as $T$. The y-axis has three points marked: one at the origin, one at a point labeled $Q_{abgegeben}$, and one at a point labeled $Q_{zugef}$. The x-axis has points marked at $-50^\circ C$, $-40^\circ C$, $-30^\circ C$, $-20^\circ C$, $-10^\circ C$, and $0^\circ C$. There is a curve starting from the origin and increasing non-linearly towards the right. A point on the curve is marked with $10K$ and another point is marked with $2t$. A vertical line is drawn from the x-axis at $0^\circ C$ intersecting the curve at a point labeled $5mbar$.
\end{itemize}

b)

\begin{itemize}
    \item A small diagram is drawn, but it is not clear what it represents.
\end{itemize}

Temperatur im Verdampfer: $T_i = G_k$

Titus Diagramm: $-10^\circ C \quad T_f = -16^\circ C$

Aus $x_2 = n$ findet man die Entropie und Enthalpie aus Tabelle A-10

\[
s_g(-10^\circ C) = 0.5298 \frac{kJ}{kgK} \quad h_g(-10^\circ C) = 237.74 \frac{kJ}{kgK}
\]

Da der Kompressor adiabt reversibel ist gilt

\[
S_2 = S_3
\]

``````latex


Aus Tabelle A12 kann man nun mittels der Entropie die Enthalpie bestimmen. Man weiß zudem aus A10

\[
s_2 = s_3 = 6.528 \frac{\text{kJ}}{\text{kg K}}
\]

\[
h_3 = \frac{h(8 \text{bar}, 90^\circ \text{C}) - h(8 \text{bar}, T_{\text{satt}})}{s_3 - s(8 \text{bar}, T_{\text{satt}})} + h(8 \text{bar}, T_{\text{satt}})
\]

\[
= 277.37 \frac{\text{kJ}}{\text{kg K}}
\]

Daher: Für die Bilanzgleichung und den Kompressor erhalten wir adiabatisch

\[
0 = \dot{m} (h_2 - h_3) + \dot{Q} - \dot{W}
\]

\[
\dot{m} = \frac{\dot{W}}{(h_2 - h_3)} = 8.34 \cdot 10^{-4} \frac{\text{kg}}{\text{s}}
\]

c) adiabate Drossel - isenthalp Tabelle A11

\[
h_4 = h_3
\]

\[
h_3 = h_f (8.0 \text{bar}) = 93.92 \frac{\text{kJ}}{\text{kg}}
\]

Der Verdampfer im Wasserdampfgebiet liegt und bei -16°C operiert. Kann man das Dampfgehalt ausrechnen Tabelle A-10

\[
X_m = \frac{h_1 - h_f (-16^\circ \text{C})}{h_g (-16^\circ \text{C}) - h_f (-16^\circ \text{C})} = 30.76\%
\]

``````latex


\section*{Aufgabe 4}

\subsection*{d)}

\begin{equation*}
    \epsilon = \frac{|Q_{20}|}{|Q_{AB}| - |Q_{20}|} = \frac{|Q_{20}|}{|W|} = 4.239
\end{equation*}

\begin{equation*}
    Q_{20} = \dot{Q_k} = \dot{m}(h_2 - h_1) = \dot{m} \cdot 120.36
\end{equation*}

\begin{equation*}
    h_1 = h_2 = 93.42 \frac{kJ}{kg}
\end{equation*}

\begin{equation*}
    h_2 = h_2(60^\circ C) = 237.70 \frac{kJ}{kg}
\end{equation*}

\begin{equation*}
    totale \Delta = 10
\end{equation*}

\subsection*{e)}

Die Temperatur würde weiterfallen während der Druck konstant bleibt. Deshalb würde es ins Nassdampfgebiet übergehen.

```