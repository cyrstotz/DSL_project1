
``````latex


\section*{Aufgabe 4}

\subsection*{i)}
\begin{tabular}{|c|c|c|c|c|c|c|c|}
\hline
Zustand & T & p & G & W & n & s & Notizen \\
\hline
1 & Ti-G & p1 & Qk & & 93\% & & Ti-Qk \\
\hline
2 & -Qk & p1 & 0 & 88\% & 249,53 & 0,91 & notwendige Verdampfung $x_2 = 1$ \\
\hline
3 & & 8 bar & & & 0,946 & & $x_3$ \\
\hline
4 & & 8 bar & 0 & & 93\% & & gesamte rote Hand $x_4 = 0$ \\
\hline
\end{tabular}

Für die adiabate Drossel gilt $h_1 = h_2$

\subsection*{ii)}
Ti konst. $p_i = 1 \text{ mbar} \quad (p_6 \text{ mbar} - p_5 \text{ mbar})$ \\
geforderte Entspannung \\
Graphisch \\
$\Pi = 10 K$ über Sublimationspunkt \\
$T_i = 10 K + 273,15 K = 283,15 K$

\subsection*{iii)}
\textbf{ges. $m_{Kreis}$}

Energieäquivalenz von (1) $\rightarrow$ (3)
\[
\frac{dE}{dt} = \sum \dot{m} (h + \frac{c^2}{2} + gz) - \dot{Q} - \sum \dot{W}
\]
\[
0 = \dot{m} (h_1 - h_3) - W_k
\]
\[
\dot{m} = \frac{W_k}{h_1 - h_3}
\]
\[
W_k = 0, da zugeführt
\]

Aus TAB A-11: \\
\[
h_f (8 \text{ bar}) = 93,48
\]
\[
h_g (8 \text{ bar}) = 204,15
\]
\[
s_f (8 \text{ bar}) = 0,3459
\]
\[
s_g (8 \text{ bar}) = 0,9106
\]

\[
\text{AM} \\
\text{Aus TAB } h_1 = h_2 = h_f (8 \text{ bar}) = 93,48
\]

\[
T_i = T_i - Q_k = 273,15 K
\]

\[
\text{Aus TAB A-10:} \quad \text{Anteil Verdampfung} \quad s_f (4^\circ C) = 249,53
\]
\[
s_g (4^\circ C) = 0,9106
\]

\[
x_3 = \frac{s_3 - s_f}{s_g - s_f} = \frac{}{} > 1 \rightarrow idealisierter Zustand
\]

``````latex


\section*{c)}
\text{Annahme:} \quad \dot{m} = 4 \frac{\text{kg}}{\text{h}}, \quad T_{se} = 52^\circ \text{C} \\
\text{ges} \quad x_1 \\

\text{Aus T\&B A-10} \quad h_{g}(52^\circ \text{C}) = 234,28 \frac{\text{kJ}}{\text{kg}} \\
\text{(wir wissen h}_4 = 93,49 \frac{\text{kJ}}{\text{kg}}) \\
\text{und aus T\&B A-10} \quad p_{4}(52^\circ \text{C}) = 1,24992 \text{bar} = p_2 \\

\text{h}_2 \text{ und h}_3 \text{ aus T\&B A-10} \\
h_{g2} = 234,28 \frac{\text{kJ}}{\text{kg}} \\
h_{f2} = 93,49 \frac{\text{kJ}}{\text{kg}} \\

x_1 = \frac{h_1 - h_{f2}}{h_{g2} - h_{f2}} \quad \text{mit} \quad h_4 = 93,49 \\

\Rightarrow x_1 = 0,3375 \\

\section*{d)}
\text{ges.} \quad \dot{E}_k = \frac{\dot{Q}_{zu}}{\dot{W}_t} \quad \text{mit} \quad \dot{W}_t = 28 \, \text{W} \\

\text{Energieanalyse} \quad 1 \rightarrow 5 \quad \dot{m} (h_1 - h_2) + \dot{Q} = 0 \\
\dot{Q} = \dot{m} (h_2 - h_1) \\
\Rightarrow \dot{Q} = 4 \frac{\text{kg}}{\text{h}} \cdot \frac{1}{3600} \left( 93,49 - 234,28 \right) \\
\Rightarrow \dot{Q} = 9,173 \, \text{kW} \\

\Rightarrow \dot{E}_k = 6,18 \\

\section*{e)}
\section*{f)}

\section*{Graph Description}
The graph is a plot with the vertical axis labeled as \( P \) and the horizontal axis labeled as \( T \). The graph shows a curve that starts from the origin and rises upwards to the right. There is a point marked on the curve, and a vertical line is drawn from this point upwards. The curve appears to be non-linear, indicating a possible exponential or logarithmic relationship between \( P \) and \( T \).

```