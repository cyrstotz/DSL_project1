
``````latex


\section*{Problem 1(a)}

\begin{itemize}
    \item The graph is a Pressure-Volume (P-v) diagram.
    \item The y-axis is labeled as \( P \) in \( \text{MPa} \).
    \item The x-axis is labeled as \( v \) in \( \text{m}^3/\text{kg} \).
    \item There are several points and curves on the graph:
        \begin{itemize}
            \item Point 1 is at the intersection of the isobar and the isotherm lines.
            \item Point 2 is on the isobar line, to the right of point 1.
            \item Point 3 is on the isotherm line, above point 2.
            \item Point 4 is on the isotherm line, above point 1.
            \item The curve labeled "isobar" connects points 1 and 2.
            \item The curve labeled "isotherm" connects points 2 and 3.
            \item The curve labeled "isotherm" connects points 1 and 4.
            \item The curve labeled "Nass-Dampf" connects points 1 and 3.
            \item The curve labeled "gas" connects points 3 and 4.
            \item The curve labeled "flüssig" connects points 1 and 2.
        \end{itemize}
    \item The critical point \( T_{\text{crit}} \) is marked on the isotherm line.
\end{itemize}

\section*{Problem 1(b)}

\begin{itemize}
    \item The energy balance equation is written as:
    \[
    \dot{Q} = \dot{m}_{\text{R134a}} \left( h_2 - h_3 \right) + \dot{Q}_{\text{stationär}} - W_{\text{t}}
    \]
\end{itemize}

```