
``````latex


\section*{Problem 3}

\[
p = \frac{F}{A}
\]

\subsection*{a)}
\[
p_1 = p_{\text{amb}} + m_k \cdot \frac{1}{\pi \left( \frac{D}{2} \right)^2} g + m_{\text{EW}} \cdot \frac{1}{\pi \left( \frac{D}{2} \right)^2} g
\]

\[
= p_{\text{amb}} + \frac{m_k \cdot g}{\pi \left( \frac{D}{2} \right)^2} + \frac{m_{\text{EW}} \cdot g}{\pi \left( \frac{D}{2} \right)^2} = 71.9005 \, \text{bar}
\]

\subsection*{b)}
\[
p \cdot V = m \cdot R \cdot T \quad \text{und} \quad R = \frac{\bar{R}}{M} = 186.28 \, \frac{\text{J}}{\text{kmol} \cdot \text{K}}
\]

\[
m_1 = \frac{p_1 V_1}{R T_1} = 3.427 \cdot 10^{-3} \, \text{kg}
\]

\subsection*{c)}
\[
m_1 = m_2 \quad \text{und} \quad p_1 = p_2 \quad \text{weil gleiches "Gewicht" von dem Druckstück!!}
\]

\[
\Rightarrow \text{isotherme Polytrope Veränderung}
\]

\[
T_2 = T_1 \left( \frac{V_1}{V_2} \right)^{n-1}
\]

\[
p_2 \cdot v_2 = R \cdot T_2 \quad \Leftrightarrow \quad v_2 = \frac{R \cdot T_2}{p_2}
\]

\[
T_2 = T_1 \left( \frac{V_1}{\frac{R T_2}{p_2}} \right)^{n-1} = T_1 \cdot \frac{R T_2}{V_1 p_2}
\]

``````latex


\section*{Problem c}

\begin{align*}
    T_{912} &= 0.003^\circ C \\
    p_2 &= 7.900 \, bar \\
    m_2 &= 3.427 \cdot 10^{-3} \, kg
\end{align*}

1. HS über Gas: \quad \text{nach Lösens S. 5/6m.}

\begin{align*}
    \Delta U_{12} &= m Q_{12} - \cancel{W^0} \\
    & \quad \text{KE + PE vernachlässigbar} \\
    T_2 &= 500^\circ C
\end{align*}

\begin{align*}
    \Delta U_{12}^d &= C_V (T_2 - T_1) = -376.49 \, kJ \\
    & \quad \text{ideales Gas}
\end{align*}

\begin{align*}
    U_{12} = m_2 \cdot \Delta u_{12} &= 1.0824 \, kJ = \underline{Q_{12}}
\end{align*}

\section*{Problem d}

1. HS über Eism. \quad \text{nach Lösens S. 5/6m.}

\begin{align*}
    m_{EW} \Delta u_{12} &= Q_{12} \quad \# \quad V^0
\end{align*}

\begin{align*}
    \Delta u_{12} &= \frac{Q_{12}}{m_{EW}}
\end{align*}

\begin{align*}
    \Delta U_{12} &= U_2 - U_1
\end{align*}

\begin{align*}
    p_{2EW} &= p_{12EW} - 7.6 \, bar
\end{align*}

\begin{align*}
    U_{12} \quad \text{bei} \quad (7.6 \, bar) &= U_{Fest} + x_1 (U_{Fusion} - U_{Fest})
\end{align*}

\begin{align*}
    U_2 &= \frac{Q_{12} + U_1 \cdot m_{EW}}{m_{EW}} = \frac{Q_{12}}{m_{EW}} + U_1
\end{align*}

\begin{align*}
    U_1 \quad \text{bei} \quad 0^\circ C &= \text{TAB.} \\
    \text{Druck auf EW:} \quad p_{12} &= 7.394 \, bar \quad 3.746 \, bar
\end{align*}

\begin{align*}
    U_2 &= U_{Fest} + x_1 (U_{Fusion} - U_{Fest}) \\
    &= \frac{137.7 \, kJ}{kg} - \frac{200.9254 \, kJ}{kg}
\end{align*}

\begin{align*}
    x_1 &= 1 - x_{Eis} = 0.0
\end{align*}

\begin{align*}
    x_2 &= \frac{U_2 - U_{Fest}}{U_{Fusion} - U_{Fest}} = 0.429
\end{align*}

\begin{align*}
    x_{2Eis} &= 1 - 0.429 = 0.570
\end{align*}

``````latex


