
``````latex


\section*{4.}

\subsection*{a)}

\begin{center}
\begin{tabular}{c}
\begin{minipage}{0.8\textwidth}
\textbf{Graph Description:} The graph is a pressure-volume (p-V) diagram. The x-axis is labeled as $T$ (temperature) and the y-axis is labeled as $p$ (pressure). There are four points labeled 1, 2, 3, and 4. The graph shows a closed loop with the following details:
\begin{itemize}
    \item Point 1 to Point 2: A horizontal line labeled "isobar".
    \item Point 2 to Point 3: A curved line moving upwards labeled "Nassdampfgebiet".
    \item Point 3 to Point 4: A horizontal line labeled "isobar".
    \item Point 4 to Point 1: A curved line moving downwards.
\end{itemize}
\end{minipage}
\end{tabular}
\end{center}

\subsection*{b)}

\[
0 = \dot{m} (h_e - h_a) + Q - W
\]

\subsection*{c)}

\[
x = \frac{m_g}{m_g + m_f}
\]

\subsection*{d)}

\[
E_k = \frac{|\dot{Q}_{2c}|}{|W_{1c}|} = \frac{|\dot{Q}_{2c}|}{|Q_{a4}| - |\dot{Q}_{2c}|}
\]

\subsection*{e)}

``````latex

\section*{4(e)}
Die Temperatur würde weiter fallen, weil Wärme abgeführt wird.

```