
``````latex


\section*{Aufgabe 4}

\subsection*{a)}
p-T Diagramm

\begin{description}
    \item[Graph:] Ein Diagramm mit der x-Achse beschriftet als $p/T$ und der y-Achse beschriftet als $p$. Es gibt vier markierte Punkte: 1, 2, 3, und 4. Punkt 1 ist unten links, Punkt 2 ist unten rechts, Punkt 3 ist oben rechts, und Punkt 4 ist oben links. Die Punkte 1 und 2 sind durch eine horizontale Linie verbunden, ebenso wie die Punkte 3 und 4. Die Punkte 1 und 4 sind durch eine vertikale Linie verbunden, ebenso wie die Punkte 2 und 3. Es gibt eine schraffierte Fläche zwischen den Punkten 1 und 4.
\end{description}

\subsection*{b)}
ges. in $p_2$\\
der Kreisprozess:

\[
0 = \dot{m} (h_e - h_a) - \dot{Q}_{kj} - \sum \dot{W}_{trm}
\]

\[
0 = \dot{m} (h_2 - h_3) + \dot{Q}_{ab}
\]

\[
\frac{\dot{Q}_{ab}}{h_2 - h_3} = \dot{m}
\]

\[
p_2 = 1 \, \text{bar}
\]

\subsection*{c)}
\[
p_i = p_s = 8 \, \text{bar}
\]

\[
p_i = \text{Inter}
\]

\subsection*{d)}
\[
E_K = \frac{\dot{Q}_{ab}}{\dot{W}_t} = \frac{\dot{Q}_{ab}}{\dot{Q}_{ab} - \dot{Q}_{zu}}
\]

\subsection*{e)}
Temperatur sinkt da höher messströmen durch Z und deshalb 1/ fließen und wegen der größeren Druckdifferenz an der Verdichter (2 > 3)
```