
``````latex


\section*{Aufg. 3}

\begin{align*}
A_2 &= 5 \text{cm}^2 \eta = 0.007854 \text{m}^2 \\
C_v &= 0.633 \frac{R}{\text{kg K}} = 633 \frac{J}{\text{kg K}} \\
M_g &= 50 \frac{\text{kg}}{\text{kmol}} = 50 \frac{g}{\text{mol}} \\
R &= \frac{\bar{R}}{M_g} = 166.28 \frac{J}{\text{kg K}} \\
p_1 V_1 &= R T_1 \\
P_1 &= p_{\text{atm}} + \frac{m_k g}{A_2} + \frac{m_{ew} g}{A_2} = 1.4009 \text{bar} \\
V_1 &= \frac{R T_1}{P_1} = 0.91769 \frac{\text{m}^3}{\text{kg}} \\
\frac{V_1}{V_2} &= m_g = 0.0034216 \text{kg} \\
P_1 &= P_2 \text{ da es gleichviel gewillt von außen hat}
\end{align*}

\noindent
Es wird mit $T_2 = 0.003^\circ \text{C}$ festgefahren

\begin{align*}
T_2 &= 273.18 \text{K} \\
V_2 &= m_g V_1 \cdot \frac{R T_2}{P_2 m_g} = 0.001076 \text{m}^3 \\
\Delta W_{12} &= \Delta V P_2 = -284.72 \text{J} \\
\Delta E &= \Delta U_g = m_g (u_1 - u_2) + \Delta W + \Delta Q_n = -316.78 \text{kJ} - 1.36667 \\
u_1 - u_2 &= C_v \Delta T
\end{align*}

\noindent
\textit{back}
``````latex

\begin{align*}
p &= 1,4 \text{bar} \\
T &= 0^\circ C \quad \text{alles bei } 1,4 \text{bar} \quad \text{Tabelle 1} \\
U_1 &= 0,64 \, u_{\text{fest}}(0^\circ) = 0,4 \, u_{\text{fest}}(0^\circ) = -4,3374 \, \frac{kJ}{kg} \\
U_2 &= U_{\text{fl}}(100^\circ) = 0,5 \, u_{\text{fl}} = 0,5 \, \frac{kJ}{kg} \\
\Delta E &= \Delta U = m_w (u_2 - u_1) = \Delta Q + \Delta A \\
\frac{Q_1 + m_w u_1}{m_w} &= U_2 = -273,28 \, \frac{kJ}{kg} = -419,72 \, \frac{kJ}{kg} = -1,89 \, \frac{kJ}{kg} \\
\frac{U_2 - U_{\text{fest}}}{U_{\text{flüss}} - U_{\text{fest}}} &= x_2 = 0,64406 = 1,5607
\end{align*}

``````latex


