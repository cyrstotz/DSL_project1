
``````latex


\section*{2(a)}

\[
\begin{array}{cccccccccccc}
0 & \rightarrow & 1 & \text{adiabat} & R < 1 & \rightarrow & 2 & \rightarrow & 3 & \text{isobar} & \rightarrow & 4 \\
\text{adiabat} & R < 1 & \rightarrow & 5 & \rightarrow & 6 & \text{isobar} & \rightarrow & 7 & \rightarrow & 8 & \text{isobar} \\
\end{array}
\]

\section*{Kernstrom}

\begin{description}
\item[Graph Description:] The graph is a Temperature-Entropy (T-s) diagram. The x-axis is labeled $s \left[ \frac{J}{kg \cdot K} \right]$ and the y-axis is labeled $T [K]$. The graph contains several lines and points:
\begin{itemize}
    \item Point 1 is at the bottom left.
    \item Point 2 is above point 1, connected by a vertical line labeled "adiabat".
    \item Point 3 is to the right of point 2, connected by a horizontal line labeled "isobar".
    \item Point 4 is below point 3, connected by a vertical line labeled "adiabat".
    \item Point 5 is to the left of point 4, connected by a horizontal line labeled "isobar".
    \item Point 6 is below point 5, connected by a vertical line labeled "adiabat".
    \item Point 7 is to the right of point 6, connected by a horizontal line labeled "isobar".
    \item Point 8 is below point 7, connected by a vertical line labeled "adiabat".
    \item There are several isothermal lines and isobaric lines drawn parallel to the x-axis and y-axis respectively.
    \item The region between points 1, 2, 3, and 4 is shaded.
    \item The pressures $P_1 = P_3 = 0.5 \, \text{bar}$ and $P_6 = P_0 = 0.15 \, \text{bar}$ are indicated.
\end{itemize}
\end{description}

\section*{b)}

\[
\omega_6 = ? \quad T_6 = ?
\]

\[
T_5 = 431.9 \, K, \quad P_3 = 0.5 \, \text{bar}, \quad \omega_5 = 220 \, \frac{m}{s}, \quad P_6 = P_0 = 0.15 \, \text{bar}
\]

\[
n = \kappa = 1.4
\]

\[
\left( \frac{T_2}{T_1} \right) = \left( \frac{P_2}{P_1} \right)^{\frac{n-1}{n}} \quad \rightarrow \quad \left( \frac{T_6}{T_5} \right) = \left( \frac{P_6}{P_5} \right)^{\frac{n-1}{n}}
\]

\[
\rightarrow T_6 = T_5 \left( \frac{P_6}{P_5} \right)^{\frac{n-1}{n}} = 328.07 \, K
\]

\[
0 = \dot{m} \left[ h_6 - h_5 + \frac{\omega_6^2}{2} - \frac{\omega_5^2}{2} \right] \quad \rightarrow \quad 0 = h_5 - h_6 + \frac{\omega_6^2}{2} - \frac{\omega_5^2}{2}
\]

\[
\rightarrow h_6 - h_5 = \frac{\omega_5^2}{2} - \frac{\omega_6^2}{2} \quad \rightarrow \quad \omega_6 = \sqrt{\omega_5^2 + 2 \left( h_5 - h_6 \right)}
\]

\[
h_5 = h_4 + (431.9 \, K - 430 \, K) \quad \rightarrow \quad h_5 = (440.6 - 431.9) \frac{kJ}{kg} + 431.9 \```latex


\section*{Problem c}

\[
\Delta e_{x,str} = e_{x,5+6} - e_{x,str,0}
\]

\[
e_{x,5+6,0} - e_{x,str,0} = \phi \left[ h_6 - h_0 - T_0 (s_4 - s_5) + \frac{\omega_2^2}{2} - \frac{\omega_1^2}{2} \right]
\]

\[
h_6 = 328.4 \frac{J}{kg}, \quad \omega_1 = 5.10 \frac{m}{s}, \quad T_6 = 340 K
\]

\[
h_6 = A-22 \left( \frac{243.15 - 240}{k} \right) = (250.01 - 240.02) \frac{k}{k} - 240.02
\]

\[
\Delta e_{x,str} = \left[ c_p (T_6 - T_0) - T_0 \left( c_p \ln \left( \frac{T_6}{T_0} \right) - R \ln \left( \frac{P_6}{P_0} \right) - \frac{\omega_2^2}{2} - \frac{\omega_1^2}{2} \right) \right]
\]

\[
= \left[ 1.006 \frac{kJ}{kg \cdot K} (340 K - 243.15 K) - 243.15 K \left( 1.006 \frac{kJ}{kg \cdot K} \ln \left( \frac{340 K}{243.15 K} \right) - R \ln \left( \frac{P_6}{P_0} \right) \right) + \frac{(5.10 \frac{m}{s})^2}{2} - \frac{(200 \frac{m}{s})^2}{2} \right]
\]

\[
= 12.127 \pm 8 J = 12.127 \pm 8 J
\]

\section*{Problem d}

\[
e_{x,usel} = T_0 s_{c2}
\]

\[
s_{c2} \Rightarrow s_2 = s_n = \sum_i \frac{Q_i}{T_i} + \delta s_{c2}
\]

\[
\Rightarrow s_{c2} = s_{c2} \Rightarrow \frac{q_B}{T_B} = \frac{q_B}{T_B} = c_p \left( \frac{T_6}{T_0} \right) - R \ln \left( \frac{P_6}{P_0} \right) - \frac{q_B}{T_B}
\]

\[
= 1.006 \frac{kJ}{kg \cdot K} \ln \left( \frac{340 K}{243.15 K} \right) - \frac{1.55 \frac{kJ}{kg}}{T_B} = 0.8411 = -0.4
\]

``````latex


