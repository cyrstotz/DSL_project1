
``````latex


\section*{4)}

\subsection*{a)}

\textbf{Verbal Description of the Graph:}

The graph is a pressure-volume (p-V) diagram with the y-axis labeled as \( p \) in bar and the x-axis labeled as \( T \) in Kelvin. The graph shows a closed cycle with four points labeled 1, 2, 3, and 4. The cycle starts at point 1, moves to point 2, then to point 3, and finally to point 4 before returning to point 1. The path from point 1 to point 2 and from point 3 to point 4 is a straight line, indicating an isobaric process at 8 bar. The path from point 2 to point 3 and from point 4 to point 1 is curved, indicating an isothermal process. The maximum pressure reached in the cycle is 150 bar.

\subsection*{b)}

\[
x_2 = 1 \quad p_1 = p_2 \quad p_3 = 8 \text{ bar} \quad x_4 = 0
\]

\[
E\text{-Bil Verdichter:} \quad \dot{Q} = \dot{m} \cdot (h_2 - h_3) - \dot{W}_K
\]

\[
\dot{m} = \frac{\dot{W}_K}{h_2 - h_3}
\]

\[
s_2 = s_3 \quad p_1 = p_2 = 8 \text{ bar} \quad p_{\text{Innenraum}} = 1 \text{ mbar}
\]

\subsection*{c)}

\[
\dot{m} = \frac{\dot{Q}}{h} \quad T_{1} = -22^\circ \text{C} \quad \text{aus annahmen}
\]

\[
p_{02} = 1.292 \text{ bar} \quad TAB A-10 \quad p_2 = p_4
\]

\[
E\text{-Bil Drossel:} \quad 0 = \dot{m} (h_4 - h_1) \Rightarrow h_2 = h_4
\]

\[
h_4 = 93.62 \frac{\text{kJ}}{\text{kg}} = h_1 \quad h_2 = 234.08 \frac{\text{kJ}}{\text{kg}} \quad TAB A-10
\]

\[
x = \frac{h_4 - h_f}{h_g - h_f} = \frac{93.62 - 0}{234.08 - 0} = 0.337
\]

\subsection*{d)}

\[
E_K = \dot{Q}_{zu} = 1
\]

\[
E_{zu} = \dot{Q}_{zu} \quad \dot{Q}_{zu} = \dot{Q}_{zu} \quad \dot{W}_{zu} = \dot{W}_K
\]

\[
E\text{-Bil Verdampfer:} \quad 0 = \dot{m} (h_1 - h_2) + \dot{Q}_{zu}
\]

\[
h_2 \text{ aus c)} \quad h_2 = 234.08 \frac{\text{kJ}}{\text{kg}} \quad TAB A-10
\]

\[
\dot{Q}_{zu} = \dot{m} (h_1 - h_2) = 0.00 \frac{1}{\text{s}} (2500 \frac{\text{kJ}}{\text{kg}} - 93.62 \frac{\text{kJ}}{\text{kg}}) = 0.146 \text{ kW}
\]

\[
E_K = \frac{\dot{Q}_{zu} + \dot{W}_{zu}}{3.68 \text{ kW}} = 0.005
\]

\subsection*{e)}

Die Temperatur würde sinken. Der Innendruck ist konstant 1 mbar und Volumen bleibt auch unverändert. Die einzige Möglichkeit Energie aus dem System zu nehmen, ist durch das Sinken der Temperatur.

```