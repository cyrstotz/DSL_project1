
``````latex


\section*{Aufgabe 3}

\subsection*{a)}

\[
A = \pi \left( \frac{D}{2} \right)^2 = \pi \left( \frac{0,1\, \text{m}}{2} \right)^2 = \pi \cdot 0,05^2 = 7,854 \cdot 10^{-3} \, \text{m}^2
\]

\[
F_{\text{gmk}} = 38 \, \text{kg} \cdot 32 \, \text{kg} \cdot 9,81 \, \frac{\text{m}}{\text{s}^2} = 3,13 \, \text{N}
\]

\[
F_{\text{gkw}} = 0,1 \, \text{kg} \cdot 9,81 \, \frac{\text{m}}{\text{s}^2} = 0,981 \, \text{N}
\]

\[
P = P_{\text{amb}} + \frac{F_{\text{sk}}}{A} + \frac{F_{\text{gkw}}}{A} = 1,401 \, \text{bar}
\]

\[
mg: \quad \rho V = m^* RT \implies m_g = \frac{\rho V}{RT} = \frac{M_g \cdot \rho_g \cdot V_g}{R \cdot T_g} = \frac{50 \, \frac{\text{kg}}{\text{mol}} \cdot 1,401 \cdot 10^5 \, \text{Pa} \cdot 3,14 \cdot 10^{-3} \, \text{m}^3}{R \cdot 773,15 \, \text{K}} = 3,42 \, \text{kg}
\]

\subsection*{b)}

Der Druck wird gleich bleiben, da die Gewichtskraft des Gases gleich bleibt.

Die Temperatur wird auf 0°C fallen, da sich die Temperatur des Gases bis zum kompletten Aufschmelzen des Eises nicht ändern wird.

\subsection*{c)}

System um das Gas:

\[
\frac{dE}{dt} = \dot{m} \cdot \dot{Q} + \dot{w} \implies \Delta U = Q - W
\]

\[
W = p \int_{V_1}^{V_2} dV = p (V_2 - V_1) = 1,401 \cdot 10^5 \, \text{Pa} (1,105 \cdot 10^{-3} - 3,14 \cdot 10^{-3}) \, \text{m}^3 = -284,483 \, \text{J}
\]

\[
V_2 = \frac{m_g RT_2}{M_g p_2} = \frac{3,42 \, \text{kg} \cdot R \cdot 273,15 \, \text{K}}{50 \, \frac{\text{kg}}{\text{mol}} \cdot 1,401 \cdot 10^5 \, \text{Pa}} = 1,105 \cdot 10^{-3} \, \text{m}^3
\]

\[
m (h_2 - h_1) = m \cdot c_v (T_2 - T_1) = Q + 284,483 \, \text{J}
\]

\[
Q = 3,42 \, \text{kg} \cdot 6,35 \, \frac{1}{\text{kg} \cdot \text{K}} (273,15 \, \text{K} - 773,15 \, \text{K}) - 284,483 \approx 1082 \, \text{J}
\]

``````latex


