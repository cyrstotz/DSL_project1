
``````latex


\section*{Aufgabe 3}

\subsection*{a)}

\begin{itemize}
    \item Diagramm: Ein horizontaler Zylinder mit einem Kolben in der Mitte. Links vom Kolben ist der Druck $p_\text{amb}$, rechts vom Kolben ist der Druck $p_\text{gas}$. Auf den Kolben wirkt eine Kraft $mg$ nach unten und eine Fläche $A$ nach oben.
\end{itemize}

\[
p_\text{gas} = p_\text{amb} + \frac{mg}{A}
\]

\[
p_\text{gas} = 1.399 \, \text{bar} \approx 1.4 \, \text{bar}
\]

\subsection*{Perfektes Gas}

\[
p_\text{gas} V_6 = m_\text{gas} RT
\]

\[
m_\text{gas} = \frac{p_\text{gas} \cdot V_6}{RT}
\]

\[
m_\text{gas} = \frac{3.14}{3.42} = 3.42 \, \text{g}
\]

\subsection*{b)}

\[
X_{Eis,1} > 0
\]

\[
X_{Eis,1} = 0.6
\]

\[
m_{EW} = 0.1 \, \text{kg}
\]

\[
T_{EW} = 0^\circ \text{C}
\]

\[
T_{s,1} = ???
\]

\begin{itemize}
    \item Text: Temperatur vom Gas wird kleiner, da Wärme in das EW fließt. Der Druck bleibt jedoch konstant, da das Equilibrium vom Außendruck abhängt.
    \item Text: Das Equilibrium ist nun an einem anderen Punkt.
\end{itemize}

\subsection*{d) Geschlossenes System}

\[
\Delta E = \Sigma Q - \Sigma W
\]

\[
\Delta U = \Sigma Q
\]

\[
m_2 (u_2 - u_1) = Q
\]

\[
p_{g,2} = p_{g,1} = 1.399 \, \text{bar}
\]

\begin{itemize}
    \item Text: (bei $p = 1.4 \, \text{bar}$ (Tabelle 1))
\end{itemize}

\[
U_1 = U_\text{flüssig} + x_1 (U_\text{fest} - U_\text{flüssig})
\]

\[
U_1 = -200.0938
\]

\[
U_2 = U_\text{flüssig} + x_2 (U_\text{fest} - U_\text{flüssig})
\]

\[
U_\text{flüssig} = -333.458
\]

\[
U_\text{fest} = -0.045
\]

\begin{itemize}
    \item Text: zu (d)
    \item Text: zu Aufgabe (3d)
\end{itemize}

``````latex


\section*{zu 07!}

\subsection*{a)}
\begin{align*}
    &\Delta m (u_2 - u_1) = Q \\
    &\Delta m (T_2 - T_1) = Q \\
    &\Delta m c_v (T_2 - T_1) = Q \\
    &Q_v = -m \Delta D
\end{align*}

\subsection*{b)}
\begin{align*}
    &\Delta U = \sum Q \\
    &m_{ges} (u_2 - u_1) = Q_{v2} \\
    &u_2 m_{neu} - u_1 m_{neu} = -Q_{v2} \\
    &u_2 = -\frac{Q_{v2}}{m_{neu}} + u_1 \\
    &u_2 = -18,9238
\end{align*}

\begin{align*}
    &T_2 = 0,005 \degree C \\
    &c_p = 0,633 \\
    &c_v = \frac{c_p - R}{R} \\
    &c_v = 0,46672
\end{align*}

\begin{align*}
    &u_1 = -200,0928 \quad \text{(siehe letzte Seite)} \\
    &u_{flüssig} = 0,045 \\
    &u_{fest} = -333,458 \\
    &u_2 = u_{flüss} + x_2 (u_{fest} - u_{flüssig}) \\
    &x_2 = \frac{u_2 - u_{flüssig}}{u_{fest} - u_{flüssig}} \\
    &x_2 = 0,05662 \\
    &x_2 = 5,662 \%
\end{align*}

``````latex


