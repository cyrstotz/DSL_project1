
``````latex


\section*{Problem 4}

\subsection*{a)}

\begin{description}
    \item[Graph Description:] The graph is a Pressure-Temperature (P-T) diagram. The x-axis is labeled $T$ (Temperature) and the y-axis is labeled $P$ (Pressure). There is a curve starting from the origin and moving upwards to the right, labeled "Fest" (solid). Another curve starts from the same origin and moves upwards to the right, labeled "Flüssig" (liquid). These two curves meet at a point labeled "Tripelpunkt" (triple point). Below the "Fest" curve, the region is labeled "gas". There is also a horizontal line extending from the "Tripelpunkt" to the right, labeled "isotherm".
\end{description}

\subsection*{b)}

\[
\begin{array}{|c|c|c|c|c|}
\hline
 & 1 & 2 & 3 & 4 \\
\hline
T & & -6^\circ C & & 37.33^\circ C \\
\hline
p & & 8 \, \text{bar} & 8 \, \text{bar} & \\
\hline
h & & & & \\
\hline
\end{array}
\]

\[
T_1 = -6^\circ C
\]

\[
\text{adiabat reversibel = isentrop}
\]

\[
s_2 = s_3 = 0.9273 + \left( \frac{-6 + 6}{8.4} \right) 0.9235 = 0.9273 - 0.9226 \, \frac{\text{kJ}}{\text{kgK}}
\]

\[
\dot{m} (h_2 - h_3) - \dot{W}_x = 0
\]

\[
\dot{m} = \frac{\dot{W}_x}{(h_2 - h_3)} = 0.00170 \, \frac{\text{kJ}}{\text{s}} = 3.973 \, \frac{\text{kg}}{\text{s}}
\]

\[
h_2 = 264.7 + \left( \frac{-6 + 6}{8.4} \right) (274.54 - 264.54) = 273.72 \, \frac{\text{kJ}}{\text{kg}}
\]

\[
h_3 = 0.0066 + \left( \frac{0.932 - 0.0066}{273.66 - 264.75} \right) (273.66 - 264.75)
\]

\[
h_3 = 0.9334 - 0.0066
\]

\[
= 65.050 \, \frac{\text{kJ}}{\text{kg}}
\]

``````latex


\section*{Student Solution}

\subsection*{a)}
\begin{align*}
x_1 \\
p_1 &= p_2 \\
p_2 &= 1.212152 \, \text{bar} = p_1 \\
p_1 &= -22 \, \text{bar} \\
p_2 &= -7.272 \, \text{bar} \\
\text{Drossel isenthalp} \\
s_2 - s_1 &= 0.345 \, \frac{\text{kJ}}{\text{kgK}} \\
\alpha &= \frac{s_1 - s_f}{s_g - s_f} = 0.303 \\
s_f &= 0.0807 \\
s_g &= 0.9357
\end{align*}

\subsection*{d)}
\begin{align*}
\epsilon_c &= \frac{\dot{Q}_{zu}}{W_t} = 5.8373 \\
\dot{V}_t &= -28 \, \text{kW} \\
\dot{Q}_{zu} &= \dot{m} (h_2 - h_1) = 762.37 \, \text{W} = 0.2 \, \text{m} \\
\dot{m} &= 0.00170 \, \frac{\text{kg}}{\text{s}} \\
h_2 &= h_f + x(h_g - h_f) = 86.703 \, \frac{\text{kJ}}{\text{kg}} \\
h_f &= 21.79 \, \frac{\text{kJ}}{\text{kg}} \\
h_g &= 234.08 \, \frac{\text{kJ}}{\text{kg}} \\
h_2 &= 234.08 \, \frac{\text{kJ}}{\text{kg}}
\end{align*}

\subsection*{e)}
Die Temperatur würde weiter sinken bis 0 Kelvin erreicht wird.

```