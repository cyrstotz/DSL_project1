
``````latex


\section*{4 a)}

\begin{tabular}{|c|c|c|c|c|c|c|}
\hline
P & T & V & W & \omega & \Omega & \dot{Q} \\
\hline
1 & $<$ 8 bar & & & & & $\dot{Q}$ \\
\hline
$P_1 = P_2$ & $T_i = T_{-6K}$ & & $\dot{W}$ & & & \\
\hline
8 bar & & & & & & \\
\hline
8 bar & & & & & & \\
\hline
\end{tabular}

\begin{itemize}
    \item Gasförmig, gesättigt
    \item flüssig
\end{itemize}

$p_1 = p_2 \quad S_2 = S_3$

\subsection*{Graph Descriptions}

\textbf{First Graph:}

The first graph is a pressure-temperature ($p$-$T$) diagram. The x-axis is labeled $T$ (kelvin) and the y-axis is labeled $p$ (bar). There is a saturation curve that starts at the origin, rises to a peak, and then falls back down. Two points are marked on the curve, labeled 1 and 2, connected by a horizontal line. The point 1 is on the left side of the peak, and point 2 is on the right side of the peak. The region to the left of the curve is labeled "sättigungslinie".

\textbf{Second Graph:}

The second graph is also a pressure-temperature ($p$-$T$) diagram. The x-axis is labeled $T$ (kelvin) and the y-axis is labeled $p$ (bar). There is a saturation curve similar to the first graph. Four points are marked on the curve, labeled 1, 2, 3, and 4. Points 1 and 2 are connected by a horizontal line labeled "1 bar". Points 3 and 4 are on the left side of the curve, with point 3 below point 4.

\textbf{Third Graph:}

The third graph is a temperature-entropy ($T$-$s$) diagram. The x-axis is labeled $T$ (K) and the y-axis is labeled $s$ (bar). There is a dome-shaped curve. Four points are marked on the curve, labeled 1, 2, 3, and 4. Points 1 and 2 are connected by a horizontal line labeled "isobar". Points 3 and 4 are connected by a vertical line labeled "isentrope". The region under the dome is labeled "isobar" and "isentrope".

``````latex


\section*{b)}

\(\dot{m}_{R134a}\)

Verdichter

1HS.

\[
0 = \dot{m}(h_e - h_a) + \dot{Q}^0 - \dot{W}_k
\]

\[
s_2 = s_3
\]

\[
T_2 = T_1 - 6K
\]

\[
T_1 = 273{,}15K + 10K = 283{,}15K
\]

\[
T_2 = 277{,}15K
\]

\[
\dot{W}_k = \dot{m}_{R134a} (h_2 - h_3)
\]

\[
h_2 (T = 277{,}15K) \quad \text{-- gasförmig gesättigt}
\]

\[
\text{TAB A-10 4°C} \quad h_{2g} (4°C) = 249{,}53 \frac{kJ}{kg} \quad s_{2g} = 0{,}9169 \frac{kJ}{kg \cdot K}
\]

\[
h_3 (8 bar, s_2)
\]

Nassdampf:

\[
s_3 = s_f (8 bar) + x (s_g (8 bar) - s_f (8 bar))
\]

\[
\text{A-11} \quad s_f = 0{,}3459 \quad s_g = 0{,}9006
\]

\[
x = 0{,}29999
\]

\[
\text{TAB} \quad h_{3x} (8 bar) = h_f (8 bar) + x (h_g (8 bar) - h_f (8 bar))
\]

\[
\text{A-11} \quad h_f = 93{,}42 \quad h_g = 264{,}15
\]

\[
h_{3x} (8 bar) = 144{,}637 \frac{kJ}{kg}
\]

\[
\dot{W}_k = \dot{m}_{R134a} \frac{28 \cdot 10^{-3} \frac{kJ}{s}}{h_2 - h_3 \left( \frac{kJ}{kg} \right)} = 4{,}9007 \cdot 10^{-4} \frac{kg}{s} = \dot{m}_{R134a}
\]

\[
\dot{W}_k = 28W = 28 \frac{J}{s}
\]

\[
2{,}669 \cdot 10^{-4} \frac{kg}{s} = \dot{m}_{R134a}
\]

``````latex

\section*{4}

\subsection*{c)}
\[ x_1 \]

\[
0 = \dot{m} (h_4 - h_1) \quad \text{-- da } \dot{W} = 0 \quad \dot{Q} = 0
\]

\[
h_4 = h_1
\]

\[
h_4 = h_f (8 \text{bar}) \quad \text{TAB A-11}
\]

\[
h_4 = 93,42 \frac{\text{kJ}}{\text{kg}} = h_1
\]

\[
h_1 = h_f + x (h_g - h_f)
\]

\[
p_1 - p_2 \quad \text{TAB A-10}
\]

\[
p_2 (4^\circ \text{C}) = 3,3765 \text{bar} \quad \text{TAB A-11}
\]

\[
h_f (3,3765 \text{bar}) = \frac{h_f (3,6 \text{bar}) - h_f (3,2 \text{bar})}{3,6 - 3,2} \cdot (3,3765 - 3,2) + h_f (3,2 \text{bar})
\]

\[
h_f = 55,3 \frac{\text{kJ}}{\text{kg}}
\]

\[
h_g (3,3765 \text{bar}) = \frac{h_g (3,6 \text{bar}) - h_g (3,2 \text{bar})}{3,6 - 3,2} \cdot (3,3765 - 3,2) + h_g (3,2 \text{bar})
\]

\[
h_g = 249,5072
\]

\[
x = \frac{h_1 - h_f}{h_g - h_f} = 0,1963 = x_1
\]

\subsection*{d)}
\[
\epsilon_k = \frac{\dot{Q}_{\text{zu}}}{\dot{Q}_{\text{ab}} - \dot{Q}_{\text{zu}}} = \frac{\dot{Q}_k}{\dot{W}_k} = 2,0836 = \epsilon_k
\]

\[
\dot{Q}_k = \dot{m} (h_2 - h_1)
\]

\[
= \frac{1,9667 \cdot 10^{-4} \frac{\text{kg}}{\text{s}}}{2,6068 \cdot 10^{-4} \frac{\text{kg}}{\text{s}}} \cdot (249,53 \frac{\text{kJ}}{\text{kg}} - 93,42 \frac{\text{kJ}}{\text{kg}}) = 0,029765 \frac{\text{kJ}}{\text{s}} = \dot{Q}_k
\]

\[
\dot{Q}_k = 41,6718 \frac{\text{J}}{\text{s}}
\]

\[
= 0,0416718 \frac{\text{kJ}}{\text{s}} = \dot{Q}_k
\]

```