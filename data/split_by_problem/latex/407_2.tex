
``````latex


\section*{Aufgabe 2}

\begin{tabular}{|c|c|c|c|c|c|c|}
\hline
 & V & T & W & S & Q \\
\hline
0 & 0.191 \, \text{bar} & & -30 & & \\
\hline
1 & 0.191 \, \text{bar} & & & & \\
\hline
2 & 0.5 \, \text{bar} & & & & \\
\hline
3 & 0.5 \, \text{bar} & & & & \\
\hline
4 & 0.5 \, \text{bar} & & & & \\
\hline
5 & 0.5 \, \text{bar} & & 4.32 \, \% & & \\
\hline
6 & 0.191 \, \text{bar} & & & & \\
\hline
\end{tabular}

\begin{itemize}
    \item $p_{1} = 0$
    \item $p_{2} = 0$
    \item $p_{3} = 1.195 \, \frac{\text{kJ}}{\text{kg}}$
    \item $p_{4} = 0$
    \item $p_{5} = 0$
\end{itemize}

$k_{s} = \frac{v_{2}}{v_{1}}$

\section*{12 R/Min rev $\rightarrow$ isentrop}

\begin{description}
    \item[Graph Description:] The graph is a plot with the y-axis labeled $T \, [K]$ and the x-axis labeled $s \, \left[\frac{\text{kJ}}{\text{kg K}}\right]$. There are three curves labeled 1, 2, and 3, each representing different states. The curves are smooth and upward sloping. The points on the curves are labeled as follows:
    \begin{itemize}
        \item Point 1 is at the bottom left of the graph.
        \item Point 2 is directly above point 1 on the same curve.
        \item Point 3 is on a higher curve to the right of point 2.
        \item Point 4 is on the highest curve to the right of point 3.
        \item Point 5 is on the middle curve to the right of point 2.
    \end{itemize}
    The curves are labeled with pressures:
    \begin{itemize}
        \item The curve passing through points 1 and 2 is labeled $0.191 \, \text{bar}$.
        \item The curve passing through points 3 and 5 is labeled $0.5 \, \text{bar}$.
    \end{itemize}
\end{description}

$p = 1$ \& nach 7 Minutenstrom abgetrennt

``````latex


b) \quad \dot{m}_s = \dot{m}_b \quad \text{stationär reversibel} \\
\Rightarrow s_s = s_b \Rightarrow \text{können polytropen Kompression} \\
\Rightarrow n = k = 1.4 \\

\frac{T_0}{T_s} = \left( \frac{p_0}{p_s} \right)^{\frac{n-1}{n}} \quad T_b = T_s \left( \frac{p_b}{p_s} \right)^{\frac{n-1}{n}} = 328.1 \, K \\

\text{für} \quad W_b \quad \text{und} \quad h_s \quad \text{ist} \quad s_b \\

0 = \dot{m} h_{es} \left[ h_s - h_{s_i} + \frac{v_s^2 - v_b^2}{2} \right] + \dot{Q} \quad \text{adiab} \quad 0-\text{weise} \\

h_s - h_{b} = \int_{T_b}^{T_s} c_p \, dT = c_p (T_s - T_b) \\

0 = h_s - h_b + \frac{v_s^2 - v_b^2}{2} = c_p (T_s - T_b) + \frac{v_s^2 - v_b^2}{2} \\

c_p (T_b - T_s) 2 = v_s^2 - v_b^2 \quad \Rightarrow \quad v_b = 390.93 \, \frac{m}{s} \\

\text{(d) C)} \quad e_{s + str} = [h_b - h_0 - \dot{p} (s - s_0)] \quad \text{KE} \\

e_{s + str_2} - e_{s + str_2} = h_b - h_0 - \dot{p} (s - s_0) \quad \text{KE} \\

\left( h_b - h_0 = \int_{T_0}^{T_b} c_p \, dT = c_v (T_b - T_0) = 41.08 \, \frac{kJ}{kg} \right) \quad \text{es geht} \\

n = k = \frac{c_p}{c_v} \quad \leftarrow c_v = \frac{c_p}{k} = 0.419 \, \frac{kJ}{kg \cdot K} \quad p_1 = p_0 \quad \text{ln(1)} = 0 \\

s_b - s_0 = \int_{T_0}^{T_b} \frac{c_p}{T} \, dT - R \ln \left( \frac{p_b}{p_0} \right) = c_p \ln \left( \frac{T_b}{T_0} \right) - R \ln \left( \frac{p_b}{p_0} \right) \\

\left( v_b - v_0 = \int_{T_0}^{T_b} \frac{c_p}{T} \, dT = c_v (T_b - T_0) = 41.08 \, \frac{kJ}{kg} \right) \quad \text{es geht} \\

v_b - v_0 \quad pv = RT \quad v = \frac{RT}{p} \quad R = \frac{R}{M} \quad c_p - c_v = 0.28 \, \frac{kJ}{kg \cdot K} \quad - 28 \, \frac{J}{kg \cdot K} \\

v_b - \frac{R T_b}{p_b} = 7.93 \, \frac{m^2}{kg}
``````latex


\section*{Handwritten Student Solution}

\subsection*{Aufgabe 2 c) weiter}

\[
v_0 = \frac{v_0}{\rho_0} = 3.65 \, \frac{m^3}{kg}
\]

\[
\frac{v_0 - v_2}{2} = 21.78 \, \frac{kJ}{kg}
\]

\[
\Delta e_{x1str} = 159.158 \, \frac{kJ}{kg}
\]

\subsection*{d) stat. Fließprozesse}

\[
Q = \dot{m} \cdot \left( \dot{m} \cdot \left( e_{x1str} + (1 - \frac{T_0}{T}) \right) - \dot{V} T - \dot{E}_{x1str} \right)
\]

\[
E_{x1str} = e_{x1str} + \dot{V} T
\]

\[
W_r = \int_{v_0}^{v_2} \rho \, dp + \Delta ke + \Delta e
\]

\[
W_e = \int_{0}^{n} \rho \, dv - \frac{v_2^2 - v_0^2}{2} = \rho_0 \, n \, (v_0 - v_2) - \frac{v_2^2 - v_0^2}{2}
\]

Für \(v_1\) und \(v_0\) siehe Aufgabe c) da dort ausgerechnet

\[
W_r =
\]

\subsection*{Aufgabe 2c von oben weiter}

\[
h_0 - h_0 = c_p \, (T_0 - T_0) = 85.96 \, \frac{kJ}{kg}
\]

\[
\Delta e_{x1str} = 145.17 \, \frac{kJ}{kg}
\]

``````latex


