
``````latex


3)

\begin{align*}
p_{3\alpha} &= p_{\text{amb}} + p_{\text{kolben}} + p_{\text{EW}} \\
&= 1 \text{bar} + \frac{(32 \text{kg} + 0,1 \text{kg}) \cdot 9,81 \frac{\text{m}}{\text{s}^2}}{(0,05 \text{m})^2 \cdot \pi} \\
&= 1 \text{bar} + \frac{(32 \text{kg} + 0,1 \text{kg}) \cdot 9,81 \frac{\text{m}}{\text{s}^2}}{(0,05 \text{m})^2 \cdot \pi} \\
&= 1,4 \text{bar}
\end{align*}

\begin{align*}
V_{3\alpha} &= 3,14 \text{L} = 0,00314 \text{m}^3 \\
R &= \frac{\bar{R}}{M} = \frac{8,314 \frac{\text{J}}{\text{mol K}}}{50 \frac{\text{kg}}{\text{mol}}} = 0,16628 \frac{\text{J}}{\text{kg K}} \\
T_{3\alpha} &= 500^\circ \text{C} = 773,15 \text{K}
\end{align*}

\[
\frac{p_{3\alpha} \cdot V_{3\alpha}}{R \cdot T_{3\alpha}} = m_{3\alpha} = \underline{3,42 \text{g}}
\]

b) Die Temperatur $T_{\text{gz}}$ ist $0^\circ \text{C}$, da das Ei in Wasser eine Temperatur von $0^\circ \text{C}$ hat und sich Gas \& EW in Thermodynamik GG befinden. Da das Gewicht von oben (Kolben, Ei, Atmosphärendruck) immer noch gleich ist $\rightarrow$ ist der Druck $p_{2\alpha} = p_{\text{gg}} = 1,4 \text{bar}$.

c) 
\[
\Delta E = E_2 - E_1 = m_{3\alpha} \cdot u_2 - m_{3\alpha} \cdot u_1 = Q - W_v
\]

\begin{align*}
T_{\text{gz}} &= 0^\circ \text{C} \\
T_{3\alpha} &= 500^\circ \text{C} \\
u_2 - u_1 &= c_v^{\text{pg}} (T_2 - T_1) = 316,5 \frac{\text{J}}{\text{kg}}
\end{align*}

\[
\frac{W_v}{m} = \frac{R(T_2 - T_1)}{1 - n}
\]

\begin{align*}
\frac{W_v}{m} &= \frac{0,16628 \cdot (500)}{1 - 1,263} = -3,16,5 \frac{\text{J}}{\text{kg}} \\
\text{reibungsfrei} &\rightarrow \text{isentrop} \\
n &= \frac{R + c_v}{c_v} = 1,263
\end{align*}

\[
\Rightarrow \text{ergibt 3,16,5}
\]

``````latex


3a)
\[
\Delta E = (u_2 - u_1) m = \frac{Q_{12}}{m} \quad \text{Volumen ändert nicht}
\]
\[
Q_{12} = 1,5 \, \text{kJ}
\]
\[
u_1 = u_f(0^\circ \text{C}) + 0,6 \left( u_{fe}(0^\circ \text{C}) - u_f(0^\circ \text{C}) \right)
\]
\[
= -0,005 + 0,6 \left( -333,6 \frac{\text{kJ}}{\text{kg}} + 0,005 \right)
\]
\[
= -200,093 \frac{\text{kJ}}{\text{kg}}
\]
\[
u_2 = \frac{Q_{12}}{m} + u_1 \quad = \quad \frac{1,5 \, \text{kJ}}{0,1 \, \text{kg}} + (-200,093 \frac{\text{kJ}}{\text{kg}}) = -185,093 \frac{\text{kJ}}{\text{kg}}
\]
\[
T_{g2} = 0,003^\circ \text{C}
\]
\[
u_2 = -185,093 = u_f(0,003^\circ \text{C}) + x_2 \left( u_{fe}(0,003^\circ \text{C}) - u_f(0,003^\circ \text{C}) \right)
\]
\[
= -0,0033 + x_2 \left( -333,4 \frac{\text{kJ}}{\text{kg}} + 0,0033 \right)
\]
\[
\Rightarrow x_2 = 0,555 \Lambda = 0,555 \quad \underline{\underline{0,555}}
\]

``````latex


