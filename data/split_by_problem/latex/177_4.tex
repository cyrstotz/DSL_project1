
``````latex


\section*{4 (a)}

\begin{itemize}
    \item There is a graph with the vertical axis labeled \( P \) and the horizontal axis labeled \( T \).
    \item The graph shows a curve that starts at the bottom left, rises to a peak labeled \( T_c \), and then falls back down towards the bottom right.
    \item The point \( T_i \) is marked on the horizontal axis with a value of \( -20^\circ C \).
    \item The graph is a smooth curve, resembling a bell shape.
\end{itemize}

\[
T_i = -20^\circ C
\]

\[
\lambda_{\text{bar}} = 0.004 \, \text{bar}
\]

\[
\lambda_{\text{mm}} = 0.001
\]

``````latex


4.

b) $\dot{m}_{R134a}$

$T_1 = 23^\circ C \quad T_2 = -22^\circ C$

1. HS über Verdichter: $2 \rightarrow 3$

\[
0 = \dot{m}_{R134a} (h_2 - h_3) + \dot{W}_k
\]

\[
h_2 = h_g (-22^\circ C) = 237.08 \frac{kJ}{kg}
\]

\[
h_3 \rightarrow s_2 = s_3 \rightarrow s_2 = s_g (-22^\circ C) = 0.9351 \frac{kJ}{kgK}
\]

\[
s_3 = 0.9351 \frac{kJ}{kgK}
\]

\[
h_3 (p_{bar}, 0.9351 \frac{kJ}{kgK}) = 0.9351 \frac{kJ}{kgK} + (0.9351 - 0.9066) \frac{kJ}{kgK} \cdot \frac{264.18 - 0.9066}{2.3} = 264.18 \frac{kJ}{kg}
\]

\[
0 = \dot{m}_{R134a} (h_2 - h_3) + \dot{W}_k
\]

\[
\dot{m}_{R134a} = \frac{-\dot{W}_k}{(h_2 - h_3)} = \frac{-0.028 kW}{237.08 \frac{kJ}{kg} - 264.18 \frac{kJ}{kg}} = 3.349 \frac{kg}{h}
\]

c) $x_4 = 0$

$p_4 = 8 bar$

Drossel isenthalp da adiabate + keine Arbeit $h_4 = h_1$

\[
h_4 = h_f (8 bar) = 93.42 \frac{kJ}{kg}
\]

\[
h_1 = h_f + x (h_g - h_f) \quad at \quad 1 mbar \quad T_{bar} \quad p_{1} = 1.5 bar (5 bar)
\]

``````latex


\section*{Student Solution}

\subsection*{c)}
\begin{align*}
    x_u &= 0 \\
    p_1 &= p_{\text{bar}} \\
    &\text{Drossel, Isenthalp} \\
    &\text{da Arbeit = 0 und adiabate} \\
    h_u &= h_1 \\
    h_u &= h_f(p_{\text{bar}}) = 93.42 \frac{\text{kJ}}{\text{kg}} \\
    h_1 &= h_f + x(h_g - h_f) \quad \text{at } T_1 \\
    x &= \frac{h_1 - h_f}{h_g - h_f} = \frac{93.42 \frac{\text{kJ}}{\text{kg}} - 27.72 \frac{\text{kJ}}{\text{kg}}}{234.08 \frac{\text{kJ}}{\text{kg}} - 27.72 \frac{\text{kJ}}{\text{kg}}} \\
    x &= 0.337
\end{align*}

\subsection*{d)}
\begin{align*}
    \epsilon_K &= \frac{(\dot{Q}_{\text{zu}})}{\dot{w} + l} = \frac{(\dot{Q}_{\text{kalt}})}{2 \dot{w}} = \\
    \dot{Q}_{\text{zu}} &= \dot{m} (h_1 - h_2) + \dot{Q}_K \\
    h_1 &= 93.42 \frac{\text{kJ}}{\text{kg}} \\
    h_2 &= h_f (-2^\circ C)
\end{align*}

\subsection*{e)}
Temperatur würde konstant bleiben bei 0K.

```