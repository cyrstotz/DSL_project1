
``````latex


\section*{a)}

\subsection*{i)}

\begin{description}
    \item[Graph 1:] The graph is a plot with the vertical axis labeled \( p \) and the horizontal axis labeled \( T \). The graph shows a curve that starts at the bottom left, oscillates up and down several times, and then curves downwards to the right.
    \item[Graph 2:] The graph is a plot with the vertical axis labeled \( p \) and the horizontal axis labeled \( T \). There are two lines: one labeled "Gas, fest" which is a straight line starting from the origin and going upwards to the right, and another line labeled "flüssig" which starts from a point on the first line and goes downwards to the right. There is a point labeled \( T \) on the first line, and a region labeled "flüssig" below the second line.
\end{description}

\[
T_i = \text{PR: } \Lambda \text{bar}
\]

\[
T_i = -10^\circ C
\]

\section*{b)}

\[
\text{EB 1} \rightarrow 2
\]

\[
Q = \dot{m} \left[ h_1 - h_2 \right] + \dot{Q}_k
\]

\[
h_2 = h_f(-22) = 234.8 \frac{\text{kJ}}{\text{kg}}
\]

\section*{c)}

\[
h_1 = h_4
\]

\[
h_4 = h_f(8 \text{bar}) = 93.42 \frac{\text{kJ}}{\text{kg}}
\]

\[
x = \frac{h_3 - h_f}{h_g - h_f}
\]

``````latex


a)

\[
\epsilon_k = \frac{\dot{W}_{\text{nutz}}}{\dot{Q}_{\text{zu}}}
\]

e) und \sout{Wärmer} du die Lebensmittel alle Wärme beitragen

```