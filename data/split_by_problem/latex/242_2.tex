
``````latex


\section*{Student Solution}

\subsection*{Graphical Content Description}

\begin{itemize}
    \item The first graph is a plot with the vertical axis labeled \( T \left( \frac{K}{s} \right) \) and the horizontal axis labeled \( s \left[ \frac{2}{u} \right] \). There are two curves drawn on the graph. The first curve starts at the origin (0,0) and rises steeply to a point labeled "1" on the vertical axis. From this point, a vertical line extends upwards to a point labeled "2". The second curve starts from the origin and rises more gradually, intersecting the first curve at point "1" and continuing upwards. There is a horizontal line extending from the origin to the right, labeled \( p_0 = 0.1 \rho \sigma \).
    
    \item The second graph is similar to the first, with the vertical axis labeled \( T \left( \frac{K}{s} \right) \) and the horizontal axis labeled \( s \left[ \frac{2}{u} \right] \). This graph contains more points and lines. The origin is labeled "0". From the origin, a line rises steeply to a point labeled "2" on the vertical axis. From point "2", a horizontal line extends to the right to a point labeled "4". From point "4", a vertical line extends upwards to a point labeled "3". There is a diagonal line connecting points "2" and "3". Another diagonal line connects points "4" and "6", with "6" being on the horizontal axis. There is a label \( 0.5 \sigma \) next to the line connecting points "2" and "3". The text "clear ideal gas" is written next to the graph, with arrows pointing upwards and downwards labeled \( \uparrow \) and \( \downarrow \) respectively. There is also a label \( p_0 \) next to the downward arrow.
\end{itemize}

``````latex


\section*{Student Solution}

\subsection*{Graph 1}

The first graph is a plot with the vertical axis labeled \( T(s) \) and the horizontal axis labeled \( SLEJ \). The graph contains six points labeled 0, 1, 2, 3, 4, and 5. The points are connected by lines as follows:

- Point 0 is at the origin.
- Point 1 is above and to the right of point 0.
- Point 2 is above and to the right of point 1.
- Point 3 is above and to the right of point 2.
- Point 4 is to the right of point 3.
- Point 5 is below and to the right of point 4.

The lines connecting the points form a closed loop. There are also several curved lines in the background, which appear to be contour lines or isotherms.

\subsection*{Graph 2}

The second graph is a plot with the vertical axis unlabeled and the horizontal axis unlabeled. The graph contains six points labeled 0, 1, 2, 3, 4, and 5. The points are connected by lines as follows:

- Point 0 is at the origin.
- Point 1 is above and to the right of point 0.
- Point 2 is above and to the right of point 1.
- Point 3 is above and to the right of point 2.
- Point 4 is to the right of point 3.
- Point 5 is below and to the right of point 4.

The lines connecting the points form a closed loop. There are also several curved lines in the background, which appear to be contour lines or isotherms.

``````latex


\section*{Problem 2}

\subsection*{b)}

\[
p_1 = 0.1 \, \text{bar}
\]

\[
\dot{m} = \frac{p_1 v_1}{R T_1}
\]

\[
Q = \dot{m} \left[ h_5 + \frac{1}{2} w_5^2 - h_6 - \frac{1}{2} w_6^2 \right] = w_t
\]

\[
\Rightarrow \text{Schubdüse:}
\]

\[
\frac{T_6}{T_5} = \left( \frac{p_6}{p_5} \right)^{\frac{\kappa - 1}{\kappa}}
\]

\[
\kappa = 1.4
\]

\[
\Rightarrow T_6 = T_5 \left( \frac{p_6}{p_5} \right)^{\frac{\kappa - 1}{\kappa}} = 328.14
\]

\[
\Rightarrow \text{Energiegleichung gesamte Turbine:}
\]

\[
Q = \dot{m} \left( h_0 + \frac{1}{2} w_0^2 - h_6 - \frac{1}{2} w_6^2 \right) + \dot{Q} \beta \left( 1 + \frac{1}{2} w_6^2 \right)
\]

\[
\frac{1}{2} w_0^2 = c_p (T_0 - T_6) + \frac{1}{2} w_0^2 + a_3 (0.2033)
\]

\[
\Rightarrow w_0 = \sqrt{2 \cdot \left( c_p (T_0 - T_6) + \frac{1}{2} w_0^2 + a_3 (0.2033) \right)}
\]

\[
= 438.866 \, \frac{m}{s}
\]

\subsection*{c)}

\[
e_{\text{ex,er,0}} - e_{\text{ex,ist,0}} = (h_0 - h_0 - T_0 (s_0 - s_0) + w_6 - h_0)
\]

\[
= \left( c_p (T_0 - T_0) - T_0 \left( c_p \ln \left( \frac{T_0}{T_0} \right) - R \left( \frac{p_0}{p_0} \right) \right) + \frac{1}{2} w_0^2 - \frac{1}{2} w_0^2 \right)
\]

\[
= 116.6 \, \frac{kJ}{kg}
\]

\subsection*{d)}

\[
Q = E_{\text{ex,ist,0-6}} + E_{\text{ex,Q}} - E_{\text{ex,vel}}
\]

\[
E_{\text{ex,vel}} = E_{\text{ex,ist,0-6}} + E_{\text{ex,Q}}
\]

\[
E_{\text{ex,Q}} = \left( 1 - \frac{T_0}{T} \right) \left| \dot{Q} \right|
\]

\[
\dot{Q} = 11.85 \, \frac{kJ}{kg}
\]

\[
\dot{m}_{\text{in}} + \dot{m}_{\text{ex}} = 6.733 \, \dot{m}
\]

\[
\dot{m}_{\text{in}} = 5.203
\]

\[
\dot{m}_{\text{in}} = 5.203 \, \dot{m}
\]

\[
\dot{m}_{\text{in}} = \dot{m} \left( \frac{m_{\text{in}}}{m_{\text{ex}}} \right)
\]

\[
\dot{m}_{\text{in}} = \dot{m} \left( \frac{m_{\text{in}}}{m_{\text{ex}}} \right) = \dot{m} \left( \frac{1```latex

\section*{Problem 2(d)}

\[
\vec{F}_{\text{ges}} = \frac{d}{dt} (m \vec{v}) = m \vec{a} + \dot{m} \vec{v} \quad \text{(1)}
\]

\[
\vec{a}_{\text{exp}} = \vec{a}_{\text{exp,stör-f}} + \frac{\dot{m}}{m} \vec{v}_{\text{exp}}
\]

\[
= -11.6 \left( \frac{6}{s^2} \right) + \frac{110.5 \frac{kg}{s}}{6 \frac{kg}{s}} = 73.3 \left( \frac{kg}{s} \right) \quad \text{(circled)}
\]

\[
\rightarrow \vec{a}_{\text{exp,neu}}
\]

``````latex


