
``````latex


\section*{Thermodynamik 1}

\subsection*{Aufgabe 4}

\subsubsection*{a)}

\begin{description}
    \item[Graph Description:] The graph is a plot with the y-axis labeled as \( P \, [Pa] \) and the x-axis labeled as \( T \, [K] \). There are two main curves:
    \begin{itemize}
        \item A blue curve representing the saturation dome, which starts at the origin, rises to a peak, and then falls back down.
        \item Two straight lines intersecting at the center of the graph, one with a positive slope and the other with a negative slope.
    \end{itemize}
    The region under the blue curve is labeled as "Nassdampfgebiet".
\end{description}

\subsubsection*{b)}

Stellen wir die Energiebilanz auf

\[
Q = m_{\text{ges}} \left[ h_2 - h_3 \right] + W_{\text{k}}
\]

\[
W_{\text{k}} = m_{\text{ges}} \left[ h_2 - h_3 \right]
\]

\[
-\frac{W_{\text{k}}}{h_2 - h_3} = m_{\text{ges}}
\]

\[
h_2 = h_g (277.15 K) = h_g (4^\circ C) = 249.33 \, \frac{\text{kJ}}{\text{kg}} \quad \text{(von TAB A.10)}
\]

\[
h_3 - h_1 (8 \, \text{bar}) = 269.15 \, \frac{\text{kJ}}{\text{kg}}
\]

\[
m_{\text{ges}} = \frac{0.28 \, \text{kW}}{0.8165 \, \frac{\text{kJ}}{\text{kg}}}
\]

\[
s_2 - s_3 (4^\circ C) = 0.8165 \, \frac{\text{kJ}}{\text{kg}}
\]

\[
s_3 = s_2 = 0.8165 \, \frac{\text{kJ}}{\text{kg}}
\]

\[
h_3 (8 \, \text{bar}, 0.8165 \, \frac{\text{kJ}}{\text{kg}}) = 267.33 \, \frac{\text{kJ}}{\text{kg}}
\]

\[
h (8 \, \text{bar}, x) = 264.15 \, \frac{\text{kJ}}{\text{kg}}
\]

\[
h (8 \, \text{bar}, y) = 273.66 \, \frac{\text{kJ}}{\text{kg}}
\]

\[
y = \frac{(x - x_1)}{(x_2 - x_1)} (y_2 - y_1) + y_1 = 267.33 \, \frac{\text{kJ}}{\text{kg}}
\]

\[
\frac{W_{\text{k}}}{m_{\text{ges}}} = h_3 - h_2 = 0.0157 \, \frac{\text{kg}}{\text{s}}
\]

``````latex


\textbf{c)} \quad \cancel{\frac{dE}{dt}} = \sum \dot{m_i} \left[ h_i + \frac{v_i^2}{2} + pe_i \right] + \sum \dot{Q_j} - \sum \dot{W_n}

\[
Q_k = m \left[ h_2 - h_1 \right]
\]

``````latex


\section*{Thermodynamik 1}

\subsection*{Aufgabe 4}

\begin{itemize}
    \item[a)] 
    \begin{description}
        \item[Graph Description:] 
        The graph is a plot with the x-axis labeled \( T \, [K] \) and the y-axis labeled \( P \, [Pa] \). There are three lines in the graph:
        \begin{itemize}
            \item A blue curve starting from the origin, curving upwards and to the right, labeled "Tripel" at a point marked with a small circle and the number 2.
            \item A purple straight line starting from the origin and sloping downwards to the right.
            \item A red straight line starting from the origin and sloping upwards to the right.
        \end{itemize}
    \end{description}
    
    \item[Graph Description:] 
    The second graph is a plot with the x-axis labeled \( T \, [K] \) and the y-axis labeled \( P \, [Pa] \). There are several lines and points in the graph:
    \begin{itemize}
        \item A blue curve starting from the origin, curving upwards and to the right.
        \item A blue triangle with vertices labeled 1, 2, 3, and 4. The triangle is oriented such that the base is parallel to the x-axis.
        \item The base of the triangle is labeled "isobar" and the side of the triangle is labeled "isochor".
        \item Arrows indicate the direction of the process along the edges of the triangle.
    \end{itemize}
\end{itemize}

```