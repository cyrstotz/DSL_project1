
``````latex


\section*{Aufgabe 4}

\subsection*{a)}

\begin{description}
    \item[Graph Description:] The graph is a plot with the vertical axis labeled as \( p(\text{bar}) \) and the horizontal axis labeled as \( T(\text{K}) \). There are three processes indicated on the graph:
    \begin{itemize}
        \item Process von (i) is a vertical line starting from point \( i \) and going upwards to point \( i2 \).
        \item Process von (ii) is a horizontal line starting from point \( i2 \) and going to the right to point \( i1 \). This line is labeled as "isobar".
        \item Process von (iii) is a vertical line starting from point \( i1 \) and going downwards to point \( i \).
    \end{itemize}
\end{description}

\subsection*{b)}

\begin{equation*}
    1. \, \text{HS}
\end{equation*}

\begin{equation*}
    \dot{m} (h_2 - h_3) = \omega
\end{equation*}

\begin{equation*}
    \dot{m} = \frac{\omega}{h_2 - h_3}
\end{equation*}

\begin{equation*}
    T_1 = -20^\circ \text{C} = T_2
\end{equation*}

\begin{equation*}
    h_{2f} = 29.26 \, \frac{\text{kJ}}{\text{kg}}
\end{equation*}

\begin{equation*}
    T_3 = T_2 \left( \frac{P_3}{P_2} \right)
\end{equation*}

\begin{equation*}
    \text{mit} \quad T_3 \quad \text{und} \quad P_3 = 4 \text{bar}
\end{equation*}

\begin{equation*}
    \text{in Tabelle} \quad h_3 = h_{2f}
\end{equation*}

``````latex


c) \\
mit \quad \dot{m}_2 = \frac{a \cdot c_s}{h} \quad \text{und} \quad T_2 = -22^\circ C \\
O = \dot{m}_e (h_{g,1} - h_1) \quad \Rightarrow \quad h_{g} = h_1 \\

x_1 = \frac{n_1 - n_x}{h_g - h_f} \quad \text{und} \quad h_1 = h_f \\

d) \\
\varepsilon_{ex} = \frac{\dot{Q}_{zu}}{\dot{Q}_{ab} + \dot{Q}_{zu}} = \frac{\dot{Q}_{zu}}{\dot{Q}_{ab} - \dot{Q}_{ab}} = \frac{\dot{Q}_{zu}}{\dot{Q} - \dot{W}} \\

\dot{Q}_{zu} = \dot{m} (h_2 - h_1) = \dot{m} (h_{g,2} - h_1) \quad \text{und} \quad \dot{W} = 28 \dot{W} \\

e) \\
Die Temp würde weiter abnehmen und alles würde vereisen

```