
``````latex


\section*{Aufgabe 2:}

\subsection*{a)}

\begin{itemize}
    \item The first graph is a plot with the x-axis labeled as $[ \frac{1}{V} \frac{kg}{kJ} ]$ and the y-axis unlabeled.
    \item There are several points labeled 0, 1, 2, 3, 4, 5, and 6.
    \item The points are connected by lines with arrows indicating the direction of the process.
    \item The process starts at point 0, moves to point 1, then to point 2, and so on until point 6.
    \item The line from point 0 to point 1 is labeled "isentrope".
    \item The line from point 2 to point 3 is labeled "isotherm".
    \item The line from point 4 to point 5 is labeled "isentrope".
    \item The line from point 5 to point 6 is labeled "isotherm".
\end{itemize}

\begin{itemize}
    \item The second graph is a plot with the x-axis labeled as $[ \frac{1}{V} \frac{kg}{kJ} ]$ and the y-axis labeled as $T [K]$.
    \item There are several points labeled 0, 1, 2, 3, 4, 5, and 6.
    \item The points are connected by lines with arrows indicating the direction of the process.
    \item The process starts at point 0, moves to point 1, then to point 2, and so on until point 6.
    \item The line from point 0 to point 1 is labeled "isentrope".
    \item The line from point 2 to point 3 is labeled "isobar".
    \item The line from point 4 to point 5 is labeled "isobar".
    \item The line from point 5 to point 6 is labeled "isotherm".
\end{itemize}

``````latex


b) $p_c = p_0 = 0.1 \text{ bar}$

$T_G = ? \quad ; \quad T_S = 431.9 \text{ K} \quad ; \quad p_S = 0.5 \text{ bar} \quad ; \quad \kappa = 1.4$

\text{isentrope Gleichung:}

\[
\frac{T_G}{T_S} = \left( \frac{p_G}{p_S} \right)^{\frac{\kappa - 1}{\kappa}} \quad \Rightarrow \quad T_G = T_S \left( \frac{p_G}{p_S} \right)^{\frac{\kappa - 1}{\kappa}} = 328.07 \text{ K}
\]

\text{Energie Bilanz}

\[
0 = \dot{m} (h_0 - h_6) + \cancel{\dot{Q}} + \cancel{\sum \dot{W}_t} - \cancel{\sum \dot{W}_t}
\]

\[
0 = \dot{m} (h_0 - h_6) + \frac{1}{2} (w_0^2 - w_6^2) + \cancel{\sum Q_j} - \cancel{\sum \dot{W}_t}
\]

\text{in Aufgabenstellung} \rightarrow \text{Annahmen}

\[
2(h_6 - h_0) = w_0^2 - w_6^2
\]

\[
w_6^2 = w_0^2 + 2(h_0 - h_6)
\]

\[
w_6^2 = w_0^2 + 2 c_{p,\text{Luft}} (T_0 - T_6)
\]

\[
w_6^2 = 200 \frac{\text{m}^2}{\text{s}^2} + 2 \cdot 1.006 \frac{\text{kJ}}{\text{kg K}} (293.15 - 328.07) \text{ K}
\]

\[
= 200 \frac{\text{m}^2}{\text{s}^2} + 2 \cdot 1.006 \frac{\text{kJ}}{\text{kg K}} \cdot \frac{1 \text{kg} \cdot \text{m}^2}{\text{s}^2 \cdot 1000} (293.15 - 328.07)
\]

\[
w_6 = \sqrt{450 \frac{\text{m}^2}{\text{s}^2}} = 459.19 \frac{\text{m}}{\text{s}}
\]

``````latex


c) $T_6 = 340 \, K \, ; \, w_6 = 510 \, \frac{m}{s} \, ; \, T_0 = 243.15 \, K$

\[
\Delta e_{x,st} = e_{x,st,6} - e_{x,st,0}
\]

\[
e_{x,st,6} = \dot{m} \left( h_6 - h_0 - T_0 (s_6 - s_0) + \frac{1}{2} w_6^2 \right)
\]

\[
= c_p^{ig,luft} (T_6 - T_0) - T_0 \left( c_p^{ig,luft} \ln \left( \frac{T_6}{T_0} \right) - R \ln \left( \frac{p_6}{p_0} \right) \right) + \frac{1}{2} w_6^2
\]

\[
\dot{m}_{luft} = 28.3 \, \frac{kg}{kmol} \, ; \, R = \frac{R}{M} = 0.2869 \, \frac{kJ}{kg \cdot K}
\]

\[
\Rightarrow e_{x,st,6} = \dot{m} \left( 130 \, \frac{kJ}{kg} \right)
\]

\[
e_{x,st,0} = \left( h_0 - h_0 - T_0 (s_0 - s_0) + \frac{1}{2} w_0^2 \right) = 40 \, \frac{kJ}{kg}
\]

\[
\Rightarrow \Delta e_{x,st} = 90 \, \frac{kJ}{kg}
\]

``````latex

d) Energiebilanz:

\[
0 = \dot{m}_{ges} \left( -\Delta e_{x,s,t} \right) + \sum_j \left( 1 - \frac{T_0}{T_j} \right) \dot{Q}_j - \sum_k \dot{W}_{h,k} - \dot{E}_{x,ver1}
\]

\[
\dot{E}_{x,ver1} = -\dot{m}_{ges} \Delta e_{x,s,t}
\]

\[
\dot{e}_{x,ver1} = -\Delta e_{x,s,t} = -100 \frac{kJ}{kg}
\]

``````latex


