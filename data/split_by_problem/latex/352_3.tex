
``````latex


3.a) 
\[
R = \frac{\bar{R}}{M} = \frac{83,14}{50} \approx 1,66 \, \frac{\text{J}}{\text{g} \cdot \text{K}}
\]

\[
p_{s4} = p_{s1} + \rho_{EV} + \rho_{amb}
\]

\[
= \frac{m_{s1}}{A} + \frac{m_{EV} \cdot g}{A} + \rho_{amb}
\]

\[
\text{Fläche} \, A = 0,008 \, \text{m}^2
\]

\[
\underline{1,40 \, \text{bar}} \, p_{s4} = 1,4 \, \text{bar}
\]

\[
m_{s4} = \frac{p_{s4} V_{s4}}{R T_{s4}} \approx 0,0034 \, \text{kg} = 3,4 \, \text{g}
\]

b) 
\[
T_{s2} = 0^\circ \text{C}
\]

Es ist noch nicht alles Eis geschmolzen, und dennoch sind die beiden Kammern im thermodynamischen Gleichgewicht. D.h. dass das EV eine Temperatur von $0^\circ \text{C}$ hat und das Gas darin auch.

\[
p_{s2} = p_{s4} = \underline{1,4 \, \text{bar}}
\]

Es drückt noch immer der Atmosphärendruck und die Gewichtskraft von Kolben und EV auf dieselbe Fläche. Das EV ist immer gleich dicht. Somit ist auch der Druck immer gleich.

``````latex


c) 1. HS. für geschlossenes System an einem Kolben:

\[
\frac{dE}{dt} = \sum \dot{Q} - \sum \dot{W}
\]

\[
\Delta U = \Delta Q_{12} - W_{12}
\]

\[
Q_{12} = \Delta U + W_{12}
\]

\[
\Delta U = (U_{S2} - U_{S1}) + (U_{EV1} - U_{EV4})
\]

\[
= m_S \int_{T1}^{T2} c_V \, dT
\]

\[
= m_S \cdot c_V (T_2 - T_1) = 0.0034 \cdot 633 \cdot (293.15 - 393.15)
\]

\[
= (-684.8 \, \text{J})
\]

\[
V_{S2} = \frac{m_S R T_{S2}}{\rho_{S2}} = \frac{0.0034 \cdot 461 \cdot 293.15}{40.094} \approx 0.001 \, \text{m}^3
\]

\[
V = \text{höhe} \times \text{Fläche} \Rightarrow \text{Höhe} = \frac{V}{A} \approx 0.14 \, \text{m}
\]

\[
W_{12} = \text{Kraft} \times \text{Weg} = (m_S + m_{EV}) \cdot g \cdot \text{höhe} = (32 + 0.1) \cdot 9.81 \cdot (0.14)
\]

\[
= (44.5 \, \text{J})
\]

\[
\Rightarrow Q_{12} = -684.8 + 44
\]

\[
= -640.8 \, \text{J}
\]

``````latex


3. d) \(\Delta U = Q_\mu\)

``````latex


