
``````latex


\section*{4a)}

\begin{itemize}
    \item There is a graph with the vertical axis labeled \( P \) and the horizontal axis labeled \( T \).
    \item A line starts from the origin and goes upwards to the right, labeled \( T_{\text{inel}} \).
    \item Another line starts from the origin and goes upwards to the right, but at a steeper angle.
    \item A horizontal line intersects the steeper line at point 1 and is labeled \( p_4 \).
    \item A vertical line goes down from point 1 to point 2 on the first line.
    \item A dashed line goes from point 2 downwards to the left, ending at point 3 on the horizontal axis, labeled \( T_z \).
\end{itemize}

\section*{4b)}

\begin{align*}
0 &= \dot{m}(h) + \dot{Q} - \dot{W}^o \\
0 &= \dot{m}(h) + \dot{Q} - W \\
W &= \dot{m}(h_2 - h_3) \\
\dot{m} &= \frac{W}{h_2 - h_3} = \\
h_2 &= h_g(T = \Phi) = \\
\text{isotrop:} \quad s_3 &= s_2 \\
\left( \frac{p_3}{p_2} \right)^{\frac{n-1}{n}} &= \frac{T_3}{T_2} \quad \Rightarrow \quad T_z = T_3 \left( \frac{p_2}{p_3} \right)^{\frac{n-1}{n}}
\end{align*}

\section*{4d)}

\begin{align*}
\mathcal{E} &= \frac{Q_{zu}}{W_t} = \frac{\dot{Q}_k}{\dot{W}_k} =
\end{align*}

``````latex


\section*{Student Solution}

\begin{itemize}
    \item[(a)] 
    \begin{align*}
        S_4 &= S_{1,4} \\
        S_4 &= S_f + \\
        X &= \frac{S_4 - S_f}{S_9 - S_f}
    \end{align*}
    
    \item[(b)] der Kreislauf wird gestoppt da der es nicht mehr verdichtet wird
\end{itemize}

\section*{Graphical Content Description}

There is a graph on the right side of the page. The graph has two axes: the horizontal axis and the vertical axis. The horizontal axis is labeled with a variable, but the label is not clearly visible. The vertical axis is also labeled with a variable, but the label is not clearly visible either.

The graph contains a curve that starts from the bottom left corner, rises steeply, and then levels off as it moves to the right. There are several points marked on the curve, and some of these points are connected by dashed lines to the axes, indicating specific values on the axes.

There is also a small diagram near the top of the page, which appears to be a schematic representation of a physical system. The diagram includes several components connected by lines, but the details are not clearly visible.

```