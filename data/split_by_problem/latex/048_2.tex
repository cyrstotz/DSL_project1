
``````latex


\section*{2.}

\begin{itemize}
    \item A graph is drawn with the x-axis labeled as $s$ and the y-axis labeled as $h$. 
    \item The graph contains several curves and points:
    \begin{itemize}
        \item Point 0 is at the origin.
        \item Point 1 is above point 0 on a curve labeled "adiabatic reversible".
        \item Point 2 is to the right of point 1 on a curve labeled "0.5 isotherm".
        \item Point 3 is above point 2 on a curve labeled "isotherm".
        \item Point 4 is to the right of point 3 on a curve labeled "0.75 isotherm".
        \item Point 5 is below point 4 on a curve labeled "adiabatic reversible".
        \item Point 6 is to the left of point 5 on a curve labeled "adiabatic reversible".
    \end{itemize}
    \item The points are connected in the following order: 0 to 1, 1 to 2, 2 to 3, 3 to 4, 4 to 5, and 5 to 6.
\end{itemize}

\subsection*{b)}

\begin{itemize}
    \item adiabatic reversible Schaltweise $\rightarrow s_5 = s_6$
    \item ideales Gas $\rightarrow$ Polytropengleichung
    \[
    \frac{T_6}{T_5} = \left( \frac{p_6}{p_5} \right)^{\frac{n-1}{n}} \rightarrow T_6 = T_5 \left( \frac{p_6}{p_5} \right)^{\frac{n-1}{n}}
    \]
    \item $e = 43.1 \, \text{J/kgK} \quad \frac{v_1}{v_2} = 0.5 \quad \left( \frac{n+1}{n} \right) = 328.07 \, \text{K} = T_6$
\end{itemize}

\subsection*{c)}

\[
\Delta e_{x, \text{ist}} = (h_0 - h_1 - T_0 (s_0 - s_1)) + (h_1 - h_2 - T_0 (s_1 - s_2)) + \ldots + (h_5 - h_6 - T_0 (s_5 - s_6))
\]
\[
= h_0 - h_6 - T_0 (s_0 - s_6)
\]

\subsection*{d)}

\[
0 = \Delta e_{x, \text{ist}} + e_{x, q} - w_t - e_{x, \text{verl}} \rightarrow e_{x, \text{verl}} = \Delta e_{x, \text{ist}} + e_{x, q} - w_t
\]

``````latex


