
``````latex


\section*{Aufgabe 1}

\subsection*{a)}

\begin{itemize}
    \item The graph is a plot of pressure \( p \) in \([ \text{bar} ]\) versus temperature \( T \) in \([ ^\circ \text{C} ]\).
    \item The x-axis is labeled \( T \) \([ ^\circ \text{C} ]\) and the y-axis is labeled \( p \) \([ \text{bar} ]\).
    \item There is a curve that starts at the origin, rises to a peak, and then falls back down, forming a closed loop.
    \item A straight line intersects the curve, labeled as \( T_{\text{tripel}} \).
    \item Points i and ii are marked on the graph, with point i on the curve and point ii on the straight line.
    \item The region to the left of the curve is labeled "Nach N. W. Z. 22-52 kJ/kg".
\end{itemize}

\subsection*{b)}

\begin{tabular}{|c|c|c|c|c|}
    \hline
    & \multicolumn{2}{c|}{\text{vollst. verd.}} & \text{ges.} & \text{s.v.k.} \\
    \hline
    & 1 & 2 & 3 & 4 \\
    \hline
    $T$ \([ ^\circ \text{C} ]\) & & & 37.1 & 31.33 \\
    \hline
    $p$ \([ \text{bar} ]\) & 3.3765 & 3.3755 & 8 & 8 \\
    \hline
    $s$ \([ \frac{\text{kJ}}{\text{kg K}} ]\) & & & 0.5165 & \\
    \hline
    $h$ \([ \frac{\text{kJ}}{\text{kg}} ]\) & & & & \\
    \hline
\end{tabular}

\[
\begin{aligned}
    T_1 &= T_{\text{Suno}} + 10 \, K = 0^\circ C + 10 \, K = 10^\circ C \\
    T_{\text{verd}} &= T_2 = 10^\circ C - 6 \, K = 4^\circ C \\
    s_2 &= s_g (4^\circ C) = 0.9165 \, \frac{\text{kJ}}{\text{kg K}} \quad \text{TAB 3-A10} \\
    T_3 &= T_3 = \frac{31.33 + \frac{4 \, C - 31.33}{(0.9374 - 0.5165)}}{(0.9374 - 0.5165)} = 37.1^\circ C
\end{aligned}
\]

\[
\begin{aligned}
    \text{1. HS (2-3):} \quad 0 &= \dot{m}_{12} (h_2 - h_3) - \dot{W}_k \\
    \dot{m}_{12} &= \frac{\dot{W}_k}{h_2 - h_3} \\
    h_2 &= h_g (10^\circ C) = 249.53 \, \frac{\text{kJ}}{\text{kg}} \\
    h_3 &= \ldots
\end{aligned}
\]

``````latex


\section*{Student Solution}

\subsection*{c)}
\[
x_{A} = \frac{h_{1} - h_{f}}{h_{g} - h_{f}} = 
\]

\[
h_{g} \Rightarrow (T_{A}) = 
\]

\[
h_{f} (T_{A}) = 
\]

\[
h_{1} = 
\]

\subsection*{d)}
\[
\varepsilon_{k} = \frac{\dot{Q}_{2u}}{\dot{W}_{t,1}} = \frac{\dot{Q}_{2u}}{\dot{Q}_{ab,ol} - \dot{Q}_{2u}}
\]

\subsection*{e)}
Sie würde noch etwas sinken und dann konstant bleiben, wenn die \textst{oszillationen} dissipation zu eis eintritt

```