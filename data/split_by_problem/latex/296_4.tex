
``````latex


\section*{Aufgabe 1}

\subsection*{a)}

\begin{center}
\textbf{Verbal Description of the Graph:}
\end{center}

The graph is a Pressure-Temperature ($P$-$T$) diagram. The $x$-axis is labeled $T$ (Temperature) and the $y$-axis is labeled $P$ (Pressure). There is a curve that starts from the origin and curves upwards to the right. This curve is labeled "Isotherm". There are two points marked on the curve, labeled 1 and 2. Point 1 is higher up on the curve than point 2. There is a horizontal line extending to the right from point 1, labeled "Höher". There is a vertical line extending downwards from point 1, labeled "Tripel". The point where the vertical line intersects the curve is labeled 2.

\subsection*{b)}

\textbf{ges.:} $\dot{m}$

\[
\dot{W}_k = \dot{m} (h_2 - h_3)
\]

\[
h_3 = h_4 (0.5 \text{ bar}) = 93.42 \frac{\text{kJ}}{\text{kg}}
\]

\[
h_2 = h_2 (-20^\circ \text{C}) = 335.37 \frac{\text{kJ}}{\text{kg}}
\]

\[
\frac{\dot{W}_k}{h_2 - h_3} = \dot{m} = \frac{28 \text{W}}{335.37 - 93.42 \frac{\text{kJ}}{\text{kg}}} = 0.072 \frac{\text{kg}}{\text{s}}
\]

\[
s_2 = s_3 = 0.906 \frac{\text{kJ}}{\text{kg K}}
\]

\[
h_2 = h_2 (0.5 \text{ bar}) = 264.7 \frac{\text{kJ}}{\text{kg}}
\]

``````latex


\section*{4c)}
\text{ges.: } x_n

\[
h_{m} = \frac{h_{f}}{h_{g}} = h_{f}(18x-1) = 264.11 \frac{kJ}{kg}
\]

\[
\dot{m} = \frac{m_{f}}{m_{g}}
\]

\[
x_{n} = \frac{S_{g} - S_{ag}}{S_{ag} - S_{a}}
\]

\section*{4d)}
\[
E_{K} = \frac{\dot{Q}_{ml}}{T_{ml}} = \frac{\dot{Q}_{kl}}{T_{kl}}
\]

\[
\dot{Q}_{k} = \dot{m}(h_{2} - h_{1})
\]

\section*{e)}
Die Temperatur im Innenraum würde weiterhin sinken, da der Innenraum des Gefrierfaches nach außen hin abgedichtet ist.

```