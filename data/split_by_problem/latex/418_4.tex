
``````latex


\section*{Aufgabe 4}

\subsection*{a)}

\begin{description}
    \item[Graph Description:] The graph is a pressure-temperature ($p$-$T$) diagram. The horizontal axis is labeled $T$ [K] and the vertical axis is labeled $p$ [bar]. There are three points labeled 1, 2, and 3. Point 1 is connected to point 2 by a horizontal line labeled "isobares Entspannen". Point 2 is connected to point 3 by a vertical line labeled "isothermes Drucksenken".
\end{description}

\subsection*{b)}

\begin{itemize}
    \item Von 3 $\rightarrow$ 4:
    \item Von 2 $\rightarrow$ 3:
\end{itemize}

\[
0 = -\dot{W}_k + \dot{m} (h_2 - h_3)
\]

\[
h_2 = 
\]

\[
h_3 = 264{,}15 \frac{\text{kJ}}{\text{kg}}
\]

\[
T_2 = T_1 - 6K
\]

\[
\Rightarrow \dot{m} = \frac{\dot{W}_k}{h_2 - h_3}
\]

\begin{flushright}
    Tabelle 4.11
\end{flushright}

``````latex


\begin{itemize}
    \item[c)]
    \begin{equation}
        \varphi = \varphi_f + x_1 \left( \varphi_g - \varphi_f \right)
    \end{equation}
    \begin{equation}
        \Rightarrow x_1 = \frac{\varphi - \varphi_f}{\varphi_g - \varphi_f}
    \end{equation}
    
    \item[d)]
    \begin{equation}
        \epsilon_u = \frac{\left| \dot{Q}_{ab} \right|}{\left| \dot{W} \right|}
    \end{equation}
    
    \item[e)]
    \text{Kälte}
\end{itemize}

```