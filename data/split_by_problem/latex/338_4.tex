
``````latex


\section*{Aufgabe 4}

\subsection*{a)}

\begin{description}
    \item[Graph:] Ein Diagramm mit einer y-Achse beschriftet mit \( p \) [mixed] und einer x-Achse beschriftet mit \( T \) [°C]. Die Kurve beginnt links unten und steigt nach rechts oben an. Es gibt drei markierte Punkte auf der Kurve: Punkt 1 liegt auf der Kurve, Punkt 2 liegt links von der Kurve und Punkt 3 liegt unterhalb der Kurve. Die Region oberhalb der Kurve ist mit "gas" beschriftet, die Region unterhalb der Kurve ist mit "flüssig" beschriftet.
\end{description}

\subsection*{b)}

\begin{description}
    \item[Diagramm:] Ein Kreisdiagramm mit einem Pfeil, der von links nach rechts durch den Kreis geht und mit den Zahlen 2 und 3 beschriftet ist. Ein weiterer Pfeil zeigt nach oben aus dem Kreis und ist mit \( \dot{W}_K \) beschriftet.
\end{description}

\[
0 = \dot{m} (h_2 - h_3) - \dot{W}_K
\]

\[
h_2 \quad h_3(8 \text{ bar}) =
\]

\[
s_2 = s_3
\]

\[
s_3 \quad s_3 \text{ superheated} \quad s_3(8 \text{ bar}) = \text{TAB 12}
\]

\subsection*{c)}

\[
x = \frac{h_1 - h_f}{h_g - h_f}
\]

\subsection*{d)}

\[
\epsilon_K = \frac{|\dot{Q}_{zu}|}{|\dot{W}_{el}|} = \frac{|\dot{Q}_{zu}|}{|\dot{Q}_{ab} - \dot{Q}_{zu}|}
\]

\subsection*{e)}

Die Temperatur würde sich immer weiter senken, bis sie irgendwann stagniert, da keine Wärme mehr entzogen werden kann.

```