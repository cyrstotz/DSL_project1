
``````latex


\section*{Problem 4}

\subsection*{a)}

\begin{itemize}
    \item The first graph is a plot with the y-axis labeled \( p(T) \) and the x-axis labeled \( T(K) \). The graph shows a curve that starts at the bottom left, rises to a peak, and then falls back down. The regions under the curve are labeled as follows:
        \begin{itemize}
            \item The region to the left of the peak is labeled "Unterkühlte Flüssigkeit".
            \item The region under the peak is labeled "Masse\_dampf".
            \item The region to the right of the peak is labeled "übersättigter Dampf".
        \end{itemize}
\end{itemize}

\subsection*{b)}

\begin{itemize}
    \item The second graph is a plot with the y-axis labeled \( p(T) \) and the x-axis labeled \( T(K) \). The graph shows a straight line starting from the origin and rising to the right. The line is labeled "gasförmig". There are two points marked on the line:
        \begin{itemize}
            \item Point 1 is at the bottom left.
            \item Point 2 is at the top right.
        \end{itemize}
    \item There is an arrow pointing from point 1 to point 2 labeled "Schritt ii".
\end{itemize}

\begin{itemize}
    \item The third graph is a plot with the y-axis labeled \( p(T) \) and the x-axis labeled \( T(K) \). The graph shows a curve that starts at the bottom left, rises to a peak, and then falls back down. The regions under the curve are labeled as follows:
        \begin{itemize}
            \item The region to the left of the peak is labeled "1".
            \item The region under the peak is labeled "2".
            \item The region to the right of the peak is labeled "3".
        \end{itemize}
    \item There is an arrow pointing from the left side of the peak to the right side labeled "Schritt i".
\end{itemize}

\subsection*{b)}

\[
0 = \dot{m}_2 (h_2 - h_3) + \dot{Q} - \dot{w}
\]

\[
h_2 = \frac{s_2 - s_3}{s_2 - s_4} h_4 (27.7.15 K) = \frac{2.95.55 \frac{kJ}{kg}}{0.916 \frac{kJ}{kg \cdot K}}
\]

\[
T_i = 10^\circ C = 283.15 K
\]

\[
p_{\text{1bar}} = \text{const}
\]

\[
p_1 = 1 \text{mbar} = p_2
\]

\[
T_2 = 283.15 K - 6 K = 277.15 K
\]

\[
h_1 = \frac{s_2 - s_3}{s_2 - s_4} (h_3 - h_2) + h_3
\]

``````latex


\begin{itemize}
    \item[c)] \( h_1 - h_4 \), da adiabate Drossel \\
    \[ h_2 = h_f (85 bar) - 93.42 \frac{kJ}{kg} \]
    
    \[
    \xi = \frac{h_1 - h_f (40^\circ C)}{h_g (40^\circ C) - h_f (40^\circ C)} \quad \text{TAD A10} = 0.139
    \]
    
    \item[d)] \[ \dot{E}_k = \frac{\dot{Q}_2 u}{\dot{w}_t} = \dot{w}_k \]
    
    mit angegebenen Werten rechnen \\
    \[ 0 = \dot{m}_R (h_1 - h_2) + \dot{Q}_k \]
    
    \[
    \Rightarrow \dot{Q}_k = \dot{m}_R (h_2 - h_1) = \diameter 156.25 W
    \]
    
    \[
    h_1 = h_4
    \]
    
    \[
    h_2 = h_g (-22^\circ C) = 259.08 \frac{kJ}{kg} \quad \text{TAD A10}
    \]
    
    \[
    \dot{E}_k = 5.58
    \]
    
    \item[e)] Die Temperatur würde absinken, da sich $T$ somit zum Wärmereservoir angleichen muss.
\end{itemize}

```