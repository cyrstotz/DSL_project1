
``````latex


\section*{A3}

\textbf{Gegeben:} $C_v = 0{,}633$ \\
$M_g = 50 \, \frac{\text{kg}}{\text{kmol}}$

\subsection*{a)}
\[
p_{g1} \cdot V_1 = m_g \cdot R \cdot T
\]
\[
R = \frac{\mathcal{R}}{M_g} = \frac{8{,}314 \, \frac{\text{kJ}}{\text{kmol} \cdot \text{K}}}{50 \, \frac{\text{kg}}{\text{kmol}}} = 0{,}16628 \, \frac{\text{kJ}}{\text{kg} \cdot \text{K}}
\]

\[
V_{g1} = 3{,}14 \, \text{L}
\]
\[
T_{g1} = 500^\circ \text{C}
\]
\[
m_{ew} \approx 0{,}1 \, \text{kg}
\]

\subsection*{K.G.W.}
\[
\begin{array}{|c|}
\hline
p_{g1} \cdot A = p_{amb} \cdot A + m_k \cdot g + m_{ew} \cdot g \\
\hline
\end{array}
\]

\[
p_{g1} = p_{amb} + \frac{m_k \cdot g}{A} + \frac{m_{ew} \cdot g}{A}
\]

\[
p_{g1} = 4 \cdot 10^5 \, \text{N/m}^2 + \frac{32{,}81 \, \text{N}}{0{,}0079 \, \text{m}^2} + \frac{0{,}1 \cdot 9{,}81 \, \text{N}}{0{,}0079 \, \text{m}^2} = 140{,}034 \, 4406 \, \frac{\text{N}}{\text{m}^2} \approx 1{,}4 \, \text{bar}
\]

\[
A = \frac{D^2}{4} \pi = \frac{100 \cdot 10^{-3} \, \text{m}^2}{4} \pi = 0{,}00785398 \, \text{m}^2
\]

\[
pV = mRT
\]

\[
\Rightarrow m_{g1} = \frac{p_{g1} \cdot V_{g1}}{R \cdot T_1} = \frac{1{,}4 \cdot 10^5 \, \text{Pa} \cdot 3{,}14 \cdot 10^{-3} \, \text{m}^3}{0{,}16628 \cdot 10^3 \, \frac{\text{J}}{\text{kg} \cdot \text{K}} \cdot (500 + 273{,}15) \, \text{K}} = 0{,}00634 \, \text{kg} \approx 3{,}422 \, \text{g}
\]

\subsection*{b)}
\[
x_{eis,2} > 0 \quad x_{eis,1} = \frac{m_{eis}}{m_{ew}} = 0{,}6
\]

\[
\Rightarrow m_{eis} = 0{,}1 \cdot m_{ew} = 0{,}06 \, \text{kg}
\]

\textbf{Gas und EW Thermodyn. GGW}

Weil Dichte von Eis und Wasser gleich sind, verändert sich die Masse (und Volumen) von Eiswasser nicht. Durch das KGW sieht man, dass $p_{g2} \stackrel{!}{=} p_{g1}$

\[
p_{g1,2} = 1{,}4 \, \text{bar}
\]

``````latex


\section*{b) fort.}

\begin{figure}[h!]
    \centering
    \begin{tabular}{c}
        \begin{tabular}{|c|}
            \hline
            EV \\
            \hline
        \end{tabular} \\
        \begin{tabular}{|c|}
            \hline
            Gas \\
            \hline
        \end{tabular} \\
    \end{tabular}
    \caption*{A diagram showing two horizontal sections. The top section is labeled "EV" and the bottom section is labeled "Gas". There is an arrow labeled "$\dot{Q}$" pointing upwards from the "Gas" section to the "EV" section.}
\end{figure}

\[
m_{g1} = m_{g2} = 3,422 \, \text{g}
\]

\noindent
Weil das EV dann wie Wasser im Nassdampf behandelt werden kann und $x_2 > 0$ ist, ist $T_{EV,2} = T_{EV,1} = 0^\circ \text{C}$

\noindent
Für thermodynamisches Gleichgewicht muss $T_{g,2} = T_{EV,2} = 0^\circ \text{C}$ sein

\section*{c) Geschl. System am Kolben, Grenzen um Gas,}

\[
\Delta E = E_2 - E_1 = Q - W_v
\]

\noindent
kin. + pot. Energien vernachlässigen

\[
m_g (u_2 - u_1) = Q - W_v
\]

\[
u_2 - u_1 = c_v (T_2 - T_1)
\]

\[
\Rightarrow Q = m_g c_v (T_2 - T_1) + W_v
\]

\[
\Rightarrow W_v = V_1^{rev} p_g = m_g R_g (T_2 - T_1) = p_g (V_2 - V_1)
\]

\[
V_2 = \frac{m_g R T_2}{p_2} = \frac{0,034 \, \text{kg} \cdot 0,116 \cdot 10^3 \, \frac{J}{\text{kg} \cdot K} \cdot (273,15 \, K)}{1,4 \cdot 10^5 \, \frac{N}{m^2}} = 0,00111 \, m^3
\]

\[
\Rightarrow W_v = 1,4 \cdot 10^5 \, \frac{N}{m^2} (0,00111 \, m^3 - 3,14 \cdot 10^{-3} \, m^3) = -284,187 \, J
\]

\[
\Rightarrow Q_{12} = 3,422 \cdot 10^{-3} \, \text{kg} \cdot 0,633 \cdot 10^3 \, \frac{J}{\text{kg} \cdot K} \left[ 273,15 - 500 + 273,15 \right] K - 284,187 \, J
\]

\[
= -1 367,2484 \, J \approx -1,367 \, kJ
\]

``````latex


\section*{A3 forts}

\subsection*{d) geschl. Kolben}

\begin{align*}
    &E_2 - E_1 = Q - W_{u}^{0}, \text{inkompressible FL} \\
    &m_{ew} (u_2 - u_1) = Q \\
    &u_2 = \frac{Q}{m_{ew}} + u_1
\end{align*}

\begin{align*}
    &u_1 = u_f + x_1 (u_g - u_f)
\end{align*}

\text{Tab. 1:} \quad
\begin{aligned}
    &m_{wf} (T = 0^\circ C) = -0,045 \frac{\text{kJ}}{\text{kg}} \\
    &u_g (T = 0^\circ C) = -333,458 \frac{\text{kJ}}{\text{kg}}
\end{aligned}

\begin{align*}
    &u_2 = -0,045 + 0,6 \cdot (-333,458 - (-0,045)) \\
    &\quad = -260,0928 \frac{\text{kJ}}{\text{kg}}
\end{align*}

\begin{align*}
    &u_2 = \frac{-1,36742 \frac{\text{kJ}}{0,1 \text{kg}} - 200,0928 \frac{\text{kJ}}{\text{kg}}}{0,1 \text{kg}} = -213,7628 \frac{\text{kJ}}{\text{kg}}
\end{align*}

\begin{align*}
    &u_2 = u_f + x_2 (u_g - u_f)
\end{align*}

\begin{align*}
    &x_2 = \frac{u_2 - u_f}{u_g - u_f} = \frac{-213,7628 - (-0,045)}{-333,458 - (-0,045)} \approx 0,641
\end{align*}

``````latex


