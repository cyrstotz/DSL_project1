
``````latex


\section*{Aufgabe 4}

\subsection*{a)}

\begin{description}
    \item[Graph Description:] The graph is a Pressure-Temperature (P-T) diagram. The x-axis is labeled \( T \) (Temperature) and the y-axis is labeled \( P \) (Pressure). The graph shows a closed loop with four key points labeled 1, 2, 3, and 4. The process between these points is described as follows:
    \begin{itemize}
        \item From point 1 to point 2: "adiabate und reversibel (isentrop) verdichtung"
        \item From point 2 to point 3: "isobare vollständige verdampfung"
        \item From point 3 to point 4: "mass dampf"
        \item From point 4 to point 1: "adiabate entspannung"
    \end{itemize}
    Additionally, there are annotations indicating "isobare kondensation" and "ND Gebiet" near the bottom right of the loop.
\end{description}

\subsection*{b)}

\begin{align*}
    \dot{m}_{R134a} & \quad \text{? 1HS} \\
    O &= \dot{m} \left[ h_2 - h_3 \right] + \dot{Q} - \dot{W}_u \\
    \frac{\dot{W}_u}{h_2 - h_3} &= \dot{m}_{R134a} \\
    \dot{W}_u &= 28 \, \text{W} \\
    T_i &= -10^\circ \text{C} \\
    &= 263.15 \, \text{K} \\
    h_3 &= h_f = 264.15 \, \frac{\text{kJ}}{\text{kg}} \quad \text{(Tabelle A-11)} \\
    \dot{W}_u &= \dot{m}_{R134a} C_p (T_2 - T_3) \\
    T_3 &= 31.33^\circ \text{C} \\
    T_2 &= \\
    \text{wir wissen } s_2 = s_3
\end{align*}

``````latex


\section*{Aufgabe 4 weiter}

\subsection*{c)}
\begin{itemize}
    \item $x_1? \quad 4 \rightarrow 1$
    \item \textbf{Dampfkreis:} 
    \[
    0 = \dot{m} [0] + \dot{Q} - \dot{W}_u
    \]
    \[
    \dot{W}_u = \dot{Q} = 0 \quad \text{Prozess ist isentrop}
    \]
\end{itemize}

\subsection*{d)}
\begin{itemize}
    \item Leistungszahl $E_k$ wird mit 
    \[
    \frac{\dot{Q}_{zu}}{|\dot{W}|} = \frac{\dot{Q}_{zu}}{(\dot{Q}_{ab} - \dot{Q}_{zu})} \quad \text{für Kältemaschine}
    \]
    \item In unserem Fall 
    \[
    \frac{\dot{Q}_k}{(\dot{Q}_{ab} - \dot{Q}_k)} \quad \text{gibt uns} \quad E_k
    \]
\end{itemize}

\subsection*{e)}
\begin{itemize}
    \item Die Temperatur würde so zu einem Gleichgewicht
    \item Die innere Temperatur kann nur so kalt werden, wie es bei der Umgebung heiss ist. Wenn 
    \item $\dot{Q}_{ab}$ langsam zu 0 wird, haben wir die tiefste Innentemperatur erreicht.
\end{itemize}

```