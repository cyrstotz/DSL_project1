
``````latex


\section*{Aufgabe 2}

\subsection*{a)}

\begin{description}
    \item[Graph Description:] The graph is a Temperature-Entropy (T-S) diagram. The y-axis is labeled \( T \) with the unit \( [K] \). The x-axis is labeled \( s \) with the unit \( \left[ \frac{kJ}{kg \cdot K} \right] \). There are three isobars labeled \( p_0 \), \( p_B \), and \( p_S \) from bottom to top. The graph shows a cycle with six states labeled \( 0, 1, 2, 3, 4, 5 \). The process moves from state 0 to 1, then to 2, 3, 4, 5, and back to 0. The paths between states are marked with arrows indicating the direction of the process.
\end{description}

\subsection*{b)}

\begin{align*}
    T_S &= 437.9 K \\
    w_s &= 220 \frac{m}{s} \\
    p_S &= 0.5 bar \\
    m_s &= \text{ingeo} \\
    \\
    h_s &= c_p \cdot T_S \quad (\text{ideales gas}) \\
    h_s &= 7.006 \frac{kJ}{kg \cdot K} \cdot 437.9 K = 43.69 \frac{kJ}{K} \\
    \\
    \text{adiabat:} \quad T_0 &= T_S \left( \frac{p_0}{p_S} \right)^{\frac{R}{c_p}} = 437.9 K \left( \frac{0.1}{0.5 bar} \right)^{\frac{0.19 \frac{kJ}{kg \cdot K}}{0.4 \frac{kJ}{kg \cdot K}}} = 325.07 K \\
    \\
    h_0 &= c_p \cdot T_0 = 330 \frac{kJ}{kg} \quad \Rightarrow \quad \text{Gesamtsystem:} \\
    \\
    \text{Für } \dot{Q} &= \dot{m} \left( h_0 - h_6 + \frac{w_2^2 - w_1^2}{2} \right) + Q \\
    \\
    h_0 &= c_p \cdot 263.15 K = 2.4 \frac{kJ}{kg}
\end{align*}

\begin{description}
    \item[Additional Notes:] 
    \begin{itemize}
        \item The student mentions "Schubdüse Adiabat: Stationär" and "reversible Schubdüse: \(\dot{S}_{erzeugt} = 0\)".
        \item There are notes about "adiabat" and "reversible (adiabat)" processes.
        \item The student also writes "V. dp c_v (p_0 - p_3)" and "c_v p_0 - p_3".
    \end{itemize}
\end{description}

``````latex


\section*{Problem c}

\[
\Delta s_{str} = h_0 - h_6 - T_0 (s_0 - s_6) + \Delta ke
\]

\textit{Kein potentielle energie} \\
\textit{Stationärer Prozess}

\[
h_0 = 244.6 \frac{kJ}{kg} \quad \text{Cp: } T_0
\]

\[
h_6 = 342.04 \frac{kJ}{kg} \quad \text{Cp: } T_0
\]

\[
T_0 = 243.75 \quad \text{(Umgebung)}
\]

\[
\Delta ke = \frac{w_6^2 - w_0^2}{2}
\]

\[
s_0 - s_6 = s_0 (T_0) - s_6 (T_0) - R \ln \left( \frac{p_0}{p_6} \right)
\]

\[
\Delta ke = \frac{200^2 - 520^2}{2} = -770.05 \frac{kJ}{kg}
\]

\[
s_0 - s_6 = Cp \ln \left( \frac{T_0}{T_6} \right) = -0.33727 \quad \left( \frac{kJ}{kg \cdot K} \right) = -82 \frac{kJ}{kg}
\]

\[
\Delta c = 244.6 \frac{kJ}{kg} - 342.04 \frac{kJ}{kg} + 82 \frac{kJ}{kg} - 770.05 \frac{kJ}{kg} = -725.46 \frac{kJ}{kg}
\]

\section*{Problem d}

\[
ex_{verl} = T_0 \cdot S_{proz}
\]

\[
ex_{verl} = 0 = ex_{str} + \sum Q_{zu} \left( 1 - \frac{T_0}{T_i} \right) - W - ex_{verl}
\]

\[
ex_{verl} = -725.46 \frac{kJ}{kg} + 7745 \left( 1 - \frac{253}{298} \right)
\]

``````latex


