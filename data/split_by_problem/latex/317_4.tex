
``````latex


\section*{A4}

\subsection*{a)}

\[
p \quad [\text{mbar}]
\]

\begin{description}
    \item[Graph Description:] The graph is a pressure-temperature ($p$-$T$) diagram. The x-axis is labeled $T$ [°C] and the y-axis is labeled $p$ [mbar]. The graph shows a phase diagram with a curve that starts from the origin and rises non-linearly. The curve is labeled "Tripel" at the point where it changes slope. Below the curve, the region is labeled "gas," and above the curve, the region is labeled "flüssig." There are three points marked on the graph: point 1 on the curve, point 2 above the curve, and point 3 below the curve. The vertical line connecting points 1 and 2 is labeled "isobare," and the vertical line connecting points 1 and 3 is labeled "isotherm." The horizontal line connecting points 2 and 3 is labeled "isotherm." The region above the curve is labeled "flüssig" and "fest."
\end{description}

\subsection*{b) Energieeinsatz am Verdichter}

\[
Q = \dot{m}_{\text{n}} \cdot (h_2 - h_3) - \dot{W}_K
\]

\[
\text{adiabat \& reversibel} \Rightarrow \text{isentrop} \Rightarrow s_2 = s_3
\]

\[
T_2 = T_1 - 5K = -22^\circ C
\]

\[
h_2 (T = -22^\circ C, \text{sat}) \text{ aus Tab. A-10} = h_g = 239.08 \frac{\text{kJ}}{\text{kg}}
\]

\[
s_2 = s_g = 0.9351 \frac{\text{kJ}}{\text{kg} \cdot K}
\]

\[
h_3 \text{ interpolieren aus Tab. A-11}
\]

``````latex


\section*{Aufgabe (c)}

\subsection*{Teil a}
\[
T_q: \text{vollst. kond. sei } p_q = 8 \, \text{bar}
\]
\[
= 37.3^\circ \text{C}
\]
\[
\text{drossel ist isotherm}
\]

\subsection*{Teil d}
\[
\eta_d = \frac{\left| \dot{Q}_{21} \right|}{\left| \dot{W}_T \right|}
\]

\subsection*{Teil e}
\begin{quote}
die Temperatur würde sich irgendwann stabilisieren, wenn das Kältemittel keine Wärme mehr aufnehmen kann, sprich $T_i = T_q$
\end{quote}

```