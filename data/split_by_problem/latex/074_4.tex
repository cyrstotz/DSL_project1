
``````latex


\section*{Aufgabe 4}

\subsection*{a)}

\begin{itemize}
    \item A graph is drawn with the x-axis labeled as \( T \, (^\circ \text{C}) \) and the y-axis labeled as \( p \, (\text{mbar}) \).
    \item The graph shows a curve starting from the origin and increasing non-linearly.
    \item There are three points marked on the graph:
        \begin{itemize}
            \item Point 1 is at the lower part of the curve.
            \item Point 2 is at the upper part of the curve, with a horizontal line extending to the right labeled as "isobar eingehen".
            \item Point 3 is below point 2, with a vertical line connecting points 2 and 3 labeled as "Tripelpunkt".
            \item The curve between points 1 and 3 is labeled as "isotherm".
        \end{itemize}
\end{itemize}

\subsection*{b) Kühlkreislauf}

\begin{tabular}{|c|c|c|}
    \hline
    Zustand & p & \\
    \hline
    1 & $p_1$ & \\
    \hline
    2 & $p_1 - 16^\circ$ & \\
    \hline
    3 & 8 bar & \\
    \hline
    4 & 8 bar & \\
    \hline
\end{tabular}

\begin{itemize}
    \item The second row is circled.
    \item There is an arrow pointing from the second row to the right, labeled as \( h_2 \).
    \item Another arrow points from the second row to the right, labeled as \( x_2 = 1 \).
    \item An arrow points from the fourth row to the right, labeled as \( x_4 = 0 \, h_4 \).
\end{itemize}

\[
T_i = -20^\circ \text{C (aus Diagramm)}
\]

``````latex

\section*{Aufgabe 4}

\subsection*{b)}

\begin{align*}
    T_A B A 11 \\
    h_1 (8 \, \text{bar}, \, x_1 = 0) &= \cancel{264,45 \, \frac{\text{kJ}}{\text{kg}}} \\
    &= 93,42 \, \frac{\text{kJ}}{\text{kg}} = h_1 \\
    h_2 (-16^\circ \text{C}, \, x_2 = 1) &= \cancel{2401,15 \, \frac{\text{kJ}}{\text{kg}}} \\
    &= 93,92 \, \frac{\text{kJ}}{\text{kg}}
\end{align*}

\[
T_1 = -20^\circ \text{C} - 0^\circ \text{C} = -10^\circ \text{C}
\]

\subsection*{Stationärer Fließprozess}

\[
0 = \dot{m} \left[ h_e - h_a \right] + \dot{Q} - \dot{W}
\]

\[
0 = m \left[ h_a - h_2 \right]
\]

``````latex


\section*{c)}
\[
x = \frac{h - h_s}{h_g - h_f}
\]
\[
x(1)
\]

\section*{d)}
\[
\epsilon_k = \frac{|\dot{Q}_u|}{|\dot{W} + |} = \frac{|\dot{Q}_u|}{|\dot{Q}_{ab}| - |\dot{Q}_u|}
\]

\section*{d)}
Die Luft würde von Gas zu Feststoff werden und es bildet sich \underline{Eis}
```