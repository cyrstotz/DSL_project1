
``````latex


\section*{Aufgabe 4}

\subsection*{a) $p$ [bar]}

\begin{description}
    \item[Graph 1:] 
    \begin{itemize}
        \item The graph is a pressure-temperature ($p$-$T$) diagram.
        \item The x-axis is labeled $T$ [K].
        \item The y-axis is labeled $p$ [bar].
        \item There is a curve that starts from the origin, rises to a peak, and then falls back down, forming a dome shape.
        \item The left side of the dome is labeled "unterkühlte Flüssigkeit".
        \item The right side of the dome is labeled "überhitzter Dampf".
        \item The region under the dome is labeled "Nassdampf".
        \item Two vertical lines are drawn from the x-axis to the curve, labeled 2 and 1 respectively.
    \end{itemize}
\end{description}

\subsection*{a) $p$ [bar]}

\begin{description}
    \item[Graph 2:] 
    \begin{itemize}
        \item The graph is another pressure-temperature ($p$-$T$) diagram.
        \item The x-axis is labeled $T$ [°C].
        \item The y-axis is labeled $p$ [bar].
        \item There is a curve that starts from the origin, rises to a peak, and then falls back down, forming a dome shape.
        \item The left side of the dome is labeled $T_{ci}$.
        \item The right side of the dome is labeled $T_{ci}$.
        \item The peak of the dome is labeled 1.
        \item A point on the left side of the dome is labeled 2.
        \item The y-axis has a label $p_T$ and a mark indicating $p_T = 6 \text{ bar}$.
    \end{itemize}
\end{description}

``````latex


\section*{a) Energiefluss um das innere Kältemittel im Wärmeübertrager}

\begin{center}
\begin{tabular}{|c|}
\hline
\begin{minipage}{0.9\textwidth}
\centering
\includegraphics[width=0.8\textwidth]{diagram1.png}
\end{minipage} \\
\hline
\end{tabular}
\end{center}

\textbf{Description of the diagram:} The diagram is a rectangular representation of a heat exchanger. The rectangle is divided into two horizontal sections. The upper section is shaded with diagonal lines, and the lower section is outlined with a dashed orange line. Inside the lower section, there is a downward arrow labeled \( \dot{Q}_K \) and a horizontal arrow labeled \( R134 \).

\[
\text{SFP: } 0 = \dot{m}_R [h_1 - h_2] + \dot{Q}_K
\]

\textbf{Erneute Energiefluss um Wasser:}

\begin{center}
\begin{tabular}{|c|}
\hline
\begin{minipage}{0.9\textwidth}
\centering
\includegraphics[width=0.8\textwidth]{diagram2.png}
\end{minipage} \\
\hline
\end{tabular}
\end{center}

\textbf{Description of the diagram:} The second diagram is similar to the first one, with a rectangular representation of a heat exchanger. The upper section is shaded with diagonal lines, and the lower section is outlined with a dashed orange line. Inside the lower section, there is a downward arrow labeled \( \dot{Q}_K \) and a horizontal arrow labeled \( R134 \). Additionally, there is a small arrow pointing to the right at the top right corner of the rectangle.

\[
Q = m (s_{2} - s_{a}) = \frac{\dot{Q}_K}{T_i} - \frac{\dot{Q}_K}{T_{ii}}
\]

\[
\dot{Q}_K = m (s_{2} - s_{a}) \cdot (-T_{ii} + T_{i})
\]

\[
Q = \dot{m} h_{1} - \dot{m}
\]

\[
\dot{m}_R = \frac{-\dot{Q}_K}{h_1 - h_2}
\]

``````latex


\begin{tabular}{|c|c|c|c|c|c|}
\hline
 & W & Q & Z & p & T \\
\hline
 & & & 0 & & \\
\hline
 & & & 7 & & \\
\hline
 & & & 2 & & \\
\hline
 & & 28W & 3 & 86w & \\
\hline
 & & & 4 & pa & \\
\hline
 & & & 7 & & \\
\hline
\end{tabular}

\vspace{1cm}

\begin{itemize}
    \item The table has 6 columns and 6 rows.
    \item The first row contains the headers: W, Q, Z, p, T.
    \item The second row has the value 0 under column Z.
    \item The third row has the value 7 under column Z.
    \item The fourth row has the value 2 under column Z.
    \item The fifth row has the value 28W under column Q, 3 under column Z, and 86w under column p.
    \item The sixth row has the value 4 under column Z and pa under column p.
    \item The seventh row has the value 7 under column Z.
\end{itemize}

```