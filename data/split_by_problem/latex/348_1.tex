
``````latex


1. a) Prozess Reaktor isotherm, da der Wasserdampf auch \\
\hspace{1cm} \rule{5cm}{0.5pt} \\
Energiebilanz Reaktor: \\
\[
Q_{\text{aus}} + \dot{Q}
\]
stationär \\
\[
0 = \dot{m} (\text{che. In}) + \dot{Q} - \sum \dot{m}_h
\]
\hspace{1cm} W = 0, da kein osser \\
\hspace{1cm} \rightarrow \text{d}e/\text{d}t \text{ Verdampfungswärme} \\
\hspace{1cm} \text{Erhaltung} \\
Wärme $T = 100^\circ C$ \rightarrow gesätt. siedend \rightarrow hf \\
TAB 42 Lf (T = 100^\circ C) = 282,98 \frac{kJ}{kg} \\
\hspace{1cm} ha: \downarrow gesätt. siedend Flüssigkeit $T_{\text{aus}} = 100^\circ C$ Lf (T = 100^\circ C) = 419,07 \frac{kJ}{kg} \\
\[
Q_{\text{aus}} = \dot{m} (\text{ha-he}) - Q_a = -62,182 \text{kW}
\]

b) $T_{\text{m}}$ Mitteltemperatur \\
\[
T = \frac{\int_{sa}^{se} T \, ds}{sa-se} = \frac{q_w}{sa-se}
\]
Da Wasser im Kühlmittel \\
\hspace{1cm} \downarrow q_w = \Delta h \rightarrow \frac{ha-he}{sa-se} = \frac{T_{\text{aus}} - T_{\text{ein}}}{\ln \left( \frac{T_{\text{aus}}}{T_{\text{ein}}} \right)} \\
\hspace{1cm} \text{Unabhängig ideal Flüssigkeit} \\
\hspace{1cm} \rightarrow \Delta h = c_p (T_{\text{aus}} - T_{\text{ein}}) \\
\hspace{1cm} \rightarrow ss = c_p \ln \left( \frac{T_{\text{aus}}}{T_{\text{ein}}} \right) \\
\[
T_{\text{m}} = 293,12 \text{K}
\]

c) Seitz \rightarrow Beim Wärmetransport \\
\hspace{1cm} Ohne Massenfluss, stationär \\
\[
0 = \sum \frac{\dot{Q}}{T_i} + \dot{S}_{\text{er}}
\]
\[
\dot{S}_{\text{er}} = -\sum \frac{\dot{Q}}{T_i} = -\frac{Q_{\text{ein}}}{T_{\text{ein}}} + \frac{\dot{Q}}{T_{\text{m}}} = 45,5 \text{W}
\]

\hspace{1cm} \rule{5cm}{0.5pt} \\
Q = 62,182 \text{kW}

``````latex


d) Halboffenes System:

\begin{align*}
\Delta E &= m_2 U_2 - m_1 U_1 = \dot{m}_e h_e + \sum \dot{Q} - \sum \dot{E}_W \\
&\downarrow \\
&= 575 \frac{kg}{s} \quad \text{(TBA 4-2)} \quad p_T = 100^\circ C \quad x = 0.005 \\
&\dot{m}_e = \dot{m}_2 \\
&\text{hor. gerade siedende Flüssigkeit} \\
&\dot{m}_e = \frac{A_{TBA}}{A_{TBA 2}} \quad \frac{U_A}{U_A} = \frac{h_f(T_{in}) + x \cdot (h_{fg}(T_{in}) - h_f(T_{in}))}{h_f(T_{in})} \\
&\text{(crossed out)} \quad = 425.38 \\
&\text{TBA 2} \\
&\text{T}_{\text{ein}} = 20^\circ C \quad h_f(T_{\text{ein}}) = 83.96 \frac{kJ}{kg} \\
&U_2 = UF(T_2) = 252.55 \frac{kJ}{kg} \\
&\text{siedende Flüssigkeit bei } T_2 = 70^\circ C \\
&\text{(m_1 + \dot{m}_e t) U_2 - m_1 U_1 = \dot{m}_e h_e + \dot{Q}} \\
&\dot{m}_e U_2 - \dot{m}_e h_e = \dot{Q} + m_1 U_1 - m_1 U_2 \\
&\dot{m}_e (U_2 - h_e) = \dot{Q} + m_1 (U_1 - U_2) \quad \text{1: U_2 - h_e} \\
&\dot{m}_e = \frac{\dot{Q} + m_1 (U_1 - U_2)}{(U_2 - h_e)} = 3589.4 \frac{kg}{s}
\end{align*}

``````latex


