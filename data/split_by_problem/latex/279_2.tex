
``````latex


\section*{Aufgabe 2:}

\subsection*{a)}

\begin{description}
    \item[Graph Description:] The graph is a Temperature-Entropy (T-s) diagram. The x-axis is labeled $s \left( \frac{kJ}{kg \cdot K} \right)$ and the y-axis is labeled $T \left( K \right)$. The graph contains a curve starting from point 0 and moving upwards to point 3, then downwards to point 6. The points are labeled sequentially from 0 to 6. The segments between points 0-1, 2-3, and 4-5 are vertical lines indicating isochoric processes. The segments between points 1-2, 3-4, and 5-6 are inclined lines indicating isobaric processes. The points are connected as follows: 0 to 1, 1 to 2, 2 to 3, 3 to 4, 4 to 5, and 5 to 6. The labels "isochore" and "isobare" are written next to the respective segments.
\end{description}

\subsection*{b)}

\begin{align*}
    w_s &= 220 \frac{m}{s} \\
    p_5 &= 0.5 \, \text{bar} \\
    T_5 &= 431.9 \, K \\
    p_6 &= 0.191 \, \text{bar} \\
    \frac{T_6}{T_5} &= \left( \frac{p_6}{p_5} \right)^{\frac{2.4-1}{2.4}} \\
    T_6 &= 328.075 \, K
\end{align*}

``````latex


