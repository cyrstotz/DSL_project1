
``````latex


\section*{Aufgabe 3}

\subsection*{a)}
\begin{align*}
    p_{g2} &= ? \\
    p_{g2} &= \frac{m_{g2}}{V} = \frac{37,4 \text{kg}}{0,03 \text{m}^3} \cdot 1 \frac{\text{J}}{\text{kg} \cdot \text{K}} \cdot 1 \text{K} = 1,4 \text{bar} \\
    p_{g2} &= p_{bar} + \frac{m_{g2}}{V} \cdot \frac{\text{J}}{\text{kg} \cdot \text{K}} \cdot \text{K} = 1,4 \text{bar} \\
    &= 125405 \text{Pa}
\end{align*}

\subsection*{b)}
\begin{align*}
    m_g &= m_g \frac{V_{g1}}{R T_{g1}} = 0,003 \text{kg} \quad R = \frac{1}{M_g} = 0,4 \frac{\text{J}}{\text{kg} \cdot \text{K}} \\
    &= 3,42 \text{g}
\end{align*}

\subsection*{b)}
Im Zustand 2, also im Endzustand, müssen das Eiswasser und das Gas die gleiche Temperatur haben, $T_{g2} = 0^\circ C$, damit keine Wärme mehr ausgetauscht wird und das System "konstant" bleibt.

$p_2$ bleibt gleich wie $p_1$, da sich die äußeren Bedingungen nicht ändern.

\subsection*{c)}
\begin{align*}
    Q_{12} &= m_g (u_2 - u_1) = m_g c_v (T_2 - T_1) = 3,42 \text{g} \cdot 0,623 \frac{\text{J}}{\text{kg} \cdot \text{K}} \cdot (500 \text{K}) \\
    &= -1082,98 \text{J}
\end{align*}

\subsection*{d)}
\begin{align*}
    m_{ew} (u_2 - u_1) &= l Q_{12} \quad \text{mit vorgegebenem } l \text{ (gilt unter anderem)} \\
    u_1 &= 0,6 \cdot (-333,455 \frac{\text{J}}{\text{kg}}) + 0,4 \cdot (-0,65 \frac{\text{J}}{\text{kg}}) = -200,933 \frac{\text{J}}{\text{kg}} \\
    u_2 &= x_{Eis2} \cdot (-333,455 \frac{\text{J}}{\text{kg}}) + (1 - x_{Eis2}) \cdot (-0,65 \frac{\text{J}}{\text{kg}}) \\
    Q_{12} &= 15 \frac{\text{J}}{\text{kg}} \\
    u_2 &= 15 \frac{\text{J}}{\text{kg}} = 200,933 \frac{\text{J}}{\text{kg}} - 185,693 \frac{\text{J}}{\text{kg}} \\
    -185,693 \frac{\text{J}}{\text{kg}} &= x_{Eis2} \cdot (-333,455 \frac{\text{J}}{\text{kg}}) \\
    x_{Eis2} &= 0,555
\end{align*}

``````latex


