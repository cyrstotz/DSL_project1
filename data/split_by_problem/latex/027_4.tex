
``````latex


\section*{Aufgabe 4}

\subsection*{4) a)}

\begin{center}
\textbf{Graph Description:}
\end{center}

The graph is a phase diagram with the y-axis labeled as "P" and the x-axis labeled as "T". There are three main regions labeled "Fest" (solid), "Flüssig" (liquid), and "Gas" (gas). The boundary between the solid and liquid regions is marked as "isotherm" (isothermal) and "isobar" (isobaric). The point where all three phases meet is labeled "TP" (triple point). An arrow labeled "water curves back" points to the boundary between the liquid and gas regions. Another arrow labeled "Steigt bei low (unter TP)" points to the boundary between the solid and liquid regions.

\subsection*{b)}

\begin{equation*}
\text{Energiebilanz um Verdichter:} \quad 0 = \dot{m} (h_e - h_a) + \dot{Q} - \dot{W}_k \quad \text{(stationär)}
\end{equation*}

\begin{itemize}
    \item \text{New Dampf (vollständig verdampft)}: $\Rightarrow$
    \item \text{Dampf-Flüssig Gemisch (8 bar)}: $\Rightarrow$
    \item \text{TAB A11:} \quad h_f = 93.42 \, \text{kJ/kg} \quad h_g = 266.15 \, \text{kJ/kg}
\end{itemize}

\subsection*{c)}

\begin{equation*}
x = h
\end{equation*}

\subsection*{d)}

\begin{equation*}
\epsilon_k = \frac{|\dot{Q}_{zu}|}{|\dot{W}_t|} = \frac{Q_u - Q_{ab}}{28 \, \text{W}}
\end{equation*}

\subsection*{e)}

The temperature would go down to absolute 0 and stop there (theoretically), but it gets harder and harder to cool the closer you get.

```