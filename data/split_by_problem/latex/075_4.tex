
``````latex


\section*{Aufgabe 6a}

\subsection*{a)}

\begin{description}
    \item[Graph Description:] The graph is a Pressure-Temperature (P-T) diagram. The x-axis is labeled \( T \) (Temperature) and the y-axis is labeled \( P \) (Pressure). There is a curve starting from the bottom left, rising steeply, and then leveling off to the right. The curve is labeled with points 1, 2, 3, and 4. Point 1 is at the bottom left, point 2 is slightly above and to the right of point 1, point 3 is further to the right and slightly above point 2, and point 4 is to the far right. The region to the left of the curve is labeled "flüssig" (liquid), and the region to the right of the curve is labeled "gas". There is a horizontal line extending from point 3 to the right, labeled "10K" and "3.1 mbar".
\end{description}

\subsection*{b)}

\[
\dot{M} R134a
\]

\begin{description}
    \item[Graph Description:] The graph is a vertical line with two points labeled 1 and 2. The y-axis is labeled \( \dot{Q}_k \) and the x-axis is labeled \( R_2 = 1 \).
\end{description}

\[
\overline{T}_v = 
\]

\subsection*{c)}

\[
\begin{array}{c|c|c|c|c}
    & T & p & h & S \\
    \hline
    1 & & 1 & & \\
    2 & & x & & \\
    3 & & 8 \text{bar} & & \\
    4 & & 8 \text{bar} & & \\
\end{array}
\]

\subsection*{d)}

\[
\text{Leistungszahl}
\]

\[
\varepsilon = \frac{|\dot{Q}_k|}{|\dot{W}|} = \frac{|\dot{Q}_k|}{|\dot{Q}_{ab} - \dot{Q}_k|}
\]

```