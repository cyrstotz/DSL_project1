
``````latex


\section*{Aufgabe 3}

\subsection*{a)}
\begin{align*}
p_{1\Delta} &= p_{\text{amb}} + \frac{F}{A} + \frac{g(m_k + m_{ew})}{\left(\frac{D}{2}\right)^2 \pi} \\
&= 100{,}000 + \frac{9{,}81(32 + 0{,}4)}{0{,}0025 \pi} \\
&= 100{,}000 + 40{,}054 = 140{,}054 \, \text{Pa}
\end{align*}

\begin{align*}
pV &= nRT \\
n &= \frac{m}{M} \quad \text{und} \quad \bar{R} = 8{,}314 \, \frac{J}{\text{mol} \cdot K} \\
\frac{g \, m_3}{s} &= \frac{p_{1\Delta} V_{1\Delta}}{\bar{R}} + Mg \\
&= \frac{140{,}054 \cdot 0{,}00314 \cdot 50 \cdot 10^{-3}}{8{,}314 \cdot 273{,}15} \\
&= 0{,}003921 \, \text{kg}
\end{align*}

\subsection*{b)}
\begin{align*}
p_{3\Delta} &= 140{,}054 \, \text{Pa}
\end{align*}

Der Druck von außen, dem das Gas entgegen halten muss, ist immer noch gleich groß, also ist auch der Gasdruck noch gleich groß.

\begin{align*}
T_{g,2} &= 0^\circ \text{C}
\end{align*}

Da es immer noch Eiswasser ist, ändert sich die Temperatur des Eiswassers nicht, bis alles Eis geschmolzen ist und da das Gas im Gleichgewicht ist, nimmt es auch 0°C an.

\subsection*{c)}
\begin{align*}
\Delta E &= \Delta U = Q_{2\Delta} - W = Q_{2\Delta} - \int p \, dV = Q_{2\Delta} - \frac{R}{\bar{R}} (T_2 - T_1)
\end{align*}

\begin{align*}
\bar{R}_g &= \frac{8{,}314}{50 \cdot 10^{-3}} = 166{,}28 = c_p - c_v
\end{align*}

\begin{align*}
Q_{2\Delta} &= \Delta U + \int p \, dV = m_g \int_{T_1}^{T_2} c_v \, dT + \int p \, dV \\
&= m_g c_v (T_2 - T_1) + p_{3\Delta} (V_2 - V_1)
\end{align*}

\begin{align*}
V_2 &= \frac{m_g \bar{R}_g T_{g,2}}{p_{3\Delta}} = 0{,}60 + 0{,}103 \, \text{m}^3
\end{align*}

\begin{align*}
Q_{2\Delta} &= -1{,}0827 \, \text{kJ} - 0{,}2853 \, \text{kJ} = -1{,}3673 \, \text{kJ}
\end{align*}

``````latex


\section*{Aufgabe d)}

\begin{align*}
\Delta E &= 0U = \lvert Q_{12} \rvert - W_{12}^0 \\
m_{ew} (h_{12} - u_1) &= \lvert Q_{12} \rvert \\
\end{align*}

\begin{align*}
u_1 &= u_{\text{Flüssig}} + X_{E5,1} (u_{\text{Fest}} - u_{\text{Flüssig}}) = -200,0028 \frac{\text{kJ}}{\text{kg}} \\
u_2 &= \frac{Q_{12}}{m_{ew}} + u_1 = -186,4198 \frac{\text{kJ}}{\text{kg}} \\
X_{E5,2} &= \frac{u_2 - u_{\text{Flüssig}}}{u_{\text{Fest}} - u_{\text{Flüssig}}} = 0,558198 \approx 0,559
\end{align*}

\text{Alle $u_1$ und $u_2$ bei 1,4 bar:}
\begin{align*}
u_{\text{Fest}} &= -333,458 \frac{\text{kJ}}{\text{kg}} \\
u_{\text{Flüssig}} &= -0,045 \frac{\text{kJ}}{\text{kg}}
\end{align*}

``````latex


