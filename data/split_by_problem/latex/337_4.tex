
``````latex


\section*{Aufgabe 4}

\subsection*{a)}

\begin{itemize}
    \item \textbf{Graph 1:} A graph with the y-axis labeled \( P \) [bar] and the x-axis labeled \( T \) [K]. The graph shows a dome-shaped curve with a peak labeled "1500 kJ". There are two horizontal dashed lines intersecting the curve, one at the top and one at the bottom. The bottom intersection is labeled "2".
    \item \textbf{Graph 2:} A similar graph with the y-axis labeled \( P \) [bar] and the x-axis labeled \( T \) [K]. The graph shows a dome-shaped curve with a peak. There is a horizontal dashed line intersecting the curve at the bottom, labeled "2". There is also a shaded region to the right of the curve.
\end{itemize}

\[
\begin{array}{|c|c|c|}
\hline
P & T & v \\
\hline
1 & p_u & \\
2 & p_{Boon} & -22^\circ C \\
3 & 8 \text{ bar} & \\
4 & 8 \text{ bar} & \\
\hline
\end{array}
\]

\[
a_{44} = 0
\]

\[
p_u = 5 \text{ mbar} + p_{ipp}
\]

\[
T_i - 10 \text{ K} = T_{\text{subimationspunkt}}
\]

\[
x_2 = 1
\]

\[
x_3 = 1
\]

\[
x_4 = 0
\]

\[
T_i - T_3 = 6 \text{ K}
\]

\begin{itemize}
    \item \textbf{Graph 3:} A graph with the y-axis labeled \( P \) [bar] and the x-axis labeled \( T \) [K]. The graph shows a dome-shaped curve with a peak. There is a horizontal dashed line intersecting the curve at the bottom, labeled "2". There is also a shaded region to the right of the curve.
    \item \textbf{Graph 4:} A graph with the y-axis labeled \( P \) [bar] and the x-axis labeled \( T \) [K]. The graph shows a dome-shaped curve with a peak. There are two horizontal lines intersecting the curve, one at the top and one at the bottom. The bottom intersection is labeled "2". The top intersection is labeled "isotrop". There is a point labeled "4" on the top line and a point labeled "3" on the bottom line.
\end{itemize}

\subsection*{b)}

\[
\dot{m}_{R134a} = ?
\]

\[
\dot{E}_{\text{stat}} = 0
\]

\[
\frac{dE}{dt} = \dot{m}_R (h_2 - h_3) + \dot{Q} - \dot{W}_k
\]

\[
\dot{W}_k = \dot{m}_{R134a} (h_2 - h_3)
\]

\[
h_2 (8 \text{ bar}, x_2 = 1) = h_g (8 \text{ bar}) \quad \text{TAB A14} \quad 264.15 \frac{\text{kJ}}{\text{kg}}
\]

\[
h_3 (8 \text{ bar}, S_2) \Rightarrow \text{interpolieren @ 8 bar mit } S_2 :
\]

\[
h_3 = \frac{h(S_x) - h(S_y)}{S_x - S_y} (S_2 - S_y) + h_y
\]

\[
S_2 (8 \text{ bar}, x = 1) \quad \text{TAB A14} \quad 0.9066 \frac{\text{kJ}}{\text{kg K}} = S_3
\]

``````latex


\section*{c)}

Drossel $\sim$ isenthalp, isotherm

\[
x_4 = 0 \quad p_4 = 8 \text{ bar} \quad \sim \quad h_4 = 93.42 \frac{\text{kJ}}{\text{kg}}
\]

\[
T_{AB} = 31.33
\]

\[
T_4 = 31.33
\]

\[
h_1 = 92.42, \quad p_u = 6 \text{ bar}
\]

Interpolieren @ TAB 4.12

\[
h = \frac{\Phi(p_1) - \Phi(p_2)}{p_1 - p_2} \left( p - p_2 \right) + \Phi(p_2)
\]

\[
\text{(Will das also unten für } \Phi(p_2) \text{)}
\]

\[
p = \frac{h - \Phi(p_2)}{\Phi(p_1) - \Phi(p_2)} (p_1 - p_2) + p_2
\]

\section*{d)}

\[
E_K = \frac{\dot{Q}_{zu}}{\dot{V}_t \dot{W}_K}
\]

\section*{e)}

\[
Q = c \cdot m \cdot (T_2 - T_1) \quad \text{- Bolle Temp würde Abnehmen}
\]

\[
\text{Phase Veränderung}
\]

```