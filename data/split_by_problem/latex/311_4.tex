
``````latex


\section*{Aufgabe 4}

\subsection*{a)}

\begin{description}
    \item[Graph 1:] 
    The graph is a Pressure-Volume (P-V) diagram. The x-axis is labeled as $V$ and the y-axis is labeled as $P$. The graph shows a closed loop with four distinct points labeled 1, 2, 3, and 4. The region inside the loop is labeled "Nassdampf". The area to the left of the loop is labeled "Flüssig" and the area to the right is labeled "Dampf". The loop starts at point 1, moves to point 2, then to point 3, and finally to point 4 before returning to point 1. The path from point 1 to point 2 is a curve moving upwards and to the right. The path from point 2 to point 3 is a curve moving downwards and to the right. The path from point 3 to point 4 is a curve moving downwards and to the left. The path from point 4 to point 1 is a curve moving upwards and to the left.
    
    \item[Graph 2:] 
    The graph is another Pressure-Volume (P-V) diagram similar to the first one. The x-axis is labeled as $V$ and the y-axis is labeled as $P$. The graph shows a closed loop with four distinct points labeled 1, 2, 3, and 4. The region inside the loop is labeled "Nassdampf". The area to the left of the loop is labeled "Flüssig" and the area to the right is labeled "Dampf". The loop starts at point 1, moves to point 2, then to point 3, and finally to point 4 before returning to point 1. The path from point 1 to point 2 is a curve moving upwards and to the right. The path from point 2 to point 3 is a curve moving downwards and to the right. The path from point 3 to point 4 is a curve moving downwards and to the left. The path from point 4 to point 1 is a curve moving upwards and to the left. Additionally, there is a label "Isobare" near the path from point 3 to point 4.
    
    \item[Graph 3:] 
    The graph is a Pressure-Volume (P-V) diagram. The x-axis is labeled as $V$ and the y-axis is labeled as $P$. The graph shows a closed loop with four distinct points labeled 1, 2, 3, and 4. The region inside the loop is labeled "Nassdampf". The area to the left of the loop is labeled "Flüssig" and the area to the right is labeled "Dampf". The loop starts at point 1, moves to point 2, then to point 3, and finally to point 4 before returning to point 1. The path from point 1 to point 2 is a curve moving upwards and to the right. The path from point 2 to point 3 is a curve moving downwards and to the right. The path from point 3 to point 4 is a curve moving downwards and to the left. The path from point 4 to point 1 is a curve moving upwards and to the left. Additionally, there are labels "P3", "P4", and "100bar" near the path from point 3 to point 4.
\end{description}

``````latex


b) \quad \text{1HS Verdichter}

\[
0 = \dot{m} (h_2 - h_3) \quad \dot{W}_k
\]

\text{(There is a diagram here showing a process with two curves. The first curve starts at a point labeled \( \dot{W}_k \) and goes up and down, ending at a point labeled \( s_2 = s_3 \). The second curve starts at a point labeled \( \dot{W}_k \) and goes up and down, ending at a point labeled \( \dot{m} \).)}

\[
\dot{m} = \frac{\dot{W}_k}{h_2 - h_3}
\]

\[
s_2 = s_3
\]

\[
p_1 = p_2
\]

\[
x_4 = 0 \quad p_4 = 8 \text{bar} \quad t_4 = 31-33^\circ \text{C}
\]

\text{1HS Drossel}

\[
\dot{m} (h_4 - h_1) = 0 \quad h_4 = h_1
\]

``````latex


\section*{Aufgabe 4}

\subsection*{c)}
\[
h_4 = h_{A} = 93.42 \frac{\text{kJ}}{\text{kg}}
\]
\[
p_2 = p_{A} = 1.2182 \text{ bar}
\]
\[
\Rightarrow x_{A} = \frac{h_4 - h_f(p_4)}{h_g(p_4) - h_f(p_4)}
\]
\[
= 0.3375
\]

\subsection*{d)}
\[
\epsilon_k = \frac{\left| \dot{Q}_{\text{zu}} \right|}{\left| \dot{W} \right|} = \frac{\left| \dot{Q}_{\text{zu}} \right|}{\left| \dot{Q}_{\text{ab}} - \dot{Q}_{\text{zu}} \right|}
\]
\[
= \frac{\left| \dot{Q}_{\text{zu}} \right|}{\left| \dot{Q}_{\text{ab}} \right| - \left| \dot{Q}_{\text{zu}} \right|}
\]
\[
Q_{\text{zu}} = \dot{m}_{\text{TRB}} (h_2 - h_1) = 156.28 \text{ W}
\]
\[
h_1 = h_f(p_1) + x_1 (h_g - h_f) = 93.42 \frac{\text{kJ}}{\text{kg}}
\]

\[
4 \frac{\text{kg}}{\text{h}} = \frac{4 \frac{\text{kg}}{\text{h}} \cdot 60 \frac{\text{min}}{\text{h}}}{60 \frac{\text{s}}{\text{min}}} = \frac{1 \frac{\text{kg}}{300 \text{s}}}
\]

``````latex


\[
\dot{Q}_{ab} = \dot{m}_{rev} (h_2 - h_3)
\]

\[
s_2 = s_3 = 0.9351 \frac{kJ}{kg \cdot K}
\]

\[
h_3 = 273.16 \frac{kJ}{kg} - 204.15 \frac{kJ}{kg} \left(0.9351 - 0.8066\right) \frac{2.04 \frac{kJ}{kg}}{kg \cdot K}
\]

\[
= 0.9351 - 0.8066
\]

\[
= 272.85 \frac{kJ}{kg}
\]

\[
\dot{Q}_{ab} = -199.47 \, W
\]

\[
\dot{S}_x = 3.6184 \, \underline{\phantom{3.6184}}
\]

```