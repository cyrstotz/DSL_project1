
``````latex


\section*{Aufgabe 4}

\subsection*{a)}

\begin{description}
    \item[Graph 1:] The graph is a plot with the vertical axis labeled \( p[\text{bar}] \) and the horizontal axis labeled \( T[\text{K}] \). The plot shows a wavy line that starts from the bottom left, rises and falls multiple times, and ends at the bottom right. There is also a shaded region in the upper left corner of the graph.
\end{description}

\begin{itemize}
    \item[1-2:] \( T \uparrow \), \( p \downarrow \)
    \item[2-3:] \( S_2 = S_3 \), \( p \uparrow \)
    \item[3-4:] isobar, \( T \uparrow \), \( p = \text{const.} \)
    \item[4-1:] isenthalpe, \( h_4 = h_1 \), \( p \downarrow \), \( T = \text{const.} \)
\end{itemize}

\begin{description}
    \item[Graph 2:] The graph is a plot with the vertical axis labeled \( p[\text{bar}] \) and the horizontal axis labeled \( T[\text{K}] \). The plot shows a closed loop with four points labeled 1, 2, 3, and 4. The path from point 1 to point 2 is curved and labeled \( T = \text{const} \). The path from point 2 to point 3 is straight and labeled \( p = \text{const} \). The path from point 3 to point 4 is curved and labeled \( T = \text{const} \). The path from point 4 to point 1 is straight and labeled \( p = \text{const} \).
\end{description}

``````latex

\section*{Aufgabe 4}

\subsection*{b)}
\begin{equation*}
    \dot{m} (h_2 - h_3) + \dot{Q} - \dot{W} = 0
\end{equation*}
\begin{equation*}
    \dot{m} = \frac{\dot{W}}{h_2 - h_3}
\end{equation*}
\begin{equation*}
    s_2 = s_3
\end{equation*}
\begin{equation*}
    S_3 = S_2 = 0.9006
\end{equation*}

\subsection*{c)}

\subsection*{d)}

\subsection*{e)}
\begin{equation*}
    e_k = \frac{\dot{Q}_{zu}}{\left| \dot{W}_c \right|} = \frac{\left| \dot{Q}_k \right|}{\dot{W}_k}
\end{equation*}
\begin{equation*}
    \dot{Q}_k = - \dot{m} (
\end{equation*}

```