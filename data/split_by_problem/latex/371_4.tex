
``````latex


\section*{Aufgabe 4}

\subsection*{a)}

\begin{center}
\textbf{Graph Description:}

The graph is a pressure-temperature ($p$-$T$) diagram. The x-axis is labeled $T [K]$ and the y-axis is labeled $p [\text{bar}]$. 

- The graph starts with a line rising from the origin, representing the phase boundary between the liquid and vapor phases.
- The line curves upwards and then downwards, forming a dome shape, which represents the saturation curve.
- Inside the dome, the region is labeled as "Nassdampf" (wet steam).
- To the left of the dome, the region is labeled as "gesättigte Flüssigkeit" (saturated liquid).
- To the right of the dome, the region is labeled as "gesättigter Dampf" (saturated vapor).
- A horizontal line is drawn inside the dome, representing an isotherm, and is labeled as "isotherm".
- The left end of the isotherm is labeled as "1" and the right end as "2".
- Another horizontal line is drawn above the dome, representing another isotherm, and is labeled as "isotherm".
- The left end of this isotherm is labeled as "3".

\end{center}

\subsection*{b)}

\textbf{1. HS am Verdichter}

\[
0 = \dot{m} (h_2 - h_3) + \dot{W}_k
\]

\[
\dot{m} = \frac{\dot{W}_k}{h_2 - h_3} = \frac{28 \, \text{W}}{h_2 - h_3}
\]

\[
(h_2 - h_3) \text{ adiabt-reversibel} = \text{isentrop} \rightarrow s_2 = s_3 = s_g (p = 8 \, \text{bar}) = 0.906 \, \frac{\text{kJ}}{\text{kgK}} \quad (\text{TAB A1.1})
\]

\[
h_2 = h_g (\text{8bar}) = 264.7 \, \frac{\text{kJ}}{\text{kg}} \quad (\text{TAB A1.1})
\]

\[
h_3 = h_2 - s_3
\]

```