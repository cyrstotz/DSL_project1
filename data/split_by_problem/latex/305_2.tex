
``````latex


\section*{2) a)}

\begin{description}
    \item[Graph 1:] The graph is a $T$-$s$ diagram. The x-axis is labeled $s$ and the y-axis is labeled $T$. There are several lines drawn on the graph:
    \begin{itemize}
        \item A line starting from point $O$ and moving upwards to the right, labeled $1_3$.
        \item A line starting from point $1_3$ and moving upwards to the right, labeled $1$.
        \item A line starting from point $1$ and moving upwards to the right, labeled $2$.
        \item A line starting from point $O$ and moving upwards to the left, labeled $2_3$.
        \item A line starting from point $2_3$ and moving upwards to the left, labeled $2_k$.
        \item A line starting from point $2_k$ and moving upwards to the left, labeled $2$.
        \item Several isochore lines are drawn parallel to each other, labeled "isochore".
        \item A line labeled $0.13 \text{ bar}$ is drawn parallel to the isochore lines.
    \end{itemize}
    
    \item[Graph 2:] The graph is a $T$-$s$ diagram. The x-axis is labeled $s$ and the y-axis is labeled $T$. There are several lines drawn on the graph:
    \begin{itemize}
        \item A line starting from point $O$ and moving upwards to the right, labeled $1_3$.
        \item A line starting from point $1_3$ and moving upwards to the right, labeled $1$.
        \item A line starting from point $1$ and moving upwards to the right, labeled $2$.
        \item A line starting from point $O$ and moving upwards to the left, labeled $2_3$.
        \item A line starting from point $2_3$ and moving upwards to the left, labeled $2_k$.
        \item A line starting from point $2_k$ and moving upwards to the left, labeled $2$.
        \item Several isochore lines are drawn parallel to each other, labeled "isochore".
        \item A line labeled $0.019 \text{ bar}$ is drawn parallel to the isochore lines.
    \end{itemize}
\end{description}

\section*{b)}

\[
h_5 + \frac{w_5^2}{2} = h_6 + \frac{w_6^2}{2}
\]

\[
\text{Düse ist isentrop} \Rightarrow s_5 = s_6 \Rightarrow n = k = 1.4
\]

\[
T_6 = T_5 \left( \frac{p_6}{p_5} \right)^{\frac{k-1}{k}} = 431.9 \text{ K} \left( \frac{0.191}{0.5} \right)^{\frac{0.4}{1.4}} = \underline{328.07 \text{ K}}
\]

\[
\Rightarrow \frac{w_6^2}{2} = h_5 - h_6 + \frac{w_5^2}{2}
\]

\[
= c_p \left( T_5 - T_6 \right) + \frac{w_5^2}{2} = 1.005 \frac{\text{kJ}}{\text{kg} \cdot \text{K}} \left( 431.9 - 328.07 \text{ K} \right) + \frac{(220 \frac{\text{m}}{\text{s}})^2}{2}
\]

\[
= 104.45 \text{ kJ} + 24.2 \text{ kJ} = 128.65 \text{ kJ}
\]

\[
\Rightarrow w_6 = \sqrt{2 \cdot 128.65 \text{ kJ}} = \underline{507.24 \frac{\text{m}}{\text{s}}}
\]

``````latex


\section*{c)}

\begin{align*}
\Delta_{ex, s12} &= h_2 - h_0 - T_0 (s_2 - s_0) + \frac{w_2^2}{2} - \frac{w_0^2}{2} \\
&= c_p (T_2 - T_0) - T_0 \left( c_p \ln \frac{T_2}{T_0} \right) - R \ln \left( \frac{p_2}{p_0} \right) + \frac{w_2^2}{2} - \frac{w_0^2}{2} \\
&= 1.006 \frac{kJ}{kg \cdot K} \left( 328.07 K - 243.15 K \right) - 243.15 K \left( 1.006 \frac{kJ}{kg \cdot K} \ln \frac{328.07 K}{243.15 K} \right) \\
&\quad + \frac{(56.72 \frac{m}{s})^2}{2} - \frac{220 \frac{m}{s}}{2} \\
&= 12.16 kJ + 128.55 kJ - 24.26 kJ \\
&= 116.6 \underline{kJ}
\end{align*}

\section*{d)}

\begin{align*}
\dot{e}_{x, \nu} &= \dot{e}_{x, s12} + \left( \frac{T_0}{T_3} \right) \dot{q} \\
\dot{s}_{ex} &= (s_0 - s_1) + \frac{\dot{q}}{T_3} = c_p \ln \left( \frac{T_0}{T_0} \right) - R \ln \left( \frac{p_0}{p_0} \right) + \frac{\dot{q}}{T_3} \\
&= 1.006 \frac{kJ}{kg \cdot K} \ln \left( \frac{243.15 K}{328.07 K} \right) + \frac{1.195 \frac{kJ}{kg}}{128 sK} = 0.6257 \frac{kJ}{kg \cdot K}
\end{align*}

\begin{align*}
\dot{E}_{x, \nu} &= \dot{s}_{ex} \cdot T_0 \\
\dot{e}_{x, \nu} &= \dot{s}_{ex} \cdot T_0 = 152.14 \frac{kJ}{kg}
\end{align*}

``````latex


