
``````latex


\section*{Aufgabe 1}

\subsection*{a)}

\begin{itemize}
    \item Ein Diagramm mit der y-Achse beschriftet als $p(\text{Pa})$ und der x-Achse beschriftet als $T(\text{K})$. 
    \item Es gibt drei Phasenbereiche, die durch Linien getrennt sind:
    \begin{itemize}
        \item Ein Bereich links oben, der als "fest" beschriftet ist.
        \item Ein Bereich rechts oben, der als "flüssig" beschriftet ist.
        \item Ein Bereich unten, der als "gas" beschriftet ist.
    \end{itemize}
    \item Eine grüne vertikale Linie, die von der x-Achse nach oben verläuft und mit "i" markiert ist.
    \item Eine grüne horizontale Linie, die von der y-Achse nach rechts verläuft und mit "ii" markiert ist.
    \item Eine blaue Kurve, die von links unten nach rechts oben verläuft und die Phasenbereiche trennt.
\end{itemize}

\subsection*{b)}

\begin{itemize}
    \item Sie würde weiterhin sinken.
\end{itemize}

\subsection*{c)}

\begin{align*}
    h_2 &= h_f + x \cdot (h_{fg} - h_f) \\
    h_q &= h_f \left( \frac{8 \, \text{bar}}{9,3 \, \text{kg}} \right) \left( \frac{T_{ab} - A - 12}{T_{ab} - A - 12} \right) \\
    x &= \frac{h_q - h_{f}}{h_{fg} - h_{f}} \\
    x &= \frac{h_q - h_{f}}{h_{fg} - h_{f}} \quad \text{Wert für } h_{f(\text{flüssig})} \text{ und } h_{f(\text{gas})} \text{ finden und einsetzen}
\end{align*}

``````latex


\section*{Solution}

\subsection*{a)}
\begin{equation*}
    \epsilon_u = \frac{\dot{Q}_{zu}}{(W+1)} = \frac{\dot{Q}_k}{28W}
\end{equation*}

\subsection*{b)}
\begin{itemize}
    \item Druck in $r_2$ Erhöht
    \item Energieübergang zur statischen Kompression
\end{itemize}

```