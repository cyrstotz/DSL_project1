
``````latex


\section*{Aufgabe 4:}

\subsection*{a)}

\begin{description}
    \item[Graph Description:] The graph is a plot with the vertical axis labeled \( p(\text{bar}) \) and the horizontal axis labeled \( T [K] \). The vertical axis has an arrow pointing upwards, and the horizontal axis has an arrow pointing to the right. There is a curve starting from point 1 on the upper left, moving downwards and to the right, and ending at point 2 on the lower right. This curve is labeled "Isotherme". There are two segments marked on the curve: segment \( i \) from point 1 to an intermediate point, and segment \( ii \) from the intermediate point to point 2. Segment \( i \) is indicated with a solid line, while segment \( ii \) is indicated with a dashed line. Point 1 is labeled at the top left of the curve, and point 2 is labeled at the bottom right of the curve. The intermediate point is labeled with a small circle and the letter \( G \).
\end{description}

\begin{itemize}
    \item[i] ist isobar \(\Delta T < 0\)
    \item[ii] ist isotherm
\end{itemize}

``````latex


\section*{b)}

\textit{mit R134a}

Energiebilanz um Verdichter $\rightarrow$ isentrop

\[
\dot{m} (h_1 - h_3) - \dot{W} = 0
\]

\[
p_2 = p_1
\]

\[
h_4 = h_1
\]

\[
S_2 = S_3
\]

\[
h_1 = 63.53 \frac{kJ}{kg}
\]

\textit{use $S_3$ to interpolate $h_3$ in A9}

\[
p_4 = p_3
\]

\[
S_2 = \text{(vorher gemacht)}
\]

\[
S_2 = \text{interpolieren mit } p_x
\]

\[
p_x = 2.2688 \text{ bar}
\]

\[
S_2 = \frac{2.2688 - 2.25}{2.2688 - 2.25} \cdot (0.9566 - 0.9636) + 0.9636
\]

\[
S_2 = 0.9632 = S_3
\]

\section*{c)}

\textit{Energ. v. drossel}

\[
t_2 = -22^\circ C
\]

\[
\dot{m} = 4 \frac{kg}{h}
\]

\[
h_4 = h_1
\]

\[
p_1 = p_2
\]

\[
p_2 = 2.2688 \text{ bar} = p_1
\]

\[
T_4 = 19.45^\circ C
\]

\textit{aus Tab.}

\[
h_1 = 63.53 \frac{kJ}{kg}
\]

\[
h_1 = 63.53 \frac{kJ}{kg}
\]

``````latex


c)
\[
63.53 = 79.82 \left( \frac{7 - x}{7} \right) + \frac{227.32 \times x}{241.24}
\]

\[
43.67 = 20.14 \times x + 227.32 \times x
\]

\[
x = \frac{43.67}{247.46} = 0.197
\]

``````latex


\section*{d)}

\begin{equation*}
\varepsilon = \frac{\dot{Q}_k}{\dot{w}_k} \quad \text{(aus def.)}
\end{equation*}

\begin{equation*}
\varepsilon = \frac{0.983745 \cdot 10^3}{28} = 7.052
\end{equation*}

\begin{equation*}
\dot{m} (h_1 - h_2) = -\dot{Q}_k
\end{equation*}

\begin{equation*}
\frac{4}{3600} \left(3.353 - 2.41424\right) = -\dot{Q}_k \quad \text{(aus c)}
\end{equation*}

\begin{equation*}
\dot{Q}_k = 6.193745 \, \text{kW}
\end{equation*}

\section*{e)}

\begin{equation*}
\text{Temperatur würde fallen da } p < p_{\text{trippel}}
\end{equation*}

```