
``````latex


\section*{Aufgabe 9}

\subsection*{a)}

\begin{description}
    \item[Graph 1:] A wavy line is drawn, labeled "Phasendiagramm" at the top. The x-axis is labeled "T (°C)" and the y-axis is labeled "P (mbar)". The graph is divided into three regions: "Fest" on the left, "isobar" in the middle, and "Flüssig" on the right.
    \item[Graph 2:] A phase diagram with a curved line starting from the bottom left and rising to the right. The x-axis is labeled "T (°C)" and the y-axis is labeled "P (mbar)". The diagram is divided into three regions: "Fest" on the left, "Flüssig" on the right, and "Gas" above the curved line. There are three points marked: point 1 on the right, point 2 in the middle, and point 3 on the left. Point 1 and point 2 are connected by a horizontal line labeled "isobar". Point 2 and point 3 are connected by a vertical line labeled "isotherm". The pressure at point 3 is labeled "5 mbar" and the temperature at point 3 is labeled "10 K".
\end{description}

\subsection*{b)}

\begin{align*}
    \dot{W}_k &= 28 \, \text{W} \\
    T_2 &= T_1 = T_i - 6^\circ \\
    T_i &= -10^\circ \text{C} \\
    T_2 &= \boxed{-76^\circ \text{C}} \\
    0 &= \dot{m} (h_e - h_a) + \dot{W}_k \\
    \dot{W}_k &= \dot{m} \frac{h_a}{h_3 - h_2}
\end{align*}

\begin{description}
    \item[Table:] A table labeled "Tab A-10: Zustand" with three columns labeled "2", "3", and "8 bar". The rows are labeled "p", "x", and "T". The values in the table are:
    \begin{itemize}
        \item Zustand 2: $p = 7.5748 \, \text{bar}$, $x = 1$, $T = -76^\circ \text{C}$
        \item Zustand 3: $p = 8 \, \text{bar}$
    \end{itemize}
    \item[Additional Information:] $S_g = 0.9298 \, \frac{kJ}{kgK} \times 0.9298 \, \frac{kJ}{kgK}$
\end{description}

``````latex


\section*{Student Solution}

\textbf{Zustand 3:} $p = 8 \, \text{bar}$ \quad $s = 0.9298 \, \frac{\text{kJ}}{\text{kgK}}$

\[
\Rightarrow \text{überhitz}
\]

\textbf{A-72:} \quad $p = 8 \, \text{bar}$

\[
h_3 = 264.15 - 273.66 \left( \frac{0.9298 - 0.9066}{0.9066 - 0.9298} \right) + 264.75
\]

\[
= \boxed{277.37 \, \frac{\text{kJ}}{\text{kg}}}
\]

\[
h_2 = h_g (T = -76^\circ \text{C}) - 237.74 \, \frac{\text{kJ}}{\text{kg}}
\]

\[
\frac{28 \, \text{kW}}{h_3 - h_2} = 0.83 \, \frac{\text{g}}{\text{s}} = \boxed{3 \, \frac{\text{kg}}{\text{h}}} \quad \text{in } R_{134a}
\]

\textbf{c)}

\[
\begin{array}{c|c|c}
 & z & z \\
p & 8 \, \text{bar} & 8 \, \text{bar} \\
x & 0 & 0 \\
h & 277.37 \, \frac{\text{kJ}}{\text{kg}} & h_f (8 \, \text{bar}) = 93.42
\end{array}
\]

\textbf{aus Tab A-10 bei} $p = p_2 = 1.5248 \, \text{bar}$

\[
x_1 = \frac{h_1 - h_f}{h_g - h_f} = \boxed{0.83} \quad \boxed{0.308}
\]

``````latex


\section*{d)}
\begin{equation*}
    E_k = \frac{| \dot{Q}_{zu} |}{| \dot{W}_{el} |} = \frac{| \dot{Q}_{zu} |}{| \dot{Q}_{ab} - \dot{Q}_{zu} |}
\end{equation*}

\begin{equation*}
    \dot{Q}_{zu} = \dot{m} (h_2 - h_1)
\end{equation*}

\begin{equation*}
    \dot{Q}_{ab} = \dot{m} (h_a - h_3)
\end{equation*}

\begin{equation*}
    h_1 = 93.42 \frac{kJ}{kg} \quad h_2 = 237.74 \frac{kJ}{kg}
\end{equation*}

\begin{equation*}
    h_3 = 277.37 \frac{kJ}{kg} \quad h_a = \cancel{93.92 \frac{kJ}{kg}}
\end{equation*}

\begin{equation*}
    E_k = \frac{| (h_2 - h_1) |}{| (h_a - h_3) - (h_2 - h_1) |} = \frac{144.32}{77.99 - 144.32}
\end{equation*}

\begin{equation*}
    \boxed{E_k = 4.299}
\end{equation*}

\section*{e)}
Die Temperatur würde weiter gesenkt werden und das Wasser wäre immer im festen Zustand, da die Sublimationslinie nie unterschritten wird.
```