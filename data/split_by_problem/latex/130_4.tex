
``````latex


4)

a)

\begin{description}
    \item[Graph 1:] The graph is a plot with the vertical axis labeled as \( p \) and the horizontal axis labeled as \( T \) (k). There are four points labeled 1, 2, 3, and 4. The points 1 and 2 are connected by a horizontal line labeled "Isobar". The points 3 and 4 are also connected by a horizontal line labeled "Isobar". There is a curved line passing through points 1, 4, 3, and 2, forming a loop. The region inside the loop is labeled "Isobar".
\end{description}

\begin{description}
    \item[Graph 2:] The graph is a plot with the vertical axis labeled as \( p \) and the horizontal axis labeled as \( T \). There are four points labeled 1, 2, 3, and 4. The points 1 and 2 are connected by a horizontal line labeled "flüssigkeit". The points 3 and 4 are connected by a horizontal line labeled "gesättigter dampf". There is a curved line passing through points 1, 4, 3, and 2, forming a loop. The region inside the loop is labeled "flüssigkeit" and "gesättigter dampf".
\end{description}

b)

\[
\dot{m}(h_2 - h_1) = \dot{W}_{\text{turb}}
\]

\[
\dot{m} = \frac{\dot{W}_{\text{turb}}}{h_2 - h_1}
\]

\[
p_3 = p_4 = 8 \text{bar}
\]

``````latex

\section*{c)}

\begin{equation*}
\epsilon_k = \frac{\dot{Q}_{zu}}{W_{t1}} - \frac{\dot{Q}_{zw}}{Q}
\end{equation*}

```