
``````latex


\section*{A2}

\subsection*{a)}

\begin{center}
\textbf{Graph Description:}
\end{center}

The graph is a plot with the vertical axis labeled \( T(K) \) and the horizontal axis labeled \( S \left( \frac{kJ}{kgK} \right) \). The plot contains a curve that starts at point 0 and moves through points 1, 2, 3, 4, 5, and 6. The curve has the following characteristics:
- From point 0 to point 1, the curve is a straight line.
- From point 1 to point 2, the curve is a steep upward line.
- From point 2 to point 3, the curve is a horizontal line.
- From point 3 to point 4, the curve is a steep upward line.
- From point 4 to point 5, the curve is a horizontal line.
- From point 5 to point 6, the curve is a steep upward line.

The points are labeled as follows:
- Point 0 is at the origin.
- Point 1 is at a higher temperature than point 0.
- Point 2 is at a higher temperature than point 1.
- Point 3 is at a higher temperature than point 2.
- Point 4 is at a higher temperature than point 3.
- Point 5 is at a higher temperature than point 4.
- Point 6 is at a higher temperature than point 5.

Below the graph, there is a table with the following values:

\[
\begin{array}{ccccccc}
0 & 1 & 2 & 3 & 4 & 5 & 6 \\
P & 0.191 & = & = & = & = & = \\
T & -30 & & & & & & \\
\end{array}
\]

\subsection*{b)}

\textbf{7. HS Gleichung:}

\[
0 = m_{ges} \left[ h_5 - h_6 + \frac{W_5^2}{2} - \frac{W_6^2}{2} \right]
\]

\[
0 = h_5 - h_6 + \frac{W_5^2}{2} - \frac{W_6^2}{2}
\]

\[
W_5^2 = 2 \left[ h_5 - h_6 + \frac{W_5^2}{2} \right]
\]

\[
W_5^2 = 2 c_p \left( T_5 - T_6 \right) + W_5^2
\]

\[
T_6 = \frac{T_s}{\left( \frac{T_s}{P_s} \right)^{\frac{k-1}{k}}} = 9 + 37.9K = 328K
\]

\[
W_6 = \sqrt{2 \cdot 7.006 \frac{kJ}{kgK} \left( 37.9K - 328K \right) + 220 \frac{m}{s}}
\]

\[
= 220.9 \frac{m}{s}
\]

``````latex


\section*{Problem c}

\[
e_{x, \text{ser}, s} = h_6 - h_0 - T_0 (s_6 - s_0) + \frac{1}{2} v_6^2
\]

\[
h_6 - h_0 = c_p (T_6 - T_0) = 97.9 \frac{\text{kJ}}{\text{kg}}
\]

\[
s_6 - s_0 = c_p \ln \left( \frac{T_6}{T_0} \right) = 0.3373 \frac{\text{kJ}}{\text{kg K}}
\]

\[
c_p (340 \text{K} - 293.15 \text{K}) = 97.9 \frac{\text{kJ}}{\text{kg}}
\]

\[
e_{x, \text{ser}, s} = 97.9 \frac{\text{kJ}}{\text{kg}} - 293.15 \text{K} \cdot 0.3373 \frac{\text{kJ}}{\text{kg K}} + \frac{1}{2} \cdot 5 \frac{\text{m}^2}{\text{s}^2}
\]

\[
= 130 \frac{\text{kJ}}{\text{kg}}
\]

\[
e_{x, \text{ser}, 0} = \cancel{0} + \frac{1}{2} \cdot 200 \frac{\text{m}^2}{\text{s}^2} = 20 \frac{\text{kJ}}{\text{kg}}
\]

\[
\Delta e_{x, \text{ser}} = 130 \frac{\text{kJ}}{\text{kg}} - 20 \frac{\text{kJ}}{\text{kg}} = 110 \frac{\text{kJ}}{\text{kg}}
\]

\section*{Problem d}

\[
0 = \dot{m} (s_0 - s_6) + \frac{Q_0}{T_0} + S_{\text{env}}
\]

\[
S_{\text{env}} = s_6 - s_0 - \frac{Q_0}{T_0}
\]

\[
= c_p \ln \left( \frac{T_6}{T_0} \right) - \frac{17.15 \frac{\text{kJ}}{\text{kg}}}{293 \text{K}} = -0.56 \frac{\text{kJ}}{\text{kg K}}
\]

\[
T_6 = 390 \text{K}
\]

\[
T_0 = 293.15 \text{K}
\]

``````latex


