
``````latex


\section*{Problem 2}

\begin{itemize}
    \item[(a)] $p_5 = p_6 \quad p_c = 1$
    
    \[
    \frac{T_6}{T_5} = \left( \frac{p_6}{p_5} \right)^{\frac{0.4}{1.4}}
    \]
    
    \[
    T_6 = \left( \frac{p_6}{p_5} \right)^{\frac{0.4}{1.4}} T_5
    \]
    
    \[
    T_6 = 328.07 K
    \]
    
    \[
    w_{5/6} = \frac{R \left( T_6 - T_5 \right)}{-0.4} = \underline{\hspace{2cm}}
    \]
    
    \[
    7.87 \times 71.498 \frac{kJ}{kg}
    \]
    
    \[
    \frac{R}{2}
    \]
    
    \[
    M = 28.97 \frac{kg}{kmol}
    \]
    
    \[
    R = 28 + \frac{1}{\log(1)}
    \]
    
    \item[(b)] 
    
    \[
    w_e = h_e - h_a + \frac{w_e^2 - w_a^2}{2} \bigg|_2
    \]
    
    \[
    w_A^2 = 2h_5 - 2h_6 + w_e^2 - 2w_e
    \]
    
    \[
    w_A' = 2c_p \left( T_5 - T_6 \right) + \left( 2000 \frac{m}{s} \right)^2 2w_e
    \]
    
    \[
    w_s = 3.16 \frac{m}{s}
    \]
    
\end{itemize}

\section*{Graph Description}

The graph is a line plot with five points labeled 1 through 5. The x-axis is labeled with a point at the origin (0,0) and extends to the right. The y-axis extends upwards from the origin. The points are connected as follows:

- Point 1 is at the origin.
- Point 2 is directly above point 1 on the y-axis.
- Point 3 is to the right of point 2 and slightly higher, indicating an increase in both x and y values.
- Point 4 is to the right of point 3 but at a lower y value, indicating a decrease in y but an increase in x.
- Point 5 is directly below point 4 on the x-axis, indicating a decrease in y value.

The graph is labeled with "Isotherm" near point 3.

``````latex

\section*{c)}

\[
\Delta e_{x, \text{ist}} = (h_6 - h_0 - T_0 (s_6 - s_0) + \frac{1}{2} \dot{m} w_6^2 - \frac{1}{2} \dot{m} w_0^2)
\]

\[
m_{05} = 
\]

\[
\dot{m} k = 5.293
\]

\[
\dot{m}_{05} = 5.293 \dot{m}
\]

\[
\Delta e_{x, \text{st}} = (cp (T_6 - T_0) - T_0 \left( cp \left( \ln \left( \frac{T_6}{T_0} \right) - R \ln \left( \frac{P_0}{P_0} \right) \right) \right) + \frac{w_6^2 - w_0^2}{2})
\]

\[
T_6, w_6 \text{ von}
\]

\textbf{Lösung:}

\[
s_{ext} = 14.413 \frac{kJ}{kg} + 110.050 \frac{kJ}{kg} = 264.18 \frac{kJ}{kg}
\]

\section*{d)}

\[
\dot{e}_{x, \text{ist}} = \frac{E}{\dot{m}}
\]

\[
\text{Exreal} = T_0 s_{erz}
\]

\[
s_{erz} = \dot{m} (s_2 - s_1) - \frac{Q}{T_J} \quad 1:m
\]

\[
s_{erz} = \dot{m}_{05} (s_8 - s_0) - \frac{q}{T_J}
\]

\[
s_{erz} = \dot{m} cp (T_2 - T_1) - 110.050 \frac{kJ}{kg}
\]

\[
m_{05} = (\dot{m} 4.5 2.5)
\]

\[
m_{05} = \frac{\dot{m}_1}{5.293}
\]

\[
\dot{m}_{05} = 3 GT 5 1 \frac{kW}{kJ}
\]

\[
\dot{E} = 80,3 12 \frac{kW}{kJ}
\]

\[
m_{05} = \dot{m}_1 (1 + 5.293 nc)
\]

``````latex


\begin{itemize}
    \item At the top left of the page, there is a small circle with the letter "e" inside it. To the right of this circle, there is an arrow pointing to the right.
\end{itemize}

\[
\Delta S_{12} = m^t \cdot s_2 - m^i \cdot s_1
\]

\begin{itemize}
    \item Below the formula, there is a scribble or crossed-out section.
    \item To the left of the next section, there is a circle with "T1" written inside it.
    \item Below the circle, there is a label "TAZ".
\end{itemize}

\[
T_1 = 100
\]

\[
s = 1.33 \frac{MJ}{kg}
\]

\[
s_2 = s_3 = 7.7533
\]

\[
\Delta S_{12} = 4.813 \frac{MJ}{kg} - \frac{35 MJ}{295 K}
\]

\begin{itemize}
    \item Below the formula, there is a horizontal arrow pointing to the left.
\end{itemize}

\[
\Delta S_{12} = 9.813 \, t
\]

``````latex


