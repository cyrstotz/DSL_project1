
``````latex


\section*{THERMO I}

\subsection*{4. Gefriertrocknung}

\subsubsection*{a)}

\begin{description}
    \item[Graph 1:] The graph is a Pressure-Volume ($P$-$v$) diagram. The x-axis is labeled $v$ [m$^3$/kg] and the y-axis is labeled $P$ [Pa]. There are three horizontal lines labeled $P_1$, $P_2$, and $P_3$ from bottom to top. Two diagonal lines intersect, forming an "X" shape. The left diagonal line is labeled "isotherm" and the right diagonal line is labeled "isotherm". The intersection point is labeled $T_3 = T_4$. There are two vertical dashed lines labeled $1$ and $2$. The area between the vertical lines is shaded and labeled "indoor". The left side of the graph is labeled $T_1 = T_2$.
\end{description}

\begin{description}
    \item[Graph 2:] The graph is a Pressure-Temperature ($P$-$T$) diagram. The x-axis is labeled $T$ [K] and the y-axis is labeled $P$ [Pa]. There are four horizontal lines labeled $P_1$, $P_2$, $P_3$, and $P_4$ from bottom to top. There is a bell-shaped curve with the left side labeled $x = 0$ and the right side labeled $x = 1$. The top of the curve is labeled "wet". There are two vertical dashed lines labeled $3$ and $4$. The area between the vertical lines is shaded and labeled "indoor". The bottom of the graph is labeled $T_3$ and $T_4$.
\end{description}

\textit{sorry, falsch gelesen} :(

\subsubsection*{b)}

\begin{align*}
    &\text{MR\_134a} \quad \text{isentrope} \\
    &\text{stationär:} \quad \dot{E} = \dot{m} \left[ h_2 + \frac{v_2^2}{2} \right] = \dot{m} \cdot WK \\
    &T_i = 10K \quad \text{über} \quad \text{unter Sub Punkt bei Indoor} \\
    &T_i = -10^\circ C \quad \rightarrow \quad T_4 = -16^\circ C \\
    &h_2 (x = 1, P_4, T_4 = -16^\circ C) = 237.74 \frac{\text{kJ}}{\text{kg}} \quad (\text{TAB-A10}) \\
    &h_3 (x = 0.95, x = 2, P_3 = 20 bar) = h_3 (x = 0.95, x = 2, P_3 = 20 bar) + \text{(scribbled out)} \\
    &h_3 = 266.497 \frac{\text{kJ}}{\text{kg}} \\
    &\dot{m} = \frac{WK}{h_2 - h_3} = \frac{0.974 \frac{\text{kJ}}{\text{s}}}{237.74 - 266.497} = 0.974 \frac{\text{kJ}}{\text{s}} = 3.5 \frac{\text{kg}}{\text{h}}
\end{align*}

\subsubsection*{c)}

\begin{align*}
    &x \quad \text{nach Drossel?} \\
    &P_1 = P_2 \quad T_2 = -10^\circ C \quad P_4 = P_3 \quad x_4 = 0 \\
    &\dot{m} = \dot{m} (h_4 - h_3) = 0.974 \cdot 10^{-6} \quad \# (93.42 - 266.497) = -769 \text{W} \\
    &h_4 (x = 0, P_4 = 20 bar) = 95.42 \frac{\text{kJ}}{\text{kg}} \quad (\text{TAB-A11}) \quad ? \quad