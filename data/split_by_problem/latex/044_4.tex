
``````latex


\section*{Aufgabe 4}

\subsection*{a)}

\begin{center}
\begin{tabular}{c}
\begin{picture}(100,100)
\put(10,10){\vector(1,0){80}}
\put(10,10){\vector(0,1){80}}
\put(10,10){\line(1,1){60}}
\put(10,10){\line(1,2){30}}
\put(40,40){\line(1,0){30}}
\put(70,40){\line(0,1){30}}
\put(40,40){\circle*{3}}
\put(70,40){\circle*{3}}
\put(70,70){\circle*{3}}
\put(25,25){\makebox(0,0){1}}
\put(55,25){\makebox(0,0){2}}
\put(55,55){\makebox(0,0){3}}
\put(75,75){\makebox(0,0){4}}
\put(5,5){\makebox(0,0){$T_1 = -10^\circ C$}}
\put(90,5){\makebox(0,0){$T$ (K)}}
\put(5,90){\makebox(0,0){$p$ (bar)}}
\put(40,80){\makebox(0,0){flüssiges}}
\put(40,75){\makebox(0,0){Gas}}
\put(80,40){\makebox(0,0){Kondensat}}
\end{picture}
\end{tabular}
\end{center}

\subsection*{b)}

\textbf{Energie-Satzliche Verhältnisse:}

\[
0 = \dot{m} \left( h_e - h_g \right) + \dot{Q} \quad \text{adiabat} \quad \dot{W}_k
\]

\[
\dot{m} = \frac{\dot{W}_k}{h_2 - h_1}
\]

\[
T_{AB} \quad h_e
\]

\[
h_2 \text{ ist mit } x_1 = 1 \quad \text{bei} \quad T_1 = -10^\circ C \quad \text{(aus Tafel Diagramm abgelesen)}
\]

\[
T_{AB} = A + 0 - \frac{N}{2} \quad h_2 = \frac{247,34 \, \frac{kJ}{kg} + 249,45 \, \frac{kJ}{kg}}{2} \quad \text{(aus Tafel Diagramm abgelesen)}
\]

\[
= 247,345 \, \frac{kJ}{kg}
\]

\[
h_3 \text{ wäre bei } s_2 \text{ mit gleicher Entropie}
\]

\[
s_2 = s_3 \quad \text{oder} \quad adiabatisch \quad reversibel
\]

\[
s_2 = 0,5233 + 0,5274 = 0,5233 \, \frac{kJ}{kg \cdot K}
\]

\[
T_{AB} = A + 1,2
\]

\[
s_3 = 0,5233 \, \frac{kJ}{kg \cdot K} \quad \text{(extrapoliert zwischen} \quad T_{sätt} = 37,33^\circ C \quad \text{und} \quad T_{4} = 0,5373)
\]

\[
h_3 = 223,66 \, \frac{kJ}{kg} + \left( 0,5233 \, \frac{kJ}{kg \cdot K} - 0,5373 \, \frac{kJ}{kg \cdot K} \right)
\]

\[
= 265,524 \, \frac{kJ}{kg}
\]

\[
\dot{m} = \frac{9,787 \, \frac{kJ}{s}}{(265,524 - 247,345) \, \frac{kJ}{kg}} = 0,2066 \, \frac{kg}{s}
\]

\end{document```latex


\section*{Problem 1}

\subsection*{c)}
\begin{align*}
    \text{Zustand 1:} \quad x_1 &= 0 \\
    p_1 &= p_2 = 8 \, \text{bar}
\end{align*}

\text{Prozess ist adiabatisch:}

\begin{align*}
    \text{Energiegleichung:} \quad \dot{Q} &= \dot{m} \left( h_{k2} - h_{k1} \right) + \dot{Q} \\
    \text{TAB:} \quad u_1 &= u_2 = 53,42 \, \frac{\text{kJ}}{\text{kg}}
\end{align*}

\text{Zustand 2:} \quad \text{Adiabatische Linie:} \quad p_2 = p_1

\begin{align*}
    \text{und es gilt:} \quad T_2 &= -20 \, \text{kJ/kg} \quad \text{gleich wie} \\
    \text{TAB:} \quad T_1 &= 470 \, \text{K}
\end{align*}

\begin{align*}
    x_1 &= \frac{h_1 - u_1}{u_g - u_f} \\
    h_f &= \frac{3,35 + 3,56}{2} \\
    \frac{u_2}{u_f} &= 36,365 \quad \frac{u_2}{u_g} \\
    u_2 &= 26,345 \quad \frac{u_2}{u_f}
\end{align*}

\begin{align*}
    \Rightarrow x_2 &= 0,2762
\end{align*}

\subsection*{d)}
\begin{align*}
    E_k &= \left( \frac{\dot{m}}{\dot{m} + 1} \right) = \left( \frac{\dot{m} + 1}{\dot{m} + 1} \right) \\
    \dot{m} \left( h_1 - h_2 \right) &= 0,164 \, \frac{\text{kJ}}{\text{s}} \\
    \dot{m} &= \frac{4 \, \text{kg}}{3600 \, \text{s}}
\end{align*}

\begin{align*}
    \Rightarrow E_k &= 5,187
\end{align*}

\subsection*{e)}
Es wird sicher, dass wir Energie aus anderen System erhalten.

```