
``````latex


\section*{Aufgabe 3}

\textit{a)} \textit{Ideale Gasgleichung}

\[
p \cdot V = n \cdot R \cdot T
\]

\textit{Den Druck finden wir mit der Gleichgewicht}

\[
p_{g,1} = \frac{m_{WE} \cdot g}{A} + \frac{m_K \cdot g}{A} + p_{ab} \quad \text{mit} \quad m_{WE} = 0,6 \cdot 0,1 \, \text{kg}
\]

\textit{und} \quad $A = \pi \left( \frac{D}{2} \right)^2$

\[
p_{g,1} = 1,6 \, \text{bar}
\]

\[
m_{g,1} = \frac{R \cdot T_{g,1}}{V_{g,1} \cdot p_{g,1}} \quad \text{mit} \quad R = \frac{R}{M_g} = 166,28 \, \frac{J}{\text{kg} \cdot K}
\]

\[
m_{g,1} = \frac{p_{g,1} \cdot V_{g,1}}{R \cdot T_{g,1}}
\]

\[
= 3,92 \, g
\]

\textit{b)} \textit{Die Temperatur von Gas weicht die von Eis weicht, weil es eine lange Fläche hat. Die Temperatur von Gas weicht die Temperatur von Eis weicht} $T_{g,z} = 273,15 \, K = 0^\circ C$

``````latex


Die volle Temperatur wird die von $T_{2}$, weil in thermodynamische Gleichgewicht.

$P_{2,g} = 1,4 \text{ bar}$, weil die Masse die auf das Los drückt gleich bleibt.

\[
\text{Energieerhaltung von 1 $\rightarrow$ 2:}
\]

\[
m_{2} \cdot u_{2} - m_{1} \cdot u_{1} + Q_{12} - \cancel{W_{V}} \quad \left\{ \text{gibt es Arbeit!} \right.
\]

\[
\text{mit } m_{1} = m_{2} = m_{g}
\]

\[
m_{g} (u_{2} - u_{1}) = Q_{12} - \cancel{W_{V}}
\]

\[
\Rightarrow Q_{12} = m_{g} \cdot c_{v} (T_{2} - T_{1}) + W_{V}
\]

\[
= 1082,43 \, \text{J} + W_{V} \quad \text{mit } V_{2,g} = 1,108 \, \text{L}
\]

\[
W_{V} = m_{g} \int_{V_{1}}^{V_{2}} p \, dv
\]

\[
= m_{g} \cdot P \left( V_{2,g} - V_{1,g} \right)
\]

\[
= -0,872 \, \text{J}
\]

\[
Q_{12} = 1081,5 \, \text{J}
\]

\[
\boxed{}
\]

``````latex


