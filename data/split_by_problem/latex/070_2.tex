
``````latex


A2)

\begin{itemize}
    \item[a)] 
    \begin{itemize}
        \item[1)] 
        \begin{description}
            \item[Graph Description:] 
            The graph is a Temperature (T) vs Entropy (S) diagram. The vertical axis is labeled \( T \) and the horizontal axis is labeled \( S \). There are several curves and lines drawn on the graph:
            \begin{itemize}
                \item There are three main curves labeled as adiabatic, reversible \(\rightarrow\) isentropic.
                \item The curves are labeled 1, 2, and 3 from bottom to top.
                \item There are two isobar lines labeled \( P_1 \) and \( P_2 \) with \( P_2 \) being higher than \( P_1 \).
                \item The isobar lines intersect the adiabatic curves at points labeled 0, 1, 2, 3, 4, 5, and 6.
                \item The points 0, 1, 2, 3, 4, 5, and 6 are connected by arrows indicating the direction of the process.
                \item The isobar lines are labeled with pressures \( P_0 = 0.75 \, \text{bar} \) and \( P_0 = 0.5 \, \text{bar} \).
            \end{itemize}
        \end{description}
    \end{itemize}
\end{itemize}

\begin{itemize}
    \item[Graph Description:] 
    The second graph is also a Temperature (T) vs Entropy (S) diagram. The vertical axis is labeled \( T \) and the horizontal axis is labeled \( S \, [\text{kJ}/\text{kgK}] \). There are several curves and lines drawn on the graph:
    \begin{itemize}
        \item There are three main curves labeled as adiabatic, reversible \(\rightarrow\) isentropic.
        \item The curves are labeled 1, 2, and 3 from bottom to top.
        \item There are two isobar lines labeled \( P_1 \) and \( P_2 \) with \( P_2 \) being higher than \( P_1 \).
        \item The isobar lines intersect the adiabatic curves at points labeled 0, 1, 2, 3, 4, 5, and 6.
        \item The points 0, 1, 2, 3, 4, 5, and 6 are connected by arrows indicating the direction of the process.
        \item The isobar lines are labeled with pressures \( P_0 = 0.75 \, \text{bar} \) and \( P_0 = 0.5 \, \text{bar} \).
        \item There is a temperature value \( T_5 = 437.96 \, \text{K} \) and another value 293.15.
    \end{itemize}
\end{itemize}

``````latex

b) \quad v_6 = ? \quad T_6 = ? 

\[
\begin{aligned}
v_5 &= 220 \, \text{m/s} \\
T_5 &= 437.5 \, \text{K} \\
p_5 &= 0.5 \, \text{bar} \\
p_6 &= p_0 = 0.197 \, \text{bar} \\
\text{isentrop} & \quad s_5 = s_6
\end{aligned}
\]

\text{Stationary Process}

\[
0 = \dot{m} \left( h_5 - h_6 + \frac{w_5^2 - w_6^2}{2} \right) + \dot{Q} - \dot{W}
\]

\text{adiabatic} \quad \text{reversible}

\[
h_5 - h_6 + \frac{w_5^2 - w_6^2}{2} = 0
\]

\[
h_5 - h_6 = c_p (T_5 - T_6) = 146,2286 \, \text{kJ/kg}
\]

\[
c_p = 1 \, \text{kJ/(kg K)} = n \cdot c_v = 1,4089 \, \text{kJ/(kg K)}
\]

\[
\Rightarrow R = c_p - c_v = 0,4024 \, \text{kJ/(kg K)}
\]

\[
\text{isentrop} \quad s_5 = s_6
\]

\[
0 = c_p \ln \left( \frac{T_6}{T_5} \right) - R \ln \left( \frac{p_6}{p_5} \right)
\]

\[
R \ln \left( \frac{p_6}{p_5} \right) = c_p \ln \left( \frac{T_6}{T_5} \right) \quad | e^x
\]

\[
e^{R \ln \left( \frac{p_6}{p_5} \right)} = e^{c_p \ln \left( \frac{T_6}{T_5} \right)}
\]

\[
\left( \frac{p_6}{p_5} \right)^R = \left( \frac{T_6}{T_5} \right)^{c_p} \Rightarrow T_6 = T_5 \left( \frac{p_6}{p_5} \right)^{\frac{R}{c_p}} = 328,075 \, \text{K}
\]

\[
w_6^2 = w_5^2 + 2 (h_5 - h_6)
\]

\[
w_6 = \sqrt{w_5^2 + 2 \cdot 146,2286 \, \text{kJ/kg}}
\]

\[
= \sqrt{220^2 + 2 \cdot 146,2286 \cdot 1000} \, \text{m/s}
\]

\[
= 1729,93 \, \text{m/s}
\]

``````latex

\section*{c)}

\[
\Delta e_{x,str} = e_{x,str,6} - e_{x,str,0} = h_6 - h_0 - T_0 (s_6 - s_0) + \frac{w_6^2}{2} - \frac{w_0^2}{2}
\]

\[
h_6 - h_0 = c_p (T_6 - T_0) = 719,608 \frac{kJ}{kg}
\]

\[
s_6 - s_0 = c_p \ln \left( \frac{T_6}{T_0} \right) - R \ln \left( \frac{p_6}{p_0} \right) = 0,927905 \frac{kJ}{kg \cdot K}
\]

\[
\frac{w_0^2}{2} = 20 \frac{kJ}{kg}
\]

\[
\frac{W_6^2}{2} = 1487,8 \frac{kJ}{kg}
\]

\[
\Delta e_{x,str} = 1487,21 \frac{kW}{kg}
\]

\section*{d)}

\[
\dot{E}_{x,vol} = T_0 \cdot \dot{S}_{ir,z}
\]

\textit{Stationäre Fließprozesse}

\[
0 = \dot{m} (s_0 - s_6) + \frac{\dot{Q}}{T_0} + \dot{S}_{ir,z}
\]

\[
\dot{S}_{ir,z} = s_6 - s_0 - \frac{q}{T_0} = 0,927905 \frac{kJ}{kg \cdot K} - \frac{1795 \frac{kJ}{kg}}{728,8}
\]

\[
= 0,505 \frac{kJ}{kg \cdot K}
\]

\[
e_{x,vol} = T_0 \cdot \dot{S}_{ir,z} = 243,15 \cdot 0,505 = 122,832 \frac{kW}{kg}
\]

``````latex


