
``````latex


\section*{Aufgabe 3}

\subsection*{a)}
\begin{align*}
M &= 50 \, \text{t/h} & \sigma &= 0.653 \, \frac{\text{kg}}{\text{m}^3} & T_{e1} &= 500^\circ \text{C} & V_{e1} &= 5.15 \cdot 10^{-3} \, \text{m}^3 \\
pV &= mRT \\
R &= \frac{R}{\mu} = 166.29 \, \frac{\text{J}}{\text{kgK}} \\
p_{e1} &= p_{amb} + (m_e + m_w) g \frac{h}{A} \\
p_{e1} &= 1.966 \, \text{bar} \\
m_{e1} &= \frac{m_i V_{e1}}{R T_{e1}} = 0.0056 \, \text{kg} = 5.6 \, \text{g}
\end{align*}

\subsection*{b)}
\begin{align*}
T_{e2} &= 0^\circ \text{C} & p_{e2} &= p_{e1} = 1.966 \, \text{bar}
\end{align*}

\textbf{Temperatur:} Da $x = 0$ ist, besteht befindet sich das EW immer noch im EW Gleichgewicht und die ganze zugeführte Wärme wurde zum Schmelzen angewendet, nicht zur Temperaturerhöhung und das Eis steht im thermischen GG $\rightarrow$ gleiche Temperatur.

\textbf{Druck:} Da der Außendruck und die Masse konstant ist, bleibt $p$ automatisch auch konstant.

\subsection*{c)}
\begin{align*}
V_{e2} &= \frac{M_e R T_{e2}}{p_{e2}} = 1.871 \, \text{L}
\end{align*}

\begin{align*}
\Delta U &= Q - W \\
Q &= \Delta U + W - 2233.02 \, \text{J} \\
|Q_{21}| &= 2233.02 \, \text{J}
\end{align*}

\begin{align*}
W &= \int_{V_1}^{V_2} p \, dV = p (V_2 - V_1) = -466.62 \, \text{J} \\
\Delta U &= m_e c_v (T_2 - T_1) = -1772.4 \, \text{J}
\end{align*}

\subsection*{d)}
\begin{align*}
x_1 &= 0.6 & \frac{dE}{dt} &= \dot{m} (h_{1} + \frac{v_1^2}{2} + gz) - \dot{Q} - \dot{W} & \dot{V} &= \text{const}
\end{align*}

\text{Wir rechnen mit } Q = 1500 \, \text{J}

\begin{align*}
\Delta U &= Q_{21} \\
m_w (u_2 - u_1) &= Q_{21} \\
u_2 &= \frac{Q_{21}}{m_w} + u_1 = -185.1 \, \frac{\text{J}}{\text{kg}}
\end{align*}

``````latex

\[
x_s = \frac{u_2 - u_{\text{min}}}{u_k - u_{\text{min}}} = 0.555
\]

``````latex


