
``````latex


\section*{Solution}

\subsection*{a)}

\begin{itemize}
    \item The first graph is a plot with the vertical axis labeled \( p \, [\text{bar}] \) and the horizontal axis labeled \( T \, [K] \). The graph shows a wavy line that starts at a high value on the vertical axis and oscillates downwards and upwards several times as it moves to the right along the horizontal axis.
\end{itemize}

\begin{itemize}
    \item The second graph is a plot with the vertical axis labeled \( p \, [\text{bar}] \) and the horizontal axis labeled \( T \, [K] \). The graph includes the following elements:
    \begin{itemize}
        \item A curve starting from the origin and rising upwards, labeled "isotherm".
        \item A horizontal arrow pointing to the right from the curve, labeled "fest".
        \item A vertical arrow pointing upwards from the horizontal arrow, labeled "gas".
        \item A horizontal arrow pointing to the right from the vertical arrow, labeled "flüssig".
        \item A vertical arrow pointing downwards from the horizontal arrow, labeled "isobar".
    \end{itemize}
\end{itemize}

``````latex


\section*{Problem 4}

\subsection*{b)}

\begin{align*}
&\text{Given:} \\
&\quad p_2 = p_2 \\
&\quad p_3 = p_4 = 3 \text{bar} \\
&\quad p_2 = p_1 - p \\
&\quad h_4 = h_2 = h_f (80 \text{bar}) = 93.42 \frac{\text{kJ}}{\text{kg}} \\
&\quad s_3 = s_2
\end{align*}

\begin{align*}
Q &= \dot{m}_{\text{R}} (h_2 - h_3) - \dot{W}_{\text{K}} \\
\dot{m}_{\text{R}} &= \frac{\dot{W}_{\text{K}}}{h_2 - h_3} = \frac{-28 \text{kW}}{(237.74 - 253.31) \frac{\text{kJ}}{\text{kg}}} = 1.80 \frac{\text{kg}}{\text{s}}
\end{align*}

\begin{align*}
h_2 &= h_g (257.15 \text{K}) = h_g (-16^\circ \text{C}) = 237.74 \frac{\text{kJ}}{\text{kg}} \quad \text{(A10)}
\end{align*}

\begin{align*}
T_2 &= T_2 = T_i - \Delta t = T_{\text{isoll}} + (0 - 0) \text{K} = (273.15 \text{K} - 20 \text{K}) + 46 \text{K} = 257.15 \text{K} = -16^\circ \text{C}
\end{align*}

\begin{align*}
h_3 (s_3 = s_2) &= 93.42 \frac{\text{kJ}}{\text{kg}} + \frac{264.15}{264.15} \\
s_2 &= s_{g1} (-16^\circ \text{C}) = 0.9298 \frac{\text{kJ}}{\text{kg K}} = s_3 \quad \text{(A10)}
\end{align*}

\begin{align*}
h_3 (80 \text{bar}, s_2) &= 264.15 \frac{\text{kJ}}{\text{kg}} \\
&= \frac{273.66 - 264.15}{0.9298 - 0.8066} \\
&= 253.74 - 0.8066 \\
&= 253.32 \frac{\text{kJ}}{\text{kg}}
\end{align*}

\subsection*{c)}

\begin{align*}
h_2 &= 93.42 \frac{\text{kJ}}{\text{kg}} = h_f (-16^\circ \text{C}) + x_2 (h_g - h_f) \quad \text{(TAB 10)}
\end{align*}

\begin{align*}
x_2 &= \frac{h_2 - h_f}{h_g - h_f} = 0.3076
\end{align*}

\subsection*{d)}

\begin{align*}
\epsilon_{\text{K}} &= \frac{Q_{\text{zu}}}{\dot{W}_{\text{K}}} = \frac{Q_{\text{zu}}}{\dot{W}_{\text{K}}} = \frac{252.8}{28} = 9.28
\end{align*}

\begin{align*}
Q_{\text{zu}} &= \dot{m}_{\text{R}} (h_2 - h_3) = 1.80 \frac{\text{kg}}{\text{s}} (237.74 - 93.42) \frac{\text{kJ}}{\text{kg}} = 258.8 \text{kW}
\end{align*}

\subsection*{e)}

\begin{align*}
\text{Die Temperatur wird sich schlussendlich auf } T_2 = -16^\circ \text{C} \text{ einpendeln!}
\end{align*}

\end