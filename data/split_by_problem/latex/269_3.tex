
``````latex


\section*{Problem 4}

\subsection*{a)}

\begin{description}
    \item[Graph Description:] The graph is a phase diagram with the y-axis labeled as $p$ and the x-axis labeled as $T$. The graph shows different regions labeled as "solid", "liquid", and "vapor". There is a curve labeled "isotherm" and another curve labeled "isobar". The point where these curves meet is labeled "triple point". There is also a line labeled "isotherm" that intersects the "solid" and "vapor" regions. The graph includes arrows indicating the direction of the phases.
\end{description}

\subsection*{b)}

\[
0 = \dot{m}_{\text{R134a}} (h_2 - h_3) - \dot{W}_k \Rightarrow \dot{W}_k
\]

\[
x_4 = 0 \quad p_4 = 8 \text{ bar} \quad h_1 = h_4 = h_f(8 \text{ bar}) = 93.92 \frac{\text{kJ}}{\text{kg}} \quad (A-11)
\]

\[
\dot{W}_k
\]

\[
T_1 = -20^\circ C + 10^\circ C = -10^\circ C
\]

\[
\Delta T_{\text{ü}} = -6^\circ C - (-10^\circ C) = -16^\circ C = T_2 - T_1
\]

\[
h_2 (-16^\circ C, x_2 = 1) = 237.74 \frac{\text{kJ}}{\text{kg}} \quad (A-10)
\]

\[
s_2 (-16^\circ C, x_2 = 1) = 0.9298 \frac{\text{kJ}}{\text{kgK}} = s_3
\]

\[
h_3 (s_3, 8 \text{ bar}) = 264.15 \frac{\text{kJ}}{\text{kg}} + \left( \frac{283.66 \frac{\text{kJ}}{\text{kg}} - 264.15 \frac{\text{kJ}}{\text{kg}}}{0.9374 - 0.9066} \right) (0.9298 - 0.9066)
\]

\[
\approx 270.92 \frac{\text{kJ}}{\text{kg}} \quad (A-12)
\]

\[
\dot{m} = \frac{\dot{W}_k}{h_2 - h_3} \approx 0.894 \frac{\text{kg}}{\text{s}}
\]

\subsection*{c)}

\[
\dot{m}_{\text{R134a}} = \frac{4 \frac{\text{kg}}{\text{h}}}{3600 \frac{\text{s}}{\text{h}}} = 0.0011 \frac{\text{kg}}{\text{s}}
\]

\[
T_2 = -22^\circ C
\]

\[
\Rightarrow p_2 = 1.2122 \text{ bar} \quad p_1
\]

\[
h_1 = 93.42 \frac{\text{kJ}}{\text{kg}} \quad (\text{Tafelaufgabe 4})
\]

\[
x = \frac{h_1 - h_f(-22^\circ C)}{h_g(-22^\circ C) - h_f(-22^\circ C)} = 0.337
\]

\[
0 = h_1 - h_2 + \dot{Q}_k
\]

\[
\dot{Q}_{zu} = \dot{Q}_k = \dot{m}_{\text{R134a}} \cdot (h_2 - h_1) = 160.35 \text{ kW}
\]

\[
\dot{W}_e = 281 \text{ W} \quad (\text{Watt nicht Kilowatt?})
\]

\subsection*{d)}

\[
\epsilon_k = \frac{\dot{Q}_{zu}}{\dot{W}_e}
\]

\[
\epsilon_k = 572.98 \quad ??
\]

``````latex


c) Es würde sich weiß abkühlen, da Wärme entzogen wird aus dem Gefrierbeutel, aber der ist langfristig nicht möglich.

```