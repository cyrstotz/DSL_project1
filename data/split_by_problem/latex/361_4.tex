
``````latex


\section*{Aufgabe 4}

\subsection*{a)}

\begin{description}
    \item[Graph Description:] The graph is a Pressure-Temperature (P-T) diagram. The x-axis is labeled \( T \) (Temperature) and the y-axis is labeled \( P \) (Pressure). There are three regions labeled "Fest" (solid), "Flüssig" (liquid), and "Gas" (gas). The boundary lines between these regions are shown, with a point labeled \( i \) on the line between "Fest" and "Flüssig", a point labeled \( ii \) on the line between "Flüssig" and "Gas", and a point labeled \( iii \) on the line between "Fest" and "Gas".
\end{description}

\subsection*{b)}

\begin{align*}
T_i &= -70^\circ C \\
\text{Tverdampfer} &= -16^\circ C \\
\\
z &= 3 \\
0 &= \dot{m} [h_e - h_a] - \dot{W}_k \\
h_e &= h_{g}(-16^\circ C) \\
&= 237.97 - 236.0 \\
&= 237.74 \frac{kJ}{kg} \\
\\
h_a &= h \left( \frac{8 bar, x=0.92}{0.9285 - 0.9322} \right) \\
&= \frac{275.96 - 264.75}{0.9284 - 0.8066} + 264.75 \\
&= 274.57 \frac{kJ}{kg} \\
\\
\dot{m} &= \frac{\dot{W}_k}{h_e - h_a} \\
&= \frac{8.34}{237.74 - 274.57} \\
&= 8.34 \times 10^{-4} \frac{kg}{s}
\end{align*}

``````latex


\section*{c)}

\textit{st. Fließprozess}

\[
\dot{m} (h_e - h_a) = 0
\]

\[
h_a = h_e = h_f (8 \text{bar}) = 93,42 \frac{\text{kJ}}{\text{kg}}
\]

\[
T_f = T_e = 70^\circ C
\]

\[
h_{fg} = 233,74
\]

\[
h_f = 93,48 - 23,77 = (-16 + x \cdot 8) + 23,77 = 29,3 \frac{\text{kJ}}{\text{kg}}
\]

\[
x = \frac{93,42 - 29,3}{233,74 - 29,3} = 0,3076
\]

```