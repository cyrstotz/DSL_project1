
``````latex


\section*{Aufgabe 4: Gefrierodnung mit einem R134a Kühlkreislauf}

\subsection*{a)}

\begin{description}
    \item[Graph Description:] The graph is a Temperature-Entropy ($T$-$s$) diagram. The x-axis is labeled $s$ (entropy) and the y-axis is labeled $T$ (temperature). The graph shows a closed loop with four points labeled 1, 2, 3, and 4. The region between points 1 and 2 is labeled "Dampf-Flüssig zweiphasiges Gebiet" (vapor-liquid two-phase region). The region to the left of point 4 is labeled "Flüssig" (liquid), and the region to the right of point 3 is labeled "Gas" (gas). There is an isobar labeled "Isobare $p_3 = 8 \text{ bar}$" extending vertically from point 3. The process from point 4 to point 1 is labeled "adiabat".
\end{description}

\subsection*{b)}

Aus der TAB A-15 interpoliert

\[
\dot{m}_{R134a} (h_2 - h_1) \neq \dot{W} \Rightarrow \dot{W} = 0
\]

\[
\dot{m}_{R134a} = \frac{28 \, \text{W}}{h_3 - h_2} = \frac{28 \, \text{W}}{7 \, \text{W}} = 4 \, \text{kg/h}
\]

\subsection*{c)}

\[
x_{1,2} = \frac{u_2 - u_f}{u_g - u_f}
\]

\[
u_1 = u_f
\]

``````latex


\section*{Student Solution}

\begin{itemize}
    \item[d)] \( G_{u} = \frac{\left| \dot{Q}_{\text{zul}} \right|}{\left| \dot{W}_{e} \right|} \)
    
    \[
    = \frac{\dot{Q}_{\text{zu}}}{\left| \dot{Q}_{\text{alt}} - \dot{Q}_{\text{zul}} \right|}
    \]
    
    \[
    = \frac{\text{---}}{28 \, w}
    \]
    
    \item[e)] Wärmestrom \( \dot{Q}_{k} \) garantiert
\end{itemize}

```