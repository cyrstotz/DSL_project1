
``````latex


\section*{Aufgabe 4}

\textbf{a)} \\
\textit{isobare und isotherme Expansion einer isobaren Verdampfung} \\
\textit{ausführlich} \\
\textit{adiabate, reversible Verdichtung} \\
$Q=0$ \quad $Q=0$ \\

\textit{ausführlich} \\
\textit{isotherme Expansion, Wärmeübertrager ohne irreversiblen} \\
\textit{Verluste} \\

\textbf{a)} \\

\begin{description}
    \item[Graph:] 
    \begin{itemize}
        \item The graph is a Pressure-Volume (P-V) diagram.
        \item The x-axis is labeled $T$ and has points $T_1 - \Delta t$, $T_1$, and $T_2$.
        \item The y-axis is labeled $p$ and has points $p_1$, $p_2$, and $p_3$.
        \item There are three curves on the graph:
        \begin{itemize}
            \item The first curve starts at $T_1 - \Delta t$ and $p_1$, and ends at $T_1$ and $p_2$.
            \item The second curve starts at $T_1$ and $p_2$, and ends at $T_2$ and $p_3$.
            \item The third curve starts at $T_1$ and $p_2$, and ends at $T_2$ and $p_3$.
        \end{itemize}
        \item There are five points labeled on the graph:
        \begin{itemize}
            \item Point 1 is at $T_1 - \Delta t$ and $p_1$.
            \item Point 2 is at $T_1$ and $p_2$.
            \item Point 3 is at $T_2$ and $p_3$.
            \item Point 4 is at $T_1$ and $p_2$.
            \item Point 5 is at $T_2$ and $p_3$.
        \end{itemize}
    \end{itemize}
\end{description}

$T_i$ aus Graph (nach Sander auch Tripelpunkt um Links) \\
$T_i = -20^\circ C \quad T_{a} = -26^\circ C$ \\

\textbf{b)} \\
\textit{System (offen) Gleichgewicht, stationär} \\
\[
\dot{E}_w = \frac{d}{dt}(PE) = \frac{d}{dt}(KE) + \dot{m}_e \cdot h_e - \dot{m}_a \cdot h_a + \dot{Q} - \dot{W}
\]
\[
\Rightarrow \dot{m}_e \cdot h_e + \dot{Q} + \dot{Q}_{ab} - \dot{W} = 0
\]
\textit{Bekannte Messwerte, aus Qe und Qab bestimmt werden} \\
$Q_e$ ist Wärme der isobaren Verdampfung von R134a \\
\[
\frac{Q_e}{\dot{m}_a} = \frac{h_{g1}(T_i) - h_{f1}(T_i)}{T_i} = 245.84 \frac{kJ}{kg} - 24.17 \frac{kJ}{kg} = 419.67 \frac{kJ}{kg}
\]
\textit{Tabelle A-10 bei $T_i = -20^\circ C$} \\
$Q_{ab}$ ist Wärme der isobaren Kondensation von R134a bei $T_{ab} = -26^\circ C$ \\
\[
\frac{Q_{ab}}{\dot{m}_a} = \frac{h_{f1}(T_{ab}) - h_{g1}(T_{ab})}{T_{ab}} = 16.25 - 242.43 = -435.68 \frac{kJ}{kg}
\]
\[
\Rightarrow \frac{Q_{ab}}{\dot{m}_a} = \frac{h_{f1}(T_{ab}) - h_{g1}(T_{ab})}{T_{ab}}
\]
\[
\Right