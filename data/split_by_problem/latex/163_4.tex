
``````latex


\section*{Aufgabe 3.4}

\subsection*{a) Energie-Massenbilanz}

\[
0 = \dot{m} \left[ h_2 - h_3 + \frac{c_p}{\lambda} (T_2 - T_3) \right] + \dot{Q}_{23} - \dot{W}_K
\]

\[
\Rightarrow \dot{W}_K = \dot{m} (h_2 - h_3)
\]

\subsection*{Annahme}

\[
\dot{m}_1 = \dot{m}_2
\]

\[
T_K = T_i, \text{ Annahme 1 } x > 0
\]

\subsection*{Drossel Isenthalpe}

\[
TAB - A11
\]

\[
p_4 (18 \text{ bar}) \quad h_4 (18 \text{ bar}) = h_f (18 \text{ bar}) = 83,142 \frac{\text{kJ}}{\text{kg}}
\]

\[
h_1 = h_4 = 83,142 \frac{\text{kJ}}{\text{kg}}
\]

\subsection*{T_2 aus Lösung}

\[
T_2 = -72^\circ \text{C}
\]

\[
\Rightarrow \text{Aus TAB A-10}
\]

\[
h_2 = h_g (-72^\circ \text{C}) = 734,08 \frac{\text{kJ}}{\text{kg}}
\]

\[
h_3 (S_3 = S_2) (8 \text{ bar})
\]

\[
S_2 = S_3 = S_{g4} (-72^\circ \text{C}) = 0,8351 \frac{\text{kJ}}{\text{kg K}}
\]

\subsection*{Lin Interpolieren TAB A-12}

\[
\Rightarrow \text{Für 8 bar überh. Dampf}
\]

\[
h_3 (0,9351 \frac{\text{kJ}}{\text{kg K}}) = \frac{h \left( 0,8374 \frac{\text{kJ}}{\text{kg K}} \right) - h \left( 0,86066 \frac{\text{kJ}}{\text{kg K}} \right)}{0,8374 - 0,86066} \cdot (0,8351 - 0,86066) + h \left( 0,86066 \frac{\text{kJ}}{\text{kg K}} \right)
\]

\[
h_3 = 727,95 \frac{\text{kJ}}{\text{kg}}
\]

\subsection*{Graphische Darstellung}

Es gibt ein Diagramm mit zwei Achsen. Die horizontale Achse ist mit $T [^\circ C]$ beschriftet und die vertikale Achse ist mit $p [\text{bar}]$ beschriftet. Es gibt vier markierte Punkte, die durch Linien verbunden sind, die ein Rechteck bilden. Die Punkte sind mit den Zahlen 1, 2, 3 und 4 beschriftet. Die Linie von Punkt 1 zu Punkt 2 ist als "isentrop" gekennzeichnet, und die Linie von Punkt 3 zu Punkt 4 ist als "isotherm" gekennzeichnet.

``````latex


\[
\dot{W}_K = -28W
\]

\[
\frac{h_2 - h_3}{(734,08 - 27,92) \frac{kJ}{kg}} = 0,1720 \frac{g}{s} = \dot{m}_{R134a}
\]

\[
\dot{m} = 1 \frac{kg}{s} \implies \dot{m} = 758 \frac{g}{s}
\]

\[
T_1 = T_2 = -22^\circ C
\]

\[
h_1 = h_{g1} = h_{g1}(8 bar) = 93,42 \frac{kJ}{kg}
\]

\[
x_1: x_1 > 0 \text{Nassdampf}
\]

\[
x_1 = \frac{h_1 - h_{f1}}{h_{g1} - h_{f1}} = \frac{93,42 - 24,76}{255,31 - 24,76}
\]

\[
x_1 = 0,3277 \implies 32,77\%
\]

\[
\epsilon_K = \frac{\dot{Q}_{zu}}{\dot{W}_K}
\]

\[
\dot{Q}_{zu} = \dot{Q}_K = \dot{m}_{R134a} \cdot (h_2 - h_1)
\]

\[
\dot{Q}_K = 0,172 \frac{kg}{s} \cdot 10^{-3} \cdot (734,08 - 93,42) \frac{kJ}{kg}
\]

\[
= 100,10 \frac{W}
\]

\[
\frac{100,13 W}{28 W} = \epsilon_K = 3,618
\]

\[
\text{e) die Temperatur wurde bis zur Temperatur Tz abkühlen}
\]

\[
\text{b) sein Thermodynamisches Gleichgewicht ent steht}
\]

\text{a) A diagram is drawn with the following details:}

\begin{itemize}
    \item The diagram is a square cycle with four points labeled 1, 2, 3, and 4 in a clockwise direction.
    \item The x-axis is labeled as $p_{abs}$.
    \item The y-axis is labeled as $T_{KL}$.
    \item The top side of the square is labeled as "Isobar".
    \item The right side of the square is labeled as "Isochore".
    \item The bottom side of the square is labeled as "Isobar".
    \item The left side of the square is labeled as "Isochore".
\end{itemize}

```