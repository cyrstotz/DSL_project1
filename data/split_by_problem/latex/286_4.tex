
``````latex


\section*{4}

\subsection*{a)}

\textbf{Graph 1:}

This graph is a pressure-volume (P-V) diagram. The x-axis is labeled with "V" and the y-axis is labeled with "P". The graph consists of a closed loop with four points labeled 1, 2, 3, and 4. 

- Point 1 is at the bottom right of the loop.
- Point 2 is at the bottom left of the loop.
- Point 3 is at the top left of the loop.
- Point 4 is at the top right of the loop.

The loop is divided into two regions by a vertical line passing through point 1. The left region is labeled "gas" and the right region is labeled "flüssig". There is a wavy line between points 1 and 4, indicating a phase change. The region between points 1 and 2 is labeled "flüssig" and the region between points 2 and 3 is labeled "gas".

\textbf{Graph 2:}

This graph is also a pressure-volume (P-V) diagram. The x-axis is labeled with "V" and the y-axis is labeled with "P". The graph consists of a closed loop with four points labeled 1, 2, 3, and 4.

- Point 1 is at the bottom left of the loop.
- Point 2 is at the bottom right of the loop.
- Point 3 is at the top right of the loop.
- Point 4 is at the top left of the loop.

The loop is divided into two regions by a diagonal line passing through points 1 and 3. The left region is labeled "flüssig" and the right region is labeled "gas". There is a wavy line between points 3 and 4, indicating a phase change.

\textbf{Graph 3:}

This graph is a pressure-temperature (P-T) diagram. The x-axis is labeled with "T" and the y-axis is labeled with "P". The graph consists of a closed loop with four points labeled 1, 2, 3, and 4.

- Point 1 is at the bottom left of the loop.
- Point 2 is at the bottom right of the loop.
- Point 3 is at the top right of the loop.
- Point 4 is at the top left of the loop.

The loop is divided into two regions by a diagonal line passing through points 1 and 3. The left region is labeled "flüssig" and the right region is labeled "gas".

\subsection*{b)}

\begin{equation}
Q = \dot{m} (h_4 - h_2) + \dot{E}_Q - \dot{E}_W
\end{equation}

\begin{equation}
\dot{m} = \frac{\dot{W}_{mech}}{h_3 - h_2}
\end{equation}

``````latex


\section*{Student Solution}

\subsection*{a)}

\[
h_{g2} \quad T_1 = -10^\circ C
\]

\[
T_2 = T_1
\]

\[
\Rightarrow h_{g2} \quad \text{interpoliert aus TAB A-10}
\]

\[
h_{g2} = \frac{hg(-8^\circ C) - hg(-12^\circ C)}{-8^\circ C - (-12^\circ C)} \cdot (-8^\circ C + 10^\circ C) + hg(-12^\circ C) = 2467,85 \frac{kJ}{kg}
\]

\[
\frac{2444,31 \frac{kJ}{kg}}{243,3}
\]

\[
h_{g}
\]

\[
\frac{T_2}{T_1} = \left( \frac{p_2}{p_1} \right)^n
\]

\[
\Rightarrow \text{aus TAB A-12}
\]

\[
p(268,1) \quad h_{g3} = h_{g2} = 264,15
\]

\subsection*{b)}

\[
\dot{m} = \frac{28 \frac{g}{s}}{266,15-243,2} = 4,1 \frac{kg}{s} \cdot 1,4 \times 10^{-3} \frac{kg}{s} = 1,4 \frac{kg}{s}
\]

\[
= 1,368 \times 10^{-3} \frac{kg}{s} = 4,9742 \frac{kg}{s}
\]

\subsection*{c)}

\[
(\dot{m} in = 4,1 \frac{kg}{s} \quad \dot{T}_1 = -22^\circ C)
\]

\[
x = \frac{h - h_f}{h_g - h_f}
\]

\subsection*{d)}

\[
E_k = \frac{\dot{Q}_{zu}}{\dot{W}_t} = \frac{\dot{Q}_{k} - \dot{Q}_{ab}}{7,8 \frac{W}{s}}
\]

\subsection*{e)}

Der Druck würde immer weniger werden und das Wasser würde sublimieren zuerst. Irgendwann würde das Wasser den Tripelpunkt erreichen und nicht aus dem Essen entweichen.

```