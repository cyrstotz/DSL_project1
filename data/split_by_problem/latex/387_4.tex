
``````latex


\section*{Aufgabe 4}

\subsection*{a)}

\begin{description}
    \item[Graph Description:] The graph consists of a horizontal axis labeled \( T(K) \) and a vertical axis labeled \( p(N/m^2) \). There are three curves on the graph:
    \begin{itemize}
        \item The first curve starts at the origin, rises to a peak, and then falls back down.
        \item The second curve starts at the same point as the first curve, rises to a higher peak, falls below the horizontal axis, rises again to a smaller peak, and then falls back down.
        \item The third curve starts at the same point as the first two curves, rises to a peak, falls below the horizontal axis, rises again to a smaller peak, and then falls back down, similar to the second curve but with different amplitudes.
    \end{itemize}
\end{description}

\subsection*{b)}

\[
\dot{m}_{\text{Re}} = \dot{m}_{\text{Ra}}
\]

\[
0 = -\dot{s}_{\text{in}} - \dot{V}_0 + Q \geq 0
\]

\[
h_2 = \frac{h_1}{2}
\]

\[
h_4 = 52, \quad h_2 \quad \left( \frac{W}{s} \right)
\]

\[
h_4 = 32.75 \quad \left( \frac{W}{s} \right)
\]

\[
h_2 = h_2(-16^\circ C)
\]

\[
s_2 = s_2 - 0.0258 \frac{h_2}{s_k}
\]

\[
\frac{A}{10} = 237.7 \left( \frac{T}{s} \right)
\]

\[
(2.3)
\]

\[
p_2 = 1 \text{bar}
\]

\[
T_i = -10^\circ C
\]

\[
T_i = -10^\circ C
\]

``````latex


\[
A_{u}(s) = const
\]

\[
\begin{array}{ccc}
A_{12} & y_{sat} \\
1 & F_{sat}
\end{array}
\]

\[
\frac{S - S_{sat}(y_{b})}{S_{sat}(y_{a}) - S_{sat}(y_{b})} \cdot \left( h_{S}(s_{a}) - h_{u}(s) \right) - h_{u}(s)
\]

\[
t_{3} = 27,31
\]

\[
\dot{U}_{u} = \dot{m}_{E} (h_{2} - h_{3})
\]

\[
\dot{m}_{E} = \frac{\dot{m}_{u} - \dot{U}_{u}}{h_{2} - h_{3}} - 9.8 \frac{kg}{s}
\]

\subsection*{c) weiter mit seseren werten}

\[
x_{1}:
\]

\[
u_{1} = u_{u} = 32,70 \frac{u_{f}}{u_{s}}
\]

\[
x = \frac{u_{f} - u_{f}(-22^{\circ}C)}{u_{s}(-22^{\circ}C) - u_{f}(-22^{\circ}C)}
\]

\[
A = 10
\]

\[
1
\]

\[
0,3682
\]

\subsection*{d)}

\[
q_{u} = \frac{\dot{Q}_{u}}{\dot{W}_{t}} = \frac{\text{(scribbled out)}}{\text{(scribbled out)}} = 1,323
\]

\[
\dot{Q}_{u} = \dot{m}_{E} (h_{2} - h_{1}) = 15 \text{(scribbled out)} \left( \text{(scribbled out)} \right) W
\]

\[
\dot{u}_{s} / \left( \Delta h_{12} \right) =
\]

\[
h_{1} = h_{f}(-22^{\circ}C) + x \left( h_{s}(-22^{\circ}C) - h_{f}(-22^{\circ}C) \right)
\]

\[
= 32,9 \frac{u_{f}}{u_{s}}
\]

\[
u_{2} = 234,08 \frac{u_{f}}{u_{s}}
\]

``````latex


\section*{Aufgabe 1}

\subsection*{a)}
Er würde Sie würde bis \\
der kondensations temperatur \\
von R134a abkühlen etwa bis zu -22°C \\
und dann würde keine wärme mehr abgegeben \\
weil ein gleichgewicht erreicht ist

\subsection*{a2)}
\[
\begin{array}{c}
\text{Graph description:}
\end{array}
\]

The graph is a plot with the horizontal axis labeled \( T (u) \) and the vertical axis labeled \( p / N_{1/2} \). The plot shows a curve that starts at the origin, rises to a peak, and then falls back down, forming a bell-shaped curve. 

- The curve starts at the origin and rises steeply.
- There is a point labeled \( i \) at the peak of the curve.
- To the left of the peak, there is a point labeled \( i' \).
- To the right of the peak, there is another point labeled \( i'' \).
- The curve then descends symmetrically after the peak.

```