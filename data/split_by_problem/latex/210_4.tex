
``````latex


\section*{4. R 134a}

\subsection*{Graph 1}
The first graph is a phase diagram with pressure \( P \) on the y-axis (labeled in \([ \text{bar} ]\)) and temperature \( T \) on the x-axis (labeled in \([ \text{K} ]\)). The graph shows three distinct regions labeled as "fest" (solid), "flüssig" (liquid), and "gas" (gas). 

- The "fest" region is on the left side of the graph.
- The "flüssig" region is in the middle.
- The "gas" region is on the right side.

There are three lines separating these regions:
- The line between "fest" and "flüssig" is sloped upwards.
- The line between "flüssig" and "gas" is also sloped upwards but less steeply.
- The line between "fest" and "gas" is curved and intersects the other two lines at a point labeled "Tripel" (triple point).

\subsection*{Graph 2}
The second graph is another phase diagram with pressure \( P \) on the y-axis (labeled in \([ \text{bar} ]\)) and temperature \( T \) on the x-axis (labeled in \([ \text{K} ]\)). This graph also shows three regions labeled "fest" (solid), "flüssig" (liquid), and "gasförmig" (gaseous).

- The "fest" region is on the left side of the graph.
- The "flüssig" region is in the middle.
- The "gasförmig" region is on the right side.

There are three lines separating these regions:
- The line between "fest" and "flüssig" is sloped upwards.
- The line between "flüssig" and "gasförmig" is also sloped upwards but less steeply.
- The line between "fest" and "gasförmig" is curved and intersects the other two lines at a point labeled "Tripel" (triple point).

Additionally, there are two points labeled \( x_1 \) and \( x_2 \) on the "fest" to "flüssig" line, with a vertical arrow labeled \( i \) pointing upwards from \( x_2 \) to \( x_1 \). There is also a horizontal arrow labeled \( ii \) pointing from \( x_1 \) to a point labeled \( O \) on the "flüssig" to "gasförmig" line.

\subsection*{Graph 3}
The third graph is another phase diagram with pressure \( P \) on the y-axis (labeled in \([ \text{bar} ]\)) and temperature \( T \) on the x-axis (labeled in \([ \text{K} ]\)). This graph shows three regions but without labels.

- The leftmost region is separated by a line sloped upwards.
- The middle region is separated by a line sloped upwards but less steeply.
- The rightmost region is separated by a curved line.

The lines intersect at a point, forming a triple point similar to the previous graphs.

``````latex


\section*{b)}

\[
\dot{m}_{\text{R134a}}
\]

\begin{itemize}
    \item A circular diagram is drawn representing a system with an inlet and an outlet.
    \item The inlet is labeled with $x=1$ and an arrow pointing into the system.
    \item The outlet is labeled with $8 \text{bar}$ and an arrow pointing out of the system.
    \item Inside the circle, the word "isentrop" is written.
    \item Below the circle, the value $28 \text{W}$ is written.
\end{itemize}

\[
\Rightarrow s_2 = s_3
\]
\[
s_2 = s_g(T_2)
\]

\[
\text{1.HS st. FP:}
\]

\[
0 = \dot{m} [h_2 - h_3] + \cancel{\dot{Q}} - \dot{W}_k \quad \text{adiabat}
\]

\[
\dot{m} = \frac{\dot{W}_k}{h_2 - h_3}
\]

\[
T_{s1} = 31.33^\circ \text{C} \quad @ \quad 8 \text{bar}
\]

\section*{c)}

\[
x_n \quad \text{nach Drossel}
\]

\begin{itemize}
    \item A diagram of a throttle is drawn with an inlet and an outlet.
    \item The inlet is labeled with $8 \text{bar}$ and an arrow pointing into the throttle.
    \item The outlet is labeled with $p_1 = p_2$ and an arrow pointing out of the throttle.
    \item The points before and after the throttle are labeled as "4" and "1" respectively.
\end{itemize}

\[
\phi = \phi_f + x (\phi_g - \phi_f)
\]

\[
\text{Drossel isenthalp}
\]

\[
\Rightarrow h_4 = h_1
\]

\[
h_4 = 264.15
\]

\[
x = \frac{h_4 - h_{f1}}{h_{g1} - h_{f1}} = \frac{264.15 - 253.43}{272.02 - 253.43}
\]

\[
h_g(p_1) = h_f(p_4)
\]

\[
h_f \quad \text{aus AB} \quad @ \quad 8 \text{bar} = 264.15
\]

```