
``````latex


\section*{Aufgabe 3:}

\subsection*{a)}

\[ p_0 + p_1 = mg \]

Für kritische gg.wi:

\[
A_{Zyl} = 0.05^2 \pi = \frac{\pi}{100} = 7.85 \cdot 10^{-3}
\]

\begin{description}
    \item[Figure Description:] A horizontal cylinder is shown with an upward arrow labeled \( p_1 \cdot 0.05^2 \pi \) and a downward arrow labeled \( 1 \text{bar} \). Another downward arrow is labeled \( 32.1 \cdot 9.81 \).
\end{description}

\[
p_1 = \frac{32.1 \cdot 9.81}{0.05^2 \pi} + 1 \text{bar} = 400394.4 \, \text{Pa} + 1 \text{bar} = 0.400394 \, \text{bar} + 1 \text{bar}
\]

\[
= 1.400394 \, \text{bar}
\]

\subsection*{Ideales Gasgesetz:}

\[
M = \frac{pV}{RT} = \frac{1.400394 \cdot 10^5 \cdot 3.14}{8.314 \cdot (500 + 273.15)} = 3.14 \cdot 10^{-3} \, \text{kg}
\]

\[
= 0.00314 \, \text{kg}
\]

\subsection*{b)}

Der Druck bleibt, da Außendruck und Masse EW/kalben gleich: \( p_2 = 1.400394 \, \text{bar} \)

Masse d. Gases, Druck, Gasconst aber Volumen, somit auch Temperatur nimmt ab, da \( x_{Eis} \geq 0 \)

\[
\text{if temp von EW} = 0 \text{ und gas somit } T_{g,2} = 0
\]

``````latex


\section*{c)}

\textbf{1 HS um Gaskammer:} \\
geschlossen:

\[
m_{\text{gas}} (u_2 - u_1) = Q - W \quad \Rightarrow \quad PQ = -1361.57
\]

\[
W = p_0 \left[ V_2 - V_1 \right] = -285.47
\]

\[
V_2 = 0.0039 \cdot \frac{8.314}{50 \cdot 10^{-3}} \cdot 273.15 = 1.1023 \cdot 10^{-3}
\]

\[
V_1 = \ldots
\]

\[
u_2 = \left[ 273.15 + 500 \right] = 3.1203 \cdot 10^{-3} = 3.4 \, \text{L} \, (4 \, \text{L})
\]

\[
m_{\text{gas}} \left( u_2 - u_1 \right) = c^v \left( T_2 - T_1 \right) = 0.633 \cdot (-500)
\]

\[
-316.5 \, \frac{J}{kg}
\]

\[
\Rightarrow \quad \text{amess} = -1076.17
\]

\section*{d)}

\textbf{1 HS Eiskammer}

\[
u_1 = 0.6 \cdot (-333.458) + 0.9 \cdot (-0.045) = -200.083 \, \frac{J}{kg}
\]

\[
X_2 = \frac{-186.1771 + 0.045}{-333.458 + 0.045} = 0.559
\]

\[
\text{Messwasser} \left[ u_2 - u_1 \right] = 1361.57
\]

\[
h_2 = 1361.57 - 200.083 = -186.1771 \, \frac{J}{kg} \quad @ \, 0^\circ
\]

``````latex


