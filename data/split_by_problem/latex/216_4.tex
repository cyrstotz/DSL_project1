
``````latex


\section*{Aufgabe 4}

\subsection*{a)}

\begin{itemize}
    \item The first graph is a plot with the vertical axis labeled \( p \, \text{in bar} \) and the horizontal axis labeled \( T \, \text{in K} \). The graph shows a curve that starts at the origin, rises to a peak labeled "cut", and then falls back down.
    \item The second graph is a plot with the vertical axis labeled \( p \, \text{in bar} \) and the horizontal axis labeled \( T \, \text{in K} \). The graph shows a closed loop with four points labeled 1, 2, 3, and 4, forming a triangular shape.
\end{itemize}

\subsection*{b)}

\begin{itemize}
    \item The text reads: "2-3 \quad 1. \text{Adiabatischer Prozessprozess}"
    \item The equations are:
    \begin{align*}
        Q &= m \cdot (h_e - h_a + \frac{c}{2} + \frac{c}{2} + \frac{c}{2} - v) \\
        W &= m \cdot (h_e - h_a) = m (h_2 - h_3) \\
        m &= \frac{W}{h_e - h_a} \\
        h_2 &= \\
        h_3 &=
    \end{align*}
\end{itemize}

``````latex


c)

\begin{align*}
\text{in } R134a & : v_{kg} = \frac{4 \text{kg}}{V} \\
T & = -22^\circ C
\end{align*}

\text{im Drossel}

\begin{align*}
h_2 & = h_4
\end{align*}

\text{Tabelle A 11}

\begin{align*}
h_{g1} (8 \text{bar}, x = 0) & = 93,42 \frac{\text{kJ}}{\text{kg}} \\
h_3 (8 \text{bar}) & \\
s_2 & = s_3
\end{align*}

\begin{align*}
h_4 & = h_4 = h_{f1} + x (h_{g1} - h_{f1}) \\
x & = \frac{h_1 - h_{f2}}{h_{g2} - h_{f2}} = \frac{h_2 (-22^\circ C, x = 1)}{h_{g1} (-10^\circ C, x = 1)} = \frac{1}{400} \frac{\text{kg}}{\text{s}}
\end{align*}

\begin{align*}
EK & = \frac{Qzu}{V4} \\
& = 5787
\end{align*}

d) \text{Interpolation } h_{f1} \text{ und } h_{g1}

\begin{align*}
h_{f1} & = -20 + 12 \left( \frac{39,54 - 34,79}{20 + 34,39} \right) = 36,965 \frac{\text{kJ}}{\text{kg}} \\
h_{g1} & = -10 + 17 \left( \frac{209,57 - 203}{20 + 203} \right) = 704,389 \frac{\text{kJ}}{\text{kg}}
\end{align*}

e) \text{Würde zuerst ohne die Drossel fließen zu lassen, dann erst geöffnet werden, dann Temperatur würde sich ändern.}

```