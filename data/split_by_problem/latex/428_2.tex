
``````latex


\section*{Aufgabe 2:}

\subsection*{a) T-s Diagramm}

\begin{description}
    \item[Graph 1:] The graph is a T-s diagram with temperature \( T \) on the vertical axis and entropy \( s \) on the horizontal axis. The graph consists of six points labeled 1 through 6. The points are connected by lines forming a closed loop. The lines between points 1 and 2, 2 and 3, 3 and 4, 4 and 5, 5 and 6, and 6 and 1 are drawn. The lines between points 1 and 2, 2 and 3, and 3 and 4 are curved, while the lines between points 4 and 5, 5 and 6, and 6 and 1 are straight. The region between points 1 and 2 is shaded with diagonal lines.
    \item[Graph 2:] The second graph is a P-v diagram with pressure \( P \) on the vertical axis and specific volume \( v \) on the horizontal axis. The graph consists of six points labeled 1 through 6. The points are connected by lines forming a closed loop. The lines between points 1 and 2, 2 and 3, 3 and 4, 4 and 5, 5 and 6, and 6 and 1 are drawn. The lines between points 1 and 2, 2 and 3, and 3 and 4 are curved, while the lines between points 4 and 5, 5 and 6, and 6 and 1 are straight. The region between points 1 and 2 is shaded with diagonal lines.
\end{description}

\subsection*{b)}

\begin{align*}
    \text{Luft als ideales Gas:} & \quad c_p \text{mittl.} = 1.006 \frac{\text{kJ}}{\text{kg K}}, \quad n = k = 1.4. \\
    & \quad P_0 = P_6 = 0.956 \text{bar} \\
    5-6 \text{ revers. + adiabate} & \Rightarrow \text{isentrop.} \Rightarrow \left( \frac{T_6}{T_5} \right) = \left( \frac{P_6}{P_5} \right)^{\frac{n-1}{n}} \\
    & \quad \left( \frac{0.956 \text{bar}}{0.5 \text{bar}} \right)^{\frac{0.4}{1.4}} = 431.8 \text{K} \\
    & = 328.074 \text{K} \\
    n = k = 1.4 & \Rightarrow \frac{c_p}{c_v} \Rightarrow c_v = \frac{c_p}{k} = \frac{1.006 \frac{\text{kJ}}{\text{kg K}}}{1.4} = 0.7186 \frac{\text{kJ}}{\text{kg K}} \\
    R & = c_p - c_v = 1.006 \frac{\text{kJ}}{\text{kg K}} - 0.7186 \frac{\text{kJ}}{\text{kg K}} = 0.2874 \frac{\text{kJ}}{\text{kg K}}
\end{align*}

``````latex


Luft als ideales Gas:

\begin{equation}
p \cdot V = R \cdot T
\end{equation}

\begin{equation}
\Rightarrow p_0 \cdot V_0 = R \cdot T_0
\end{equation}

\begin{equation}
p_6 \cdot V_6 = R \cdot T_6
\end{equation}

\begin{equation}
\Rightarrow \frac{p_0}{R \cdot T_0} = \frac{p_6}{R \cdot T_6}
\end{equation}

\begin{equation}
\Rightarrow \frac{p_0}{p_6} = \frac{T_0}{T_6}
\end{equation}

\begin{equation}
\frac{p_6}{R \cdot T_6} = \frac{p_0}{R \cdot T_0}
\end{equation}

\begin{equation}
\Rightarrow \frac{p_6}{T_6} = \frac{p_0}{T_0}
\end{equation}

da \(\dot{m} = \rho \cdot A \cdot w\) und \(M_6 = M_0\), \(A_6 = A_0\)

\begin{equation}
w = \frac{\dot{m}}{\rho \cdot A} \Rightarrow w_6 = \frac{\dot{m}_6}{\rho_6 \cdot A_6}
\end{equation}

\begin{equation}
w_0 = \frac{\dot{m}_0}{\rho_0 \cdot A_0}
\end{equation}

\begin{equation}
\Rightarrow w_6 = w_0 \cdot \frac{\rho_0}{\rho_6}
\end{equation}

\begin{equation}
\rho_0 = \frac{p_0}{R \cdot T_0} = \frac{0.1596 \, \text{kg}}{0.2874 \, \frac{\text{kg}}{\text{K}} \cdot 294.15 \, \text{K}} = 2.7332
\end{equation}

\begin{equation}
\rho_6 = \frac{p_6}{R \cdot T_6} = \frac{0.1596 \, \text{kg}}{0.2874 \, \frac{\text{kg}}{\text{K}} \cdot 320.47 \, \text{K}} = 2.0217
\end{equation}

\begin{equation}
\Rightarrow w_6 = w_0 \cdot \frac{\rho_0}{\rho_6}
\end{equation}

``````latex


