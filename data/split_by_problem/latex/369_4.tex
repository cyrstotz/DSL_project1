
``````latex


\section*{Aufgabe 4}

\subsection*{a)}

\begin{center}
\textbf{Verbal Description of the Graph:}

The graph is a phase diagram with pressure \( P \) on the y-axis and temperature \( T \) on the x-axis. The y-axis is labeled with \( P \) in \([ \text{bar} ]\) and \([ \text{Pa} ]\). The x-axis is labeled with \( T \) in \([ K ]\). 

The graph shows three regions labeled "Fest" (solid), "Flüssig" (liquid), and "Gas" (gas). The boundary between the solid and liquid regions is a curve starting from the y-axis and moving upwards to the right. The boundary between the liquid and gas regions is another curve starting from the x-axis and moving upwards to the right, ending at a point labeled "Kritischer Punkt" (critical point). 

There is a point labeled "Tripel Punkt" (triple point) where the three regions meet. 

Two processes are shown on the graph:
1. A horizontal line labeled "IsoTherm" (isothermal) from point I to point II.
2. A vertical line labeled "IsoTherm druckabnahme" (isothermal pressure decrease) from point III to point II.

\end{center}

\subsection*{b)}

\begin{align*}
&\dot{m} \quad \text{stationär:} \\
&\dot{W}_k = 28 \, \text{W} \\
&Q = \dot{m} (h_e - h_a) \quad t \cdot l - \dot{W}_k \\
&\dot{W}_k = 28 \, \text{W} = \dot{m} (h_2 - h_3) \\
&h_e \Rightarrow x_2 = ? \quad \text{gesättigt gedämpft} \\
&h_a \\
&h_a = p = 3 \, \text{bar} \quad \Rightarrow \text{enthalpi resoribel} \quad os = 0 \quad s_2 = s_3 \\
&p_3 = p_{\text{max}} \\
&p_4 = p_{\text{vor}} \quad x_e \\
&h_4 = 93.62 \, \frac{\text{kJ}}{\text{kg}} \\
&h_1 = h_e \quad h_2 = \\
&\text{Trockendampf} = \quad \text{T}_w = -4 \, ^\circ \text{C} \\
&\text{T}_e = -4 \, ^\circ \text{C}
\end{align*}

``````latex


\section*{c)}
\begin{align*}
    \dot{m} &= \frac{\dot{Q}}{h_1} \\
    T_2 &= -22^\circ C \\
    x &= 0 \\
    h_f &= h_1 = 93.62 \, \text{kJ/kg} \\
    T_4 &= 37.37^\circ C \\
    T_7 &= ?
\end{align*}
Temperatur im Verdampfer = -4^\circ C

\section*{d)}
\begin{equation*}
    \varepsilon_K = \frac{E_{\text{nutz}}}{E_{\text{zu}}} = \frac{\text{Wärme}}{Q_{\text{ab}}}
\end{equation*}

\section*{e)}
Die Temperatur würde um 6K sinken, bis sie gleich der Temperatur im Kondensator ist.

```