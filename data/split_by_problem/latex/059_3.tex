
``````latex


\section*{Aufgabe 3}

\subsection*{a)}
\begin{align*}
    A &= r^2 \pi \\
    r &= 0{,}05 \, \text{m} \\
    A &= 7{,}854 \cdot 10^{-3} \, \text{m}^2 \\
    P_{k} &= \frac{F}{A} \\
    F &= m \cdot g \\
    P_{k} &= \frac{32 \cdot 9{,}81}{7{,}854 \cdot 10^{-3}} = 39369{,}53 \, \text{Pa} \approx 0{,}399 \, \text{bar} \\
    P_{ew} &= \frac{m_{ew} \cdot 9{,}81}{A} = \frac{0{,}1 \cdot 9{,}81}{7{,}854 \cdot 10^{-3}} = 124{,}00 \, \text{Pa} \approx 1{,}249 \cdot 10^{-3} \, \text{bar} \\
    P_{g1} &= P_{k} + P_{ew} + P_{amb} = 1{,}4 \, \text{bar} \\
    P_{g1} V_{g1} &= \frac{m_{g} R T}{M} \\
    m_{g} &= \frac{P_{g1} V_{g1} M}{R T} = \frac{1{,}4 \cdot 10^5 \, \text{Pa} \cdot 3{,}14 \cdot 10^{-3} \, \text{m}^3 \cdot 50 \, \text{kg/kmol}}{8{,}314 \cdot (500 + 273{,}15)} \\
    m_{g} &= 0{,}818 \, \text{g}
\end{align*}

\subsection*{b)}

\subsection*{c)}
\begin{align*}
    \Delta Q &= m_{g} \cdot C_p \cdot \Delta T \\
    \Delta Q &= 3{,}6 \cdot 10^{-3} \cdot 0{,}8 \cdot (T_2 - T_1) \\
    &= 3{,}6 \cdot 10^{-3} \cdot 0{,}8 \cdot (0{,}003 \degree \text{C} - 500 \degree \text{C}) \\
    &= -1{,}4393 \, \text{J}
\end{align*}

\begin{align*}
    C_p &= \frac{R}{M} + C_v \\
    &= \frac{8{,}314 \, \text{kJ}}{50 \, \text{kg/kmol} \cdot \text{K}} + 0{,}633 \\
    C_p &= 0{,}793 \approx 0{,}8 \, \frac{\text{kJ}}{\text{kg} \cdot \text{K}}
\end{align*}

\subsection*{d)}
\begin{align*}
    u_2 &= u_f + x (u_{est} - u_f) \\
    \Delta U &= \Delta Q \\
    u_2 (T_2) - u_f (T_1) &= C_r (T_2 - T_1) \\
    T_1 &= 0 \degree \text{C} \\
    T_2 &= 0{,}003 \degree \text{C} \\
    u_2 (T_2) - u_f (T_1) &= \frac{\Delta Q}{m_w} = \frac{+1{,}5 \, \text{kJ}}{0{,}1 \, \text{kg}} = 15 \, \frac{\text{kJ}}{\text{kg}} \\
    u_f (T_1) &= u_f + x_1 (u_{est} - u_f) = -0{,}045 + 0{,}6 \cdot (-333{,}158 + 0{```latex


\begin{equation*}
x_2 = \frac{(U_2 - U_x)}{(U_{\text{rest}} - U_y)} = \text{(unreadable scribbles)}
\end{equation*}

\begin{equation*}
U_2 = 15 \frac{\text{kJ}}{\text{kg}} + U_1(T) = 15 \frac{\text{kJ}}{\text{kg}} - 200 \frac{\text{kJ}}{\text{kg}} = -185,0 \frac{\text{kJ}}{\text{kg}}
\end{equation*}

\begin{equation*}
x_2 = \frac{(-185 + 0,033)}{(-333,442 + 0,033)} = 0,554
\end{equation*}

\begin{flushleft}
\text{Theoretisch nicht möglich,}\\
\text{weil, wenn Eis gibt, 0°C}\\
\text{die Temperatur sollte sein.}
\end{flushleft}

``````latex


