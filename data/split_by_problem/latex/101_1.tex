
``````latex


\section*{2)}

\begin{tabular}{|c|c|c|c|}
\hline
 & T & S & \\
\hline
1 & $<p_c$ & & \\
\hline
2 & & & $S_2 = S_1$ \\
\hline
3 & $>T_2$ & & \\
\hline
4 & & & \\
\hline
5 & 0,15 bar & 431,9 & \\
\hline
6 & & & $S_5 = S_6$ \\
\hline
\end{tabular}

\bigskip

\textbf{Graph Description:}

The graph is a Temperature-Entropy (T-S) diagram. The x-axis is labeled $S \left[\frac{kJ}{kg \cdot K}\right]$ and the y-axis is labeled $T [K]$. The graph consists of a cycle with the following points:

- Point 1: Starting point on the isentrope line.
- Point 2: Moves vertically up along the isentrope.
- Point 3: Moves horizontally right along the isotherm.
- Point 4: Moves vertically down along the isobar.
- Point 5: Moves diagonally down to the left along the isobar.
- Point 6: Moves vertically down along the isentrope back to point 1.

The graph includes labels for "isotherm", "isobar", "isentrope", and "adiabat". There is also a note indicating "steiler als isobar" (steeper than isobar).

\section*{b)}

\begin{align*}
\omega_5 &= 220 \frac{m}{s} \\
p_5 &= 0,5 \text{ bar} \\
T_5 &= 431,9 \text{ K} \\
\\
\text{isentrope} & \quad 5-6 \\
\\
\frac{T_6}{T_5} &= \left( \frac{p_6}{p_5} \right)^{\frac{n-1}{n}} \quad n = k \\
\\
T_6 &= \left( \frac{0,131 \text{ bar}}{0,5 \text{ bar}} \right)^{\frac{0,4}{1,4}} \cdot 431,9 \text{ K} = 328,07 \text{ K} = T_6 \\
\\
\dot{m} \left( h_5 - h_6 + \frac{\omega_5^2 - \omega_6^2}{2} \right) + \dot{Q} - \dot{Q} = \dot{m} \left( h_6 - h_5 \right) + \frac{\omega_5^2 - \omega_6^2}{2}
\end{align*}

``````latex


c)

\[
0 = m_{\text{ges}} \left( h_1 - h_0 - T_0 (s_1 - s_0) \right) = 0
\]

``````latex


