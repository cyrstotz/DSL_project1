
``````latex


\section*{Student Solution}

\subsection*{2 a)}

\begin{description}
    \item[Graph 1:] The graph is a Pressure-Volume (P-V) diagram. The x-axis is labeled $V$ and the y-axis is labeled $p$. There are several curves and lines:
    \begin{itemize}
        \item A straight line labeled $2$ starting from the origin and extending diagonally upwards to the right.
        \item A curve labeled $P_0$ starting from the bottom left and curving upwards to the right.
        \item Another curve labeled $P_2$ above $P_0$.
        \item A point labeled $1$ on the $P_0$ curve.
        \item A point labeled $2$ on the $P_2$ curve.
        \item A vertical dashed line connecting points $1$ and $2$.
        \item A point labeled $3$ on the $P_2$ curve, above point $2$.
        \item A point labeled $4$ on the $P_2$ curve, above point $3$.
        \item A point labeled $5$ on the $P_2$ curve, above point $4$.
        \item A horizontal dashed line connecting points $3$, $4$, and $5$.
        \item The word "isobar" written near the top right of the graph.
    \end{itemize}
    
    \item[Graph 2:] The graph is a Temperature-Volume (T-V) diagram. The x-axis is labeled $V$ and the y-axis is labeled $T$. There are several curves and lines:
    \begin{itemize}
        \item A curve labeled $P_0$ starting from the bottom left and curving upwards to the right.
        \item Another curve labeled $P_2$ above $P_0$.
        \item A curve labeled $P_5$ above $P_2$.
        \item A point labeled $1$ on the $P_0$ curve.
        \item A point labeled $2$ on the $P_2$ curve.
        \item A point labeled $3$ on the $P_2$ curve, above point $2$.
        \item A point labeled $4$ on the $P_2$ curve, above point $3$.
        \item A point labeled $5$ on the $P_5$ curve, above point $4$.
        \item A vertical dashed line connecting points $1$, $2$, $3$, $4$, and $5$.
        \item A point labeled $6$ on the $P_0$ curve, below point $1$.
        \item A horizontal dashed line connecting points $1$ and $6$.
    \end{itemize}
\end{description}

\subsection*{b)}

\begin{align*}
    & w_{e1}, T_{e1} \\
    & 0 = \dot{m} (h_5 - h_6 + \frac{w_5^2}{2} - \frac{w_6^2}{2}) + \dot{W}_{56} \\
    & \text{mit} \quad \dot{W}_{56} = \dot{m} \int_{5}^{6} v \, dp + \Delta pe \quad \text{isobar} \\
    & \Rightarrow h_5 - h_6 + \frac{w_5^2 - w_6^2}{2} = 0 \\
    & \Rightarrow w_5^2 = 2 \cdot (h_6 - h_5) + w_6^2 \\
    & = \sqrt{2 \cdot c_p (T_{e6} - T_{e5}) + w_6^2} \\
    & T_{e6} = T_5 \cdot \left( \frac{P_6}{P_5} \right)^{\frac{n-1}{n}} = 328.07 \, K
\end{align*}

``````latex


