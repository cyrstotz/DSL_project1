
``````latex


\section*{Aufgabe 3}

\subsection*{a)}
\begin{align*}
p_{s1, A} &= p_{amb} + p_{m1} + p_{m, ew} - p_{amb} + \frac{F_G}{A} \\
&= 1 \text{bar} + \frac{9.81 (0.15 \text{kg} + 32 \text{kg})}{\frac{\pi}{4} \left( \frac{d}{2} \right)^2} \\
&= 1 \text{bar} + 40034.2 \text{Pa} = 1.400342 \text{bar} = p_{s1, A}
\end{align*}

\begin{align*}
p_{s1} V_{s1} &= m_{s1} R T_{s1} \\
R &= 8314 \, \text{J/(kmol K)} = 166.28 \, \text{J/(kg K)} \\
p V &= R T \\
m_{s1} &= \frac{(166.28 \cdot 773.15 \, \text{K})}{1.400342 \cdot 10^5 \, \text{Pa} \cdot 3.14 \cdot 10^{-2} \, \text{m}^3} = 0.003426 \, \text{kg} \\
m_{s1} &= 3.42 \, \text{g}
\end{align*}

\subsection*{b)}
\begin{align*}
X_{\text{Eis2}} &> 0 \rightarrow T_{\text{Ew2}} = T_{\text{Ew1}} = 0^\circ \text{C} \\
\text{Therm. GW} &\Rightarrow Q = 0 \Rightarrow \Delta T_{G2} = 0^\circ \text{C} = T_{\text{Ew}} \\
p_{G2} &\Rightarrow Gewicht auf Kolben ändert sich nicht \\
p_{G2} &= p_{s1, A} = 1.400342 \, \text{bar}
\end{align*}

\begin{quote}
Da der Eisanteil größer als komplett bedeutet, dass die Eisküsse-Temperatur gleich bleibt und zudem im Zustand 2 keine Wärmeströme mehr existieren muss, die zweite Gas-Temperatur 0°C sein. Zudem bleibt der Druck im Gas konstant, da die Kolbenkraft konstant bleibt.
\end{quote}

\subsection*{c)}
\begin{align*}
V_{G2} &= \frac{m_{G}}{\rho} = \frac{0.003426 \cdot 166.28 \cdot 273.15 \, \text{K}}{1.400342 \, \text{bar}} = 0.0071 \, \text{m}^3 = 7.1 \, \text{L} \\
\Delta T &= 500 \, \text{K}
\end{align*}

\subsubsection*{Energiebilanz}
\begin{align*}
\Delta U &= \dot{Q}_{12} - W_v \\
\Delta u &= u_2 - u_1 = m_s c_v (T_2 - T_1) \\
V_v &= \int p \, dV = p_G (V_2 - V_1)
\end{align*}

\begin{align*}
q + \dot{Q}_{12} &= \cancelto{0}{\Delta U} + W_v \\
&= 0.003426 \cdot 633 \frac{\text{J}}{\text{kgK}} \cdot 500 \text{K} + 1.400342 \cdot 10^5 \text{Pa} \cdot (-0.00204 \text{m}^3) \\
&= 1388.2 \, \text{J} = \dot{Q}_{12}
\end{align*}

\begin{quote}
EV gewinnt an Q \Rightarrow Q_{12} \rightarrow 0 \Rightarrow Q_{m2} = 1388.2 \, \text{J}
\end{quote}```latex


d)
\[
p_{EV} = 16 \text{bar} + \frac{32 \cdot 9.81}{(0.05)^2 \pi} = 1.46 \text{bar} = p_{sat}
\]
\[
\theta_{1,sat} = 0^\circ C
\]

Energiebilanz:
\[
E_1 = E_2 + Q_{12}
\]
\[
\Delta E = Q_{12} \quad \text{mit} \quad \Delta E = E_1 - E_2
\]

\[
\Delta E = (0.6 \cdot u_{test} + 0.4 \cdot u_{liquid}) \cdot 1.0 \, \text{kg} - \left( (1 - x_2) \cdot u_{test} + x_2 \cdot u_{liquid} \right) \cdot 0.2 \, \text{kg}
\]

\[
= 0.2 \left( -200.1 \right) - \left( -333.458 + x_2 (333.493) \right)
\]

\[
= -13.34 - x_2 \cdot 33.34 = Q_{12} = 1368.2
\]

``````latex


