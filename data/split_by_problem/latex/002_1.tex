
``````latex


\section*{Aufgabe 1}

\textbf{Gegeben:}
\begin{align*}
    \dot{m} &= 0{,}3 \, \frac{\text{kg}}{\text{s}} \\
    m_{\text{Holz}} &= 5{,}775 \, \text{kg} \\
    x_0 &= 0{,}05 \\
    \dot{Q}_R &= 100 \, \text{kW}
\end{align*}

\begin{align*}
    T_{\text{ein}} &= 70^\circ \text{C} = 343{,}15 \, \text{K} \\
    T_{\text{aus}} &= 100^\circ \text{C} = 373{,}15 \, \text{K} \\
    T_{\text{reakt}} &= 100^\circ \text{C} = 373{,}15 \, \text{K}
\end{align*}

\begin{align*}
    T_1 &= 298{,}15 \, \text{K} \\
    T_{\text{Kreis, ein}} &= 298{,}15 \, \text{K} \\
    T_{\text{Kreis, aus}} &= 298{,}15 \, \text{K}
\end{align*}

\textbf{Diagramm:}

Es gibt ein Rechteck, das als "Kühlmantel" bezeichnet wird. Innerhalb des Rechtecks gibt es einen Pfeil, der nach unten zeigt und als $\dot{Q}_{\text{aus}}$ beschriftet ist. 

Links vom Rechteck gibt es einen Pfeil, der nach rechts zeigt und als $\dot{m}_{\text{ein}}$ beschriftet ist. Dieser Pfeil ist mit $T_1 = 298{,}15 \, \text{K}$ beschriftet. 

Rechts vom Rechteck gibt es einen Pfeil, der nach rechts zeigt und als $\dot{m}_{\text{Kühl, aus}}$ beschriftet ist. Dieser Pfeil ist mit $T_2 = 298{,}15 \, \text{K}$ beschriftet. 

Über dem Rechteck steht $p_2 = p_1$ und unter dem Rechteck steht $p_1 - p_2$.

\textbf{Perfekte Flüssigkeit:}

\textbf{Energiebilanz um Kühlmantel:}
\begin{align*}
    0 &= \dot{m} \left[ h_1 - h_2 \right] + Q - \cancel{0}
\end{align*}

\textbf{st.}
\begin{align*}
    Q_{\text{ab}} &= \dot{m} \left[ h_2 - h_1 \right] \\
    &= \dot{m} \left[ c_i \cdot dT \right] \text{ für } \left( p_2 / p_1 \right) \\
    &= -m \, c_i \left( T_c - T_h \right)
\end{align*}

\textbf{Energiebilanz um Reaktor:}
\begin{align*}
    0 &= \dot{m} \left[ h_e - h_a \right] + \dot{Q}_R + \dot{Q}_{\text{Ab}} - \cancel{0}
\end{align*}

\begin{align*}
    Q_{\text{ab}} &= m \left[ h_a - h_e \right] - \dot{Q}_R
\end{align*}

``````latex


