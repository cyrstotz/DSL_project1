
``````latex


4.a)

\textbf{Description of the graph:} The graph shows a plot with the vertical axis labeled \( p \) and the horizontal axis labeled \( T \). There are two main curves in the graph. The first curve is a downward sloping straight line. The second curve is a wavy line that starts from the origin, rises and falls in a sinusoidal manner, and intersects the straight line at several points. There is also a shaded region near the horizontal axis on the right side of the graph.

\[
\begin{array}{c}
\includegraphics[width=0.5\textwidth]{graph1.png}
\end{array}
\]

\textbf{Description of the second graph:} The second graph is simpler, with the vertical axis labeled \( P \) and the horizontal axis labeled \( T \) with the unit \([K]\). There are no curves or lines drawn in this graph, only the axes are shown.

\[
\begin{array}{c}
\includegraphics[width=0.5\textwidth]{graph2.png}
\end{array}
\]

b) \( T_a < T_i \)

\[
0 = \dot{m}_{R134a} \left[ h_2 - h_3 \right] + \dot{W}_a
\]

\[
\dot{m}_{R134a} = -\frac{\dot{V}_a}{h_2 - h_3} = \frac{\dot{V}_K}{h_3 - h_2}
\]

\[
S_3 = S_2 \quad (\text{weil adiabatisch + reversibel})
\]

d)

\[
E_K = \frac{\left| \dot{Q}_m \right|}{\left| \dot{V}_K \right|} = \frac{\left| \dot{Q}_1 \right|}{\left| \dot{V}_K \right|}
\]

\[
\dot{Q}_K = \dot{Q}_x
\]

\[
0 = \dot{m}_{R134a} \left( h_1 - h_2 \right) + \dot{Q}_x \quad \Rightarrow \quad \dot{Q}_x = \dot{m} \left( h_2 - h_1 \right)
\]

```