
``````latex


\section*{Problem 2}

\subsection*{a)}

\begin{description}
    \item[Left Graph:] 
    The graph is a Temperature (T) vs. Entropy (S) diagram. The x-axis is labeled as S and the y-axis is labeled as T. There are several curves and lines:
    \begin{itemize}
        \item A black curve starting from the origin and curving upwards to the right.
        \item A blue curve starting from the origin, curving upwards, and then sharply turning downwards.
        \item A straight black line with a negative slope intersecting the blue curve.
        \item A point labeled $P_0$ on the straight black line.
        \item A vertical blue line from the x-axis to the intersection point of the blue curve and the straight black line.
    \end{itemize}
    
    \item[Right Graph:]
    The graph is also a Temperature (T) vs. Entropy (S) diagram. The x-axis is labeled as S and the y-axis is labeled as T. There are several curves and lines:
    \begin{itemize}
        \item A straight line with a positive slope.
        \item A blue curve starting from the origin, curving upwards, and then sharply turning downwards.
        \item A point labeled $P_0$ on the straight line.
        \item A vertical blue line from the x-axis to the intersection point of the blue curve and the straight line.
    \end{itemize}
\end{description}

\subsection*{b)}

\[
0 = \dot{m}(h_1) + \frac{\dot{V}^2}{2} \bigg|_{\text{isobar}}
\]

\[
0 = -\dot{m}(h_0 - h_6) + \frac{\omega_1^2}{2} - \frac{\omega_6^2}{2} \Rightarrow -\dot{m}(C_p^g(T_0 - T_6) + \frac{\omega_0^2}{2} - \frac{\omega_2^2}{2})
\]

\[
\frac{\omega_1^2}{2} - \frac{\omega_2^2}{2} = -C_p^g(T_0 - T_6)
\]

\[
\omega_1 = \sqrt{\omega_0^2 - 2C_p(T_0 - T_6)}
\]

\[
\omega_1 = 200.48 \, \text{m/s}
\]

\[
mRT = P \cdot V
\]

\[
\dot{m}RT_1 = \frac{mR T_2}{V_2} \bigg|_{\text{isobar}} \quad P_0 \frac{V_1}{T_0}
\]

\[
\frac{T_6}{T_0} = \left( \frac{P_6}{P_0} \right)^{\frac{n-1}{n}} T_0 = T_6 = 340
\]

\subsection*{c)}

\[
\dot{e}_{\text{ext}} = (h_6 - h_0 - T_0(S_6 - S_0) + ke)
\]

\[
= C_p(T_6 - T_0) - T_0 \left( C_p \ln \left( \frac{T_6}{T_0} \right) - R \ln \left( \frac{P_6}{P_0} \right) \right) + \frac{\omega_6^2}{2} - \frac{\omega_0^2}{2}
\]

\[
= \ldots
\]

``````latex


\section*{Problem d}

\[
0 = \dot{e}_{\text{exstr}} + \dot{e}_{xq}^0 - W_n(t) - \dot{E}_{\text{exvel}}
\]

\[
\dot{e}_{\text{exvel}} = \dot{e}_{\text{exstr}} - W_n(t) = 11050 \, \frac{\text{kJ}}{\text{kg}}
\]

\[
W_n(t) = \text{Prozverdichter}
\]

\[
W_n = -\int \dot{v} dp + \Delta \text{ke}
\]

\[
= -\left( \frac{\omega_2^2}{2} - \frac{\omega_6^2}{2} \right) = \frac{1}{2} (\omega_2^2 - \omega_6^2) = -110050 \, \text{kJ}
\]

\section*{Problem b}

\[
\frac{T_6}{T_5} = \left( \frac{p_6^0}{p_5^0} \right)^{\frac{n-1}{n}} \implies T_6 = T_5 \left( \frac{p_6^0}{p_5^0} \right)^{\frac{n-1}{n}} = 328.07 \, \text{K}
\]

\[
0 = \dot{m}(h_6) + \dot{e}_{xq}^0 - W
\]

\[
W^0 = \dot{m} \left( h_5 - h_6 + \frac{\omega_5^2}{2} - \frac{\omega_6^2}{2} \right)
\]

\[
c_p (T_5 - T_6) = \frac{\omega_6^2}{2} - \frac{\omega_5^2}{2} \implies \omega_6 = \sqrt{2 c_p (T_5 - T_6) + \frac{\omega_5^2}{2}} = 220.47 \, \frac{\text{m}}{\text{s}}
\]

\[
\text{adiabat \& reversibel : isentrop} \implies s_6 = s_5
\]

\[
0 = \int_{T_5}^{T_6} \ln \left( \frac{T_6}{T_5} \right) c_p - R \ln \left( \frac{p_6^0}{p_5^0} \right)
\]

\[
\implies T_5 = T_6
\]

\section*{Problem 2a}

The diagram is a Temperature (T) vs. Entropy (S) graph. The x-axis represents entropy (S) and the y-axis represents temperature (T). There are several points labeled on the graph:

- Point 1 is at the bottom left.
- Point 2 is slightly to the right and above point 1.
- Point 3 is further to the right and above point 2.
- Point 4 is to the right of point 3.
- Point 5 is above point 4.
- Point 6 is to the left of point 5.

There are curves connecting these points:

- A curve from point 1 to point 2.
- A curve from point 2 to point 3.
- A curve from point 3 to point 4.
- A curve from point 4 to point 5.
- A curve from point 5 to point 6.

There is a vertical line labeled "C" between points 5 and 6. The region between points 1, 2, and 3 is shaded with diagonal lines. The region between points 3, 4, and 5 is shaded with horizontal lines. The region between points 5 and 6 is shaded with vertical lines.

``````latex


