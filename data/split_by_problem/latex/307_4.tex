
``````latex


\section*{Aufgabe 4:}

\subsection*{a)}

\begin{description}
    \item[Graph Description:] The graph is a phase diagram with the y-axis labeled as "p" and the x-axis labeled as "T [°C]". The y-axis represents pressure and the x-axis represents temperature. The graph shows three regions labeled "solid", "liquid", and "gas". There are three points labeled "1", "2", and "3". Point "1" is in the liquid region, point "2" is in the solid region, and point "3" is in the gas region. There is a line connecting points "1" and "2" labeled "isobar", and a line connecting points "2" and "3" labeled "isotherm". The line separating the solid and liquid regions is labeled "Schmelzlinie", and the line separating the liquid and gas regions is labeled "Siedelinie".
\end{description}

\subsection*{b)}

2-3:

\[
\dot{W}_K = 28 \, \text{W}
\]

\[
p_3 = 8 \, \text{bar}
\]

Energiebilanz:

\[
0 = \dot{m} (h_2 - h_3) + \dot{Q}_{23} \rightarrow 0, \text{adiabat}
\]

\[
\dot{m} (h_2 - h_3) = \frac{\dot{W}_K}{h_2 - h_3}
\]

Zustand 3: flüssig: TAB 4 HÜ: \(T_3 = 31.33 \, \degree \text{C}\)

\[
h_3 = 93.62 \, \frac{\text{kJ}}{\text{kg}}, \quad s_3 = 0.3459 \, \frac{\text{kJ}}{\text{kg K}}
\]

``````latex

\section*{Handwritten Student Solution}

\subsection*{a)}
\textit{Der Zustand 2: gesättigten Dampf}

\[ T_4 = T_1 = 31.33^\circ C = T_3 \]

\subsection*{c)}
\[ P_4 = P_3 = 8 \text{ bar} \]
\textit{und vollständig kondensiert}

``````latex


d) 
\[
\varepsilon_{\text{N}} = \frac{| \dot{Q}_{zu} |}{| \dot{W} |} - \frac{| \dot{Q}_{ab} |}{| \dot{Q}_{zu} | - | \dot{Q}_{ab} |}
\]

\[
\dot{Q}_{ab} = \dot{m}_{R134a} (h_3 - h_1)
\]

\[
0 = \dot{m}_{R134a} (h_3 - h_2) + \dot{Q}_{ab} \quad \rightarrow \quad 0 \rightarrow \dot{Q}_{ab} = \dot{m}_{R134a} (h_4 - h_3)
\]

\[
0 = \dot{m}_{R134a} (h_1 - h_2) + \dot{Q}_{ab} \quad \rightarrow \quad \dot{Q}_{ab} = \dot{m}_{R134a} (h_2 - h_1)
\]

\[
\varepsilon_{\text{N}} = \frac{| h_3 - h_1 |}{| h_3 - h_1 | - | h_2 - h_1 |}
\]

e) T im Inneren wird tiefer werden aber immer weniger, kann nicht unendlich nach tief gehen.

(so ein Verlauf:)

\begin{description}
    \item[Graph Description:] The graph is a plot with the x-axis labeled as $t$ and the y-axis labeled as $T$. The curve starts at a high value on the y-axis and decreases steeply at first, then gradually levels off as it moves to the right along the x-axis, indicating that the temperature $T$ decreases over time $t$ but at a decreasing rate.
\end{description}

```