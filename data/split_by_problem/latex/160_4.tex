
``````latex


\section*{A4}

\subsection*{a) Wasser der Lebensmittel}

\begin{tabular}{|c|c|c|}
\hline
p & T & \\
\hline
1 & $p_1 = p_2$ & $> T_i$ \\
2 & $p_2 = p_3$ & $T_i$ \\
3 & $p_3 < f(T,p)$ & $T_i$ \\
\hline
\end{tabular}

\subsection*{Graphical Descriptions}

\textbf{First Graph:}

The first graph is a pressure-temperature ($p$-$T$) diagram. The x-axis is labeled $T$ [°C] and the y-axis is labeled $p$ [mbar]. The y-axis has a logarithmic scale with values marked at 0.1, 1, and 10. 

- There is a curve starting from the bottom left, labeled "Wasser" (water), which rises steeply and then levels off as it moves to the right.
- Another curve, labeled "Eis" (ice), starts from the bottom left and rises more gently, intersecting the "Wasser" curve at a point labeled "Tripel" (triple point).
- From the "Tripel" point, a line extends horizontally to the right, labeled "Dampf" (steam).
- The graph is divided into three regions: "Eis" (ice) below the "Wasser" curve, "Wasser" (water) between the "Wasser" and "Dampf" curves, and "Dampf" (steam) above the "Dampf" curve.

\textbf{Second Graph:}

The second graph is another pressure-temperature ($p$-$T$) diagram. The x-axis is labeled $T$ [°C] and the y-axis is labeled $p$ [mbar]. 

- The y-axis has values marked at 1 and 5.
- There is a curve starting from the bottom left, labeled "Wasser" (water), which rises steeply and then levels off as it moves to the right.
- Another curve, labeled "Flüssig" (liquid), starts from the bottom left and rises more gently, intersecting the "Wasser" curve at a point labeled "Tripel" (triple point).
- From the "Tripel" point, a line extends horizontally to the right, labeled "Isotherm" (isotherm).
- The graph is divided into three regions: "Wasser" (water) below the "Flüssig" curve, "Flüssig" (liquid) between the "Flüssig" and "Dampf" curves, and "Dampf" (steam) above the "Dampf" curve.
- Points labeled 1, 2, and 3 are marked on the graph, with point 1 in the "Dampf" region, point 2 on the "Isotherm" line, and point 3 in the "Wasser" region.

``````latex


\section*{b)}

\[
\dot{m}_{\text{R134a}}
\]

\[
T_i = 10^\circ C
\]
\[
T_{\Lambda} = 4^\circ C
\]

\begin{center}
\begin{tabular}{c}
\includegraphics[scale=0.5]{diagram.png}
\end{tabular}
\end{center}

\textbf{Description of the diagram:} The diagram is a circular representation with three points labeled 1, 2, and 3. Point 1 is at the top of the circle, point 2 is on the left side, and point 3 is on the right side. There is an arrow pointing from point 1 to point 2 labeled $\dot{W}_K$. Another arrow points from point 2 to point 3.

\[
\text{Stat. Fließg.}
\]
\[
0 = \dot{m} (h_2 - h_3) + \dot{W}_K
\]

\[
\text{adiabat reversibel: } s_2 = s_3
\]

\[
\text{Tab A-11}
\]
\[
h_4 = h_{e\Phi}(8 \text{bar}) = 93,42 \frac{\text{kJ}}{\text{kg}} = h_1
\]

\[
p_2 = p_1
\]
\[
p_3 = p_4 = 8 \text{bar}
\]
\[
h_3 = h_4 \rightarrow \text{drossel}
\]

\[
\dot{m} = \frac{\dot{W}_K}{h_2 - h_3}
\]

\section*{c)}

\[
h_1 = h_4 = 93,42 \frac{\text{kJ}}{\text{kg}}
\]

\[
h_1 = h_f + x_1 (h_g - h_f)
\]

\[
x_1 = \frac{h_1 - h_f}{h_fg}
\]

\[
x_1 = \frac{93,42 - 55,35}{184,15} = 0,196
\]

\text{Tab A-10} \quad T = 4^\circ C

\[
h_f = 55,35
\]
\[
h_fg = 184,15
\]

\section*{d)}

\[
\epsilon_K = \frac{Q_{zu}}{W_T}
\]

\section*{e)}

Da der Druck abnimmt, wird auch die $T_i$ abnehmen mit weiterlaufendem Kreislauf, da der Prozess nicht mehr isotherm ist.

```