
``````latex


\section*{Üb 4}

\subsection*{a)}

\begin{description}
    \item[Graph Description:] The graph is a plot with the y-axis labeled as \( p(\text{bar}) \) and the x-axis labeled as \( T(\degree C) \). There are two curves: one labeled "flüssig" (liquid) and the other labeled "Dampf" (vapor). The intersection of these curves is marked with a point labeled \( i \). A vertical line from \( i \) down to the x-axis is labeled \( T_{\text{Krit}} \). Another point on the "Dampf" curve is labeled \( ii \). The region below the curves is labeled "flüssig" and the region above the curves is labeled "Dampf".
\end{description}

\subsection*{b)}

\begin{description}
    \item[Graph Description:] The graph is a plot with the y-axis labeled as \( p(\text{bar}) \) and the x-axis labeled as \( T(\degree C) \). There are three points labeled \( 1 \), \( 2 \), and \( 3 \). Point \( 1 \) is in the "flüssig" region, point \( 2 \) is in the "fest" (solid) region, and point \( 3 \) is in the "Dampf" (vapor) region. There are horizontal and vertical lines connecting these points, forming a step-like structure.
\end{description}

\subsection*{d)}

\[
\epsilon_K = \frac{|Q_{\text{sol}}|}{|\dot{W}_{\text{kl}}|}
\]

\[
|Q_{\text{sol}}| = \dot{m}_{\text{ex}} \left[ h_{n3} - h_{n1} \right]
\]

\[
h_2 = h_g (8 \text{bar}) = g_2 \cdot u_2 \frac{R}{g}
\]

\[
s_2 = s_3 \quad s_2 = s_g (T_i - 6 \degree C)
\]

\[
x_3 = \frac{s_2 - s_f (8 \text{bar})}{s_g (8 \text{bar}) - s_f (8 \text{bar})}
\]

\[
h_3 = h_f (8 \text{bar}) + x \left( h_g (8 \text{bar}) - h_f (8 \text{bar}) \right)
\]

``````latex


```