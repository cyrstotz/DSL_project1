
``````latex


\section*{4.}
\subsection*{(a)}
\[
p_i = p_3 = 8 \text{bar}
\]
\[
p_4 = 
\]
\[
\text{Siehe Abb. 5}
\]

\subsubsection*{Verbal Description of Graphs}

The first graph is a plot with the vertical axis labeled \( p \) and the horizontal axis labeled \( T \). The graph shows a dome-shaped curve starting from the origin, rising to a peak, and then falling back down symmetrically.

The second graph is also a plot with the vertical axis labeled \( p \) and the horizontal axis labeled \( T \). There are several lines and points marked on this graph:
- A line labeled "4" starting from the lower left and moving upwards to the right.
- A line labeled "2" starting from the lower left and moving upwards to the right, parallel to the line labeled "4".
- A line labeled "g" starting from the lower left and moving upwards to the right, parallel to the other two lines.
- A point labeled "Tripel" where the lines converge.
- An arrow labeled "Mengegebiet" pointing upwards from the point "Tripel".

\subsection*{(b)}
\[
0 = \dot{m}_{R134a} \cdot (h_3 - h_4) - \dot{Q}_{ab} = \dot{m}_{R134a} \cdot (h_1 - h_2) + \dot{Q}_K
\]
\[
\dot{m}_{R134a} \cdot \frac{\dot{Q}_{ab}}{h_3 - h_4} = - \frac{\dot{Q}_K}{h_1 - h_2}
\]
\[
0 = \dot{m}_{R134a} \cdot (h_2 - h_3) - \dot{W}_K
\]
\[
\dot{m}_{R134a} \cdot \frac{\dot{W}_K}{h_2 - h_3}
\]
\[
h_2 = (p_2, x=1) = h_g(p_2) = 259 \frac{\text{kJ}}{\text{kg}}
\]
\[
p_2 = 6 \text{bar} - 5 \text{mbar} = 5.995 \text{bar} = p_1
\]
\[
p_3 = p_4 = 8 \text{bar}
\]

``````latex


4.(c) 
\[
\dot{m}_{R134a} = 4 \frac{kg}{h} = 0.0011 \frac{kg}{s}
\]
\[
T_2 = -22^\circ C
\]
\[
p_1 = p_2 = 5.95 \text{bar}
\]
\[
h_1 = h_4 = h(8 \text{bar}, x=0) = 53.42 \frac{kJ}{kg}
\]
\[
X_n = \frac{h_n - h_f(p_1)}{h_g(p_1) - h_f(p_1)} = 0.0777
\]

(d) 
\[
\varepsilon_k = \frac{|\dot{Q}_{zu}|}{|W_t|} = \frac{\dot{Q}_k}{W_k} = \frac{\dot{Q}_k}{28W}
\]

(e) 
Die Temperatur würde sich solange sinken verändern bis die Innentemperatur im therm. GGW mit dem Kühlkreislauf liegt.

```