
``````latex


\section*{Ideales Gas, Luft:}

\begin{tabular}{|c|c|c|}
\hline
$P[\text{bar}]$ & $T[^\circ C]$ & $T[\text{K}]$ \\
\hline
0 & 0.011 & -30 \\
1 & & \\
2 & & \\
3 & & \\
4 & & \\
5 & & \\
6 & 328.07 & \\
\hline
\end{tabular}

\begin{itemize}
    \item $\Delta Q = 0$, isentrop wg. $c < 1$ \\
    \item isentrop = adiabatic reversible \\
    \item adiabatic \\
    \item isobar $P_4 = P_1 = P_5$ \\
    \item isentrop = rev/adiabatic \\
\end{itemize}

\subsection*{a)}

Graph description: The graph is a Pressure-Volume (P-V) diagram with the x-axis labeled as $s \left[ \frac{u^3}{u \cdot k} \right]$ and the y-axis labeled as $T \left[ u \right]$. There are six points labeled 1 through 6. The points are connected by lines indicating different processes:

\begin{itemize}
    \item Point 1 to Point 2: Isentropic process
    \item Point 2 to Point 3: Isobaric process
    \item Point 3 to Point 4: Isentropic process
    \item Point 4 to Point 5: Isobaric process
    \item Point 5 to Point 6: Isentropic process
\end{itemize}

The graph also includes curves labeled as "isobare" and "isentrope".

\subsection*{b)}

\begin{align*}
    \text{Energie Bilanz} &\rightarrow \text{isentrop, da reversibel} \rightarrow \Delta Q = 0 \\
    &\rightarrow n = k = \frac{c_p}{c_v} = 1.4 \\
    T_0 &= T_s \left( \frac{P_0}{P_s} \right)^{\frac{n-1}{n}} \\
    &= 328.07 \, K
\end{align*}

Wo: Energie Bilanz um Schubdüse, stationär, ein Massenstrom:
\begin{itemize}
    \item $P_e = 0$
\end{itemize}

\[
0 = \dot{m} \left( h_s - h_e + \frac{w_s^2 - w_e^2}{2} \right) + \dot{Q}_{s \rightarrow e} - \dot{W}_{t, re}
\]

\[
\text{Wiese: } \dot{W}_{t, re} = \dot{m} \cdot \left( h_s - h_e \right)
\]

``````latex

\section*{Problem 2}

\subsection*{c)}

\begin{align*}
W_e &= 5.10 \frac{m}{s}, \quad T_0 = 340 \, K \\
\text{Exergiebilanz um Turbine, stationär, 1 massenstrom:} \\
\text{alles im Trockenlauf} \\
\Delta e_{\text{ex,str}} &= -\left[ h_0 - h_e - T_0 \left( s_0 - s_e \right) + \text{alle } \frac{w^2}{2} \right] \\
\text{alle} &= \frac{w_0^2 - w_e^2}{2}
\end{align*}

\begin{align*}
h_0 - h_e &= c_p \left( T_0 - T_e \right) = -97.4311 \frac{kJ}{kg} \\
s_0 - s_e &= c_p \ln \left( \frac{T_0}{T_e} \right) = -0.57927 \frac{kJ}{kg \cdot K} \\
\frac{w_0^2 - w_e^2}{2} &= -1.00050 \frac{kJ}{kg}
\end{align*}

\begin{align*}
\Delta e_{\text{ex,str}} &= -\left[ 207.878 \frac{kJ}{kg} \right] \\
\Delta e_{\text{ex,str}} &= 115.797 \frac{kJ}{kg}
\end{align*}

\subsection*{d)}

\begin{align*}
\dot{e}_{\text{ex,verl}} &= T_0 \cdot \dot{s}_{\text{erz}} \\
\text{alles im Trockenlauf} \\
\text{Entropiebilanz um Turbine:} \\
0 &= \dot{m} \left( s_0 - s_e \right) + \frac{\dot{Q}_{0e}}{T_{0e}} + \dot{s}_{\text{erz}} \\
&\Rightarrow \text{alles adiabatisch außer Brennkammer} \\
\frac{\dot{Q}_{0e}}{\dot{m}} &= -q_B = -1195 \frac{kJ}{kg} \\
T_{0e} &= \overline{T}_B = 1289 \, K
\end{align*}

\begin{align*}
\frac{\dot{s}_{\text{erz}}}{\dot{m}} &= s_e - s_0 - \frac{q_B}{T_B} = 1.304 \frac{kJ}{kg \cdot K}
\end{align*}

\begin{align*}
e_{\text{ex,verl}} &= 317.15 \frac{kJ}{kg}
\end{align*}

``````latex


