
``````latex


\section*{Aufgabe 3}

\subsection*{a)}

\textbf{Geg:} $p_1, T_1$

\[
m_2 = \frac{m}{50 \, \frac{\text{g}}{\text{mol}}}
\]

\[
\Pi = \frac{8,314 \, \frac{\text{J}}{\text{mol} \cdot \text{K}}}{50 \, \frac{\text{g}}{\text{mol}}} = 166,28 \, \frac{\text{J}}{\text{kg} \cdot \text{K}}
\]

\[
R = c_p - c_v
\]

\[
c_p = R + c_v = 166,28 \, \frac{\text{J}}{\text{kg} \cdot \text{K}} + 0,633 \, \frac{\text{kJ}}{\text{kg} \cdot \text{K}} = 0,79928 \, \frac{\text{kJ}}{\text{kg} \cdot \text{K}}
\]

\subsection*{b)}

\textbf{Die Masse rechnet man dann durch ideales Gasgesetz:}

\[
m = \frac{\rho V}{R T}
\]

\[
= \frac{1,5 \, \text{bar} \cdot 3,14 \, \text{L}}{0,16628 \, \frac{\text{kJ}}{\text{kg} \cdot \text{K}} \cdot 273,15 \, \text{K}}
\]

\[
m = 3,664 \, \text{kg}
\]

\[
\boxed{m = 3,664 \, \text{kg}}
\]

\subsection*{c)}

\textbf{Durch 1. Hauptsatz:} $Q = \Delta U + p \Delta V - W_{12}$

\textbf{Die Temperatur des Gases ist kleiner als $T_1$, da dieser das Gas zuständig war, um das Eis zu schmelzen.}

\textbf{Der Druck bleibt konstant, da sich die Temperatur ...}

``````latex


Der Druck bleibt Constant \( p_{1,s} = p_{2+s} \), da sich die Temperatur verändert, aber gleichzeitig auch das Volumen, da die Masse es nach unten drückt.

\begin{equation*}
\begin{aligned}
c) \quad \Delta U &= \cancel{m} \cdot \cancel{c_v} \cdot \Delta T \quad \text{1. Hauptsatz} \\
Q_{12} &= \cancel{m} \cdot (u_2 - u_1) \\
&= m \cdot c_v (T_2 - T_1) \\
Q_{12} &= 3,6 \, g \cdot c_v (T_2 - T_1)
\end{aligned}
\end{equation*}

``````latex


