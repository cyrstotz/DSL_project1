
``````latex


\section*{Aufgabe: 2}

\subsection*{a)}

\[
\begin{array}{c}
\includegraphics[width=0.8\textwidth]{graph.png}
\end{array}
\]

\textbf{Verbal Description of the Graph:}

The graph is a plot with the vertical axis labeled \( T [K] \) and the horizontal axis labeled \( S \left[ \frac{kJ}{kg \cdot K} \right] \). The plot contains a closed loop with six points labeled 0 through 6. The points are connected as follows:

- Point 0 to Point 1
- Point 1 to Point 2 (labeled as "adiabat")
- Point 2 to Point 3 (labeled as "isobar")
- Point 3 to Point 4
- Point 4 to Point 5 (labeled as "isobar")
- Point 5 to Point 6
- Point 6 to Point 0

\textbf{Zustand 0:}
\[
T_0 = -30^\circ C
\]

\textbf{Zustand 1:}
\[
\text{adiabat} \quad S_1 > S_0, \quad T_1 > T_0
\]

\textbf{Zustand 2:}
\[
\text{adiabat, reversibel} \quad S_2 = S_1, \quad T_2 > T_1
\]

\textbf{Zustand 3:}
\[
\text{isobar} \quad T_3 > T_2, \quad S_3 >> S_2
\]

\textbf{Zustand 4:}
\[
T_4 < T_3, \quad S_4 > S_3, \quad \text{irreversibel}
\]

\textbf{Zustand 5:}
\[
S_5 \neq S_4, \quad T_5 \neq T_4
\]

\textbf{Zustand 6:}
\[
S_6 = S_5, \quad T_6 < T_5
\]

\subsection*{b)}

\[
\dot{Q} = \dot{m} \left( h_e - h_a + \frac{w_e^2 - w_a^2}{2} \right)
\]

\[
\left( \frac{P_6}{P_5} \right)^{\frac{k-1}{k}} = \frac{V_5}{V_6} = \frac{T_6}{T_5}
\]

\[
T_6 = T_5 \left( \frac{P_6}{P_5} \right)^{\frac{k-1}{k}} = 328.07 \, K
\]

\[
P_5 V_5 = R T_5
\]

\[
R = \frac{\bar{R}}{M} = 0.2867
\]

\[
V_5 = \frac{R T_5}{P_5}
\]

\[
V_5 = 7.48 \, \frac{m^3}{kg}
\]

\[
V_6 = V_5 \sqrt{\frac{T_6}{T_5}}
\]

\[
V_6 = 4.8377 \, \frac{m^3}{kg}
\]

``````latex


Massen ist konstant

\begin{equation}
m_{\text{ein}} = \rho_1 A_1 w_1 = \rho_2 A_2 w_2 \quad \rightarrow \quad A_3 = A_2 \quad \rightarrow \text{Annahme} \quad \text{(keine andere Idee)}
\end{equation}

\begin{equation}
T_1 = 431,9 \, K \quad \quad V_5 = 2,42 \, \frac{m^3}{kg}
\end{equation}

\begin{equation}
T_6 = 322,07 \, K \quad \quad V_6 = 4,93717 \, \frac{m^3}{kg}
\end{equation}

\begin{equation}
s_2 = s_6
\end{equation}

\begin{equation}
w_6 = \frac{p_5}{\rho_6} w_5
\end{equation}

\begin{equation}
w_6 = \frac{V_6}{V_5} w_5
\end{equation}

\begin{equation}
w_6 = 437,49 \, \frac{m}{s}
\end{equation}

c) keine Zeit mehr

``````latex


