
``````latex


\section*{Aufgabe 4}

\subsection*{Graph 1}

The first graph is a phase diagram with the following details:

- The y-axis is labeled \( p \) (mbar) and is logarithmic, with tick marks at 0.01, 0.1, 1, 10, and 100.
- The x-axis is labeled \( T \) (°C) and is linear, with tick marks at -50, 0, and 50.
- There are three regions labeled "fest", "flüssig", and "gas".
- The "fest" region is in the upper left, the "flüssig" region is in the middle, and the "gas" region is in the lower right.
- The "Tripel" point is marked where the three regions meet.
- There are several lines separating the regions, with one line curving from the lower left to the upper right, and another line curving from the upper left to the lower right.
- The "flüssig" region is shaded with diagonal lines.

\subsection*{Graph 2}

The second graph is a more detailed phase diagram with the following details:

- The y-axis is labeled \( p \) (mbar) and is logarithmic, with tick marks at 0.01, 0.1, 1, and 10.
- The x-axis is labeled \( T \) (°C) and is linear, with tick marks at -30, -20, and 0.
- There are three regions labeled "Fest", "flüssig", and "gas".
- The "Fest" region is in the upper left, the "flüssig" region is in the upper right, and the "gas" region is in the lower part of the graph.
- The "Tripel" point is marked where the three regions meet.
- There is a line curving from the lower left to the upper right, separating the "gas" and "flüssig" regions.
- There is a point labeled \((ii) T\) with an arrow pointing to \((T_i)\).

``````latex


6)

\[
\dot{Q} = \dot{m}_{\text{pass}} (h_{in} - h_{out}) + \dot{Q}_k
\]

\[
\dot{m}_R = \frac{\dot{Q}_k}{h_2 - h_1}
\]

\[
h_2 = h_g (
\]

\begin{description}
    \item[Graphical Content:] 
    There is a table with four rows and three columns. The first column is labeled with a vector symbol $\vec{z}$, the second column is labeled with $P$, and the third column is labeled with $T$. The rows are numbered from 1 to 4 in the first column. The second column has the numbers 1, 2, 3, and 4 listed vertically. The third column has the numbers 7, 8, 8, and 8 listed vertically.
\end{description}

```